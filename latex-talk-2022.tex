% !TeX encoding = UTF-8

\documentclass[UTF8]{ctexbeamer}
\usetheme[default,topright]{sjtubeamer}
% \logo{\includegraphics[height=.9cm]{figures/sjtug-logo.pdf}}
\usepackage{parskip}
\begin{document}
    \begin{frame}
        \frametitle{\LaTeX{} 新语}

        {\scriptsize \textbf{摘要}~本讲座首先带领大家从零跑起来 \LaTeX{},只用十步速通 \LaTeX{} 的基础操作。接着展示如何用 \textsc{SJTUThesis} 排版论文、如何使用 \textsc{SJTUBeamer} 制作幻灯片。最后介绍一些 \LaTeX{} 相关的辅助工具,以提高工作效率。\par}

        \begin{enumerate}
            \item 学习 \LaTeX{}
            
            {\scriptsize 概述网站 Learn\LaTeX{}.org 的内容,10 步速通 \LaTeX{} 基本概念与基础操作,主要介绍一些最佳实践。 \par}
            \item \textsc{SJTUThesis}
            
            {\scriptsize 展现 \LaTeX{} 的重要思想:格式与内容分离,修改内容即可完成一份排版优良的论文。学习 \LaTeX3 的范例。 \par}
            \item \textsc{SJTUBeamer}
            
            {\scriptsize 制作学术幻灯片的模板,快速上手并与 PowerPoint 比较,展示基于 GitHub 的开源社区建设。 \par}

            \item 像模像样 \LaTeX{}
            
            {\scriptsize 概述本人的博客,介绍 \LaTeX{} 辅助工具,展现当下 \LaTeX{} 与其他高级语言结合的可能。提供加速 \LaTeX{} 编译的进阶方法。推荐一些参考资料。\par}
        \end{enumerate}
    \end{frame}

    \begin{frame}
        \frametitle{从零开始使用 \LaTeX{} 排版论文}
        \framesubtitle{上海交通大学图书馆专题培训讲座}

        {\scriptsize \textbf{摘要}~本次讲座帮助学生从零开始跑起来 \LaTeX{},只用十步了解其基本概念和基础操作。学会使用 \textsc{SJTUThesis} 交大学位论文模板,并了解提升排版效率的相关技巧。\par}

        \begin{enumerate}
            \item 学习 \LaTeX{}
            
            {\scriptsize 概述网站 Learn\LaTeX{}.org 的内容,10 步速通 \LaTeX{} 基本概念与基础操作,主要介绍一些最佳实践。 \par}
            \item \textsc{SJTUThesis}
            
            {\scriptsize 带领大家使用 \textsc{SJTUThesis} 排版论文,并提供解决常见错误的解决方案。 \par}
            
            \item 提升排版效率
            
            {\scriptsize 简要提及 \textsc{SJTUBeamer} 制作学术幻灯片的模板,简要展示一些像模像样 \LaTeX{} 上的高效排版技巧。提供一些其他深入学习的参考资料。\par}
        \end{enumerate}
    \end{frame}
\end{document}