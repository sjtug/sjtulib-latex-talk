% !TeX encoding = UTF-8

\documentclass[UTF8,space=auto,xcolor=table,dvipsnames,svgnames,aspectratio=169]{ctexbeamer}
% Author: 2022 Log Creative <logcreative@outlook.com>
%         2021 Alexara Wu <alexarawu@outlook.com>

\usetheme[my,default,topright]{sjtubeamer}
\usemytheme{sjtug}
\setbeamertemplate{background}{}
\graphicspath{{support/}{support/contents/}{support/examples/}{support/figures/}{support/images/}{support/thesis/}{support/beamer/}{support/pgfplots/}}

%% If you don't use sjtubeamer theme.
%% Uncomment the following lines.
% \usepackage{listings}
% \newcommand{\highlight}[2][]{\alert{#2}}
% \newcommand{\paragraph}[1]{\bfseries #1\hspace{.5em}}
% \newenvironment{stampbox}{}{}
% \newcommand{\makebottom}{
%   \begin{frame}
%     \begin{center}
%       \Huge
%       Thank You!
%     \end{center}
%   \end{frame}
% }
% \newenvironment{\codeblock}[2][]{\begin{verbatim}}{\end{verbatim}}
% \newenvironment{bibliolist}[1]{\begin{thebibliography}{#1}}{\end{thebibliography}}
% \newcommand{\articleitem}{\item}
% \newcommand{\bookitem}{\item}
% \newcommand{\onlineitem}{\item}
% \newenvironment{shadedsection}{}{}

\usepackage{fontawesome5}
\usepackage{hologo}
\usepackage{hyperxmp}
\usepackage{qrcode}
\usepackage{ulem}
\usepackage{subcaption}
\usepackage{booktabs}

\newcommand{\SJTUThesis}{\textsc{SJTUThesis}}
\newcommand{\SJTUThesisVersion}{1.1.0}
\newcommand{\SJTUThesisDate}{2022/3/26}
\newcommand{\SJTUBeamer}{\textsc{SJTUBeamer}}
\newcommand{\SJTUBeamerVersion}{2.6.1}
\newcommand{\SJTUBeamerDate}{2022/4/21}

\newcommand{\thesisissue}[1]{\href{https://github.com/sjtug/SJTUThesis/issues/#1}{\textsuperscript{\textcolor{.!90!DarkBlue}{\underline{\faCheckCircle{}#1}}}}}
\newcommand{\thesisdiscuss}[1]{\href{https://github.com/sjtug/SJTUThesis/discussions/#1}{\textsuperscript{\textcolor{.!90!DarkBlue}{\underline{\faComments{}#1}}}}}
\newcommand\link[1]{\href{#1}{\faLink}}
\newcommand\pkg[1]{\texttt{#1}}

\def\cmd#1{\texttt{\color{DarkBlue}$\backslash$#1}}
\def\env#1{\texttt{\color{DarkBlue}#1}}
\def\cmdxmp#1#2#3{\small{\texttt{\color{DarkBlue}$\backslash$#1}\{#2\}\hspace{1em}\\ $\Rightarrow$\hspace{1em} {#3}\par\vskip1em}}

%% Overleaf and some distributions omit the Adobe icon ...
%% If you need to display the Adobe logo, comment the following line.
\def\faAdobe{}

\lstset{
  escapechar=|
}

\newcommand{\includepdflarge}[1]{\framebox{\includegraphics[width=\linewidth,trim=4cm 17cm 3cm 2cm,clip]{#1}}\vspace{-10pt}}
\newcommand{\includebeamerlarge}[2][1]{\framebox{\includegraphics[width=0.75\linewidth,page=#1]{#2}}}

%% Use pympress to show the notes on the second screen.
% \setbeameroption{show notes on second screen}

\title{\LaTeX{} 新语}
\subtitle{上海交通大学图书馆专题培训讲座}
\author{李子龙}
\institute[SJTU Linux User Group]{上海交通大学 Linux 用户组}
\date{\the\year\,年\,\the\month\,月}
\subject{LaTeX, SJTUThesis, SJTUBeamer}

\hypersetup{
  pdfsubject         = {上海交通大学图书馆专题培训讲座},
  pdfauthor          = {Log Creative},
  pdfcopyright       = {Licensed under CC-BY-SA 4.0. Some rights reserved.},
  pdflicenseurl      = {http://creativecommons.org/licenses/by-sa/4.0/},
  unicode            = true,
  psdextra           = true,
  pdfdisplaydoctitle = true
}

\includeonly{
  support/contents/learnlatex,
  support/contents/sjtuthesis,
  support/contents/intermission,
  support/contents/sjtubeamer,
  support/contents/visualization,
  support/contents/boost
}

\begin{document}

% !TeX root = ../../latex-talk.tex

\title{从零开始使用 \LaTeX{} 排版论文}
\subtitle{上海交通大学图书馆专题培训讲座}
\author{李子龙}
\institute[SJTU Linux User Group]{上海交通大学 Linux 用户组}
\date{2022 年 4 月}
\subject{LaTeX, 论文排版, SJTUThesis}
\maketitle


\begin{frame}{关于}
    \begin{columns}[c]
      \begin{column}{.7\textwidth}
        \begin{block}{从零开始使用 \LaTeX{} 排版论文}
          \alert{\url{https://github.com/sjtug/sjtulib-latex-talk}}
          
          \begin{flushleft}
            \small 本次讲座帮助学生从零开始跑起来 \LaTeX{},只用十步了解其基本概念和基础操作。学会使用 \textsc{SJTUThesis} 交大学位论文模板,并了解提升排版效率的相关技巧。
          \end{flushleft}
  
          \begin{tabular*}{0.8\linewidth}{@{\extracolsep{\fill}}lll@{}}
            \scriptsize 最后更新 & \scriptsize 幻灯片下载 & \scriptsize 许可证 \\
            \today & Overleaf \link{https://www.overleaf.com/read/fvwxzvcxhcwd} & \href{https://creativecommons.org/licenses/by-sa/4.0/}{\faCreativeCommons\,\faCreativeCommonsBy\,\faCreativeCommonsSa} \\ 
          \end{tabular*}
        \end{block}
        \vspace{0.2cm}
      \end{column}
      \begin{column}{.3\textwidth}
        \qrcode[hyperlink, height=4cm]{https://www.overleaf.com/read/fvwxzvcxhcwd}
      \end{column}
    \end{columns}
  \end{frame}
  
  % \begin{frame}
  %   \frametitle{主要参考}
  %   \printbibliography[heading=none]
  % \end{frame}
  
% !TeX root = ..\..\latex-talk.tex

\part{学习 \LaTeX{}}
% FIXME: Part Page miniframe overflow
% FIXME: footnote fault numbering

\begin{frame}[plain]
  \vfil
  \begin{center}
    \href{https://learnlatex.org}{
      \rmfamily
      Learn\,\lower1ex\hbox{\Huge\LaTeX{}}.org
    }
  \end{center}
  \vfil
  \begin{center}
    \parbox{0.75\linewidth}{
      Learn\LaTeX{}.org\cite{learnlatex} 提供了解 \LaTeX{} 的 16 篇简短的教程,并包含了一些可以在线运行的示例,可以通过亲自动手查看实验效果。本部分主要参考由 C\TeX{}-org 提供的中文翻译版本 \link{https://github.com/CTeX-org/learnlatex.github.io/tree/zh-Hans/zh-Hans/}。
    }
  \end{center}
  \vfil
\end{frame}

{ % Start of shaded number logo

\newcommand{\shadedfont}[2][1pt]{
  % #1 (optional): shadow distance
  % #2: the text needed to be shaded
  \hbox{\rlap{\color{gray}\hskip#1#2}#2}
}
\newcounter{learnsec}
\setcounter{learnsec}{-1}
\newcommand{\updatelogo}{
  % update the logo corresponding to current counter.
  \stepcounter{learnsec}
  \logo{
    \raise.3ex\hbox{\tiny\insertsection}\shadedfont{\arabic{learnsec}}
  }
}
\let\oldsection=\section
\renewcommand{\section}[1]{\oldsection{#1}\updatelogo}

\section{是什么}
\begin{frame}
  \frametitle{\TeX{}}
  \begin{columns}[c]
    \begin{column}{0.7\textwidth}
      \begin{center}
        \rmfamily\Huge
        \hologo{La}\highlight[structure!70]{\TeX{}}
      \end{center}
      \begin{center}
        \parbox{0.75\textwidth}{
          \TeX{} 是由斯坦福大学教授高德纳
          (Donald E.~Knuth)于 1977 年开始开发的排版引擎。目前仍在更新,最新版本号为 3.141592653 \link{https://tug.org/TUGboat/tb42-1/tb130knuth-tuneup21.pdf}。
        }
      \end{center}
    \end{column}
    \begin{column}{0.3\textwidth}
      \includegraphics[width=.8\columnwidth]{Knuth.jpg}
    \end{column}
  \end{columns}
\end{frame}

\begin{frame}
  \frametitle{\LaTeX{}}
  \begin{columns}[c]
    \begin{column}{0.7\textwidth}
      \begin{center}
        \rmfamily\Huge
        \highlight[structure]{\LaTeX{}}
      \end{center}
      \begin{center}
        \parbox{0.75\textwidth}{
          \LaTeX{} 是最早在 1985 年由现就职于微软的 Leslie Lamport 开发的一种 \TeX{} \textbf{格式}\footnotemark,使用一些列宏和扩展宏包来简化 \TeX{} 的使用。现在由 \LaTeX{} Project 的成员维护。现在广泛使用的版本是 \LaTeXe{},最新的版本为 \LaTeX3(2020 年 10 月后默认内置)。
        }
      \end{center}
    \end{column}
    \begin{column}{0.3\textwidth}
      \includegraphics[width=.8\columnwidth]{Lamport.jpg}
    \end{column}
  \end{columns}
  \footnotetext{\hologo{ConTeXt} 也是一种 \TeX{} 格式 \link{https://www.contextgarden.net/}。}
\end{frame}

\begin{frame}
  \frametitle{程序}
  \begin{columns}[c]
    \begin{column}{0.7\textwidth}
      \begin{center}
        \rmfamily\Huge
        \highlight[structure]{\hologo{pdfLaTeX}}
      \end{center}
      \begin{center}
        \parbox{0.7\textwidth}{
          \hologo{pdfLaTeX} 是为了编译一个 \LaTeX{} 文档而运行的程序。实际上底层在运行一个叫 \hologo{pdfTeX} 的引擎,并预装了对应的 \LaTeX{} \textbf{格式}。为了利用临时文件,可能就需要多次运行程序。
        }
      \end{center}
    \end{column}
    \begin{column}{0.3\textwidth}
      \begin{block}{}
        \ttfamily\small
        > \highlight{pdflatex} main.tex\\
        This is pdfTeX, Version 3.141592653-
        2.6-1.40.23 (MiKTeX 21.10)\\
        entering extended mode\\
        \highlight{LaTeX2e} <2021-11-15>\\
        \highlight{L3} programming layer <2021-11-22>
      \end{block}
    \end{column}
  \end{columns}
\end{frame}

\begin{frame}
  \frametitle{引擎}
  \begin{columns}[c]
    \begin{column}{0.7\textwidth}
      \begin{center}
        \rmfamily\Huge
        \highlight[structure!70]{pdf}\hologo{La}\highlight[structure!70]{\TeX{}}
      \end{center}
      \begin{center}
        \parbox{0.7\textwidth}{
          pdf\TeX{} 是编译 \TeX{} 文档(以 \texttt{.tex} 结尾)的\textbf{引擎}---可以理解 \TeX{} 指令的\textbf{程序}。
        }
      \end{center}
    \end{column}
    \begin{column}{0.3\textwidth}
      \begin{block}{}
        \ttfamily\small
        > pdflatex main.tex\\
        This is \highlight[structure!70]{pdfTeX}, Version 3.141592653-
        2.6-1.40.23 (MiKTeX 21.10)
        entering extended mode\\
        LaTeX2e <2021-11-15>\\
        L3 programming layer <2021-11-22>
      \end{block}
    \end{column}
  \end{columns}
\end{frame}

\begin{frame}
  \frametitle{Unicode 引擎}
  \begin{table}
    \caption{主流 \hologo{(La)TeX} 程序
    \footnote{(u)p\TeX{} 是日语最常用的引擎,生成 \texttt{.dvi},支持 Unicode。}\footnote{Ap\TeX{} 具有底层 CJK 支持,内联 Ruby,Color Emoji。}}
    \footnotesize
    \begin{stampbox}
      \begin{tabular}{c>{\raggedright}*{3}{p{3.5cm}}}
        \alert{引擎}     & \hologo{pdfTeX}   & \hologo{XeTeX}   & \hologo{LuaTeX}   \\
        \alert{程序}     & \hologo{pdfLaTeX} & \hologo{XeLaTeX} & \hologo{LuaLaTeX} \\
        \alert{特点}     & 直接生成 PDF,支持 micro-typography  & 支持 Unicode、OpenType 与复杂文字编排 (CTL) & 支持 Unicode,内联 Lua,支持 OpenType \\
      \end{tabular}
    \end{stampbox}
  \end{table}

  \begin{center}
    \parbox{.9\textwidth}{
      \hologo{pdfLaTeX} 不支持 Unicode。为了排版中文,大部分情况下 \faApple{}\,\faLinux{} 应当使用 \hologo{XeLaTeX},而 \hologo{LuaLaTeX} 速度相对较慢。\faWindows{} 可以在一些情况下使用 \hologo{pdfLaTeX}。
    }
  \end{center}
\end{frame}

% \begin{frame}
%   \paragraph{\hologo{pdfLaTeX}} \TeX{} 和 \LaTeX{} 被广泛使用之前,它们只需内置支持欧洲语言即可。在 Unicode 出现之前,\LaTeX{} 提供了许多种\textbf{文件编码}来允许很多语言的文字以原生的方式输入,\hologo{pdfLaTeX} 也只需要使用 8 位文件编码和 8 位字体。
% \end{frame}

\section{跑起来}
\begin{frame}
  \frametitle{发行版}
  \begin{table}
    \caption{\hologo{TeX} 发行版}
    \footnotesize
    \begin{stampbox}
      \begin{tabular}{c>{\raggedright}*{3}{p{3.2cm}}}
        \alert{发行版}     & \hologo{MiKTeX} \link{https://mirrors.sjtug.sjtu.edu.cn/ctan/systems/win32/miktex/setup/windows-x64/basic-miktex-21.12-x64.exe}   & \TeX{} Live \link{https://mirrors.sjtug.sjtu.edu.cn/ctan/systems/texlive/tlnet/install-tl.zip}   & Mac\TeX{} \link{https://mirrors.sjtug.sjtu.edu.cn/ctan/systems/mac/mactex/mactex-20210328.pkg}  \\[2pt]
        \alert{特点}      &  只安装必要文件,依赖用时更新  &  所有平台均可使用,每年发布一次 & Mac 系统专用,对 \TeX{} Live 的进一步打包 \\
        \alert{推荐平台}  & \faWindows  & \faLinux &  \faApple \\
      \end{tabular}
    \end{stampbox}
  \end{table}
  \begin{center}
    \parbox{.9\textwidth}{
      要让 \LaTeX{} 跑起来,核心就是要有一套 \TeX{} 发行版,来获取让 \LaTeX{} 工作所需的一组程序和文件。参考《一份简短的关于 \LaTeX{} 安装的介绍》\link{https://mirrors.sjtug.sjtu.edu.cn/ctan/info/install-latex-guide-zh-cn/install-latex-guide-zh-cn.pdf} 安装想使用的发行版。推荐使用发行版的最新版本\footnote{老版本 Linux 系统的包管理器自带 \TeX{} Live 发行版可能不是最新的 \link{https://repology.org/project/texlive/versions},尽量使用镜像安装,并手动将相关环境变量添加到路径 \link{https://www.tug.org/texlive/doc/texlive-zh-cn/texlive-zh-cn.pdf}。},并使用国内镜像。
    }
  \end{center}
\end{frame}

\begin{frame}[plain]
  \hbox to \textwidth{
    \hfil
    \vbox to 3cm{
      \hbox{
        \resizebox{3cm}{!}{\includegraphics{\getcontribpath{sjtug}{vi/sjtug.pdf}}}
      }
    }
    \hfil
    \vbox to 3cm{
      \vfill
      \hbox{\Large\bfseries\color{cprimary} 稳定、快速、现代的镜像服务。}
      \vskip2pt
      \hbox{托管于华东教育网骨干节点上海交通大学。}
      \vfill
    }
    \hskip20pt
    \hfil
  }

  \begin{center}
    \parbox{0.8\textwidth}{
      推荐使用 SJTUG 软件镜像服务,离得近,下得快。
      
      \begin{description}
        \footnotesize
        \item[\TeX{} Live]  {\ttfamily tlmgr option repository https://mirrors.sjtug.sjtu.edu.cn/CTAN/systems/texlive/tlnet}
        \item[\hologo{MiKTeX}] 在 \hologo{MiKTeX} Console 中设置镜像源为 \url{https://mirrors.sjtug.sjtu.edu.cn}
      \end{description}
    }
  \end{center}
\end{frame}

\begin{frame}
  \frametitle{编辑器}
  \begin{table}
    \caption{开源编辑器推荐}
    \footnotesize
    \begin{stampbox}
      \begin{tabular}{c>{\raggedright}*{3}{p{3.5cm}}}
        \alert{编辑器}     & \begin{tabular}{c}Visual Studio Code\\ \LaTeX{} Workshop\end{tabular}  & \TeX{}studio & \TeX{}works \\[5pt]
        \alert{特点}      &  搭配 VS Code 使用非常方便,易扩展  & 可以使用大量的菜单选项输入代码块,用户友好 & 只提供基础的高亮与运行方法,发行版自带\footnote{Mac\TeX{} 打包的是 \TeX{}Shop 编辑器。} \\
      \end{tabular}
    \end{stampbox}
  \end{table}
  \begin{center}
    \parbox{.9\textwidth}{
      使用专为 \LaTeX{} 设计的编辑器将带来更多便利,因为它们往往会提供一键编译、内置 PDF 阅读器以及语法高亮等功能。几乎所有现代的 \LaTeX{} 编辑器都提供 Sync\TeX{} 这一强大的功能,以在 PDF 和 代码间对应跳转。
    }
  \end{center}
\end{frame}

\begin{frame}
  \frametitle{在线平台}
  \begin{table}
    \caption{在线协作平台推荐}
    \footnotesize
    \begin{stampbox}
      \begin{tabular}{c>{\raggedright}*{2}{p{4cm}}}
        \alert{在线平台}     & Overleaf \link{https://www.overleaf.com/}  & 交大 \LaTeX{} 助手 \link{https://latex.sjtu.edu.cn/} \\[2pt]
        \alert{特点}      & 最流行的在线平台,提供大量的模板,但国内访问慢 & 校内平台,隐私保护有保障,共享项目限制少 \\
      \end{tabular}
    \end{stampbox}
  \end{table}
  \begin{center}
    \parbox{.9\textwidth}{
      在线平台允许你直接在网页中编辑文档,无需本地安装即可在后台运行 \LaTeX{},并显示生成的 PDF。可以参照 Overleaf 官方文档学习如何使用在线平台 \link{https://www.overleaf.com/learn}。
    }
  \end{center}
\end{frame}

\section{基本结构}
\begin{frame}[fragile]%
  \frametitle{文档部件}
  \begin{columns}[c]
    \begin{column}{0.4\textwidth}
      \only<1>{
        \cmd{documentclass} 命令加载了\textbf{文档类}。\pkg{article} 是由 \LaTeX{}提供的用于排版短文档的基本文档类。
        \begin{description}
          \footnotesize
          \item[\pkg{article}] 不包含章的短文档
          \item[\pkg{report}] 含有章的单面印刷文档
          \item[\pkg{book}] 含有章的双面印刷文档
          \item[\pkg{beamer}] 制作幻灯片
        \end{description}
      }
      \only<2>{
        \env{document} 环境用于指示文档主体的范围。\LaTeX{} 还有其他的使用 \cmd{begin} 和 \cmd{end} 的搭配,我们称这些为\textbf{环境}。它们将用来设定局部格式命令\footnotemark。
      }
      \only<3>{
        \includepdflarge{enminimal}
      }
    \end{column}
    \begin{column}{0.6\textwidth}
      \begin{codeblock}[]{排版英文最简示例}
|\only<1>{\highlightline}|\documentclass{article}
|\only<2>{\highlightline}|\begin{document}
|\only<3>{\highlightline}|  Together for a Shared Future
|\only<2>{\highlightline}|\end{document}
      \end{codeblock}
    \end{column}
  \end{columns}
  \only<2>{\footnotetext{环境实际上是一个组,只不过通过语义化的形式预装了对应的格式命令。普通的组可以直接使用一对大括号之间的内容 \{$\cdots$\} 表示。}}
\end{frame}

\section{扩展}
\begin{frame}[fragile]%
  \frametitle{中文排版}
  \begin{columns}[c]
    \begin{column}{0.4\textwidth}
      \only<1>{
        \cmd{usepackage} 用于使用宏包以向 \LaTeX{} 添加或修改功能,需要在\textbf{导言区}调用。
        这里使用 \pkg{ctex} 宏集以获得中文支持。其调用底层因随不同的引擎而不同。
        {
          \footnotesize
          \begin{stampbox}
            \begin{tabular}{c*{3}{c}}
              \alert{引擎}     & \hologo{pdfTeX}   & \hologo{XeTeX}   & \hologo{LuaTeX}   \\
              \alert{程序}     & \hologo{pdfLaTeX} & \hologo{XeLaTeX} & \hologo{LuaLaTeX} \\
              \alert{宏包}     & CJK\footnotemark & xeCJK & luatexja \\
              \alert{封装}     & \multicolumn{3}{c}{ctex} \\
            \end{tabular}
          \end{stampbox}
        }
        \vspace{-1cm}
      }
      \only<2>{
        C\TeX{} 建议对于之前提到的常规文档类,最佳实践是使用该宏集提供的四种中文文档类,以对特定类型提供额外的中文排版适配。
        \begin{center}
          \begin{stampbox}
            \footnotesize
            \begin{tabular}{cc}
              \pkg{ctexart} & \pkg{ctexrep} \\
              \pkg{ctexbook} & \pkg{ctexbeamer} \\
            \end{tabular}
          \end{stampbox}
        \end{center}
      }
      \only<3>{
        \includepdflarge{cnminimal}
      }
      \only<4>{
        大部分情况下,你都不应当在 \LaTeX{} 中强制断行:你几乎只是想另起一段,或者是想在段落之间添加空行(使用 \pkg{parskip} 宏包就可实现)。
        只有\alert{很少的}情况下你需要使用 \textbackslash{}\textbackslash{} 来另起一行而不另起一段。
      }
    \end{column}
    \begin{column}{0.6\textwidth}
      \begin{codeblock}[]{排版中文\only<2->{最佳实践}}
|\only<2>{\highlightline}|\documentclass{|\only<1>{article}\only<2->{ctexart}|}
|\only<1>{\highlightline\textbackslash{}usepackage\{ctex\}\hfill\color{cprimary}\% 导言区}|
\begin{document}
|\only<3>{\highlightline}|    一起向未来
|\only<4>{\highlightline}|
  Together for a Shared Future
\end{document}
      \end{codeblock}
    \end{column}
  \end{columns}
  \only<1>{\footnotetext{ctex 在 \faApple\,\faLinux{} 上已经不可以使用 \hologo{pdfLaTeX} 编译,以及在 \faWindows{} 上使用该引擎也会变更自动间距调整等功能的默认行为。}}
\end{frame}

\section{设定格式}
\begin{frame}[fragile]%
  \frametitle{字体样式}
  \begin{columns}
    \begin{column}{0.4\textwidth}
      \only<1>{
        \includepdflarge{fontstyle}
      }
      \only<2>{
        可以使用显示样式设定命令对小段做加粗、斜体、等宽等等的处理。
        \begin{center}
          \footnotesize
          \begin{stampbox}
            \begin{tabular}{rl}
              \cmd{textrm} & \textrm{衬线} \\
              \cmd{textbf} & \textbf{加粗} \\
              \cmd{textit} & \kaishu 斜体 \\
              \cmd{texttt} & \texttt{等宽} \\
              \cmd{textsf} & \textsf{无衬线} \\
              \cmd{textsc} & \textsc{Small Caps} \\
              \cmd{textsl} & \textsl{Slanted} \\
            \end{tabular}
          \end{stampbox}
        \end{center}
      }
      \only<3>{
        与 Word 不同的是,\LaTeX{} 一般情况下并不需要使用上面的显式命令,而是采用逻辑标记的方法,
        比如 \cmd{emph} 可以强调文字,以及下面将要提到的目次命令(第 \ref{sectioning} 页)。
        这样可以统一管理格式。
      }
    \end{column}
    \begin{column}{0.6\textwidth}
      \begin{codeblock}[]{样式}
\documentclass{ctexart}
\begin{document}
|\only<2>{\highlightline}|  \textbf{||一起向未来}

|\only<3>{\highlightline}|  \emph{Together for a Shared Future}
\end{document}
      \end{codeblock}
    \end{column}
  \end{columns}
\end{frame}

\begin{frame}[fragile]%
  \frametitle{\only<1-2>{字体大小}\only<3>{字体样式}}
  \begin{columns}
    \begin{column}{0.4\textwidth}
      \only<1>{
        \includepdflarge{fontsize}
      }
      \only<2>{
        同样地,你也可以显式地设定字体大小,但是这种命令会更改行文设置,所以需要使用一个组来限定作用范围\footnotemark。
        \begin{center}
          \footnotesize
          \begin{stampbox}
            \begin{tabular}{rl}
              \cmd{tiny} & \tiny 极小 \\
              \cmd{scriptsize} & \scriptsize 抄本大小  \\
              \cmd{footnotesize} & \footnotesize 脚注大小 \\
              \cmd{small} & \small 小 \\
              \cmd{normalsize} & \normalsize 正常大小 \\
              \cmd{large} & \large 大 \\
              \cmd{huge} & \Huge 巨大 \\
            \end{tabular}
          \end{stampbox}
        \end{center}
      }
      \only<3>{
        也可以使用字体样式对应的更改字体设置的命令,这对于大段文段的设置而言也是很方便的。
        \begin{center}
          \footnotesize
          \begin{stampbox}
            \begin{tabular}{ll}
              \cmd{textrm} & \cmd{rmfamily}\\
              \cmd{texttt} & \cmd{ttfamily}\\
              \cmd{textsf} & \cmd{sffamily}\\
              \cmd{textbf} & \cmd{bfseries}\\
              \cmd{textit} & \cmd{itshape}\\
              \cmd{textsc} & \cmd{scshape}\\
              \cmd{textsl} & \cmd{slshape}\\
            \end{tabular}
          \end{stampbox}
        \end{center}
      }
    \end{column}
    \begin{column}{0.6\textwidth}
      \begin{codeblock}[]{大小}
\documentclass{ctexart}
\begin{document}
|\only<2>{\highlightline}|  {\bfseries\Large 一起向未来\par}
|\only<3>{\highlightline}|  {\itshape Together for a Shared Future}
\end{document}
      \end{codeblock}
    \end{column}
  \end{columns}
  \only<2>{\footnotetext{注意最后显式地使用 \cmd{par} 在改回大小前结束该段,否则会导致下一行的行间距异常!}}
\end{frame}

\section{逻辑结构}
\begin{frame}[fragile]
  \frametitle{列表}
  \begin{columns}
    \begin{column}{0.35\textwidth}
      \begin{codeblock}[]{无序列表}
\documentclass{ctexart}
\begin{document}
|\highlightline|  \begin{itemize}
    \item 第一项
    \item 第二项
    \item 第三项
|\highlightline|  \end{itemize}
\end{document}
      \end{codeblock}
    \end{column}
    \begin{column}{0.35\textwidth}
      \begin{codeblock}[]{有序列表}
\documentclass{ctexart}
\begin{document}
|\highlightline|  \begin{enumerate}
    \item 第一项
    \item 第二项
    \item 第三项
|\highlightline|  \end{enumerate}
\end{document}
      \end{codeblock}
    \end{column}
    \begin{column}{0.35\textwidth}
      \begin{codeblock}[]{描述列表}
\documentclass{ctexart}
\begin{document}
|\highlightline|  \begin{description}
    \item[||第一] 文本
    \item[||第二] 文本
    \item[||第三] 文本  
|\highlightline|  \end{description}
\end{document}
      \end{codeblock}
    \end{column}
  \end{columns}
\end{frame}

%更深的列表技巧,定理环境等

\begin{frame}
  \frametitle{列表}
  \begin{columns}
    \begin{column}{0.35\textwidth}
      \includepdflarge{unordered}
    \end{column}
    \begin{column}{0.35\textwidth}
      \includepdflarge{ordered}
    \end{column}
    \begin{column}{0.35\textwidth}
      \includepdflarge{description}
    \end{column}
  \end{columns}
\end{frame}

\begin{frame}[fragile,label=sectioning]%
  \frametitle{目次结构}
  \begin{columns}
    \begin{column}{0.4\textwidth}
      \LaTeX{} 可以使用目次命令将文档划分层级\footnotemark,并自动设定对应字体样式和大小。
      \begin{center}
        \begin{stampbox}
          \footnotesize
          \begin{tabular}{rll}
           命令 & 中文 & 层次 \\
           \cmd{chapter} & 章\footnotemark & \sout{0} \\
           \cmd{section} & 节 & 1 \\
           \cmd{subsection} & 小节 & 2 \\
           \cmd{subsubsection} & 小小节 & 3 \\
          \end{tabular}
        \end{stampbox}
      \end{center}
    \end{column}
    \begin{column}{0.6\textwidth}
      \begin{codeblock}[]{目次}
\documentclass{ctexart}
\begin{document}
|\highlightline|  \section{||概念}
|\highlightline|  \subsection{\LaTeX{}}
  \LaTeX{} 是一个用以排版高质量作品的文档准备系统。
\end{document}
      \end{codeblock}
    \end{column}
  \end{columns}
  \footnotetext{章这一级只在 \pkg{report} 和 \pkg{book} 文档类(包括对应的中文文档类)有定义。还有不常用的 \cmd{part} (0@\pkg{article}/-1@\pkg{report}\&\pkg{book}\&\pkg{beamer}) 以及更低层次的 \cmd{paragraph} (4) 与 \cmd{subparagraph} (5)。 }
\end{frame}

\begin{frame}[fragile]%
  \frametitle{组织文档}
  \begin{columns}
    \begin{column}{0.4\textwidth}
      \only<1>{
        \cmd{tableofcontents} 用来生成对于目次命令的目录。如果你想设定显示到哪个层级,在这个命令前使用 \cmd{setcounter\{tocdepth\}\{层次\}}
      }
      \only<2>{
        对于大型文档而言,使用多个文件管理源文件通常是更方便的。而 \cmd{include} 和 \cmd{input} 都以相对路径的方式插入其他 \TeX{} 文档。
        区别在于,\cmd{include} 命令会从新页开始并做一些内部调整,这基本上只对 \pkg{chapter} 这一级有用。而 \cmd{input} 会原样插入源代码。
      }
      \only<3>{
        但是 \cmd{include} 插入的文档可以使用 \cmd{includeonly} 管理当前要排印哪一部分的内容,利用所有章节辅助文件的同时,减少编译时间并专注于该部分的内容。
      }
    \end{column}
    \begin{column}{0.6\textwidth}
      \begin{codeblock}[]{主文档}
\documentclass{ctexrep}
|\only<3>{\highlightline}|\includeonly{learnlatex,sjtuthesis}
\begin{document}
|\only<1>{\highlightline}|  \tableofcontents
|\only<2-3>{\highlightline}|  % !TeX root = ..\..\latex-talk.tex

\part{学习 \LaTeX{}}
% FIXME: Part Page miniframe overflow
% FIXME: footnote fault numbering

\begin{frame}[plain]
  \vfil
  \begin{center}
    \href{https://learnlatex.org}{
      \rmfamily
      Learn\,\lower1ex\hbox{\Huge\LaTeX{}}.org
    }
  \end{center}
  \vfil
  \begin{center}
    \parbox{0.75\linewidth}{
      Learn\LaTeX{}.org\cite{learnlatex} 提供了解 \LaTeX{} 的 16 篇简短的教程,并包含了一些可以在线运行的示例,可以通过亲自动手查看实验效果。本部分主要参考由 C\TeX{}-org 提供的中文翻译版本 \link{https://github.com/CTeX-org/learnlatex.github.io/tree/zh-Hans/zh-Hans/}。
    }
  \end{center}
  \vfil
\end{frame}

{ % Start of shaded number logo

\newcommand{\shadedfont}[2][1pt]{
  % #1 (optional): shadow distance
  % #2: the text needed to be shaded
  \hbox{\rlap{\color{gray}\hskip#1#2}#2}
}
\newcounter{learnsec}
\setcounter{learnsec}{-1}
\newcommand{\updatelogo}{
  % update the logo corresponding to current counter.
  \stepcounter{learnsec}
  \logo{
    \raise.3ex\hbox{\tiny\insertsection}\shadedfont{\arabic{learnsec}}
  }
}
\let\oldsection=\section
\renewcommand{\section}[1]{\oldsection{#1}\updatelogo}

\section{是什么}
\begin{frame}
  \frametitle{\TeX{}}
  \begin{columns}[c]
    \begin{column}{0.7\textwidth}
      \begin{center}
        \rmfamily\Huge
        \hologo{La}\highlight[structure!70]{\TeX{}}
      \end{center}
      \begin{center}
        \parbox{0.75\textwidth}{
          \TeX{} 是由斯坦福大学教授高德纳
          (Donald E.~Knuth)于 1977 年开始开发的排版引擎。目前仍在更新,最新版本号为 3.141592653 \link{https://tug.org/TUGboat/tb42-1/tb130knuth-tuneup21.pdf}。
        }
      \end{center}
    \end{column}
    \begin{column}{0.3\textwidth}
      \includegraphics[width=.8\columnwidth]{Knuth.jpg}
    \end{column}
  \end{columns}
\end{frame}

\begin{frame}
  \frametitle{\LaTeX{}}
  \begin{columns}[c]
    \begin{column}{0.7\textwidth}
      \begin{center}
        \rmfamily\Huge
        \highlight[structure]{\LaTeX{}}
      \end{center}
      \begin{center}
        \parbox{0.75\textwidth}{
          \LaTeX{} 是最早在 1985 年由现就职于微软的 Leslie Lamport 开发的一种 \TeX{} \textbf{格式}\footnotemark,使用一些列宏和扩展宏包来简化 \TeX{} 的使用。现在由 \LaTeX{} Project 的成员维护。现在广泛使用的版本是 \LaTeXe{},最新的版本为 \LaTeX3(2020 年 10 月后默认内置)。
        }
      \end{center}
    \end{column}
    \begin{column}{0.3\textwidth}
      \includegraphics[width=.8\columnwidth]{Lamport.jpg}
    \end{column}
  \end{columns}
  \footnotetext{\hologo{ConTeXt} 也是一种 \TeX{} 格式 \link{https://www.contextgarden.net/}。}
\end{frame}

\begin{frame}
  \frametitle{程序}
  \begin{columns}[c]
    \begin{column}{0.7\textwidth}
      \begin{center}
        \rmfamily\Huge
        \highlight[structure]{\hologo{pdfLaTeX}}
      \end{center}
      \begin{center}
        \parbox{0.7\textwidth}{
          \hologo{pdfLaTeX} 是为了编译一个 \LaTeX{} 文档而运行的程序。实际上底层在运行一个叫 \hologo{pdfTeX} 的引擎,并预装了对应的 \LaTeX{} \textbf{格式}。为了利用临时文件,可能就需要多次运行程序。
        }
      \end{center}
    \end{column}
    \begin{column}{0.3\textwidth}
      \begin{block}{}
        \ttfamily\small
        > \highlight{pdflatex} main.tex\\
        This is pdfTeX, Version 3.141592653-
        2.6-1.40.23 (MiKTeX 21.10)\\
        entering extended mode\\
        \highlight{LaTeX2e} <2021-11-15>\\
        \highlight{L3} programming layer <2021-11-22>
      \end{block}
    \end{column}
  \end{columns}
\end{frame}

\begin{frame}
  \frametitle{引擎}
  \begin{columns}[c]
    \begin{column}{0.7\textwidth}
      \begin{center}
        \rmfamily\Huge
        \highlight[structure!70]{pdf}\hologo{La}\highlight[structure!70]{\TeX{}}
      \end{center}
      \begin{center}
        \parbox{0.7\textwidth}{
          pdf\TeX{} 是编译 \TeX{} 文档(以 \texttt{.tex} 结尾)的\textbf{引擎}---可以理解 \TeX{} 指令的\textbf{程序}。
        }
      \end{center}
    \end{column}
    \begin{column}{0.3\textwidth}
      \begin{block}{}
        \ttfamily\small
        > pdflatex main.tex\\
        This is \highlight[structure!70]{pdfTeX}, Version 3.141592653-
        2.6-1.40.23 (MiKTeX 21.10)
        entering extended mode\\
        LaTeX2e <2021-11-15>\\
        L3 programming layer <2021-11-22>
      \end{block}
    \end{column}
  \end{columns}
\end{frame}

\begin{frame}
  \frametitle{Unicode 引擎}
  \begin{table}
    \caption{主流 \hologo{(La)TeX} 程序
    \footnote{(u)p\TeX{} 是日语最常用的引擎,生成 \texttt{.dvi},支持 Unicode。}\footnote{Ap\TeX{} 具有底层 CJK 支持,内联 Ruby,Color Emoji。}}
    \footnotesize
    \begin{stampbox}
      \begin{tabular}{c>{\raggedright}*{3}{p{3.5cm}}}
        \alert{引擎}     & \hologo{pdfTeX}   & \hologo{XeTeX}   & \hologo{LuaTeX}   \\
        \alert{程序}     & \hologo{pdfLaTeX} & \hologo{XeLaTeX} & \hologo{LuaLaTeX} \\
        \alert{特点}     & 直接生成 PDF,支持 micro-typography  & 支持 Unicode、OpenType 与复杂文字编排 (CTL) & 支持 Unicode,内联 Lua,支持 OpenType \\
      \end{tabular}
    \end{stampbox}
  \end{table}

  \begin{center}
    \parbox{.9\textwidth}{
      \hologo{pdfLaTeX} 不支持 Unicode。为了排版中文,大部分情况下 \faApple{}\,\faLinux{} 应当使用 \hologo{XeLaTeX},而 \hologo{LuaLaTeX} 速度相对较慢。\faWindows{} 可以在一些情况下使用 \hologo{pdfLaTeX}。
    }
  \end{center}
\end{frame}

% \begin{frame}
%   \paragraph{\hologo{pdfLaTeX}} \TeX{} 和 \LaTeX{} 被广泛使用之前,它们只需内置支持欧洲语言即可。在 Unicode 出现之前,\LaTeX{} 提供了许多种\textbf{文件编码}来允许很多语言的文字以原生的方式输入,\hologo{pdfLaTeX} 也只需要使用 8 位文件编码和 8 位字体。
% \end{frame}

\section{跑起来}
\begin{frame}
  \frametitle{发行版}
  \begin{table}
    \caption{\hologo{TeX} 发行版}
    \footnotesize
    \begin{stampbox}
      \begin{tabular}{c>{\raggedright}*{3}{p{3.2cm}}}
        \alert{发行版}     & \hologo{MiKTeX} \link{https://mirrors.sjtug.sjtu.edu.cn/ctan/systems/win32/miktex/setup/windows-x64/basic-miktex-21.12-x64.exe}   & \TeX{} Live \link{https://mirrors.sjtug.sjtu.edu.cn/ctan/systems/texlive/tlnet/install-tl.zip}   & Mac\TeX{} \link{https://mirrors.sjtug.sjtu.edu.cn/ctan/systems/mac/mactex/mactex-20210328.pkg}  \\[2pt]
        \alert{特点}      &  只安装必要文件,依赖用时更新  &  所有平台均可使用,每年发布一次 & Mac 系统专用,对 \TeX{} Live 的进一步打包 \\
        \alert{推荐平台}  & \faWindows  & \faLinux &  \faApple \\
      \end{tabular}
    \end{stampbox}
  \end{table}
  \begin{center}
    \parbox{.9\textwidth}{
      要让 \LaTeX{} 跑起来,核心就是要有一套 \TeX{} 发行版,来获取让 \LaTeX{} 工作所需的一组程序和文件。参考《一份简短的关于 \LaTeX{} 安装的介绍》\link{https://mirrors.sjtug.sjtu.edu.cn/ctan/info/install-latex-guide-zh-cn/install-latex-guide-zh-cn.pdf} 安装想使用的发行版。推荐使用发行版的最新版本\footnote{老版本 Linux 系统的包管理器自带 \TeX{} Live 发行版可能不是最新的 \link{https://repology.org/project/texlive/versions},尽量使用镜像安装,并手动将相关环境变量添加到路径 \link{https://www.tug.org/texlive/doc/texlive-zh-cn/texlive-zh-cn.pdf}。},并使用国内镜像。
    }
  \end{center}
\end{frame}

\begin{frame}[plain]
  \hbox to \textwidth{
    \hfil
    \vbox to 3cm{
      \hbox{
        \resizebox{3cm}{!}{\includegraphics{\getcontribpath{sjtug}{vi/sjtug.pdf}}}
      }
    }
    \hfil
    \vbox to 3cm{
      \vfill
      \hbox{\Large\bfseries\color{cprimary} 稳定、快速、现代的镜像服务。}
      \vskip2pt
      \hbox{托管于华东教育网骨干节点上海交通大学。}
      \vfill
    }
    \hskip20pt
    \hfil
  }

  \begin{center}
    \parbox{0.8\textwidth}{
      推荐使用 SJTUG 软件镜像服务,离得近,下得快。
      
      \begin{description}
        \footnotesize
        \item[\TeX{} Live]  {\ttfamily tlmgr option repository https://mirrors.sjtug.sjtu.edu.cn/CTAN/systems/texlive/tlnet}
        \item[\hologo{MiKTeX}] 在 \hologo{MiKTeX} Console 中设置镜像源为 \url{https://mirrors.sjtug.sjtu.edu.cn}
      \end{description}
    }
  \end{center}
\end{frame}

\begin{frame}
  \frametitle{编辑器}
  \begin{table}
    \caption{开源编辑器推荐}
    \footnotesize
    \begin{stampbox}
      \begin{tabular}{c>{\raggedright}*{3}{p{3.5cm}}}
        \alert{编辑器}     & \begin{tabular}{c}Visual Studio Code\\ \LaTeX{} Workshop\end{tabular}  & \TeX{}studio & \TeX{}works \\[5pt]
        \alert{特点}      &  搭配 VS Code 使用非常方便,易扩展  & 可以使用大量的菜单选项输入代码块,用户友好 & 只提供基础的高亮与运行方法,发行版自带\footnote{Mac\TeX{} 打包的是 \TeX{}Shop 编辑器。} \\
      \end{tabular}
    \end{stampbox}
  \end{table}
  \begin{center}
    \parbox{.9\textwidth}{
      使用专为 \LaTeX{} 设计的编辑器将带来更多便利,因为它们往往会提供一键编译、内置 PDF 阅读器以及语法高亮等功能。几乎所有现代的 \LaTeX{} 编辑器都提供 Sync\TeX{} 这一强大的功能,以在 PDF 和 代码间对应跳转。
    }
  \end{center}
\end{frame}

\begin{frame}
  \frametitle{在线平台}
  \begin{table}
    \caption{在线协作平台推荐}
    \footnotesize
    \begin{stampbox}
      \begin{tabular}{c>{\raggedright}*{2}{p{4cm}}}
        \alert{在线平台}     & Overleaf \link{https://www.overleaf.com/}  & 交大 \LaTeX{} 助手 \link{https://latex.sjtu.edu.cn/} \\[2pt]
        \alert{特点}      & 最流行的在线平台,提供大量的模板,但国内访问慢 & 校内平台,隐私保护有保障,共享项目限制少 \\
      \end{tabular}
    \end{stampbox}
  \end{table}
  \begin{center}
    \parbox{.9\textwidth}{
      在线平台允许你直接在网页中编辑文档,无需本地安装即可在后台运行 \LaTeX{},并显示生成的 PDF。可以参照 Overleaf 官方文档学习如何使用在线平台 \link{https://www.overleaf.com/learn}。
    }
  \end{center}
\end{frame}

\section{基本结构}
\begin{frame}[fragile]%
  \frametitle{文档部件}
  \begin{columns}[c]
    \begin{column}{0.4\textwidth}
      \only<1>{
        \cmd{documentclass} 命令加载了\textbf{文档类}。\pkg{article} 是由 \LaTeX{}提供的用于排版短文档的基本文档类。
        \begin{description}
          \footnotesize
          \item[\pkg{article}] 不包含章的短文档
          \item[\pkg{report}] 含有章的单面印刷文档
          \item[\pkg{book}] 含有章的双面印刷文档
          \item[\pkg{beamer}] 制作幻灯片
        \end{description}
      }
      \only<2>{
        \env{document} 环境用于指示文档主体的范围。\LaTeX{} 还有其他的使用 \cmd{begin} 和 \cmd{end} 的搭配,我们称这些为\textbf{环境}。它们将用来设定局部格式命令\footnotemark。
      }
      \only<3>{
        \includepdflarge{enminimal}
      }
    \end{column}
    \begin{column}{0.6\textwidth}
      \begin{codeblock}[]{排版英文最简示例}
|\only<1>{\highlightline}|\documentclass{article}
|\only<2>{\highlightline}|\begin{document}
|\only<3>{\highlightline}|  Together for a Shared Future
|\only<2>{\highlightline}|\end{document}
      \end{codeblock}
    \end{column}
  \end{columns}
  \only<2>{\footnotetext{环境实际上是一个组,只不过通过语义化的形式预装了对应的格式命令。普通的组可以直接使用一对大括号之间的内容 \{$\cdots$\} 表示。}}
\end{frame}

\section{扩展}
\begin{frame}[fragile]%
  \frametitle{中文排版}
  \begin{columns}[c]
    \begin{column}{0.4\textwidth}
      \only<1>{
        \cmd{usepackage} 用于使用宏包以向 \LaTeX{} 添加或修改功能,需要在\textbf{导言区}调用。
        这里使用 \pkg{ctex} 宏集以获得中文支持。其调用底层因随不同的引擎而不同。
        {
          \footnotesize
          \begin{stampbox}
            \begin{tabular}{c*{3}{c}}
              \alert{引擎}     & \hologo{pdfTeX}   & \hologo{XeTeX}   & \hologo{LuaTeX}   \\
              \alert{程序}     & \hologo{pdfLaTeX} & \hologo{XeLaTeX} & \hologo{LuaLaTeX} \\
              \alert{宏包}     & CJK\footnotemark & xeCJK & luatexja \\
              \alert{封装}     & \multicolumn{3}{c}{ctex} \\
            \end{tabular}
          \end{stampbox}
        }
        \vspace{-1cm}
      }
      \only<2>{
        C\TeX{} 建议对于之前提到的常规文档类,最佳实践是使用该宏集提供的四种中文文档类,以对特定类型提供额外的中文排版适配。
        \begin{center}
          \begin{stampbox}
            \footnotesize
            \begin{tabular}{cc}
              \pkg{ctexart} & \pkg{ctexrep} \\
              \pkg{ctexbook} & \pkg{ctexbeamer} \\
            \end{tabular}
          \end{stampbox}
        \end{center}
      }
      \only<3>{
        \includepdflarge{cnminimal}
      }
      \only<4>{
        大部分情况下,你都不应当在 \LaTeX{} 中强制断行:你几乎只是想另起一段,或者是想在段落之间添加空行(使用 \pkg{parskip} 宏包就可实现)。
        只有\alert{很少的}情况下你需要使用 \textbackslash{}\textbackslash{} 来另起一行而不另起一段。
      }
    \end{column}
    \begin{column}{0.6\textwidth}
      \begin{codeblock}[]{排版中文\only<2->{最佳实践}}
|\only<2>{\highlightline}|\documentclass{|\only<1>{article}\only<2->{ctexart}|}
|\only<1>{\highlightline\textbackslash{}usepackage\{ctex\}\hfill\color{cprimary}\% 导言区}|
\begin{document}
|\only<3>{\highlightline}|    一起向未来
|\only<4>{\highlightline}|
  Together for a Shared Future
\end{document}
      \end{codeblock}
    \end{column}
  \end{columns}
  \only<1>{\footnotetext{ctex 在 \faApple\,\faLinux{} 上已经不可以使用 \hologo{pdfLaTeX} 编译,以及在 \faWindows{} 上使用该引擎也会变更自动间距调整等功能的默认行为。}}
\end{frame}

\section{设定格式}
\begin{frame}[fragile]%
  \frametitle{字体样式}
  \begin{columns}
    \begin{column}{0.4\textwidth}
      \only<1>{
        \includepdflarge{fontstyle}
      }
      \only<2>{
        可以使用显示样式设定命令对小段做加粗、斜体、等宽等等的处理。
        \begin{center}
          \footnotesize
          \begin{stampbox}
            \begin{tabular}{rl}
              \cmd{textrm} & \textrm{衬线} \\
              \cmd{textbf} & \textbf{加粗} \\
              \cmd{textit} & \kaishu 斜体 \\
              \cmd{texttt} & \texttt{等宽} \\
              \cmd{textsf} & \textsf{无衬线} \\
              \cmd{textsc} & \textsc{Small Caps} \\
              \cmd{textsl} & \textsl{Slanted} \\
            \end{tabular}
          \end{stampbox}
        \end{center}
      }
      \only<3>{
        与 Word 不同的是,\LaTeX{} 一般情况下并不需要使用上面的显式命令,而是采用逻辑标记的方法,
        比如 \cmd{emph} 可以强调文字,以及下面将要提到的目次命令(第 \ref{sectioning} 页)。
        这样可以统一管理格式。
      }
    \end{column}
    \begin{column}{0.6\textwidth}
      \begin{codeblock}[]{样式}
\documentclass{ctexart}
\begin{document}
|\only<2>{\highlightline}|  \textbf{||一起向未来}

|\only<3>{\highlightline}|  \emph{Together for a Shared Future}
\end{document}
      \end{codeblock}
    \end{column}
  \end{columns}
\end{frame}

\begin{frame}[fragile]%
  \frametitle{\only<1-2>{字体大小}\only<3>{字体样式}}
  \begin{columns}
    \begin{column}{0.4\textwidth}
      \only<1>{
        \includepdflarge{fontsize}
      }
      \only<2>{
        同样地,你也可以显式地设定字体大小,但是这种命令会更改行文设置,所以需要使用一个组来限定作用范围\footnotemark。
        \begin{center}
          \footnotesize
          \begin{stampbox}
            \begin{tabular}{rl}
              \cmd{tiny} & \tiny 极小 \\
              \cmd{scriptsize} & \scriptsize 抄本大小  \\
              \cmd{footnotesize} & \footnotesize 脚注大小 \\
              \cmd{small} & \small 小 \\
              \cmd{normalsize} & \normalsize 正常大小 \\
              \cmd{large} & \large 大 \\
              \cmd{huge} & \Huge 巨大 \\
            \end{tabular}
          \end{stampbox}
        \end{center}
      }
      \only<3>{
        也可以使用字体样式对应的更改字体设置的命令,这对于大段文段的设置而言也是很方便的。
        \begin{center}
          \footnotesize
          \begin{stampbox}
            \begin{tabular}{ll}
              \cmd{textrm} & \cmd{rmfamily}\\
              \cmd{texttt} & \cmd{ttfamily}\\
              \cmd{textsf} & \cmd{sffamily}\\
              \cmd{textbf} & \cmd{bfseries}\\
              \cmd{textit} & \cmd{itshape}\\
              \cmd{textsc} & \cmd{scshape}\\
              \cmd{textsl} & \cmd{slshape}\\
            \end{tabular}
          \end{stampbox}
        \end{center}
      }
    \end{column}
    \begin{column}{0.6\textwidth}
      \begin{codeblock}[]{大小}
\documentclass{ctexart}
\begin{document}
|\only<2>{\highlightline}|  {\bfseries\Large 一起向未来\par}
|\only<3>{\highlightline}|  {\itshape Together for a Shared Future}
\end{document}
      \end{codeblock}
    \end{column}
  \end{columns}
  \only<2>{\footnotetext{注意最后显式地使用 \cmd{par} 在改回大小前结束该段,否则会导致下一行的行间距异常!}}
\end{frame}

\section{逻辑结构}
\begin{frame}[fragile]
  \frametitle{列表}
  \begin{columns}
    \begin{column}{0.35\textwidth}
      \begin{codeblock}[]{无序列表}
\documentclass{ctexart}
\begin{document}
|\highlightline|  \begin{itemize}
    \item 第一项
    \item 第二项
    \item 第三项
|\highlightline|  \end{itemize}
\end{document}
      \end{codeblock}
    \end{column}
    \begin{column}{0.35\textwidth}
      \begin{codeblock}[]{有序列表}
\documentclass{ctexart}
\begin{document}
|\highlightline|  \begin{enumerate}
    \item 第一项
    \item 第二项
    \item 第三项
|\highlightline|  \end{enumerate}
\end{document}
      \end{codeblock}
    \end{column}
    \begin{column}{0.35\textwidth}
      \begin{codeblock}[]{描述列表}
\documentclass{ctexart}
\begin{document}
|\highlightline|  \begin{description}
    \item[||第一] 文本
    \item[||第二] 文本
    \item[||第三] 文本  
|\highlightline|  \end{description}
\end{document}
      \end{codeblock}
    \end{column}
  \end{columns}
\end{frame}

%更深的列表技巧,定理环境等

\begin{frame}
  \frametitle{列表}
  \begin{columns}
    \begin{column}{0.35\textwidth}
      \includepdflarge{unordered}
    \end{column}
    \begin{column}{0.35\textwidth}
      \includepdflarge{ordered}
    \end{column}
    \begin{column}{0.35\textwidth}
      \includepdflarge{description}
    \end{column}
  \end{columns}
\end{frame}

\begin{frame}[fragile,label=sectioning]%
  \frametitle{目次结构}
  \begin{columns}
    \begin{column}{0.4\textwidth}
      \LaTeX{} 可以使用目次命令将文档划分层级\footnotemark,并自动设定对应字体样式和大小。
      \begin{center}
        \begin{stampbox}
          \footnotesize
          \begin{tabular}{rll}
           命令 & 中文 & 层次 \\
           \cmd{chapter} & 章\footnotemark & \sout{0} \\
           \cmd{section} & 节 & 1 \\
           \cmd{subsection} & 小节 & 2 \\
           \cmd{subsubsection} & 小小节 & 3 \\
          \end{tabular}
        \end{stampbox}
      \end{center}
    \end{column}
    \begin{column}{0.6\textwidth}
      \begin{codeblock}[]{目次}
\documentclass{ctexart}
\begin{document}
|\highlightline|  \section{||概念}
|\highlightline|  \subsection{\LaTeX{}}
  \LaTeX{} 是一个用以排版高质量作品的文档准备系统。
\end{document}
      \end{codeblock}
    \end{column}
  \end{columns}
  \footnotetext{章这一级只在 \pkg{report} 和 \pkg{book} 文档类(包括对应的中文文档类)有定义。还有不常用的 \cmd{part} (0@\pkg{article}/-1@\pkg{report}\&\pkg{book}\&\pkg{beamer}) 以及更低层次的 \cmd{paragraph} (4) 与 \cmd{subparagraph} (5)。 }
\end{frame}

\begin{frame}[fragile]%
  \frametitle{组织文档}
  \begin{columns}
    \begin{column}{0.4\textwidth}
      \only<1>{
        \cmd{tableofcontents} 用来生成对于目次命令的目录。如果你想设定显示到哪个层级,在这个命令前使用 \cmd{setcounter\{tocdepth\}\{层次\}}
      }
      \only<2>{
        对于大型文档而言,使用多个文件管理源文件通常是更方便的。而 \cmd{include} 和 \cmd{input} 都以相对路径的方式插入其他 \TeX{} 文档。
        区别在于,\cmd{include} 命令会从新页开始并做一些内部调整,这基本上只对 \pkg{chapter} 这一级有用。而 \cmd{input} 会原样插入源代码。
      }
      \only<3>{
        但是 \cmd{include} 插入的文档可以使用 \cmd{includeonly} 管理当前要排印哪一部分的内容,利用所有章节辅助文件的同时,减少编译时间并专注于该部分的内容。
      }
    \end{column}
    \begin{column}{0.6\textwidth}
      \begin{codeblock}[]{主文档}
\documentclass{ctexrep}
|\only<3>{\highlightline}|\includeonly{learnlatex,sjtuthesis}
\begin{document}
|\only<1>{\highlightline}|  \tableofcontents
|\only<2-3>{\highlightline}|  \include{learnlatex}
|\only<3>{\highlightline}|  \include{sjtuthesis}
\end{document}
      \end{codeblock}
    \end{column}
  \end{columns}
\end{frame}

\begin{frame}[fragile]
  \frametitle{组织文档}
  \begin{columns}
    \begin{column}{0.4\textwidth}
      \begin{codeblock}[]{learnlatex.tex}
|\highlightline|\chapter{||学习 \LaTeX{}}
\section{||概念}
\subsection{\LaTeX{}}
\LaTeX{} 是一个用以排版高质量作品的文档准备系统。
      \end{codeblock}
      子文件中就不需要添加 \env{document} 环境了\footnotemark。
    \end{column}
    \begin{column}{0.6\textwidth}
      \begin{codeblock}[]{主文档}
|\highlightline|\documentclass{ctexrep}
\includeonly{learnlatex,sjtuthesis}
\begin{document}
  \tableofcontents
  \include{learnlatex}
  \include{sjtuthesis}
\end{document}
      \end{codeblock}
    \end{column}
  \end{columns}
  \footnotetext{如果想强制指定子文档的主文档,可以在文件第一行输入魔术命令:\texttt{\% !TeX root = main.tex}}
\end{frame}

\section{图}
\begin{frame}[fragile]%
  \frametitle{\temporal<5>{插图}{浮动体}{插图}}
  \begin{columns}
    \begin{column}{0.6\textwidth}
      \begin{codeblock}[]{插入单图\only<4->{最佳实践}}
\documentclass{ctexart}
|\only<2>{\highlightline}|\usepackage{graphicx}
|\only<2>{\highlightline}|\graphicspath{{figs/}{pics/}}
\begin{document}
|\only<5>{\highlightline}|\begin{figure}[ht]
|\only<6>{\highlightline}|  \centering
|\only<3>{\highlightline}|  \includegraphics[width=|\only<1-3>{4cm}\only<4->{0.4\textbackslash{}textwidth}|]{sjtug}
|\only<7>{\highlightline}|  \caption{SJTUG 徽标}\label{fig:sjtug}
|\only<5>{\highlightline}|\end{figure}
\end{document}
      \end{codeblock}
    \end{column}
    \begin{column}{0.4\textwidth}
      \only<1>{
        \includepdflarge{insertimage}
      }
      \only<2>{
        为了插入外部图片,需要使用 \pkg{graphicx} 宏包。之后在文档主体便可以使用 \cmd{includegraphics} 插入图片。导言区也可以加入 \cmd{graphicspath} 指定图片文件夹\footnotemark。
      }
      \only<3>{
        \cmd{includegraphics} 命令便以相对路径的方式插入图片,如果无同名图片,那么后缀名可以省略。可以使用可选参数指定插入的图片尺寸,最佳实践是使用 \cmd{textwidth} 或 \cmd{linewidth} 的相对值指定尺寸大小,以在未来可能的布局更改中保留一定的灵活性。
      }
      \only<4>{
        也可以通过键值对的方法设置图片的其他属性。
        \begin{center}
          \footnotesize
          \begin{stampbox}
            \begin{tabular}{rl}
              \pkg{width} & 宽度 \\
              \pkg{height} & 高度 \\
              \pkg{scale} & 缩放 \\
              \pkg{angle} & 角度 \\
            \end{tabular}
          \end{stampbox}
        \end{center}
      }
      \only<5>{
        \env{figure} 为一个浮动体环境(\env{table} 也是),可以将其移动到其他位置上以减少行文中的空白。可以添加可选参数以指定如何放置浮动体,最多可以使用四种位置描述符:
        \begin{center}
          \footnotesize
          \begin{stampbox}
            \begin{tabular}{cll}
              \pkg{h} & Here & 尽可能在这里 \\
              \pkg{t} & Top & 页面顶部 \\
              \pkg{b} & Bottom & 页面底部 \\
              \pkg{p} & Page & 浮动体专页 \\
            \end{tabular}
          \end{stampbox}
        \end{center}
        还可以只使用 \pkg{float} 宏包提供的 \pkg{H} 描述符以强制置于此处。
      }
      \only<6>{
        采用 \cmd{centering} 命令而不是 \env{center} 环境来水平居中图片。这将避免多余的纵向间距。
      }
      \only<7>{
        使用 \cmd{caption} 命令输入题注,如果这个命令写在插入图片前面,题注将会在上方(中文中一般对 \env{table} 环境这么做)。后面将会看到如何对留有标记(\cmd{label})的图片进行引用。
      }
    \end{column}
  \end{columns}
  \only<2>{\footnotetext{其命令参数每个为一个以 \texttt{/} 结尾的文件夹,每个文件夹需要使用大括号包裹起来。}}
\end{frame}

\begin{frame}[fragile]
  \begin{columns}
    \begin{column}{0.6\textwidth}
      \begin{codeblock}[]{插入双图}
\documentclass{ctexart}
\usepackage{graphicx}
\graphicspath{{figs/}{pics/}}
\begin{document}
  \begin{figure}[ht]
|\only<1>{\highlightline}|    \begin{minipage}{0.48\textwidth}
      \centering
      \includegraphics[height=2cm]{sjtug}
|\only<2>{\highlightline}|      \caption{SJTUG 徽标}\label{fig:sjtug}
|\only<1>{\highlightline}|    \end{minipage}\hfill
|\only<1>{\highlightline}|    \begin{minipage}{0.48\textwidth}
      \centering
      \includegraphics[height=2cm]{sjtugt}
|\only<2>{\highlightline}|      \caption{SJTUG||文字}\label{fig:sjtugt}
|\only<1>{\highlightline}|    \end{minipage}
  \end{figure}
\end{document}
      \end{codeblock}
    \end{column}
    \begin{column}{0.4\textwidth}
      \only<1>{
        在 \env{figure} 环境里使用 \env{minipage} 小页制作列盒子,以并排两图并分别编号,需要设定强制参数以指定列宽。两个小页之间添加 \cmd{hfill} 使两个小页两端对齐。
      }
      \only<2>{
        在每个小页内部分别使用 \cmd{caption} 添加标签。
      }
      \only<3>{
        \includepdflarge{doubleimages}
      }
    \end{column}
  \end{columns}
\end{frame}

\begin{frame}[fragile]%
  \begin{columns}
    \begin{column}{0.6\textwidth}
      \begin{codeblock}[]{}
\documentclass{ctexart}
\usepackage{graphicx}
|\highlightline|\usepackage{subcaption}
\graphicspath{{figs/}{pics/}}
\begin{document}
  \begin{figure}[ht]
|\highlightline|    \begin{subfigure}{0.48\textwidth}
      \centering
      \includegraphics[height=2cm]{sjtug}
      \caption{||徽标}
|\highlightline|    \end{subfigure}\hfill
|\highlightline|    \begin{subfigure}{0.48\textwidth}
      \centering
      \includegraphics[height=2cm]{sjtugt}
      \caption{||文字}
|\highlightline|    \end{subfigure}
    \caption{SJTUG}\label{fig:sjtug}
  \end{figure}
\end{document}
      \end{codeblock}
    \end{column}
    \begin{column}{0.4\textwidth}
      \includepdflarge{subfigures}\vspace{15pt}
      \pkg{subcaption} 宏包提供了 \env{subfigure} 环境(以及 \env{subtable}),可以用于以子图的形式插入,编号会增加一级。也可以为子图添加子集引用编号。
    \end{column}
  \end{columns}
\end{frame}

\section{表}
\begin{frame}[fragile]
  \frametitle{简单表格}
  \begin{columns}
    \begin{column}{0.6\textwidth}
      \begin{codeblock}[]{}
\documentclass{ctexart}
|\only<1-2>{\highlightline}|\usepackage{|\temporal<1>{array}{\highlight{array}}{array},\temporal<2>{booktabs}{\highlight{booktabs}}{booktabs}|}
\begin{document}
\begin{table}[ht]
  \centering
  \caption{||北京冬奥会吉祥物}
|\only<1>{\highlightline}|  \begin{tabular}{lp{3cm}}
|\only<2>{\highlightline}|    \toprule
|\only<3>{\highlightline}|吉祥物 & 描述                          \\
|\only<2>{\highlightline}|    \midrule
|\only<3>{\highlightline}|冰墩墩 & 2022 年北京冬季奥运会吉祥物  \\
|\only<3>{\highlightline}|雪容融 & 2022 年北京冬季残奥会吉祥物  \\
|\only<2>{\highlightline}|    \bottomrule
|\only<1>{\highlightline}|  \end{tabular}
\end{table}
\end{document}
      \end{codeblock}
    \end{column}
    \begin{column}{0.4\textwidth}
      \only<1>{
        使用 \env{tabular} 环境绘制表格。需要添加参数(称为\textbf{表格导言})以确定每一列的对齐方式。引入 \pkg{array} 宏包来使用更多的\textbf{引导符}。
        \begin{center}
          \footnotesize
          \begin{stampbox}
            \begin{tabular}{>{\ttfamily}ll}
              \alert{l} & 向左对齐 \\
              \alert{c} & 居中对齐 \\
              \alert{r} & 向右对齐 \\
              \alert{p\{3cm\}} & 固定列宽,两端对齐 \\
              \alert{m\{3cm\}} & \texttt{p} + 垂直居中对齐 \\
              \alert{>\{\textbackslash{}bfseries\}} & 后一列单元格前加命令 \\
              \alert{*\{3\}\{l\}} & 三个左对齐列 \\
            \end{tabular}
          \end{stampbox}
        \end{center}
      }
      \only<2>{
        \pkg{booktabs} 宏包提供了标准三线表格所需要的行分割线:\cmd{toprule},\cmd{midrule},\cmd{bottomrule}。也可以使用 \cmd{cmidrule\{1-2\}} 来部分地绘制行分割线。一般不推荐在专业表格中使用纵向分割线。
      }
      \only<3>{
        每行内容使用 \textbackslash\textbackslash{} 分隔开,每行中的单元格使用 \& 分隔开。
      }
      \only<4>{
        \includepdflarge{table}
      }
    \end{column}
  \end{columns}
\end{frame}

\begin{frame}[fragile]%
  \begin{columns}
    \begin{column}{0.6\textwidth}
      \begin{codeblock}[]{表头居中}
\documentclass{ctexart}
\usepackage{array,booktabs}
\begin{document}
\begin{table}[ht]
  \centering
  \caption{||北京冬奥会吉祥物}
  \begin{tabular}{lp{3cm}}
    \toprule
|\highlightline|\multicolumn{1}{c}{||吉祥物} &
|\highlightline|\multicolumn{1}{c}{||描述} \\
    \midrule
||冰墩墩 & 2022 年北京冬季奥运会吉祥物  \\
||雪容融 & 2022 年北京冬季残奥会吉祥物  \\
    \bottomrule
  \end{tabular}
\end{table}
\end{document}
      \end{codeblock}
    \end{column}
    \begin{column}{0.4\textwidth}
      \cmd{multicolumn} 命令不仅可以用于合并同行的单元格,还可以用于临时地屏蔽表格导言对该列的对齐方式定义。这里用于居中表头。
      \begin{center}
        \begin{stampbox}
          \parbox{0.85\linewidth}{
            \ttfamily\color{blue}\textbackslash{}multicolumn\{格数\}\{对齐方式\}\{内容\}
          }
        \end{stampbox}
      \end{center}
      跨页表格考虑使用 \pkg{longtable} 宏包。带标注的表格可以考虑使用 \pkg{threeparttable} 宏包。考虑使用在线工具生成表格代码 \link{https://www.tablesgenerator.com/latex_tables}。
    \end{column}
  \end{columns}
\end{frame}

\section{数学公式}
\begin{frame}
  \frametitle{数学模式}
  \begin{alertblock}{}
  输入数学公式是 \LaTeX{} 的绝对强项,很多常见的公式服务依赖于一些轻量级渲染引擎比如 MathJax, K\kern-.3ex\raise.4ex\hbox{\footnotesize A}\kern-.3ex\TeX{}。但是它们实际上使用的是 \LaTeX{} 语法变种,也就是说并没有使用 \LaTeX{} 后端。所以不要期望有完全一致的输出。
  \end{alertblock}
  
  为了更好的获得数学公式输入支持,请使用 \hologo{AmS}math 宏包。数学模式分为两种:
  \begin{description}
    \item[行内模式] 一般通过一对美元符号(\$$\cdots$\$)标记,可以使用编辑器内建的符号表输入数学符号,也可以使用在线工具手写识别 \link{https://detexify.kirelabs.org/classify.html}。
    \item[行间模式] 一般通过 \env{equation} 环境\footnote{这是有编号环境,其加星号的变种 \env{equation*} 用于生成无编号环境。}输入。如果需要使用多行公式,请使用 \env{align} 环境,并按照类似表格输入的方式,使用 \& 对齐符号,\textbackslash\textbackslash{} 换行。如果不想手动居中,可以考虑多行自动居中的 \env{gather} 和单个大型公式首尾两端对齐 \env{multline}。
  \end{description}
\end{frame}

\begin{frame}
  \frametitle{数学命令展示}
  \begin{columns}
    \begin{column}{0.33\textwidth}
      \begin{exampleblock}{}
        $PV=nRT$
      \end{exampleblock}
      \begin{exampleblock}{}
        $\sum_{i=1}^ki^2=\frac{n(n+1)(2n+1)}{6}$
      \end{exampleblock}
      \begin{exampleblock}{}
        $T(n) = aT\left(\left\lceil\frac{n}{b}\right\rceil\right) + \mathcal{O}(n^d)$
      \end{exampleblock}
      \begin{exampleblock}{}
        $\frac{x_{1}+x_{2}+x_{3}}{3}\geq \sqrt[3]{x_{1}x_{2}x_{3}}$
      \end{exampleblock}
      \begin{exampleblock}{}
        $n=(\underbrace{1\cdots 1}_{k\text{ of 1's}})_2=2^{k+1}-1$
      \end{exampleblock}
      \begin{exampleblock}{}
        $\nabla f (P)= \left.\left(\frac{\partial f}{\partial x},\frac{\partial f}{\partial y},\frac{\partial f}{\partial z}\right)\right|_{P}$
      \end{exampleblock}
    \end{column}
    \begin{column}{0.67\textwidth}
      \begin{exampleblock}{}
        \begin{equation*}
          \int_{a}^b f(x)\,\mathrm{d}x=\lim_{|P|\rightarrow 0}\sum_{i=1}^n f(\xi_i)\Delta x_i
        \end{equation*}
      \end{exampleblock}
      \begin{exampleblock}{}
        \begin{equation}
          T(n) = \begin{cases}
            \mathcal{O}(n^d),&\textrm{if } d>\log_b a, \\
            \mathcal{O}(n^d\log n), &\textrm{if } d=\log_b a,\\
            \mathcal{O}(n^{\log_b a}), &\textrm{if } d<\log_b a.
          \end{cases}
        \end{equation}
      \end{exampleblock}
      \begin{exampleblock}{}
        \begin{align}
          Q^{T}A&=R \\
          \begin{pmatrix}
            q_1^T \\ q_2^T \\ q_3^T
          \end{pmatrix}
          \begin{pmatrix}
            a_1 & a_2 & a_3
          \end{pmatrix}
          &=R
        \end{align}
      \end{exampleblock}
    \end{column}
  \end{columns}
\end{frame}

%更深入地讲解 mathtools, unicode-math, siunix

\section{引用}
\begin{frame}[fragile]
  \frametitle{交叉引用}
  \only<1>{
    正如之前所提到的,\LaTeX{} 中使用 \cmd{label} 标记,然后可以使用 \cmd{ref} 来引用这个标记。 \cmd{label} 可以放在使用计数器的对象之后。
  }
  \only<2>{
    为了使得对公式编号的引用带有括号,推荐使用 \hologo{AmS}math 宏包中的 \cmd{eqref} 命令。对于多行公式环境,每一个换行符前都可以添加一个 \cmd{label} 用于引用该行公式。
  }
  \begin{columns}
    \begin{column}{0.5\textwidth}
      \begin{codeblock}[]{图}
\begin{figure}
|\only<1>{\highlightline}|  \caption{||示例}\label{fig:example}
\end{figure}
      \end{codeblock}
      \begin{codeblock}[]{表}
\begin{table}
|\only<1>{\highlightline}|  \caption{||示例}\label{tab:example}
\end{table}
      \end{codeblock}
    \end{column}
    \begin{column}{0.5\textwidth}
\begin{codeblock}[]{目次}
|\only<1>{\highlightline}|\section{||示例}\label{sec:example}
\end{codeblock}

\begin{codeblock}[]{公式}
\begin{equation}
  a = b + c
|\only<1>{\highlightline}|\label{eq:example}
\end{equation}
|\only<2>{\highlightline}|如公式 \eqref{eq:example} 所示,
\end{codeblock}
    \end{column}
  \end{columns}
\end{frame}

\begin{frame}[fragile]
  \frametitle{文献引用}
  \LaTeX{} 管理参考文献可以采用专用数据库文件 \texttt{.bib},很多的文献管理文件比如 EndNote \link{https://lic.sjtu.edu.cn/Default/softshow/tag/MDAwMDAwMDAwMLGImKE}, Zotero \link{https://www.zotero.org/}, JabRef \link{https://www.jabref.org/} 都可以直接导出这种格式的文件用于 \LaTeX{} 论文中的引用。一般不需要手写数据库文件,你只需要注意每一个文献会在数据库中有一个主键,这个类似于上文的 \cmd{label} 标记,只是要引用数据库中的文献需要使用 \cmd{cite} 命令。
  
  \begin{codeblock}[]{ref.bib}
|\highlightline|@phdthesis{devoftech,|\hfill\alert{\% 类型为博士论文,主键为\texttt{devoftech}}|
  title={||新时期我国信息技术产业的发展},
  author={||江泽民},
  year={2008}
}
  \end{codeblock}
\end{frame}

\begin{frame}
  \frametitle{文献引用}
  而让 \LaTeX{} 处理 \texttt{.bib} 数据库文件与引用有两种工作流。你可能需要查清楚如何在编辑器中设置对应的工作流,或者采用后面所提到的高级编译方式 \texttt{latexmk}。
  \begin{columns}
    \begin{column}{0.5\textwidth}
      \begin{block}{\hologo{BibTeX} + \pkg{gbt7714}}
        一般期刊提交使用这种方法,\pkg{natbib} 宏包将提供命令 \cmd{citet}(文本引用) 和 \cmd{citep}(括号引用)。中文引用可以直接使用 \pkg{gbt7714} 宏包,但是角模式和正文模式不能混用。
      \end{block}
    \end{column}
    \begin{column}{0.5\textwidth}
      \begin{block}{\hologo{biber} + \pkg{biblatex}}
        这是更容易自定义的方法,与 \hologo{BibTeX} 的运作方式稍有不同。\pkg{biblatex} 提供了更加智能的引用命令。而中文引用可以使用 \pkg{biblatex} 宏包的样式 \pkg{gb7714-2015},使用该样式需要使用 \hologo{XeLaTeX} 编译。
      \end{block}
    \end{column}
  \end{columns}
\end{frame}

\begin{frame}[fragile]
  \frametitle{文献引用}
  \begin{columns}
    \begin{column}{0.5\textwidth}
      \begin{codeblock}[]{\hologo{BibTeX} + \pkg{gbt7714}}
\documentclass{ctexart}
\usepackage{gbt7714}
\bibliographystyle{gbt7714-numerial}
% \citestyle{numbers}  % 正文模式
\begin{document}
  ||他指出了...\cite{devoftech}
  \bibliography{ref}
\end{document}
      \end{codeblock}
    \end{column}
    \begin{column}{0.5\textwidth}
      \begin{codeblock}[]{\hologo{biber} + \pkg{biblatex}}
\documentclass{ctexart}
\usepackage[backend=biber,style=gb7714-2015]{biblatex}
\addbibresource{ref.bib}
\begin{document}
  ||他在文献 \parencite{devoftech}
  ||指出了...\cite{devoftech}
  \printbibliography
\end{document}
      \end{codeblock}
    \end{column}
  \end{columns}
\end{frame}

\begin{frame}
  \frametitle{文献引用}
  \begin{columns}
    \begin{column}{0.5\textwidth}
      \includepdflarge{bibtex}
    \end{column}
    \begin{column}{0.5\textwidth}
      \includepdflarge{biblatex}
    \end{column}
  \end{columns}
\end{frame}

} % End of customized shaded number logo

|\only<3>{\highlightline}|  % !TeX root = ..\..\latex-talk.tex

\part{SJTUThesis}

\begin{frame}
  \frametitle{简介}
  \begin{columns}
    \begin{column}{0.6\textwidth}
      \begin{itemize}
        \item 最早由韦建文于 2009 年 11 月发布 0.1a 版,2018 年起由 SJTUG 接手维护
        \item 最新版:\SJTUThesisVersion{} (\SJTUThesisDate)
        \item 支持本科、硕士、博士学位论文以及课程论文的排版
      \end{itemize}
    \end{column}
    \begin{column}{0.4\textwidth}
      \begin{exampleblock}{}
        \begin{minipage}[c]{1cm}
          \includegraphics[width=0.8cm]{\getcontribpath{sjtug}{vi/sjtug}}
        \end{minipage}
        \begin{minipage}[c]{2cm}
          \href{https://github.com/sjtug}{sjtug}/\href{https://github.com/sjtug/SJTUThesis}{SJTUThesis}
        \end{minipage}
      \end{exampleblock}
      \vspace{-8pt}
      \begin{block}{}
        \scriptsize
        上海交通大学 \hologo{XeLaTeX} 学位论文及课程论文模板 | Shanghai Jiao Tong University \hologo{XeLaTeX} Thesis Template
      \end{block}
      \vspace{-8pt}
      \begin{alertblock}{}
        \scriptsize
        \begin{tabular}{cl}
          \faStar & 2.4k \\
          \faEye & 55 \\
          \faCodeBranch & 701 \\
        \end{tabular}
      \end{alertblock}
    \end{column}
  \end{columns}
\end{frame}

\begin{frame}
  \frametitle{下载与编译}
  \alert{下载} 推荐安装 Git \link{https://git-scm.com/} 后,克隆 SJTUG 镜像仓库
  \begin{exampleblock}{\faGit*}
    \ttfamily\small
    git clone https://mirror.sjtu.edu.cn/git/SJTUThesis.git/
  \end{exampleblock}

  \alert{编译} 推荐使用 \pkg{latexmk} 编译\footnote{\hologo{MiKTeX} 用户需要手动安装 Perl 解释器 \link{https://www.perl.org/get.html} 才能使用 \pkg{latexmk}。},在不能够利用自带的 \texttt{.latexmkrc} 配置文件的情况下,需要查清楚在对应的编辑器中如何使用 \hologo{XeLaTeX} + \hologo{biber} 编译 \link{https://github.com/sjtug/SJTUThesis/blob/master/README.md}。
  \begin{exampleblock}{\faTerminal}
    \ttfamily\small
    latexmk -xelatex main
  \end{exampleblock}

  Overleaf 用户可以下载压缩包后,上传并采用 \hologo{XeLaTeX} 编译方式。
\end{frame}

\begin{frame}
  \frametitle{手动编译}
  \alert{第一次编译失败} 如果没有办法通过通常方式编译成功,请尝试使用文件夹内附带 \faLinux{}\,\faApple{} \texttt{Makefile} 和 \faWindows{} \texttt{Compile.bat} 进行编译。

  \alert{统计字数} 编写过程中也可以使用对应的命令调用 \TeX{}count 来统计正文字数。
  \begin{columns}
    \begin{column}{0.38\textwidth}
      \begin{exampleblock}{\faLinux{}\,\faApple}
        \ttfamily
        make all\\
        make clean\\
        make cleanall\\
        make wordcount
      \end{exampleblock}
    \end{column}
    \begin{column}{0.38\textwidth}
      \begin{exampleblock}{\faWindows}
        \ttfamily
        ./Compile.bat thesis\\
        ./Compile.bat clean\\
        ./Compile.bat cleanall\\
        ./Compile.bat wordcount
      \end{exampleblock}
    \end{column}
    \begin{column}{0.24\textwidth}
      \begin{block}{\faInfo}
        \ttfamily
        编译论文\\
        清理中间文件\\
        $\hookrightarrow +$删除论文\\
        统计字数
      \end{block}
    \end{column}
  \end{columns}
\end{frame}

\begin{frame}[label=compile]
  \frametitle{编译问题排查}
  \begin{columns}
    \begin{column}{0.33\textwidth}
      \begin{alertblock}{无法使用 \texttt{latexmk}\thesisissue{578}}
        \hologo{MiKTeX} 需要安装 Perl 解释器。
      \end{alertblock}  
      \begin{alertblock}{C\TeX{} 套装无法编译\thesisissue{446}}
        使用最新 \TeX{} 发行版。
      \end{alertblock}
      \begin{alertblock}{\hologo{pdfLaTeX} 无法编译\thesisissue{444}}
        请使用 \texttt{latexmk},或更改编辑器设置以 \hologo{XeLaTeX} 编译。
      \end{alertblock}
    \end{column}
    \begin{column}{0.33\textwidth}
      \begin{alertblock}{缺少字体\thesisissue{564} \thesisdiscuss{598}}
        更换字体集,或者安装对应字体。
      \end{alertblock}
      \begin{alertblock}{缺少汉字\thesisissue{533} \thesisdiscuss{617}}
        去除使用 fandol 字体集的设定。或者是安装字体后,改用 \texttt{fontset=adobe} 或 \texttt{fontset=founder}。
      \end{alertblock}
    \end{column}
    \begin{column}{0.33\textwidth}
      \begin{block}{\faInfoCircle{} README}
        不同编辑器的设置请首先参阅 README \link{https://github.com/sjtug/SJTUThesis/blob/master/README.md} 文档。
      \end{block}
      \begin{block}{\faBookOpen{} Wiki}
        其他编译问题推荐查阅 Wiki \link{https://github.com/sjtug/SJTUThesis/wiki} 的使用说明部分。
      \end{block}
    \end{column}
  \end{columns}
\end{frame}

\begin{frame}[fragile, label=covers]
  \begin{codeblock}[firstnumber=3]{main.tex}
|\alert{\% 载入 SJTUThesis 模版}|
\documentclass[|\only<1>{\highlight{type}}\only<2>{type}|=|\only<1>{bachelor}\only<2>{\highlight{bachelor}}|]{sjtuthesis}
  \end{codeblock}
  \begin{figure}
    \parbox{0.9\textwidth}{
      \begin{subfigure}{0.20\textwidth}
        \framebox{\includegraphics[width=\linewidth]{support/thesis/bachelor}}
        \caption{\only<1>{学士}\only<2>{\texttt{bachelor}}}
      \end{subfigure}\hfill
      \begin{subfigure}{0.20\textwidth}
        \framebox{\includegraphics[width=\linewidth]{support/thesis/master}}
        \caption{\only<1>{硕士}\only<2>{\texttt{master}}}
      \end{subfigure}\hfill
      \begin{subfigure}{0.20\textwidth}
        \framebox{\includegraphics[width=\linewidth]{support/thesis/doctor}}
        \caption{\only<1>{博士}\only<2>{\texttt{doctor}}}
      \end{subfigure}\hfill
      \begin{subfigure}{0.20\textwidth}
        \framebox{\includegraphics[width=\linewidth]{support/thesis/course}}
        \caption{\only<1>{课程}\only<2>{\texttt{course}}}
      \end{subfigure}
      \caption{论文类型示例\only<2>{ \texttt{type}}}
    }
  \end{figure}
\end{frame}

\begin{frame}[fragile]
  \frametitle{文档类选项}
  % \framesubtitle{\textbackslash{}documentclass\{sjtuthesis\}}
  \begin{columns}
    \begin{column}{0.45\textwidth}
      \includegraphics[page=10]{thesisdir}
    \end{column}
    \begin{column}{0.55\textwidth}
      \begin{table}[H]
        \caption{文档类选项}
        \footnotesize
        \begin{tabular}{>{\ttfamily}rll}
          \toprule
          选项 & 含义 & 相关 \\
          \midrule
          type= & 指定论文类型 & 第 \ref{covers} 页\\
          fontset= & 指定字体 & 第 \ref{compile} 页\\
          \midrule
          review & 开启盲审模式 & \thesisissue{195} \thesisissue{686} \\
          twoside & 双页模式 & \thesisissue{554} \\
          oneside & 单页模式 & \thesisissue{694} \\
          openright & 章从奇数页开始 & \thesisdiscuss{724} \\
          openany & 章从任意页开始 & \thesisissue{446} \\
          \bottomrule
        \end{tabular}
      \end{table}
    \end{column}
  \end{columns}
\end{frame}

\begin{frame}[fragile]
  \frametitle{基本配置}
  \framesubtitle{\textbackslash{}input\{setup\}}
  \begin{columns}
    \begin{column}{0.45\textwidth}
      \includegraphics[page=9]{thesisdir}
    \end{column}
    \begin{column}{0.55\textwidth}
      \begin{codeblock}[firstnumber=12]{main.tex}
|\highlightline<1>|% 论文基本配置,加载宏包等全局配置
|\highlightline<1>|\input{setup}

\begin{document}

%TC:ignore

|\highlightline<2>|% 标题页
|\highlightline<2>|\maketitle
      \end{codeblock}
      \visible<2>{
        \cmd{sjtusetup} 中的 \pkg{info} 将会修改封面的信息设置(见第 \ref{covers} 页)。
      }
    \end{column}
  \end{columns}
\end{frame}

\begin{frame}[fragile]
  \frametitle{基本配置}
  \framesubtitle{\textbackslash{}sjtusetup}
  \begin{columns}
    \begin{column}{0.45\textwidth}
      \includegraphics[page=1]{thesisdir}
    \end{column}
    \begin{column}{0.55\textwidth}
      \begin{codeblock}[firstnumber=3]{setup.tex}
\sjtusetup{
  info = {
    title    = {||上海交通大学学位论文 \LaTeX{} 模板示例文档},
    title*   = {A Sample for \LaTeX-based SJTU Thesis Template},
    author   = {||某\quad{}某},
    author* = {Mo Mo},
  },
  style = { header-logo-color = red, 
  },
  name = {
    publications = {||攻读学位期间完成的论文},
  },
}
      \end{codeblock}
    \end{column}
  \end{columns}
\end{frame}

\begin{frame}
  \frametitle{基本配置}
  \framesubtitle{\textbackslash{}sjtusetup}
  \begin{columns}
    \begin{column}{0.45\textwidth}
      \includegraphics[page=1]{thesisdir}
    \end{column}
    \begin{column}{0.55\textwidth}
      \begin{table}[H]
        \centering
        \caption{info 域}
        \footnotesize
        \begin{tabular}{lll} \toprule
          命令作用 & 中文对应选项 & 英文对应选项 \\ \midrule
          论文标题 & \texttt{title} & \texttt{title*} \\
          关键字列表 & \texttt{keywords} & \texttt{keywords*} \\
          作者姓名&  \texttt{author} &\texttt{author*}\\
          申请学位名称 & \texttt{degree}&\texttt{degree*}\\
          院系名称 & \texttt{department} & \texttt{department*}\\
          专业名称 & \texttt{major} & \texttt{major*}\\
          导师 & \texttt{supervisor} & \texttt{supervisor*}\\
          副导师 & \texttt{assisupervisor} & \texttt{assisupervisor*}\\
          日期 & \multicolumn{2}{c}{\texttt{date}}\\
          学号 & \multicolumn{2}{c}{\texttt{id}}\\ \bottomrule
          \end{tabular}
      \end{table}
    \end{column}
  \end{columns}
\end{frame}

\begin{frame}[fragile]
  \frametitle{版权页}
  \framesubtitle{\textbackslash{}copyrightpage}
  \begin{columns}
    \begin{column}{0.45\textwidth}
      \only<1>{
        \includegraphics[page=9]{thesisdir}
      }
      \only<2>{
        \includegraphics[page=2]{thesisdir}
      }
      \only<3>{
        \begin{figure}[H]
          \framebox{\includegraphics[page=2,width=0.4\linewidth]{bachelor}}
          \caption{版权页}
        \end{figure}
      }
    \end{column}
    \begin{column}{0.55\textwidth}
      \begin{codeblock}[firstnumber=22]{main.tex}
|\highlightline<1>|% 原创性声明及使用授权书
|\highlightline<1>|\copyrightpage
|\highlightline<2>|% 插入外置原创性声明及使用授权书
|\highlightline<2>|% \copyrightpage[scans/sample-copyright-old.pdf]
      \end{codeblock}
      \only<1>{
        \cmd{copyrightpages} 可以用于插入版权页。
      }
      \only<2>{
        \cmd{copyrightpages} 也接受一个可选参数,用于直接使用扫描件。
      }
    \end{column}
  \end{columns}
\end{frame}

\begin{frame}[fragile]
  \frametitle{前置部分}
  \framesubtitle{\textbackslash{}frontmatter}
  \begin{columns}
    \begin{column}{0.45\textwidth}
      \only<1>{
        \includegraphics[page=9]{thesisdir}
      }
      \only<2>{
        \includegraphics[page=3]{thesisdir}
      }
      \only<3>{
        \begin{figure}[H]
          \begin{subfigure}{0.45\textwidth}
            \framebox{\includegraphics[page=3,width=\linewidth]{bachelor}}
            \caption{中文}
          \end{subfigure}\hfill
          \begin{subfigure}{0.45\textwidth}
            \framebox{\includegraphics[page=4,width=\linewidth]{bachelor}}
            \caption{英文}
          \end{subfigure}
          \caption{摘要}
        \end{figure}
      }
      \only<4>{
        \begin{figure}[H]
          \begin{subfigure}{0.30\linewidth}
            \centering
            \framebox{\includegraphics[page=5,width=0.6\linewidth]{bachelor}}
            \caption{目录}
          \end{subfigure}
          \begin{subfigure}{0.30\linewidth}
            \centering
            \framebox{\includegraphics[page=6,width=0.6\linewidth]{bachelor}}
            \caption{插图}
          \end{subfigure}

          \begin{subfigure}{0.30\linewidth}
            \centering
            \framebox{\includegraphics[page=7,width=0.6\linewidth]{bachelor}}
            \caption{表格}
          \end{subfigure}
          \begin{subfigure}{0.30\linewidth}
            \centering
            \framebox{\includegraphics[page=8,width=0.6\linewidth]{bachelor}}
            \caption{算法}
          \end{subfigure}
          \caption{索引}
        \end{figure}
      }
      \only<5>{
        \includegraphics[page=4]{thesisdir}
      }
      \only<6>{
        \begin{figure}[H]
          \framebox{\includegraphics[page=9,width=0.5\linewidth]{bachelor}}
          \caption{符号对照表}
        \end{figure}
      }
    \end{column}
    \begin{column}{0.55\textwidth}
      \begin{codeblock}[firstnumber=30]{main.tex}
|\highlightline<2-3>|% 摘要
|\highlightline<2-3>|\input{contents/abstract}

|\highlightline<4>|% 目录
|\highlightline<4>|\tableofcontents
|\highlightline<4>|% 插图索引
|\highlightline<4>|\listoffigures*
|\highlightline<4>|% 表格索引
|\highlightline<4>|\listoftables*
|\highlightline<4>|% 算法索引
|\highlightline<4>|\listofalgorithms*

|\highlightline<5-6>|% 符号对照表
|\highlightline<5-6>|\input{contents/nomenclature}
      \end{codeblock}
    \end{column}
  \end{columns}
\end{frame}

\begin{frame}[fragile]
  \frametitle{主体部分}
  \framesubtitle{\textbackslash{}mainmatter}
  \begin{columns}
    \begin{column}{0.45\textwidth}
      \only<1>{
        \includegraphics[page=5]{thesisdir}
      }
      \only<2>{
        \begin{figure}[H]
          \begin{subfigure}{0.30\linewidth}
            \centering
            \framebox{\includegraphics[page=11,width=0.6\linewidth]{bachelor}}
            \caption{简介}
          \end{subfigure}
          \begin{subfigure}{0.30\linewidth}
            \centering
            \framebox{\includegraphics[page=13,width=0.6\linewidth]{bachelor}}
            \caption{数学}
          \end{subfigure}

          \begin{subfigure}{0.30\linewidth}
            \centering
            \framebox{\includegraphics[page=16,width=0.6\linewidth]{bachelor}}
            \caption{浮动体}
          \end{subfigure}
          \begin{subfigure}{0.30\linewidth}
            \centering
            \framebox{\includegraphics[page=22,width=0.6\linewidth]{bachelor}}
            \caption{总结}
          \end{subfigure}
          \caption{主体部分}
        \end{figure}
      }
    \end{column}
    \begin{column}{0.55\textwidth}
      \begin{codeblock}[firstnumber=47]{main.tex}
|\highlightline|% 正文内容
|\highlightline|\input{contents/intro}
|\highlightline|\input{contents/math_and_citations}
|\highlightline|\input{contents/floats}
|\highlightline|\input{contents/summary}

%TC:ignore

% 参考文献
\printbibliography[heading=bibintoc]
      \end{codeblock}
    \end{column}
  \end{columns}
\end{frame}

\begin{frame}
  \frametitle{数学}
  \begin{itemize}
    \item 公式示例:\nolinkurl{contents/math_and_citations.tex}
    \item \SJTUThesis{} 定义了常用的数学环境(需要手工引入 \texttt{ntheorem} 宏包):
      \begin{table}[h]
        \centering
        \footnotesize
        \begin{tabular}{*{7}{l}}\toprule
          assumption  & axiom   & conjecture & corollary    & definition  & example   & exercise  \\
          假设        & 公理    & 猜想       & 推论         & 定义        & 例        & 练习      \\\midrule
          lemma       & problem & proof      & proposition  & remark      & solution  & theorem   \\
          引理        & 问题    & 证明       & 命题         & 注          & 解        & 定理      \\\bottomrule
        \end{tabular}
      \end{table}
      \item \SJTUThesis{} 可以通过 \texttt{unimath} 选项使用 \pkg{unicode-math} 进行数学输入,注意与传统方式的区别。\thesisissue{555}
  \end{itemize}
\end{frame}

\begin{frame}[fragile]
  \frametitle{参考文献}
  \begin{columns}
    \begin{column}{0.45\textwidth}
      \includegraphics[page=6]{thesisdir}
    \end{column}
    \begin{column}{0.55\textwidth}
      \begin{codeblock}[firstnumber=111,numbersep=2pt]{setup.tex}
% 使用 BibLaTeX 处理参考文献
%   biblatex-gb7714-2015 常用选项
%     gbnamefmt=lowercase     姓名大小写由输入信息确定
%     gbpub=false             禁用出版信息缺失处理
\usepackage[backend=biber,style=gb7714-2015]{biblatex}
% 文献表字体
% \renewcommand{\bibfont}{\zihao{-5}}
% 文献表条目间的间距
\setlength{\bibitemsep}{0pt}
|\highlightline|% 导入参考文献数据库
|\highlightline|\addbibresource{bibdata/thesis.bib}
      \end{codeblock}
    \end{column}
  \end{columns}
\end{frame}

\begin{frame}[fragile]
  \frametitle{附录}
  \framesubtitle{\textbackslash{}appendix}
  \begin{columns}
    \begin{column}{0.45\textwidth}
      \only<1>{
        \includegraphics[page=7]{thesisdir}
      }
      \only<2>{
        \begin{figure}[H]
          \begin{subfigure}{0.45\linewidth}
            \framebox{\includegraphics[width=\linewidth,page=24]{bachelor}}
            \caption{}
          \end{subfigure}\hfill
          \begin{subfigure}{0.45\textwidth}
            \framebox{\includegraphics[width=\linewidth,page=25]{bachelor}}
            \caption{}
          \end{subfigure}
          \caption{附录}
        \end{figure}
      }
    \end{column}
    \begin{column}{0.55\textwidth}
      \begin{codeblock}[firstnumber=61]{main.tex}
% 附录中图表不加入索引
\captionsetup{list=no}

% 附录内容
|\highlightline|\input{contents/app_maxwell_equations}
|\highlightline|\input{contents/app_flow_chart}
      \end{codeblock}
    \end{column}
  \end{columns}
\end{frame}

\begin{frame}[fragile]
  \frametitle{结尾部分}
  \framesubtitle{\textbackslash{}backmatter}
  \begin{columns}
    \begin{column}{0.45\textwidth}
      \only<1>{
        \includegraphics[page=8]{thesisdir}
      }
      \only<2>{
        \begin{figure}[H]
          \begin{subfigure}{0.30\linewidth}
            \centering
            \framebox{\includegraphics[page=26,width=0.6\linewidth]{bachelor}}
            \caption{致谢}
          \end{subfigure}
          \begin{subfigure}{0.30\linewidth}
            \centering
            \framebox{\includegraphics[page=27,width=0.6\linewidth]{bachelor}}
            \caption{成就}
          \end{subfigure}

          \begin{subfigure}{0.30\linewidth}
            \centering
            \framebox{\includegraphics[page=28,width=0.6\linewidth]{bachelor}}
            \caption{简历}
          \end{subfigure}
          \begin{subfigure}{0.30\linewidth}
            \centering
            \framebox{\includegraphics[page=29,width=0.6\linewidth]{bachelor}}
            \caption{大摘要*}
          \end{subfigure}
          \caption{结尾部分}
        \end{figure}
      }
    \end{column}
    \begin{column}{0.55\textwidth}
      \begin{codeblock}[firstnumber=76]{main.tex}
% 致谢
\input{contents/acknowledgements}

% 发表论文及科研成果
% 盲审论文中,发表论文及科研成果等仅以第几作者注明即可,不要出现作者或他人姓名
\input{contents/achievements}

% 简历
\input{contents/resume}

% 学士学位论文要求在最后有一个大摘要,单独编页码
\input{contents/digest}
      \end{codeblock}
    \end{column}
  \end{columns}
\end{frame}

\begin{frame}
  \frametitle{还有其他问题?}
  \begin{columns}
    \begin{column}{0.75\textwidth}
    \begin{itemize}
      \item[{\faComment*[regular]}] 日常模板或 \LaTeX{} 使用问题可以前往 Discussions \link{https://github.com/sjtug/SJTUThesis/discussions} 提问
      
      (解决后别忘了 \textcolor{green}{\faCheckCircle{} Mark as answer}
      \item[{\faDotCircle[regular]}] 如果是 \textsc{SJTUThesis} 项目本身的 bug 和 feature request
      
      可以通过 Issues \link{https://github.com/sjtug/SJTUThesis/issues} 反馈。
      \item[{\faCodeBranch}] 如果你有好点子,可以贡献代码
     
      向 \textsc{SJTU\TeX{}}(v1) \link{https://github.com/sjtug/SJTUTeX/tree/v1} 存储库发 PR,\par
      而后把解包结果同步到 \textsc{SJTUThesis}。
  
      \item[{\faTag}] 如果你对正在基于 \LaTeX3 开发的新版感兴趣,\par
      也欢迎向 \textsc{SJTU\TeX{}}(v2) \link{https://github.com/sjtug/SJTUTeX/tree/v2} 发 PR。
  
      \item[{\faQq}] 也欢迎在 QQ 群即时讨论。
    \end{itemize}
    \end{column}
    \begin{column}{0.25\textwidth}
      \includegraphics[height=0.7\textheight]{qq.jpg}
    \end{column}
  \end{columns}
\end{frame}
\end{document}
      \end{codeblock}
    \end{column}
  \end{columns}
\end{frame}

\begin{frame}[fragile]
  \frametitle{组织文档}
  \begin{columns}
    \begin{column}{0.4\textwidth}
      \begin{codeblock}[]{learnlatex.tex}
|\highlightline|\chapter{||学习 \LaTeX{}}
\section{||概念}
\subsection{\LaTeX{}}
\LaTeX{} 是一个用以排版高质量作品的文档准备系统。
      \end{codeblock}
      子文件中就不需要添加 \env{document} 环境了\footnotemark。
    \end{column}
    \begin{column}{0.6\textwidth}
      \begin{codeblock}[]{主文档}
|\highlightline|\documentclass{ctexrep}
\includeonly{learnlatex,sjtuthesis}
\begin{document}
  \tableofcontents
  % !TeX root = ..\..\latex-talk.tex

\part{学习 \LaTeX{}}
% FIXME: Part Page miniframe overflow
% FIXME: footnote fault numbering

\begin{frame}[plain]
  \vfil
  \begin{center}
    \href{https://learnlatex.org}{
      \rmfamily
      Learn\,\lower1ex\hbox{\Huge\LaTeX{}}.org
    }
  \end{center}
  \vfil
  \begin{center}
    \parbox{0.75\linewidth}{
      Learn\LaTeX{}.org\cite{learnlatex} 提供了解 \LaTeX{} 的 16 篇简短的教程,并包含了一些可以在线运行的示例,可以通过亲自动手查看实验效果。本部分主要参考由 C\TeX{}-org 提供的中文翻译版本 \link{https://github.com/CTeX-org/learnlatex.github.io/tree/zh-Hans/zh-Hans/}。
    }
  \end{center}
  \vfil
\end{frame}

{ % Start of shaded number logo

\newcommand{\shadedfont}[2][1pt]{
  % #1 (optional): shadow distance
  % #2: the text needed to be shaded
  \hbox{\rlap{\color{gray}\hskip#1#2}#2}
}
\newcounter{learnsec}
\setcounter{learnsec}{-1}
\newcommand{\updatelogo}{
  % update the logo corresponding to current counter.
  \stepcounter{learnsec}
  \logo{
    \raise.3ex\hbox{\tiny\insertsection}\shadedfont{\arabic{learnsec}}
  }
}
\let\oldsection=\section
\renewcommand{\section}[1]{\oldsection{#1}\updatelogo}

\section{是什么}
\begin{frame}
  \frametitle{\TeX{}}
  \begin{columns}[c]
    \begin{column}{0.7\textwidth}
      \begin{center}
        \rmfamily\Huge
        \hologo{La}\highlight[structure!70]{\TeX{}}
      \end{center}
      \begin{center}
        \parbox{0.75\textwidth}{
          \TeX{} 是由斯坦福大学教授高德纳
          (Donald E.~Knuth)于 1977 年开始开发的排版引擎。目前仍在更新,最新版本号为 3.141592653 \link{https://tug.org/TUGboat/tb42-1/tb130knuth-tuneup21.pdf}。
        }
      \end{center}
    \end{column}
    \begin{column}{0.3\textwidth}
      \includegraphics[width=.8\columnwidth]{Knuth.jpg}
    \end{column}
  \end{columns}
\end{frame}

\begin{frame}
  \frametitle{\LaTeX{}}
  \begin{columns}[c]
    \begin{column}{0.7\textwidth}
      \begin{center}
        \rmfamily\Huge
        \highlight[structure]{\LaTeX{}}
      \end{center}
      \begin{center}
        \parbox{0.75\textwidth}{
          \LaTeX{} 是最早在 1985 年由现就职于微软的 Leslie Lamport 开发的一种 \TeX{} \textbf{格式}\footnotemark,使用一些列宏和扩展宏包来简化 \TeX{} 的使用。现在由 \LaTeX{} Project 的成员维护。现在广泛使用的版本是 \LaTeXe{},最新的版本为 \LaTeX3(2020 年 10 月后默认内置)。
        }
      \end{center}
    \end{column}
    \begin{column}{0.3\textwidth}
      \includegraphics[width=.8\columnwidth]{Lamport.jpg}
    \end{column}
  \end{columns}
  \footnotetext{\hologo{ConTeXt} 也是一种 \TeX{} 格式 \link{https://www.contextgarden.net/}。}
\end{frame}

\begin{frame}
  \frametitle{程序}
  \begin{columns}[c]
    \begin{column}{0.7\textwidth}
      \begin{center}
        \rmfamily\Huge
        \highlight[structure]{\hologo{pdfLaTeX}}
      \end{center}
      \begin{center}
        \parbox{0.7\textwidth}{
          \hologo{pdfLaTeX} 是为了编译一个 \LaTeX{} 文档而运行的程序。实际上底层在运行一个叫 \hologo{pdfTeX} 的引擎,并预装了对应的 \LaTeX{} \textbf{格式}。为了利用临时文件,可能就需要多次运行程序。
        }
      \end{center}
    \end{column}
    \begin{column}{0.3\textwidth}
      \begin{block}{}
        \ttfamily\small
        > \highlight{pdflatex} main.tex\\
        This is pdfTeX, Version 3.141592653-
        2.6-1.40.23 (MiKTeX 21.10)\\
        entering extended mode\\
        \highlight{LaTeX2e} <2021-11-15>\\
        \highlight{L3} programming layer <2021-11-22>
      \end{block}
    \end{column}
  \end{columns}
\end{frame}

\begin{frame}
  \frametitle{引擎}
  \begin{columns}[c]
    \begin{column}{0.7\textwidth}
      \begin{center}
        \rmfamily\Huge
        \highlight[structure!70]{pdf}\hologo{La}\highlight[structure!70]{\TeX{}}
      \end{center}
      \begin{center}
        \parbox{0.7\textwidth}{
          pdf\TeX{} 是编译 \TeX{} 文档(以 \texttt{.tex} 结尾)的\textbf{引擎}---可以理解 \TeX{} 指令的\textbf{程序}。
        }
      \end{center}
    \end{column}
    \begin{column}{0.3\textwidth}
      \begin{block}{}
        \ttfamily\small
        > pdflatex main.tex\\
        This is \highlight[structure!70]{pdfTeX}, Version 3.141592653-
        2.6-1.40.23 (MiKTeX 21.10)
        entering extended mode\\
        LaTeX2e <2021-11-15>\\
        L3 programming layer <2021-11-22>
      \end{block}
    \end{column}
  \end{columns}
\end{frame}

\begin{frame}
  \frametitle{Unicode 引擎}
  \begin{table}
    \caption{主流 \hologo{(La)TeX} 程序
    \footnote{(u)p\TeX{} 是日语最常用的引擎,生成 \texttt{.dvi},支持 Unicode。}\footnote{Ap\TeX{} 具有底层 CJK 支持,内联 Ruby,Color Emoji。}}
    \footnotesize
    \begin{stampbox}
      \begin{tabular}{c>{\raggedright}*{3}{p{3.5cm}}}
        \alert{引擎}     & \hologo{pdfTeX}   & \hologo{XeTeX}   & \hologo{LuaTeX}   \\
        \alert{程序}     & \hologo{pdfLaTeX} & \hologo{XeLaTeX} & \hologo{LuaLaTeX} \\
        \alert{特点}     & 直接生成 PDF,支持 micro-typography  & 支持 Unicode、OpenType 与复杂文字编排 (CTL) & 支持 Unicode,内联 Lua,支持 OpenType \\
      \end{tabular}
    \end{stampbox}
  \end{table}

  \begin{center}
    \parbox{.9\textwidth}{
      \hologo{pdfLaTeX} 不支持 Unicode。为了排版中文,大部分情况下 \faApple{}\,\faLinux{} 应当使用 \hologo{XeLaTeX},而 \hologo{LuaLaTeX} 速度相对较慢。\faWindows{} 可以在一些情况下使用 \hologo{pdfLaTeX}。
    }
  \end{center}
\end{frame}

% \begin{frame}
%   \paragraph{\hologo{pdfLaTeX}} \TeX{} 和 \LaTeX{} 被广泛使用之前,它们只需内置支持欧洲语言即可。在 Unicode 出现之前,\LaTeX{} 提供了许多种\textbf{文件编码}来允许很多语言的文字以原生的方式输入,\hologo{pdfLaTeX} 也只需要使用 8 位文件编码和 8 位字体。
% \end{frame}

\section{跑起来}
\begin{frame}
  \frametitle{发行版}
  \begin{table}
    \caption{\hologo{TeX} 发行版}
    \footnotesize
    \begin{stampbox}
      \begin{tabular}{c>{\raggedright}*{3}{p{3.2cm}}}
        \alert{发行版}     & \hologo{MiKTeX} \link{https://mirrors.sjtug.sjtu.edu.cn/ctan/systems/win32/miktex/setup/windows-x64/basic-miktex-21.12-x64.exe}   & \TeX{} Live \link{https://mirrors.sjtug.sjtu.edu.cn/ctan/systems/texlive/tlnet/install-tl.zip}   & Mac\TeX{} \link{https://mirrors.sjtug.sjtu.edu.cn/ctan/systems/mac/mactex/mactex-20210328.pkg}  \\[2pt]
        \alert{特点}      &  只安装必要文件,依赖用时更新  &  所有平台均可使用,每年发布一次 & Mac 系统专用,对 \TeX{} Live 的进一步打包 \\
        \alert{推荐平台}  & \faWindows  & \faLinux &  \faApple \\
      \end{tabular}
    \end{stampbox}
  \end{table}
  \begin{center}
    \parbox{.9\textwidth}{
      要让 \LaTeX{} 跑起来,核心就是要有一套 \TeX{} 发行版,来获取让 \LaTeX{} 工作所需的一组程序和文件。参考《一份简短的关于 \LaTeX{} 安装的介绍》\link{https://mirrors.sjtug.sjtu.edu.cn/ctan/info/install-latex-guide-zh-cn/install-latex-guide-zh-cn.pdf} 安装想使用的发行版。推荐使用发行版的最新版本\footnote{老版本 Linux 系统的包管理器自带 \TeX{} Live 发行版可能不是最新的 \link{https://repology.org/project/texlive/versions},尽量使用镜像安装,并手动将相关环境变量添加到路径 \link{https://www.tug.org/texlive/doc/texlive-zh-cn/texlive-zh-cn.pdf}。},并使用国内镜像。
    }
  \end{center}
\end{frame}

\begin{frame}[plain]
  \hbox to \textwidth{
    \hfil
    \vbox to 3cm{
      \hbox{
        \resizebox{3cm}{!}{\includegraphics{\getcontribpath{sjtug}{vi/sjtug.pdf}}}
      }
    }
    \hfil
    \vbox to 3cm{
      \vfill
      \hbox{\Large\bfseries\color{cprimary} 稳定、快速、现代的镜像服务。}
      \vskip2pt
      \hbox{托管于华东教育网骨干节点上海交通大学。}
      \vfill
    }
    \hskip20pt
    \hfil
  }

  \begin{center}
    \parbox{0.8\textwidth}{
      推荐使用 SJTUG 软件镜像服务,离得近,下得快。
      
      \begin{description}
        \footnotesize
        \item[\TeX{} Live]  {\ttfamily tlmgr option repository https://mirrors.sjtug.sjtu.edu.cn/CTAN/systems/texlive/tlnet}
        \item[\hologo{MiKTeX}] 在 \hologo{MiKTeX} Console 中设置镜像源为 \url{https://mirrors.sjtug.sjtu.edu.cn}
      \end{description}
    }
  \end{center}
\end{frame}

\begin{frame}
  \frametitle{编辑器}
  \begin{table}
    \caption{开源编辑器推荐}
    \footnotesize
    \begin{stampbox}
      \begin{tabular}{c>{\raggedright}*{3}{p{3.5cm}}}
        \alert{编辑器}     & \begin{tabular}{c}Visual Studio Code\\ \LaTeX{} Workshop\end{tabular}  & \TeX{}studio & \TeX{}works \\[5pt]
        \alert{特点}      &  搭配 VS Code 使用非常方便,易扩展  & 可以使用大量的菜单选项输入代码块,用户友好 & 只提供基础的高亮与运行方法,发行版自带\footnote{Mac\TeX{} 打包的是 \TeX{}Shop 编辑器。} \\
      \end{tabular}
    \end{stampbox}
  \end{table}
  \begin{center}
    \parbox{.9\textwidth}{
      使用专为 \LaTeX{} 设计的编辑器将带来更多便利,因为它们往往会提供一键编译、内置 PDF 阅读器以及语法高亮等功能。几乎所有现代的 \LaTeX{} 编辑器都提供 Sync\TeX{} 这一强大的功能,以在 PDF 和 代码间对应跳转。
    }
  \end{center}
\end{frame}

\begin{frame}
  \frametitle{在线平台}
  \begin{table}
    \caption{在线协作平台推荐}
    \footnotesize
    \begin{stampbox}
      \begin{tabular}{c>{\raggedright}*{2}{p{4cm}}}
        \alert{在线平台}     & Overleaf \link{https://www.overleaf.com/}  & 交大 \LaTeX{} 助手 \link{https://latex.sjtu.edu.cn/} \\[2pt]
        \alert{特点}      & 最流行的在线平台,提供大量的模板,但国内访问慢 & 校内平台,隐私保护有保障,共享项目限制少 \\
      \end{tabular}
    \end{stampbox}
  \end{table}
  \begin{center}
    \parbox{.9\textwidth}{
      在线平台允许你直接在网页中编辑文档,无需本地安装即可在后台运行 \LaTeX{},并显示生成的 PDF。可以参照 Overleaf 官方文档学习如何使用在线平台 \link{https://www.overleaf.com/learn}。
    }
  \end{center}
\end{frame}

\section{基本结构}
\begin{frame}[fragile]%
  \frametitle{文档部件}
  \begin{columns}[c]
    \begin{column}{0.4\textwidth}
      \only<1>{
        \cmd{documentclass} 命令加载了\textbf{文档类}。\pkg{article} 是由 \LaTeX{}提供的用于排版短文档的基本文档类。
        \begin{description}
          \footnotesize
          \item[\pkg{article}] 不包含章的短文档
          \item[\pkg{report}] 含有章的单面印刷文档
          \item[\pkg{book}] 含有章的双面印刷文档
          \item[\pkg{beamer}] 制作幻灯片
        \end{description}
      }
      \only<2>{
        \env{document} 环境用于指示文档主体的范围。\LaTeX{} 还有其他的使用 \cmd{begin} 和 \cmd{end} 的搭配,我们称这些为\textbf{环境}。它们将用来设定局部格式命令\footnotemark。
      }
      \only<3>{
        \includepdflarge{enminimal}
      }
    \end{column}
    \begin{column}{0.6\textwidth}
      \begin{codeblock}[]{排版英文最简示例}
|\only<1>{\highlightline}|\documentclass{article}
|\only<2>{\highlightline}|\begin{document}
|\only<3>{\highlightline}|  Together for a Shared Future
|\only<2>{\highlightline}|\end{document}
      \end{codeblock}
    \end{column}
  \end{columns}
  \only<2>{\footnotetext{环境实际上是一个组,只不过通过语义化的形式预装了对应的格式命令。普通的组可以直接使用一对大括号之间的内容 \{$\cdots$\} 表示。}}
\end{frame}

\section{扩展}
\begin{frame}[fragile]%
  \frametitle{中文排版}
  \begin{columns}[c]
    \begin{column}{0.4\textwidth}
      \only<1>{
        \cmd{usepackage} 用于使用宏包以向 \LaTeX{} 添加或修改功能,需要在\textbf{导言区}调用。
        这里使用 \pkg{ctex} 宏集以获得中文支持。其调用底层因随不同的引擎而不同。
        {
          \footnotesize
          \begin{stampbox}
            \begin{tabular}{c*{3}{c}}
              \alert{引擎}     & \hologo{pdfTeX}   & \hologo{XeTeX}   & \hologo{LuaTeX}   \\
              \alert{程序}     & \hologo{pdfLaTeX} & \hologo{XeLaTeX} & \hologo{LuaLaTeX} \\
              \alert{宏包}     & CJK\footnotemark & xeCJK & luatexja \\
              \alert{封装}     & \multicolumn{3}{c}{ctex} \\
            \end{tabular}
          \end{stampbox}
        }
        \vspace{-1cm}
      }
      \only<2>{
        C\TeX{} 建议对于之前提到的常规文档类,最佳实践是使用该宏集提供的四种中文文档类,以对特定类型提供额外的中文排版适配。
        \begin{center}
          \begin{stampbox}
            \footnotesize
            \begin{tabular}{cc}
              \pkg{ctexart} & \pkg{ctexrep} \\
              \pkg{ctexbook} & \pkg{ctexbeamer} \\
            \end{tabular}
          \end{stampbox}
        \end{center}
      }
      \only<3>{
        \includepdflarge{cnminimal}
      }
      \only<4>{
        大部分情况下,你都不应当在 \LaTeX{} 中强制断行:你几乎只是想另起一段,或者是想在段落之间添加空行(使用 \pkg{parskip} 宏包就可实现)。
        只有\alert{很少的}情况下你需要使用 \textbackslash{}\textbackslash{} 来另起一行而不另起一段。
      }
    \end{column}
    \begin{column}{0.6\textwidth}
      \begin{codeblock}[]{排版中文\only<2->{最佳实践}}
|\only<2>{\highlightline}|\documentclass{|\only<1>{article}\only<2->{ctexart}|}
|\only<1>{\highlightline\textbackslash{}usepackage\{ctex\}\hfill\color{cprimary}\% 导言区}|
\begin{document}
|\only<3>{\highlightline}|    一起向未来
|\only<4>{\highlightline}|
  Together for a Shared Future
\end{document}
      \end{codeblock}
    \end{column}
  \end{columns}
  \only<1>{\footnotetext{ctex 在 \faApple\,\faLinux{} 上已经不可以使用 \hologo{pdfLaTeX} 编译,以及在 \faWindows{} 上使用该引擎也会变更自动间距调整等功能的默认行为。}}
\end{frame}

\section{设定格式}
\begin{frame}[fragile]%
  \frametitle{字体样式}
  \begin{columns}
    \begin{column}{0.4\textwidth}
      \only<1>{
        \includepdflarge{fontstyle}
      }
      \only<2>{
        可以使用显示样式设定命令对小段做加粗、斜体、等宽等等的处理。
        \begin{center}
          \footnotesize
          \begin{stampbox}
            \begin{tabular}{rl}
              \cmd{textrm} & \textrm{衬线} \\
              \cmd{textbf} & \textbf{加粗} \\
              \cmd{textit} & \kaishu 斜体 \\
              \cmd{texttt} & \texttt{等宽} \\
              \cmd{textsf} & \textsf{无衬线} \\
              \cmd{textsc} & \textsc{Small Caps} \\
              \cmd{textsl} & \textsl{Slanted} \\
            \end{tabular}
          \end{stampbox}
        \end{center}
      }
      \only<3>{
        与 Word 不同的是,\LaTeX{} 一般情况下并不需要使用上面的显式命令,而是采用逻辑标记的方法,
        比如 \cmd{emph} 可以强调文字,以及下面将要提到的目次命令(第 \ref{sectioning} 页)。
        这样可以统一管理格式。
      }
    \end{column}
    \begin{column}{0.6\textwidth}
      \begin{codeblock}[]{样式}
\documentclass{ctexart}
\begin{document}
|\only<2>{\highlightline}|  \textbf{||一起向未来}

|\only<3>{\highlightline}|  \emph{Together for a Shared Future}
\end{document}
      \end{codeblock}
    \end{column}
  \end{columns}
\end{frame}

\begin{frame}[fragile]%
  \frametitle{\only<1-2>{字体大小}\only<3>{字体样式}}
  \begin{columns}
    \begin{column}{0.4\textwidth}
      \only<1>{
        \includepdflarge{fontsize}
      }
      \only<2>{
        同样地,你也可以显式地设定字体大小,但是这种命令会更改行文设置,所以需要使用一个组来限定作用范围\footnotemark。
        \begin{center}
          \footnotesize
          \begin{stampbox}
            \begin{tabular}{rl}
              \cmd{tiny} & \tiny 极小 \\
              \cmd{scriptsize} & \scriptsize 抄本大小  \\
              \cmd{footnotesize} & \footnotesize 脚注大小 \\
              \cmd{small} & \small 小 \\
              \cmd{normalsize} & \normalsize 正常大小 \\
              \cmd{large} & \large 大 \\
              \cmd{huge} & \Huge 巨大 \\
            \end{tabular}
          \end{stampbox}
        \end{center}
      }
      \only<3>{
        也可以使用字体样式对应的更改字体设置的命令,这对于大段文段的设置而言也是很方便的。
        \begin{center}
          \footnotesize
          \begin{stampbox}
            \begin{tabular}{ll}
              \cmd{textrm} & \cmd{rmfamily}\\
              \cmd{texttt} & \cmd{ttfamily}\\
              \cmd{textsf} & \cmd{sffamily}\\
              \cmd{textbf} & \cmd{bfseries}\\
              \cmd{textit} & \cmd{itshape}\\
              \cmd{textsc} & \cmd{scshape}\\
              \cmd{textsl} & \cmd{slshape}\\
            \end{tabular}
          \end{stampbox}
        \end{center}
      }
    \end{column}
    \begin{column}{0.6\textwidth}
      \begin{codeblock}[]{大小}
\documentclass{ctexart}
\begin{document}
|\only<2>{\highlightline}|  {\bfseries\Large 一起向未来\par}
|\only<3>{\highlightline}|  {\itshape Together for a Shared Future}
\end{document}
      \end{codeblock}
    \end{column}
  \end{columns}
  \only<2>{\footnotetext{注意最后显式地使用 \cmd{par} 在改回大小前结束该段,否则会导致下一行的行间距异常!}}
\end{frame}

\section{逻辑结构}
\begin{frame}[fragile]
  \frametitle{列表}
  \begin{columns}
    \begin{column}{0.35\textwidth}
      \begin{codeblock}[]{无序列表}
\documentclass{ctexart}
\begin{document}
|\highlightline|  \begin{itemize}
    \item 第一项
    \item 第二项
    \item 第三项
|\highlightline|  \end{itemize}
\end{document}
      \end{codeblock}
    \end{column}
    \begin{column}{0.35\textwidth}
      \begin{codeblock}[]{有序列表}
\documentclass{ctexart}
\begin{document}
|\highlightline|  \begin{enumerate}
    \item 第一项
    \item 第二项
    \item 第三项
|\highlightline|  \end{enumerate}
\end{document}
      \end{codeblock}
    \end{column}
    \begin{column}{0.35\textwidth}
      \begin{codeblock}[]{描述列表}
\documentclass{ctexart}
\begin{document}
|\highlightline|  \begin{description}
    \item[||第一] 文本
    \item[||第二] 文本
    \item[||第三] 文本  
|\highlightline|  \end{description}
\end{document}
      \end{codeblock}
    \end{column}
  \end{columns}
\end{frame}

%更深的列表技巧,定理环境等

\begin{frame}
  \frametitle{列表}
  \begin{columns}
    \begin{column}{0.35\textwidth}
      \includepdflarge{unordered}
    \end{column}
    \begin{column}{0.35\textwidth}
      \includepdflarge{ordered}
    \end{column}
    \begin{column}{0.35\textwidth}
      \includepdflarge{description}
    \end{column}
  \end{columns}
\end{frame}

\begin{frame}[fragile,label=sectioning]%
  \frametitle{目次结构}
  \begin{columns}
    \begin{column}{0.4\textwidth}
      \LaTeX{} 可以使用目次命令将文档划分层级\footnotemark,并自动设定对应字体样式和大小。
      \begin{center}
        \begin{stampbox}
          \footnotesize
          \begin{tabular}{rll}
           命令 & 中文 & 层次 \\
           \cmd{chapter} & 章\footnotemark & \sout{0} \\
           \cmd{section} & 节 & 1 \\
           \cmd{subsection} & 小节 & 2 \\
           \cmd{subsubsection} & 小小节 & 3 \\
          \end{tabular}
        \end{stampbox}
      \end{center}
    \end{column}
    \begin{column}{0.6\textwidth}
      \begin{codeblock}[]{目次}
\documentclass{ctexart}
\begin{document}
|\highlightline|  \section{||概念}
|\highlightline|  \subsection{\LaTeX{}}
  \LaTeX{} 是一个用以排版高质量作品的文档准备系统。
\end{document}
      \end{codeblock}
    \end{column}
  \end{columns}
  \footnotetext{章这一级只在 \pkg{report} 和 \pkg{book} 文档类(包括对应的中文文档类)有定义。还有不常用的 \cmd{part} (0@\pkg{article}/-1@\pkg{report}\&\pkg{book}\&\pkg{beamer}) 以及更低层次的 \cmd{paragraph} (4) 与 \cmd{subparagraph} (5)。 }
\end{frame}

\begin{frame}[fragile]%
  \frametitle{组织文档}
  \begin{columns}
    \begin{column}{0.4\textwidth}
      \only<1>{
        \cmd{tableofcontents} 用来生成对于目次命令的目录。如果你想设定显示到哪个层级,在这个命令前使用 \cmd{setcounter\{tocdepth\}\{层次\}}
      }
      \only<2>{
        对于大型文档而言,使用多个文件管理源文件通常是更方便的。而 \cmd{include} 和 \cmd{input} 都以相对路径的方式插入其他 \TeX{} 文档。
        区别在于,\cmd{include} 命令会从新页开始并做一些内部调整,这基本上只对 \pkg{chapter} 这一级有用。而 \cmd{input} 会原样插入源代码。
      }
      \only<3>{
        但是 \cmd{include} 插入的文档可以使用 \cmd{includeonly} 管理当前要排印哪一部分的内容,利用所有章节辅助文件的同时,减少编译时间并专注于该部分的内容。
      }
    \end{column}
    \begin{column}{0.6\textwidth}
      \begin{codeblock}[]{主文档}
\documentclass{ctexrep}
|\only<3>{\highlightline}|\includeonly{learnlatex,sjtuthesis}
\begin{document}
|\only<1>{\highlightline}|  \tableofcontents
|\only<2-3>{\highlightline}|  \include{learnlatex}
|\only<3>{\highlightline}|  \include{sjtuthesis}
\end{document}
      \end{codeblock}
    \end{column}
  \end{columns}
\end{frame}

\begin{frame}[fragile]
  \frametitle{组织文档}
  \begin{columns}
    \begin{column}{0.4\textwidth}
      \begin{codeblock}[]{learnlatex.tex}
|\highlightline|\chapter{||学习 \LaTeX{}}
\section{||概念}
\subsection{\LaTeX{}}
\LaTeX{} 是一个用以排版高质量作品的文档准备系统。
      \end{codeblock}
      子文件中就不需要添加 \env{document} 环境了\footnotemark。
    \end{column}
    \begin{column}{0.6\textwidth}
      \begin{codeblock}[]{主文档}
|\highlightline|\documentclass{ctexrep}
\includeonly{learnlatex,sjtuthesis}
\begin{document}
  \tableofcontents
  \include{learnlatex}
  \include{sjtuthesis}
\end{document}
      \end{codeblock}
    \end{column}
  \end{columns}
  \footnotetext{如果想强制指定子文档的主文档,可以在文件第一行输入魔术命令:\texttt{\% !TeX root = main.tex}}
\end{frame}

\section{图}
\begin{frame}[fragile]%
  \frametitle{\temporal<5>{插图}{浮动体}{插图}}
  \begin{columns}
    \begin{column}{0.6\textwidth}
      \begin{codeblock}[]{插入单图\only<4->{最佳实践}}
\documentclass{ctexart}
|\only<2>{\highlightline}|\usepackage{graphicx}
|\only<2>{\highlightline}|\graphicspath{{figs/}{pics/}}
\begin{document}
|\only<5>{\highlightline}|\begin{figure}[ht]
|\only<6>{\highlightline}|  \centering
|\only<3>{\highlightline}|  \includegraphics[width=|\only<1-3>{4cm}\only<4->{0.4\textbackslash{}textwidth}|]{sjtug}
|\only<7>{\highlightline}|  \caption{SJTUG 徽标}\label{fig:sjtug}
|\only<5>{\highlightline}|\end{figure}
\end{document}
      \end{codeblock}
    \end{column}
    \begin{column}{0.4\textwidth}
      \only<1>{
        \includepdflarge{insertimage}
      }
      \only<2>{
        为了插入外部图片,需要使用 \pkg{graphicx} 宏包。之后在文档主体便可以使用 \cmd{includegraphics} 插入图片。导言区也可以加入 \cmd{graphicspath} 指定图片文件夹\footnotemark。
      }
      \only<3>{
        \cmd{includegraphics} 命令便以相对路径的方式插入图片,如果无同名图片,那么后缀名可以省略。可以使用可选参数指定插入的图片尺寸,最佳实践是使用 \cmd{textwidth} 或 \cmd{linewidth} 的相对值指定尺寸大小,以在未来可能的布局更改中保留一定的灵活性。
      }
      \only<4>{
        也可以通过键值对的方法设置图片的其他属性。
        \begin{center}
          \footnotesize
          \begin{stampbox}
            \begin{tabular}{rl}
              \pkg{width} & 宽度 \\
              \pkg{height} & 高度 \\
              \pkg{scale} & 缩放 \\
              \pkg{angle} & 角度 \\
            \end{tabular}
          \end{stampbox}
        \end{center}
      }
      \only<5>{
        \env{figure} 为一个浮动体环境(\env{table} 也是),可以将其移动到其他位置上以减少行文中的空白。可以添加可选参数以指定如何放置浮动体,最多可以使用四种位置描述符:
        \begin{center}
          \footnotesize
          \begin{stampbox}
            \begin{tabular}{cll}
              \pkg{h} & Here & 尽可能在这里 \\
              \pkg{t} & Top & 页面顶部 \\
              \pkg{b} & Bottom & 页面底部 \\
              \pkg{p} & Page & 浮动体专页 \\
            \end{tabular}
          \end{stampbox}
        \end{center}
        还可以只使用 \pkg{float} 宏包提供的 \pkg{H} 描述符以强制置于此处。
      }
      \only<6>{
        采用 \cmd{centering} 命令而不是 \env{center} 环境来水平居中图片。这将避免多余的纵向间距。
      }
      \only<7>{
        使用 \cmd{caption} 命令输入题注,如果这个命令写在插入图片前面,题注将会在上方(中文中一般对 \env{table} 环境这么做)。后面将会看到如何对留有标记(\cmd{label})的图片进行引用。
      }
    \end{column}
  \end{columns}
  \only<2>{\footnotetext{其命令参数每个为一个以 \texttt{/} 结尾的文件夹,每个文件夹需要使用大括号包裹起来。}}
\end{frame}

\begin{frame}[fragile]
  \begin{columns}
    \begin{column}{0.6\textwidth}
      \begin{codeblock}[]{插入双图}
\documentclass{ctexart}
\usepackage{graphicx}
\graphicspath{{figs/}{pics/}}
\begin{document}
  \begin{figure}[ht]
|\only<1>{\highlightline}|    \begin{minipage}{0.48\textwidth}
      \centering
      \includegraphics[height=2cm]{sjtug}
|\only<2>{\highlightline}|      \caption{SJTUG 徽标}\label{fig:sjtug}
|\only<1>{\highlightline}|    \end{minipage}\hfill
|\only<1>{\highlightline}|    \begin{minipage}{0.48\textwidth}
      \centering
      \includegraphics[height=2cm]{sjtugt}
|\only<2>{\highlightline}|      \caption{SJTUG||文字}\label{fig:sjtugt}
|\only<1>{\highlightline}|    \end{minipage}
  \end{figure}
\end{document}
      \end{codeblock}
    \end{column}
    \begin{column}{0.4\textwidth}
      \only<1>{
        在 \env{figure} 环境里使用 \env{minipage} 小页制作列盒子,以并排两图并分别编号,需要设定强制参数以指定列宽。两个小页之间添加 \cmd{hfill} 使两个小页两端对齐。
      }
      \only<2>{
        在每个小页内部分别使用 \cmd{caption} 添加标签。
      }
      \only<3>{
        \includepdflarge{doubleimages}
      }
    \end{column}
  \end{columns}
\end{frame}

\begin{frame}[fragile]%
  \begin{columns}
    \begin{column}{0.6\textwidth}
      \begin{codeblock}[]{}
\documentclass{ctexart}
\usepackage{graphicx}
|\highlightline|\usepackage{subcaption}
\graphicspath{{figs/}{pics/}}
\begin{document}
  \begin{figure}[ht]
|\highlightline|    \begin{subfigure}{0.48\textwidth}
      \centering
      \includegraphics[height=2cm]{sjtug}
      \caption{||徽标}
|\highlightline|    \end{subfigure}\hfill
|\highlightline|    \begin{subfigure}{0.48\textwidth}
      \centering
      \includegraphics[height=2cm]{sjtugt}
      \caption{||文字}
|\highlightline|    \end{subfigure}
    \caption{SJTUG}\label{fig:sjtug}
  \end{figure}
\end{document}
      \end{codeblock}
    \end{column}
    \begin{column}{0.4\textwidth}
      \includepdflarge{subfigures}\vspace{15pt}
      \pkg{subcaption} 宏包提供了 \env{subfigure} 环境(以及 \env{subtable}),可以用于以子图的形式插入,编号会增加一级。也可以为子图添加子集引用编号。
    \end{column}
  \end{columns}
\end{frame}

\section{表}
\begin{frame}[fragile]
  \frametitle{简单表格}
  \begin{columns}
    \begin{column}{0.6\textwidth}
      \begin{codeblock}[]{}
\documentclass{ctexart}
|\only<1-2>{\highlightline}|\usepackage{|\temporal<1>{array}{\highlight{array}}{array},\temporal<2>{booktabs}{\highlight{booktabs}}{booktabs}|}
\begin{document}
\begin{table}[ht]
  \centering
  \caption{||北京冬奥会吉祥物}
|\only<1>{\highlightline}|  \begin{tabular}{lp{3cm}}
|\only<2>{\highlightline}|    \toprule
|\only<3>{\highlightline}|吉祥物 & 描述                          \\
|\only<2>{\highlightline}|    \midrule
|\only<3>{\highlightline}|冰墩墩 & 2022 年北京冬季奥运会吉祥物  \\
|\only<3>{\highlightline}|雪容融 & 2022 年北京冬季残奥会吉祥物  \\
|\only<2>{\highlightline}|    \bottomrule
|\only<1>{\highlightline}|  \end{tabular}
\end{table}
\end{document}
      \end{codeblock}
    \end{column}
    \begin{column}{0.4\textwidth}
      \only<1>{
        使用 \env{tabular} 环境绘制表格。需要添加参数(称为\textbf{表格导言})以确定每一列的对齐方式。引入 \pkg{array} 宏包来使用更多的\textbf{引导符}。
        \begin{center}
          \footnotesize
          \begin{stampbox}
            \begin{tabular}{>{\ttfamily}ll}
              \alert{l} & 向左对齐 \\
              \alert{c} & 居中对齐 \\
              \alert{r} & 向右对齐 \\
              \alert{p\{3cm\}} & 固定列宽,两端对齐 \\
              \alert{m\{3cm\}} & \texttt{p} + 垂直居中对齐 \\
              \alert{>\{\textbackslash{}bfseries\}} & 后一列单元格前加命令 \\
              \alert{*\{3\}\{l\}} & 三个左对齐列 \\
            \end{tabular}
          \end{stampbox}
        \end{center}
      }
      \only<2>{
        \pkg{booktabs} 宏包提供了标准三线表格所需要的行分割线:\cmd{toprule},\cmd{midrule},\cmd{bottomrule}。也可以使用 \cmd{cmidrule\{1-2\}} 来部分地绘制行分割线。一般不推荐在专业表格中使用纵向分割线。
      }
      \only<3>{
        每行内容使用 \textbackslash\textbackslash{} 分隔开,每行中的单元格使用 \& 分隔开。
      }
      \only<4>{
        \includepdflarge{table}
      }
    \end{column}
  \end{columns}
\end{frame}

\begin{frame}[fragile]%
  \begin{columns}
    \begin{column}{0.6\textwidth}
      \begin{codeblock}[]{表头居中}
\documentclass{ctexart}
\usepackage{array,booktabs}
\begin{document}
\begin{table}[ht]
  \centering
  \caption{||北京冬奥会吉祥物}
  \begin{tabular}{lp{3cm}}
    \toprule
|\highlightline|\multicolumn{1}{c}{||吉祥物} &
|\highlightline|\multicolumn{1}{c}{||描述} \\
    \midrule
||冰墩墩 & 2022 年北京冬季奥运会吉祥物  \\
||雪容融 & 2022 年北京冬季残奥会吉祥物  \\
    \bottomrule
  \end{tabular}
\end{table}
\end{document}
      \end{codeblock}
    \end{column}
    \begin{column}{0.4\textwidth}
      \cmd{multicolumn} 命令不仅可以用于合并同行的单元格,还可以用于临时地屏蔽表格导言对该列的对齐方式定义。这里用于居中表头。
      \begin{center}
        \begin{stampbox}
          \parbox{0.85\linewidth}{
            \ttfamily\color{blue}\textbackslash{}multicolumn\{格数\}\{对齐方式\}\{内容\}
          }
        \end{stampbox}
      \end{center}
      跨页表格考虑使用 \pkg{longtable} 宏包。带标注的表格可以考虑使用 \pkg{threeparttable} 宏包。考虑使用在线工具生成表格代码 \link{https://www.tablesgenerator.com/latex_tables}。
    \end{column}
  \end{columns}
\end{frame}

\section{数学公式}
\begin{frame}
  \frametitle{数学模式}
  \begin{alertblock}{}
  输入数学公式是 \LaTeX{} 的绝对强项,很多常见的公式服务依赖于一些轻量级渲染引擎比如 MathJax, K\kern-.3ex\raise.4ex\hbox{\footnotesize A}\kern-.3ex\TeX{}。但是它们实际上使用的是 \LaTeX{} 语法变种,也就是说并没有使用 \LaTeX{} 后端。所以不要期望有完全一致的输出。
  \end{alertblock}
  
  为了更好的获得数学公式输入支持,请使用 \hologo{AmS}math 宏包。数学模式分为两种:
  \begin{description}
    \item[行内模式] 一般通过一对美元符号(\$$\cdots$\$)标记,可以使用编辑器内建的符号表输入数学符号,也可以使用在线工具手写识别 \link{https://detexify.kirelabs.org/classify.html}。
    \item[行间模式] 一般通过 \env{equation} 环境\footnote{这是有编号环境,其加星号的变种 \env{equation*} 用于生成无编号环境。}输入。如果需要使用多行公式,请使用 \env{align} 环境,并按照类似表格输入的方式,使用 \& 对齐符号,\textbackslash\textbackslash{} 换行。如果不想手动居中,可以考虑多行自动居中的 \env{gather} 和单个大型公式首尾两端对齐 \env{multline}。
  \end{description}
\end{frame}

\begin{frame}
  \frametitle{数学命令展示}
  \begin{columns}
    \begin{column}{0.33\textwidth}
      \begin{exampleblock}{}
        $PV=nRT$
      \end{exampleblock}
      \begin{exampleblock}{}
        $\sum_{i=1}^ki^2=\frac{n(n+1)(2n+1)}{6}$
      \end{exampleblock}
      \begin{exampleblock}{}
        $T(n) = aT\left(\left\lceil\frac{n}{b}\right\rceil\right) + \mathcal{O}(n^d)$
      \end{exampleblock}
      \begin{exampleblock}{}
        $\frac{x_{1}+x_{2}+x_{3}}{3}\geq \sqrt[3]{x_{1}x_{2}x_{3}}$
      \end{exampleblock}
      \begin{exampleblock}{}
        $n=(\underbrace{1\cdots 1}_{k\text{ of 1's}})_2=2^{k+1}-1$
      \end{exampleblock}
      \begin{exampleblock}{}
        $\nabla f (P)= \left.\left(\frac{\partial f}{\partial x},\frac{\partial f}{\partial y},\frac{\partial f}{\partial z}\right)\right|_{P}$
      \end{exampleblock}
    \end{column}
    \begin{column}{0.67\textwidth}
      \begin{exampleblock}{}
        \begin{equation*}
          \int_{a}^b f(x)\,\mathrm{d}x=\lim_{|P|\rightarrow 0}\sum_{i=1}^n f(\xi_i)\Delta x_i
        \end{equation*}
      \end{exampleblock}
      \begin{exampleblock}{}
        \begin{equation}
          T(n) = \begin{cases}
            \mathcal{O}(n^d),&\textrm{if } d>\log_b a, \\
            \mathcal{O}(n^d\log n), &\textrm{if } d=\log_b a,\\
            \mathcal{O}(n^{\log_b a}), &\textrm{if } d<\log_b a.
          \end{cases}
        \end{equation}
      \end{exampleblock}
      \begin{exampleblock}{}
        \begin{align}
          Q^{T}A&=R \\
          \begin{pmatrix}
            q_1^T \\ q_2^T \\ q_3^T
          \end{pmatrix}
          \begin{pmatrix}
            a_1 & a_2 & a_3
          \end{pmatrix}
          &=R
        \end{align}
      \end{exampleblock}
    \end{column}
  \end{columns}
\end{frame}

%更深入地讲解 mathtools, unicode-math, siunix

\section{引用}
\begin{frame}[fragile]
  \frametitle{交叉引用}
  \only<1>{
    正如之前所提到的,\LaTeX{} 中使用 \cmd{label} 标记,然后可以使用 \cmd{ref} 来引用这个标记。 \cmd{label} 可以放在使用计数器的对象之后。
  }
  \only<2>{
    为了使得对公式编号的引用带有括号,推荐使用 \hologo{AmS}math 宏包中的 \cmd{eqref} 命令。对于多行公式环境,每一个换行符前都可以添加一个 \cmd{label} 用于引用该行公式。
  }
  \begin{columns}
    \begin{column}{0.5\textwidth}
      \begin{codeblock}[]{图}
\begin{figure}
|\only<1>{\highlightline}|  \caption{||示例}\label{fig:example}
\end{figure}
      \end{codeblock}
      \begin{codeblock}[]{表}
\begin{table}
|\only<1>{\highlightline}|  \caption{||示例}\label{tab:example}
\end{table}
      \end{codeblock}
    \end{column}
    \begin{column}{0.5\textwidth}
\begin{codeblock}[]{目次}
|\only<1>{\highlightline}|\section{||示例}\label{sec:example}
\end{codeblock}

\begin{codeblock}[]{公式}
\begin{equation}
  a = b + c
|\only<1>{\highlightline}|\label{eq:example}
\end{equation}
|\only<2>{\highlightline}|如公式 \eqref{eq:example} 所示,
\end{codeblock}
    \end{column}
  \end{columns}
\end{frame}

\begin{frame}[fragile]
  \frametitle{文献引用}
  \LaTeX{} 管理参考文献可以采用专用数据库文件 \texttt{.bib},很多的文献管理文件比如 EndNote \link{https://lic.sjtu.edu.cn/Default/softshow/tag/MDAwMDAwMDAwMLGImKE}, Zotero \link{https://www.zotero.org/}, JabRef \link{https://www.jabref.org/} 都可以直接导出这种格式的文件用于 \LaTeX{} 论文中的引用。一般不需要手写数据库文件,你只需要注意每一个文献会在数据库中有一个主键,这个类似于上文的 \cmd{label} 标记,只是要引用数据库中的文献需要使用 \cmd{cite} 命令。
  
  \begin{codeblock}[]{ref.bib}
|\highlightline|@phdthesis{devoftech,|\hfill\alert{\% 类型为博士论文,主键为\texttt{devoftech}}|
  title={||新时期我国信息技术产业的发展},
  author={||江泽民},
  year={2008}
}
  \end{codeblock}
\end{frame}

\begin{frame}
  \frametitle{文献引用}
  而让 \LaTeX{} 处理 \texttt{.bib} 数据库文件与引用有两种工作流。你可能需要查清楚如何在编辑器中设置对应的工作流,或者采用后面所提到的高级编译方式 \texttt{latexmk}。
  \begin{columns}
    \begin{column}{0.5\textwidth}
      \begin{block}{\hologo{BibTeX} + \pkg{gbt7714}}
        一般期刊提交使用这种方法,\pkg{natbib} 宏包将提供命令 \cmd{citet}(文本引用) 和 \cmd{citep}(括号引用)。中文引用可以直接使用 \pkg{gbt7714} 宏包,但是角模式和正文模式不能混用。
      \end{block}
    \end{column}
    \begin{column}{0.5\textwidth}
      \begin{block}{\hologo{biber} + \pkg{biblatex}}
        这是更容易自定义的方法,与 \hologo{BibTeX} 的运作方式稍有不同。\pkg{biblatex} 提供了更加智能的引用命令。而中文引用可以使用 \pkg{biblatex} 宏包的样式 \pkg{gb7714-2015},使用该样式需要使用 \hologo{XeLaTeX} 编译。
      \end{block}
    \end{column}
  \end{columns}
\end{frame}

\begin{frame}[fragile]
  \frametitle{文献引用}
  \begin{columns}
    \begin{column}{0.5\textwidth}
      \begin{codeblock}[]{\hologo{BibTeX} + \pkg{gbt7714}}
\documentclass{ctexart}
\usepackage{gbt7714}
\bibliographystyle{gbt7714-numerial}
% \citestyle{numbers}  % 正文模式
\begin{document}
  ||他指出了...\cite{devoftech}
  \bibliography{ref}
\end{document}
      \end{codeblock}
    \end{column}
    \begin{column}{0.5\textwidth}
      \begin{codeblock}[]{\hologo{biber} + \pkg{biblatex}}
\documentclass{ctexart}
\usepackage[backend=biber,style=gb7714-2015]{biblatex}
\addbibresource{ref.bib}
\begin{document}
  ||他在文献 \parencite{devoftech}
  ||指出了...\cite{devoftech}
  \printbibliography
\end{document}
      \end{codeblock}
    \end{column}
  \end{columns}
\end{frame}

\begin{frame}
  \frametitle{文献引用}
  \begin{columns}
    \begin{column}{0.5\textwidth}
      \includepdflarge{bibtex}
    \end{column}
    \begin{column}{0.5\textwidth}
      \includepdflarge{biblatex}
    \end{column}
  \end{columns}
\end{frame}

} % End of customized shaded number logo

  % !TeX root = ..\..\latex-talk.tex

\part{SJTUThesis}

\begin{frame}
  \frametitle{简介}
  \begin{columns}
    \begin{column}{0.6\textwidth}
      \begin{itemize}
        \item 最早由韦建文于 2009 年 11 月发布 0.1a 版,2018 年起由 SJTUG 接手维护
        \item 最新版:\SJTUThesisVersion{} (\SJTUThesisDate)
        \item 支持本科、硕士、博士学位论文以及课程论文的排版
      \end{itemize}
    \end{column}
    \begin{column}{0.4\textwidth}
      \begin{exampleblock}{}
        \begin{minipage}[c]{1cm}
          \includegraphics[width=0.8cm]{\getcontribpath{sjtug}{vi/sjtug}}
        \end{minipage}
        \begin{minipage}[c]{2cm}
          \href{https://github.com/sjtug}{sjtug}/\href{https://github.com/sjtug/SJTUThesis}{SJTUThesis}
        \end{minipage}
      \end{exampleblock}
      \vspace{-8pt}
      \begin{block}{}
        \scriptsize
        上海交通大学 \hologo{XeLaTeX} 学位论文及课程论文模板 | Shanghai Jiao Tong University \hologo{XeLaTeX} Thesis Template
      \end{block}
      \vspace{-8pt}
      \begin{alertblock}{}
        \scriptsize
        \begin{tabular}{cl}
          \faStar & 2.4k \\
          \faEye & 55 \\
          \faCodeBranch & 701 \\
        \end{tabular}
      \end{alertblock}
    \end{column}
  \end{columns}
\end{frame}

\begin{frame}
  \frametitle{下载与编译}
  \alert{下载} 推荐安装 Git \link{https://git-scm.com/} 后,克隆 SJTUG 镜像仓库
  \begin{exampleblock}{\faGit*}
    \ttfamily\small
    git clone https://mirror.sjtu.edu.cn/git/SJTUThesis.git/
  \end{exampleblock}

  \alert{编译} 推荐使用 \pkg{latexmk} 编译\footnote{\hologo{MiKTeX} 用户需要手动安装 Perl 解释器 \link{https://www.perl.org/get.html} 才能使用 \pkg{latexmk}。},在不能够利用自带的 \texttt{.latexmkrc} 配置文件的情况下,需要查清楚在对应的编辑器中如何使用 \hologo{XeLaTeX} + \hologo{biber} 编译 \link{https://github.com/sjtug/SJTUThesis/blob/master/README.md}。
  \begin{exampleblock}{\faTerminal}
    \ttfamily\small
    latexmk -xelatex main
  \end{exampleblock}

  Overleaf 用户可以下载压缩包后,上传并采用 \hologo{XeLaTeX} 编译方式。
\end{frame}

\begin{frame}
  \frametitle{手动编译}
  \alert{第一次编译失败} 如果没有办法通过通常方式编译成功,请尝试使用文件夹内附带 \faLinux{}\,\faApple{} \texttt{Makefile} 和 \faWindows{} \texttt{Compile.bat} 进行编译。

  \alert{统计字数} 编写过程中也可以使用对应的命令调用 \TeX{}count 来统计正文字数。
  \begin{columns}
    \begin{column}{0.38\textwidth}
      \begin{exampleblock}{\faLinux{}\,\faApple}
        \ttfamily
        make all\\
        make clean\\
        make cleanall\\
        make wordcount
      \end{exampleblock}
    \end{column}
    \begin{column}{0.38\textwidth}
      \begin{exampleblock}{\faWindows}
        \ttfamily
        ./Compile.bat thesis\\
        ./Compile.bat clean\\
        ./Compile.bat cleanall\\
        ./Compile.bat wordcount
      \end{exampleblock}
    \end{column}
    \begin{column}{0.24\textwidth}
      \begin{block}{\faInfo}
        \ttfamily
        编译论文\\
        清理中间文件\\
        $\hookrightarrow +$删除论文\\
        统计字数
      \end{block}
    \end{column}
  \end{columns}
\end{frame}

\begin{frame}[label=compile]
  \frametitle{编译问题排查}
  \begin{columns}
    \begin{column}{0.33\textwidth}
      \begin{alertblock}{无法使用 \texttt{latexmk}\thesisissue{578}}
        \hologo{MiKTeX} 需要安装 Perl 解释器。
      \end{alertblock}  
      \begin{alertblock}{C\TeX{} 套装无法编译\thesisissue{446}}
        使用最新 \TeX{} 发行版。
      \end{alertblock}
      \begin{alertblock}{\hologo{pdfLaTeX} 无法编译\thesisissue{444}}
        请使用 \texttt{latexmk},或更改编辑器设置以 \hologo{XeLaTeX} 编译。
      \end{alertblock}
    \end{column}
    \begin{column}{0.33\textwidth}
      \begin{alertblock}{缺少字体\thesisissue{564} \thesisdiscuss{598}}
        更换字体集,或者安装对应字体。
      \end{alertblock}
      \begin{alertblock}{缺少汉字\thesisissue{533} \thesisdiscuss{617}}
        去除使用 fandol 字体集的设定。或者是安装字体后,改用 \texttt{fontset=adobe} 或 \texttt{fontset=founder}。
      \end{alertblock}
    \end{column}
    \begin{column}{0.33\textwidth}
      \begin{block}{\faInfoCircle{} README}
        不同编辑器的设置请首先参阅 README \link{https://github.com/sjtug/SJTUThesis/blob/master/README.md} 文档。
      \end{block}
      \begin{block}{\faBookOpen{} Wiki}
        其他编译问题推荐查阅 Wiki \link{https://github.com/sjtug/SJTUThesis/wiki} 的使用说明部分。
      \end{block}
    \end{column}
  \end{columns}
\end{frame}

\begin{frame}[fragile, label=covers]
  \begin{codeblock}[firstnumber=3]{main.tex}
|\alert{\% 载入 SJTUThesis 模版}|
\documentclass[|\only<1>{\highlight{type}}\only<2>{type}|=|\only<1>{bachelor}\only<2>{\highlight{bachelor}}|]{sjtuthesis}
  \end{codeblock}
  \begin{figure}
    \parbox{0.9\textwidth}{
      \begin{subfigure}{0.20\textwidth}
        \framebox{\includegraphics[width=\linewidth]{support/thesis/bachelor}}
        \caption{\only<1>{学士}\only<2>{\texttt{bachelor}}}
      \end{subfigure}\hfill
      \begin{subfigure}{0.20\textwidth}
        \framebox{\includegraphics[width=\linewidth]{support/thesis/master}}
        \caption{\only<1>{硕士}\only<2>{\texttt{master}}}
      \end{subfigure}\hfill
      \begin{subfigure}{0.20\textwidth}
        \framebox{\includegraphics[width=\linewidth]{support/thesis/doctor}}
        \caption{\only<1>{博士}\only<2>{\texttt{doctor}}}
      \end{subfigure}\hfill
      \begin{subfigure}{0.20\textwidth}
        \framebox{\includegraphics[width=\linewidth]{support/thesis/course}}
        \caption{\only<1>{课程}\only<2>{\texttt{course}}}
      \end{subfigure}
      \caption{论文类型示例\only<2>{ \texttt{type}}}
    }
  \end{figure}
\end{frame}

\begin{frame}[fragile]
  \frametitle{文档类选项}
  % \framesubtitle{\textbackslash{}documentclass\{sjtuthesis\}}
  \begin{columns}
    \begin{column}{0.45\textwidth}
      \includegraphics[page=10]{thesisdir}
    \end{column}
    \begin{column}{0.55\textwidth}
      \begin{table}[H]
        \caption{文档类选项}
        \footnotesize
        \begin{tabular}{>{\ttfamily}rll}
          \toprule
          选项 & 含义 & 相关 \\
          \midrule
          type= & 指定论文类型 & 第 \ref{covers} 页\\
          fontset= & 指定字体 & 第 \ref{compile} 页\\
          \midrule
          review & 开启盲审模式 & \thesisissue{195} \thesisissue{686} \\
          twoside & 双页模式 & \thesisissue{554} \\
          oneside & 单页模式 & \thesisissue{694} \\
          openright & 章从奇数页开始 & \thesisdiscuss{724} \\
          openany & 章从任意页开始 & \thesisissue{446} \\
          \bottomrule
        \end{tabular}
      \end{table}
    \end{column}
  \end{columns}
\end{frame}

\begin{frame}[fragile]
  \frametitle{基本配置}
  \framesubtitle{\textbackslash{}input\{setup\}}
  \begin{columns}
    \begin{column}{0.45\textwidth}
      \includegraphics[page=9]{thesisdir}
    \end{column}
    \begin{column}{0.55\textwidth}
      \begin{codeblock}[firstnumber=12]{main.tex}
|\highlightline<1>|% 论文基本配置,加载宏包等全局配置
|\highlightline<1>|\input{setup}

\begin{document}

%TC:ignore

|\highlightline<2>|% 标题页
|\highlightline<2>|\maketitle
      \end{codeblock}
      \visible<2>{
        \cmd{sjtusetup} 中的 \pkg{info} 将会修改封面的信息设置(见第 \ref{covers} 页)。
      }
    \end{column}
  \end{columns}
\end{frame}

\begin{frame}[fragile]
  \frametitle{基本配置}
  \framesubtitle{\textbackslash{}sjtusetup}
  \begin{columns}
    \begin{column}{0.45\textwidth}
      \includegraphics[page=1]{thesisdir}
    \end{column}
    \begin{column}{0.55\textwidth}
      \begin{codeblock}[firstnumber=3]{setup.tex}
\sjtusetup{
  info = {
    title    = {||上海交通大学学位论文 \LaTeX{} 模板示例文档},
    title*   = {A Sample for \LaTeX-based SJTU Thesis Template},
    author   = {||某\quad{}某},
    author* = {Mo Mo},
  },
  style = { header-logo-color = red, 
  },
  name = {
    publications = {||攻读学位期间完成的论文},
  },
}
      \end{codeblock}
    \end{column}
  \end{columns}
\end{frame}

\begin{frame}
  \frametitle{基本配置}
  \framesubtitle{\textbackslash{}sjtusetup}
  \begin{columns}
    \begin{column}{0.45\textwidth}
      \includegraphics[page=1]{thesisdir}
    \end{column}
    \begin{column}{0.55\textwidth}
      \begin{table}[H]
        \centering
        \caption{info 域}
        \footnotesize
        \begin{tabular}{lll} \toprule
          命令作用 & 中文对应选项 & 英文对应选项 \\ \midrule
          论文标题 & \texttt{title} & \texttt{title*} \\
          关键字列表 & \texttt{keywords} & \texttt{keywords*} \\
          作者姓名&  \texttt{author} &\texttt{author*}\\
          申请学位名称 & \texttt{degree}&\texttt{degree*}\\
          院系名称 & \texttt{department} & \texttt{department*}\\
          专业名称 & \texttt{major} & \texttt{major*}\\
          导师 & \texttt{supervisor} & \texttt{supervisor*}\\
          副导师 & \texttt{assisupervisor} & \texttt{assisupervisor*}\\
          日期 & \multicolumn{2}{c}{\texttt{date}}\\
          学号 & \multicolumn{2}{c}{\texttt{id}}\\ \bottomrule
          \end{tabular}
      \end{table}
    \end{column}
  \end{columns}
\end{frame}

\begin{frame}[fragile]
  \frametitle{版权页}
  \framesubtitle{\textbackslash{}copyrightpage}
  \begin{columns}
    \begin{column}{0.45\textwidth}
      \only<1>{
        \includegraphics[page=9]{thesisdir}
      }
      \only<2>{
        \includegraphics[page=2]{thesisdir}
      }
      \only<3>{
        \begin{figure}[H]
          \framebox{\includegraphics[page=2,width=0.4\linewidth]{bachelor}}
          \caption{版权页}
        \end{figure}
      }
    \end{column}
    \begin{column}{0.55\textwidth}
      \begin{codeblock}[firstnumber=22]{main.tex}
|\highlightline<1>|% 原创性声明及使用授权书
|\highlightline<1>|\copyrightpage
|\highlightline<2>|% 插入外置原创性声明及使用授权书
|\highlightline<2>|% \copyrightpage[scans/sample-copyright-old.pdf]
      \end{codeblock}
      \only<1>{
        \cmd{copyrightpages} 可以用于插入版权页。
      }
      \only<2>{
        \cmd{copyrightpages} 也接受一个可选参数,用于直接使用扫描件。
      }
    \end{column}
  \end{columns}
\end{frame}

\begin{frame}[fragile]
  \frametitle{前置部分}
  \framesubtitle{\textbackslash{}frontmatter}
  \begin{columns}
    \begin{column}{0.45\textwidth}
      \only<1>{
        \includegraphics[page=9]{thesisdir}
      }
      \only<2>{
        \includegraphics[page=3]{thesisdir}
      }
      \only<3>{
        \begin{figure}[H]
          \begin{subfigure}{0.45\textwidth}
            \framebox{\includegraphics[page=3,width=\linewidth]{bachelor}}
            \caption{中文}
          \end{subfigure}\hfill
          \begin{subfigure}{0.45\textwidth}
            \framebox{\includegraphics[page=4,width=\linewidth]{bachelor}}
            \caption{英文}
          \end{subfigure}
          \caption{摘要}
        \end{figure}
      }
      \only<4>{
        \begin{figure}[H]
          \begin{subfigure}{0.30\linewidth}
            \centering
            \framebox{\includegraphics[page=5,width=0.6\linewidth]{bachelor}}
            \caption{目录}
          \end{subfigure}
          \begin{subfigure}{0.30\linewidth}
            \centering
            \framebox{\includegraphics[page=6,width=0.6\linewidth]{bachelor}}
            \caption{插图}
          \end{subfigure}

          \begin{subfigure}{0.30\linewidth}
            \centering
            \framebox{\includegraphics[page=7,width=0.6\linewidth]{bachelor}}
            \caption{表格}
          \end{subfigure}
          \begin{subfigure}{0.30\linewidth}
            \centering
            \framebox{\includegraphics[page=8,width=0.6\linewidth]{bachelor}}
            \caption{算法}
          \end{subfigure}
          \caption{索引}
        \end{figure}
      }
      \only<5>{
        \includegraphics[page=4]{thesisdir}
      }
      \only<6>{
        \begin{figure}[H]
          \framebox{\includegraphics[page=9,width=0.5\linewidth]{bachelor}}
          \caption{符号对照表}
        \end{figure}
      }
    \end{column}
    \begin{column}{0.55\textwidth}
      \begin{codeblock}[firstnumber=30]{main.tex}
|\highlightline<2-3>|% 摘要
|\highlightline<2-3>|\input{contents/abstract}

|\highlightline<4>|% 目录
|\highlightline<4>|\tableofcontents
|\highlightline<4>|% 插图索引
|\highlightline<4>|\listoffigures*
|\highlightline<4>|% 表格索引
|\highlightline<4>|\listoftables*
|\highlightline<4>|% 算法索引
|\highlightline<4>|\listofalgorithms*

|\highlightline<5-6>|% 符号对照表
|\highlightline<5-6>|\input{contents/nomenclature}
      \end{codeblock}
    \end{column}
  \end{columns}
\end{frame}

\begin{frame}[fragile]
  \frametitle{主体部分}
  \framesubtitle{\textbackslash{}mainmatter}
  \begin{columns}
    \begin{column}{0.45\textwidth}
      \only<1>{
        \includegraphics[page=5]{thesisdir}
      }
      \only<2>{
        \begin{figure}[H]
          \begin{subfigure}{0.30\linewidth}
            \centering
            \framebox{\includegraphics[page=11,width=0.6\linewidth]{bachelor}}
            \caption{简介}
          \end{subfigure}
          \begin{subfigure}{0.30\linewidth}
            \centering
            \framebox{\includegraphics[page=13,width=0.6\linewidth]{bachelor}}
            \caption{数学}
          \end{subfigure}

          \begin{subfigure}{0.30\linewidth}
            \centering
            \framebox{\includegraphics[page=16,width=0.6\linewidth]{bachelor}}
            \caption{浮动体}
          \end{subfigure}
          \begin{subfigure}{0.30\linewidth}
            \centering
            \framebox{\includegraphics[page=22,width=0.6\linewidth]{bachelor}}
            \caption{总结}
          \end{subfigure}
          \caption{主体部分}
        \end{figure}
      }
    \end{column}
    \begin{column}{0.55\textwidth}
      \begin{codeblock}[firstnumber=47]{main.tex}
|\highlightline|% 正文内容
|\highlightline|\input{contents/intro}
|\highlightline|\input{contents/math_and_citations}
|\highlightline|\input{contents/floats}
|\highlightline|\input{contents/summary}

%TC:ignore

% 参考文献
\printbibliography[heading=bibintoc]
      \end{codeblock}
    \end{column}
  \end{columns}
\end{frame}

\begin{frame}
  \frametitle{数学}
  \begin{itemize}
    \item 公式示例:\nolinkurl{contents/math_and_citations.tex}
    \item \SJTUThesis{} 定义了常用的数学环境(需要手工引入 \texttt{ntheorem} 宏包):
      \begin{table}[h]
        \centering
        \footnotesize
        \begin{tabular}{*{7}{l}}\toprule
          assumption  & axiom   & conjecture & corollary    & definition  & example   & exercise  \\
          假设        & 公理    & 猜想       & 推论         & 定义        & 例        & 练习      \\\midrule
          lemma       & problem & proof      & proposition  & remark      & solution  & theorem   \\
          引理        & 问题    & 证明       & 命题         & 注          & 解        & 定理      \\\bottomrule
        \end{tabular}
      \end{table}
      \item \SJTUThesis{} 可以通过 \texttt{unimath} 选项使用 \pkg{unicode-math} 进行数学输入,注意与传统方式的区别。\thesisissue{555}
  \end{itemize}
\end{frame}

\begin{frame}[fragile]
  \frametitle{参考文献}
  \begin{columns}
    \begin{column}{0.45\textwidth}
      \includegraphics[page=6]{thesisdir}
    \end{column}
    \begin{column}{0.55\textwidth}
      \begin{codeblock}[firstnumber=111,numbersep=2pt]{setup.tex}
% 使用 BibLaTeX 处理参考文献
%   biblatex-gb7714-2015 常用选项
%     gbnamefmt=lowercase     姓名大小写由输入信息确定
%     gbpub=false             禁用出版信息缺失处理
\usepackage[backend=biber,style=gb7714-2015]{biblatex}
% 文献表字体
% \renewcommand{\bibfont}{\zihao{-5}}
% 文献表条目间的间距
\setlength{\bibitemsep}{0pt}
|\highlightline|% 导入参考文献数据库
|\highlightline|\addbibresource{bibdata/thesis.bib}
      \end{codeblock}
    \end{column}
  \end{columns}
\end{frame}

\begin{frame}[fragile]
  \frametitle{附录}
  \framesubtitle{\textbackslash{}appendix}
  \begin{columns}
    \begin{column}{0.45\textwidth}
      \only<1>{
        \includegraphics[page=7]{thesisdir}
      }
      \only<2>{
        \begin{figure}[H]
          \begin{subfigure}{0.45\linewidth}
            \framebox{\includegraphics[width=\linewidth,page=24]{bachelor}}
            \caption{}
          \end{subfigure}\hfill
          \begin{subfigure}{0.45\textwidth}
            \framebox{\includegraphics[width=\linewidth,page=25]{bachelor}}
            \caption{}
          \end{subfigure}
          \caption{附录}
        \end{figure}
      }
    \end{column}
    \begin{column}{0.55\textwidth}
      \begin{codeblock}[firstnumber=61]{main.tex}
% 附录中图表不加入索引
\captionsetup{list=no}

% 附录内容
|\highlightline|\input{contents/app_maxwell_equations}
|\highlightline|\input{contents/app_flow_chart}
      \end{codeblock}
    \end{column}
  \end{columns}
\end{frame}

\begin{frame}[fragile]
  \frametitle{结尾部分}
  \framesubtitle{\textbackslash{}backmatter}
  \begin{columns}
    \begin{column}{0.45\textwidth}
      \only<1>{
        \includegraphics[page=8]{thesisdir}
      }
      \only<2>{
        \begin{figure}[H]
          \begin{subfigure}{0.30\linewidth}
            \centering
            \framebox{\includegraphics[page=26,width=0.6\linewidth]{bachelor}}
            \caption{致谢}
          \end{subfigure}
          \begin{subfigure}{0.30\linewidth}
            \centering
            \framebox{\includegraphics[page=27,width=0.6\linewidth]{bachelor}}
            \caption{成就}
          \end{subfigure}

          \begin{subfigure}{0.30\linewidth}
            \centering
            \framebox{\includegraphics[page=28,width=0.6\linewidth]{bachelor}}
            \caption{简历}
          \end{subfigure}
          \begin{subfigure}{0.30\linewidth}
            \centering
            \framebox{\includegraphics[page=29,width=0.6\linewidth]{bachelor}}
            \caption{大摘要*}
          \end{subfigure}
          \caption{结尾部分}
        \end{figure}
      }
    \end{column}
    \begin{column}{0.55\textwidth}
      \begin{codeblock}[firstnumber=76]{main.tex}
% 致谢
\input{contents/acknowledgements}

% 发表论文及科研成果
% 盲审论文中,发表论文及科研成果等仅以第几作者注明即可,不要出现作者或他人姓名
\input{contents/achievements}

% 简历
\input{contents/resume}

% 学士学位论文要求在最后有一个大摘要,单独编页码
\input{contents/digest}
      \end{codeblock}
    \end{column}
  \end{columns}
\end{frame}

\begin{frame}
  \frametitle{还有其他问题?}
  \begin{columns}
    \begin{column}{0.75\textwidth}
    \begin{itemize}
      \item[{\faComment*[regular]}] 日常模板或 \LaTeX{} 使用问题可以前往 Discussions \link{https://github.com/sjtug/SJTUThesis/discussions} 提问
      
      (解决后别忘了 \textcolor{green}{\faCheckCircle{} Mark as answer}
      \item[{\faDotCircle[regular]}] 如果是 \textsc{SJTUThesis} 项目本身的 bug 和 feature request
      
      可以通过 Issues \link{https://github.com/sjtug/SJTUThesis/issues} 反馈。
      \item[{\faCodeBranch}] 如果你有好点子,可以贡献代码
     
      向 \textsc{SJTU\TeX{}}(v1) \link{https://github.com/sjtug/SJTUTeX/tree/v1} 存储库发 PR,\par
      而后把解包结果同步到 \textsc{SJTUThesis}。
  
      \item[{\faTag}] 如果你对正在基于 \LaTeX3 开发的新版感兴趣,\par
      也欢迎向 \textsc{SJTU\TeX{}}(v2) \link{https://github.com/sjtug/SJTUTeX/tree/v2} 发 PR。
  
      \item[{\faQq}] 也欢迎在 QQ 群即时讨论。
    \end{itemize}
    \end{column}
    \begin{column}{0.25\textwidth}
      \includegraphics[height=0.7\textheight]{qq.jpg}
    \end{column}
  \end{columns}
\end{frame}
\end{document}
      \end{codeblock}
    \end{column}
  \end{columns}
  \footnotetext{如果想强制指定子文档的主文档,可以在文件第一行输入魔术命令:\texttt{\% !TeX root = main.tex}}
\end{frame}

\section{图}
\begin{frame}[fragile]%
  \frametitle{\temporal<5>{插图}{浮动体}{插图}}
  \begin{columns}
    \begin{column}{0.6\textwidth}
      \begin{codeblock}[]{插入单图\only<4->{最佳实践}}
\documentclass{ctexart}
|\only<2>{\highlightline}|\usepackage{graphicx}
|\only<2>{\highlightline}|\graphicspath{{figs/}{pics/}}
\begin{document}
|\only<5>{\highlightline}|\begin{figure}[ht]
|\only<6>{\highlightline}|  \centering
|\only<3>{\highlightline}|  \includegraphics[width=|\only<1-3>{4cm}\only<4->{0.4\textbackslash{}textwidth}|]{sjtug}
|\only<7>{\highlightline}|  \caption{SJTUG 徽标}\label{fig:sjtug}
|\only<5>{\highlightline}|\end{figure}
\end{document}
      \end{codeblock}
    \end{column}
    \begin{column}{0.4\textwidth}
      \only<1>{
        \includepdflarge{insertimage}
      }
      \only<2>{
        为了插入外部图片,需要使用 \pkg{graphicx} 宏包。之后在文档主体便可以使用 \cmd{includegraphics} 插入图片。导言区也可以加入 \cmd{graphicspath} 指定图片文件夹\footnotemark。
      }
      \only<3>{
        \cmd{includegraphics} 命令便以相对路径的方式插入图片,如果无同名图片,那么后缀名可以省略。可以使用可选参数指定插入的图片尺寸,最佳实践是使用 \cmd{textwidth} 或 \cmd{linewidth} 的相对值指定尺寸大小,以在未来可能的布局更改中保留一定的灵活性。
      }
      \only<4>{
        也可以通过键值对的方法设置图片的其他属性。
        \begin{center}
          \footnotesize
          \begin{stampbox}
            \begin{tabular}{rl}
              \pkg{width} & 宽度 \\
              \pkg{height} & 高度 \\
              \pkg{scale} & 缩放 \\
              \pkg{angle} & 角度 \\
            \end{tabular}
          \end{stampbox}
        \end{center}
      }
      \only<5>{
        \env{figure} 为一个浮动体环境(\env{table} 也是),可以将其移动到其他位置上以减少行文中的空白。可以添加可选参数以指定如何放置浮动体,最多可以使用四种位置描述符:
        \begin{center}
          \footnotesize
          \begin{stampbox}
            \begin{tabular}{cll}
              \pkg{h} & Here & 尽可能在这里 \\
              \pkg{t} & Top & 页面顶部 \\
              \pkg{b} & Bottom & 页面底部 \\
              \pkg{p} & Page & 浮动体专页 \\
            \end{tabular}
          \end{stampbox}
        \end{center}
        还可以只使用 \pkg{float} 宏包提供的 \pkg{H} 描述符以强制置于此处。
      }
      \only<6>{
        采用 \cmd{centering} 命令而不是 \env{center} 环境来水平居中图片。这将避免多余的纵向间距。
      }
      \only<7>{
        使用 \cmd{caption} 命令输入题注,如果这个命令写在插入图片前面,题注将会在上方(中文中一般对 \env{table} 环境这么做)。后面将会看到如何对留有标记(\cmd{label})的图片进行引用。
      }
    \end{column}
  \end{columns}
  \only<2>{\footnotetext{其命令参数每个为一个以 \texttt{/} 结尾的文件夹,每个文件夹需要使用大括号包裹起来。}}
\end{frame}

\begin{frame}[fragile]
  \begin{columns}
    \begin{column}{0.6\textwidth}
      \begin{codeblock}[]{插入双图}
\documentclass{ctexart}
\usepackage{graphicx}
\graphicspath{{figs/}{pics/}}
\begin{document}
  \begin{figure}[ht]
|\only<1>{\highlightline}|    \begin{minipage}{0.48\textwidth}
      \centering
      \includegraphics[height=2cm]{sjtug}
|\only<2>{\highlightline}|      \caption{SJTUG 徽标}\label{fig:sjtug}
|\only<1>{\highlightline}|    \end{minipage}\hfill
|\only<1>{\highlightline}|    \begin{minipage}{0.48\textwidth}
      \centering
      \includegraphics[height=2cm]{sjtugt}
|\only<2>{\highlightline}|      \caption{SJTUG||文字}\label{fig:sjtugt}
|\only<1>{\highlightline}|    \end{minipage}
  \end{figure}
\end{document}
      \end{codeblock}
    \end{column}
    \begin{column}{0.4\textwidth}
      \only<1>{
        在 \env{figure} 环境里使用 \env{minipage} 小页制作列盒子,以并排两图并分别编号,需要设定强制参数以指定列宽。两个小页之间添加 \cmd{hfill} 使两个小页两端对齐。
      }
      \only<2>{
        在每个小页内部分别使用 \cmd{caption} 添加标签。
      }
      \only<3>{
        \includepdflarge{doubleimages}
      }
    \end{column}
  \end{columns}
\end{frame}

\begin{frame}[fragile]%
  \begin{columns}
    \begin{column}{0.6\textwidth}
      \begin{codeblock}[]{}
\documentclass{ctexart}
\usepackage{graphicx}
|\highlightline|\usepackage{subcaption}
\graphicspath{{figs/}{pics/}}
\begin{document}
  \begin{figure}[ht]
|\highlightline|    \begin{subfigure}{0.48\textwidth}
      \centering
      \includegraphics[height=2cm]{sjtug}
      \caption{||徽标}
|\highlightline|    \end{subfigure}\hfill
|\highlightline|    \begin{subfigure}{0.48\textwidth}
      \centering
      \includegraphics[height=2cm]{sjtugt}
      \caption{||文字}
|\highlightline|    \end{subfigure}
    \caption{SJTUG}\label{fig:sjtug}
  \end{figure}
\end{document}
      \end{codeblock}
    \end{column}
    \begin{column}{0.4\textwidth}
      \includepdflarge{subfigures}\vspace{15pt}
      \pkg{subcaption} 宏包提供了 \env{subfigure} 环境(以及 \env{subtable}),可以用于以子图的形式插入,编号会增加一级。也可以为子图添加子集引用编号。
    \end{column}
  \end{columns}
\end{frame}

\section{表}
\begin{frame}[fragile]
  \frametitle{简单表格}
  \begin{columns}
    \begin{column}{0.6\textwidth}
      \begin{codeblock}[]{}
\documentclass{ctexart}
|\only<1-2>{\highlightline}|\usepackage{|\temporal<1>{array}{\highlight{array}}{array},\temporal<2>{booktabs}{\highlight{booktabs}}{booktabs}|}
\begin{document}
\begin{table}[ht]
  \centering
  \caption{||北京冬奥会吉祥物}
|\only<1>{\highlightline}|  \begin{tabular}{lp{3cm}}
|\only<2>{\highlightline}|    \toprule
|\only<3>{\highlightline}|吉祥物 & 描述                          \\
|\only<2>{\highlightline}|    \midrule
|\only<3>{\highlightline}|冰墩墩 & 2022 年北京冬季奥运会吉祥物  \\
|\only<3>{\highlightline}|雪容融 & 2022 年北京冬季残奥会吉祥物  \\
|\only<2>{\highlightline}|    \bottomrule
|\only<1>{\highlightline}|  \end{tabular}
\end{table}
\end{document}
      \end{codeblock}
    \end{column}
    \begin{column}{0.4\textwidth}
      \only<1>{
        使用 \env{tabular} 环境绘制表格。需要添加参数(称为\textbf{表格导言})以确定每一列的对齐方式。引入 \pkg{array} 宏包来使用更多的\textbf{引导符}。
        \begin{center}
          \footnotesize
          \begin{stampbox}
            \begin{tabular}{>{\ttfamily}ll}
              \alert{l} & 向左对齐 \\
              \alert{c} & 居中对齐 \\
              \alert{r} & 向右对齐 \\
              \alert{p\{3cm\}} & 固定列宽,两端对齐 \\
              \alert{m\{3cm\}} & \texttt{p} + 垂直居中对齐 \\
              \alert{>\{\textbackslash{}bfseries\}} & 后一列单元格前加命令 \\
              \alert{*\{3\}\{l\}} & 三个左对齐列 \\
            \end{tabular}
          \end{stampbox}
        \end{center}
      }
      \only<2>{
        \pkg{booktabs} 宏包提供了标准三线表格所需要的行分割线:\cmd{toprule},\cmd{midrule},\cmd{bottomrule}。也可以使用 \cmd{cmidrule\{1-2\}} 来部分地绘制行分割线。一般不推荐在专业表格中使用纵向分割线。
      }
      \only<3>{
        每行内容使用 \textbackslash\textbackslash{} 分隔开,每行中的单元格使用 \& 分隔开。
      }
      \only<4>{
        \includepdflarge{table}
      }
    \end{column}
  \end{columns}
\end{frame}

\begin{frame}[fragile]%
  \begin{columns}
    \begin{column}{0.6\textwidth}
      \begin{codeblock}[]{表头居中}
\documentclass{ctexart}
\usepackage{array,booktabs}
\begin{document}
\begin{table}[ht]
  \centering
  \caption{||北京冬奥会吉祥物}
  \begin{tabular}{lp{3cm}}
    \toprule
|\highlightline|\multicolumn{1}{c}{||吉祥物} &
|\highlightline|\multicolumn{1}{c}{||描述} \\
    \midrule
||冰墩墩 & 2022 年北京冬季奥运会吉祥物  \\
||雪容融 & 2022 年北京冬季残奥会吉祥物  \\
    \bottomrule
  \end{tabular}
\end{table}
\end{document}
      \end{codeblock}
    \end{column}
    \begin{column}{0.4\textwidth}
      \cmd{multicolumn} 命令不仅可以用于合并同行的单元格,还可以用于临时地屏蔽表格导言对该列的对齐方式定义。这里用于居中表头。
      \begin{center}
        \begin{stampbox}
          \parbox{0.85\linewidth}{
            \ttfamily\color{blue}\textbackslash{}multicolumn\{格数\}\{对齐方式\}\{内容\}
          }
        \end{stampbox}
      \end{center}
      跨页表格考虑使用 \pkg{longtable} 宏包。带标注的表格可以考虑使用 \pkg{threeparttable} 宏包。考虑使用在线工具生成表格代码 \link{https://www.tablesgenerator.com/latex_tables}。
    \end{column}
  \end{columns}
\end{frame}

\section{数学公式}
\begin{frame}
  \frametitle{数学模式}
  \begin{alertblock}{}
  输入数学公式是 \LaTeX{} 的绝对强项,很多常见的公式服务依赖于一些轻量级渲染引擎比如 MathJax, K\kern-.3ex\raise.4ex\hbox{\footnotesize A}\kern-.3ex\TeX{}。但是它们实际上使用的是 \LaTeX{} 语法变种,也就是说并没有使用 \LaTeX{} 后端。所以不要期望有完全一致的输出。
  \end{alertblock}
  
  为了更好的获得数学公式输入支持,请使用 \hologo{AmS}math 宏包。数学模式分为两种:
  \begin{description}
    \item[行内模式] 一般通过一对美元符号(\$$\cdots$\$)标记,可以使用编辑器内建的符号表输入数学符号,也可以使用在线工具手写识别 \link{https://detexify.kirelabs.org/classify.html}。
    \item[行间模式] 一般通过 \env{equation} 环境\footnote{这是有编号环境,其加星号的变种 \env{equation*} 用于生成无编号环境。}输入。如果需要使用多行公式,请使用 \env{align} 环境,并按照类似表格输入的方式,使用 \& 对齐符号,\textbackslash\textbackslash{} 换行。如果不想手动居中,可以考虑多行自动居中的 \env{gather} 和单个大型公式首尾两端对齐 \env{multline}。
  \end{description}
\end{frame}

\begin{frame}
  \frametitle{数学命令展示}
  \begin{columns}
    \begin{column}{0.33\textwidth}
      \begin{exampleblock}{}
        $PV=nRT$
      \end{exampleblock}
      \begin{exampleblock}{}
        $\sum_{i=1}^ki^2=\frac{n(n+1)(2n+1)}{6}$
      \end{exampleblock}
      \begin{exampleblock}{}
        $T(n) = aT\left(\left\lceil\frac{n}{b}\right\rceil\right) + \mathcal{O}(n^d)$
      \end{exampleblock}
      \begin{exampleblock}{}
        $\frac{x_{1}+x_{2}+x_{3}}{3}\geq \sqrt[3]{x_{1}x_{2}x_{3}}$
      \end{exampleblock}
      \begin{exampleblock}{}
        $n=(\underbrace{1\cdots 1}_{k\text{ of 1's}})_2=2^{k+1}-1$
      \end{exampleblock}
      \begin{exampleblock}{}
        $\nabla f (P)= \left.\left(\frac{\partial f}{\partial x},\frac{\partial f}{\partial y},\frac{\partial f}{\partial z}\right)\right|_{P}$
      \end{exampleblock}
    \end{column}
    \begin{column}{0.67\textwidth}
      \begin{exampleblock}{}
        \begin{equation*}
          \int_{a}^b f(x)\,\mathrm{d}x=\lim_{|P|\rightarrow 0}\sum_{i=1}^n f(\xi_i)\Delta x_i
        \end{equation*}
      \end{exampleblock}
      \begin{exampleblock}{}
        \begin{equation}
          T(n) = \begin{cases}
            \mathcal{O}(n^d),&\textrm{if } d>\log_b a, \\
            \mathcal{O}(n^d\log n), &\textrm{if } d=\log_b a,\\
            \mathcal{O}(n^{\log_b a}), &\textrm{if } d<\log_b a.
          \end{cases}
        \end{equation}
      \end{exampleblock}
      \begin{exampleblock}{}
        \begin{align}
          Q^{T}A&=R \\
          \begin{pmatrix}
            q_1^T \\ q_2^T \\ q_3^T
          \end{pmatrix}
          \begin{pmatrix}
            a_1 & a_2 & a_3
          \end{pmatrix}
          &=R
        \end{align}
      \end{exampleblock}
    \end{column}
  \end{columns}
\end{frame}

%更深入地讲解 mathtools, unicode-math, siunix

\section{引用}
\begin{frame}[fragile]
  \frametitle{交叉引用}
  \only<1>{
    正如之前所提到的,\LaTeX{} 中使用 \cmd{label} 标记,然后可以使用 \cmd{ref} 来引用这个标记。 \cmd{label} 可以放在使用计数器的对象之后。
  }
  \only<2>{
    为了使得对公式编号的引用带有括号,推荐使用 \hologo{AmS}math 宏包中的 \cmd{eqref} 命令。对于多行公式环境,每一个换行符前都可以添加一个 \cmd{label} 用于引用该行公式。
  }
  \begin{columns}
    \begin{column}{0.5\textwidth}
      \begin{codeblock}[]{图}
\begin{figure}
|\only<1>{\highlightline}|  \caption{||示例}\label{fig:example}
\end{figure}
      \end{codeblock}
      \begin{codeblock}[]{表}
\begin{table}
|\only<1>{\highlightline}|  \caption{||示例}\label{tab:example}
\end{table}
      \end{codeblock}
    \end{column}
    \begin{column}{0.5\textwidth}
\begin{codeblock}[]{目次}
|\only<1>{\highlightline}|\section{||示例}\label{sec:example}
\end{codeblock}

\begin{codeblock}[]{公式}
\begin{equation}
  a = b + c
|\only<1>{\highlightline}|\label{eq:example}
\end{equation}
|\only<2>{\highlightline}|如公式 \eqref{eq:example} 所示,
\end{codeblock}
    \end{column}
  \end{columns}
\end{frame}

\begin{frame}[fragile]
  \frametitle{文献引用}
  \LaTeX{} 管理参考文献可以采用专用数据库文件 \texttt{.bib},很多的文献管理文件比如 EndNote \link{https://lic.sjtu.edu.cn/Default/softshow/tag/MDAwMDAwMDAwMLGImKE}, Zotero \link{https://www.zotero.org/}, JabRef \link{https://www.jabref.org/} 都可以直接导出这种格式的文件用于 \LaTeX{} 论文中的引用。一般不需要手写数据库文件,你只需要注意每一个文献会在数据库中有一个主键,这个类似于上文的 \cmd{label} 标记,只是要引用数据库中的文献需要使用 \cmd{cite} 命令。
  
  \begin{codeblock}[]{ref.bib}
|\highlightline|@phdthesis{devoftech,|\hfill\alert{\% 类型为博士论文,主键为\texttt{devoftech}}|
  title={||新时期我国信息技术产业的发展},
  author={||江泽民},
  year={2008}
}
  \end{codeblock}
\end{frame}

\begin{frame}
  \frametitle{文献引用}
  而让 \LaTeX{} 处理 \texttt{.bib} 数据库文件与引用有两种工作流。你可能需要查清楚如何在编辑器中设置对应的工作流,或者采用后面所提到的高级编译方式 \texttt{latexmk}。
  \begin{columns}
    \begin{column}{0.5\textwidth}
      \begin{block}{\hologo{BibTeX} + \pkg{gbt7714}}
        一般期刊提交使用这种方法,\pkg{natbib} 宏包将提供命令 \cmd{citet}(文本引用) 和 \cmd{citep}(括号引用)。中文引用可以直接使用 \pkg{gbt7714} 宏包,但是角模式和正文模式不能混用。
      \end{block}
    \end{column}
    \begin{column}{0.5\textwidth}
      \begin{block}{\hologo{biber} + \pkg{biblatex}}
        这是更容易自定义的方法,与 \hologo{BibTeX} 的运作方式稍有不同。\pkg{biblatex} 提供了更加智能的引用命令。而中文引用可以使用 \pkg{biblatex} 宏包的样式 \pkg{gb7714-2015},使用该样式需要使用 \hologo{XeLaTeX} 编译。
      \end{block}
    \end{column}
  \end{columns}
\end{frame}

\begin{frame}[fragile]
  \frametitle{文献引用}
  \begin{columns}
    \begin{column}{0.5\textwidth}
      \begin{codeblock}[]{\hologo{BibTeX} + \pkg{gbt7714}}
\documentclass{ctexart}
\usepackage{gbt7714}
\bibliographystyle{gbt7714-numerial}
% \citestyle{numbers}  % 正文模式
\begin{document}
  ||他指出了...\cite{devoftech}
  \bibliography{ref}
\end{document}
      \end{codeblock}
    \end{column}
    \begin{column}{0.5\textwidth}
      \begin{codeblock}[]{\hologo{biber} + \pkg{biblatex}}
\documentclass{ctexart}
\usepackage[backend=biber,style=gb7714-2015]{biblatex}
\addbibresource{ref.bib}
\begin{document}
  ||他在文献 \parencite{devoftech}
  ||指出了...\cite{devoftech}
  \printbibliography
\end{document}
      \end{codeblock}
    \end{column}
  \end{columns}
\end{frame}

\begin{frame}
  \frametitle{文献引用}
  \begin{columns}
    \begin{column}{0.5\textwidth}
      \includepdflarge{bibtex}
    \end{column}
    \begin{column}{0.5\textwidth}
      \includepdflarge{biblatex}
    \end{column}
  \end{columns}
\end{frame}

} % End of customized shaded number logo

% !TeX root = ..\..\latex-talk.tex

\part{SJTUThesis}

\begin{frame}
  \frametitle{简介}
  \begin{columns}
    \begin{column}{0.6\textwidth}
      \begin{itemize}
        \item 最早由韦建文于 2009 年 11 月发布 0.1a 版,2018 年起由 SJTUG 接手维护
        \item 最新版:\SJTUThesisVersion{} (\SJTUThesisDate)
        \item 支持本科、硕士、博士学位论文以及课程论文的排版
      \end{itemize}
    \end{column}
    \begin{column}{0.4\textwidth}
      \begin{exampleblock}{}
        \begin{minipage}[c]{1cm}
          \includegraphics[width=0.8cm]{\getcontribpath{sjtug}{vi/sjtug}}
        \end{minipage}
        \begin{minipage}[c]{2cm}
          \href{https://github.com/sjtug}{sjtug}/\href{https://github.com/sjtug/SJTUThesis}{SJTUThesis}
        \end{minipage}
      \end{exampleblock}
      \vspace{-8pt}
      \begin{block}{}
        \scriptsize
        上海交通大学 \hologo{XeLaTeX} 学位论文及课程论文模板 | Shanghai Jiao Tong University \hologo{XeLaTeX} Thesis Template
      \end{block}
      \vspace{-8pt}
      \begin{alertblock}{}
        \scriptsize
        \begin{tabular}{cl}
          \faStar & 2.4k \\
          \faEye & 55 \\
          \faCodeBranch & 701 \\
        \end{tabular}
      \end{alertblock}
    \end{column}
  \end{columns}
\end{frame}

\begin{frame}
  \frametitle{下载与编译}
  \alert{下载} 推荐安装 Git \link{https://git-scm.com/} 后,克隆 SJTUG 镜像仓库
  \begin{exampleblock}{\faGit*}
    \ttfamily\small
    git clone https://mirror.sjtu.edu.cn/git/SJTUThesis.git/
  \end{exampleblock}

  \alert{编译} 推荐使用 \pkg{latexmk} 编译\footnote{\hologo{MiKTeX} 用户需要手动安装 Perl 解释器 \link{https://www.perl.org/get.html} 才能使用 \pkg{latexmk}。},在不能够利用自带的 \texttt{.latexmkrc} 配置文件的情况下,需要查清楚在对应的编辑器中如何使用 \hologo{XeLaTeX} + \hologo{biber} 编译 \link{https://github.com/sjtug/SJTUThesis/blob/master/README.md}。
  \begin{exampleblock}{\faTerminal}
    \ttfamily\small
    latexmk -xelatex main
  \end{exampleblock}

  Overleaf 用户可以下载压缩包后,上传并采用 \hologo{XeLaTeX} 编译方式。
\end{frame}

\begin{frame}
  \frametitle{手动编译}
  \alert{第一次编译失败} 如果没有办法通过通常方式编译成功,请尝试使用文件夹内附带 \faLinux{}\,\faApple{} \texttt{Makefile} 和 \faWindows{} \texttt{Compile.bat} 进行编译。

  \alert{统计字数} 编写过程中也可以使用对应的命令调用 \TeX{}count 来统计正文字数。
  \begin{columns}
    \begin{column}{0.38\textwidth}
      \begin{exampleblock}{\faLinux{}\,\faApple}
        \ttfamily
        make all\\
        make clean\\
        make cleanall\\
        make wordcount
      \end{exampleblock}
    \end{column}
    \begin{column}{0.38\textwidth}
      \begin{exampleblock}{\faWindows}
        \ttfamily
        ./Compile.bat thesis\\
        ./Compile.bat clean\\
        ./Compile.bat cleanall\\
        ./Compile.bat wordcount
      \end{exampleblock}
    \end{column}
    \begin{column}{0.24\textwidth}
      \begin{block}{\faInfo}
        \ttfamily
        编译论文\\
        清理中间文件\\
        $\hookrightarrow +$删除论文\\
        统计字数
      \end{block}
    \end{column}
  \end{columns}
\end{frame}

\begin{frame}[label=compile]
  \frametitle{编译问题排查}
  \begin{columns}
    \begin{column}{0.33\textwidth}
      \begin{alertblock}{无法使用 \texttt{latexmk}\thesisissue{578}}
        \hologo{MiKTeX} 需要安装 Perl 解释器。
      \end{alertblock}  
      \begin{alertblock}{C\TeX{} 套装无法编译\thesisissue{446}}
        使用最新 \TeX{} 发行版。
      \end{alertblock}
      \begin{alertblock}{\hologo{pdfLaTeX} 无法编译\thesisissue{444}}
        请使用 \texttt{latexmk},或更改编辑器设置以 \hologo{XeLaTeX} 编译。
      \end{alertblock}
    \end{column}
    \begin{column}{0.33\textwidth}
      \begin{alertblock}{缺少字体\thesisissue{564} \thesisdiscuss{598}}
        更换字体集,或者安装对应字体。
      \end{alertblock}
      \begin{alertblock}{缺少汉字\thesisissue{533} \thesisdiscuss{617}}
        去除使用 fandol 字体集的设定。或者是安装字体后,改用 \texttt{fontset=adobe} 或 \texttt{fontset=founder}。
      \end{alertblock}
    \end{column}
    \begin{column}{0.33\textwidth}
      \begin{block}{\faInfoCircle{} README}
        不同编辑器的设置请首先参阅 README \link{https://github.com/sjtug/SJTUThesis/blob/master/README.md} 文档。
      \end{block}
      \begin{block}{\faBookOpen{} Wiki}
        其他编译问题推荐查阅 Wiki \link{https://github.com/sjtug/SJTUThesis/wiki} 的使用说明部分。
      \end{block}
    \end{column}
  \end{columns}
\end{frame}

\begin{frame}[fragile, label=covers]
  \begin{codeblock}[firstnumber=3]{main.tex}
|\alert{\% 载入 SJTUThesis 模版}|
\documentclass[|\only<1>{\highlight{type}}\only<2>{type}|=|\only<1>{bachelor}\only<2>{\highlight{bachelor}}|]{sjtuthesis}
  \end{codeblock}
  \begin{figure}
    \parbox{0.9\textwidth}{
      \begin{subfigure}{0.20\textwidth}
        \framebox{\includegraphics[width=\linewidth]{support/thesis/bachelor}}
        \caption{\only<1>{学士}\only<2>{\texttt{bachelor}}}
      \end{subfigure}\hfill
      \begin{subfigure}{0.20\textwidth}
        \framebox{\includegraphics[width=\linewidth]{support/thesis/master}}
        \caption{\only<1>{硕士}\only<2>{\texttt{master}}}
      \end{subfigure}\hfill
      \begin{subfigure}{0.20\textwidth}
        \framebox{\includegraphics[width=\linewidth]{support/thesis/doctor}}
        \caption{\only<1>{博士}\only<2>{\texttt{doctor}}}
      \end{subfigure}\hfill
      \begin{subfigure}{0.20\textwidth}
        \framebox{\includegraphics[width=\linewidth]{support/thesis/course}}
        \caption{\only<1>{课程}\only<2>{\texttt{course}}}
      \end{subfigure}
      \caption{论文类型示例\only<2>{ \texttt{type}}}
    }
  \end{figure}
\end{frame}

\begin{frame}[fragile]
  \frametitle{文档类选项}
  % \framesubtitle{\textbackslash{}documentclass\{sjtuthesis\}}
  \begin{columns}
    \begin{column}{0.45\textwidth}
      \includegraphics[page=10]{thesisdir}
    \end{column}
    \begin{column}{0.55\textwidth}
      \begin{table}[H]
        \caption{文档类选项}
        \footnotesize
        \begin{tabular}{>{\ttfamily}rll}
          \toprule
          选项 & 含义 & 相关 \\
          \midrule
          type= & 指定论文类型 & 第 \ref{covers} 页\\
          fontset= & 指定字体 & 第 \ref{compile} 页\\
          \midrule
          review & 开启盲审模式 & \thesisissue{195} \thesisissue{686} \\
          twoside & 双页模式 & \thesisissue{554} \\
          oneside & 单页模式 & \thesisissue{694} \\
          openright & 章从奇数页开始 & \thesisdiscuss{724} \\
          openany & 章从任意页开始 & \thesisissue{446} \\
          \bottomrule
        \end{tabular}
      \end{table}
    \end{column}
  \end{columns}
\end{frame}

\begin{frame}[fragile]
  \frametitle{基本配置}
  \framesubtitle{\textbackslash{}input\{setup\}}
  \begin{columns}
    \begin{column}{0.45\textwidth}
      \includegraphics[page=9]{thesisdir}
    \end{column}
    \begin{column}{0.55\textwidth}
      \begin{codeblock}[firstnumber=12]{main.tex}
|\highlightline<1>|% 论文基本配置,加载宏包等全局配置
|\highlightline<1>|\input{setup}

\begin{document}

%TC:ignore

|\highlightline<2>|% 标题页
|\highlightline<2>|\maketitle
      \end{codeblock}
      \visible<2>{
        \cmd{sjtusetup} 中的 \pkg{info} 将会修改封面的信息设置(见第 \ref{covers} 页)。
      }
    \end{column}
  \end{columns}
\end{frame}

\begin{frame}[fragile]
  \frametitle{基本配置}
  \framesubtitle{\textbackslash{}sjtusetup}
  \begin{columns}
    \begin{column}{0.45\textwidth}
      \includegraphics[page=1]{thesisdir}
    \end{column}
    \begin{column}{0.55\textwidth}
      \begin{codeblock}[firstnumber=3]{setup.tex}
\sjtusetup{
  info = {
    title    = {||上海交通大学学位论文 \LaTeX{} 模板示例文档},
    title*   = {A Sample for \LaTeX-based SJTU Thesis Template},
    author   = {||某\quad{}某},
    author* = {Mo Mo},
  },
  style = { header-logo-color = red, 
  },
  name = {
    publications = {||攻读学位期间完成的论文},
  },
}
      \end{codeblock}
    \end{column}
  \end{columns}
\end{frame}

\begin{frame}
  \frametitle{基本配置}
  \framesubtitle{\textbackslash{}sjtusetup}
  \begin{columns}
    \begin{column}{0.45\textwidth}
      \includegraphics[page=1]{thesisdir}
    \end{column}
    \begin{column}{0.55\textwidth}
      \begin{table}[H]
        \centering
        \caption{info 域}
        \footnotesize
        \begin{tabular}{lll} \toprule
          命令作用 & 中文对应选项 & 英文对应选项 \\ \midrule
          论文标题 & \texttt{title} & \texttt{title*} \\
          关键字列表 & \texttt{keywords} & \texttt{keywords*} \\
          作者姓名&  \texttt{author} &\texttt{author*}\\
          申请学位名称 & \texttt{degree}&\texttt{degree*}\\
          院系名称 & \texttt{department} & \texttt{department*}\\
          专业名称 & \texttt{major} & \texttt{major*}\\
          导师 & \texttt{supervisor} & \texttt{supervisor*}\\
          副导师 & \texttt{assisupervisor} & \texttt{assisupervisor*}\\
          日期 & \multicolumn{2}{c}{\texttt{date}}\\
          学号 & \multicolumn{2}{c}{\texttt{id}}\\ \bottomrule
          \end{tabular}
      \end{table}
    \end{column}
  \end{columns}
\end{frame}

\begin{frame}[fragile]
  \frametitle{版权页}
  \framesubtitle{\textbackslash{}copyrightpage}
  \begin{columns}
    \begin{column}{0.45\textwidth}
      \only<1>{
        \includegraphics[page=9]{thesisdir}
      }
      \only<2>{
        \includegraphics[page=2]{thesisdir}
      }
      \only<3>{
        \begin{figure}[H]
          \framebox{\includegraphics[page=2,width=0.4\linewidth]{bachelor}}
          \caption{版权页}
        \end{figure}
      }
    \end{column}
    \begin{column}{0.55\textwidth}
      \begin{codeblock}[firstnumber=22]{main.tex}
|\highlightline<1>|% 原创性声明及使用授权书
|\highlightline<1>|\copyrightpage
|\highlightline<2>|% 插入外置原创性声明及使用授权书
|\highlightline<2>|% \copyrightpage[scans/sample-copyright-old.pdf]
      \end{codeblock}
      \only<1>{
        \cmd{copyrightpages} 可以用于插入版权页。
      }
      \only<2>{
        \cmd{copyrightpages} 也接受一个可选参数,用于直接使用扫描件。
      }
    \end{column}
  \end{columns}
\end{frame}

\begin{frame}[fragile]
  \frametitle{前置部分}
  \framesubtitle{\textbackslash{}frontmatter}
  \begin{columns}
    \begin{column}{0.45\textwidth}
      \only<1>{
        \includegraphics[page=9]{thesisdir}
      }
      \only<2>{
        \includegraphics[page=3]{thesisdir}
      }
      \only<3>{
        \begin{figure}[H]
          \begin{subfigure}{0.45\textwidth}
            \framebox{\includegraphics[page=3,width=\linewidth]{bachelor}}
            \caption{中文}
          \end{subfigure}\hfill
          \begin{subfigure}{0.45\textwidth}
            \framebox{\includegraphics[page=4,width=\linewidth]{bachelor}}
            \caption{英文}
          \end{subfigure}
          \caption{摘要}
        \end{figure}
      }
      \only<4>{
        \begin{figure}[H]
          \begin{subfigure}{0.30\linewidth}
            \centering
            \framebox{\includegraphics[page=5,width=0.6\linewidth]{bachelor}}
            \caption{目录}
          \end{subfigure}
          \begin{subfigure}{0.30\linewidth}
            \centering
            \framebox{\includegraphics[page=6,width=0.6\linewidth]{bachelor}}
            \caption{插图}
          \end{subfigure}

          \begin{subfigure}{0.30\linewidth}
            \centering
            \framebox{\includegraphics[page=7,width=0.6\linewidth]{bachelor}}
            \caption{表格}
          \end{subfigure}
          \begin{subfigure}{0.30\linewidth}
            \centering
            \framebox{\includegraphics[page=8,width=0.6\linewidth]{bachelor}}
            \caption{算法}
          \end{subfigure}
          \caption{索引}
        \end{figure}
      }
      \only<5>{
        \includegraphics[page=4]{thesisdir}
      }
      \only<6>{
        \begin{figure}[H]
          \framebox{\includegraphics[page=9,width=0.5\linewidth]{bachelor}}
          \caption{符号对照表}
        \end{figure}
      }
    \end{column}
    \begin{column}{0.55\textwidth}
      \begin{codeblock}[firstnumber=30]{main.tex}
|\highlightline<2-3>|% 摘要
|\highlightline<2-3>|\input{contents/abstract}

|\highlightline<4>|% 目录
|\highlightline<4>|\tableofcontents
|\highlightline<4>|% 插图索引
|\highlightline<4>|\listoffigures*
|\highlightline<4>|% 表格索引
|\highlightline<4>|\listoftables*
|\highlightline<4>|% 算法索引
|\highlightline<4>|\listofalgorithms*

|\highlightline<5-6>|% 符号对照表
|\highlightline<5-6>|\input{contents/nomenclature}
      \end{codeblock}
    \end{column}
  \end{columns}
\end{frame}

\begin{frame}[fragile]
  \frametitle{主体部分}
  \framesubtitle{\textbackslash{}mainmatter}
  \begin{columns}
    \begin{column}{0.45\textwidth}
      \only<1>{
        \includegraphics[page=5]{thesisdir}
      }
      \only<2>{
        \begin{figure}[H]
          \begin{subfigure}{0.30\linewidth}
            \centering
            \framebox{\includegraphics[page=11,width=0.6\linewidth]{bachelor}}
            \caption{简介}
          \end{subfigure}
          \begin{subfigure}{0.30\linewidth}
            \centering
            \framebox{\includegraphics[page=13,width=0.6\linewidth]{bachelor}}
            \caption{数学}
          \end{subfigure}

          \begin{subfigure}{0.30\linewidth}
            \centering
            \framebox{\includegraphics[page=16,width=0.6\linewidth]{bachelor}}
            \caption{浮动体}
          \end{subfigure}
          \begin{subfigure}{0.30\linewidth}
            \centering
            \framebox{\includegraphics[page=22,width=0.6\linewidth]{bachelor}}
            \caption{总结}
          \end{subfigure}
          \caption{主体部分}
        \end{figure}
      }
    \end{column}
    \begin{column}{0.55\textwidth}
      \begin{codeblock}[firstnumber=47]{main.tex}
|\highlightline|% 正文内容
|\highlightline|% !TeX root = ../../../latex-talk.tex

\section{是什么}

\begin{frame}
  \frametitle{\TeX{}}
  \begin{columns}[c]
    \begin{column}{0.7\textwidth}
      \begin{center}
        \rmfamily\Huge
        \highlight[structure]{\TeX{}}
      \end{center}
      \begin{center}
        \parbox{0.75\textwidth}{
          \TeX{} 是由斯坦福大学教授高德纳
          (Donald E.~Knuth)于 1977 年开始开发的排版引擎。目前仍在更新,最新版本号为 3.141592653 \link{https://tug.org/TUGboat/tb42-1/tb130knuth-tuneup21.pdf}。
        }
      \end{center}
    \end{column}
    \begin{column}{0.3\textwidth}
      \includegraphics[width=.8\columnwidth]{support/images/Knuth.jpg}
    \end{column}
  \end{columns}
  \note{\emph{这一部分背景介绍大家可以了解一下,暂时跳过。}
  \LaTeX{} 这个词由两个部分组成,\hologo{La} 和 \TeX{}。那我们首先了解一下 \TeX{} 是什么。
  \TeX{} 是由斯坦福大学的教授高德纳于 1977 年开始开发的排版引擎,它已经有三十多年的历史了,
  目前仍在更新,版本号(3.141592653)将会趋近于 $\pi$ 的取值,高德纳最近还在给 \textsl{TUGBoat} 写稿子
  \link{https://tug.org/TUGboat/tb42-1/tb130knuth-tuneup21.pdf},
  关于 \TeX{} 今年又做了哪些改进。}
\end{frame}

\begin{frame}
  \frametitle{\LaTeX{}}
  \begin{columns}[c]
    \begin{column}{0.7\textwidth}
      \begin{center}
        \rmfamily\Huge
        \highlight[structure]{\LaTeX{}}
      \end{center}
      \begin{center}
        \parbox{0.75\textwidth}{
          \LaTeX{} 是最早在 1985 年由现就职于微软的 Leslie Lamport 开发的一种 \TeX{} \textbf{格式}\footnotemark,使用一些列宏和扩展宏包来简化 \TeX{} 的使用。现在由 \LaTeX{} Project 的成员维护。现在广泛使用的版本是 \LaTeXe{},最新的版本为 \LaTeX3(2020 年 10 月后默认内置)。
        }
      \end{center}
    \end{column}
    \begin{column}{0.3\textwidth}
      \includegraphics[width=.8\columnwidth]{support/images/Lamport.jpg}
    \end{column}
  \end{columns}
  \footnotetext{\hologo{ConTeXt} 也是一种 \TeX{} 格式 \link{https://www.contextgarden.net/}。}
  \note{\emph{这一部分的背景介绍大家可以了解一下,暂时跳过。}
  \LaTeX{} 是最早由现就职于微软的 Leslie Lamport 开发的一种 \TeX{} 格式(与其对标的是
  \hologo{ConTeXt}\link{https://www.contextgarden.net/}),主要也是为了简化 \TeX{} 的使用。
  现在主要由 \LaTeX{} 开发组维护,现在广泛使用的版本是 \LaTeXe{},最新的版本为 \LaTeX3,
  在 2020 年 10 月后默认内置,所以要尽可能使用较新的发行版,以充分发挥其功能。}
\end{frame}

\begin{frame}
  \frametitle{程序}
  \begin{columns}[c]
    \begin{column}{0.7\textwidth}
      \begin{center}
        \rmfamily\Huge
        \highlight[structure]{\hologo{pdfLaTeX}}
      \end{center}
      \begin{center}
        \parbox{0.7\textwidth}{
          \hologo{pdfLaTeX} 是为了编译一个 \LaTeX{} 文档而运行的程序。实际上底层在运行一个叫 \hologo{pdfTeX} 的引擎,并预装了对应的 \LaTeX{} \textbf{格式}。为了利用临时文件,可能就需要多次运行程序。
        }
      \end{center}
    \end{column}
    \begin{column}{0.3\textwidth}
      \begin{block}{}
        \ttfamily\small
        > \highlight{pdflatex} main.tex\\
        This is pdfTeX, Version 3.141592653-
        2.6-1.40.23 (MiKTeX 21.10)\\
        entering extended mode\\
        \highlight{LaTeX2e} <2021-11-15>\\
        \highlight{L3} programming layer <2021-11-22>
      \end{block}
    \end{column}
  \end{columns}
  \note{\hologo{pdfLaTeX} 是为了编译一个 \LaTeX{} 文档而运行的程序。}
\end{frame}

% \begin{frame}
%   \frametitle{引擎}
%   \begin{columns}[c]
%     \begin{column}{0.7\textwidth}
%       \begin{center}
%         \rmfamily\Huge
%         \highlight[structure!70]{pdf}\hologo{La}\highlight[structure!70]{\TeX{}}
%       \end{center}
%       \begin{center}
%         \parbox{0.7\textwidth}{
%           pdf\TeX{} 是编译 \TeX{} 文档(以 \texttt{.tex} 结尾)的\textbf{引擎}---可以理解 \TeX{} 指令的\textbf{程序}。
%         }
%       \end{center}
%     \end{column}
%     \begin{column}{0.3\textwidth}
%       \begin{block}{}
%         \ttfamily\small
%         > pdflatex main.tex\\
%         This is \highlight[structure!70]{pdfTeX}, Version 3.141592653-
%         2.6-1.40.23 (MiKTeX 21.10)
%         entering extended mode\\
%         LaTeX2e <2021-11-15>\\
%         L3 programming layer <2021-11-22>
%       \end{block}
%     \end{column}
%   \end{columns}
%   \note{实际上底层在运行一个叫 \hologo{pdfTeX} 的引擎,并预装了对应的 \LaTeX{} 格式。}
% \end{frame}

\begin{frame}[label={frame:engine}]
  \frametitle{程序}
  \begin{table}
    \caption{主流 \hologo{(La)TeX} 程序
    \footnote{(u)p\TeX{} 是日语最常用的引擎,生成 \texttt{.dvi},支持 Unicode。}\footnote{Ap\TeX{} \link{https://github.com/clerkma/ptex-ng} 具有底层 CJK 支持,内联 Ruby,Color Emoji。}}
    \footnotesize
    \begin{stampbox}
      \begin{tabular}{c>{\raggedright}*{3}{p{3.5cm}}}
        \alert{引擎}     & \hologo{pdfTeX}   & \hologo{XeTeX}   & \hologo{LuaTeX}   \\
        \alert{程序}     & \hologo{pdfLaTeX} & \hologo{XeLaTeX} & \hologo{LuaLaTeX} \\
        \alert{特点}     & 直接生成 PDF,支持 micro-typography  & 支持 Unicode、OpenType 与复杂文字编排 (CTL) & 支持 Unicode,内联 Lua,支持 OpenType \\
      \end{tabular}
    \end{stampbox}
  \end{table}

  \begin{center}
    \parbox{.9\textwidth}{
      \hologo{pdfLaTeX} 不支持 Unicode。为了排版中文,大部分情况下应当使用 \hologo{XeLaTeX},而 \hologo{LuaLaTeX} 速度相对较慢。\faWindows{} 可以在一些情况下使用 \hologo{pdfLaTeX}。
    }
  \end{center}
  \note{当然为了排版中文,已经不再推荐使用 \hologo{pdfLaTeX} 了,应该使用
  \hologo{XeLaTeX} 或者 \hologo{LuaLaTeX},当然后者的速度还是相对较慢,
  它们支持 Unicode 编码,并可以使用 OpenType 字体的全部功能。
  当然 \faWindows{} 平台下在某些追求速度的情况下,
  还是可以试着使用 \hologo{pdfLaTeX} 的。

  \hologo{LuaLaTeX} 理想情况下不慢,但是使用一些宏包后会破坏理想状态,
  也会因配置产生不同的结果,不同的操作系统在 I/O 速度上的不同也会导致不同的时间。

  \hologo{pdfLaTeX} 也支持,只不过需要先生成 tfm \TeX{} 字体度量文件,后续使用 \TeX{}
  自身的配置方法,只能使用 7 比特或 8 比特字体。}
\end{frame}

% \begin{frame}
%   \paragraph{\hologo{pdfLaTeX}} \TeX{} 和 \LaTeX{} 被广泛使用之前,它们只需内置支持欧洲语言即可。在 Unicode 出现之前,\LaTeX{} 提供了许多种\textbf{文件编码}来允许很多语言的文字以原生的方式输入,\hologo{pdfLaTeX} 也只需要使用 8 位文件编码和 8 位字体。
% \end{frame}


|\highlightline|\input{contents/math_and_citations}
|\highlightline|\input{contents/floats}
|\highlightline|\input{contents/summary}

%TC:ignore

% 参考文献
\printbibliography[heading=bibintoc]
      \end{codeblock}
    \end{column}
  \end{columns}
\end{frame}

\begin{frame}
  \frametitle{数学}
  \begin{itemize}
    \item 公式示例:\nolinkurl{contents/math_and_citations.tex}
    \item \SJTUThesis{} 定义了常用的数学环境(需要手工引入 \texttt{ntheorem} 宏包):
      \begin{table}[h]
        \centering
        \footnotesize
        \begin{tabular}{*{7}{l}}\toprule
          assumption  & axiom   & conjecture & corollary    & definition  & example   & exercise  \\
          假设        & 公理    & 猜想       & 推论         & 定义        & 例        & 练习      \\\midrule
          lemma       & problem & proof      & proposition  & remark      & solution  & theorem   \\
          引理        & 问题    & 证明       & 命题         & 注          & 解        & 定理      \\\bottomrule
        \end{tabular}
      \end{table}
      \item \SJTUThesis{} 可以通过 \texttt{unimath} 选项使用 \pkg{unicode-math} 进行数学输入,注意与传统方式的区别。\thesisissue{555}
  \end{itemize}
\end{frame}

\begin{frame}[fragile]
  \frametitle{参考文献}
  \begin{columns}
    \begin{column}{0.45\textwidth}
      \includegraphics[page=6]{thesisdir}
    \end{column}
    \begin{column}{0.55\textwidth}
      \begin{codeblock}[firstnumber=111,numbersep=2pt]{setup.tex}
% 使用 BibLaTeX 处理参考文献
%   biblatex-gb7714-2015 常用选项
%     gbnamefmt=lowercase     姓名大小写由输入信息确定
%     gbpub=false             禁用出版信息缺失处理
\usepackage[backend=biber,style=gb7714-2015]{biblatex}
% 文献表字体
% \renewcommand{\bibfont}{\zihao{-5}}
% 文献表条目间的间距
\setlength{\bibitemsep}{0pt}
|\highlightline|% 导入参考文献数据库
|\highlightline|\addbibresource{bibdata/thesis.bib}
      \end{codeblock}
    \end{column}
  \end{columns}
\end{frame}

\begin{frame}[fragile]
  \frametitle{附录}
  \framesubtitle{\textbackslash{}appendix}
  \begin{columns}
    \begin{column}{0.45\textwidth}
      \only<1>{
        \includegraphics[page=7]{thesisdir}
      }
      \only<2>{
        \begin{figure}[H]
          \begin{subfigure}{0.45\linewidth}
            \framebox{\includegraphics[width=\linewidth,page=24]{bachelor}}
            \caption{}
          \end{subfigure}\hfill
          \begin{subfigure}{0.45\textwidth}
            \framebox{\includegraphics[width=\linewidth,page=25]{bachelor}}
            \caption{}
          \end{subfigure}
          \caption{附录}
        \end{figure}
      }
    \end{column}
    \begin{column}{0.55\textwidth}
      \begin{codeblock}[firstnumber=61]{main.tex}
% 附录中图表不加入索引
\captionsetup{list=no}

% 附录内容
|\highlightline|\input{contents/app_maxwell_equations}
|\highlightline|\input{contents/app_flow_chart}
      \end{codeblock}
    \end{column}
  \end{columns}
\end{frame}

\begin{frame}[fragile]
  \frametitle{结尾部分}
  \framesubtitle{\textbackslash{}backmatter}
  \begin{columns}
    \begin{column}{0.45\textwidth}
      \only<1>{
        \includegraphics[page=8]{thesisdir}
      }
      \only<2>{
        \begin{figure}[H]
          \begin{subfigure}{0.30\linewidth}
            \centering
            \framebox{\includegraphics[page=26,width=0.6\linewidth]{bachelor}}
            \caption{致谢}
          \end{subfigure}
          \begin{subfigure}{0.30\linewidth}
            \centering
            \framebox{\includegraphics[page=27,width=0.6\linewidth]{bachelor}}
            \caption{成就}
          \end{subfigure}

          \begin{subfigure}{0.30\linewidth}
            \centering
            \framebox{\includegraphics[page=28,width=0.6\linewidth]{bachelor}}
            \caption{简历}
          \end{subfigure}
          \begin{subfigure}{0.30\linewidth}
            \centering
            \framebox{\includegraphics[page=29,width=0.6\linewidth]{bachelor}}
            \caption{大摘要*}
          \end{subfigure}
          \caption{结尾部分}
        \end{figure}
      }
    \end{column}
    \begin{column}{0.55\textwidth}
      \begin{codeblock}[firstnumber=76]{main.tex}
% 致谢
\input{contents/acknowledgements}

% 发表论文及科研成果
% 盲审论文中,发表论文及科研成果等仅以第几作者注明即可,不要出现作者或他人姓名
\input{contents/achievements}

% 简历
\input{contents/resume}

% 学士学位论文要求在最后有一个大摘要,单独编页码
\input{contents/digest}
      \end{codeblock}
    \end{column}
  \end{columns}
\end{frame}

\begin{frame}
  \frametitle{还有其他问题?}
  \begin{columns}
    \begin{column}{0.75\textwidth}
    \begin{itemize}
      \item[{\faComment*[regular]}] 日常模板或 \LaTeX{} 使用问题可以前往 Discussions \link{https://github.com/sjtug/SJTUThesis/discussions} 提问
      
      (解决后别忘了 \textcolor{green}{\faCheckCircle{} Mark as answer}
      \item[{\faDotCircle[regular]}] 如果是 \textsc{SJTUThesis} 项目本身的 bug 和 feature request
      
      可以通过 Issues \link{https://github.com/sjtug/SJTUThesis/issues} 反馈。
      \item[{\faCodeBranch}] 如果你有好点子,可以贡献代码
     
      向 \textsc{SJTU\TeX{}}(v1) \link{https://github.com/sjtug/SJTUTeX/tree/v1} 存储库发 PR,\par
      而后把解包结果同步到 \textsc{SJTUThesis}。
  
      \item[{\faTag}] 如果你对正在基于 \LaTeX3 开发的新版感兴趣,\par
      也欢迎向 \textsc{SJTU\TeX{}}(v2) \link{https://github.com/sjtug/SJTUTeX/tree/v2} 发 PR。
  
      \item[{\faQq}] 也欢迎在 QQ 群即时讨论。
    \end{itemize}
    \end{column}
    \begin{column}{0.25\textwidth}
      \includegraphics[height=0.7\textheight]{qq.jpg}
    \end{column}
  \end{columns}
\end{frame}
% !TeX root = ../../latex-talk.tex

\part{更进一步}

\begin{frame}
  \frametitle{其他 \LaTeX\ 中文入门资料}
  \begin{bibliolist}{00}
    \onlineitem \textsc{Reckdahl~K.}
    王磊~译.
    \newblock \LaTeX{} 插图指南[EB/OL].
    \newblock 第三版. 2017. url{https://github.com/WenboSheng/epslatex-cn}.

    \onlineitem 包太雷.
    \newblock \LaTeX\ Notes[EB/OL].
    \newblock 第二版. 2021. \url{https://github.com/huangxg/lnotes}.

    \bookitem 刘海洋.
    \newblock \LaTeX\ 入门[M]. 
    \newblock 北京:电子工业出版社, 2013.
  \end{bibliolist}
\end{frame}

\begin{frame}
  \frametitle{其他 \LaTeX\ 经典资料}
  \begin{bibliolist}{00}
    \onlineitem \textsc{Overleaf}.
    \newblock Learn \LaTeX\ in 30 minutes[EB/OL]. 
    \newblock \url{https://www.overleaf.com/learn/latex/Learn_LaTeX_in_30_minutes}.

    \bookitem \textsc{M.R.C. van Dongen}.
    \newblock \LaTeX{} and Friends[M/OL].
    \newblock New York: Springer, 2012. 
    \url{https://link.springer.com/book/10.1007/978-3-642-23816-1}

    \bookitem \textsc{Mittelbach F.}, \textsc{Goossens M.}, \textsc{Braams J.}, et al.
    \newblock The \LaTeX\ Companion[M].
    \newblock 2nd ed. Boston: Addison-Wesley Professional. 2004.
  \end{bibliolist}
\end{frame}

\begin{frame}
  \frametitle{仍有 \LaTeX\ 的问题?}
  \begin{itemize}
    \item 首先可以查阅宏包的附带文档。在命令行输入
    
    \hfill
    \begin{minipage}{0.3\textwidth}
      \begin{exampleblock}{\faTerminal}
        \ttfamily
        texdoc \textit{<pkg>}
      \end{exampleblock}
    \end{minipage}
    \item 其次可以考虑向 \TeX{} StackExchange \link{https://tex.stackexchange.com/} 寻找解答或提出问题。
    \item 中文相关的问题可以在 \CTeX{} 临时论坛 \link{https://github.com/CTeX-org/forum} 搜寻。
    \item 提问时请附上最小工作案例(MWE)与日志文件。
  \end{itemize}
\end{frame}

\begin{frame}
  \frametitle{敬请期待}
  
  5 月 18 日还会有一讲来讨论 \LaTeX\ 更加注重视觉效果的相关知识。

  \begin{columns}
    \begin{column}{0.5\textwidth}
      \begin{figure}
        \begin{stampbox}[sjtuRedPrimary]
          \includegraphics[height=0.4\textheight]{hello}
        \end{stampbox}
        \caption{\textsc{SJTUBeamer} \link{https://github.com/sjtug/SJTUBeamer}}
      \end{figure}
    \end{column}
    \begin{column}{0.5\textwidth}
      \begin{figure}
        \begin{stampbox}[sjtuRedPrimary]
          \includegraphics[height=0.4\textheight]{latexsparkle}
        \end{stampbox}
        \caption{像模像样 \LaTeX\ \link{https://logcreative.tech/LaTeXSparkle}}
      \end{figure}
    \end{column}
  \end{columns}
\end{frame}

\def\bottomthanks{Happy \TeX{}ing!}
\makebottom
% !TeX root = ../latex-talk.tex

\title{如何使用 beamer 制作学术报告幻灯片}
\subtitle{上海交通大学图书馆专题培训讲座}
\author{李子龙}
\institute[SJTU Linux User Group]{上海交通大学 Linux 用户组}
\date{2022 年 5 月}
\subject{LaTeX, 幻灯片制作, SJTUBeamer}
\maketitle

\providecommand{\TikZ}{Ti\textit{k}Z}
\providecommand{\pgf}{\textsc{pgf}}
\providecommand{\pgfplots}{\textsc{pgfplots}}
\providecommand{\pgfplotstable}{\textsc{PgfplotsTable}}

\begin{frame}{关于}
  \begin{columns}[c]
  \begin{column}{.7\textwidth}
    \begin{block}{如何使用 beamer 制作学术报告幻灯片}
    \alert{\url{https://github.com/sjtug/sjtulib-latex-talk}}
    
    \begin{flushleft}
      \small 介绍 \LaTeX{} 幻灯片模板 \textsc{SJTUBeamer} 的使用方法,以及 \LaTeX{} 图形绘制的简明方法,帮助学生制作出像模像样的学术幻灯片。
    \end{flushleft}

    \begin{tabular*}{0.8\linewidth}{@{\extracolsep{\fill}}lll@{}}
      \scriptsize 最后更新 & \scriptsize 幻灯片下载 & \scriptsize 许可证 \\
      \today & Overleaf \link{https://www.overleaf.com/read/fvwxzvcxhcwd} & \href{https://creativecommons.org/licenses/by-sa/4.0/}{\faCreativeCommons\,\faCreativeCommonsBy\,\faCreativeCommonsSa} \\ 
    \end{tabular*}
    \end{block}
    \vspace{0.2cm}
  \end{column}
  \begin{column}{.3\textwidth}
    \qrcode[hyperlink, height=4cm]{https://www.overleaf.com/read/fvwxzvcxhcwd}
  \end{column}
  \end{columns}
\end{frame}

\begin{frame}
  \frametitle{讲稿主要参考}
  \begin{bibliolist}{00}
    \onlineitem 吕荐瑞.
    \newblock \LaTeX{} 文档排版教程[EB/OL].
    \newblock 2018. \url{https://lvjr.bitbucket.io/tutorial/learn-latex.pdf}
    
    \onlineitem 朱俸民.
    \newblock \TikZ{} 画“图”指南[EB/OL].
    \newblock 2020. \url{https://tuna.moe/event/2020/tikz/}

    \onlineitem \textsc{Hansimov}. 
    \newblock 如何加速 \LaTeX{} 编译[EB/OL].
    \newblock 2019. \url{https://zhuanlan.zhihu.com/p/55043560}
  \end{bibliolist}
\end{frame}

% !TeX root = ../../latex-talk.tex

\part{SJTUBeamer}

\begin{frame}
  \frametitle{简介}
  \begin{columns}
    \begin{column}{0.6\textwidth}
      \begin{itemize}
        \item 最早由谌翔于 2021 年 4 月发布
        \item 2021 年 5 月起由 SJTUG 接手,发布 1.0 版
        \item 2021 年 9 月李子龙重构了整个宏包的代码,升级版本号为 2.0
        \item 最新版:\SJTUBeamerVersion{} (\SJTUBeamerDate)
        \item 支持三种基本样式与扩展样式的幻灯片模板
        \item 支持镜像站克隆 \link{https://mirror.sjtu.edu.cn/git/SJTUBeamer.git/},支持使用任一引擎编译
      \end{itemize}
    \end{column}
    \begin{column}{0.4\textwidth}
      \begin{exampleblock}{}
        \begin{minipage}[c]{1cm}
          \includegraphics[width=0.8cm]{support/examples/pics/sjtug.pdf}
        \end{minipage}
        \begin{minipage}[c]{4cm}
          \href{https://github.com/sjtug}{sjtug}/\href{https://github.com/sjtug/SJTUBeamer}{SJTUBeamer}
        \end{minipage}
      \end{exampleblock}
      \vspace{-8pt}
      \begin{block}{}
        \scriptsize
        上海交通大学 Beamer 模版 | Beamer template for Shanghai Jiao Tong University
      \end{block}
      \vspace{-8pt}
      \begin{alertblock}{}
        \scriptsize
        \begin{tabular}{cl}
          \faStar       & 357 \\
          \faEye        & 7   \\
          \faCodeBranch & 29  \\
        \end{tabular}
      \end{alertblock}
    \end{column}
  \end{columns}
\end{frame}

\begin{frame}
  \frametitle{\only<1>{为什么使用 beamer 制作幻灯片?}\only<2>{当然也有新兴替代物}}
  \begin{columns}[t]

    \only<2>{
      \begin{column}{0.25\textwidth}
        \begin{block}{\faMarkdown{} 代码翻译}
          \begin{itemize}
            \item[\faVuejs{}] Slidev \link{https://cn.sli.dev/}
            \item[\faJsSquare{}] Marp \link{https://marp.app/}
            \item[\faJs{}] AutoBeamer \link{https://github.com/LogCreative/AutoBeamer}
          \end{itemize}
        \end{block}
        \note[item]<2>{\textbf{Markdown} 当然也有新兴替代物,比如可以从 Markdown 代码转换为幻灯片的一些软件。基于 Vue 的 Slidev 可以快速地输出灵活布局的幻灯片,基于 TypeScript 的 Marp 可以作为 VS Code 的插件来实时预览 Markdown 幻灯片,还有我制作的转换 Markdown 文档为 beamer 代码的 AutoBeamer 也欢迎大家的试用。}
      \end{column}
    }

    \begin{column}{0.25\textwidth}
      \begin{block}{\LaTeX{} SJTUBeamer}
        \begin{itemize}
          \item[\faPlus] \LaTeX{} 原生
          \item[\faPlus] 规范的格式
          \item[\faMinus] 美术功能麻烦
        \end{itemize}
      \end{block}
      \note[item]<1>{\textbf{Beamer} 是 \LaTeX{} 原生的幻灯片类,可以输入专业的公式,其编程式的方法会促使用户规整幻灯片的布局,更为容易地产生正式而不失优雅的幻灯片。我们使用 \LaTeX{} 主要也是为了符合规范,对于很多不擅长排版的小伙伴有个这样的工具自动美化自己的作品是一件非常方便的事情。

        当然缺点也是存在的,就是美术功能的实现较为复杂,以及一些多媒体的支持可能需要使用链接的形式。}
    \end{column}

    \only<2>{
      \begin{column}{0.25\textwidth}
        \begin{exampleblock}{\faApple{} Keynote}
          \begin{itemize}
            \item[\faPlus] 友好的界面
            \item[\faPlus] 精致的动画
            \item[\faMinus] 缺少进阶功能
          \end{itemize}
        \end{exampleblock}
        \note[item]<2>{\textbf{Keynote} 是苹果独占的幻灯片软件,相比于 PowerPoint 有了更加用户友好的界面,辅以一些精致的动画,可以较快地做出“果味”幻灯片。

          当然简单也会带来进阶功能的缺失,缺乏插件体系的支持,能做出的东西相比于 PowerPoint 少了不少。}
      \end{column}
    }

    \begin{column}{0.25\textwidth}
      \begin{exampleblock}{\faFilePowerpoint{} PowerPoint}
        \begin{itemize}
          \item[\faPlus] 灵活的插件
          \item[\faPlus] 强大的动画
          \item[\faMinus] 布局不够规整
        \end{itemize}
      \end{exampleblock}
      \note[item]<1>{\textbf{PowerPoint} 我们在前半部分的讲座中看到了 PowerPoint 制作幻灯片的强大之处,可见即可得,辅以强大的动画功能,并可以使用灵活的插件提高效率。

        当然灵活性也会让用户在布局上更加随意一些,并在这样一些细节上让整个演讲稿看起来慢慢地变得没有那么正式。当然你也可以只使用大纲视图编写幻灯片,来规整自己的布局。}
    \end{column}
  \end{columns}
\end{frame}

\begin{frame}
  \frametitle{浅学手册}

  \only<1>{
    \begin{columns}
      \begin{column}{0.7\textwidth}
        前往 SJTUBeamer 的发布页 \link{https://github.com/sjtug/SJTUBeamer/releases} 可以得到一个干净的 \pkg{sjtubeamer-ctan.zip},用于将该 SJTUBeamer 安装到本地目录。以及一些其他文档的下载。当然你也可以在 SJTUG 镜像站 \link{https://mirror.sjtu.edu.cn/} 上下载 SJTUBeamer 的这些 Assets。

        \begin{exampleblock}{\faGit* 从镜像站克隆最新版}
          \ttfamily\footnotesize
          git clone https://mirror.sjtu.edu.cn/git/SJTUBeamer.git/
        \end{exampleblock}
      \end{column}
      \begin{column}{0.3\textwidth}
        \fzerobox{\includegraphics[width=\linewidth]{support/images/mirrorclone.jpg}}
      \end{column}
    \end{columns}
    \note[item]{GitHub 上克隆的是开发版(社区版),Release 得到的是稳定版,虽然开发版比稳定版稳定,但是稳定版在镜像站的加持下下载会更快一些。}
    \note[item]{进入镜像站页面后,点击 github/SJTUBeamer(点击 git/SJTUBeamer.git 会有 BUG)。}
  }

  \only<2>{
    \begin{columns}
      \begin{column}{0.7\textwidth}
        《SJTUBeamer 使用手册》\link{https://github.com/sjtug/SJTUBeamer/releases/download/v\SJTUBeamerVersion{}/sjtubeamer.pdf} 给出了大量的示例,用于快速上手 SJTUBeamer 的基本使用方法。推荐大家从一个干净的文档一步步码出来这些例子,或者遇到问题先按手册查询。本讲座仅做概览。

        \note{这些例子含有 beamer 文档类的基础用法,一步步地完成可以较快地学会这些知识。}

        \begin{exampleblock}{\faTerminal{} 从源码编译文档}
          \ttfamily\footnotesize
          cd src\\
          l3build doc
        \end{exampleblock}
      \end{column}
      \begin{column}{0.3\textwidth}
        \fzerobox{\includegraphics[width=\linewidth]{support/images/sjtubeamerfirst.pdf}}
      \end{column}
    \end{columns}
  }

  \only<3>{
    \begin{columns}
      \begin{column}{0.7\textwidth}
        \textit{Development Guide of SJTUBeamer} \link{https://github.com/sjtug/SJTUBeamer/releases/download/v\SJTUBeamerVersion{}/sjtubeamerdevguide.pdf} 对于想要了解 SJTUBeamer 框架与原理的人会比较有用,感兴趣的可以通过阅读该手册尝试对代码更改提 PR 或贡献插件 \link{https://github.com/sjtug/SJTUBeamer/issues/81}。本讲座不会涉及这一部分。
        \begin{exampleblock}{\faTerminal{} 代码同步}
          \ttfamily\footnotesize
          cd src\\
          \textcolor{gray}{\# 将 Doc\TeX{} 解包为 sty 文件并移至根目录} \\
          l3build check
        \end{exampleblock}
      \end{column}
      \begin{column}{0.3\textwidth}
        \fzerobox{\includegraphics[width=\linewidth]{support/images/sjtubeamerdevguidefirst.pdf}}
      \end{column}
    \end{columns}
    \note{用英文写开发手册实际上是因为国内做 beamer 的比较少。sad。}
  }

\end{frame}

\begin{frame}[fragile]
  \frametitle{文档类设置}

  \only<1>{
    为文档类设置 \opt{aspectratio} 选项可以更改幻灯片的长宽比。其他通用选项详见 \textit{The Beamer Class -- User Guide} \link{https://mirrors.sjtug.sjtu.edu.cn/CTAN/macros/latex/contrib/beamer/doc/beameruserguide.pdf}。帧环境 \env{frame} 详见第 \ref{frame} 页。
    \note{有些教室可能还在用 4:3 比例(不调整默认是 4:3 比例),在 beamer 里调整比例要比 PowerPoint 里要方便的多,因为内容与格式分离了,不会导致比例失调的情况。}
  }

  \only<2>{
    SJTUBeamer 依赖于 \cls{ctexbeamer} 文档类,相关字体设置详见《\CTeX{} 宏集手册》\link{https://mirrors.sjtug.sjtu.edu.cn/CTAN/language/chinese/ctex/ctex.pdf}。如果编写英文幻灯片,请直接使用 \cls{beamer} 文档类。
    \note{\opt{fontset} 可以被指定为诸如 \opt{ubuntu}, \opt{adobe}, \opt{founder} 等,注意 SJTUThesis 中只能使用 \opt{cjk-font} 调整,暂时没有为 SJTUBeamer 设定单独字体样式的想法,毕竟幻灯片跟论文还是不太相同的。当然如果以后有什么“交大字体”啦,我们可以考虑适配一下。}
  }

  \begin{columns}
    \begin{column}{0.62\textwidth}
      \begin{codeblock}[]{\only<1>{更改长宽比}\only<2>{更改字体}}
|\highlightline|\documentclass[|\only<1>{aspectratio=169}\only<2>{fontset=fandol}|]{ctexbeamer}
\usetheme{sjtubeamer}
\begin{document}
  \begin{frame}
    \frametitle{|\phantom{}|欢迎}
    |\phantom{}|你好,SJTUBeamer|\phantom{}|!
|\phantom{}|  \end{frame}
|\phantom{}|\end{document}
      \end{codeblock}
    \end{column}
    \begin{column}{0.38\textwidth}
      \begin{center}
        \includebeamerlarge{support/beamer/welcome.pdf}
      \end{center}
    \end{column}
  \end{columns}
\end{frame}

\begin{frame}[fragile]
  \frametitle{切换设计}
  \begin{columns}
    \begin{column}{0.6\textwidth}

      \only<1>{
        为了使用 SJTUBeamer,需要在根目录中创建一个文件,并添加 \cmd{usetheme\{sjtubeamer\}}。你可以添加用逗号分隔的选项切换不同的设计样式。
        \note{使用 \opt{maxplus}, \opt{max}, \opt{min} 切换不同的封面样式(包含节次页)。使用 \opt{red}, \opt{blue} 来切换主色调。使用 \opt{dark} 和 \opt{light} 来切换不同的亮度主题。}
      }

      \only<2>{
        还可以适配 \cls{beamer} 自带的外样式,可以通过 \opt{topright} 或 \opt{bottomright} 选项切换徽标位置。
        \note{
          使用外样式关键字切换页面布局,使用 \opt{topright} 和 \opt{bottomright} 切换徽标的位置。
        
          sidebar 外样式略有一些 BUG,就是封面页会有文字偏移,感兴趣地可以考虑修正一下,暂时没有找到太好的通用方法,这种情况下可以考虑不使用标题页或只使用 maxplus 版本。冷知识,10 年前也有学长做了个 \href{https://github.com/X-Wei/aBeamerTemplate4SJTU}{SJTU Beamer 模板}发布于水源 BBS,使用的就是 sidebar 外样式。
          
          别小看这两页,SJTUBeamer 很多代码都放在这两页上了,这两页已经可以组合出 $3\times 2\times 2\times 9\times 2=216$ 种可能了(忽略徽标位置的话正好 108 种)。每次展示都用同一种样式显得比较单调,适当的时候可以改改选项试试这些样式。}
      }

      \begin{codeblock}[]{}
\documentclass{ctexbeamer}
|\highlightline|\usetheme[|\only<1>{maxplus,blue,light}\only<2>{default,topright}|]{sjtubeamer}
\begin{document}
  \title{|\phantom{}|使用 Beamer 制作学术报告幻灯片}
  \subtitle{|\phantom{}|上海交通大学图书馆专题培训讲座}
  \author{|\phantom{}|李子龙}
  \institute[SJTUG]{SJTU Linux User Group}
  \date{2022 年 5 月}
|\highlightline<1>|  \maketitle
|\highlightline<1>|  \part{|\phantom{}|介绍}
|\highlightline<1>|  \makebottom
|\phantom{}|\end{document}
      \end{codeblock}
    \end{column}
    \begin{column}{0.4\textwidth}
      \only<1>{
        \begin{figure}
          \includegraphics[width=\linewidth]{support/images/sjtubeamercover.jpg}
          \caption{封面样式}
        \end{figure}
      }
      \only<2>{
        \begin{figure}
          \includegraphics[width=\linewidth]{support/images/sjtubeamerouter.jpg}
          \caption{外样式}
        \end{figure}
      }
    \end{column}
  \end{columns}
\end{frame}

\begin{frame}[fragile]
  \frametitle{目录}
  \cmd{tableofcontents} 命令在 \cls{beamer} 中可以添加参数以更改要展示的节次标题。\opt{hideallsubsections} 参数用于不排印小节标题。
  \begin{columns}
    \begin{column}{0.6\textwidth}
      
      \begin{codeblock}[]{目录页}
|\phantom{}|\documentclass{ctexbeamer}
\usetheme{sjtubeamer}
\begin{document}
  \begin{frame}
    \frametitle{|\phantom{}|目录}
|\highlightline|    \tableofcontents[hideallsubsections]
  |\phantom{}|\end{frame}
|\phantom{}|\end{document}
      \end{codeblock}
      
    \end{column}
    \begin{column}{0.4\textwidth}
      \centering
      \includebeamerlarge[1]{support/beamer/toc.pdf}
    \end{column}
  \end{columns}
\end{frame}

\begin{frame}[fragile]
  \frametitle{目录}
  \only<1>{
    \cmd{AtBeginSection} 用于向每节开始添加内容。
  }

  \only<2>{
    最常见的做法是添加当前节的高亮目录,添加 \opt{currentsection} 参数高亮该节。
  }

  \begin{columns}
    \begin{column}{0.6\textwidth}
      
      \begin{codeblock}[]{节目录}
|\phantom{}|\documentclass{ctexbeamer}
\usetheme{sjtubeamer}
|\highlightline<1>|\AtBeginSection[]{
  \begin{frame}
|\highlightline<2>|    \tableofcontents[currentsection, hideallsubsections]
  |\phantom{}|\end{frame}
|\highlightline<1>|}
\begin{document}
  \section{|\phantom{}|第一节}
|\phantom{}|\end{document}
      \end{codeblock}
      
    \end{column}
    \begin{column}{0.4\textwidth}
      \centering
      \includebeamerlarge[2]{support/beamer/toc.pdf}
    \end{column}
  \end{columns}
\end{frame}

\begin{frame}[fragile]
  \frametitle{目录}
  也可以通过 \cmd{sectionpage} 添加节页,在 SJTUBeamer 里最好在帧上添加 \opt{plain} 参数,以隐藏导航栏。

  \begin{columns}
    \begin{column}{0.6\textwidth}
      
      \begin{codeblock}[]{节页}
|\phantom{}|\documentclass{ctexbeamer}
\usetheme{sjtubeamer}
\AtBeginSection[]{
|\highlightline|  \begin{frame}[plain]
|\highlightline|    \sectionpage
  |\phantom{}|\end{frame}
}
\begin{document}
  \section{|\phantom{}|第一节}
|\phantom{}|\end{document}
      \end{codeblock}
      
    \end{column}
    \begin{column}{0.4\textwidth}
      \centering
      \includebeamerlarge[5]{support/beamer/toc.pdf}
    \end{column}
  \end{columns}
\end{frame}

\begin{frame}
  \frametitle{符合交大视觉形象识别系统}

  \begin{table}
    \caption{上海交通大学视觉形象识别系统规范(节选)}
    \vspace*{-5pt}
    \begin{stampbox}
      \footnotesize
      \begin{tabular}{llll}
        \alert{编号}                                                   & \alert{说明}                   & \alert{编号}                                                   & \alert{说明}     \\
        \href{https://vi.sjtu.edu.cn/index.php/articles/base/1}{A1-06} & 校徽需 $\frac{1}{5}h$ 安全空间 & \href{https://vi.sjtu.edu.cn/index.php/articles/base/5}{A5-03} & 辅助图形使用规范 \\
        \href{https://vi.sjtu.edu.cn/index.php/articles/base/3}{A3-01} & 标准色规范(含色阶)           & \href{https://vi.sjtu.edu.cn/index.php/articles/base/5}{A5-05} & 辅助图形底纹制作 \\
        \href{https://vi.sjtu.edu.cn/index.php/articles/base/3}{A3-02} & 辅助色彩规范(含色阶)         & \href{https://vi.sjtu.edu.cn/index.php/articles/app/7}{B1-01}  & 名片             \\
        \href{https://vi.sjtu.edu.cn/index.php/articles/base/3}{A3-05} & 品牌专用色彩搭配表             & \href{https://vi.sjtu.edu.cn/index.php/articles/app/7}{B1-20}  & 文件封套         \\
        \href{https://vi.sjtu.edu.cn/index.php/articles/base/4}{A4-08} & 二级机构中英文名称横式组合     & \href{https://vi.sjtu.edu.cn/index.php/articles/app/8}{B2-02}  & PPT 模板         \\
      \end{tabular}
    \end{stampbox}
  \end{table}

  正如 SJTUThesis 需要依照学位论文规范一样,SJTUBeamer 借助于 \TikZ{} 实现了《上海交通大学视觉形象识别系统》\link{https://vi.sjtu.edu.cn/} 的绝大部分规范\footnote{使用规范所规定的用品需要遵守相关许可条例 \link{https://vi.sjtu.edu.cn/index.php/articles/bulletin/16}。}。

  \note[item]{我相信小米的二百万 logo 更改让大家意识到了视觉形象识别系统的存在。交大在 2016 年的时候颁布了自己的视觉形象识别系统,并配套了对应的 PPT 模板。}

  \note[item]{许可条例第十三条规定:“校属各单位及个人可以自行或委托其他单位制作《手册》中有明确范例的物品,须严格遵照《手册》的要求,不得自行改动”,\textsc{SJTUBeamer} 也就需要依照该规范制作产物。}
\end{frame}

\begin{frame}
  \frametitle{SJTU VI 对应的特殊方法}

  \begin{columns}
    \begin{column}{0.7\textwidth}

      \stamptext[sjtuRedPrimary]{注} \cmd{stamptext} 用于创建一个印记形状文本框。

      \begin{stampbox}[sjtuRedPrimary]
        \env{stampbox} 环境用于创建一个印记盒子。
      \end{stampbox}

      \cmd{stamparray} 用于在 \env{tikzpicture} 环境(第 \ref{tikz} 页)中创建印记矩阵图案。\textcolor{structure}{$\blacktriangleright$}
    \end{column}
    \begin{column}{0.3\textwidth}
      \begin{figure}
        \centering
        \begin{tikzpicture}
          \setbeamercolor{palette primary}{bg=sjtuRedPrimary,fg=white}
          \stamparray{16pt}{(0,0)}{(4,3)}
        \end{tikzpicture}
        \caption{印记矩阵}
      \end{figure}
    \end{column}
  \end{columns}

  \note{SJTU 视觉形象识别系统规定的一些图案,采用 \TikZ{} 代码的方式绘制出来。}
\end{frame}

\begin{frame}[fragile]
  \frametitle{SJTUThesis 对应的特殊方法}
  \begin{columns}
    \begin{column}{0.55\textwidth}
      \env{codeblock} 环境用于创建代码抄录盒子。与 SJTUThesis 略有不同的是,它需要一个强制参数指定标题,并且注意在所在帧上添加 \opt{fragile} 选项。
    \end{column}
    \begin{column}{0.45\textwidth}
      \begin{codeblock}{代码盒子}
|\textbackslash{}begin\{codeblock\}|[language=python]{Python 代码}
|\phantom{}|print("Hello, world!")
|\textbackslash{}end\{codeblock\}|
      \end{codeblock}
    \end{column}
  \end{columns}

  \note[item]{当然你可以将 \env{codeblock} 环境的强制参数留空,但要保留大括号。这样就不会产生代码块的标题。}

  \begin{columns}
    \begin{column}{0.55\textwidth}
      \env{bibliolist} 环境用于手动创建引用文献列表。可以直接使用 SJTUThesis 的 \cmd{item} 式创建条目,也可以使用 \cmd{articleitem}, \cmd{bookitem}, \cmd{onlineitem} 配合 \cmd{newblock} 创建不同图标的条目。
    \end{column}
    \begin{column}{0.45\textwidth}
      \begin{bibliolist}{00}
        \onlineitem SJTUG.
        \newblock SJTUBeamer 使用手册[EB/OL].
        \newblock 2022. \url{https://github.com/sjtug/SJTUBeamer}
      \end{bibliolist}
    \end{column}
  \end{columns}

  \note[item]{\env{bibliolist} 的环境定义与 SJTUThesis 相同,但是对 \cls{beamer} 有了特殊的优化。}

\end{frame}

\begin{frame}
  \frametitle{SJTUBeamer 的其他附加方法}
  \highlight[sjtuBluePrimary]{高亮} \cmd{highlight} 用于产生高亮效果。
  
  \paragraph{段落} \cmd{paragraph} 在 \cls{beamer} 中原本是无效的。

  \vspace*{0.25em}

  \cmd{makebottom} 用于创建底页,\cls{beamer} 中只有 \cmd{maketitle}。

  \vspace*{1em}

  如果你需要迁移幻灯片至其他样式,可以尝试通过 \cmd{usetheme} 加载该存储库中的 \pkg{nosjtubeamer} 样式。\sout{或者是直接改个徽标}
\end{frame}

\begin{frame}[fragile]
  \frametitle{欢迎贡献主题}
  SJTUBeamer 设计了一套简单的接口,用于将幻灯片修改成你的主题!使用 \cmd{usesjtutheme} 调用社区版 \pkg{contrib} 文件夹内的插件 \link{https://github.com/sjtug/SJTUBeamer/issues/81},可以阅读对应插件附带的文档源码了解如何使用该插件。
  
  \begin{columns}
    \begin{column}{0.55\textwidth}
      \begin{codeblock}{使用子主题}
\documentclass{ctexbeamer}
\usetheme[my]{sjtubeamer}
|\highlightline|\usesjtutheme{sjtug}
\begin{document}
  \begin{frame}
    |\phantom{}|使用 SJTUG 子主题。
|\phantom{}|  \end{frame}
|\phantom{}|\end{document}
      \end{codeblock}
    \end{column}
    \begin{column}{0.45\textwidth}
      \begin{figure}
        \centering
        \includebeamerlarge{support/beamer/sjtug.pdf}
        \caption{SJTUG 子主题}
      \end{figure}
    \end{column}
  \end{columns}

  \note{有的插件并不需要向 sjtubeamer 设定 my 选项。下面来了解一下 \cls{beamer} 的基础模型,因为我觉得这些部分可能要比前面这些设定参数要难一点,所以放在了偏后的位置。以及 SJTUBeamer 对下面的内容优化程度可能还稍显不足。}
  \note{展示几个自建主题。}
  
\end{frame}

\begin{shadedsection}

\section{是什么}

\begin{frame}
  \frametitle{slides}

  \cls{slides} 是为 20 世纪 80 年代中期物理投影片开发的文档类。

  \begin{columns}
    \begin{column}{0.33\textwidth}
      \begin{figure}
        \centering
        \begin{stampbox}[sjtuRedPrimary]
          \scriptsize
          \includegraphics[width=0.88\linewidth]{support/images/slides.jpg}
        \end{stampbox}
        \caption{幻灯影片}
      \end{figure}
    \end{column}
    \begin{column}{0.33\textwidth}
      \begin{figure}
        \centering
        \begin{stampbox}[sjtuRedPrimary]
          \scriptsize
          \includegraphics[width=0.88\linewidth]{support/images/screen.jpg}
        \end{stampbox}
        \caption{准备荧幕}
      \end{figure}
    \end{column}
    \begin{column}{0.33\textwidth}
      \begin{figure}
        \centering
        \begin{stampbox}[sjtuRedPrimary]
          \scriptsize
          \includegraphics[width=0.88\linewidth]{support/images/project.jpg}
        \end{stampbox}
        \caption{投影放映}
      \end{figure}
    \end{column}
  \end{columns}
  \footnotetext{图像由 nickelodeon 持有版权。} % Overleaf 版本将会将图像设为 draft
  \note{我拿海绵宝宝举例说明,首先准备好一卷幻灯影片,放入幻灯机后准备个荧幕,然后按下按钮切换不同的幻灯片,如果是以前的话,你还可以专门找个助手帮你切片,然后你示意一下去切片。}
\end{frame}

\begin{frame}
  \frametitle{beamer}

  \cls{beamer} 是含有一些 PDF 交互式演示功能的文档类。

  \begin{figure}
    \centering
    \includegraphics[width=0.7\linewidth]{support/images/pympress.jpg}
    \caption{Pympress 阅读器\footnote{需要使用 \cmd{note} 添加笔记,并在导言区添加 \cmd{setbeameroption\{notes on second screen\}} 以启用第二屏功能。} \link{https://github.com/Cimbali/pympress}}
    \note[item]{使用 Pympress 阅读器以最大化 beamer 的交互功能(左侧为放映窗口,右侧为演示者视图),并使用类似于 PowerPoint 演示者视图的界面方便演讲,可便携安装。我也参与了界面的汉化工作。}
    \note[item]{为了使用目前的汉化版本,你需要使用 1.7.2 版本,需要从 PyPi 上手动安装包并且按照指示安装依赖(目前只推荐 \faLinux{} 和 \faApple{} 这么做)。}
    \note[item]{也就是 \texttt{pip install pympress --user} 并安装依赖。之后就是 \texttt{pympress} 启动。}
    \note[item]{Pympress 与 LibreOffice 的 Impress Presentation 名称有关。}
  \end{figure}

\end{frame}

\section{帧}

\begin{frame}[fragile,label=frame]
  \frametitle{帧}
  \begin{columns}
    \begin{column}{0.5\textwidth}
      \begin{codeblock}{帧}
\documentclass{ctexbeamer}
\begin{document}
|\highlightline<1>|\begin{frame}
|\highlightline<2>|  \frametitle{|\phantom{}|beamer 介绍}
|\highlightline<2>|  \framesubtitle{|\phantom{}|发起人}
|\highlightline<3>|  beamer 由|\phantom{}| Till Tantau 发起。
|\highlightline<1>|\end{frame}
\end{document}
      \end{codeblock}
    \end{column}
    \begin{column}{0.5\textwidth}
      
      \only<1>{
        \cls{beamer} 的自然切割单位为 \env{frame}(帧),一帧可以派生出很多张幻灯片(slide)。
        \note{当然后来由 Joseph Wright 和 Vedran Miletic 维护,由 samcarter(后文提到的 \pkg{tikzlings} 的作者)提供支持服务。}
        \note{以及 Till Tantau 是当时为了他 PhD 答辩时制作的 beamer 宏包,SJTUBeamer 发起人谌翔是为了他本科答辩时制作的,我重构的一部分是为了我自己某课程大作业 pre 准备的。}
      }

      \only<2>{
        \cmd{frametitle} 用于指定该帧的标题,\cmd{framesubtitle} 还可以用来指定该帧的子标题。
        \note{当然也可以在 \env{frame} 环境后用大括号添加这两个参数,两者都是可行的。}
      }

      \only<3>{
        \begin{center}
          \includebeamerlarge{support/beamer/frame.pdf}
        \end{center}
      }
    \end{column}
  \end{columns}
\end{frame}

\section{渐进切换}

\begin{frame}[fragile]
  \frametitle{占位覆盖}
  
  \begin{columns}
    \begin{column}{0.5\textwidth}
      \begin{codeblock}{暂停式}
\documentclass{ctexbeamer}
\begin{document}
\begin{frame}
|\highlightline<2-4>|    第一段
|\highlightline<1>|  \pause
|\highlightline<3-4>|
|\highlightline<3-4>|    第二段
|\highlightline<1>|  \pause
|\highlightline<4>|        
|\highlightline<4>|    第三段
|\phantom{}|\end{frame}
\end{document}
      \end{codeblock}
    \end{column}
    \begin{column}{0.5\textwidth}

      \only<1>{
        \cmd{pause} 用来渐进地展示每一部分的内容,未展示的内容会有占位。
        \note{可以注意一下下面几张幻灯片,默认情况下应该是垂直居中对齐(\env{frame} 默认的 \opt{c} 参数)的,而下面几张会有一定的偏移,就是因为占位的内容被实际排印出来,但是通过内部机制隐藏掉了这些内容。}
      }

      \only<2-4>{
        \makebox[0.8cm][l]{\scalebox{0.8}{\includebeamerlarge[1]{support/beamer/pause.pdf}}}
      }
      \only<3-4>{
        \makebox[0.8cm][l]{\scalebox{0.8}{\includebeamerlarge[2]{support/beamer/pause.pdf}}}
      }
      \only<4>{
        \makebox[0.8cm][l]{\scalebox{0.8}{\includebeamerlarge[3]{support/beamer/pause.pdf}}}
      }
    \end{column}
  \end{columns}
  
\end{frame}

\begin{frame}[fragile]
  \frametitle{替代覆盖}

  \begin{columns}
    \begin{column}{0.5\textwidth}
      \begin{codeblock}{替代覆盖}
|\phantom{}|\documentclass{ctexbeamer}
|\phantom{}|\begin{document}
  \begin{frame}
|\highlightline<3->|    \only<2->{|\phantom{}|第二}
|\highlightline<2>|    \only<1>{|\phantom{}|第一}
|\highlightline<2,4>|    \only<1,3>{|\phantom{}|第三}
|\phantom{}|  \end{frame}
|\phantom{}|\end{document}
      \end{codeblock}
    \end{column}
    \begin{column}{0.5\textwidth}

      \only<1>{

      \cmd{only} 需要通过\alert{修饰符}来指定内容排印的子幻灯片编号,没有显示的内容将不会占位。由于内容没有被实际排印出来,可能会在不同的子幻灯片间产生版面抖动。

      \begin{table}
        \centering
        \footnotesize
        \begin{tabular}{>{\color{structure}}cl}
          \texttt{<x>} & 
            \begin{tabular}[t]{l}  
              显示在第 \texttt{x} 子页
            \end{tabular} \\
          \texttt{<x,y>} & 
            \begin{tabular}[t]{l}
              显示在第 \texttt{x} 和第 \texttt{y} 子页上\\
              \scriptsize (每一项也可以是一个范围)
            \end{tabular} \\
          \texttt{<x-y>} & 
            \begin{tabular}[t]{l}
              显示在第 \texttt{x} 到第 \texttt{y} 子页上 \\
              \scriptsize (如果 \texttt{y} 被省略,默认到最后一个子页)
            \end{tabular}
            \\
        \end{tabular}
      \end{table}

      \note{注意要分段的话,\cmd{only} 前后的内容要空行,否则会有诡异的首行空格。}
      }

      \only<2-4>{
        \begin{center}
          \begin{tikzpicture}[scale=0.8, every node/.style={scale=0.8}]
            \node at (0,0) {\includebeamerlarge[1]{support/beamer/only.pdf}};
            \only<3->{\node at (0,-2) {\includebeamerlarge[2]{support/beamer/only.pdf}};}
            \only<4>{\node at (0,-4) {\includebeamerlarge[3]{support/beamer/only.pdf}};}
          \end{tikzpicture}
        \end{center}
      }

      \only<5>{
        与 \cmd{only} 类似的还有 \cmd{alt},\cmd{temporal}。

        \begin{table}
          \centering
          \footnotesize
          \begin{tabular}{>{\ttfamily}lccc}
             & \scriptsize 1 & \scriptsize 2 & \scriptsize 3 \\
             \midrule
            \cmd{only}<2>\{二\} & & 二 & \\
            \cmd{alt}<2>\{二\}\{一\} & 一 & 二 & 一 \\
            \cmd{temporal}<2>\{一\}\{二\}\{三\} & 一 & 二 & 三 \\
          \end{tabular}
        \end{table}
      }

    \end{column}
  \end{columns}

\end{frame}


\begin{frame}[fragile]
  \frametitle{渐进切换}

  \begin{columns}
    \begin{column}{0.5\textwidth}
      \begin{codeblock}{占位覆盖}
\documentclass{ctexbeamer}
\begin{document}
|\phantom{}|  \begin{frame}
|\highlightline<3->|    \onslide<2->{|\phantom{}|第二}
|\highlightline<2>|    \onslide<1>{|\phantom{}|第一}
|\highlightline<2,4>|    \onslide<1,3>{|\phantom{}|第三}
|\phantom{}|  \end{frame}
\end{document}
      \end{codeblock}
    \end{column}
    \begin{column}{0.5\textwidth}
      
      \only<1>{
        \cmd{onslide} 可以通过修饰符来指定显示的子幻灯片页码,\cmd{pause} 与此类似但不能指定子页码。
      }

      \only<2-4>{
        \begin{center}
          \begin{tikzpicture}[scale=0.8, every node/.style={scale=0.8}]
            \node at (0,0) {\includebeamerlarge[1]{support/beamer/onslide.pdf}};
            \only<3->{\node at (0,-2) {\includebeamerlarge[2]{support/beamer/onslide.pdf}};}
            \only<4>{\node at (0,-4) {\includebeamerlarge[3]{support/beamer/onslide.pdf}};}
          \end{tikzpicture}
        \end{center}
      }

    \end{column}
  \end{columns}
\end{frame}

\begin{frame}
  \frametitle{渐进切换}
  实际上 \cmd{onslide} 是覆盖命令的基础命令,其变种可以等同其他覆盖命令 \link{https://github.com/CTeX-org/beamer-code-cn}。

  \begin{table}
    \centering
    \begin{tabular}{>{\ttfamily}lc>{\ttfamily}l}
      \cmd{pause} & $\Rightarrow$ & \cmd{onslide}<c-> \\
      \cmd{only}<x> & $=$ & \cmd{onslide}*<x> \\
      \cmd{uncover}<x> & $=$ & \cmd{onslide}<x> \\
      \cmd{visible}<x> & $=$ & \cmd{onslide}+<x>\footnote{\cmd{onslide+} 和 \cmd{onslide} 的区别在于前者即使在全局使用透明覆盖模式时 \cmd{setbeamercovered\{transparent\}} 透明度也是无效的(只能全显示或不显示)。} \\
    \end{tabular}
  \end{table}
  
  \note{在导言区使用 \cmd{setbeamercovered\{transparent\}} 将会产生半透明的效果。}

  渐进切换是 \cls{beamer} 非常强大的编程模型,合理使用可以减少重复代码的编写。但也希望它的文本参数不要太过复杂,否则可能无法解析。\textsl{Keep It Simple, Stupid.}

  \note{后来 PowerPoint 才有渐进切换的动画效果。}
\end{frame}

\begin{frame}[fragile]
  \frametitle{列表覆盖}
  \begin{columns}
    \begin{column}{0.5\textwidth}
      \begin{codeblock}{列表覆盖}
\documentclass{ctexbeamer}
\begin{document}
  \begin{frame}
|\highlightline<1>|    \begin{itemize}[<+->]
|\highlightline<2-4>|      \item 第一点
|\highlightline<3-4>|      \item 第二点
|\highlightline<4>|      \item 第三点
    \end{itemize}
|\phantom{}|  \end{frame}
\end{document}
      \end{codeblock}
    \end{column}
    \begin{column}{0.5\textwidth}
      
      \only<1>{
        对列表环境添加 \opt{<+->} 修饰符可以逐项添加覆盖效果。\opt{<+-|alert@+>} 还可以高亮当前的覆盖项目。
        \note{还有一些特殊的渐进切换语法糖,比如列表覆盖可以直接在环境上添加简单的修饰符。还可以使用 \opt{alert} 参数来高亮当前项目。以及 \cmd{beamerdefaultoverlayspecification}\{\opt{<+->}\} 可以设定默认覆盖选项。}
      }
      \only<2-4>{
        \makebox[1cm][l]{\includebeamerlarge[1]{support/beamer/itemizeoverlay.pdf}}
      }
      \only<3-4>{
        \makebox[1cm][l]{\includebeamerlarge[2]{support/beamer/itemizeoverlay.pdf}}
      }
      \only<4>{
        \makebox[1cm][l]{\includebeamerlarge[3]{support/beamer/itemizeoverlay.pdf}}
      }
    \end{column}
  \end{columns}
\end{frame}

\begin{frame}[fragile]
  \frametitle{其他覆盖}
  \begin{columns}[b]
    \begin{column}{0.5\textwidth}
      一些样式命令可以添加与 \cmd{alt} 行为类似的可选修饰符参数(替换覆盖)。
      \note[item]{这些样式命令可以直接添加修饰符来设定高亮的子幻灯片编号,采用当前子幻灯片替换宏的方法。注意这是替换覆盖。}
      \begin{table}
        \footnotesize
        \begin{tabular}{lll}
          \toprule
          \cmd{textbf} & \cmd{textmd} & \cmd{textit} \\
          \cmd{textnormal} & \cmd{textrm} & \cmd{textsc} \\
          \cmd{textsf} & \cmd{textsl} & \cmd{texttt} \\
          \cmd{textup} & \cmd{emph} & \cmd{color} \\
          \cmd{textcolor} & \cmd{alert} & \cmd{structure} \\
          \bottomrule
        \end{tabular}
      \end{table}
      \begin{codeblock}[]{样式覆盖}
\textbf<1>{|\phantom{}|第一页加粗}
\alert<2>{|\phantom{}|第二页强调}
\textcolor<3>{blue}{|\phantom{}|第三页变蓝}
      \end{codeblock}
    \end{column}
    \begin{column}{0.5\textwidth}
      区块和定理环境可以添加与 \cmd{onslide} 行为类似的可选修饰符参数(占位覆盖)。
      \note[item]{这些预定义的 block 和 theorem 环境也可以添加修饰符参数,用于占位覆盖的情况。}
      \begin{table}
        \footnotesize
        \begin{tabular}{lll}
          \toprule
          \env{block} & \env{alertblock} & \env{exampleblock} \\
          \midrule
          \env{theorem} & \env{corollary} & \env{definition} \\
          \env{definitions} & \env{fact} & \env{example} \\
          \env{examples} & \env{proof} &  \\ 
          \bottomrule
        \end{tabular}
      \end{table}
      \vspace*{0.2cm}
      \begin{codeblock}[]{环境覆盖}
\begin{theorem}<2->[|\phantom{}|定理名]
|\phantom{}|  第二页之后显示该定理。
\end{theorem}
      \end{codeblock}
    \end{column}
  \end{columns}
\end{frame}

\section{断区接续}

\begin{frame}[fragile]
  \frametitle{分栏}
  \begin{columns}[t]
    \begin{column}{.5\textwidth}
      \cls{beamer} 提供的标准分栏方法是 \env{columns} 环境,硬分栏方法有助于保持设计。
      
      \begin{codeblock}{}
\documentclass{ctexbeamer}
\begin{document}
  \begin{frame}
|\highlightline<1>|    \begin{columns}[c]
|\highlightline<2>|      \begin{column}{.5\textwidth}
        |\phantom{}|第一栏
|\highlightline<2>|      \end{column}
|\highlightline<2>|      \begin{column}{.5\textwidth}
        |\phantom{}|第二栏
|\highlightline<2>|      \end{column}
|\highlightline<1>|    \end{columns}
|\phantom{}|  \end{frame}
\end{document}
      \end{codeblock}
      
    \end{column}
    \begin{column}{.5\textwidth}
      如果你想让过长的内容自动分栏,不妨尝试 \pkg{multicol} 宏包中的 \env{multicols} 环境。

      \note[item]<1>{我个人还是蛮推荐使用 \env{multicols} 的,因为 \env{columns} 带来不便的同时,可能会导致一些宏包无法正常工作(比如 \pkg{parskip})。当然对于本幻灯片而言,更多的应当使用 \env{columns} 因为我需要配合 \env{codeblock} 环境排印代码,需要硬分栏(不希望这一栏的代码跑到另一栏的说明中去)。}
      
      \begin{codeblock}{}
\documentclass{ctexbeamer}
|\highlightline|\usepackage{multicol}
\begin{document}
  \begin{frame}
|\highlightline|    \begin{multicols}{2}
      |\phantom{}|所有内容
|\highlightline|    \end{multicols}
|\phantom{}|  \end{frame}
\end{document}
      \end{codeblock}

      \small
      \vspace*{1em}

      \only<1>{
        \env{columns} 环境可以添加可选参数,用于指定两栏纵向对齐方式(\opt{b},\opt{c},\opt{t},\opt{T})。\textcolor{structure}{$\blacktriangleleft$}
        \note[item]{\opt{b} 底部对齐,\opt{c} 中心对齐,\opt{t} 顶部基线对齐,\opt{T} 顶部对齐(\opt{t} 奇怪的时候不妨试试这个参数)。}
      }
      
      \only<2>{
        \env{column} 环境用于分栏,需要强制参数用于指定栏宽。\textcolor{structure}{$\blacktriangleleft$}
        \note[item]{\env{columns} 环境的下一级子环境一般情况下应当是 \env{column}。}
      }
      
    \end{column}
  \end{columns}
\end{frame}

\begin{frame}[fragile]
  \frametitle{断帧接续}
  \note[item]{标题名称断帧接续,意为断开帧继续,也就是自动将一帧的内容切分为多张幻灯片。}
  \begin{columns}[t]
    \begin{column}{0.5\textwidth}
      对 \env{frame} 环境赋予 \opt{allowframebreaks} 参数,一张幻灯片内多余的内容就会流入到下一张幻灯片中。
      
      \begin{codeblock}[]{}
\documentclass{ctexbeamer}
\usepackage{zhlipsum}
|\phantom{}|\begin{document}
|\highlightline|  \begin{frame}[allowframebreaks]
    \zhlipsum[1-2]
|\phantom{}|  \end{frame}
|\phantom{}|\end{document}
      \end{codeblock}

      \vspace*{2em}
      \small

      % \begin{alertblock}{}
        \opt{allowframebreaks=0.8} 将指定每一个子页的高度占比不超过 0.8\cmd{textheight}。\textcolor{structure}{$\blacktriangle$}
      % \end{alertblock}
      
    \end{column}
    \begin{column}{0.5\textwidth}
      对于数学公式而言,可以对 \env{frame} 环境\textbf{再}赋予 \opt{allowdisplaybreaks} 参数,就可以对公式按行截断。

      \begin{codeblock}{}
\documentclass{ctexbeamer}
\begin{document}
|\highlightline|    \begin{frame}[allowframebreaks,
|\highlightline|  allowdisplaybreaks]
\begin{align*}
    a =& b  \\
       &+ c \\
    % 长公式...
\end{align*}
|\phantom{}|    \end{frame}
|\phantom{}|\end{document}
      \end{codeblock}
    \end{column}
  \end{columns}
\end{frame}

\begin{frame}[fragile]
  \frametitle{自动切割}
  \begin{block}{}
    \begin{description}
      \item[\faExclamationTriangle] 慎用断帧接续,会导致渐进切换无法使用!
      \item[\faExclamationTriangle{} \faExclamationTriangle{}] 慎用全局自动切割,滥用将降低可维护性!
    \end{description}
  \end{block}
  \note[item]{这将导致后面的并行程序失效,变成了 \LaTeX{} 串行排版,毕竟鱼和熊掌不可兼得。}
  \note[item]{\LaTeX{} 幻灯片工具人震怒。}
  紧急的时候,可以尝试配合 \cmd{framebreak} 手动分割子页进行全局幻灯片自动切分。
  \begin{columns}
    \begin{column}{0.5\textwidth}
      \begin{codeblock}[]{}
\documentclass{beamer}
|\highlightline|\usepackage{autobeamer}
\begin{document}
    \begin{frame}[allowframebreaks=0.8,fragile]
        % Your full article ...
|\phantom{}|    \end{frame}
|\phantom{}|\end{document}
      \end{codeblock}
      \small
      \vspace*{1em}
      % \begin{alertblock}{}
        更多详见 AutoBeamer 的 \faCodeBranch{} \pkg{pkg} 分支 \link{https://github.com/LogCreative/AutoBeamer/tree/pkg}
      % \end{alertblock}
    \end{column}
    \begin{column}{0.5\textwidth}
      \begin{codeblock}[]{autobeamer.sty}
\def\section#1{\par\framebreak
  {\bfseries \color{red} #1}}
\def\chapter#1{\framebreak
  \vspace*{0.3\paperheight}
  \begin{center}
    \Large\color{red} #1 
  \end{center}
  \vspace*{0.3\paperheight}
  \newpage}
      \end{codeblock}
    \end{column}
  \end{columns}
  \note{AutoBeamer 自动切割未来可能是一个较大的程序,要想真的完成自动从论文生成幻灯片,需要将后文的那些机制都结合起来,如果我有时间的话或许会写成 VS Code 插件。再高级点就要加 Machine Learning/NLP,提取每一段语义的思想分割并生成标题,这个 idea 或许有商业价值(当然最近的 Google I/O 大会已经发布了 Google Docs 的最新 TL;DR 功能。\link{https://support.google.com/docs/thread/150906266/google-docs-new-feature-summaries?hl=en}),但我觉得谁想做谁做吧,现在还是处于相对初级的阶段。}
\end{frame}

\section{跳转}

\begin{frame}[fragile,label=jump]
  \frametitle{引用跳转}
  最基本的跳转方式是对 \env{frame} 添加一个标签,之后使用 \cmd{ref} 该标签跳转。

  \note{或许可以直接在 \env{frame} 环境内和往常一样添加 \cmd{label} 标签,但是这样会导致无法精确地使用修饰符跳转到对应的子页。所以更推荐的做法是直接在 \env{frame} 环境上添加 \opt{label} 参数,注意这个参数的内容不能有冒号。}
  
  \begin{columns}
    \begin{column}{0.5\textwidth}
      \begin{codeblock}[]{}
\documentclass{ctexbeamer}
\begin{document}
|\highlightline|  \begin{frame}[label=target]
    |\phantom{}|跳转到的帧。
|\phantom{}|  \end{frame}
  \begin{frame}
|\highlightline|    |\phantom{}|详见第 \ref{target} 页。
|\phantom{}|  \end{frame}
|\phantom{}|\end{document}
      \end{codeblock}
    \end{column}
    \begin{column}{0.5\textwidth}
      \begin{center}
        \includebeamerlarge[2]{support/beamer/label.pdf}
      \end{center}
    \end{column}
  \end{columns}
\end{frame}

\begin{frame}[fragile]
  \frametitle{按钮跳转}
  还可以采用 \cmd{hyperlink} 外加自带的按钮命令跳转。

  \note{\cmd{hyperlink} 用于跳转标签,而 \cmd{href} 用于跳转特定的地址 URL。}

  \begin{columns}
    \begin{column}{0.5\textwidth}
      \begin{codeblock}[]{}
\documentclass{ctexbeamer}
\begin{document}
|\highlightline|  \begin{frame}[label=target]
    |\phantom{}|跳转到的帧。
|\phantom{}|  \end{frame}
  \begin{frame}
|\highlightline|    \hyperlink{target}{
|\highlightline|  \beamergotobutton{|\phantom{}|跳转}}
|\phantom{}|  \end{frame}
|\phantom{}|\end{document}
      \end{codeblock}
    \end{column}
    \begin{column}{0.5\textwidth}
      \begin{table}
        \centering
        \caption{按钮样式}
        \footnotesize
        \begin{tabular}{ll}
          \cmd{beamerbutton} & \hyperlink{jump}{\beamerbutton{跳转}} \\
          \cmd{beamergotobutton} & \hyperlink{jump}{\beamergotobutton{跳转}} \\
          \cmd{beamerskipbutton} & \hyperlink{jump}{\beamerskipbutton{跳转}} \\
          \cmd{beamerreturnbutton} & \hyperlink{jump}{\beamerreturnbutton{跳转}} \\  
        \end{tabular}
      \end{table}
    \end{column}
  \end{columns}
\end{frame}

\section{建议}

\begin{frame}
  \frametitle{注意事项}
  \begin{itemize}
    \item 建议在制作幻灯片时尽量分点展示,分段会导致段与段之间的分界不够明显。
    \item 尽量不使用代码抄录环境,使用时需要在 \env{frame} 上添加 \opt{fragile} 选项。
    \note{用抄录环境的话最好定义一个命令,然后使用 \cmd{texttt} 更改为等宽字体样式即可。}
    \item 可以多用图标代替文字,比如使用 \pkg{fontawesome5} 宏包。
    \item 尽可能使用矢量图形,大型图片先压缩再插入,必要时开启 \opt{draft} 选项。
    \item 避免嵌入动画与视频,转而采用超链接的方式。
    \begin{center}
      \ttfamily
      \cmd{href}\{run:\,地址\}\{链接标识\}
    \end{center}
  \end{itemize}
\end{frame}

\end{shadedsection}

% !TeX root = ../../latex-talk.tex

\part{像模像样 \LaTeX{}}

\section{TikZ}



\section{PGFPlots}

\begin{frame}
    \frametitle{PGFPlotsEdt 帮助我完成物理实验报告图...}
    \begin{columns}
        \begin{column}{0.25\textwidth}
            \includegraphics[width=\linewidth]{a}
            \includegraphics[width=\linewidth]{b}
        \end{column}
        \begin{column}{0.50\textwidth}
            \includegraphics[width=\linewidth]{c}
            \begin{columns}
                \begin{column}{0.5\linewidth}
                    \includegraphics[width=\linewidth]{d}
                \end{column}
                \begin{column}{0.5\linewidth}
                    \includegraphics[width=\linewidth]{e}
                \end{column}
            \end{columns}
        \end{column}
        \begin{column}{0.25\textwidth}
            \includegraphics[width=\linewidth]{f}
            \includegraphics[width=\linewidth]{g}
        \end{column}
    \end{columns}
\end{frame}
% !TeX root = ../../latex-talk.tex

\part{\LaTeX{} 无止境}

\begin{frame}
  \frametitle{宏包}

  \begin{itemize}
    \item 绝大多数 \LaTeX{} 宏包发布于 CTAN (Comprehensive \TeX{} Archive Network) \link{https://www.ctan.org/},1992 年建站以来,至今已经吸引 2855 名贡献者发布其 6215 个宏包。 
    \item \note{正如一般不会在简历上写精通 C 一样,也都}基本不敢写精通 \LaTeX{} \sout{(虽然没有什么大用}
    \item 可以通过编写宏来简化输入,可以阅读 \pkg{clsguide}\footnote{(中文版)刘海洋~译. 写给 \LaTeXe{} 类与宏包的作者[EB/OL]. 2010. \link{https://github.com/CTeX-org/ctex-doc/blob/master/clsguide-zh-cn/clsguide-zh-cn.tex}}, 
    \pkg{xparse}\footnote{(中文版)盛文博~译. \pkg{xparse} 宏包:进行文档命令解析[EB/OL]. 2018. \link{https://github.com/WenboSheng/LaTeX3-doc-cn/blob/master/xparse-doc-cn/xparse-cn.pdf}} 文档。
  \end{itemize}

  \begin{exampleblock}{定义宏举例}
    \begin{description}
      \item[\TeX] \cmd{def}\texttt{\textbackslash{}foo}\#1\#2\{\#1 of \#2\}
      \item[\LaTeXe] \cmd{newcommand}\texttt{\textbackslash{}foo}[2]\{\#1 of \#2\}
      \item[\LaTeX3] \cmd{NewDocumentCommand}\texttt{\textbackslash{}foo}\{mm\}\{\#1 of \#2\}
    \end{description}
  \end{exampleblock}

  \note{简单了解即可。}
  
\end{frame}

\begin{frame}[fragile]
  \frametitle{错误}
  \begin{columns}
    \begin{column}{0.55\textwidth}
      \begin{codeblock}[]{未定义命令}
\newcommand\mycommand{\textbold{hmmm}}
My command is used here \mycommand.
      \end{codeblock}
      \begin{codeblock}[]{括号不配对}
\usepackage[leqno}{amsmath}
      \end{codeblock}
    \end{column}
    \begin{column}{0.45\textwidth}
      \begin{codeblock}[]{文件缺失}
\usepackage{amsmathz}
      \end{codeblock}
      \begin{codeblock}[]{数学模式有空行}
\begin{equation}

  1=2

\end{equation}
      \end{codeblock}
    \end{column}
  \end{columns}
\end{frame}

\begin{frame}[fragile]
  \frametitle{错误}
  \begin{columns}
    \begin{column}{0.55\textwidth}
      \begin{block}{未定义命令}
        \begin{lstlisting}
! Undefined control sequence.
\mycommand ->\textbold 
                        {hmmm}
l.8 My command is used here \mycommand
                                      .
? 
        \end{lstlisting}
      \end{block}
      \begin{block}{括号不配对}
        \begin{lstlisting}
! Argument of \@fileswith@ptions has an extra }.
        \end{lstlisting}
      \end{block}
    \end{column}
    \begin{column}{0.45\textwidth}
      \begin{block}{文件缺失}
        \begin{lstlisting}
! LaTeX Error: File `amsmathz.sty' not found.
        \end{lstlisting}
        \note{检查是否安装了该宏包。或者宏包是否在搜索路径内。}
      \end{block}
      \begin{block}{数学模式有空行}
        \begin{lstlisting}
! Missing $ inserted.
        \end{lstlisting}
      \end{block}
    \end{column}
  \end{columns}
\end{frame}

\begin{frame}
  \frametitle{编译速度}
  \begin{columns}
    \begin{column}{0.5\textwidth}
      \begin{exampleblock}{\faWindows}
        \begin{itemize}
          \item \hologo{pdfLaTeX} 仍是目前最快的。
          \item 中文支持方面需要做变通方法。
          \item I/O 速度成为瓶颈。
        \end{itemize}
      \end{exampleblock}
    \end{column}
    \begin{column}{0.5\textwidth}
      \begin{exampleblock}{\faLinux{} \faApple}
        \begin{itemize}
          \item \hologo{XeLaTeX} 已经足够快。
          \item \CTeX{} 需要 \hologo{XeLaTeX} 或 \hologo{LuaLaTeX}。
          \item 主要取决于单核性能。
        \end{itemize}
      \end{exampleblock}
    \end{column}
  \end{columns}
  \begin{columns}
    \begin{column}{0.5\textwidth}
      \begin{block}{文章}
        \begin{itemize}
          \item \LaTeX{} 的主要设计用途。
          \item \hologo{pdfLaTeX}, \hologo{XeLaTeX} 和 \hologo{LuaLaTeX} 理想状态下速度差距不大 \link{https://stone-zeng.github.io/2019-07-24-tex-benchmark/}。
        \end{itemize}
      \end{block}
    \end{column}
    \begin{column}{0.5\textwidth}
      \begin{block}{幻灯}
        \begin{itemize}
          \item \LaTeX{} 的图形偏门流派。
          \item 由于 \textsc{pgf} 的引入使得 \hologo{LuaLaTeX} 明显偏慢。
        \end{itemize}
      \end{block}
    \end{column}
  \end{columns}

  \note[item]{魔兽争霸 3 当年出来的时候战役模式是主流,后来大家都开始打 RPG, DotA(没错,发源地),自定义的一些内容也可以很精彩。}

  \note[item]{用 PowerPoint 做完幻灯片会有一种空虚感,因为可能会做大量的重复工作。而 beamer 如果使用的好的话不会。}
\end{frame}

\begin{frame}
  \frametitle{加速轮子}
  \begin{columns}
    \begin{column}{0.33\textwidth}
      \begin{exampleblock}{ReportBoost \link{https://github.com/LogCreative/ReportBoost}}
        \begin{tabular}{p{0.9\textwidth}}
          提供一个已经配置完备的 Visual Studio Code 预编译工程目录。\\
          \midrule
          主要使用 \hologo{eTeX} 对 \hologo{pdfLaTeX} 在 \faWindows{} 上转储头文件为一个中间文件,减少 I/O 操作。
          \note{\hologo{eTeX} 是原生 \TeX{} 的扩展版本,支持更多的字符数,\hologo{XeTeX} 和 \hologo{LuaTeX} 基于其演变而来。}\\
          \midrule
          对 \hologo{XeLaTeX} 和 \hologo{LuaLaTeX} 支持不佳。
        \end{tabular}
      \end{exampleblock}
    \end{column}
    \begin{column}{0.33\textwidth}
      \begin{exampleblock}{AutoBeamer \link{https://github.com/LogCreative/AutoBeamer}}
        \begin{tabular}{p{0.9\textwidth}}
          主要提供将 \faMarkdown{} 文件翻译为 \LaTeX{} \cls{beamer} 代码的功能。\\
          \midrule
          \pkg{pandoc} \link{https://pandoc.org/index.html} 也可以将 \faMarkdown{} 转换为 \LaTeX{} 幻灯片代码,但结果不够干净。\\
          \midrule
          使用网页脚本语言 \faJs{} 造轮子主要考虑跨平台与未来功能迁移的可能。
        \end{tabular}
      \end{exampleblock}
    \end{column}
    \begin{column}{0.33\textwidth}
      \begin{exampleblock}{BeamerBoost \link{https://github.com/LogCreative/BeamerBoost}}
        \begin{tabular}{p{0.9\textwidth}}
          主要提供对 \cls{beamer} 文档类自然切割单位(帧)的脏区与并行渲染实现,更早拿到每帧结果。\\
          \midrule
          目前只能采用预编译的手段减少进程切换开销,主要面向 \faWindows{} 编译较慢情况的预览。\\
          \midrule
          导航栏结果未必准确。
        \end{tabular}
      \end{exampleblock}
    \end{column}
  \end{columns}

  \note{这些程序还都处于相互分离的阶段。但是也可以看到这种使用 Unix 哲学所涉及的程序带来的好处:每一个程序只做好一件事。模块化、尽可能使用文本输入输出流有助于程序的构建与修改,每一个部分都可以换成新的零件。}

  \note{beamer(与动画 \pkg{animate}) 可能是目前 \LaTeX{} 方面唯一有可能并行化的东西。}
\end{frame}


\def\bottomthanks{Happy \TeX{}ing!}
\makebottom

\end{document}
