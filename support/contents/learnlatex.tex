% !TeX root = ../../latex-talk.tex

\part{学习 \LaTeX{}}
% FIXME: footnote fault numbering
% FIXME: section pop up in navigation in advance

\begin{frame}
  \frametitle{本部分主要参考}
  \begin{bibliolist}{00}
    \onlineitem 陈晟祺~等.
    \newblock 如何使用 \LaTeX\ 排版论文[EB/OL].
    \newblock 2021. \url{https://github.com/tuna/thulib-latex-talk}.

    \onlineitem 曾祥东.
    \newblock 现代 \LaTeX\ 入门讲座[EB/OL].
    \newblock 2022. \url{https://github.com/stone-zeng/latex-talk}.

    \onlineitem \LaTeX\ Project.
    \CTeX\ 开发小组~译.
    \newblock learnlatex.org[EB/OL].
    \newblock 2022. \url{https://github.com/CTeX-org/learnlatex.github.io}.

    \onlineitem \textsc{Oetiker T}, \textsc{Partl H}, \textsc{Hyna I}, \textsc{Schlegl E}.
    \CTeX\ 开发小组~译.
    \newblock 一份(不太)简短的 \LaTeXe{} 介绍:或 111 分钟了解 \LaTeXe{}[EB/OL]. \newblock\newblock 2021.
    \url{https://ctan.org/pkg/lshort-zh-cn}.
  \end{bibliolist}

  \note{推荐大家去阅读这些入门材料。}
\end{frame}

\begin{frame}[plain]
  \vfil
  \begin{center}
    \href{https://learnlatex.org}{
      \rmfamily
      Learn\,\lower1ex\hbox{\Huge\LaTeX{}}.org
    }
  \end{center}
  \vfil
  \begin{center}
    \parbox{0.75\linewidth}{
      Learn\LaTeX{}.org 提供了解 \LaTeX{} 的 16 篇简短的教程,并包含了一些可以在线运行的示例,可以通过亲自动手查看实验效果。本部分主要参考由 \CTeX{}-org 提供的中文翻译版本 \link{https://github.com/CTeX-org/learnlatex.github.io/tree/zh-Hans/zh-Hans/}。
    }
  \end{center}
  \vfil
  \note{这一部分主要参考了 Learn\LaTeX{}.org 的系列教程,内容简洁,适合入门,符合
  教育学观念 \link{https://www.tug.org/TUGboat/tb41-2/tb128reviews-learnlatex.pdf},
  并参考由 \CTeX{}-org 提供的中文翻译版本
  \link{https://github.com/CTeX-org/learnlatex.github.io/tree/zh-Hans/zh-Hans/}。
  如果你认为下面一个小时的入门教程没有讲得非常细致的话,
  欢迎直接阅读这个网站的全文。}
\end{frame}

\begin{shadedsection}

\section{是什么}

\begin{frame}
  \frametitle{\TeX{}}
  \begin{columns}[c]
    \begin{column}{0.7\textwidth}
      \begin{center}
        \rmfamily\Huge
        \highlight[structure]{\TeX{}}
      \end{center}
      \begin{center}
        \parbox{0.75\textwidth}{
          \TeX{} 是由斯坦福大学教授高德纳
          (Donald E.~Knuth)于 1977 年开始开发的排版引擎。目前仍在更新,最新版本号为 3.141592653 \link{https://tug.org/TUGboat/tb42-1/tb130knuth-tuneup21.pdf}。
        }
      \end{center}
    \end{column}
    \begin{column}{0.3\textwidth}
      \includegraphics[width=.8\columnwidth]{support/images/Knuth.jpg}
    \end{column}
  \end{columns}
  \note{\emph{这一部分背景介绍大家可以了解一下,暂时跳过。}
  \LaTeX{} 这个词由两个部分组成,\hologo{La} 和 \TeX{}。那我们首先了解一下 \TeX{} 是什么。
  \TeX{} 是由斯坦福大学的教授高德纳于 1977 年开始开发的排版引擎,它已经有三十多年的历史了,
  目前仍在更新,版本号(3.141592653)将会趋近于 $\pi$ 的取值,高德纳最近还在给 \textsl{TUGBoat} 写稿子
  \link{https://tug.org/TUGboat/tb42-1/tb130knuth-tuneup21.pdf},
  关于 \TeX{} 今年又做了哪些改进。}
\end{frame}

\begin{frame}
  \frametitle{\LaTeX{}}
  \begin{columns}[c]
    \begin{column}{0.7\textwidth}
      \begin{center}
        \rmfamily\Huge
        \highlight[structure]{\LaTeX{}}
      \end{center}
      \begin{center}
        \parbox{0.75\textwidth}{
          \LaTeX{} 是最早在 1985 年由现就职于微软的 Leslie Lamport 开发的一种 \TeX{} \textbf{格式}\footnotemark,使用一些列宏和扩展宏包来简化 \TeX{} 的使用。现在由 \LaTeX{} Project 的成员维护。现在广泛使用的版本是 \LaTeXe{},最新的版本为 \LaTeX3(2020 年 10 月后默认内置)。
        }
      \end{center}
    \end{column}
    \begin{column}{0.3\textwidth}
      \includegraphics[width=.8\columnwidth]{support/images/Lamport.jpg}
    \end{column}
  \end{columns}
  \footnotetext{\hologo{ConTeXt} 也是一种 \TeX{} 格式 \link{https://www.contextgarden.net/}。}
  \note{\emph{这一部分的背景介绍大家可以了解一下,暂时跳过。}
  \LaTeX{} 是最早由现就职于微软的 Leslie Lamport 开发的一种 \TeX{} 格式(与其对标的是
  \hologo{ConTeXt}\link{https://www.contextgarden.net/}),主要也是为了简化 \TeX{} 的使用。
  现在主要由 \LaTeX{} 开发组维护,现在广泛使用的版本是 \LaTeXe{},最新的版本为 \LaTeX3,
  在 2020 年 10 月后默认内置,所以要尽可能使用较新的发行版,以充分发挥其功能。}
\end{frame}

\begin{frame}
  \frametitle{程序}
  \begin{columns}[c]
    \begin{column}{0.7\textwidth}
      \begin{center}
        \rmfamily\Huge
        \highlight[structure]{\hologo{pdfLaTeX}}
      \end{center}
      \begin{center}
        \parbox{0.7\textwidth}{
          \hologo{pdfLaTeX} 是为了编译一个 \LaTeX{} 文档而运行的程序。实际上底层在运行一个叫 \hologo{pdfTeX} 的引擎,并预装了对应的 \LaTeX{} \textbf{格式}。为了利用临时文件,可能就需要多次运行程序。
        }
      \end{center}
    \end{column}
    \begin{column}{0.3\textwidth}
      \begin{block}{}
        \ttfamily\small
        > \highlight{pdflatex} main.tex\\
        This is pdfTeX, Version 3.141592653-
        2.6-1.40.23 (MiKTeX 21.10)\\
        entering extended mode\\
        \highlight{LaTeX2e} <2021-11-15>\\
        \highlight{L3} programming layer <2021-11-22>
      \end{block}
    \end{column}
  \end{columns}
  \note{\hologo{pdfLaTeX} 是为了编译一个 \LaTeX{} 文档而运行的程序。}
\end{frame}

% \begin{frame}
%   \frametitle{引擎}
%   \begin{columns}[c]
%     \begin{column}{0.7\textwidth}
%       \begin{center}
%         \rmfamily\Huge
%         \highlight[structure!70]{pdf}\hologo{La}\highlight[structure!70]{\TeX{}}
%       \end{center}
%       \begin{center}
%         \parbox{0.7\textwidth}{
%           pdf\TeX{} 是编译 \TeX{} 文档(以 \texttt{.tex} 结尾)的\textbf{引擎}---可以理解 \TeX{} 指令的\textbf{程序}。
%         }
%       \end{center}
%     \end{column}
%     \begin{column}{0.3\textwidth}
%       \begin{block}{}
%         \ttfamily\small
%         > pdflatex main.tex\\
%         This is \highlight[structure!70]{pdfTeX}, Version 3.141592653-
%         2.6-1.40.23 (MiKTeX 21.10)
%         entering extended mode\\
%         LaTeX2e <2021-11-15>\\
%         L3 programming layer <2021-11-22>
%       \end{block}
%     \end{column}
%   \end{columns}
%   \note{实际上底层在运行一个叫 \hologo{pdfTeX} 的引擎,并预装了对应的 \LaTeX{} 格式。}
% \end{frame}

\begin{frame}[label={frame:engine}]
  \frametitle{程序}
  \begin{table}
    \caption{主流 \hologo{(La)TeX} 程序
    \footnote{(u)p\TeX{} 是日语最常用的引擎,生成 \texttt{.dvi},支持 Unicode。}\footnote{Ap\TeX{} \link{https://github.com/clerkma/ptex-ng} 具有底层 CJK 支持,内联 Ruby,Color Emoji。}}
    \footnotesize
    \begin{stampbox}
      \begin{tabular}{c>{\raggedright}*{3}{p{3.5cm}}}
        \alert{引擎}     & \hologo{pdfTeX}   & \hologo{XeTeX}   & \hologo{LuaTeX}   \\
        \alert{程序}     & \hologo{pdfLaTeX} & \hologo{XeLaTeX} & \hologo{LuaLaTeX} \\
        \alert{特点}     & 直接生成 PDF,支持 micro-typography  & 支持 Unicode、OpenType 与复杂文字编排 (CTL) & 支持 Unicode,内联 Lua,支持 OpenType \\
      \end{tabular}
    \end{stampbox}
  \end{table}

  \begin{center}
    \parbox{.9\textwidth}{
      \hologo{pdfLaTeX} 不支持 Unicode。为了排版中文,大部分情况下应当使用 \hologo{XeLaTeX},而 \hologo{LuaLaTeX} 速度相对较慢。\faWindows{} 可以在一些情况下使用 \hologo{pdfLaTeX}。
    }
  \end{center}
  \note{当然为了排版中文,已经不再推荐使用 \hologo{pdfLaTeX} 了,应该使用
  \hologo{XeLaTeX} 或者 \hologo{LuaLaTeX},当然后者的速度还是相对较慢,
  它们支持 Unicode 编码,并可以使用 OpenType 字体的全部功能。
  当然 \faWindows{} 平台下在某些追求速度的情况下,
  还是可以试着使用 \hologo{pdfLaTeX} 的。

  \hologo{LuaLaTeX} 理想情况下不慢,但是使用一些宏包后会破坏理想状态,
  也会因配置产生不同的结果,不同的操作系统在 I/O 速度上的不同也会导致不同的时间。

  \hologo{pdfLaTeX} 也支持,只不过需要先生成 tfm \TeX{} 字体度量文件,后续使用 \TeX{}
  自身的配置方法,只能使用 7 比特或 8 比特字体。}
\end{frame}

% \begin{frame}
%   \paragraph{\hologo{pdfLaTeX}} \TeX{} 和 \LaTeX{} 被广泛使用之前,它们只需内置支持欧洲语言即可。在 Unicode 出现之前,\LaTeX{} 提供了许多种\textbf{文件编码}来允许很多语言的文字以原生的方式输入,\hologo{pdfLaTeX} 也只需要使用 8 位文件编码和 8 位字体。
% \end{frame}

\section{安装和编辑}

\begin{frame}
  \frametitle{发行版}
  \begin{table}
    \caption{\hologo{TeX} 发行版}
    \footnotesize
    \begin{stampbox}
      \begin{tabular}{c>{\raggedright}*{3}{p{3.2cm}}}
        \alert{发行版}     & \hologo{MiKTeX} \link{https://miktex.org/}   & \TeX{} Live \link{https://www.tug.org/texlive/}   & Mac\TeX{} \link{https://www.tug.org/mactex/}  \\[2pt]
        \alert{特点}      &  只安装必要文件,依赖用时更新  &  所有平台均可使用,每年发布一次 & Mac 系统专用,对 \TeX{} Live 的进一步打包 \\
        \alert{推荐平台}  & \faWindows  & \faWindows\,\faLinux &  \faApple \\
      \end{tabular}
    \end{stampbox}
  \end{table}
  \begin{center}
    \parbox{.9\textwidth}{
      要让 \LaTeX{} 跑起来,核心就是要有一套 \TeX{} 发行版,来获取让 \LaTeX{} 工作所需的一组程序和文件。参考《一份简短的关于 \LaTeX{} 安装的介绍》\link{https://mirrors.sjtug.sjtu.edu.cn/ctan/info/install-latex-guide-zh-cn/install-latex-guide-zh-cn.pdf} 安装想使用的发行版。推荐使用发行版的最新版本\footnote{老版本 Linux 系统的包管理器自带 \TeX{} Live 发行版可能不是最新的 \link{https://repology.org/project/texlive/versions},尽量使用镜像安装,并手动将相关环境变量添加到路径 \link{https://www.tug.org/texlive/doc/texlive-zh-cn/texlive-zh-cn.pdf}。},并使用国内镜像。
    }
  \end{center}
  \note{要让 \LaTeX{} 跑起来,核心就是要有一套 \TeX{} 发行版,来获取让
  \LaTeX{} 工作所需的一组程序和文件。参考《一份简短的关于 \LaTeX{} 安装的介绍》
  安装想使用的发行版,里面介绍了 \faWindows{}, \faApple{}, \faLinux{}, WSL 等系统上
  \TeX{} Live 的安装,非常全面,一步一步做就可以成功安装。目前最新的 \TeX{} Live 版本为
  2022,\SJTUThesis{} 用户不应当安装 \TeX{} Live 2020 以下的版本(后面会讲)。

  事实上,我认为这几个发行版各有操作系统偏好,虽然前两者是跨平台的。

  \hologo{MiKTeX} 对 Windows 用户较为友好,安装简单,占用空间不大,安装时间短,
  而且有完整的安装与卸载程序。可以给大家看一下 \hologo{MiKTeX} Console 的情况。

  \TeX{} Live 更符合 Linux 的更新传统。老版本 Linux 系统的包管理器自带 \TeX{} Live
  发行版可能不是最新的(到时间也会锁定依赖库的版本),尽量使用镜像安装(当然也推荐使用最新的
  Linux 发行版,这样它的版本也就一直是最新的),并手动将相关环境变量添加到路径。

  Mac\TeX{} 发行版有 pkg 安装包封装,并且附带了 \TeX{}Shop 基本编辑软件,更加适合 Mac OS。
  我用下来的话,感觉除了 \TeX{} Live Utility 外都有点过时了。}
\end{frame}

\begin{frame}[plain]
  \hbox to \textwidth{
    \hfil
    \vbox to 3cm{
      \hbox{
        \resizebox{3cm}{!}{\includegraphics{support/examples/pics/sjtug}}
      }
    }
    \hfil
    \vbox to 3cm{
      \vfill
      \hbox{\Large\bfseries\color{structure} 稳定、快速、现代的镜像服务。}
      \vskip2pt
      \hbox{托管于华东教育网骨干节点上海交通大学。}
      \vfill
    }
    \hskip20pt
    \hfil
  }

  \begin{center}
    \parbox{0.8\textwidth}{
      推荐使用 SJTUG 软件镜像服务 \link{https://mirror.sjtu.edu.cn/},离得近,下得快。

      \begin{description}
        \footnotesize
        \item[\hologo{MiKTeX}] \url{https://mirrors.sjtug.sjtu.edu.cn/CTAN/systems/win32/miktex/setup/windows-x64/} \\ 并在 \hologo{MiKTeX} Console 中设置镜像源为 \url{https://mirrors.sjtug.sjtu.edu.cn}
        \item[\TeX{} Live] \url{https://mirrors.sjtug.sjtu.edu.cn/CTAN/systems/texlive/tlnet}
        \item[Mac\TeX{}] \url{https://mirrors.sjtug.sjtu.edu.cn/CTAN/systems/mac/mactex/}
        \item[\faTelegram] 可以在 SJTUG 镜像站通知频道 \link{https://t.me/sjtug_mirrors_news} 获得更多信息,加入关联群组参与讨论。
      \end{description}
    }
  \end{center}
  \note{说到镜像,像后两者的安装包都很大(4GB 左右),由于一些原因,不采用镜像的话不知道要下到什么时候,对下载速度的要求高;
  而 \hologo{MiKTeX} 需要随时更新,宏包大小颗粒度大,对延迟的要求高。
  那么采用 SJTUG 镜像源将同时解决这两个问题,位于图信大楼的机房,凭借校内的高速网络,稳定快速下载,
  现在由 LightQuantum 维护的镜像站欢迎大家的使用,主页上还有更多的其他镜像可供使用,加入 Telegram 群组参与讨论。}
\end{frame}

\begin{frame}
  \frametitle{编辑器}
  \begin{table}
    \caption{开源编辑器推荐}
    \footnotesize
    \begin{stampbox}
      \begin{tabular}{c>{\raggedright}*{3}{p{3.5cm}}}
        \alert{编辑器}     & \begin{tabular}{c}Visual Studio Code \link{https://code.visualstudio.com}\\ +\,\LaTeX{} Workshop \link{https://marketplace.visualstudio.com/items?itemName=James-Yu.latex-workshop}\end{tabular}  & \TeX{}studio \link{https://texstudio.org} & \TeX{}works \\[5pt]
        \alert{特点}      &  搭配 VS Code 使用非常方便,易扩展  & 可以使用大量的菜单选项输入代码块,用户友好 & 只提供基础的高亮与运行方法,发行版自带\footnote{Mac\TeX{} 打包的是 \TeX{}Shop 编辑器。} \\
      \end{tabular}
    \end{stampbox}
  \end{table}
  \begin{center}
    \parbox{.9\textwidth}{
      使用专为 \LaTeX{} 设计的编辑器将带来更多便利,因为它们往往会提供一键编译、内置 PDF 阅读器以及语法高亮等功能。几乎所有现代的 \LaTeX{} 编辑器都提供 Sync\TeX{} 这一强大的功能,以在 PDF 和代码间对应跳转。
    }
  \end{center}
  \note{编辑器的种类很多,我无法一一列举,但是对编写 \TeX{} 常用的开源编辑器我推荐这三个。
  其中 \TeX{}studio 的安装包可能下得有点慢。这里我对这些编辑器都演示一下,初学者我更推荐使用
  \TeX{} studio 编辑器,如果平时就码很多代码的话,我更推荐使用 VS Code 加插件这种方式。

  使用专为 \LaTeX{} 设计的编辑器将带来更多便利,因为它们往往会提供一键编译、内置 PDF 阅读器
  以及语法高亮等功能。几乎所有现代的 \LaTeX{} 编辑器都提供 Sync\TeX{} 这一强大的功能
  (VS Code 的方法是 Ctrl + 某处,Overleaf 的方法是直接双击),以在 PDF 和代码间对应跳转。
  当然如果你不喜欢使用这种 GUI 编辑器,\TeX{} 文档本身就是纯文本,对 Vim, Emacs 等终端用户
  也很友好。}
\end{frame}

\begin{frame}
  \frametitle{在线平台}
  \begin{table}
    \caption{在线协作平台推荐}
    \footnotesize
    \begin{stampbox}
      \begin{tabular}{c>{\raggedright}*{2}{p{4cm}}}
        \alert{在线平台}     & Overleaf \link{https://www.overleaf.com/}  & 交大 \LaTeX{} 助手 \link{https://latex.sjtu.edu.cn/} \\[2pt]
        \alert{特点}      & 最流行的在线平台,提供大量的模板,但国内访问慢 & 校内平台,隐私保护有保障,共享项目限制少 \\
      \end{tabular}
    \end{stampbox}
  \end{table}
  \begin{center}
    \parbox{.9\textwidth}{
      在线平台允许你直接在网页中编辑文档,无需本地安装即可在后台运行 \LaTeX{},并显示生成的 PDF。可以参照 Overleaf 官方文档学习如何使用在线平台 \link{https://www.overleaf.com/learn}。
    }
  \end{center}
  \note{当然使用在线平台省去了安装发行版的麻烦,这里列出两种在线写作平台。

  如果有数据合规需求的话,可以考虑使用由网络信息中心维护的交大 \LaTeX{} 助手,最近更新了 \TeX{} Live 2022,还是很不错的。

  当然国内还有 \TeX{} Page \link{https://www.texpage.com/},Slagger \link{https://www.slager.cn/} 等。
  一般来讲,这种平台使用的都是 Linux 操作系统,所以在排版中文的时候考虑将编译引擎更改为 \hologo{XeLaTeX},
  学习如何使用在线平台参见 Overleaf 的官方文档 \link{https://www.overleaf.com/learn}。}
\end{frame}

\section{基本要素}

\begin{frame}[fragile]%
  \frametitle{最小示例}
  \begin{columns}[c]
    \begin{column}{0.4\textwidth}
      \only<1>{
        \includepdflarge{support/examples/enminimal.pdf}
      }

      \only<2>{
        \begin{center}\highlight[structure]{文档类}\end{center}
        \cmd{documentclass} 命令加载了\textbf{文档类}。\cls{article} 是由 \LaTeX{} 提供的用于排版短文档的基本文档类。
        \begin{description}
          \footnotesize
          \item[\cls{article}] 不包含章的短文档
          \item[\cls{report}] 含有章的单面印刷文档
          \item[\cls{book}] 含有章的双面印刷文档
          \item[\cls{beamer}] 幻灯片
        \end{description}
      }

      \only<3>{
        \begin{center}\highlight[structure]{\texttt{document} 环境}\end{center}
        \cmd{begin} 和 \cmd{end} 用于创建\textbf{环境},可以多次、嵌套使用。环境用来指定一组文档元素的局部格式\footnotemark。\env{document} 环境是文档中必须有的环境,用于指示文档主体的范围。
      }
    \end{column}
    \begin{column}{0.6\textwidth}
      \begin{codeblock}[]{排版英文最小示例}
|\highlightline<2>|\documentclass{article}
|\highlightline<3>|\begin{document}
  Together for a Shared Future
|\highlightline<3>|\end{document}
      \end{codeblock}
    \end{column}
  \end{columns}

  \only<3>{\footnotetext{环境实际上是一个组,只不过通过语义化的形式预装了对应的格式命令。普通的组可以直接使用一对大括号之间的内容 \{$\cdots$\} 表示。}}
\end{frame}

\begin{frame}[fragile]%
  \frametitle{中文最小示例}
  \begin{columns}[c]
    \begin{column}{0.4\textwidth}

      \only<1>{
        \begin{center}\highlight[structure]{导言区}\end{center}
        \cmd{usepackage} 用于引入宏包,从而使用扩展功能,需要在\textbf{导言区}调用。这里使用 \pkg{ctex} 宏集以获得中文支持。
      }

    \end{column}
    \begin{column}{0.6\textwidth}
      \begin{codeblock}[]{排版中文最小示例}
\documentclass{article}
|\highlightline<1>\textbackslash{}usepackage\{ctex\}\hfill\color{structure}\% 导言区|
\begin{document}
  一起向未来

  Together for a Shared Future
\end{document}
      \end{codeblock}
    \end{column}
  \end{columns}
\end{frame}

\begin{frame}[fragile]%
  \frametitle{中文最小示例(更好版本)}
  \begin{columns}[c]
    \begin{column}{0.4\textwidth}

      \only<1>{
        \CTeX{} 建议对于之前提到的常规文档类,使用该宏集提供的四种中文文档类,以对特定类型提供额外的中文排版适配。
        \begin{center}
          \footnotesize
          \begin{tabular}{cc}
            \cls{ctexart} & \cls{ctexrep} \\
            \cls{ctexbook} & \cls{ctexbeamer} \\
          \end{tabular}
        \end{center}
      }

      \only<2>{
        \includepdflarge{support/examples/cnminimal.pdf}
      }

      \only<3>{
        \LaTeX{} 中通过空行来开启新的段落。一般情况下,\alert{不建议}在一段中强制断行(使用 \textbackslash{}\textbackslash{})。
      }

    \end{column}
    \begin{column}{0.6\textwidth}
      \begin{codeblock}[]{排版中文最小示例(更好版本)}
|\highlightline<1>|\documentclass{ctexart}
\begin{document}
|\highlightline<3>|  一起向未来
|\highlightline<3>|
|\highlightline<3>|  Together for a Shared Future
\end{document}
      \end{codeblock}
    \end{column}
  \end{columns}
\end{frame}

\section{文字格式}

\begin{frame}[fragile]%
  \frametitle{字体样式}
  \begin{columns}
    \begin{column}{0.4\textwidth}
      \only<1>{
        \includepdflarge{support/examples/fontstyle.pdf}
      }

      \only<2>{
        可以使用显式样式设定命令对小段做加粗、斜体、等宽等等的处理。

        \begin{center}
          \footnotesize
          \begin{tabular}{rl}
            \cmd{textrm} & \textrm{衬线} \\
            \cmd{textbf} & \textbf{加粗} \\
            \cmd{textit} & \kaishu 斜体 \\
            \cmd{texttt} & \texttt{等宽} \\
            \cmd{textsf} & \textsf{无衬线} \\
            \cmd{textsc} & \textsc{Small Caps} \\
            \cmd{textsl} & \textsl{Slanted} \\
          \end{tabular}
        \end{center}
      }

      \only<3>{
        也可以使用对应的更改当前组字体设置的命令,对于大段文字较为方便。

        \begin{center}
          \footnotesize
          \begin{tabular}{ll}
            \cmd{rmfamily} & \textrm{衬线} \\
            \cmd{ttfamily} & \textbf{加粗} \\
            \cmd{sffamily} & \kaishu 斜体 \\
            \cmd{bfseries} & \texttt{等宽} \\
            \cmd{itshape} & \textsf{无衬线} \\
            \cmd{scshape} & \textsc{Small Caps} \\
            \cmd{slshape} & \textsl{Slanted} \\
          \end{tabular}
        \end{center}
      }

      \only<4>{
        \LaTeX{} 建议采用语义化的逻辑标记来设置样式,以便对全文同类文字进行统一修改。比如使用 \cmd{emph} 强调文字,以及使用下面将要提到的目次命令(第 \ref{sectioning} 页)设置标题等。
      }

    \end{column}
    \begin{column}{0.6\textwidth}
      \begin{codeblock}[]{样式}
\documentclass{ctexart}
\begin{document}
|\highlightline<2>|  \textbf{一起向未来}

|\highlightline<3>|  {\sffamily
|\highlightline<3>|    一段无衬线文字
|\highlightline<3>|  }

|\highlightline<4>|  \emph{Together for a Shared Future}
\end{document}
      \end{codeblock}
    \end{column}
  \end{columns}
\end{frame}

\begin{frame}[fragile]%
  \frametitle{字体大小}
  \begin{columns}
    \begin{column}{0.4\textwidth}

      \only<1>{
        \includepdflarge{support/examples/fontsize.pdf}
      }

      \only<2>{
        同样地,你也可以显式地设定字体大小,和 \cmd{rmfamily} 类似,这会修改当前组的字体设置\footnotemark。

        \begin{center}
          \footnotesize
          \begin{tabular}{rl}
            \cmd{tiny} & \tiny 极小 \\
            \cmd{scriptsize} & \scriptsize 角标大小  \\
            \cmd{footnotesize} & \footnotesize 脚注大小 \\
            \cmd{small} & \small 小 \\
            \cmd{normalsize} & \normalsize 正常大小 \\
            \cmd{large} & \large 大 \\
            \cmd{Large} & \Large 比大更大 \\
            \cmd{LARGE} & ... \\
            \cmd{huge} & \huge 巨大 \\
            \cmd{Huge} & ... \\
          \end{tabular}
        \end{center}
      }

    \end{column}
    \begin{column}{0.6\textwidth}
      \begin{codeblock}[]{大小}
\documentclass{ctexart}
\begin{document}
|\highlightline<2>|  {\huge 一起向未来\par}
  Together for a Shared Future
\end{document}
      \end{codeblock}
    \end{column}
  \end{columns}

  \only<2>{\footnotetext{注意最后显式地使用 \cmd{par} 在改回大小前结束该段,否则会导致下一行的行间距异常!}}
\end{frame}

\section{逻辑结构}
\begin{frame}[fragile]
  \frametitle{列表}
  \begin{columns}
    \begin{column}{0.35\textwidth}
      \begin{codeblock}[]{无序列表}
\documentclass{ctexart}
\begin{document}
|\highlightline|  \begin{itemize}
    \item 第一项
    \item 第二项
    \item 第三项
|\highlightline|  \end{itemize}
\end{document}
      \end{codeblock}
    \end{column}
    \begin{column}{0.35\textwidth}
      \begin{codeblock}[]{有序列表}
\documentclass{ctexart}
\begin{document}
|\highlightline|  \begin{enumerate}
    \item 第一项
    \item 第二项
    \item 第三项
|\highlightline|  \end{enumerate}
\end{document}
      \end{codeblock}
    \end{column}
    \begin{column}{0.35\textwidth}
      \begin{codeblock}[]{描述列表}
\documentclass{ctexart}
\begin{document}
|\highlightline|  \begin{description}
    \item[|\phantom{}|第一] 文本
    \item[|\phantom{}|第二] 文本
    \item[|\phantom{}|第三] 文本
|\highlightline|  \end{description}
\end{document}
      \end{codeblock}
    \end{column}
  \end{columns}
  \note{接下来我们概览一下三种列表:无序列表、有序列表、描述列表。这些列表可以相互嵌套,但最多嵌套四层。}
\end{frame}

%更深的列表技巧,定理环境等

\begin{frame}
  \frametitle{列表}
  \begin{columns}
    \begin{column}{0.35\textwidth}
      \includepdflarge{support/examples/unordered.pdf}
    \end{column}
    \begin{column}{0.35\textwidth}
      \includepdflarge{support/examples/ordered.pdf}
    \end{column}
    \begin{column}{0.35\textwidth}
      \includepdflarge{support/examples/description.pdf}
    \end{column}
  \end{columns}
\end{frame}

\begin{frame}[fragile,label=sectioning]%
  \frametitle{目次结构}
  \begin{columns}
    \begin{column}{0.4\textwidth}
      \LaTeX{} 可以使用目次命令将文档划分层级\footnotemark,并自动设定对应字体样式和大小。
      \begin{center}
        \footnotesize
        \begin{tabular}{rll}
          命令 & 含义 & 层次 \\
          \cmd{chapter} & 章\footnotemark & \sout{0} \\
          \cmd{section} & 节 & 1 \\
          \cmd{subsection} & 小节 & 2 \\
          \cmd{subsubsection} & 小小节 & 3 \\
        \end{tabular}
      \end{center}
    \end{column}
    \begin{column}{0.6\textwidth}
      \begin{codeblock}[]{目次}
\documentclass{ctexart}
\begin{document}
|\highlightline|  \section{|\phantom{}|概念}
|\highlightline|  \subsection{\LaTeX{}}
  \LaTeX{} 是一个用以排版高质量作品的文档准备系统。
\end{document}
      \end{codeblock}
    \end{column}
  \end{columns}
  \footnotetext{章这一级只在 \cls{report} 和 \cls{book} 文档类(包括对应的中文文档类)有定义。还有不常用的 \cmd{part} (0@\cls{article}/-1@\cls{report}\&\cls{book}\&\cls{beamer}) 以及更低层次的 \cmd{paragraph} (4) 与 \cmd{subparagraph} (5)。 }
  \note{而知道层次,对我们下面生成目录有帮助。}
\end{frame}

\begin{frame}[fragile]%
  \frametitle{组织文档}
  \begin{columns}
    \begin{column}{0.4\textwidth}

      \only<1>{
        \cmd{tableofcontents} 用来生成对于目次命令的目录。如果你想设定显示到哪个层级,在这个命令前使用 \cmd{setcounter\{tocdepth\}\{层次\}}

        如果你想在目录中使用更短的标题:

            \cmd{section[短标题]\{长标题\}}

        如果你想让本目次的标题不显示在目录中:

            \cmd{section*\{目录没这个标题\}}
      }

      \only<2>{
        对于大型文档而言,使用多个文件管理源文件通常是更方便的。而 \cmd{include} 和 \cmd{input} 都以相对路径的方式插入其他 \TeX{} 文档。
        区别在于,\cmd{include} 命令会从新页开始并做一些内部调整,这基本上只对 \pkg{chapter} 这一级有用。而 \cmd{input} 会原样插入源代码。
      }

      \only<3>{
        但是 \cmd{include} 插入的文档可以使用 \cmd{includeonly} 管理当前要排印哪一部分的内容,利用所有章节辅助文件的同时,减少编译时间并专注于该部分的内容。
      }
    \end{column}
    \begin{column}{0.6\textwidth}
      \begin{codeblock}[]{主文档}
\documentclass{ctexrep}
|\highlightline<3>|\includeonly{learnlatex,sjtuthesis}
\begin{document}
|\highlightline<1>|  \tableofcontents
|\only<2-3>{\highlightline}|  % !TeX root = ..\..\latex-talk.tex

\part{学习 \LaTeX{}}
% FIXME: Part Page miniframe overflow
% FIXME: footnote fault numbering

\begin{frame}[plain]
  \vfil
  \begin{center}
    \href{https://learnlatex.org}{
      \rmfamily
      Learn\,\lower1ex\hbox{\Huge\LaTeX{}}.org
    }
  \end{center}
  \vfil
  \begin{center}
    \parbox{0.75\linewidth}{
      Learn\LaTeX{}.org\cite{learnlatex} 提供了解 \LaTeX{} 的 16 篇简短的教程,并包含了一些可以在线运行的示例,可以通过亲自动手查看实验效果。本部分主要参考由 C\TeX{}-org 提供的中文翻译版本 \link{https://github.com/CTeX-org/learnlatex.github.io/tree/zh-Hans/zh-Hans/}。
    }
  \end{center}
  \vfil
\end{frame}

{ % Start of shaded number logo

\newcommand{\shadedfont}[2][1pt]{
  % #1 (optional): shadow distance
  % #2: the text needed to be shaded
  \hbox{\rlap{\color{gray}\hskip#1#2}#2}
}
\newcounter{learnsec}
\setcounter{learnsec}{-1}
\newcommand{\updatelogo}{
  % update the logo corresponding to current counter.
  \stepcounter{learnsec}
  \logo{
    \raise.3ex\hbox{\tiny\insertsection}\shadedfont{\arabic{learnsec}}
  }
}
\let\oldsection=\section
\renewcommand{\section}[1]{\oldsection{#1}\updatelogo}

\section{是什么}
\begin{frame}
  \frametitle{\TeX{}}
  \begin{columns}[c]
    \begin{column}{0.7\textwidth}
      \begin{center}
        \rmfamily\Huge
        \hologo{La}\highlight[structure!70]{\TeX{}}
      \end{center}
      \begin{center}
        \parbox{0.75\textwidth}{
          \TeX{} 是由斯坦福大学教授高德纳
          (Donald E.~Knuth)于 1977 年开始开发的排版引擎。目前仍在更新,最新版本号为 3.141592653 \link{https://tug.org/TUGboat/tb42-1/tb130knuth-tuneup21.pdf}。
        }
      \end{center}
    \end{column}
    \begin{column}{0.3\textwidth}
      \includegraphics[width=.8\columnwidth]{Knuth.jpg}
    \end{column}
  \end{columns}
\end{frame}

\begin{frame}
  \frametitle{\LaTeX{}}
  \begin{columns}[c]
    \begin{column}{0.7\textwidth}
      \begin{center}
        \rmfamily\Huge
        \highlight[structure]{\LaTeX{}}
      \end{center}
      \begin{center}
        \parbox{0.75\textwidth}{
          \LaTeX{} 是最早在 1985 年由现就职于微软的 Leslie Lamport 开发的一种 \TeX{} \textbf{格式}\footnotemark,使用一些列宏和扩展宏包来简化 \TeX{} 的使用。现在由 \LaTeX{} Project 的成员维护。现在广泛使用的版本是 \LaTeXe{},最新的版本为 \LaTeX3(2020 年 10 月后默认内置)。
        }
      \end{center}
    \end{column}
    \begin{column}{0.3\textwidth}
      \includegraphics[width=.8\columnwidth]{Lamport.jpg}
    \end{column}
  \end{columns}
  \footnotetext{\hologo{ConTeXt} 也是一种 \TeX{} 格式 \link{https://www.contextgarden.net/}。}
\end{frame}

\begin{frame}
  \frametitle{程序}
  \begin{columns}[c]
    \begin{column}{0.7\textwidth}
      \begin{center}
        \rmfamily\Huge
        \highlight[structure]{\hologo{pdfLaTeX}}
      \end{center}
      \begin{center}
        \parbox{0.7\textwidth}{
          \hologo{pdfLaTeX} 是为了编译一个 \LaTeX{} 文档而运行的程序。实际上底层在运行一个叫 \hologo{pdfTeX} 的引擎,并预装了对应的 \LaTeX{} \textbf{格式}。为了利用临时文件,可能就需要多次运行程序。
        }
      \end{center}
    \end{column}
    \begin{column}{0.3\textwidth}
      \begin{block}{}
        \ttfamily\small
        > \highlight{pdflatex} main.tex\\
        This is pdfTeX, Version 3.141592653-
        2.6-1.40.23 (MiKTeX 21.10)\\
        entering extended mode\\
        \highlight{LaTeX2e} <2021-11-15>\\
        \highlight{L3} programming layer <2021-11-22>
      \end{block}
    \end{column}
  \end{columns}
\end{frame}

\begin{frame}
  \frametitle{引擎}
  \begin{columns}[c]
    \begin{column}{0.7\textwidth}
      \begin{center}
        \rmfamily\Huge
        \highlight[structure!70]{pdf}\hologo{La}\highlight[structure!70]{\TeX{}}
      \end{center}
      \begin{center}
        \parbox{0.7\textwidth}{
          pdf\TeX{} 是编译 \TeX{} 文档(以 \texttt{.tex} 结尾)的\textbf{引擎}---可以理解 \TeX{} 指令的\textbf{程序}。
        }
      \end{center}
    \end{column}
    \begin{column}{0.3\textwidth}
      \begin{block}{}
        \ttfamily\small
        > pdflatex main.tex\\
        This is \highlight[structure!70]{pdfTeX}, Version 3.141592653-
        2.6-1.40.23 (MiKTeX 21.10)
        entering extended mode\\
        LaTeX2e <2021-11-15>\\
        L3 programming layer <2021-11-22>
      \end{block}
    \end{column}
  \end{columns}
\end{frame}

\begin{frame}
  \frametitle{Unicode 引擎}
  \begin{table}
    \caption{主流 \hologo{(La)TeX} 程序
    \footnote{(u)p\TeX{} 是日语最常用的引擎,生成 \texttt{.dvi},支持 Unicode。}\footnote{Ap\TeX{} 具有底层 CJK 支持,内联 Ruby,Color Emoji。}}
    \footnotesize
    \begin{stampbox}
      \begin{tabular}{c>{\raggedright}*{3}{p{3.5cm}}}
        \alert{引擎}     & \hologo{pdfTeX}   & \hologo{XeTeX}   & \hologo{LuaTeX}   \\
        \alert{程序}     & \hologo{pdfLaTeX} & \hologo{XeLaTeX} & \hologo{LuaLaTeX} \\
        \alert{特点}     & 直接生成 PDF,支持 micro-typography  & 支持 Unicode、OpenType 与复杂文字编排 (CTL) & 支持 Unicode,内联 Lua,支持 OpenType \\
      \end{tabular}
    \end{stampbox}
  \end{table}

  \begin{center}
    \parbox{.9\textwidth}{
      \hologo{pdfLaTeX} 不支持 Unicode。为了排版中文,大部分情况下 \faApple{}\,\faLinux{} 应当使用 \hologo{XeLaTeX},而 \hologo{LuaLaTeX} 速度相对较慢。\faWindows{} 可以在一些情况下使用 \hologo{pdfLaTeX}。
    }
  \end{center}
\end{frame}

% \begin{frame}
%   \paragraph{\hologo{pdfLaTeX}} \TeX{} 和 \LaTeX{} 被广泛使用之前,它们只需内置支持欧洲语言即可。在 Unicode 出现之前,\LaTeX{} 提供了许多种\textbf{文件编码}来允许很多语言的文字以原生的方式输入,\hologo{pdfLaTeX} 也只需要使用 8 位文件编码和 8 位字体。
% \end{frame}

\section{跑起来}
\begin{frame}
  \frametitle{发行版}
  \begin{table}
    \caption{\hologo{TeX} 发行版}
    \footnotesize
    \begin{stampbox}
      \begin{tabular}{c>{\raggedright}*{3}{p{3.2cm}}}
        \alert{发行版}     & \hologo{MiKTeX} \link{https://mirrors.sjtug.sjtu.edu.cn/ctan/systems/win32/miktex/setup/windows-x64/basic-miktex-21.12-x64.exe}   & \TeX{} Live \link{https://mirrors.sjtug.sjtu.edu.cn/ctan/systems/texlive/tlnet/install-tl.zip}   & Mac\TeX{} \link{https://mirrors.sjtug.sjtu.edu.cn/ctan/systems/mac/mactex/mactex-20210328.pkg}  \\[2pt]
        \alert{特点}      &  只安装必要文件,依赖用时更新  &  所有平台均可使用,每年发布一次 & Mac 系统专用,对 \TeX{} Live 的进一步打包 \\
        \alert{推荐平台}  & \faWindows  & \faLinux &  \faApple \\
      \end{tabular}
    \end{stampbox}
  \end{table}
  \begin{center}
    \parbox{.9\textwidth}{
      要让 \LaTeX{} 跑起来,核心就是要有一套 \TeX{} 发行版,来获取让 \LaTeX{} 工作所需的一组程序和文件。参考《一份简短的关于 \LaTeX{} 安装的介绍》\link{https://mirrors.sjtug.sjtu.edu.cn/ctan/info/install-latex-guide-zh-cn/install-latex-guide-zh-cn.pdf} 安装想使用的发行版。推荐使用发行版的最新版本\footnote{老版本 Linux 系统的包管理器自带 \TeX{} Live 发行版可能不是最新的 \link{https://repology.org/project/texlive/versions},尽量使用镜像安装,并手动将相关环境变量添加到路径 \link{https://www.tug.org/texlive/doc/texlive-zh-cn/texlive-zh-cn.pdf}。},并使用国内镜像。
    }
  \end{center}
\end{frame}

\begin{frame}[plain]
  \hbox to \textwidth{
    \hfil
    \vbox to 3cm{
      \hbox{
        \resizebox{3cm}{!}{\includegraphics{\getcontribpath{sjtug}{vi/sjtug.pdf}}}
      }
    }
    \hfil
    \vbox to 3cm{
      \vfill
      \hbox{\Large\bfseries\color{cprimary} 稳定、快速、现代的镜像服务。}
      \vskip2pt
      \hbox{托管于华东教育网骨干节点上海交通大学。}
      \vfill
    }
    \hskip20pt
    \hfil
  }

  \begin{center}
    \parbox{0.8\textwidth}{
      推荐使用 SJTUG 软件镜像服务,离得近,下得快。
      
      \begin{description}
        \footnotesize
        \item[\TeX{} Live]  {\ttfamily tlmgr option repository https://mirrors.sjtug.sjtu.edu.cn/CTAN/systems/texlive/tlnet}
        \item[\hologo{MiKTeX}] 在 \hologo{MiKTeX} Console 中设置镜像源为 \url{https://mirrors.sjtug.sjtu.edu.cn}
      \end{description}
    }
  \end{center}
\end{frame}

\begin{frame}
  \frametitle{编辑器}
  \begin{table}
    \caption{开源编辑器推荐}
    \footnotesize
    \begin{stampbox}
      \begin{tabular}{c>{\raggedright}*{3}{p{3.5cm}}}
        \alert{编辑器}     & \begin{tabular}{c}Visual Studio Code\\ \LaTeX{} Workshop\end{tabular}  & \TeX{}studio & \TeX{}works \\[5pt]
        \alert{特点}      &  搭配 VS Code 使用非常方便,易扩展  & 可以使用大量的菜单选项输入代码块,用户友好 & 只提供基础的高亮与运行方法,发行版自带\footnote{Mac\TeX{} 打包的是 \TeX{}Shop 编辑器。} \\
      \end{tabular}
    \end{stampbox}
  \end{table}
  \begin{center}
    \parbox{.9\textwidth}{
      使用专为 \LaTeX{} 设计的编辑器将带来更多便利,因为它们往往会提供一键编译、内置 PDF 阅读器以及语法高亮等功能。几乎所有现代的 \LaTeX{} 编辑器都提供 Sync\TeX{} 这一强大的功能,以在 PDF 和 代码间对应跳转。
    }
  \end{center}
\end{frame}

\begin{frame}
  \frametitle{在线平台}
  \begin{table}
    \caption{在线协作平台推荐}
    \footnotesize
    \begin{stampbox}
      \begin{tabular}{c>{\raggedright}*{2}{p{4cm}}}
        \alert{在线平台}     & Overleaf \link{https://www.overleaf.com/}  & 交大 \LaTeX{} 助手 \link{https://latex.sjtu.edu.cn/} \\[2pt]
        \alert{特点}      & 最流行的在线平台,提供大量的模板,但国内访问慢 & 校内平台,隐私保护有保障,共享项目限制少 \\
      \end{tabular}
    \end{stampbox}
  \end{table}
  \begin{center}
    \parbox{.9\textwidth}{
      在线平台允许你直接在网页中编辑文档,无需本地安装即可在后台运行 \LaTeX{},并显示生成的 PDF。可以参照 Overleaf 官方文档学习如何使用在线平台 \link{https://www.overleaf.com/learn}。
    }
  \end{center}
\end{frame}

\section{基本结构}
\begin{frame}[fragile]%
  \frametitle{文档部件}
  \begin{columns}[c]
    \begin{column}{0.4\textwidth}
      \only<1>{
        \cmd{documentclass} 命令加载了\textbf{文档类}。\pkg{article} 是由 \LaTeX{}提供的用于排版短文档的基本文档类。
        \begin{description}
          \footnotesize
          \item[\pkg{article}] 不包含章的短文档
          \item[\pkg{report}] 含有章的单面印刷文档
          \item[\pkg{book}] 含有章的双面印刷文档
          \item[\pkg{beamer}] 制作幻灯片
        \end{description}
      }
      \only<2>{
        \env{document} 环境用于指示文档主体的范围。\LaTeX{} 还有其他的使用 \cmd{begin} 和 \cmd{end} 的搭配,我们称这些为\textbf{环境}。它们将用来设定局部格式命令\footnotemark。
      }
      \only<3>{
        \includepdflarge{enminimal}
      }
    \end{column}
    \begin{column}{0.6\textwidth}
      \begin{codeblock}[]{排版英文最简示例}
|\only<1>{\highlightline}|\documentclass{article}
|\only<2>{\highlightline}|\begin{document}
|\only<3>{\highlightline}|  Together for a Shared Future
|\only<2>{\highlightline}|\end{document}
      \end{codeblock}
    \end{column}
  \end{columns}
  \only<2>{\footnotetext{环境实际上是一个组,只不过通过语义化的形式预装了对应的格式命令。普通的组可以直接使用一对大括号之间的内容 \{$\cdots$\} 表示。}}
\end{frame}

\section{扩展}
\begin{frame}[fragile]%
  \frametitle{中文排版}
  \begin{columns}[c]
    \begin{column}{0.4\textwidth}
      \only<1>{
        \cmd{usepackage} 用于使用宏包以向 \LaTeX{} 添加或修改功能,需要在\textbf{导言区}调用。
        这里使用 \pkg{ctex} 宏集以获得中文支持。其调用底层因随不同的引擎而不同。
        {
          \footnotesize
          \begin{stampbox}
            \begin{tabular}{c*{3}{c}}
              \alert{引擎}     & \hologo{pdfTeX}   & \hologo{XeTeX}   & \hologo{LuaTeX}   \\
              \alert{程序}     & \hologo{pdfLaTeX} & \hologo{XeLaTeX} & \hologo{LuaLaTeX} \\
              \alert{宏包}     & CJK\footnotemark & xeCJK & luatexja \\
              \alert{封装}     & \multicolumn{3}{c}{ctex} \\
            \end{tabular}
          \end{stampbox}
        }
        \vspace{-1cm}
      }
      \only<2>{
        C\TeX{} 建议对于之前提到的常规文档类,最佳实践是使用该宏集提供的四种中文文档类,以对特定类型提供额外的中文排版适配。
        \begin{center}
          \begin{stampbox}
            \footnotesize
            \begin{tabular}{cc}
              \pkg{ctexart} & \pkg{ctexrep} \\
              \pkg{ctexbook} & \pkg{ctexbeamer} \\
            \end{tabular}
          \end{stampbox}
        \end{center}
      }
      \only<3>{
        \includepdflarge{cnminimal}
      }
      \only<4>{
        大部分情况下,你都不应当在 \LaTeX{} 中强制断行:你几乎只是想另起一段,或者是想在段落之间添加空行(使用 \pkg{parskip} 宏包就可实现)。
        只有\alert{很少的}情况下你需要使用 \textbackslash{}\textbackslash{} 来另起一行而不另起一段。
      }
    \end{column}
    \begin{column}{0.6\textwidth}
      \begin{codeblock}[]{排版中文\only<2->{最佳实践}}
|\only<2>{\highlightline}|\documentclass{|\only<1>{article}\only<2->{ctexart}|}
|\only<1>{\highlightline\textbackslash{}usepackage\{ctex\}\hfill\color{cprimary}\% 导言区}|
\begin{document}
|\only<3>{\highlightline}|    一起向未来
|\only<4>{\highlightline}|
  Together for a Shared Future
\end{document}
      \end{codeblock}
    \end{column}
  \end{columns}
  \only<1>{\footnotetext{ctex 在 \faApple\,\faLinux{} 上已经不可以使用 \hologo{pdfLaTeX} 编译,以及在 \faWindows{} 上使用该引擎也会变更自动间距调整等功能的默认行为。}}
\end{frame}

\section{设定格式}
\begin{frame}[fragile]%
  \frametitle{字体样式}
  \begin{columns}
    \begin{column}{0.4\textwidth}
      \only<1>{
        \includepdflarge{fontstyle}
      }
      \only<2>{
        可以使用显示样式设定命令对小段做加粗、斜体、等宽等等的处理。
        \begin{center}
          \footnotesize
          \begin{stampbox}
            \begin{tabular}{rl}
              \cmd{textrm} & \textrm{衬线} \\
              \cmd{textbf} & \textbf{加粗} \\
              \cmd{textit} & \kaishu 斜体 \\
              \cmd{texttt} & \texttt{等宽} \\
              \cmd{textsf} & \textsf{无衬线} \\
              \cmd{textsc} & \textsc{Small Caps} \\
              \cmd{textsl} & \textsl{Slanted} \\
            \end{tabular}
          \end{stampbox}
        \end{center}
      }
      \only<3>{
        与 Word 不同的是,\LaTeX{} 一般情况下并不需要使用上面的显式命令,而是采用逻辑标记的方法,
        比如 \cmd{emph} 可以强调文字,以及下面将要提到的目次命令(第 \ref{sectioning} 页)。
        这样可以统一管理格式。
      }
    \end{column}
    \begin{column}{0.6\textwidth}
      \begin{codeblock}[]{样式}
\documentclass{ctexart}
\begin{document}
|\only<2>{\highlightline}|  \textbf{||一起向未来}

|\only<3>{\highlightline}|  \emph{Together for a Shared Future}
\end{document}
      \end{codeblock}
    \end{column}
  \end{columns}
\end{frame}

\begin{frame}[fragile]%
  \frametitle{\only<1-2>{字体大小}\only<3>{字体样式}}
  \begin{columns}
    \begin{column}{0.4\textwidth}
      \only<1>{
        \includepdflarge{fontsize}
      }
      \only<2>{
        同样地,你也可以显式地设定字体大小,但是这种命令会更改行文设置,所以需要使用一个组来限定作用范围\footnotemark。
        \begin{center}
          \footnotesize
          \begin{stampbox}
            \begin{tabular}{rl}
              \cmd{tiny} & \tiny 极小 \\
              \cmd{scriptsize} & \scriptsize 抄本大小  \\
              \cmd{footnotesize} & \footnotesize 脚注大小 \\
              \cmd{small} & \small 小 \\
              \cmd{normalsize} & \normalsize 正常大小 \\
              \cmd{large} & \large 大 \\
              \cmd{huge} & \Huge 巨大 \\
            \end{tabular}
          \end{stampbox}
        \end{center}
      }
      \only<3>{
        也可以使用字体样式对应的更改字体设置的命令,这对于大段文段的设置而言也是很方便的。
        \begin{center}
          \footnotesize
          \begin{stampbox}
            \begin{tabular}{ll}
              \cmd{textrm} & \cmd{rmfamily}\\
              \cmd{texttt} & \cmd{ttfamily}\\
              \cmd{textsf} & \cmd{sffamily}\\
              \cmd{textbf} & \cmd{bfseries}\\
              \cmd{textit} & \cmd{itshape}\\
              \cmd{textsc} & \cmd{scshape}\\
              \cmd{textsl} & \cmd{slshape}\\
            \end{tabular}
          \end{stampbox}
        \end{center}
      }
    \end{column}
    \begin{column}{0.6\textwidth}
      \begin{codeblock}[]{大小}
\documentclass{ctexart}
\begin{document}
|\only<2>{\highlightline}|  {\bfseries\Large 一起向未来\par}
|\only<3>{\highlightline}|  {\itshape Together for a Shared Future}
\end{document}
      \end{codeblock}
    \end{column}
  \end{columns}
  \only<2>{\footnotetext{注意最后显式地使用 \cmd{par} 在改回大小前结束该段,否则会导致下一行的行间距异常!}}
\end{frame}

\section{逻辑结构}
\begin{frame}[fragile]
  \frametitle{列表}
  \begin{columns}
    \begin{column}{0.35\textwidth}
      \begin{codeblock}[]{无序列表}
\documentclass{ctexart}
\begin{document}
|\highlightline|  \begin{itemize}
    \item 第一项
    \item 第二项
    \item 第三项
|\highlightline|  \end{itemize}
\end{document}
      \end{codeblock}
    \end{column}
    \begin{column}{0.35\textwidth}
      \begin{codeblock}[]{有序列表}
\documentclass{ctexart}
\begin{document}
|\highlightline|  \begin{enumerate}
    \item 第一项
    \item 第二项
    \item 第三项
|\highlightline|  \end{enumerate}
\end{document}
      \end{codeblock}
    \end{column}
    \begin{column}{0.35\textwidth}
      \begin{codeblock}[]{描述列表}
\documentclass{ctexart}
\begin{document}
|\highlightline|  \begin{description}
    \item[||第一] 文本
    \item[||第二] 文本
    \item[||第三] 文本  
|\highlightline|  \end{description}
\end{document}
      \end{codeblock}
    \end{column}
  \end{columns}
\end{frame}

%更深的列表技巧,定理环境等

\begin{frame}
  \frametitle{列表}
  \begin{columns}
    \begin{column}{0.35\textwidth}
      \includepdflarge{unordered}
    \end{column}
    \begin{column}{0.35\textwidth}
      \includepdflarge{ordered}
    \end{column}
    \begin{column}{0.35\textwidth}
      \includepdflarge{description}
    \end{column}
  \end{columns}
\end{frame}

\begin{frame}[fragile,label=sectioning]%
  \frametitle{目次结构}
  \begin{columns}
    \begin{column}{0.4\textwidth}
      \LaTeX{} 可以使用目次命令将文档划分层级\footnotemark,并自动设定对应字体样式和大小。
      \begin{center}
        \begin{stampbox}
          \footnotesize
          \begin{tabular}{rll}
           命令 & 中文 & 层次 \\
           \cmd{chapter} & 章\footnotemark & \sout{0} \\
           \cmd{section} & 节 & 1 \\
           \cmd{subsection} & 小节 & 2 \\
           \cmd{subsubsection} & 小小节 & 3 \\
          \end{tabular}
        \end{stampbox}
      \end{center}
    \end{column}
    \begin{column}{0.6\textwidth}
      \begin{codeblock}[]{目次}
\documentclass{ctexart}
\begin{document}
|\highlightline|  \section{||概念}
|\highlightline|  \subsection{\LaTeX{}}
  \LaTeX{} 是一个用以排版高质量作品的文档准备系统。
\end{document}
      \end{codeblock}
    \end{column}
  \end{columns}
  \footnotetext{章这一级只在 \pkg{report} 和 \pkg{book} 文档类(包括对应的中文文档类)有定义。还有不常用的 \cmd{part} (0@\pkg{article}/-1@\pkg{report}\&\pkg{book}\&\pkg{beamer}) 以及更低层次的 \cmd{paragraph} (4) 与 \cmd{subparagraph} (5)。 }
\end{frame}

\begin{frame}[fragile]%
  \frametitle{组织文档}
  \begin{columns}
    \begin{column}{0.4\textwidth}
      \only<1>{
        \cmd{tableofcontents} 用来生成对于目次命令的目录。如果你想设定显示到哪个层级,在这个命令前使用 \cmd{setcounter\{tocdepth\}\{层次\}}
      }
      \only<2>{
        对于大型文档而言,使用多个文件管理源文件通常是更方便的。而 \cmd{include} 和 \cmd{input} 都以相对路径的方式插入其他 \TeX{} 文档。
        区别在于,\cmd{include} 命令会从新页开始并做一些内部调整,这基本上只对 \pkg{chapter} 这一级有用。而 \cmd{input} 会原样插入源代码。
      }
      \only<3>{
        但是 \cmd{include} 插入的文档可以使用 \cmd{includeonly} 管理当前要排印哪一部分的内容,利用所有章节辅助文件的同时,减少编译时间并专注于该部分的内容。
      }
    \end{column}
    \begin{column}{0.6\textwidth}
      \begin{codeblock}[]{主文档}
\documentclass{ctexrep}
|\only<3>{\highlightline}|\includeonly{learnlatex,sjtuthesis}
\begin{document}
|\only<1>{\highlightline}|  \tableofcontents
|\only<2-3>{\highlightline}|  % !TeX root = ..\..\latex-talk.tex

\part{学习 \LaTeX{}}
% FIXME: Part Page miniframe overflow
% FIXME: footnote fault numbering

\begin{frame}[plain]
  \vfil
  \begin{center}
    \href{https://learnlatex.org}{
      \rmfamily
      Learn\,\lower1ex\hbox{\Huge\LaTeX{}}.org
    }
  \end{center}
  \vfil
  \begin{center}
    \parbox{0.75\linewidth}{
      Learn\LaTeX{}.org\cite{learnlatex} 提供了解 \LaTeX{} 的 16 篇简短的教程,并包含了一些可以在线运行的示例,可以通过亲自动手查看实验效果。本部分主要参考由 C\TeX{}-org 提供的中文翻译版本 \link{https://github.com/CTeX-org/learnlatex.github.io/tree/zh-Hans/zh-Hans/}。
    }
  \end{center}
  \vfil
\end{frame}

{ % Start of shaded number logo

\newcommand{\shadedfont}[2][1pt]{
  % #1 (optional): shadow distance
  % #2: the text needed to be shaded
  \hbox{\rlap{\color{gray}\hskip#1#2}#2}
}
\newcounter{learnsec}
\setcounter{learnsec}{-1}
\newcommand{\updatelogo}{
  % update the logo corresponding to current counter.
  \stepcounter{learnsec}
  \logo{
    \raise.3ex\hbox{\tiny\insertsection}\shadedfont{\arabic{learnsec}}
  }
}
\let\oldsection=\section
\renewcommand{\section}[1]{\oldsection{#1}\updatelogo}

\section{是什么}
\begin{frame}
  \frametitle{\TeX{}}
  \begin{columns}[c]
    \begin{column}{0.7\textwidth}
      \begin{center}
        \rmfamily\Huge
        \hologo{La}\highlight[structure!70]{\TeX{}}
      \end{center}
      \begin{center}
        \parbox{0.75\textwidth}{
          \TeX{} 是由斯坦福大学教授高德纳
          (Donald E.~Knuth)于 1977 年开始开发的排版引擎。目前仍在更新,最新版本号为 3.141592653 \link{https://tug.org/TUGboat/tb42-1/tb130knuth-tuneup21.pdf}。
        }
      \end{center}
    \end{column}
    \begin{column}{0.3\textwidth}
      \includegraphics[width=.8\columnwidth]{Knuth.jpg}
    \end{column}
  \end{columns}
\end{frame}

\begin{frame}
  \frametitle{\LaTeX{}}
  \begin{columns}[c]
    \begin{column}{0.7\textwidth}
      \begin{center}
        \rmfamily\Huge
        \highlight[structure]{\LaTeX{}}
      \end{center}
      \begin{center}
        \parbox{0.75\textwidth}{
          \LaTeX{} 是最早在 1985 年由现就职于微软的 Leslie Lamport 开发的一种 \TeX{} \textbf{格式}\footnotemark,使用一些列宏和扩展宏包来简化 \TeX{} 的使用。现在由 \LaTeX{} Project 的成员维护。现在广泛使用的版本是 \LaTeXe{},最新的版本为 \LaTeX3(2020 年 10 月后默认内置)。
        }
      \end{center}
    \end{column}
    \begin{column}{0.3\textwidth}
      \includegraphics[width=.8\columnwidth]{Lamport.jpg}
    \end{column}
  \end{columns}
  \footnotetext{\hologo{ConTeXt} 也是一种 \TeX{} 格式 \link{https://www.contextgarden.net/}。}
\end{frame}

\begin{frame}
  \frametitle{程序}
  \begin{columns}[c]
    \begin{column}{0.7\textwidth}
      \begin{center}
        \rmfamily\Huge
        \highlight[structure]{\hologo{pdfLaTeX}}
      \end{center}
      \begin{center}
        \parbox{0.7\textwidth}{
          \hologo{pdfLaTeX} 是为了编译一个 \LaTeX{} 文档而运行的程序。实际上底层在运行一个叫 \hologo{pdfTeX} 的引擎,并预装了对应的 \LaTeX{} \textbf{格式}。为了利用临时文件,可能就需要多次运行程序。
        }
      \end{center}
    \end{column}
    \begin{column}{0.3\textwidth}
      \begin{block}{}
        \ttfamily\small
        > \highlight{pdflatex} main.tex\\
        This is pdfTeX, Version 3.141592653-
        2.6-1.40.23 (MiKTeX 21.10)\\
        entering extended mode\\
        \highlight{LaTeX2e} <2021-11-15>\\
        \highlight{L3} programming layer <2021-11-22>
      \end{block}
    \end{column}
  \end{columns}
\end{frame}

\begin{frame}
  \frametitle{引擎}
  \begin{columns}[c]
    \begin{column}{0.7\textwidth}
      \begin{center}
        \rmfamily\Huge
        \highlight[structure!70]{pdf}\hologo{La}\highlight[structure!70]{\TeX{}}
      \end{center}
      \begin{center}
        \parbox{0.7\textwidth}{
          pdf\TeX{} 是编译 \TeX{} 文档(以 \texttt{.tex} 结尾)的\textbf{引擎}---可以理解 \TeX{} 指令的\textbf{程序}。
        }
      \end{center}
    \end{column}
    \begin{column}{0.3\textwidth}
      \begin{block}{}
        \ttfamily\small
        > pdflatex main.tex\\
        This is \highlight[structure!70]{pdfTeX}, Version 3.141592653-
        2.6-1.40.23 (MiKTeX 21.10)
        entering extended mode\\
        LaTeX2e <2021-11-15>\\
        L3 programming layer <2021-11-22>
      \end{block}
    \end{column}
  \end{columns}
\end{frame}

\begin{frame}
  \frametitle{Unicode 引擎}
  \begin{table}
    \caption{主流 \hologo{(La)TeX} 程序
    \footnote{(u)p\TeX{} 是日语最常用的引擎,生成 \texttt{.dvi},支持 Unicode。}\footnote{Ap\TeX{} 具有底层 CJK 支持,内联 Ruby,Color Emoji。}}
    \footnotesize
    \begin{stampbox}
      \begin{tabular}{c>{\raggedright}*{3}{p{3.5cm}}}
        \alert{引擎}     & \hologo{pdfTeX}   & \hologo{XeTeX}   & \hologo{LuaTeX}   \\
        \alert{程序}     & \hologo{pdfLaTeX} & \hologo{XeLaTeX} & \hologo{LuaLaTeX} \\
        \alert{特点}     & 直接生成 PDF,支持 micro-typography  & 支持 Unicode、OpenType 与复杂文字编排 (CTL) & 支持 Unicode,内联 Lua,支持 OpenType \\
      \end{tabular}
    \end{stampbox}
  \end{table}

  \begin{center}
    \parbox{.9\textwidth}{
      \hologo{pdfLaTeX} 不支持 Unicode。为了排版中文,大部分情况下 \faApple{}\,\faLinux{} 应当使用 \hologo{XeLaTeX},而 \hologo{LuaLaTeX} 速度相对较慢。\faWindows{} 可以在一些情况下使用 \hologo{pdfLaTeX}。
    }
  \end{center}
\end{frame}

% \begin{frame}
%   \paragraph{\hologo{pdfLaTeX}} \TeX{} 和 \LaTeX{} 被广泛使用之前,它们只需内置支持欧洲语言即可。在 Unicode 出现之前,\LaTeX{} 提供了许多种\textbf{文件编码}来允许很多语言的文字以原生的方式输入,\hologo{pdfLaTeX} 也只需要使用 8 位文件编码和 8 位字体。
% \end{frame}

\section{跑起来}
\begin{frame}
  \frametitle{发行版}
  \begin{table}
    \caption{\hologo{TeX} 发行版}
    \footnotesize
    \begin{stampbox}
      \begin{tabular}{c>{\raggedright}*{3}{p{3.2cm}}}
        \alert{发行版}     & \hologo{MiKTeX} \link{https://mirrors.sjtug.sjtu.edu.cn/ctan/systems/win32/miktex/setup/windows-x64/basic-miktex-21.12-x64.exe}   & \TeX{} Live \link{https://mirrors.sjtug.sjtu.edu.cn/ctan/systems/texlive/tlnet/install-tl.zip}   & Mac\TeX{} \link{https://mirrors.sjtug.sjtu.edu.cn/ctan/systems/mac/mactex/mactex-20210328.pkg}  \\[2pt]
        \alert{特点}      &  只安装必要文件,依赖用时更新  &  所有平台均可使用,每年发布一次 & Mac 系统专用,对 \TeX{} Live 的进一步打包 \\
        \alert{推荐平台}  & \faWindows  & \faLinux &  \faApple \\
      \end{tabular}
    \end{stampbox}
  \end{table}
  \begin{center}
    \parbox{.9\textwidth}{
      要让 \LaTeX{} 跑起来,核心就是要有一套 \TeX{} 发行版,来获取让 \LaTeX{} 工作所需的一组程序和文件。参考《一份简短的关于 \LaTeX{} 安装的介绍》\link{https://mirrors.sjtug.sjtu.edu.cn/ctan/info/install-latex-guide-zh-cn/install-latex-guide-zh-cn.pdf} 安装想使用的发行版。推荐使用发行版的最新版本\footnote{老版本 Linux 系统的包管理器自带 \TeX{} Live 发行版可能不是最新的 \link{https://repology.org/project/texlive/versions},尽量使用镜像安装,并手动将相关环境变量添加到路径 \link{https://www.tug.org/texlive/doc/texlive-zh-cn/texlive-zh-cn.pdf}。},并使用国内镜像。
    }
  \end{center}
\end{frame}

\begin{frame}[plain]
  \hbox to \textwidth{
    \hfil
    \vbox to 3cm{
      \hbox{
        \resizebox{3cm}{!}{\includegraphics{\getcontribpath{sjtug}{vi/sjtug.pdf}}}
      }
    }
    \hfil
    \vbox to 3cm{
      \vfill
      \hbox{\Large\bfseries\color{cprimary} 稳定、快速、现代的镜像服务。}
      \vskip2pt
      \hbox{托管于华东教育网骨干节点上海交通大学。}
      \vfill
    }
    \hskip20pt
    \hfil
  }

  \begin{center}
    \parbox{0.8\textwidth}{
      推荐使用 SJTUG 软件镜像服务,离得近,下得快。
      
      \begin{description}
        \footnotesize
        \item[\TeX{} Live]  {\ttfamily tlmgr option repository https://mirrors.sjtug.sjtu.edu.cn/CTAN/systems/texlive/tlnet}
        \item[\hologo{MiKTeX}] 在 \hologo{MiKTeX} Console 中设置镜像源为 \url{https://mirrors.sjtug.sjtu.edu.cn}
      \end{description}
    }
  \end{center}
\end{frame}

\begin{frame}
  \frametitle{编辑器}
  \begin{table}
    \caption{开源编辑器推荐}
    \footnotesize
    \begin{stampbox}
      \begin{tabular}{c>{\raggedright}*{3}{p{3.5cm}}}
        \alert{编辑器}     & \begin{tabular}{c}Visual Studio Code\\ \LaTeX{} Workshop\end{tabular}  & \TeX{}studio & \TeX{}works \\[5pt]
        \alert{特点}      &  搭配 VS Code 使用非常方便,易扩展  & 可以使用大量的菜单选项输入代码块,用户友好 & 只提供基础的高亮与运行方法,发行版自带\footnote{Mac\TeX{} 打包的是 \TeX{}Shop 编辑器。} \\
      \end{tabular}
    \end{stampbox}
  \end{table}
  \begin{center}
    \parbox{.9\textwidth}{
      使用专为 \LaTeX{} 设计的编辑器将带来更多便利,因为它们往往会提供一键编译、内置 PDF 阅读器以及语法高亮等功能。几乎所有现代的 \LaTeX{} 编辑器都提供 Sync\TeX{} 这一强大的功能,以在 PDF 和 代码间对应跳转。
    }
  \end{center}
\end{frame}

\begin{frame}
  \frametitle{在线平台}
  \begin{table}
    \caption{在线协作平台推荐}
    \footnotesize
    \begin{stampbox}
      \begin{tabular}{c>{\raggedright}*{2}{p{4cm}}}
        \alert{在线平台}     & Overleaf \link{https://www.overleaf.com/}  & 交大 \LaTeX{} 助手 \link{https://latex.sjtu.edu.cn/} \\[2pt]
        \alert{特点}      & 最流行的在线平台,提供大量的模板,但国内访问慢 & 校内平台,隐私保护有保障,共享项目限制少 \\
      \end{tabular}
    \end{stampbox}
  \end{table}
  \begin{center}
    \parbox{.9\textwidth}{
      在线平台允许你直接在网页中编辑文档,无需本地安装即可在后台运行 \LaTeX{},并显示生成的 PDF。可以参照 Overleaf 官方文档学习如何使用在线平台 \link{https://www.overleaf.com/learn}。
    }
  \end{center}
\end{frame}

\section{基本结构}
\begin{frame}[fragile]%
  \frametitle{文档部件}
  \begin{columns}[c]
    \begin{column}{0.4\textwidth}
      \only<1>{
        \cmd{documentclass} 命令加载了\textbf{文档类}。\pkg{article} 是由 \LaTeX{}提供的用于排版短文档的基本文档类。
        \begin{description}
          \footnotesize
          \item[\pkg{article}] 不包含章的短文档
          \item[\pkg{report}] 含有章的单面印刷文档
          \item[\pkg{book}] 含有章的双面印刷文档
          \item[\pkg{beamer}] 制作幻灯片
        \end{description}
      }
      \only<2>{
        \env{document} 环境用于指示文档主体的范围。\LaTeX{} 还有其他的使用 \cmd{begin} 和 \cmd{end} 的搭配,我们称这些为\textbf{环境}。它们将用来设定局部格式命令\footnotemark。
      }
      \only<3>{
        \includepdflarge{enminimal}
      }
    \end{column}
    \begin{column}{0.6\textwidth}
      \begin{codeblock}[]{排版英文最简示例}
|\only<1>{\highlightline}|\documentclass{article}
|\only<2>{\highlightline}|\begin{document}
|\only<3>{\highlightline}|  Together for a Shared Future
|\only<2>{\highlightline}|\end{document}
      \end{codeblock}
    \end{column}
  \end{columns}
  \only<2>{\footnotetext{环境实际上是一个组,只不过通过语义化的形式预装了对应的格式命令。普通的组可以直接使用一对大括号之间的内容 \{$\cdots$\} 表示。}}
\end{frame}

\section{扩展}
\begin{frame}[fragile]%
  \frametitle{中文排版}
  \begin{columns}[c]
    \begin{column}{0.4\textwidth}
      \only<1>{
        \cmd{usepackage} 用于使用宏包以向 \LaTeX{} 添加或修改功能,需要在\textbf{导言区}调用。
        这里使用 \pkg{ctex} 宏集以获得中文支持。其调用底层因随不同的引擎而不同。
        {
          \footnotesize
          \begin{stampbox}
            \begin{tabular}{c*{3}{c}}
              \alert{引擎}     & \hologo{pdfTeX}   & \hologo{XeTeX}   & \hologo{LuaTeX}   \\
              \alert{程序}     & \hologo{pdfLaTeX} & \hologo{XeLaTeX} & \hologo{LuaLaTeX} \\
              \alert{宏包}     & CJK\footnotemark & xeCJK & luatexja \\
              \alert{封装}     & \multicolumn{3}{c}{ctex} \\
            \end{tabular}
          \end{stampbox}
        }
        \vspace{-1cm}
      }
      \only<2>{
        C\TeX{} 建议对于之前提到的常规文档类,最佳实践是使用该宏集提供的四种中文文档类,以对特定类型提供额外的中文排版适配。
        \begin{center}
          \begin{stampbox}
            \footnotesize
            \begin{tabular}{cc}
              \pkg{ctexart} & \pkg{ctexrep} \\
              \pkg{ctexbook} & \pkg{ctexbeamer} \\
            \end{tabular}
          \end{stampbox}
        \end{center}
      }
      \only<3>{
        \includepdflarge{cnminimal}
      }
      \only<4>{
        大部分情况下,你都不应当在 \LaTeX{} 中强制断行:你几乎只是想另起一段,或者是想在段落之间添加空行(使用 \pkg{parskip} 宏包就可实现)。
        只有\alert{很少的}情况下你需要使用 \textbackslash{}\textbackslash{} 来另起一行而不另起一段。
      }
    \end{column}
    \begin{column}{0.6\textwidth}
      \begin{codeblock}[]{排版中文\only<2->{最佳实践}}
|\only<2>{\highlightline}|\documentclass{|\only<1>{article}\only<2->{ctexart}|}
|\only<1>{\highlightline\textbackslash{}usepackage\{ctex\}\hfill\color{cprimary}\% 导言区}|
\begin{document}
|\only<3>{\highlightline}|    一起向未来
|\only<4>{\highlightline}|
  Together for a Shared Future
\end{document}
      \end{codeblock}
    \end{column}
  \end{columns}
  \only<1>{\footnotetext{ctex 在 \faApple\,\faLinux{} 上已经不可以使用 \hologo{pdfLaTeX} 编译,以及在 \faWindows{} 上使用该引擎也会变更自动间距调整等功能的默认行为。}}
\end{frame}

\section{设定格式}
\begin{frame}[fragile]%
  \frametitle{字体样式}
  \begin{columns}
    \begin{column}{0.4\textwidth}
      \only<1>{
        \includepdflarge{fontstyle}
      }
      \only<2>{
        可以使用显示样式设定命令对小段做加粗、斜体、等宽等等的处理。
        \begin{center}
          \footnotesize
          \begin{stampbox}
            \begin{tabular}{rl}
              \cmd{textrm} & \textrm{衬线} \\
              \cmd{textbf} & \textbf{加粗} \\
              \cmd{textit} & \kaishu 斜体 \\
              \cmd{texttt} & \texttt{等宽} \\
              \cmd{textsf} & \textsf{无衬线} \\
              \cmd{textsc} & \textsc{Small Caps} \\
              \cmd{textsl} & \textsl{Slanted} \\
            \end{tabular}
          \end{stampbox}
        \end{center}
      }
      \only<3>{
        与 Word 不同的是,\LaTeX{} 一般情况下并不需要使用上面的显式命令,而是采用逻辑标记的方法,
        比如 \cmd{emph} 可以强调文字,以及下面将要提到的目次命令(第 \ref{sectioning} 页)。
        这样可以统一管理格式。
      }
    \end{column}
    \begin{column}{0.6\textwidth}
      \begin{codeblock}[]{样式}
\documentclass{ctexart}
\begin{document}
|\only<2>{\highlightline}|  \textbf{||一起向未来}

|\only<3>{\highlightline}|  \emph{Together for a Shared Future}
\end{document}
      \end{codeblock}
    \end{column}
  \end{columns}
\end{frame}

\begin{frame}[fragile]%
  \frametitle{\only<1-2>{字体大小}\only<3>{字体样式}}
  \begin{columns}
    \begin{column}{0.4\textwidth}
      \only<1>{
        \includepdflarge{fontsize}
      }
      \only<2>{
        同样地,你也可以显式地设定字体大小,但是这种命令会更改行文设置,所以需要使用一个组来限定作用范围\footnotemark。
        \begin{center}
          \footnotesize
          \begin{stampbox}
            \begin{tabular}{rl}
              \cmd{tiny} & \tiny 极小 \\
              \cmd{scriptsize} & \scriptsize 抄本大小  \\
              \cmd{footnotesize} & \footnotesize 脚注大小 \\
              \cmd{small} & \small 小 \\
              \cmd{normalsize} & \normalsize 正常大小 \\
              \cmd{large} & \large 大 \\
              \cmd{huge} & \Huge 巨大 \\
            \end{tabular}
          \end{stampbox}
        \end{center}
      }
      \only<3>{
        也可以使用字体样式对应的更改字体设置的命令,这对于大段文段的设置而言也是很方便的。
        \begin{center}
          \footnotesize
          \begin{stampbox}
            \begin{tabular}{ll}
              \cmd{textrm} & \cmd{rmfamily}\\
              \cmd{texttt} & \cmd{ttfamily}\\
              \cmd{textsf} & \cmd{sffamily}\\
              \cmd{textbf} & \cmd{bfseries}\\
              \cmd{textit} & \cmd{itshape}\\
              \cmd{textsc} & \cmd{scshape}\\
              \cmd{textsl} & \cmd{slshape}\\
            \end{tabular}
          \end{stampbox}
        \end{center}
      }
    \end{column}
    \begin{column}{0.6\textwidth}
      \begin{codeblock}[]{大小}
\documentclass{ctexart}
\begin{document}
|\only<2>{\highlightline}|  {\bfseries\Large 一起向未来\par}
|\only<3>{\highlightline}|  {\itshape Together for a Shared Future}
\end{document}
      \end{codeblock}
    \end{column}
  \end{columns}
  \only<2>{\footnotetext{注意最后显式地使用 \cmd{par} 在改回大小前结束该段,否则会导致下一行的行间距异常!}}
\end{frame}

\section{逻辑结构}
\begin{frame}[fragile]
  \frametitle{列表}
  \begin{columns}
    \begin{column}{0.35\textwidth}
      \begin{codeblock}[]{无序列表}
\documentclass{ctexart}
\begin{document}
|\highlightline|  \begin{itemize}
    \item 第一项
    \item 第二项
    \item 第三项
|\highlightline|  \end{itemize}
\end{document}
      \end{codeblock}
    \end{column}
    \begin{column}{0.35\textwidth}
      \begin{codeblock}[]{有序列表}
\documentclass{ctexart}
\begin{document}
|\highlightline|  \begin{enumerate}
    \item 第一项
    \item 第二项
    \item 第三项
|\highlightline|  \end{enumerate}
\end{document}
      \end{codeblock}
    \end{column}
    \begin{column}{0.35\textwidth}
      \begin{codeblock}[]{描述列表}
\documentclass{ctexart}
\begin{document}
|\highlightline|  \begin{description}
    \item[||第一] 文本
    \item[||第二] 文本
    \item[||第三] 文本  
|\highlightline|  \end{description}
\end{document}
      \end{codeblock}
    \end{column}
  \end{columns}
\end{frame}

%更深的列表技巧,定理环境等

\begin{frame}
  \frametitle{列表}
  \begin{columns}
    \begin{column}{0.35\textwidth}
      \includepdflarge{unordered}
    \end{column}
    \begin{column}{0.35\textwidth}
      \includepdflarge{ordered}
    \end{column}
    \begin{column}{0.35\textwidth}
      \includepdflarge{description}
    \end{column}
  \end{columns}
\end{frame}

\begin{frame}[fragile,label=sectioning]%
  \frametitle{目次结构}
  \begin{columns}
    \begin{column}{0.4\textwidth}
      \LaTeX{} 可以使用目次命令将文档划分层级\footnotemark,并自动设定对应字体样式和大小。
      \begin{center}
        \begin{stampbox}
          \footnotesize
          \begin{tabular}{rll}
           命令 & 中文 & 层次 \\
           \cmd{chapter} & 章\footnotemark & \sout{0} \\
           \cmd{section} & 节 & 1 \\
           \cmd{subsection} & 小节 & 2 \\
           \cmd{subsubsection} & 小小节 & 3 \\
          \end{tabular}
        \end{stampbox}
      \end{center}
    \end{column}
    \begin{column}{0.6\textwidth}
      \begin{codeblock}[]{目次}
\documentclass{ctexart}
\begin{document}
|\highlightline|  \section{||概念}
|\highlightline|  \subsection{\LaTeX{}}
  \LaTeX{} 是一个用以排版高质量作品的文档准备系统。
\end{document}
      \end{codeblock}
    \end{column}
  \end{columns}
  \footnotetext{章这一级只在 \pkg{report} 和 \pkg{book} 文档类(包括对应的中文文档类)有定义。还有不常用的 \cmd{part} (0@\pkg{article}/-1@\pkg{report}\&\pkg{book}\&\pkg{beamer}) 以及更低层次的 \cmd{paragraph} (4) 与 \cmd{subparagraph} (5)。 }
\end{frame}

\begin{frame}[fragile]%
  \frametitle{组织文档}
  \begin{columns}
    \begin{column}{0.4\textwidth}
      \only<1>{
        \cmd{tableofcontents} 用来生成对于目次命令的目录。如果你想设定显示到哪个层级,在这个命令前使用 \cmd{setcounter\{tocdepth\}\{层次\}}
      }
      \only<2>{
        对于大型文档而言,使用多个文件管理源文件通常是更方便的。而 \cmd{include} 和 \cmd{input} 都以相对路径的方式插入其他 \TeX{} 文档。
        区别在于,\cmd{include} 命令会从新页开始并做一些内部调整,这基本上只对 \pkg{chapter} 这一级有用。而 \cmd{input} 会原样插入源代码。
      }
      \only<3>{
        但是 \cmd{include} 插入的文档可以使用 \cmd{includeonly} 管理当前要排印哪一部分的内容,利用所有章节辅助文件的同时,减少编译时间并专注于该部分的内容。
      }
    \end{column}
    \begin{column}{0.6\textwidth}
      \begin{codeblock}[]{主文档}
\documentclass{ctexrep}
|\only<3>{\highlightline}|\includeonly{learnlatex,sjtuthesis}
\begin{document}
|\only<1>{\highlightline}|  \tableofcontents
|\only<2-3>{\highlightline}|  % !TeX root = ..\..\latex-talk.tex

\part{学习 \LaTeX{}}
% FIXME: Part Page miniframe overflow
% FIXME: footnote fault numbering

\begin{frame}[plain]
  \vfil
  \begin{center}
    \href{https://learnlatex.org}{
      \rmfamily
      Learn\,\lower1ex\hbox{\Huge\LaTeX{}}.org
    }
  \end{center}
  \vfil
  \begin{center}
    \parbox{0.75\linewidth}{
      Learn\LaTeX{}.org\cite{learnlatex} 提供了解 \LaTeX{} 的 16 篇简短的教程,并包含了一些可以在线运行的示例,可以通过亲自动手查看实验效果。本部分主要参考由 C\TeX{}-org 提供的中文翻译版本 \link{https://github.com/CTeX-org/learnlatex.github.io/tree/zh-Hans/zh-Hans/}。
    }
  \end{center}
  \vfil
\end{frame}

{ % Start of shaded number logo

\newcommand{\shadedfont}[2][1pt]{
  % #1 (optional): shadow distance
  % #2: the text needed to be shaded
  \hbox{\rlap{\color{gray}\hskip#1#2}#2}
}
\newcounter{learnsec}
\setcounter{learnsec}{-1}
\newcommand{\updatelogo}{
  % update the logo corresponding to current counter.
  \stepcounter{learnsec}
  \logo{
    \raise.3ex\hbox{\tiny\insertsection}\shadedfont{\arabic{learnsec}}
  }
}
\let\oldsection=\section
\renewcommand{\section}[1]{\oldsection{#1}\updatelogo}

\section{是什么}
\begin{frame}
  \frametitle{\TeX{}}
  \begin{columns}[c]
    \begin{column}{0.7\textwidth}
      \begin{center}
        \rmfamily\Huge
        \hologo{La}\highlight[structure!70]{\TeX{}}
      \end{center}
      \begin{center}
        \parbox{0.75\textwidth}{
          \TeX{} 是由斯坦福大学教授高德纳
          (Donald E.~Knuth)于 1977 年开始开发的排版引擎。目前仍在更新,最新版本号为 3.141592653 \link{https://tug.org/TUGboat/tb42-1/tb130knuth-tuneup21.pdf}。
        }
      \end{center}
    \end{column}
    \begin{column}{0.3\textwidth}
      \includegraphics[width=.8\columnwidth]{Knuth.jpg}
    \end{column}
  \end{columns}
\end{frame}

\begin{frame}
  \frametitle{\LaTeX{}}
  \begin{columns}[c]
    \begin{column}{0.7\textwidth}
      \begin{center}
        \rmfamily\Huge
        \highlight[structure]{\LaTeX{}}
      \end{center}
      \begin{center}
        \parbox{0.75\textwidth}{
          \LaTeX{} 是最早在 1985 年由现就职于微软的 Leslie Lamport 开发的一种 \TeX{} \textbf{格式}\footnotemark,使用一些列宏和扩展宏包来简化 \TeX{} 的使用。现在由 \LaTeX{} Project 的成员维护。现在广泛使用的版本是 \LaTeXe{},最新的版本为 \LaTeX3(2020 年 10 月后默认内置)。
        }
      \end{center}
    \end{column}
    \begin{column}{0.3\textwidth}
      \includegraphics[width=.8\columnwidth]{Lamport.jpg}
    \end{column}
  \end{columns}
  \footnotetext{\hologo{ConTeXt} 也是一种 \TeX{} 格式 \link{https://www.contextgarden.net/}。}
\end{frame}

\begin{frame}
  \frametitle{程序}
  \begin{columns}[c]
    \begin{column}{0.7\textwidth}
      \begin{center}
        \rmfamily\Huge
        \highlight[structure]{\hologo{pdfLaTeX}}
      \end{center}
      \begin{center}
        \parbox{0.7\textwidth}{
          \hologo{pdfLaTeX} 是为了编译一个 \LaTeX{} 文档而运行的程序。实际上底层在运行一个叫 \hologo{pdfTeX} 的引擎,并预装了对应的 \LaTeX{} \textbf{格式}。为了利用临时文件,可能就需要多次运行程序。
        }
      \end{center}
    \end{column}
    \begin{column}{0.3\textwidth}
      \begin{block}{}
        \ttfamily\small
        > \highlight{pdflatex} main.tex\\
        This is pdfTeX, Version 3.141592653-
        2.6-1.40.23 (MiKTeX 21.10)\\
        entering extended mode\\
        \highlight{LaTeX2e} <2021-11-15>\\
        \highlight{L3} programming layer <2021-11-22>
      \end{block}
    \end{column}
  \end{columns}
\end{frame}

\begin{frame}
  \frametitle{引擎}
  \begin{columns}[c]
    \begin{column}{0.7\textwidth}
      \begin{center}
        \rmfamily\Huge
        \highlight[structure!70]{pdf}\hologo{La}\highlight[structure!70]{\TeX{}}
      \end{center}
      \begin{center}
        \parbox{0.7\textwidth}{
          pdf\TeX{} 是编译 \TeX{} 文档(以 \texttt{.tex} 结尾)的\textbf{引擎}---可以理解 \TeX{} 指令的\textbf{程序}。
        }
      \end{center}
    \end{column}
    \begin{column}{0.3\textwidth}
      \begin{block}{}
        \ttfamily\small
        > pdflatex main.tex\\
        This is \highlight[structure!70]{pdfTeX}, Version 3.141592653-
        2.6-1.40.23 (MiKTeX 21.10)
        entering extended mode\\
        LaTeX2e <2021-11-15>\\
        L3 programming layer <2021-11-22>
      \end{block}
    \end{column}
  \end{columns}
\end{frame}

\begin{frame}
  \frametitle{Unicode 引擎}
  \begin{table}
    \caption{主流 \hologo{(La)TeX} 程序
    \footnote{(u)p\TeX{} 是日语最常用的引擎,生成 \texttt{.dvi},支持 Unicode。}\footnote{Ap\TeX{} 具有底层 CJK 支持,内联 Ruby,Color Emoji。}}
    \footnotesize
    \begin{stampbox}
      \begin{tabular}{c>{\raggedright}*{3}{p{3.5cm}}}
        \alert{引擎}     & \hologo{pdfTeX}   & \hologo{XeTeX}   & \hologo{LuaTeX}   \\
        \alert{程序}     & \hologo{pdfLaTeX} & \hologo{XeLaTeX} & \hologo{LuaLaTeX} \\
        \alert{特点}     & 直接生成 PDF,支持 micro-typography  & 支持 Unicode、OpenType 与复杂文字编排 (CTL) & 支持 Unicode,内联 Lua,支持 OpenType \\
      \end{tabular}
    \end{stampbox}
  \end{table}

  \begin{center}
    \parbox{.9\textwidth}{
      \hologo{pdfLaTeX} 不支持 Unicode。为了排版中文,大部分情况下 \faApple{}\,\faLinux{} 应当使用 \hologo{XeLaTeX},而 \hologo{LuaLaTeX} 速度相对较慢。\faWindows{} 可以在一些情况下使用 \hologo{pdfLaTeX}。
    }
  \end{center}
\end{frame}

% \begin{frame}
%   \paragraph{\hologo{pdfLaTeX}} \TeX{} 和 \LaTeX{} 被广泛使用之前,它们只需内置支持欧洲语言即可。在 Unicode 出现之前,\LaTeX{} 提供了许多种\textbf{文件编码}来允许很多语言的文字以原生的方式输入,\hologo{pdfLaTeX} 也只需要使用 8 位文件编码和 8 位字体。
% \end{frame}

\section{跑起来}
\begin{frame}
  \frametitle{发行版}
  \begin{table}
    \caption{\hologo{TeX} 发行版}
    \footnotesize
    \begin{stampbox}
      \begin{tabular}{c>{\raggedright}*{3}{p{3.2cm}}}
        \alert{发行版}     & \hologo{MiKTeX} \link{https://mirrors.sjtug.sjtu.edu.cn/ctan/systems/win32/miktex/setup/windows-x64/basic-miktex-21.12-x64.exe}   & \TeX{} Live \link{https://mirrors.sjtug.sjtu.edu.cn/ctan/systems/texlive/tlnet/install-tl.zip}   & Mac\TeX{} \link{https://mirrors.sjtug.sjtu.edu.cn/ctan/systems/mac/mactex/mactex-20210328.pkg}  \\[2pt]
        \alert{特点}      &  只安装必要文件,依赖用时更新  &  所有平台均可使用,每年发布一次 & Mac 系统专用,对 \TeX{} Live 的进一步打包 \\
        \alert{推荐平台}  & \faWindows  & \faLinux &  \faApple \\
      \end{tabular}
    \end{stampbox}
  \end{table}
  \begin{center}
    \parbox{.9\textwidth}{
      要让 \LaTeX{} 跑起来,核心就是要有一套 \TeX{} 发行版,来获取让 \LaTeX{} 工作所需的一组程序和文件。参考《一份简短的关于 \LaTeX{} 安装的介绍》\link{https://mirrors.sjtug.sjtu.edu.cn/ctan/info/install-latex-guide-zh-cn/install-latex-guide-zh-cn.pdf} 安装想使用的发行版。推荐使用发行版的最新版本\footnote{老版本 Linux 系统的包管理器自带 \TeX{} Live 发行版可能不是最新的 \link{https://repology.org/project/texlive/versions},尽量使用镜像安装,并手动将相关环境变量添加到路径 \link{https://www.tug.org/texlive/doc/texlive-zh-cn/texlive-zh-cn.pdf}。},并使用国内镜像。
    }
  \end{center}
\end{frame}

\begin{frame}[plain]
  \hbox to \textwidth{
    \hfil
    \vbox to 3cm{
      \hbox{
        \resizebox{3cm}{!}{\includegraphics{\getcontribpath{sjtug}{vi/sjtug.pdf}}}
      }
    }
    \hfil
    \vbox to 3cm{
      \vfill
      \hbox{\Large\bfseries\color{cprimary} 稳定、快速、现代的镜像服务。}
      \vskip2pt
      \hbox{托管于华东教育网骨干节点上海交通大学。}
      \vfill
    }
    \hskip20pt
    \hfil
  }

  \begin{center}
    \parbox{0.8\textwidth}{
      推荐使用 SJTUG 软件镜像服务,离得近,下得快。
      
      \begin{description}
        \footnotesize
        \item[\TeX{} Live]  {\ttfamily tlmgr option repository https://mirrors.sjtug.sjtu.edu.cn/CTAN/systems/texlive/tlnet}
        \item[\hologo{MiKTeX}] 在 \hologo{MiKTeX} Console 中设置镜像源为 \url{https://mirrors.sjtug.sjtu.edu.cn}
      \end{description}
    }
  \end{center}
\end{frame}

\begin{frame}
  \frametitle{编辑器}
  \begin{table}
    \caption{开源编辑器推荐}
    \footnotesize
    \begin{stampbox}
      \begin{tabular}{c>{\raggedright}*{3}{p{3.5cm}}}
        \alert{编辑器}     & \begin{tabular}{c}Visual Studio Code\\ \LaTeX{} Workshop\end{tabular}  & \TeX{}studio & \TeX{}works \\[5pt]
        \alert{特点}      &  搭配 VS Code 使用非常方便,易扩展  & 可以使用大量的菜单选项输入代码块,用户友好 & 只提供基础的高亮与运行方法,发行版自带\footnote{Mac\TeX{} 打包的是 \TeX{}Shop 编辑器。} \\
      \end{tabular}
    \end{stampbox}
  \end{table}
  \begin{center}
    \parbox{.9\textwidth}{
      使用专为 \LaTeX{} 设计的编辑器将带来更多便利,因为它们往往会提供一键编译、内置 PDF 阅读器以及语法高亮等功能。几乎所有现代的 \LaTeX{} 编辑器都提供 Sync\TeX{} 这一强大的功能,以在 PDF 和 代码间对应跳转。
    }
  \end{center}
\end{frame}

\begin{frame}
  \frametitle{在线平台}
  \begin{table}
    \caption{在线协作平台推荐}
    \footnotesize
    \begin{stampbox}
      \begin{tabular}{c>{\raggedright}*{2}{p{4cm}}}
        \alert{在线平台}     & Overleaf \link{https://www.overleaf.com/}  & 交大 \LaTeX{} 助手 \link{https://latex.sjtu.edu.cn/} \\[2pt]
        \alert{特点}      & 最流行的在线平台,提供大量的模板,但国内访问慢 & 校内平台,隐私保护有保障,共享项目限制少 \\
      \end{tabular}
    \end{stampbox}
  \end{table}
  \begin{center}
    \parbox{.9\textwidth}{
      在线平台允许你直接在网页中编辑文档,无需本地安装即可在后台运行 \LaTeX{},并显示生成的 PDF。可以参照 Overleaf 官方文档学习如何使用在线平台 \link{https://www.overleaf.com/learn}。
    }
  \end{center}
\end{frame}

\section{基本结构}
\begin{frame}[fragile]%
  \frametitle{文档部件}
  \begin{columns}[c]
    \begin{column}{0.4\textwidth}
      \only<1>{
        \cmd{documentclass} 命令加载了\textbf{文档类}。\pkg{article} 是由 \LaTeX{}提供的用于排版短文档的基本文档类。
        \begin{description}
          \footnotesize
          \item[\pkg{article}] 不包含章的短文档
          \item[\pkg{report}] 含有章的单面印刷文档
          \item[\pkg{book}] 含有章的双面印刷文档
          \item[\pkg{beamer}] 制作幻灯片
        \end{description}
      }
      \only<2>{
        \env{document} 环境用于指示文档主体的范围。\LaTeX{} 还有其他的使用 \cmd{begin} 和 \cmd{end} 的搭配,我们称这些为\textbf{环境}。它们将用来设定局部格式命令\footnotemark。
      }
      \only<3>{
        \includepdflarge{enminimal}
      }
    \end{column}
    \begin{column}{0.6\textwidth}
      \begin{codeblock}[]{排版英文最简示例}
|\only<1>{\highlightline}|\documentclass{article}
|\only<2>{\highlightline}|\begin{document}
|\only<3>{\highlightline}|  Together for a Shared Future
|\only<2>{\highlightline}|\end{document}
      \end{codeblock}
    \end{column}
  \end{columns}
  \only<2>{\footnotetext{环境实际上是一个组,只不过通过语义化的形式预装了对应的格式命令。普通的组可以直接使用一对大括号之间的内容 \{$\cdots$\} 表示。}}
\end{frame}

\section{扩展}
\begin{frame}[fragile]%
  \frametitle{中文排版}
  \begin{columns}[c]
    \begin{column}{0.4\textwidth}
      \only<1>{
        \cmd{usepackage} 用于使用宏包以向 \LaTeX{} 添加或修改功能,需要在\textbf{导言区}调用。
        这里使用 \pkg{ctex} 宏集以获得中文支持。其调用底层因随不同的引擎而不同。
        {
          \footnotesize
          \begin{stampbox}
            \begin{tabular}{c*{3}{c}}
              \alert{引擎}     & \hologo{pdfTeX}   & \hologo{XeTeX}   & \hologo{LuaTeX}   \\
              \alert{程序}     & \hologo{pdfLaTeX} & \hologo{XeLaTeX} & \hologo{LuaLaTeX} \\
              \alert{宏包}     & CJK\footnotemark & xeCJK & luatexja \\
              \alert{封装}     & \multicolumn{3}{c}{ctex} \\
            \end{tabular}
          \end{stampbox}
        }
        \vspace{-1cm}
      }
      \only<2>{
        C\TeX{} 建议对于之前提到的常规文档类,最佳实践是使用该宏集提供的四种中文文档类,以对特定类型提供额外的中文排版适配。
        \begin{center}
          \begin{stampbox}
            \footnotesize
            \begin{tabular}{cc}
              \pkg{ctexart} & \pkg{ctexrep} \\
              \pkg{ctexbook} & \pkg{ctexbeamer} \\
            \end{tabular}
          \end{stampbox}
        \end{center}
      }
      \only<3>{
        \includepdflarge{cnminimal}
      }
      \only<4>{
        大部分情况下,你都不应当在 \LaTeX{} 中强制断行:你几乎只是想另起一段,或者是想在段落之间添加空行(使用 \pkg{parskip} 宏包就可实现)。
        只有\alert{很少的}情况下你需要使用 \textbackslash{}\textbackslash{} 来另起一行而不另起一段。
      }
    \end{column}
    \begin{column}{0.6\textwidth}
      \begin{codeblock}[]{排版中文\only<2->{最佳实践}}
|\only<2>{\highlightline}|\documentclass{|\only<1>{article}\only<2->{ctexart}|}
|\only<1>{\highlightline\textbackslash{}usepackage\{ctex\}\hfill\color{cprimary}\% 导言区}|
\begin{document}
|\only<3>{\highlightline}|    一起向未来
|\only<4>{\highlightline}|
  Together for a Shared Future
\end{document}
      \end{codeblock}
    \end{column}
  \end{columns}
  \only<1>{\footnotetext{ctex 在 \faApple\,\faLinux{} 上已经不可以使用 \hologo{pdfLaTeX} 编译,以及在 \faWindows{} 上使用该引擎也会变更自动间距调整等功能的默认行为。}}
\end{frame}

\section{设定格式}
\begin{frame}[fragile]%
  \frametitle{字体样式}
  \begin{columns}
    \begin{column}{0.4\textwidth}
      \only<1>{
        \includepdflarge{fontstyle}
      }
      \only<2>{
        可以使用显示样式设定命令对小段做加粗、斜体、等宽等等的处理。
        \begin{center}
          \footnotesize
          \begin{stampbox}
            \begin{tabular}{rl}
              \cmd{textrm} & \textrm{衬线} \\
              \cmd{textbf} & \textbf{加粗} \\
              \cmd{textit} & \kaishu 斜体 \\
              \cmd{texttt} & \texttt{等宽} \\
              \cmd{textsf} & \textsf{无衬线} \\
              \cmd{textsc} & \textsc{Small Caps} \\
              \cmd{textsl} & \textsl{Slanted} \\
            \end{tabular}
          \end{stampbox}
        \end{center}
      }
      \only<3>{
        与 Word 不同的是,\LaTeX{} 一般情况下并不需要使用上面的显式命令,而是采用逻辑标记的方法,
        比如 \cmd{emph} 可以强调文字,以及下面将要提到的目次命令(第 \ref{sectioning} 页)。
        这样可以统一管理格式。
      }
    \end{column}
    \begin{column}{0.6\textwidth}
      \begin{codeblock}[]{样式}
\documentclass{ctexart}
\begin{document}
|\only<2>{\highlightline}|  \textbf{||一起向未来}

|\only<3>{\highlightline}|  \emph{Together for a Shared Future}
\end{document}
      \end{codeblock}
    \end{column}
  \end{columns}
\end{frame}

\begin{frame}[fragile]%
  \frametitle{\only<1-2>{字体大小}\only<3>{字体样式}}
  \begin{columns}
    \begin{column}{0.4\textwidth}
      \only<1>{
        \includepdflarge{fontsize}
      }
      \only<2>{
        同样地,你也可以显式地设定字体大小,但是这种命令会更改行文设置,所以需要使用一个组来限定作用范围\footnotemark。
        \begin{center}
          \footnotesize
          \begin{stampbox}
            \begin{tabular}{rl}
              \cmd{tiny} & \tiny 极小 \\
              \cmd{scriptsize} & \scriptsize 抄本大小  \\
              \cmd{footnotesize} & \footnotesize 脚注大小 \\
              \cmd{small} & \small 小 \\
              \cmd{normalsize} & \normalsize 正常大小 \\
              \cmd{large} & \large 大 \\
              \cmd{huge} & \Huge 巨大 \\
            \end{tabular}
          \end{stampbox}
        \end{center}
      }
      \only<3>{
        也可以使用字体样式对应的更改字体设置的命令,这对于大段文段的设置而言也是很方便的。
        \begin{center}
          \footnotesize
          \begin{stampbox}
            \begin{tabular}{ll}
              \cmd{textrm} & \cmd{rmfamily}\\
              \cmd{texttt} & \cmd{ttfamily}\\
              \cmd{textsf} & \cmd{sffamily}\\
              \cmd{textbf} & \cmd{bfseries}\\
              \cmd{textit} & \cmd{itshape}\\
              \cmd{textsc} & \cmd{scshape}\\
              \cmd{textsl} & \cmd{slshape}\\
            \end{tabular}
          \end{stampbox}
        \end{center}
      }
    \end{column}
    \begin{column}{0.6\textwidth}
      \begin{codeblock}[]{大小}
\documentclass{ctexart}
\begin{document}
|\only<2>{\highlightline}|  {\bfseries\Large 一起向未来\par}
|\only<3>{\highlightline}|  {\itshape Together for a Shared Future}
\end{document}
      \end{codeblock}
    \end{column}
  \end{columns}
  \only<2>{\footnotetext{注意最后显式地使用 \cmd{par} 在改回大小前结束该段,否则会导致下一行的行间距异常!}}
\end{frame}

\section{逻辑结构}
\begin{frame}[fragile]
  \frametitle{列表}
  \begin{columns}
    \begin{column}{0.35\textwidth}
      \begin{codeblock}[]{无序列表}
\documentclass{ctexart}
\begin{document}
|\highlightline|  \begin{itemize}
    \item 第一项
    \item 第二项
    \item 第三项
|\highlightline|  \end{itemize}
\end{document}
      \end{codeblock}
    \end{column}
    \begin{column}{0.35\textwidth}
      \begin{codeblock}[]{有序列表}
\documentclass{ctexart}
\begin{document}
|\highlightline|  \begin{enumerate}
    \item 第一项
    \item 第二项
    \item 第三项
|\highlightline|  \end{enumerate}
\end{document}
      \end{codeblock}
    \end{column}
    \begin{column}{0.35\textwidth}
      \begin{codeblock}[]{描述列表}
\documentclass{ctexart}
\begin{document}
|\highlightline|  \begin{description}
    \item[||第一] 文本
    \item[||第二] 文本
    \item[||第三] 文本  
|\highlightline|  \end{description}
\end{document}
      \end{codeblock}
    \end{column}
  \end{columns}
\end{frame}

%更深的列表技巧,定理环境等

\begin{frame}
  \frametitle{列表}
  \begin{columns}
    \begin{column}{0.35\textwidth}
      \includepdflarge{unordered}
    \end{column}
    \begin{column}{0.35\textwidth}
      \includepdflarge{ordered}
    \end{column}
    \begin{column}{0.35\textwidth}
      \includepdflarge{description}
    \end{column}
  \end{columns}
\end{frame}

\begin{frame}[fragile,label=sectioning]%
  \frametitle{目次结构}
  \begin{columns}
    \begin{column}{0.4\textwidth}
      \LaTeX{} 可以使用目次命令将文档划分层级\footnotemark,并自动设定对应字体样式和大小。
      \begin{center}
        \begin{stampbox}
          \footnotesize
          \begin{tabular}{rll}
           命令 & 中文 & 层次 \\
           \cmd{chapter} & 章\footnotemark & \sout{0} \\
           \cmd{section} & 节 & 1 \\
           \cmd{subsection} & 小节 & 2 \\
           \cmd{subsubsection} & 小小节 & 3 \\
          \end{tabular}
        \end{stampbox}
      \end{center}
    \end{column}
    \begin{column}{0.6\textwidth}
      \begin{codeblock}[]{目次}
\documentclass{ctexart}
\begin{document}
|\highlightline|  \section{||概念}
|\highlightline|  \subsection{\LaTeX{}}
  \LaTeX{} 是一个用以排版高质量作品的文档准备系统。
\end{document}
      \end{codeblock}
    \end{column}
  \end{columns}
  \footnotetext{章这一级只在 \pkg{report} 和 \pkg{book} 文档类(包括对应的中文文档类)有定义。还有不常用的 \cmd{part} (0@\pkg{article}/-1@\pkg{report}\&\pkg{book}\&\pkg{beamer}) 以及更低层次的 \cmd{paragraph} (4) 与 \cmd{subparagraph} (5)。 }
\end{frame}

\begin{frame}[fragile]%
  \frametitle{组织文档}
  \begin{columns}
    \begin{column}{0.4\textwidth}
      \only<1>{
        \cmd{tableofcontents} 用来生成对于目次命令的目录。如果你想设定显示到哪个层级,在这个命令前使用 \cmd{setcounter\{tocdepth\}\{层次\}}
      }
      \only<2>{
        对于大型文档而言,使用多个文件管理源文件通常是更方便的。而 \cmd{include} 和 \cmd{input} 都以相对路径的方式插入其他 \TeX{} 文档。
        区别在于,\cmd{include} 命令会从新页开始并做一些内部调整,这基本上只对 \pkg{chapter} 这一级有用。而 \cmd{input} 会原样插入源代码。
      }
      \only<3>{
        但是 \cmd{include} 插入的文档可以使用 \cmd{includeonly} 管理当前要排印哪一部分的内容,利用所有章节辅助文件的同时,减少编译时间并专注于该部分的内容。
      }
    \end{column}
    \begin{column}{0.6\textwidth}
      \begin{codeblock}[]{主文档}
\documentclass{ctexrep}
|\only<3>{\highlightline}|\includeonly{learnlatex,sjtuthesis}
\begin{document}
|\only<1>{\highlightline}|  \tableofcontents
|\only<2-3>{\highlightline}|  \include{learnlatex}
|\only<3>{\highlightline}|  \include{sjtuthesis}
\end{document}
      \end{codeblock}
    \end{column}
  \end{columns}
\end{frame}

\begin{frame}[fragile]
  \frametitle{组织文档}
  \begin{columns}
    \begin{column}{0.4\textwidth}
      \begin{codeblock}[]{learnlatex.tex}
|\highlightline|\chapter{||学习 \LaTeX{}}
\section{||概念}
\subsection{\LaTeX{}}
\LaTeX{} 是一个用以排版高质量作品的文档准备系统。
      \end{codeblock}
      子文件中就不需要添加 \env{document} 环境了\footnotemark。
    \end{column}
    \begin{column}{0.6\textwidth}
      \begin{codeblock}[]{主文档}
|\highlightline|\documentclass{ctexrep}
\includeonly{learnlatex,sjtuthesis}
\begin{document}
  \tableofcontents
  \include{learnlatex}
  \include{sjtuthesis}
\end{document}
      \end{codeblock}
    \end{column}
  \end{columns}
  \footnotetext{如果想强制指定子文档的主文档,可以在文件第一行输入魔术命令:\texttt{\% !TeX root = main.tex}}
\end{frame}

\section{图}
\begin{frame}[fragile]%
  \frametitle{\temporal<5>{插图}{浮动体}{插图}}
  \begin{columns}
    \begin{column}{0.6\textwidth}
      \begin{codeblock}[]{插入单图\only<4->{最佳实践}}
\documentclass{ctexart}
|\only<2>{\highlightline}|\usepackage{graphicx}
|\only<2>{\highlightline}|\graphicspath{{figs/}{pics/}}
\begin{document}
|\only<5>{\highlightline}|\begin{figure}[ht]
|\only<6>{\highlightline}|  \centering
|\only<3>{\highlightline}|  \includegraphics[width=|\only<1-3>{4cm}\only<4->{0.4\textbackslash{}textwidth}|]{sjtug}
|\only<7>{\highlightline}|  \caption{SJTUG 徽标}\label{fig:sjtug}
|\only<5>{\highlightline}|\end{figure}
\end{document}
      \end{codeblock}
    \end{column}
    \begin{column}{0.4\textwidth}
      \only<1>{
        \includepdflarge{insertimage}
      }
      \only<2>{
        为了插入外部图片,需要使用 \pkg{graphicx} 宏包。之后在文档主体便可以使用 \cmd{includegraphics} 插入图片。导言区也可以加入 \cmd{graphicspath} 指定图片文件夹\footnotemark。
      }
      \only<3>{
        \cmd{includegraphics} 命令便以相对路径的方式插入图片,如果无同名图片,那么后缀名可以省略。可以使用可选参数指定插入的图片尺寸,最佳实践是使用 \cmd{textwidth} 或 \cmd{linewidth} 的相对值指定尺寸大小,以在未来可能的布局更改中保留一定的灵活性。
      }
      \only<4>{
        也可以通过键值对的方法设置图片的其他属性。
        \begin{center}
          \footnotesize
          \begin{stampbox}
            \begin{tabular}{rl}
              \pkg{width} & 宽度 \\
              \pkg{height} & 高度 \\
              \pkg{scale} & 缩放 \\
              \pkg{angle} & 角度 \\
            \end{tabular}
          \end{stampbox}
        \end{center}
      }
      \only<5>{
        \env{figure} 为一个浮动体环境(\env{table} 也是),可以将其移动到其他位置上以减少行文中的空白。可以添加可选参数以指定如何放置浮动体,最多可以使用四种位置描述符:
        \begin{center}
          \footnotesize
          \begin{stampbox}
            \begin{tabular}{cll}
              \pkg{h} & Here & 尽可能在这里 \\
              \pkg{t} & Top & 页面顶部 \\
              \pkg{b} & Bottom & 页面底部 \\
              \pkg{p} & Page & 浮动体专页 \\
            \end{tabular}
          \end{stampbox}
        \end{center}
        还可以只使用 \pkg{float} 宏包提供的 \pkg{H} 描述符以强制置于此处。
      }
      \only<6>{
        采用 \cmd{centering} 命令而不是 \env{center} 环境来水平居中图片。这将避免多余的纵向间距。
      }
      \only<7>{
        使用 \cmd{caption} 命令输入题注,如果这个命令写在插入图片前面,题注将会在上方(中文中一般对 \env{table} 环境这么做)。后面将会看到如何对留有标记(\cmd{label})的图片进行引用。
      }
    \end{column}
  \end{columns}
  \only<2>{\footnotetext{其命令参数每个为一个以 \texttt{/} 结尾的文件夹,每个文件夹需要使用大括号包裹起来。}}
\end{frame}

\begin{frame}[fragile]
  \begin{columns}
    \begin{column}{0.6\textwidth}
      \begin{codeblock}[]{插入双图}
\documentclass{ctexart}
\usepackage{graphicx}
\graphicspath{{figs/}{pics/}}
\begin{document}
  \begin{figure}[ht]
|\only<1>{\highlightline}|    \begin{minipage}{0.48\textwidth}
      \centering
      \includegraphics[height=2cm]{sjtug}
|\only<2>{\highlightline}|      \caption{SJTUG 徽标}\label{fig:sjtug}
|\only<1>{\highlightline}|    \end{minipage}\hfill
|\only<1>{\highlightline}|    \begin{minipage}{0.48\textwidth}
      \centering
      \includegraphics[height=2cm]{sjtugt}
|\only<2>{\highlightline}|      \caption{SJTUG||文字}\label{fig:sjtugt}
|\only<1>{\highlightline}|    \end{minipage}
  \end{figure}
\end{document}
      \end{codeblock}
    \end{column}
    \begin{column}{0.4\textwidth}
      \only<1>{
        在 \env{figure} 环境里使用 \env{minipage} 小页制作列盒子,以并排两图并分别编号,需要设定强制参数以指定列宽。两个小页之间添加 \cmd{hfill} 使两个小页两端对齐。
      }
      \only<2>{
        在每个小页内部分别使用 \cmd{caption} 添加标签。
      }
      \only<3>{
        \includepdflarge{doubleimages}
      }
    \end{column}
  \end{columns}
\end{frame}

\begin{frame}[fragile]%
  \begin{columns}
    \begin{column}{0.6\textwidth}
      \begin{codeblock}[]{}
\documentclass{ctexart}
\usepackage{graphicx}
|\highlightline|\usepackage{subcaption}
\graphicspath{{figs/}{pics/}}
\begin{document}
  \begin{figure}[ht]
|\highlightline|    \begin{subfigure}{0.48\textwidth}
      \centering
      \includegraphics[height=2cm]{sjtug}
      \caption{||徽标}
|\highlightline|    \end{subfigure}\hfill
|\highlightline|    \begin{subfigure}{0.48\textwidth}
      \centering
      \includegraphics[height=2cm]{sjtugt}
      \caption{||文字}
|\highlightline|    \end{subfigure}
    \caption{SJTUG}\label{fig:sjtug}
  \end{figure}
\end{document}
      \end{codeblock}
    \end{column}
    \begin{column}{0.4\textwidth}
      \includepdflarge{subfigures}\vspace{15pt}
      \pkg{subcaption} 宏包提供了 \env{subfigure} 环境(以及 \env{subtable}),可以用于以子图的形式插入,编号会增加一级。也可以为子图添加子集引用编号。
    \end{column}
  \end{columns}
\end{frame}

\section{表}
\begin{frame}[fragile]
  \frametitle{简单表格}
  \begin{columns}
    \begin{column}{0.6\textwidth}
      \begin{codeblock}[]{}
\documentclass{ctexart}
|\only<1-2>{\highlightline}|\usepackage{|\temporal<1>{array}{\highlight{array}}{array},\temporal<2>{booktabs}{\highlight{booktabs}}{booktabs}|}
\begin{document}
\begin{table}[ht]
  \centering
  \caption{||北京冬奥会吉祥物}
|\only<1>{\highlightline}|  \begin{tabular}{lp{3cm}}
|\only<2>{\highlightline}|    \toprule
|\only<3>{\highlightline}|吉祥物 & 描述                          \\
|\only<2>{\highlightline}|    \midrule
|\only<3>{\highlightline}|冰墩墩 & 2022 年北京冬季奥运会吉祥物  \\
|\only<3>{\highlightline}|雪容融 & 2022 年北京冬季残奥会吉祥物  \\
|\only<2>{\highlightline}|    \bottomrule
|\only<1>{\highlightline}|  \end{tabular}
\end{table}
\end{document}
      \end{codeblock}
    \end{column}
    \begin{column}{0.4\textwidth}
      \only<1>{
        使用 \env{tabular} 环境绘制表格。需要添加参数(称为\textbf{表格导言})以确定每一列的对齐方式。引入 \pkg{array} 宏包来使用更多的\textbf{引导符}。
        \begin{center}
          \footnotesize
          \begin{stampbox}
            \begin{tabular}{>{\ttfamily}ll}
              \alert{l} & 向左对齐 \\
              \alert{c} & 居中对齐 \\
              \alert{r} & 向右对齐 \\
              \alert{p\{3cm\}} & 固定列宽,两端对齐 \\
              \alert{m\{3cm\}} & \texttt{p} + 垂直居中对齐 \\
              \alert{>\{\textbackslash{}bfseries\}} & 后一列单元格前加命令 \\
              \alert{*\{3\}\{l\}} & 三个左对齐列 \\
            \end{tabular}
          \end{stampbox}
        \end{center}
      }
      \only<2>{
        \pkg{booktabs} 宏包提供了标准三线表格所需要的行分割线:\cmd{toprule},\cmd{midrule},\cmd{bottomrule}。也可以使用 \cmd{cmidrule\{1-2\}} 来部分地绘制行分割线。一般不推荐在专业表格中使用纵向分割线。
      }
      \only<3>{
        每行内容使用 \textbackslash\textbackslash{} 分隔开,每行中的单元格使用 \& 分隔开。
      }
      \only<4>{
        \includepdflarge{table}
      }
    \end{column}
  \end{columns}
\end{frame}

\begin{frame}[fragile]%
  \begin{columns}
    \begin{column}{0.6\textwidth}
      \begin{codeblock}[]{表头居中}
\documentclass{ctexart}
\usepackage{array,booktabs}
\begin{document}
\begin{table}[ht]
  \centering
  \caption{||北京冬奥会吉祥物}
  \begin{tabular}{lp{3cm}}
    \toprule
|\highlightline|\multicolumn{1}{c}{||吉祥物} &
|\highlightline|\multicolumn{1}{c}{||描述} \\
    \midrule
||冰墩墩 & 2022 年北京冬季奥运会吉祥物  \\
||雪容融 & 2022 年北京冬季残奥会吉祥物  \\
    \bottomrule
  \end{tabular}
\end{table}
\end{document}
      \end{codeblock}
    \end{column}
    \begin{column}{0.4\textwidth}
      \cmd{multicolumn} 命令不仅可以用于合并同行的单元格,还可以用于临时地屏蔽表格导言对该列的对齐方式定义。这里用于居中表头。
      \begin{center}
        \begin{stampbox}
          \parbox{0.85\linewidth}{
            \ttfamily\color{blue}\textbackslash{}multicolumn\{格数\}\{对齐方式\}\{内容\}
          }
        \end{stampbox}
      \end{center}
      跨页表格考虑使用 \pkg{longtable} 宏包。带标注的表格可以考虑使用 \pkg{threeparttable} 宏包。考虑使用在线工具生成表格代码 \link{https://www.tablesgenerator.com/latex_tables}。
    \end{column}
  \end{columns}
\end{frame}

\section{数学公式}
\begin{frame}
  \frametitle{数学模式}
  \begin{alertblock}{}
  输入数学公式是 \LaTeX{} 的绝对强项,很多常见的公式服务依赖于一些轻量级渲染引擎比如 MathJax, K\kern-.3ex\raise.4ex\hbox{\footnotesize A}\kern-.3ex\TeX{}。但是它们实际上使用的是 \LaTeX{} 语法变种,也就是说并没有使用 \LaTeX{} 后端。所以不要期望有完全一致的输出。
  \end{alertblock}
  
  为了更好的获得数学公式输入支持,请使用 \hologo{AmS}math 宏包。数学模式分为两种:
  \begin{description}
    \item[行内模式] 一般通过一对美元符号(\$$\cdots$\$)标记,可以使用编辑器内建的符号表输入数学符号,也可以使用在线工具手写识别 \link{https://detexify.kirelabs.org/classify.html}。
    \item[行间模式] 一般通过 \env{equation} 环境\footnote{这是有编号环境,其加星号的变种 \env{equation*} 用于生成无编号环境。}输入。如果需要使用多行公式,请使用 \env{align} 环境,并按照类似表格输入的方式,使用 \& 对齐符号,\textbackslash\textbackslash{} 换行。如果不想手动居中,可以考虑多行自动居中的 \env{gather} 和单个大型公式首尾两端对齐 \env{multline}。
  \end{description}
\end{frame}

\begin{frame}
  \frametitle{数学命令展示}
  \begin{columns}
    \begin{column}{0.33\textwidth}
      \begin{exampleblock}{}
        $PV=nRT$
      \end{exampleblock}
      \begin{exampleblock}{}
        $\sum_{i=1}^ki^2=\frac{n(n+1)(2n+1)}{6}$
      \end{exampleblock}
      \begin{exampleblock}{}
        $T(n) = aT\left(\left\lceil\frac{n}{b}\right\rceil\right) + \mathcal{O}(n^d)$
      \end{exampleblock}
      \begin{exampleblock}{}
        $\frac{x_{1}+x_{2}+x_{3}}{3}\geq \sqrt[3]{x_{1}x_{2}x_{3}}$
      \end{exampleblock}
      \begin{exampleblock}{}
        $n=(\underbrace{1\cdots 1}_{k\text{ of 1's}})_2=2^{k+1}-1$
      \end{exampleblock}
      \begin{exampleblock}{}
        $\nabla f (P)= \left.\left(\frac{\partial f}{\partial x},\frac{\partial f}{\partial y},\frac{\partial f}{\partial z}\right)\right|_{P}$
      \end{exampleblock}
    \end{column}
    \begin{column}{0.67\textwidth}
      \begin{exampleblock}{}
        \begin{equation*}
          \int_{a}^b f(x)\,\mathrm{d}x=\lim_{|P|\rightarrow 0}\sum_{i=1}^n f(\xi_i)\Delta x_i
        \end{equation*}
      \end{exampleblock}
      \begin{exampleblock}{}
        \begin{equation}
          T(n) = \begin{cases}
            \mathcal{O}(n^d),&\textrm{if } d>\log_b a, \\
            \mathcal{O}(n^d\log n), &\textrm{if } d=\log_b a,\\
            \mathcal{O}(n^{\log_b a}), &\textrm{if } d<\log_b a.
          \end{cases}
        \end{equation}
      \end{exampleblock}
      \begin{exampleblock}{}
        \begin{align}
          Q^{T}A&=R \\
          \begin{pmatrix}
            q_1^T \\ q_2^T \\ q_3^T
          \end{pmatrix}
          \begin{pmatrix}
            a_1 & a_2 & a_3
          \end{pmatrix}
          &=R
        \end{align}
      \end{exampleblock}
    \end{column}
  \end{columns}
\end{frame}

%更深入地讲解 mathtools, unicode-math, siunix

\section{引用}
\begin{frame}[fragile]
  \frametitle{交叉引用}
  \only<1>{
    正如之前所提到的,\LaTeX{} 中使用 \cmd{label} 标记,然后可以使用 \cmd{ref} 来引用这个标记。 \cmd{label} 可以放在使用计数器的对象之后。
  }
  \only<2>{
    为了使得对公式编号的引用带有括号,推荐使用 \hologo{AmS}math 宏包中的 \cmd{eqref} 命令。对于多行公式环境,每一个换行符前都可以添加一个 \cmd{label} 用于引用该行公式。
  }
  \begin{columns}
    \begin{column}{0.5\textwidth}
      \begin{codeblock}[]{图}
\begin{figure}
|\only<1>{\highlightline}|  \caption{||示例}\label{fig:example}
\end{figure}
      \end{codeblock}
      \begin{codeblock}[]{表}
\begin{table}
|\only<1>{\highlightline}|  \caption{||示例}\label{tab:example}
\end{table}
      \end{codeblock}
    \end{column}
    \begin{column}{0.5\textwidth}
\begin{codeblock}[]{目次}
|\only<1>{\highlightline}|\section{||示例}\label{sec:example}
\end{codeblock}

\begin{codeblock}[]{公式}
\begin{equation}
  a = b + c
|\only<1>{\highlightline}|\label{eq:example}
\end{equation}
|\only<2>{\highlightline}|如公式 \eqref{eq:example} 所示,
\end{codeblock}
    \end{column}
  \end{columns}
\end{frame}

\begin{frame}[fragile]
  \frametitle{文献引用}
  \LaTeX{} 管理参考文献可以采用专用数据库文件 \texttt{.bib},很多的文献管理文件比如 EndNote \link{https://lic.sjtu.edu.cn/Default/softshow/tag/MDAwMDAwMDAwMLGImKE}, Zotero \link{https://www.zotero.org/}, JabRef \link{https://www.jabref.org/} 都可以直接导出这种格式的文件用于 \LaTeX{} 论文中的引用。一般不需要手写数据库文件,你只需要注意每一个文献会在数据库中有一个主键,这个类似于上文的 \cmd{label} 标记,只是要引用数据库中的文献需要使用 \cmd{cite} 命令。
  
  \begin{codeblock}[]{ref.bib}
|\highlightline|@phdthesis{devoftech,|\hfill\alert{\% 类型为博士论文,主键为\texttt{devoftech}}|
  title={||新时期我国信息技术产业的发展},
  author={||江泽民},
  year={2008}
}
  \end{codeblock}
\end{frame}

\begin{frame}
  \frametitle{文献引用}
  而让 \LaTeX{} 处理 \texttt{.bib} 数据库文件与引用有两种工作流。你可能需要查清楚如何在编辑器中设置对应的工作流,或者采用后面所提到的高级编译方式 \texttt{latexmk}。
  \begin{columns}
    \begin{column}{0.5\textwidth}
      \begin{block}{\hologo{BibTeX} + \pkg{gbt7714}}
        一般期刊提交使用这种方法,\pkg{natbib} 宏包将提供命令 \cmd{citet}(文本引用) 和 \cmd{citep}(括号引用)。中文引用可以直接使用 \pkg{gbt7714} 宏包,但是角模式和正文模式不能混用。
      \end{block}
    \end{column}
    \begin{column}{0.5\textwidth}
      \begin{block}{\hologo{biber} + \pkg{biblatex}}
        这是更容易自定义的方法,与 \hologo{BibTeX} 的运作方式稍有不同。\pkg{biblatex} 提供了更加智能的引用命令。而中文引用可以使用 \pkg{biblatex} 宏包的样式 \pkg{gb7714-2015},使用该样式需要使用 \hologo{XeLaTeX} 编译。
      \end{block}
    \end{column}
  \end{columns}
\end{frame}

\begin{frame}[fragile]
  \frametitle{文献引用}
  \begin{columns}
    \begin{column}{0.5\textwidth}
      \begin{codeblock}[]{\hologo{BibTeX} + \pkg{gbt7714}}
\documentclass{ctexart}
\usepackage{gbt7714}
\bibliographystyle{gbt7714-numerial}
% \citestyle{numbers}  % 正文模式
\begin{document}
  ||他指出了...\cite{devoftech}
  \bibliography{ref}
\end{document}
      \end{codeblock}
    \end{column}
    \begin{column}{0.5\textwidth}
      \begin{codeblock}[]{\hologo{biber} + \pkg{biblatex}}
\documentclass{ctexart}
\usepackage[backend=biber,style=gb7714-2015]{biblatex}
\addbibresource{ref.bib}
\begin{document}
  ||他在文献 \parencite{devoftech}
  ||指出了...\cite{devoftech}
  \printbibliography
\end{document}
      \end{codeblock}
    \end{column}
  \end{columns}
\end{frame}

\begin{frame}
  \frametitle{文献引用}
  \begin{columns}
    \begin{column}{0.5\textwidth}
      \includepdflarge{bibtex}
    \end{column}
    \begin{column}{0.5\textwidth}
      \includepdflarge{biblatex}
    \end{column}
  \end{columns}
\end{frame}

} % End of customized shaded number logo

|\only<3>{\highlightline}|  % !TeX root = ..\..\latex-talk.tex

\part{SJTUThesis}

\begin{frame}
  \frametitle{简介}
  \begin{columns}
    \begin{column}{0.6\textwidth}
      \begin{itemize}
        \item 最早由韦建文于 2009 年 11 月发布 0.1a 版,2018 年起由 SJTUG 接手维护
        \item 最新版:\SJTUThesisVersion{} (\SJTUThesisDate)
        \item 支持本科、硕士、博士学位论文以及课程论文的排版
      \end{itemize}
    \end{column}
    \begin{column}{0.4\textwidth}
      \begin{exampleblock}{}
        \begin{minipage}[c]{1cm}
          \includegraphics[width=0.8cm]{\getcontribpath{sjtug}{vi/sjtug}}
        \end{minipage}
        \begin{minipage}[c]{2cm}
          \href{https://github.com/sjtug}{sjtug}/\href{https://github.com/sjtug/SJTUThesis}{SJTUThesis}
        \end{minipage}
      \end{exampleblock}
      \vspace{-8pt}
      \begin{block}{}
        \scriptsize
        上海交通大学 \hologo{XeLaTeX} 学位论文及课程论文模板 | Shanghai Jiao Tong University \hologo{XeLaTeX} Thesis Template
      \end{block}
      \vspace{-8pt}
      \begin{alertblock}{}
        \scriptsize
        \begin{tabular}{cl}
          \faStar & 2.4k \\
          \faEye & 55 \\
          \faCodeBranch & 701 \\
        \end{tabular}
      \end{alertblock}
    \end{column}
  \end{columns}
\end{frame}

\begin{frame}
  \frametitle{下载与编译}
  \alert{下载} 推荐安装 Git \link{https://git-scm.com/} 后,克隆 SJTUG 镜像仓库
  \begin{exampleblock}{\faGit*}
    \ttfamily\small
    git clone https://mirror.sjtu.edu.cn/git/SJTUThesis.git/
  \end{exampleblock}

  \alert{编译} 推荐使用 \pkg{latexmk} 编译\footnote{\hologo{MiKTeX} 用户需要手动安装 Perl 解释器 \link{https://www.perl.org/get.html} 才能使用 \pkg{latexmk}。},在不能够利用自带的 \texttt{.latexmkrc} 配置文件的情况下,需要查清楚在对应的编辑器中如何使用 \hologo{XeLaTeX} + \hologo{biber} 编译 \link{https://github.com/sjtug/SJTUThesis/blob/master/README.md}。
  \begin{exampleblock}{\faTerminal}
    \ttfamily\small
    latexmk -xelatex main
  \end{exampleblock}

  Overleaf 用户可以下载压缩包后,上传并采用 \hologo{XeLaTeX} 编译方式。
\end{frame}

\begin{frame}
  \frametitle{手动编译}
  \alert{第一次编译失败} 如果没有办法通过通常方式编译成功,请尝试使用文件夹内附带 \faLinux{}\,\faApple{} \texttt{Makefile} 和 \faWindows{} \texttt{Compile.bat} 进行编译。

  \alert{统计字数} 编写过程中也可以使用对应的命令调用 \TeX{}count 来统计正文字数。
  \begin{columns}
    \begin{column}{0.38\textwidth}
      \begin{exampleblock}{\faLinux{}\,\faApple}
        \ttfamily
        make all\\
        make clean\\
        make cleanall\\
        make wordcount
      \end{exampleblock}
    \end{column}
    \begin{column}{0.38\textwidth}
      \begin{exampleblock}{\faWindows}
        \ttfamily
        ./Compile.bat thesis\\
        ./Compile.bat clean\\
        ./Compile.bat cleanall\\
        ./Compile.bat wordcount
      \end{exampleblock}
    \end{column}
    \begin{column}{0.24\textwidth}
      \begin{block}{\faInfo}
        \ttfamily
        编译论文\\
        清理中间文件\\
        $\hookrightarrow +$删除论文\\
        统计字数
      \end{block}
    \end{column}
  \end{columns}
\end{frame}

\begin{frame}[label=compile]
  \frametitle{编译问题排查}
  \begin{columns}
    \begin{column}{0.33\textwidth}
      \begin{alertblock}{无法使用 \texttt{latexmk}\thesisissue{578}}
        \hologo{MiKTeX} 需要安装 Perl 解释器。
      \end{alertblock}  
      \begin{alertblock}{C\TeX{} 套装无法编译\thesisissue{446}}
        使用最新 \TeX{} 发行版。
      \end{alertblock}
      \begin{alertblock}{\hologo{pdfLaTeX} 无法编译\thesisissue{444}}
        请使用 \texttt{latexmk},或更改编辑器设置以 \hologo{XeLaTeX} 编译。
      \end{alertblock}
    \end{column}
    \begin{column}{0.33\textwidth}
      \begin{alertblock}{缺少字体\thesisissue{564} \thesisdiscuss{598}}
        更换字体集,或者安装对应字体。
      \end{alertblock}
      \begin{alertblock}{缺少汉字\thesisissue{533} \thesisdiscuss{617}}
        去除使用 fandol 字体集的设定。或者是安装字体后,改用 \texttt{fontset=adobe} 或 \texttt{fontset=founder}。
      \end{alertblock}
    \end{column}
    \begin{column}{0.33\textwidth}
      \begin{block}{\faInfoCircle{} README}
        不同编辑器的设置请首先参阅 README \link{https://github.com/sjtug/SJTUThesis/blob/master/README.md} 文档。
      \end{block}
      \begin{block}{\faBookOpen{} Wiki}
        其他编译问题推荐查阅 Wiki \link{https://github.com/sjtug/SJTUThesis/wiki} 的使用说明部分。
      \end{block}
    \end{column}
  \end{columns}
\end{frame}

\begin{frame}[fragile, label=covers]
  \begin{codeblock}[firstnumber=3]{main.tex}
|\alert{\% 载入 SJTUThesis 模版}|
\documentclass[|\only<1>{\highlight{type}}\only<2>{type}|=|\only<1>{bachelor}\only<2>{\highlight{bachelor}}|]{sjtuthesis}
  \end{codeblock}
  \begin{figure}
    \parbox{0.9\textwidth}{
      \begin{subfigure}{0.20\textwidth}
        \framebox{\includegraphics[width=\linewidth]{support/thesis/bachelor}}
        \caption{\only<1>{学士}\only<2>{\texttt{bachelor}}}
      \end{subfigure}\hfill
      \begin{subfigure}{0.20\textwidth}
        \framebox{\includegraphics[width=\linewidth]{support/thesis/master}}
        \caption{\only<1>{硕士}\only<2>{\texttt{master}}}
      \end{subfigure}\hfill
      \begin{subfigure}{0.20\textwidth}
        \framebox{\includegraphics[width=\linewidth]{support/thesis/doctor}}
        \caption{\only<1>{博士}\only<2>{\texttt{doctor}}}
      \end{subfigure}\hfill
      \begin{subfigure}{0.20\textwidth}
        \framebox{\includegraphics[width=\linewidth]{support/thesis/course}}
        \caption{\only<1>{课程}\only<2>{\texttt{course}}}
      \end{subfigure}
      \caption{论文类型示例\only<2>{ \texttt{type}}}
    }
  \end{figure}
\end{frame}

\begin{frame}[fragile]
  \frametitle{文档类选项}
  % \framesubtitle{\textbackslash{}documentclass\{sjtuthesis\}}
  \begin{columns}
    \begin{column}{0.45\textwidth}
      \includegraphics[page=10]{thesisdir}
    \end{column}
    \begin{column}{0.55\textwidth}
      \begin{table}[H]
        \caption{文档类选项}
        \footnotesize
        \begin{tabular}{>{\ttfamily}rll}
          \toprule
          选项 & 含义 & 相关 \\
          \midrule
          type= & 指定论文类型 & 第 \ref{covers} 页\\
          fontset= & 指定字体 & 第 \ref{compile} 页\\
          \midrule
          review & 开启盲审模式 & \thesisissue{195} \thesisissue{686} \\
          twoside & 双页模式 & \thesisissue{554} \\
          oneside & 单页模式 & \thesisissue{694} \\
          openright & 章从奇数页开始 & \thesisdiscuss{724} \\
          openany & 章从任意页开始 & \thesisissue{446} \\
          \bottomrule
        \end{tabular}
      \end{table}
    \end{column}
  \end{columns}
\end{frame}

\begin{frame}[fragile]
  \frametitle{基本配置}
  \framesubtitle{\textbackslash{}input\{setup\}}
  \begin{columns}
    \begin{column}{0.45\textwidth}
      \includegraphics[page=9]{thesisdir}
    \end{column}
    \begin{column}{0.55\textwidth}
      \begin{codeblock}[firstnumber=12]{main.tex}
|\highlightline<1>|% 论文基本配置,加载宏包等全局配置
|\highlightline<1>|\input{setup}

\begin{document}

%TC:ignore

|\highlightline<2>|% 标题页
|\highlightline<2>|\maketitle
      \end{codeblock}
      \visible<2>{
        \cmd{sjtusetup} 中的 \pkg{info} 将会修改封面的信息设置(见第 \ref{covers} 页)。
      }
    \end{column}
  \end{columns}
\end{frame}

\begin{frame}[fragile]
  \frametitle{基本配置}
  \framesubtitle{\textbackslash{}sjtusetup}
  \begin{columns}
    \begin{column}{0.45\textwidth}
      \includegraphics[page=1]{thesisdir}
    \end{column}
    \begin{column}{0.55\textwidth}
      \begin{codeblock}[firstnumber=3]{setup.tex}
\sjtusetup{
  info = {
    title    = {||上海交通大学学位论文 \LaTeX{} 模板示例文档},
    title*   = {A Sample for \LaTeX-based SJTU Thesis Template},
    author   = {||某\quad{}某},
    author* = {Mo Mo},
  },
  style = { header-logo-color = red, 
  },
  name = {
    publications = {||攻读学位期间完成的论文},
  },
}
      \end{codeblock}
    \end{column}
  \end{columns}
\end{frame}

\begin{frame}
  \frametitle{基本配置}
  \framesubtitle{\textbackslash{}sjtusetup}
  \begin{columns}
    \begin{column}{0.45\textwidth}
      \includegraphics[page=1]{thesisdir}
    \end{column}
    \begin{column}{0.55\textwidth}
      \begin{table}[H]
        \centering
        \caption{info 域}
        \footnotesize
        \begin{tabular}{lll} \toprule
          命令作用 & 中文对应选项 & 英文对应选项 \\ \midrule
          论文标题 & \texttt{title} & \texttt{title*} \\
          关键字列表 & \texttt{keywords} & \texttt{keywords*} \\
          作者姓名&  \texttt{author} &\texttt{author*}\\
          申请学位名称 & \texttt{degree}&\texttt{degree*}\\
          院系名称 & \texttt{department} & \texttt{department*}\\
          专业名称 & \texttt{major} & \texttt{major*}\\
          导师 & \texttt{supervisor} & \texttt{supervisor*}\\
          副导师 & \texttt{assisupervisor} & \texttt{assisupervisor*}\\
          日期 & \multicolumn{2}{c}{\texttt{date}}\\
          学号 & \multicolumn{2}{c}{\texttt{id}}\\ \bottomrule
          \end{tabular}
      \end{table}
    \end{column}
  \end{columns}
\end{frame}

\begin{frame}[fragile]
  \frametitle{版权页}
  \framesubtitle{\textbackslash{}copyrightpage}
  \begin{columns}
    \begin{column}{0.45\textwidth}
      \only<1>{
        \includegraphics[page=9]{thesisdir}
      }
      \only<2>{
        \includegraphics[page=2]{thesisdir}
      }
      \only<3>{
        \begin{figure}[H]
          \framebox{\includegraphics[page=2,width=0.4\linewidth]{bachelor}}
          \caption{版权页}
        \end{figure}
      }
    \end{column}
    \begin{column}{0.55\textwidth}
      \begin{codeblock}[firstnumber=22]{main.tex}
|\highlightline<1>|% 原创性声明及使用授权书
|\highlightline<1>|\copyrightpage
|\highlightline<2>|% 插入外置原创性声明及使用授权书
|\highlightline<2>|% \copyrightpage[scans/sample-copyright-old.pdf]
      \end{codeblock}
      \only<1>{
        \cmd{copyrightpages} 可以用于插入版权页。
      }
      \only<2>{
        \cmd{copyrightpages} 也接受一个可选参数,用于直接使用扫描件。
      }
    \end{column}
  \end{columns}
\end{frame}

\begin{frame}[fragile]
  \frametitle{前置部分}
  \framesubtitle{\textbackslash{}frontmatter}
  \begin{columns}
    \begin{column}{0.45\textwidth}
      \only<1>{
        \includegraphics[page=9]{thesisdir}
      }
      \only<2>{
        \includegraphics[page=3]{thesisdir}
      }
      \only<3>{
        \begin{figure}[H]
          \begin{subfigure}{0.45\textwidth}
            \framebox{\includegraphics[page=3,width=\linewidth]{bachelor}}
            \caption{中文}
          \end{subfigure}\hfill
          \begin{subfigure}{0.45\textwidth}
            \framebox{\includegraphics[page=4,width=\linewidth]{bachelor}}
            \caption{英文}
          \end{subfigure}
          \caption{摘要}
        \end{figure}
      }
      \only<4>{
        \begin{figure}[H]
          \begin{subfigure}{0.30\linewidth}
            \centering
            \framebox{\includegraphics[page=5,width=0.6\linewidth]{bachelor}}
            \caption{目录}
          \end{subfigure}
          \begin{subfigure}{0.30\linewidth}
            \centering
            \framebox{\includegraphics[page=6,width=0.6\linewidth]{bachelor}}
            \caption{插图}
          \end{subfigure}

          \begin{subfigure}{0.30\linewidth}
            \centering
            \framebox{\includegraphics[page=7,width=0.6\linewidth]{bachelor}}
            \caption{表格}
          \end{subfigure}
          \begin{subfigure}{0.30\linewidth}
            \centering
            \framebox{\includegraphics[page=8,width=0.6\linewidth]{bachelor}}
            \caption{算法}
          \end{subfigure}
          \caption{索引}
        \end{figure}
      }
      \only<5>{
        \includegraphics[page=4]{thesisdir}
      }
      \only<6>{
        \begin{figure}[H]
          \framebox{\includegraphics[page=9,width=0.5\linewidth]{bachelor}}
          \caption{符号对照表}
        \end{figure}
      }
    \end{column}
    \begin{column}{0.55\textwidth}
      \begin{codeblock}[firstnumber=30]{main.tex}
|\highlightline<2-3>|% 摘要
|\highlightline<2-3>|\input{contents/abstract}

|\highlightline<4>|% 目录
|\highlightline<4>|\tableofcontents
|\highlightline<4>|% 插图索引
|\highlightline<4>|\listoffigures*
|\highlightline<4>|% 表格索引
|\highlightline<4>|\listoftables*
|\highlightline<4>|% 算法索引
|\highlightline<4>|\listofalgorithms*

|\highlightline<5-6>|% 符号对照表
|\highlightline<5-6>|\input{contents/nomenclature}
      \end{codeblock}
    \end{column}
  \end{columns}
\end{frame}

\begin{frame}[fragile]
  \frametitle{主体部分}
  \framesubtitle{\textbackslash{}mainmatter}
  \begin{columns}
    \begin{column}{0.45\textwidth}
      \only<1>{
        \includegraphics[page=5]{thesisdir}
      }
      \only<2>{
        \begin{figure}[H]
          \begin{subfigure}{0.30\linewidth}
            \centering
            \framebox{\includegraphics[page=11,width=0.6\linewidth]{bachelor}}
            \caption{简介}
          \end{subfigure}
          \begin{subfigure}{0.30\linewidth}
            \centering
            \framebox{\includegraphics[page=13,width=0.6\linewidth]{bachelor}}
            \caption{数学}
          \end{subfigure}

          \begin{subfigure}{0.30\linewidth}
            \centering
            \framebox{\includegraphics[page=16,width=0.6\linewidth]{bachelor}}
            \caption{浮动体}
          \end{subfigure}
          \begin{subfigure}{0.30\linewidth}
            \centering
            \framebox{\includegraphics[page=22,width=0.6\linewidth]{bachelor}}
            \caption{总结}
          \end{subfigure}
          \caption{主体部分}
        \end{figure}
      }
    \end{column}
    \begin{column}{0.55\textwidth}
      \begin{codeblock}[firstnumber=47]{main.tex}
|\highlightline|% 正文内容
|\highlightline|\input{contents/intro}
|\highlightline|\input{contents/math_and_citations}
|\highlightline|\input{contents/floats}
|\highlightline|\input{contents/summary}

%TC:ignore

% 参考文献
\printbibliography[heading=bibintoc]
      \end{codeblock}
    \end{column}
  \end{columns}
\end{frame}

\begin{frame}
  \frametitle{数学}
  \begin{itemize}
    \item 公式示例:\nolinkurl{contents/math_and_citations.tex}
    \item \SJTUThesis{} 定义了常用的数学环境(需要手工引入 \texttt{ntheorem} 宏包):
      \begin{table}[h]
        \centering
        \footnotesize
        \begin{tabular}{*{7}{l}}\toprule
          assumption  & axiom   & conjecture & corollary    & definition  & example   & exercise  \\
          假设        & 公理    & 猜想       & 推论         & 定义        & 例        & 练习      \\\midrule
          lemma       & problem & proof      & proposition  & remark      & solution  & theorem   \\
          引理        & 问题    & 证明       & 命题         & 注          & 解        & 定理      \\\bottomrule
        \end{tabular}
      \end{table}
      \item \SJTUThesis{} 可以通过 \texttt{unimath} 选项使用 \pkg{unicode-math} 进行数学输入,注意与传统方式的区别。\thesisissue{555}
  \end{itemize}
\end{frame}

\begin{frame}[fragile]
  \frametitle{参考文献}
  \begin{columns}
    \begin{column}{0.45\textwidth}
      \includegraphics[page=6]{thesisdir}
    \end{column}
    \begin{column}{0.55\textwidth}
      \begin{codeblock}[firstnumber=111,numbersep=2pt]{setup.tex}
% 使用 BibLaTeX 处理参考文献
%   biblatex-gb7714-2015 常用选项
%     gbnamefmt=lowercase     姓名大小写由输入信息确定
%     gbpub=false             禁用出版信息缺失处理
\usepackage[backend=biber,style=gb7714-2015]{biblatex}
% 文献表字体
% \renewcommand{\bibfont}{\zihao{-5}}
% 文献表条目间的间距
\setlength{\bibitemsep}{0pt}
|\highlightline|% 导入参考文献数据库
|\highlightline|\addbibresource{bibdata/thesis.bib}
      \end{codeblock}
    \end{column}
  \end{columns}
\end{frame}

\begin{frame}[fragile]
  \frametitle{附录}
  \framesubtitle{\textbackslash{}appendix}
  \begin{columns}
    \begin{column}{0.45\textwidth}
      \only<1>{
        \includegraphics[page=7]{thesisdir}
      }
      \only<2>{
        \begin{figure}[H]
          \begin{subfigure}{0.45\linewidth}
            \framebox{\includegraphics[width=\linewidth,page=24]{bachelor}}
            \caption{}
          \end{subfigure}\hfill
          \begin{subfigure}{0.45\textwidth}
            \framebox{\includegraphics[width=\linewidth,page=25]{bachelor}}
            \caption{}
          \end{subfigure}
          \caption{附录}
        \end{figure}
      }
    \end{column}
    \begin{column}{0.55\textwidth}
      \begin{codeblock}[firstnumber=61]{main.tex}
% 附录中图表不加入索引
\captionsetup{list=no}

% 附录内容
|\highlightline|\input{contents/app_maxwell_equations}
|\highlightline|\input{contents/app_flow_chart}
      \end{codeblock}
    \end{column}
  \end{columns}
\end{frame}

\begin{frame}[fragile]
  \frametitle{结尾部分}
  \framesubtitle{\textbackslash{}backmatter}
  \begin{columns}
    \begin{column}{0.45\textwidth}
      \only<1>{
        \includegraphics[page=8]{thesisdir}
      }
      \only<2>{
        \begin{figure}[H]
          \begin{subfigure}{0.30\linewidth}
            \centering
            \framebox{\includegraphics[page=26,width=0.6\linewidth]{bachelor}}
            \caption{致谢}
          \end{subfigure}
          \begin{subfigure}{0.30\linewidth}
            \centering
            \framebox{\includegraphics[page=27,width=0.6\linewidth]{bachelor}}
            \caption{成就}
          \end{subfigure}

          \begin{subfigure}{0.30\linewidth}
            \centering
            \framebox{\includegraphics[page=28,width=0.6\linewidth]{bachelor}}
            \caption{简历}
          \end{subfigure}
          \begin{subfigure}{0.30\linewidth}
            \centering
            \framebox{\includegraphics[page=29,width=0.6\linewidth]{bachelor}}
            \caption{大摘要*}
          \end{subfigure}
          \caption{结尾部分}
        \end{figure}
      }
    \end{column}
    \begin{column}{0.55\textwidth}
      \begin{codeblock}[firstnumber=76]{main.tex}
% 致谢
\input{contents/acknowledgements}

% 发表论文及科研成果
% 盲审论文中,发表论文及科研成果等仅以第几作者注明即可,不要出现作者或他人姓名
\input{contents/achievements}

% 简历
\input{contents/resume}

% 学士学位论文要求在最后有一个大摘要,单独编页码
\input{contents/digest}
      \end{codeblock}
    \end{column}
  \end{columns}
\end{frame}

\begin{frame}
  \frametitle{还有其他问题?}
  \begin{columns}
    \begin{column}{0.75\textwidth}
    \begin{itemize}
      \item[{\faComment*[regular]}] 日常模板或 \LaTeX{} 使用问题可以前往 Discussions \link{https://github.com/sjtug/SJTUThesis/discussions} 提问
      
      (解决后别忘了 \textcolor{green}{\faCheckCircle{} Mark as answer}
      \item[{\faDotCircle[regular]}] 如果是 \textsc{SJTUThesis} 项目本身的 bug 和 feature request
      
      可以通过 Issues \link{https://github.com/sjtug/SJTUThesis/issues} 反馈。
      \item[{\faCodeBranch}] 如果你有好点子,可以贡献代码
     
      向 \textsc{SJTU\TeX{}}(v1) \link{https://github.com/sjtug/SJTUTeX/tree/v1} 存储库发 PR,\par
      而后把解包结果同步到 \textsc{SJTUThesis}。
  
      \item[{\faTag}] 如果你对正在基于 \LaTeX3 开发的新版感兴趣,\par
      也欢迎向 \textsc{SJTU\TeX{}}(v2) \link{https://github.com/sjtug/SJTUTeX/tree/v2} 发 PR。
  
      \item[{\faQq}] 也欢迎在 QQ 群即时讨论。
    \end{itemize}
    \end{column}
    \begin{column}{0.25\textwidth}
      \includegraphics[height=0.7\textheight]{qq.jpg}
    \end{column}
  \end{columns}
\end{frame}
\end{document}
      \end{codeblock}
    \end{column}
  \end{columns}
\end{frame}

\begin{frame}[fragile]
  \frametitle{组织文档}
  \begin{columns}
    \begin{column}{0.4\textwidth}
      \begin{codeblock}[]{learnlatex.tex}
|\highlightline|\chapter{||学习 \LaTeX{}}
\section{||概念}
\subsection{\LaTeX{}}
\LaTeX{} 是一个用以排版高质量作品的文档准备系统。
      \end{codeblock}
      子文件中就不需要添加 \env{document} 环境了\footnotemark。
    \end{column}
    \begin{column}{0.6\textwidth}
      \begin{codeblock}[]{主文档}
|\highlightline|\documentclass{ctexrep}
\includeonly{learnlatex,sjtuthesis}
\begin{document}
  \tableofcontents
  % !TeX root = ..\..\latex-talk.tex

\part{学习 \LaTeX{}}
% FIXME: Part Page miniframe overflow
% FIXME: footnote fault numbering

\begin{frame}[plain]
  \vfil
  \begin{center}
    \href{https://learnlatex.org}{
      \rmfamily
      Learn\,\lower1ex\hbox{\Huge\LaTeX{}}.org
    }
  \end{center}
  \vfil
  \begin{center}
    \parbox{0.75\linewidth}{
      Learn\LaTeX{}.org\cite{learnlatex} 提供了解 \LaTeX{} 的 16 篇简短的教程,并包含了一些可以在线运行的示例,可以通过亲自动手查看实验效果。本部分主要参考由 C\TeX{}-org 提供的中文翻译版本 \link{https://github.com/CTeX-org/learnlatex.github.io/tree/zh-Hans/zh-Hans/}。
    }
  \end{center}
  \vfil
\end{frame}

{ % Start of shaded number logo

\newcommand{\shadedfont}[2][1pt]{
  % #1 (optional): shadow distance
  % #2: the text needed to be shaded
  \hbox{\rlap{\color{gray}\hskip#1#2}#2}
}
\newcounter{learnsec}
\setcounter{learnsec}{-1}
\newcommand{\updatelogo}{
  % update the logo corresponding to current counter.
  \stepcounter{learnsec}
  \logo{
    \raise.3ex\hbox{\tiny\insertsection}\shadedfont{\arabic{learnsec}}
  }
}
\let\oldsection=\section
\renewcommand{\section}[1]{\oldsection{#1}\updatelogo}

\section{是什么}
\begin{frame}
  \frametitle{\TeX{}}
  \begin{columns}[c]
    \begin{column}{0.7\textwidth}
      \begin{center}
        \rmfamily\Huge
        \hologo{La}\highlight[structure!70]{\TeX{}}
      \end{center}
      \begin{center}
        \parbox{0.75\textwidth}{
          \TeX{} 是由斯坦福大学教授高德纳
          (Donald E.~Knuth)于 1977 年开始开发的排版引擎。目前仍在更新,最新版本号为 3.141592653 \link{https://tug.org/TUGboat/tb42-1/tb130knuth-tuneup21.pdf}。
        }
      \end{center}
    \end{column}
    \begin{column}{0.3\textwidth}
      \includegraphics[width=.8\columnwidth]{Knuth.jpg}
    \end{column}
  \end{columns}
\end{frame}

\begin{frame}
  \frametitle{\LaTeX{}}
  \begin{columns}[c]
    \begin{column}{0.7\textwidth}
      \begin{center}
        \rmfamily\Huge
        \highlight[structure]{\LaTeX{}}
      \end{center}
      \begin{center}
        \parbox{0.75\textwidth}{
          \LaTeX{} 是最早在 1985 年由现就职于微软的 Leslie Lamport 开发的一种 \TeX{} \textbf{格式}\footnotemark,使用一些列宏和扩展宏包来简化 \TeX{} 的使用。现在由 \LaTeX{} Project 的成员维护。现在广泛使用的版本是 \LaTeXe{},最新的版本为 \LaTeX3(2020 年 10 月后默认内置)。
        }
      \end{center}
    \end{column}
    \begin{column}{0.3\textwidth}
      \includegraphics[width=.8\columnwidth]{Lamport.jpg}
    \end{column}
  \end{columns}
  \footnotetext{\hologo{ConTeXt} 也是一种 \TeX{} 格式 \link{https://www.contextgarden.net/}。}
\end{frame}

\begin{frame}
  \frametitle{程序}
  \begin{columns}[c]
    \begin{column}{0.7\textwidth}
      \begin{center}
        \rmfamily\Huge
        \highlight[structure]{\hologo{pdfLaTeX}}
      \end{center}
      \begin{center}
        \parbox{0.7\textwidth}{
          \hologo{pdfLaTeX} 是为了编译一个 \LaTeX{} 文档而运行的程序。实际上底层在运行一个叫 \hologo{pdfTeX} 的引擎,并预装了对应的 \LaTeX{} \textbf{格式}。为了利用临时文件,可能就需要多次运行程序。
        }
      \end{center}
    \end{column}
    \begin{column}{0.3\textwidth}
      \begin{block}{}
        \ttfamily\small
        > \highlight{pdflatex} main.tex\\
        This is pdfTeX, Version 3.141592653-
        2.6-1.40.23 (MiKTeX 21.10)\\
        entering extended mode\\
        \highlight{LaTeX2e} <2021-11-15>\\
        \highlight{L3} programming layer <2021-11-22>
      \end{block}
    \end{column}
  \end{columns}
\end{frame}

\begin{frame}
  \frametitle{引擎}
  \begin{columns}[c]
    \begin{column}{0.7\textwidth}
      \begin{center}
        \rmfamily\Huge
        \highlight[structure!70]{pdf}\hologo{La}\highlight[structure!70]{\TeX{}}
      \end{center}
      \begin{center}
        \parbox{0.7\textwidth}{
          pdf\TeX{} 是编译 \TeX{} 文档(以 \texttt{.tex} 结尾)的\textbf{引擎}---可以理解 \TeX{} 指令的\textbf{程序}。
        }
      \end{center}
    \end{column}
    \begin{column}{0.3\textwidth}
      \begin{block}{}
        \ttfamily\small
        > pdflatex main.tex\\
        This is \highlight[structure!70]{pdfTeX}, Version 3.141592653-
        2.6-1.40.23 (MiKTeX 21.10)
        entering extended mode\\
        LaTeX2e <2021-11-15>\\
        L3 programming layer <2021-11-22>
      \end{block}
    \end{column}
  \end{columns}
\end{frame}

\begin{frame}
  \frametitle{Unicode 引擎}
  \begin{table}
    \caption{主流 \hologo{(La)TeX} 程序
    \footnote{(u)p\TeX{} 是日语最常用的引擎,生成 \texttt{.dvi},支持 Unicode。}\footnote{Ap\TeX{} 具有底层 CJK 支持,内联 Ruby,Color Emoji。}}
    \footnotesize
    \begin{stampbox}
      \begin{tabular}{c>{\raggedright}*{3}{p{3.5cm}}}
        \alert{引擎}     & \hologo{pdfTeX}   & \hologo{XeTeX}   & \hologo{LuaTeX}   \\
        \alert{程序}     & \hologo{pdfLaTeX} & \hologo{XeLaTeX} & \hologo{LuaLaTeX} \\
        \alert{特点}     & 直接生成 PDF,支持 micro-typography  & 支持 Unicode、OpenType 与复杂文字编排 (CTL) & 支持 Unicode,内联 Lua,支持 OpenType \\
      \end{tabular}
    \end{stampbox}
  \end{table}

  \begin{center}
    \parbox{.9\textwidth}{
      \hologo{pdfLaTeX} 不支持 Unicode。为了排版中文,大部分情况下 \faApple{}\,\faLinux{} 应当使用 \hologo{XeLaTeX},而 \hologo{LuaLaTeX} 速度相对较慢。\faWindows{} 可以在一些情况下使用 \hologo{pdfLaTeX}。
    }
  \end{center}
\end{frame}

% \begin{frame}
%   \paragraph{\hologo{pdfLaTeX}} \TeX{} 和 \LaTeX{} 被广泛使用之前,它们只需内置支持欧洲语言即可。在 Unicode 出现之前,\LaTeX{} 提供了许多种\textbf{文件编码}来允许很多语言的文字以原生的方式输入,\hologo{pdfLaTeX} 也只需要使用 8 位文件编码和 8 位字体。
% \end{frame}

\section{跑起来}
\begin{frame}
  \frametitle{发行版}
  \begin{table}
    \caption{\hologo{TeX} 发行版}
    \footnotesize
    \begin{stampbox}
      \begin{tabular}{c>{\raggedright}*{3}{p{3.2cm}}}
        \alert{发行版}     & \hologo{MiKTeX} \link{https://mirrors.sjtug.sjtu.edu.cn/ctan/systems/win32/miktex/setup/windows-x64/basic-miktex-21.12-x64.exe}   & \TeX{} Live \link{https://mirrors.sjtug.sjtu.edu.cn/ctan/systems/texlive/tlnet/install-tl.zip}   & Mac\TeX{} \link{https://mirrors.sjtug.sjtu.edu.cn/ctan/systems/mac/mactex/mactex-20210328.pkg}  \\[2pt]
        \alert{特点}      &  只安装必要文件,依赖用时更新  &  所有平台均可使用,每年发布一次 & Mac 系统专用,对 \TeX{} Live 的进一步打包 \\
        \alert{推荐平台}  & \faWindows  & \faLinux &  \faApple \\
      \end{tabular}
    \end{stampbox}
  \end{table}
  \begin{center}
    \parbox{.9\textwidth}{
      要让 \LaTeX{} 跑起来,核心就是要有一套 \TeX{} 发行版,来获取让 \LaTeX{} 工作所需的一组程序和文件。参考《一份简短的关于 \LaTeX{} 安装的介绍》\link{https://mirrors.sjtug.sjtu.edu.cn/ctan/info/install-latex-guide-zh-cn/install-latex-guide-zh-cn.pdf} 安装想使用的发行版。推荐使用发行版的最新版本\footnote{老版本 Linux 系统的包管理器自带 \TeX{} Live 发行版可能不是最新的 \link{https://repology.org/project/texlive/versions},尽量使用镜像安装,并手动将相关环境变量添加到路径 \link{https://www.tug.org/texlive/doc/texlive-zh-cn/texlive-zh-cn.pdf}。},并使用国内镜像。
    }
  \end{center}
\end{frame}

\begin{frame}[plain]
  \hbox to \textwidth{
    \hfil
    \vbox to 3cm{
      \hbox{
        \resizebox{3cm}{!}{\includegraphics{\getcontribpath{sjtug}{vi/sjtug.pdf}}}
      }
    }
    \hfil
    \vbox to 3cm{
      \vfill
      \hbox{\Large\bfseries\color{cprimary} 稳定、快速、现代的镜像服务。}
      \vskip2pt
      \hbox{托管于华东教育网骨干节点上海交通大学。}
      \vfill
    }
    \hskip20pt
    \hfil
  }

  \begin{center}
    \parbox{0.8\textwidth}{
      推荐使用 SJTUG 软件镜像服务,离得近,下得快。
      
      \begin{description}
        \footnotesize
        \item[\TeX{} Live]  {\ttfamily tlmgr option repository https://mirrors.sjtug.sjtu.edu.cn/CTAN/systems/texlive/tlnet}
        \item[\hologo{MiKTeX}] 在 \hologo{MiKTeX} Console 中设置镜像源为 \url{https://mirrors.sjtug.sjtu.edu.cn}
      \end{description}
    }
  \end{center}
\end{frame}

\begin{frame}
  \frametitle{编辑器}
  \begin{table}
    \caption{开源编辑器推荐}
    \footnotesize
    \begin{stampbox}
      \begin{tabular}{c>{\raggedright}*{3}{p{3.5cm}}}
        \alert{编辑器}     & \begin{tabular}{c}Visual Studio Code\\ \LaTeX{} Workshop\end{tabular}  & \TeX{}studio & \TeX{}works \\[5pt]
        \alert{特点}      &  搭配 VS Code 使用非常方便,易扩展  & 可以使用大量的菜单选项输入代码块,用户友好 & 只提供基础的高亮与运行方法,发行版自带\footnote{Mac\TeX{} 打包的是 \TeX{}Shop 编辑器。} \\
      \end{tabular}
    \end{stampbox}
  \end{table}
  \begin{center}
    \parbox{.9\textwidth}{
      使用专为 \LaTeX{} 设计的编辑器将带来更多便利,因为它们往往会提供一键编译、内置 PDF 阅读器以及语法高亮等功能。几乎所有现代的 \LaTeX{} 编辑器都提供 Sync\TeX{} 这一强大的功能,以在 PDF 和 代码间对应跳转。
    }
  \end{center}
\end{frame}

\begin{frame}
  \frametitle{在线平台}
  \begin{table}
    \caption{在线协作平台推荐}
    \footnotesize
    \begin{stampbox}
      \begin{tabular}{c>{\raggedright}*{2}{p{4cm}}}
        \alert{在线平台}     & Overleaf \link{https://www.overleaf.com/}  & 交大 \LaTeX{} 助手 \link{https://latex.sjtu.edu.cn/} \\[2pt]
        \alert{特点}      & 最流行的在线平台,提供大量的模板,但国内访问慢 & 校内平台,隐私保护有保障,共享项目限制少 \\
      \end{tabular}
    \end{stampbox}
  \end{table}
  \begin{center}
    \parbox{.9\textwidth}{
      在线平台允许你直接在网页中编辑文档,无需本地安装即可在后台运行 \LaTeX{},并显示生成的 PDF。可以参照 Overleaf 官方文档学习如何使用在线平台 \link{https://www.overleaf.com/learn}。
    }
  \end{center}
\end{frame}

\section{基本结构}
\begin{frame}[fragile]%
  \frametitle{文档部件}
  \begin{columns}[c]
    \begin{column}{0.4\textwidth}
      \only<1>{
        \cmd{documentclass} 命令加载了\textbf{文档类}。\pkg{article} 是由 \LaTeX{}提供的用于排版短文档的基本文档类。
        \begin{description}
          \footnotesize
          \item[\pkg{article}] 不包含章的短文档
          \item[\pkg{report}] 含有章的单面印刷文档
          \item[\pkg{book}] 含有章的双面印刷文档
          \item[\pkg{beamer}] 制作幻灯片
        \end{description}
      }
      \only<2>{
        \env{document} 环境用于指示文档主体的范围。\LaTeX{} 还有其他的使用 \cmd{begin} 和 \cmd{end} 的搭配,我们称这些为\textbf{环境}。它们将用来设定局部格式命令\footnotemark。
      }
      \only<3>{
        \includepdflarge{enminimal}
      }
    \end{column}
    \begin{column}{0.6\textwidth}
      \begin{codeblock}[]{排版英文最简示例}
|\only<1>{\highlightline}|\documentclass{article}
|\only<2>{\highlightline}|\begin{document}
|\only<3>{\highlightline}|  Together for a Shared Future
|\only<2>{\highlightline}|\end{document}
      \end{codeblock}
    \end{column}
  \end{columns}
  \only<2>{\footnotetext{环境实际上是一个组,只不过通过语义化的形式预装了对应的格式命令。普通的组可以直接使用一对大括号之间的内容 \{$\cdots$\} 表示。}}
\end{frame}

\section{扩展}
\begin{frame}[fragile]%
  \frametitle{中文排版}
  \begin{columns}[c]
    \begin{column}{0.4\textwidth}
      \only<1>{
        \cmd{usepackage} 用于使用宏包以向 \LaTeX{} 添加或修改功能,需要在\textbf{导言区}调用。
        这里使用 \pkg{ctex} 宏集以获得中文支持。其调用底层因随不同的引擎而不同。
        {
          \footnotesize
          \begin{stampbox}
            \begin{tabular}{c*{3}{c}}
              \alert{引擎}     & \hologo{pdfTeX}   & \hologo{XeTeX}   & \hologo{LuaTeX}   \\
              \alert{程序}     & \hologo{pdfLaTeX} & \hologo{XeLaTeX} & \hologo{LuaLaTeX} \\
              \alert{宏包}     & CJK\footnotemark & xeCJK & luatexja \\
              \alert{封装}     & \multicolumn{3}{c}{ctex} \\
            \end{tabular}
          \end{stampbox}
        }
        \vspace{-1cm}
      }
      \only<2>{
        C\TeX{} 建议对于之前提到的常规文档类,最佳实践是使用该宏集提供的四种中文文档类,以对特定类型提供额外的中文排版适配。
        \begin{center}
          \begin{stampbox}
            \footnotesize
            \begin{tabular}{cc}
              \pkg{ctexart} & \pkg{ctexrep} \\
              \pkg{ctexbook} & \pkg{ctexbeamer} \\
            \end{tabular}
          \end{stampbox}
        \end{center}
      }
      \only<3>{
        \includepdflarge{cnminimal}
      }
      \only<4>{
        大部分情况下,你都不应当在 \LaTeX{} 中强制断行:你几乎只是想另起一段,或者是想在段落之间添加空行(使用 \pkg{parskip} 宏包就可实现)。
        只有\alert{很少的}情况下你需要使用 \textbackslash{}\textbackslash{} 来另起一行而不另起一段。
      }
    \end{column}
    \begin{column}{0.6\textwidth}
      \begin{codeblock}[]{排版中文\only<2->{最佳实践}}
|\only<2>{\highlightline}|\documentclass{|\only<1>{article}\only<2->{ctexart}|}
|\only<1>{\highlightline\textbackslash{}usepackage\{ctex\}\hfill\color{cprimary}\% 导言区}|
\begin{document}
|\only<3>{\highlightline}|    一起向未来
|\only<4>{\highlightline}|
  Together for a Shared Future
\end{document}
      \end{codeblock}
    \end{column}
  \end{columns}
  \only<1>{\footnotetext{ctex 在 \faApple\,\faLinux{} 上已经不可以使用 \hologo{pdfLaTeX} 编译,以及在 \faWindows{} 上使用该引擎也会变更自动间距调整等功能的默认行为。}}
\end{frame}

\section{设定格式}
\begin{frame}[fragile]%
  \frametitle{字体样式}
  \begin{columns}
    \begin{column}{0.4\textwidth}
      \only<1>{
        \includepdflarge{fontstyle}
      }
      \only<2>{
        可以使用显示样式设定命令对小段做加粗、斜体、等宽等等的处理。
        \begin{center}
          \footnotesize
          \begin{stampbox}
            \begin{tabular}{rl}
              \cmd{textrm} & \textrm{衬线} \\
              \cmd{textbf} & \textbf{加粗} \\
              \cmd{textit} & \kaishu 斜体 \\
              \cmd{texttt} & \texttt{等宽} \\
              \cmd{textsf} & \textsf{无衬线} \\
              \cmd{textsc} & \textsc{Small Caps} \\
              \cmd{textsl} & \textsl{Slanted} \\
            \end{tabular}
          \end{stampbox}
        \end{center}
      }
      \only<3>{
        与 Word 不同的是,\LaTeX{} 一般情况下并不需要使用上面的显式命令,而是采用逻辑标记的方法,
        比如 \cmd{emph} 可以强调文字,以及下面将要提到的目次命令(第 \ref{sectioning} 页)。
        这样可以统一管理格式。
      }
    \end{column}
    \begin{column}{0.6\textwidth}
      \begin{codeblock}[]{样式}
\documentclass{ctexart}
\begin{document}
|\only<2>{\highlightline}|  \textbf{||一起向未来}

|\only<3>{\highlightline}|  \emph{Together for a Shared Future}
\end{document}
      \end{codeblock}
    \end{column}
  \end{columns}
\end{frame}

\begin{frame}[fragile]%
  \frametitle{\only<1-2>{字体大小}\only<3>{字体样式}}
  \begin{columns}
    \begin{column}{0.4\textwidth}
      \only<1>{
        \includepdflarge{fontsize}
      }
      \only<2>{
        同样地,你也可以显式地设定字体大小,但是这种命令会更改行文设置,所以需要使用一个组来限定作用范围\footnotemark。
        \begin{center}
          \footnotesize
          \begin{stampbox}
            \begin{tabular}{rl}
              \cmd{tiny} & \tiny 极小 \\
              \cmd{scriptsize} & \scriptsize 抄本大小  \\
              \cmd{footnotesize} & \footnotesize 脚注大小 \\
              \cmd{small} & \small 小 \\
              \cmd{normalsize} & \normalsize 正常大小 \\
              \cmd{large} & \large 大 \\
              \cmd{huge} & \Huge 巨大 \\
            \end{tabular}
          \end{stampbox}
        \end{center}
      }
      \only<3>{
        也可以使用字体样式对应的更改字体设置的命令,这对于大段文段的设置而言也是很方便的。
        \begin{center}
          \footnotesize
          \begin{stampbox}
            \begin{tabular}{ll}
              \cmd{textrm} & \cmd{rmfamily}\\
              \cmd{texttt} & \cmd{ttfamily}\\
              \cmd{textsf} & \cmd{sffamily}\\
              \cmd{textbf} & \cmd{bfseries}\\
              \cmd{textit} & \cmd{itshape}\\
              \cmd{textsc} & \cmd{scshape}\\
              \cmd{textsl} & \cmd{slshape}\\
            \end{tabular}
          \end{stampbox}
        \end{center}
      }
    \end{column}
    \begin{column}{0.6\textwidth}
      \begin{codeblock}[]{大小}
\documentclass{ctexart}
\begin{document}
|\only<2>{\highlightline}|  {\bfseries\Large 一起向未来\par}
|\only<3>{\highlightline}|  {\itshape Together for a Shared Future}
\end{document}
      \end{codeblock}
    \end{column}
  \end{columns}
  \only<2>{\footnotetext{注意最后显式地使用 \cmd{par} 在改回大小前结束该段,否则会导致下一行的行间距异常!}}
\end{frame}

\section{逻辑结构}
\begin{frame}[fragile]
  \frametitle{列表}
  \begin{columns}
    \begin{column}{0.35\textwidth}
      \begin{codeblock}[]{无序列表}
\documentclass{ctexart}
\begin{document}
|\highlightline|  \begin{itemize}
    \item 第一项
    \item 第二项
    \item 第三项
|\highlightline|  \end{itemize}
\end{document}
      \end{codeblock}
    \end{column}
    \begin{column}{0.35\textwidth}
      \begin{codeblock}[]{有序列表}
\documentclass{ctexart}
\begin{document}
|\highlightline|  \begin{enumerate}
    \item 第一项
    \item 第二项
    \item 第三项
|\highlightline|  \end{enumerate}
\end{document}
      \end{codeblock}
    \end{column}
    \begin{column}{0.35\textwidth}
      \begin{codeblock}[]{描述列表}
\documentclass{ctexart}
\begin{document}
|\highlightline|  \begin{description}
    \item[||第一] 文本
    \item[||第二] 文本
    \item[||第三] 文本  
|\highlightline|  \end{description}
\end{document}
      \end{codeblock}
    \end{column}
  \end{columns}
\end{frame}

%更深的列表技巧,定理环境等

\begin{frame}
  \frametitle{列表}
  \begin{columns}
    \begin{column}{0.35\textwidth}
      \includepdflarge{unordered}
    \end{column}
    \begin{column}{0.35\textwidth}
      \includepdflarge{ordered}
    \end{column}
    \begin{column}{0.35\textwidth}
      \includepdflarge{description}
    \end{column}
  \end{columns}
\end{frame}

\begin{frame}[fragile,label=sectioning]%
  \frametitle{目次结构}
  \begin{columns}
    \begin{column}{0.4\textwidth}
      \LaTeX{} 可以使用目次命令将文档划分层级\footnotemark,并自动设定对应字体样式和大小。
      \begin{center}
        \begin{stampbox}
          \footnotesize
          \begin{tabular}{rll}
           命令 & 中文 & 层次 \\
           \cmd{chapter} & 章\footnotemark & \sout{0} \\
           \cmd{section} & 节 & 1 \\
           \cmd{subsection} & 小节 & 2 \\
           \cmd{subsubsection} & 小小节 & 3 \\
          \end{tabular}
        \end{stampbox}
      \end{center}
    \end{column}
    \begin{column}{0.6\textwidth}
      \begin{codeblock}[]{目次}
\documentclass{ctexart}
\begin{document}
|\highlightline|  \section{||概念}
|\highlightline|  \subsection{\LaTeX{}}
  \LaTeX{} 是一个用以排版高质量作品的文档准备系统。
\end{document}
      \end{codeblock}
    \end{column}
  \end{columns}
  \footnotetext{章这一级只在 \pkg{report} 和 \pkg{book} 文档类(包括对应的中文文档类)有定义。还有不常用的 \cmd{part} (0@\pkg{article}/-1@\pkg{report}\&\pkg{book}\&\pkg{beamer}) 以及更低层次的 \cmd{paragraph} (4) 与 \cmd{subparagraph} (5)。 }
\end{frame}

\begin{frame}[fragile]%
  \frametitle{组织文档}
  \begin{columns}
    \begin{column}{0.4\textwidth}
      \only<1>{
        \cmd{tableofcontents} 用来生成对于目次命令的目录。如果你想设定显示到哪个层级,在这个命令前使用 \cmd{setcounter\{tocdepth\}\{层次\}}
      }
      \only<2>{
        对于大型文档而言,使用多个文件管理源文件通常是更方便的。而 \cmd{include} 和 \cmd{input} 都以相对路径的方式插入其他 \TeX{} 文档。
        区别在于,\cmd{include} 命令会从新页开始并做一些内部调整,这基本上只对 \pkg{chapter} 这一级有用。而 \cmd{input} 会原样插入源代码。
      }
      \only<3>{
        但是 \cmd{include} 插入的文档可以使用 \cmd{includeonly} 管理当前要排印哪一部分的内容,利用所有章节辅助文件的同时,减少编译时间并专注于该部分的内容。
      }
    \end{column}
    \begin{column}{0.6\textwidth}
      \begin{codeblock}[]{主文档}
\documentclass{ctexrep}
|\only<3>{\highlightline}|\includeonly{learnlatex,sjtuthesis}
\begin{document}
|\only<1>{\highlightline}|  \tableofcontents
|\only<2-3>{\highlightline}|  \include{learnlatex}
|\only<3>{\highlightline}|  \include{sjtuthesis}
\end{document}
      \end{codeblock}
    \end{column}
  \end{columns}
\end{frame}

\begin{frame}[fragile]
  \frametitle{组织文档}
  \begin{columns}
    \begin{column}{0.4\textwidth}
      \begin{codeblock}[]{learnlatex.tex}
|\highlightline|\chapter{||学习 \LaTeX{}}
\section{||概念}
\subsection{\LaTeX{}}
\LaTeX{} 是一个用以排版高质量作品的文档准备系统。
      \end{codeblock}
      子文件中就不需要添加 \env{document} 环境了\footnotemark。
    \end{column}
    \begin{column}{0.6\textwidth}
      \begin{codeblock}[]{主文档}
|\highlightline|\documentclass{ctexrep}
\includeonly{learnlatex,sjtuthesis}
\begin{document}
  \tableofcontents
  \include{learnlatex}
  \include{sjtuthesis}
\end{document}
      \end{codeblock}
    \end{column}
  \end{columns}
  \footnotetext{如果想强制指定子文档的主文档,可以在文件第一行输入魔术命令:\texttt{\% !TeX root = main.tex}}
\end{frame}

\section{图}
\begin{frame}[fragile]%
  \frametitle{\temporal<5>{插图}{浮动体}{插图}}
  \begin{columns}
    \begin{column}{0.6\textwidth}
      \begin{codeblock}[]{插入单图\only<4->{最佳实践}}
\documentclass{ctexart}
|\only<2>{\highlightline}|\usepackage{graphicx}
|\only<2>{\highlightline}|\graphicspath{{figs/}{pics/}}
\begin{document}
|\only<5>{\highlightline}|\begin{figure}[ht]
|\only<6>{\highlightline}|  \centering
|\only<3>{\highlightline}|  \includegraphics[width=|\only<1-3>{4cm}\only<4->{0.4\textbackslash{}textwidth}|]{sjtug}
|\only<7>{\highlightline}|  \caption{SJTUG 徽标}\label{fig:sjtug}
|\only<5>{\highlightline}|\end{figure}
\end{document}
      \end{codeblock}
    \end{column}
    \begin{column}{0.4\textwidth}
      \only<1>{
        \includepdflarge{insertimage}
      }
      \only<2>{
        为了插入外部图片,需要使用 \pkg{graphicx} 宏包。之后在文档主体便可以使用 \cmd{includegraphics} 插入图片。导言区也可以加入 \cmd{graphicspath} 指定图片文件夹\footnotemark。
      }
      \only<3>{
        \cmd{includegraphics} 命令便以相对路径的方式插入图片,如果无同名图片,那么后缀名可以省略。可以使用可选参数指定插入的图片尺寸,最佳实践是使用 \cmd{textwidth} 或 \cmd{linewidth} 的相对值指定尺寸大小,以在未来可能的布局更改中保留一定的灵活性。
      }
      \only<4>{
        也可以通过键值对的方法设置图片的其他属性。
        \begin{center}
          \footnotesize
          \begin{stampbox}
            \begin{tabular}{rl}
              \pkg{width} & 宽度 \\
              \pkg{height} & 高度 \\
              \pkg{scale} & 缩放 \\
              \pkg{angle} & 角度 \\
            \end{tabular}
          \end{stampbox}
        \end{center}
      }
      \only<5>{
        \env{figure} 为一个浮动体环境(\env{table} 也是),可以将其移动到其他位置上以减少行文中的空白。可以添加可选参数以指定如何放置浮动体,最多可以使用四种位置描述符:
        \begin{center}
          \footnotesize
          \begin{stampbox}
            \begin{tabular}{cll}
              \pkg{h} & Here & 尽可能在这里 \\
              \pkg{t} & Top & 页面顶部 \\
              \pkg{b} & Bottom & 页面底部 \\
              \pkg{p} & Page & 浮动体专页 \\
            \end{tabular}
          \end{stampbox}
        \end{center}
        还可以只使用 \pkg{float} 宏包提供的 \pkg{H} 描述符以强制置于此处。
      }
      \only<6>{
        采用 \cmd{centering} 命令而不是 \env{center} 环境来水平居中图片。这将避免多余的纵向间距。
      }
      \only<7>{
        使用 \cmd{caption} 命令输入题注,如果这个命令写在插入图片前面,题注将会在上方(中文中一般对 \env{table} 环境这么做)。后面将会看到如何对留有标记(\cmd{label})的图片进行引用。
      }
    \end{column}
  \end{columns}
  \only<2>{\footnotetext{其命令参数每个为一个以 \texttt{/} 结尾的文件夹,每个文件夹需要使用大括号包裹起来。}}
\end{frame}

\begin{frame}[fragile]
  \begin{columns}
    \begin{column}{0.6\textwidth}
      \begin{codeblock}[]{插入双图}
\documentclass{ctexart}
\usepackage{graphicx}
\graphicspath{{figs/}{pics/}}
\begin{document}
  \begin{figure}[ht]
|\only<1>{\highlightline}|    \begin{minipage}{0.48\textwidth}
      \centering
      \includegraphics[height=2cm]{sjtug}
|\only<2>{\highlightline}|      \caption{SJTUG 徽标}\label{fig:sjtug}
|\only<1>{\highlightline}|    \end{minipage}\hfill
|\only<1>{\highlightline}|    \begin{minipage}{0.48\textwidth}
      \centering
      \includegraphics[height=2cm]{sjtugt}
|\only<2>{\highlightline}|      \caption{SJTUG||文字}\label{fig:sjtugt}
|\only<1>{\highlightline}|    \end{minipage}
  \end{figure}
\end{document}
      \end{codeblock}
    \end{column}
    \begin{column}{0.4\textwidth}
      \only<1>{
        在 \env{figure} 环境里使用 \env{minipage} 小页制作列盒子,以并排两图并分别编号,需要设定强制参数以指定列宽。两个小页之间添加 \cmd{hfill} 使两个小页两端对齐。
      }
      \only<2>{
        在每个小页内部分别使用 \cmd{caption} 添加标签。
      }
      \only<3>{
        \includepdflarge{doubleimages}
      }
    \end{column}
  \end{columns}
\end{frame}

\begin{frame}[fragile]%
  \begin{columns}
    \begin{column}{0.6\textwidth}
      \begin{codeblock}[]{}
\documentclass{ctexart}
\usepackage{graphicx}
|\highlightline|\usepackage{subcaption}
\graphicspath{{figs/}{pics/}}
\begin{document}
  \begin{figure}[ht]
|\highlightline|    \begin{subfigure}{0.48\textwidth}
      \centering
      \includegraphics[height=2cm]{sjtug}
      \caption{||徽标}
|\highlightline|    \end{subfigure}\hfill
|\highlightline|    \begin{subfigure}{0.48\textwidth}
      \centering
      \includegraphics[height=2cm]{sjtugt}
      \caption{||文字}
|\highlightline|    \end{subfigure}
    \caption{SJTUG}\label{fig:sjtug}
  \end{figure}
\end{document}
      \end{codeblock}
    \end{column}
    \begin{column}{0.4\textwidth}
      \includepdflarge{subfigures}\vspace{15pt}
      \pkg{subcaption} 宏包提供了 \env{subfigure} 环境(以及 \env{subtable}),可以用于以子图的形式插入,编号会增加一级。也可以为子图添加子集引用编号。
    \end{column}
  \end{columns}
\end{frame}

\section{表}
\begin{frame}[fragile]
  \frametitle{简单表格}
  \begin{columns}
    \begin{column}{0.6\textwidth}
      \begin{codeblock}[]{}
\documentclass{ctexart}
|\only<1-2>{\highlightline}|\usepackage{|\temporal<1>{array}{\highlight{array}}{array},\temporal<2>{booktabs}{\highlight{booktabs}}{booktabs}|}
\begin{document}
\begin{table}[ht]
  \centering
  \caption{||北京冬奥会吉祥物}
|\only<1>{\highlightline}|  \begin{tabular}{lp{3cm}}
|\only<2>{\highlightline}|    \toprule
|\only<3>{\highlightline}|吉祥物 & 描述                          \\
|\only<2>{\highlightline}|    \midrule
|\only<3>{\highlightline}|冰墩墩 & 2022 年北京冬季奥运会吉祥物  \\
|\only<3>{\highlightline}|雪容融 & 2022 年北京冬季残奥会吉祥物  \\
|\only<2>{\highlightline}|    \bottomrule
|\only<1>{\highlightline}|  \end{tabular}
\end{table}
\end{document}
      \end{codeblock}
    \end{column}
    \begin{column}{0.4\textwidth}
      \only<1>{
        使用 \env{tabular} 环境绘制表格。需要添加参数(称为\textbf{表格导言})以确定每一列的对齐方式。引入 \pkg{array} 宏包来使用更多的\textbf{引导符}。
        \begin{center}
          \footnotesize
          \begin{stampbox}
            \begin{tabular}{>{\ttfamily}ll}
              \alert{l} & 向左对齐 \\
              \alert{c} & 居中对齐 \\
              \alert{r} & 向右对齐 \\
              \alert{p\{3cm\}} & 固定列宽,两端对齐 \\
              \alert{m\{3cm\}} & \texttt{p} + 垂直居中对齐 \\
              \alert{>\{\textbackslash{}bfseries\}} & 后一列单元格前加命令 \\
              \alert{*\{3\}\{l\}} & 三个左对齐列 \\
            \end{tabular}
          \end{stampbox}
        \end{center}
      }
      \only<2>{
        \pkg{booktabs} 宏包提供了标准三线表格所需要的行分割线:\cmd{toprule},\cmd{midrule},\cmd{bottomrule}。也可以使用 \cmd{cmidrule\{1-2\}} 来部分地绘制行分割线。一般不推荐在专业表格中使用纵向分割线。
      }
      \only<3>{
        每行内容使用 \textbackslash\textbackslash{} 分隔开,每行中的单元格使用 \& 分隔开。
      }
      \only<4>{
        \includepdflarge{table}
      }
    \end{column}
  \end{columns}
\end{frame}

\begin{frame}[fragile]%
  \begin{columns}
    \begin{column}{0.6\textwidth}
      \begin{codeblock}[]{表头居中}
\documentclass{ctexart}
\usepackage{array,booktabs}
\begin{document}
\begin{table}[ht]
  \centering
  \caption{||北京冬奥会吉祥物}
  \begin{tabular}{lp{3cm}}
    \toprule
|\highlightline|\multicolumn{1}{c}{||吉祥物} &
|\highlightline|\multicolumn{1}{c}{||描述} \\
    \midrule
||冰墩墩 & 2022 年北京冬季奥运会吉祥物  \\
||雪容融 & 2022 年北京冬季残奥会吉祥物  \\
    \bottomrule
  \end{tabular}
\end{table}
\end{document}
      \end{codeblock}
    \end{column}
    \begin{column}{0.4\textwidth}
      \cmd{multicolumn} 命令不仅可以用于合并同行的单元格,还可以用于临时地屏蔽表格导言对该列的对齐方式定义。这里用于居中表头。
      \begin{center}
        \begin{stampbox}
          \parbox{0.85\linewidth}{
            \ttfamily\color{blue}\textbackslash{}multicolumn\{格数\}\{对齐方式\}\{内容\}
          }
        \end{stampbox}
      \end{center}
      跨页表格考虑使用 \pkg{longtable} 宏包。带标注的表格可以考虑使用 \pkg{threeparttable} 宏包。考虑使用在线工具生成表格代码 \link{https://www.tablesgenerator.com/latex_tables}。
    \end{column}
  \end{columns}
\end{frame}

\section{数学公式}
\begin{frame}
  \frametitle{数学模式}
  \begin{alertblock}{}
  输入数学公式是 \LaTeX{} 的绝对强项,很多常见的公式服务依赖于一些轻量级渲染引擎比如 MathJax, K\kern-.3ex\raise.4ex\hbox{\footnotesize A}\kern-.3ex\TeX{}。但是它们实际上使用的是 \LaTeX{} 语法变种,也就是说并没有使用 \LaTeX{} 后端。所以不要期望有完全一致的输出。
  \end{alertblock}
  
  为了更好的获得数学公式输入支持,请使用 \hologo{AmS}math 宏包。数学模式分为两种:
  \begin{description}
    \item[行内模式] 一般通过一对美元符号(\$$\cdots$\$)标记,可以使用编辑器内建的符号表输入数学符号,也可以使用在线工具手写识别 \link{https://detexify.kirelabs.org/classify.html}。
    \item[行间模式] 一般通过 \env{equation} 环境\footnote{这是有编号环境,其加星号的变种 \env{equation*} 用于生成无编号环境。}输入。如果需要使用多行公式,请使用 \env{align} 环境,并按照类似表格输入的方式,使用 \& 对齐符号,\textbackslash\textbackslash{} 换行。如果不想手动居中,可以考虑多行自动居中的 \env{gather} 和单个大型公式首尾两端对齐 \env{multline}。
  \end{description}
\end{frame}

\begin{frame}
  \frametitle{数学命令展示}
  \begin{columns}
    \begin{column}{0.33\textwidth}
      \begin{exampleblock}{}
        $PV=nRT$
      \end{exampleblock}
      \begin{exampleblock}{}
        $\sum_{i=1}^ki^2=\frac{n(n+1)(2n+1)}{6}$
      \end{exampleblock}
      \begin{exampleblock}{}
        $T(n) = aT\left(\left\lceil\frac{n}{b}\right\rceil\right) + \mathcal{O}(n^d)$
      \end{exampleblock}
      \begin{exampleblock}{}
        $\frac{x_{1}+x_{2}+x_{3}}{3}\geq \sqrt[3]{x_{1}x_{2}x_{3}}$
      \end{exampleblock}
      \begin{exampleblock}{}
        $n=(\underbrace{1\cdots 1}_{k\text{ of 1's}})_2=2^{k+1}-1$
      \end{exampleblock}
      \begin{exampleblock}{}
        $\nabla f (P)= \left.\left(\frac{\partial f}{\partial x},\frac{\partial f}{\partial y},\frac{\partial f}{\partial z}\right)\right|_{P}$
      \end{exampleblock}
    \end{column}
    \begin{column}{0.67\textwidth}
      \begin{exampleblock}{}
        \begin{equation*}
          \int_{a}^b f(x)\,\mathrm{d}x=\lim_{|P|\rightarrow 0}\sum_{i=1}^n f(\xi_i)\Delta x_i
        \end{equation*}
      \end{exampleblock}
      \begin{exampleblock}{}
        \begin{equation}
          T(n) = \begin{cases}
            \mathcal{O}(n^d),&\textrm{if } d>\log_b a, \\
            \mathcal{O}(n^d\log n), &\textrm{if } d=\log_b a,\\
            \mathcal{O}(n^{\log_b a}), &\textrm{if } d<\log_b a.
          \end{cases}
        \end{equation}
      \end{exampleblock}
      \begin{exampleblock}{}
        \begin{align}
          Q^{T}A&=R \\
          \begin{pmatrix}
            q_1^T \\ q_2^T \\ q_3^T
          \end{pmatrix}
          \begin{pmatrix}
            a_1 & a_2 & a_3
          \end{pmatrix}
          &=R
        \end{align}
      \end{exampleblock}
    \end{column}
  \end{columns}
\end{frame}

%更深入地讲解 mathtools, unicode-math, siunix

\section{引用}
\begin{frame}[fragile]
  \frametitle{交叉引用}
  \only<1>{
    正如之前所提到的,\LaTeX{} 中使用 \cmd{label} 标记,然后可以使用 \cmd{ref} 来引用这个标记。 \cmd{label} 可以放在使用计数器的对象之后。
  }
  \only<2>{
    为了使得对公式编号的引用带有括号,推荐使用 \hologo{AmS}math 宏包中的 \cmd{eqref} 命令。对于多行公式环境,每一个换行符前都可以添加一个 \cmd{label} 用于引用该行公式。
  }
  \begin{columns}
    \begin{column}{0.5\textwidth}
      \begin{codeblock}[]{图}
\begin{figure}
|\only<1>{\highlightline}|  \caption{||示例}\label{fig:example}
\end{figure}
      \end{codeblock}
      \begin{codeblock}[]{表}
\begin{table}
|\only<1>{\highlightline}|  \caption{||示例}\label{tab:example}
\end{table}
      \end{codeblock}
    \end{column}
    \begin{column}{0.5\textwidth}
\begin{codeblock}[]{目次}
|\only<1>{\highlightline}|\section{||示例}\label{sec:example}
\end{codeblock}

\begin{codeblock}[]{公式}
\begin{equation}
  a = b + c
|\only<1>{\highlightline}|\label{eq:example}
\end{equation}
|\only<2>{\highlightline}|如公式 \eqref{eq:example} 所示,
\end{codeblock}
    \end{column}
  \end{columns}
\end{frame}

\begin{frame}[fragile]
  \frametitle{文献引用}
  \LaTeX{} 管理参考文献可以采用专用数据库文件 \texttt{.bib},很多的文献管理文件比如 EndNote \link{https://lic.sjtu.edu.cn/Default/softshow/tag/MDAwMDAwMDAwMLGImKE}, Zotero \link{https://www.zotero.org/}, JabRef \link{https://www.jabref.org/} 都可以直接导出这种格式的文件用于 \LaTeX{} 论文中的引用。一般不需要手写数据库文件,你只需要注意每一个文献会在数据库中有一个主键,这个类似于上文的 \cmd{label} 标记,只是要引用数据库中的文献需要使用 \cmd{cite} 命令。
  
  \begin{codeblock}[]{ref.bib}
|\highlightline|@phdthesis{devoftech,|\hfill\alert{\% 类型为博士论文,主键为\texttt{devoftech}}|
  title={||新时期我国信息技术产业的发展},
  author={||江泽民},
  year={2008}
}
  \end{codeblock}
\end{frame}

\begin{frame}
  \frametitle{文献引用}
  而让 \LaTeX{} 处理 \texttt{.bib} 数据库文件与引用有两种工作流。你可能需要查清楚如何在编辑器中设置对应的工作流,或者采用后面所提到的高级编译方式 \texttt{latexmk}。
  \begin{columns}
    \begin{column}{0.5\textwidth}
      \begin{block}{\hologo{BibTeX} + \pkg{gbt7714}}
        一般期刊提交使用这种方法,\pkg{natbib} 宏包将提供命令 \cmd{citet}(文本引用) 和 \cmd{citep}(括号引用)。中文引用可以直接使用 \pkg{gbt7714} 宏包,但是角模式和正文模式不能混用。
      \end{block}
    \end{column}
    \begin{column}{0.5\textwidth}
      \begin{block}{\hologo{biber} + \pkg{biblatex}}
        这是更容易自定义的方法,与 \hologo{BibTeX} 的运作方式稍有不同。\pkg{biblatex} 提供了更加智能的引用命令。而中文引用可以使用 \pkg{biblatex} 宏包的样式 \pkg{gb7714-2015},使用该样式需要使用 \hologo{XeLaTeX} 编译。
      \end{block}
    \end{column}
  \end{columns}
\end{frame}

\begin{frame}[fragile]
  \frametitle{文献引用}
  \begin{columns}
    \begin{column}{0.5\textwidth}
      \begin{codeblock}[]{\hologo{BibTeX} + \pkg{gbt7714}}
\documentclass{ctexart}
\usepackage{gbt7714}
\bibliographystyle{gbt7714-numerial}
% \citestyle{numbers}  % 正文模式
\begin{document}
  ||他指出了...\cite{devoftech}
  \bibliography{ref}
\end{document}
      \end{codeblock}
    \end{column}
    \begin{column}{0.5\textwidth}
      \begin{codeblock}[]{\hologo{biber} + \pkg{biblatex}}
\documentclass{ctexart}
\usepackage[backend=biber,style=gb7714-2015]{biblatex}
\addbibresource{ref.bib}
\begin{document}
  ||他在文献 \parencite{devoftech}
  ||指出了...\cite{devoftech}
  \printbibliography
\end{document}
      \end{codeblock}
    \end{column}
  \end{columns}
\end{frame}

\begin{frame}
  \frametitle{文献引用}
  \begin{columns}
    \begin{column}{0.5\textwidth}
      \includepdflarge{bibtex}
    \end{column}
    \begin{column}{0.5\textwidth}
      \includepdflarge{biblatex}
    \end{column}
  \end{columns}
\end{frame}

} % End of customized shaded number logo

  % !TeX root = ..\..\latex-talk.tex

\part{SJTUThesis}

\begin{frame}
  \frametitle{简介}
  \begin{columns}
    \begin{column}{0.6\textwidth}
      \begin{itemize}
        \item 最早由韦建文于 2009 年 11 月发布 0.1a 版,2018 年起由 SJTUG 接手维护
        \item 最新版:\SJTUThesisVersion{} (\SJTUThesisDate)
        \item 支持本科、硕士、博士学位论文以及课程论文的排版
      \end{itemize}
    \end{column}
    \begin{column}{0.4\textwidth}
      \begin{exampleblock}{}
        \begin{minipage}[c]{1cm}
          \includegraphics[width=0.8cm]{\getcontribpath{sjtug}{vi/sjtug}}
        \end{minipage}
        \begin{minipage}[c]{2cm}
          \href{https://github.com/sjtug}{sjtug}/\href{https://github.com/sjtug/SJTUThesis}{SJTUThesis}
        \end{minipage}
      \end{exampleblock}
      \vspace{-8pt}
      \begin{block}{}
        \scriptsize
        上海交通大学 \hologo{XeLaTeX} 学位论文及课程论文模板 | Shanghai Jiao Tong University \hologo{XeLaTeX} Thesis Template
      \end{block}
      \vspace{-8pt}
      \begin{alertblock}{}
        \scriptsize
        \begin{tabular}{cl}
          \faStar & 2.4k \\
          \faEye & 55 \\
          \faCodeBranch & 701 \\
        \end{tabular}
      \end{alertblock}
    \end{column}
  \end{columns}
\end{frame}

\begin{frame}
  \frametitle{下载与编译}
  \alert{下载} 推荐安装 Git \link{https://git-scm.com/} 后,克隆 SJTUG 镜像仓库
  \begin{exampleblock}{\faGit*}
    \ttfamily\small
    git clone https://mirror.sjtu.edu.cn/git/SJTUThesis.git/
  \end{exampleblock}

  \alert{编译} 推荐使用 \pkg{latexmk} 编译\footnote{\hologo{MiKTeX} 用户需要手动安装 Perl 解释器 \link{https://www.perl.org/get.html} 才能使用 \pkg{latexmk}。},在不能够利用自带的 \texttt{.latexmkrc} 配置文件的情况下,需要查清楚在对应的编辑器中如何使用 \hologo{XeLaTeX} + \hologo{biber} 编译 \link{https://github.com/sjtug/SJTUThesis/blob/master/README.md}。
  \begin{exampleblock}{\faTerminal}
    \ttfamily\small
    latexmk -xelatex main
  \end{exampleblock}

  Overleaf 用户可以下载压缩包后,上传并采用 \hologo{XeLaTeX} 编译方式。
\end{frame}

\begin{frame}
  \frametitle{手动编译}
  \alert{第一次编译失败} 如果没有办法通过通常方式编译成功,请尝试使用文件夹内附带 \faLinux{}\,\faApple{} \texttt{Makefile} 和 \faWindows{} \texttt{Compile.bat} 进行编译。

  \alert{统计字数} 编写过程中也可以使用对应的命令调用 \TeX{}count 来统计正文字数。
  \begin{columns}
    \begin{column}{0.38\textwidth}
      \begin{exampleblock}{\faLinux{}\,\faApple}
        \ttfamily
        make all\\
        make clean\\
        make cleanall\\
        make wordcount
      \end{exampleblock}
    \end{column}
    \begin{column}{0.38\textwidth}
      \begin{exampleblock}{\faWindows}
        \ttfamily
        ./Compile.bat thesis\\
        ./Compile.bat clean\\
        ./Compile.bat cleanall\\
        ./Compile.bat wordcount
      \end{exampleblock}
    \end{column}
    \begin{column}{0.24\textwidth}
      \begin{block}{\faInfo}
        \ttfamily
        编译论文\\
        清理中间文件\\
        $\hookrightarrow +$删除论文\\
        统计字数
      \end{block}
    \end{column}
  \end{columns}
\end{frame}

\begin{frame}[label=compile]
  \frametitle{编译问题排查}
  \begin{columns}
    \begin{column}{0.33\textwidth}
      \begin{alertblock}{无法使用 \texttt{latexmk}\thesisissue{578}}
        \hologo{MiKTeX} 需要安装 Perl 解释器。
      \end{alertblock}  
      \begin{alertblock}{C\TeX{} 套装无法编译\thesisissue{446}}
        使用最新 \TeX{} 发行版。
      \end{alertblock}
      \begin{alertblock}{\hologo{pdfLaTeX} 无法编译\thesisissue{444}}
        请使用 \texttt{latexmk},或更改编辑器设置以 \hologo{XeLaTeX} 编译。
      \end{alertblock}
    \end{column}
    \begin{column}{0.33\textwidth}
      \begin{alertblock}{缺少字体\thesisissue{564} \thesisdiscuss{598}}
        更换字体集,或者安装对应字体。
      \end{alertblock}
      \begin{alertblock}{缺少汉字\thesisissue{533} \thesisdiscuss{617}}
        去除使用 fandol 字体集的设定。或者是安装字体后,改用 \texttt{fontset=adobe} 或 \texttt{fontset=founder}。
      \end{alertblock}
    \end{column}
    \begin{column}{0.33\textwidth}
      \begin{block}{\faInfoCircle{} README}
        不同编辑器的设置请首先参阅 README \link{https://github.com/sjtug/SJTUThesis/blob/master/README.md} 文档。
      \end{block}
      \begin{block}{\faBookOpen{} Wiki}
        其他编译问题推荐查阅 Wiki \link{https://github.com/sjtug/SJTUThesis/wiki} 的使用说明部分。
      \end{block}
    \end{column}
  \end{columns}
\end{frame}

\begin{frame}[fragile, label=covers]
  \begin{codeblock}[firstnumber=3]{main.tex}
|\alert{\% 载入 SJTUThesis 模版}|
\documentclass[|\only<1>{\highlight{type}}\only<2>{type}|=|\only<1>{bachelor}\only<2>{\highlight{bachelor}}|]{sjtuthesis}
  \end{codeblock}
  \begin{figure}
    \parbox{0.9\textwidth}{
      \begin{subfigure}{0.20\textwidth}
        \framebox{\includegraphics[width=\linewidth]{support/thesis/bachelor}}
        \caption{\only<1>{学士}\only<2>{\texttt{bachelor}}}
      \end{subfigure}\hfill
      \begin{subfigure}{0.20\textwidth}
        \framebox{\includegraphics[width=\linewidth]{support/thesis/master}}
        \caption{\only<1>{硕士}\only<2>{\texttt{master}}}
      \end{subfigure}\hfill
      \begin{subfigure}{0.20\textwidth}
        \framebox{\includegraphics[width=\linewidth]{support/thesis/doctor}}
        \caption{\only<1>{博士}\only<2>{\texttt{doctor}}}
      \end{subfigure}\hfill
      \begin{subfigure}{0.20\textwidth}
        \framebox{\includegraphics[width=\linewidth]{support/thesis/course}}
        \caption{\only<1>{课程}\only<2>{\texttt{course}}}
      \end{subfigure}
      \caption{论文类型示例\only<2>{ \texttt{type}}}
    }
  \end{figure}
\end{frame}

\begin{frame}[fragile]
  \frametitle{文档类选项}
  % \framesubtitle{\textbackslash{}documentclass\{sjtuthesis\}}
  \begin{columns}
    \begin{column}{0.45\textwidth}
      \includegraphics[page=10]{thesisdir}
    \end{column}
    \begin{column}{0.55\textwidth}
      \begin{table}[H]
        \caption{文档类选项}
        \footnotesize
        \begin{tabular}{>{\ttfamily}rll}
          \toprule
          选项 & 含义 & 相关 \\
          \midrule
          type= & 指定论文类型 & 第 \ref{covers} 页\\
          fontset= & 指定字体 & 第 \ref{compile} 页\\
          \midrule
          review & 开启盲审模式 & \thesisissue{195} \thesisissue{686} \\
          twoside & 双页模式 & \thesisissue{554} \\
          oneside & 单页模式 & \thesisissue{694} \\
          openright & 章从奇数页开始 & \thesisdiscuss{724} \\
          openany & 章从任意页开始 & \thesisissue{446} \\
          \bottomrule
        \end{tabular}
      \end{table}
    \end{column}
  \end{columns}
\end{frame}

\begin{frame}[fragile]
  \frametitle{基本配置}
  \framesubtitle{\textbackslash{}input\{setup\}}
  \begin{columns}
    \begin{column}{0.45\textwidth}
      \includegraphics[page=9]{thesisdir}
    \end{column}
    \begin{column}{0.55\textwidth}
      \begin{codeblock}[firstnumber=12]{main.tex}
|\highlightline<1>|% 论文基本配置,加载宏包等全局配置
|\highlightline<1>|\input{setup}

\begin{document}

%TC:ignore

|\highlightline<2>|% 标题页
|\highlightline<2>|\maketitle
      \end{codeblock}
      \visible<2>{
        \cmd{sjtusetup} 中的 \pkg{info} 将会修改封面的信息设置(见第 \ref{covers} 页)。
      }
    \end{column}
  \end{columns}
\end{frame}

\begin{frame}[fragile]
  \frametitle{基本配置}
  \framesubtitle{\textbackslash{}sjtusetup}
  \begin{columns}
    \begin{column}{0.45\textwidth}
      \includegraphics[page=1]{thesisdir}
    \end{column}
    \begin{column}{0.55\textwidth}
      \begin{codeblock}[firstnumber=3]{setup.tex}
\sjtusetup{
  info = {
    title    = {||上海交通大学学位论文 \LaTeX{} 模板示例文档},
    title*   = {A Sample for \LaTeX-based SJTU Thesis Template},
    author   = {||某\quad{}某},
    author* = {Mo Mo},
  },
  style = { header-logo-color = red, 
  },
  name = {
    publications = {||攻读学位期间完成的论文},
  },
}
      \end{codeblock}
    \end{column}
  \end{columns}
\end{frame}

\begin{frame}
  \frametitle{基本配置}
  \framesubtitle{\textbackslash{}sjtusetup}
  \begin{columns}
    \begin{column}{0.45\textwidth}
      \includegraphics[page=1]{thesisdir}
    \end{column}
    \begin{column}{0.55\textwidth}
      \begin{table}[H]
        \centering
        \caption{info 域}
        \footnotesize
        \begin{tabular}{lll} \toprule
          命令作用 & 中文对应选项 & 英文对应选项 \\ \midrule
          论文标题 & \texttt{title} & \texttt{title*} \\
          关键字列表 & \texttt{keywords} & \texttt{keywords*} \\
          作者姓名&  \texttt{author} &\texttt{author*}\\
          申请学位名称 & \texttt{degree}&\texttt{degree*}\\
          院系名称 & \texttt{department} & \texttt{department*}\\
          专业名称 & \texttt{major} & \texttt{major*}\\
          导师 & \texttt{supervisor} & \texttt{supervisor*}\\
          副导师 & \texttt{assisupervisor} & \texttt{assisupervisor*}\\
          日期 & \multicolumn{2}{c}{\texttt{date}}\\
          学号 & \multicolumn{2}{c}{\texttt{id}}\\ \bottomrule
          \end{tabular}
      \end{table}
    \end{column}
  \end{columns}
\end{frame}

\begin{frame}[fragile]
  \frametitle{版权页}
  \framesubtitle{\textbackslash{}copyrightpage}
  \begin{columns}
    \begin{column}{0.45\textwidth}
      \only<1>{
        \includegraphics[page=9]{thesisdir}
      }
      \only<2>{
        \includegraphics[page=2]{thesisdir}
      }
      \only<3>{
        \begin{figure}[H]
          \framebox{\includegraphics[page=2,width=0.4\linewidth]{bachelor}}
          \caption{版权页}
        \end{figure}
      }
    \end{column}
    \begin{column}{0.55\textwidth}
      \begin{codeblock}[firstnumber=22]{main.tex}
|\highlightline<1>|% 原创性声明及使用授权书
|\highlightline<1>|\copyrightpage
|\highlightline<2>|% 插入外置原创性声明及使用授权书
|\highlightline<2>|% \copyrightpage[scans/sample-copyright-old.pdf]
      \end{codeblock}
      \only<1>{
        \cmd{copyrightpages} 可以用于插入版权页。
      }
      \only<2>{
        \cmd{copyrightpages} 也接受一个可选参数,用于直接使用扫描件。
      }
    \end{column}
  \end{columns}
\end{frame}

\begin{frame}[fragile]
  \frametitle{前置部分}
  \framesubtitle{\textbackslash{}frontmatter}
  \begin{columns}
    \begin{column}{0.45\textwidth}
      \only<1>{
        \includegraphics[page=9]{thesisdir}
      }
      \only<2>{
        \includegraphics[page=3]{thesisdir}
      }
      \only<3>{
        \begin{figure}[H]
          \begin{subfigure}{0.45\textwidth}
            \framebox{\includegraphics[page=3,width=\linewidth]{bachelor}}
            \caption{中文}
          \end{subfigure}\hfill
          \begin{subfigure}{0.45\textwidth}
            \framebox{\includegraphics[page=4,width=\linewidth]{bachelor}}
            \caption{英文}
          \end{subfigure}
          \caption{摘要}
        \end{figure}
      }
      \only<4>{
        \begin{figure}[H]
          \begin{subfigure}{0.30\linewidth}
            \centering
            \framebox{\includegraphics[page=5,width=0.6\linewidth]{bachelor}}
            \caption{目录}
          \end{subfigure}
          \begin{subfigure}{0.30\linewidth}
            \centering
            \framebox{\includegraphics[page=6,width=0.6\linewidth]{bachelor}}
            \caption{插图}
          \end{subfigure}

          \begin{subfigure}{0.30\linewidth}
            \centering
            \framebox{\includegraphics[page=7,width=0.6\linewidth]{bachelor}}
            \caption{表格}
          \end{subfigure}
          \begin{subfigure}{0.30\linewidth}
            \centering
            \framebox{\includegraphics[page=8,width=0.6\linewidth]{bachelor}}
            \caption{算法}
          \end{subfigure}
          \caption{索引}
        \end{figure}
      }
      \only<5>{
        \includegraphics[page=4]{thesisdir}
      }
      \only<6>{
        \begin{figure}[H]
          \framebox{\includegraphics[page=9,width=0.5\linewidth]{bachelor}}
          \caption{符号对照表}
        \end{figure}
      }
    \end{column}
    \begin{column}{0.55\textwidth}
      \begin{codeblock}[firstnumber=30]{main.tex}
|\highlightline<2-3>|% 摘要
|\highlightline<2-3>|\input{contents/abstract}

|\highlightline<4>|% 目录
|\highlightline<4>|\tableofcontents
|\highlightline<4>|% 插图索引
|\highlightline<4>|\listoffigures*
|\highlightline<4>|% 表格索引
|\highlightline<4>|\listoftables*
|\highlightline<4>|% 算法索引
|\highlightline<4>|\listofalgorithms*

|\highlightline<5-6>|% 符号对照表
|\highlightline<5-6>|\input{contents/nomenclature}
      \end{codeblock}
    \end{column}
  \end{columns}
\end{frame}

\begin{frame}[fragile]
  \frametitle{主体部分}
  \framesubtitle{\textbackslash{}mainmatter}
  \begin{columns}
    \begin{column}{0.45\textwidth}
      \only<1>{
        \includegraphics[page=5]{thesisdir}
      }
      \only<2>{
        \begin{figure}[H]
          \begin{subfigure}{0.30\linewidth}
            \centering
            \framebox{\includegraphics[page=11,width=0.6\linewidth]{bachelor}}
            \caption{简介}
          \end{subfigure}
          \begin{subfigure}{0.30\linewidth}
            \centering
            \framebox{\includegraphics[page=13,width=0.6\linewidth]{bachelor}}
            \caption{数学}
          \end{subfigure}

          \begin{subfigure}{0.30\linewidth}
            \centering
            \framebox{\includegraphics[page=16,width=0.6\linewidth]{bachelor}}
            \caption{浮动体}
          \end{subfigure}
          \begin{subfigure}{0.30\linewidth}
            \centering
            \framebox{\includegraphics[page=22,width=0.6\linewidth]{bachelor}}
            \caption{总结}
          \end{subfigure}
          \caption{主体部分}
        \end{figure}
      }
    \end{column}
    \begin{column}{0.55\textwidth}
      \begin{codeblock}[firstnumber=47]{main.tex}
|\highlightline|% 正文内容
|\highlightline|\input{contents/intro}
|\highlightline|\input{contents/math_and_citations}
|\highlightline|\input{contents/floats}
|\highlightline|\input{contents/summary}

%TC:ignore

% 参考文献
\printbibliography[heading=bibintoc]
      \end{codeblock}
    \end{column}
  \end{columns}
\end{frame}

\begin{frame}
  \frametitle{数学}
  \begin{itemize}
    \item 公式示例:\nolinkurl{contents/math_and_citations.tex}
    \item \SJTUThesis{} 定义了常用的数学环境(需要手工引入 \texttt{ntheorem} 宏包):
      \begin{table}[h]
        \centering
        \footnotesize
        \begin{tabular}{*{7}{l}}\toprule
          assumption  & axiom   & conjecture & corollary    & definition  & example   & exercise  \\
          假设        & 公理    & 猜想       & 推论         & 定义        & 例        & 练习      \\\midrule
          lemma       & problem & proof      & proposition  & remark      & solution  & theorem   \\
          引理        & 问题    & 证明       & 命题         & 注          & 解        & 定理      \\\bottomrule
        \end{tabular}
      \end{table}
      \item \SJTUThesis{} 可以通过 \texttt{unimath} 选项使用 \pkg{unicode-math} 进行数学输入,注意与传统方式的区别。\thesisissue{555}
  \end{itemize}
\end{frame}

\begin{frame}[fragile]
  \frametitle{参考文献}
  \begin{columns}
    \begin{column}{0.45\textwidth}
      \includegraphics[page=6]{thesisdir}
    \end{column}
    \begin{column}{0.55\textwidth}
      \begin{codeblock}[firstnumber=111,numbersep=2pt]{setup.tex}
% 使用 BibLaTeX 处理参考文献
%   biblatex-gb7714-2015 常用选项
%     gbnamefmt=lowercase     姓名大小写由输入信息确定
%     gbpub=false             禁用出版信息缺失处理
\usepackage[backend=biber,style=gb7714-2015]{biblatex}
% 文献表字体
% \renewcommand{\bibfont}{\zihao{-5}}
% 文献表条目间的间距
\setlength{\bibitemsep}{0pt}
|\highlightline|% 导入参考文献数据库
|\highlightline|\addbibresource{bibdata/thesis.bib}
      \end{codeblock}
    \end{column}
  \end{columns}
\end{frame}

\begin{frame}[fragile]
  \frametitle{附录}
  \framesubtitle{\textbackslash{}appendix}
  \begin{columns}
    \begin{column}{0.45\textwidth}
      \only<1>{
        \includegraphics[page=7]{thesisdir}
      }
      \only<2>{
        \begin{figure}[H]
          \begin{subfigure}{0.45\linewidth}
            \framebox{\includegraphics[width=\linewidth,page=24]{bachelor}}
            \caption{}
          \end{subfigure}\hfill
          \begin{subfigure}{0.45\textwidth}
            \framebox{\includegraphics[width=\linewidth,page=25]{bachelor}}
            \caption{}
          \end{subfigure}
          \caption{附录}
        \end{figure}
      }
    \end{column}
    \begin{column}{0.55\textwidth}
      \begin{codeblock}[firstnumber=61]{main.tex}
% 附录中图表不加入索引
\captionsetup{list=no}

% 附录内容
|\highlightline|\input{contents/app_maxwell_equations}
|\highlightline|\input{contents/app_flow_chart}
      \end{codeblock}
    \end{column}
  \end{columns}
\end{frame}

\begin{frame}[fragile]
  \frametitle{结尾部分}
  \framesubtitle{\textbackslash{}backmatter}
  \begin{columns}
    \begin{column}{0.45\textwidth}
      \only<1>{
        \includegraphics[page=8]{thesisdir}
      }
      \only<2>{
        \begin{figure}[H]
          \begin{subfigure}{0.30\linewidth}
            \centering
            \framebox{\includegraphics[page=26,width=0.6\linewidth]{bachelor}}
            \caption{致谢}
          \end{subfigure}
          \begin{subfigure}{0.30\linewidth}
            \centering
            \framebox{\includegraphics[page=27,width=0.6\linewidth]{bachelor}}
            \caption{成就}
          \end{subfigure}

          \begin{subfigure}{0.30\linewidth}
            \centering
            \framebox{\includegraphics[page=28,width=0.6\linewidth]{bachelor}}
            \caption{简历}
          \end{subfigure}
          \begin{subfigure}{0.30\linewidth}
            \centering
            \framebox{\includegraphics[page=29,width=0.6\linewidth]{bachelor}}
            \caption{大摘要*}
          \end{subfigure}
          \caption{结尾部分}
        \end{figure}
      }
    \end{column}
    \begin{column}{0.55\textwidth}
      \begin{codeblock}[firstnumber=76]{main.tex}
% 致谢
\input{contents/acknowledgements}

% 发表论文及科研成果
% 盲审论文中,发表论文及科研成果等仅以第几作者注明即可,不要出现作者或他人姓名
\input{contents/achievements}

% 简历
\input{contents/resume}

% 学士学位论文要求在最后有一个大摘要,单独编页码
\input{contents/digest}
      \end{codeblock}
    \end{column}
  \end{columns}
\end{frame}

\begin{frame}
  \frametitle{还有其他问题?}
  \begin{columns}
    \begin{column}{0.75\textwidth}
    \begin{itemize}
      \item[{\faComment*[regular]}] 日常模板或 \LaTeX{} 使用问题可以前往 Discussions \link{https://github.com/sjtug/SJTUThesis/discussions} 提问
      
      (解决后别忘了 \textcolor{green}{\faCheckCircle{} Mark as answer}
      \item[{\faDotCircle[regular]}] 如果是 \textsc{SJTUThesis} 项目本身的 bug 和 feature request
      
      可以通过 Issues \link{https://github.com/sjtug/SJTUThesis/issues} 反馈。
      \item[{\faCodeBranch}] 如果你有好点子,可以贡献代码
     
      向 \textsc{SJTU\TeX{}}(v1) \link{https://github.com/sjtug/SJTUTeX/tree/v1} 存储库发 PR,\par
      而后把解包结果同步到 \textsc{SJTUThesis}。
  
      \item[{\faTag}] 如果你对正在基于 \LaTeX3 开发的新版感兴趣,\par
      也欢迎向 \textsc{SJTU\TeX{}}(v2) \link{https://github.com/sjtug/SJTUTeX/tree/v2} 发 PR。
  
      \item[{\faQq}] 也欢迎在 QQ 群即时讨论。
    \end{itemize}
    \end{column}
    \begin{column}{0.25\textwidth}
      \includegraphics[height=0.7\textheight]{qq.jpg}
    \end{column}
  \end{columns}
\end{frame}
\end{document}
      \end{codeblock}
    \end{column}
  \end{columns}
  \footnotetext{如果想强制指定子文档的主文档,可以在文件第一行输入魔术命令:\texttt{\% !TeX root = main.tex}}
\end{frame}

\section{图}
\begin{frame}[fragile]%
  \frametitle{\temporal<5>{插图}{浮动体}{插图}}
  \begin{columns}
    \begin{column}{0.6\textwidth}
      \begin{codeblock}[]{插入单图\only<4->{最佳实践}}
\documentclass{ctexart}
|\only<2>{\highlightline}|\usepackage{graphicx}
|\only<2>{\highlightline}|\graphicspath{{figs/}{pics/}}
\begin{document}
|\only<5>{\highlightline}|\begin{figure}[ht]
|\only<6>{\highlightline}|  \centering
|\only<3>{\highlightline}|  \includegraphics[width=|\only<1-3>{4cm}\only<4->{0.4\textbackslash{}textwidth}|]{sjtug}
|\only<7>{\highlightline}|  \caption{SJTUG 徽标}\label{fig:sjtug}
|\only<5>{\highlightline}|\end{figure}
\end{document}
      \end{codeblock}
    \end{column}
    \begin{column}{0.4\textwidth}
      \only<1>{
        \includepdflarge{insertimage}
      }
      \only<2>{
        为了插入外部图片,需要使用 \pkg{graphicx} 宏包。之后在文档主体便可以使用 \cmd{includegraphics} 插入图片。导言区也可以加入 \cmd{graphicspath} 指定图片文件夹\footnotemark。
      }
      \only<3>{
        \cmd{includegraphics} 命令便以相对路径的方式插入图片,如果无同名图片,那么后缀名可以省略。可以使用可选参数指定插入的图片尺寸,最佳实践是使用 \cmd{textwidth} 或 \cmd{linewidth} 的相对值指定尺寸大小,以在未来可能的布局更改中保留一定的灵活性。
      }
      \only<4>{
        也可以通过键值对的方法设置图片的其他属性。
        \begin{center}
          \footnotesize
          \begin{stampbox}
            \begin{tabular}{rl}
              \pkg{width} & 宽度 \\
              \pkg{height} & 高度 \\
              \pkg{scale} & 缩放 \\
              \pkg{angle} & 角度 \\
            \end{tabular}
          \end{stampbox}
        \end{center}
      }
      \only<5>{
        \env{figure} 为一个浮动体环境(\env{table} 也是),可以将其移动到其他位置上以减少行文中的空白。可以添加可选参数以指定如何放置浮动体,最多可以使用四种位置描述符:
        \begin{center}
          \footnotesize
          \begin{stampbox}
            \begin{tabular}{cll}
              \pkg{h} & Here & 尽可能在这里 \\
              \pkg{t} & Top & 页面顶部 \\
              \pkg{b} & Bottom & 页面底部 \\
              \pkg{p} & Page & 浮动体专页 \\
            \end{tabular}
          \end{stampbox}
        \end{center}
        还可以只使用 \pkg{float} 宏包提供的 \pkg{H} 描述符以强制置于此处。
      }
      \only<6>{
        采用 \cmd{centering} 命令而不是 \env{center} 环境来水平居中图片。这将避免多余的纵向间距。
      }
      \only<7>{
        使用 \cmd{caption} 命令输入题注,如果这个命令写在插入图片前面,题注将会在上方(中文中一般对 \env{table} 环境这么做)。后面将会看到如何对留有标记(\cmd{label})的图片进行引用。
      }
    \end{column}
  \end{columns}
  \only<2>{\footnotetext{其命令参数每个为一个以 \texttt{/} 结尾的文件夹,每个文件夹需要使用大括号包裹起来。}}
\end{frame}

\begin{frame}[fragile]
  \begin{columns}
    \begin{column}{0.6\textwidth}
      \begin{codeblock}[]{插入双图}
\documentclass{ctexart}
\usepackage{graphicx}
\graphicspath{{figs/}{pics/}}
\begin{document}
  \begin{figure}[ht]
|\only<1>{\highlightline}|    \begin{minipage}{0.48\textwidth}
      \centering
      \includegraphics[height=2cm]{sjtug}
|\only<2>{\highlightline}|      \caption{SJTUG 徽标}\label{fig:sjtug}
|\only<1>{\highlightline}|    \end{minipage}\hfill
|\only<1>{\highlightline}|    \begin{minipage}{0.48\textwidth}
      \centering
      \includegraphics[height=2cm]{sjtugt}
|\only<2>{\highlightline}|      \caption{SJTUG||文字}\label{fig:sjtugt}
|\only<1>{\highlightline}|    \end{minipage}
  \end{figure}
\end{document}
      \end{codeblock}
    \end{column}
    \begin{column}{0.4\textwidth}
      \only<1>{
        在 \env{figure} 环境里使用 \env{minipage} 小页制作列盒子,以并排两图并分别编号,需要设定强制参数以指定列宽。两个小页之间添加 \cmd{hfill} 使两个小页两端对齐。
      }
      \only<2>{
        在每个小页内部分别使用 \cmd{caption} 添加标签。
      }
      \only<3>{
        \includepdflarge{doubleimages}
      }
    \end{column}
  \end{columns}
\end{frame}

\begin{frame}[fragile]%
  \begin{columns}
    \begin{column}{0.6\textwidth}
      \begin{codeblock}[]{}
\documentclass{ctexart}
\usepackage{graphicx}
|\highlightline|\usepackage{subcaption}
\graphicspath{{figs/}{pics/}}
\begin{document}
  \begin{figure}[ht]
|\highlightline|    \begin{subfigure}{0.48\textwidth}
      \centering
      \includegraphics[height=2cm]{sjtug}
      \caption{||徽标}
|\highlightline|    \end{subfigure}\hfill
|\highlightline|    \begin{subfigure}{0.48\textwidth}
      \centering
      \includegraphics[height=2cm]{sjtugt}
      \caption{||文字}
|\highlightline|    \end{subfigure}
    \caption{SJTUG}\label{fig:sjtug}
  \end{figure}
\end{document}
      \end{codeblock}
    \end{column}
    \begin{column}{0.4\textwidth}
      \includepdflarge{subfigures}\vspace{15pt}
      \pkg{subcaption} 宏包提供了 \env{subfigure} 环境(以及 \env{subtable}),可以用于以子图的形式插入,编号会增加一级。也可以为子图添加子集引用编号。
    \end{column}
  \end{columns}
\end{frame}

\section{表}
\begin{frame}[fragile]
  \frametitle{简单表格}
  \begin{columns}
    \begin{column}{0.6\textwidth}
      \begin{codeblock}[]{}
\documentclass{ctexart}
|\only<1-2>{\highlightline}|\usepackage{|\temporal<1>{array}{\highlight{array}}{array},\temporal<2>{booktabs}{\highlight{booktabs}}{booktabs}|}
\begin{document}
\begin{table}[ht]
  \centering
  \caption{||北京冬奥会吉祥物}
|\only<1>{\highlightline}|  \begin{tabular}{lp{3cm}}
|\only<2>{\highlightline}|    \toprule
|\only<3>{\highlightline}|吉祥物 & 描述                          \\
|\only<2>{\highlightline}|    \midrule
|\only<3>{\highlightline}|冰墩墩 & 2022 年北京冬季奥运会吉祥物  \\
|\only<3>{\highlightline}|雪容融 & 2022 年北京冬季残奥会吉祥物  \\
|\only<2>{\highlightline}|    \bottomrule
|\only<1>{\highlightline}|  \end{tabular}
\end{table}
\end{document}
      \end{codeblock}
    \end{column}
    \begin{column}{0.4\textwidth}
      \only<1>{
        使用 \env{tabular} 环境绘制表格。需要添加参数(称为\textbf{表格导言})以确定每一列的对齐方式。引入 \pkg{array} 宏包来使用更多的\textbf{引导符}。
        \begin{center}
          \footnotesize
          \begin{stampbox}
            \begin{tabular}{>{\ttfamily}ll}
              \alert{l} & 向左对齐 \\
              \alert{c} & 居中对齐 \\
              \alert{r} & 向右对齐 \\
              \alert{p\{3cm\}} & 固定列宽,两端对齐 \\
              \alert{m\{3cm\}} & \texttt{p} + 垂直居中对齐 \\
              \alert{>\{\textbackslash{}bfseries\}} & 后一列单元格前加命令 \\
              \alert{*\{3\}\{l\}} & 三个左对齐列 \\
            \end{tabular}
          \end{stampbox}
        \end{center}
      }
      \only<2>{
        \pkg{booktabs} 宏包提供了标准三线表格所需要的行分割线:\cmd{toprule},\cmd{midrule},\cmd{bottomrule}。也可以使用 \cmd{cmidrule\{1-2\}} 来部分地绘制行分割线。一般不推荐在专业表格中使用纵向分割线。
      }
      \only<3>{
        每行内容使用 \textbackslash\textbackslash{} 分隔开,每行中的单元格使用 \& 分隔开。
      }
      \only<4>{
        \includepdflarge{table}
      }
    \end{column}
  \end{columns}
\end{frame}

\begin{frame}[fragile]%
  \begin{columns}
    \begin{column}{0.6\textwidth}
      \begin{codeblock}[]{表头居中}
\documentclass{ctexart}
\usepackage{array,booktabs}
\begin{document}
\begin{table}[ht]
  \centering
  \caption{||北京冬奥会吉祥物}
  \begin{tabular}{lp{3cm}}
    \toprule
|\highlightline|\multicolumn{1}{c}{||吉祥物} &
|\highlightline|\multicolumn{1}{c}{||描述} \\
    \midrule
||冰墩墩 & 2022 年北京冬季奥运会吉祥物  \\
||雪容融 & 2022 年北京冬季残奥会吉祥物  \\
    \bottomrule
  \end{tabular}
\end{table}
\end{document}
      \end{codeblock}
    \end{column}
    \begin{column}{0.4\textwidth}
      \cmd{multicolumn} 命令不仅可以用于合并同行的单元格,还可以用于临时地屏蔽表格导言对该列的对齐方式定义。这里用于居中表头。
      \begin{center}
        \begin{stampbox}
          \parbox{0.85\linewidth}{
            \ttfamily\color{blue}\textbackslash{}multicolumn\{格数\}\{对齐方式\}\{内容\}
          }
        \end{stampbox}
      \end{center}
      跨页表格考虑使用 \pkg{longtable} 宏包。带标注的表格可以考虑使用 \pkg{threeparttable} 宏包。考虑使用在线工具生成表格代码 \link{https://www.tablesgenerator.com/latex_tables}。
    \end{column}
  \end{columns}
\end{frame}

\section{数学公式}
\begin{frame}
  \frametitle{数学模式}
  \begin{alertblock}{}
  输入数学公式是 \LaTeX{} 的绝对强项,很多常见的公式服务依赖于一些轻量级渲染引擎比如 MathJax, K\kern-.3ex\raise.4ex\hbox{\footnotesize A}\kern-.3ex\TeX{}。但是它们实际上使用的是 \LaTeX{} 语法变种,也就是说并没有使用 \LaTeX{} 后端。所以不要期望有完全一致的输出。
  \end{alertblock}
  
  为了更好的获得数学公式输入支持,请使用 \hologo{AmS}math 宏包。数学模式分为两种:
  \begin{description}
    \item[行内模式] 一般通过一对美元符号(\$$\cdots$\$)标记,可以使用编辑器内建的符号表输入数学符号,也可以使用在线工具手写识别 \link{https://detexify.kirelabs.org/classify.html}。
    \item[行间模式] 一般通过 \env{equation} 环境\footnote{这是有编号环境,其加星号的变种 \env{equation*} 用于生成无编号环境。}输入。如果需要使用多行公式,请使用 \env{align} 环境,并按照类似表格输入的方式,使用 \& 对齐符号,\textbackslash\textbackslash{} 换行。如果不想手动居中,可以考虑多行自动居中的 \env{gather} 和单个大型公式首尾两端对齐 \env{multline}。
  \end{description}
\end{frame}

\begin{frame}
  \frametitle{数学命令展示}
  \begin{columns}
    \begin{column}{0.33\textwidth}
      \begin{exampleblock}{}
        $PV=nRT$
      \end{exampleblock}
      \begin{exampleblock}{}
        $\sum_{i=1}^ki^2=\frac{n(n+1)(2n+1)}{6}$
      \end{exampleblock}
      \begin{exampleblock}{}
        $T(n) = aT\left(\left\lceil\frac{n}{b}\right\rceil\right) + \mathcal{O}(n^d)$
      \end{exampleblock}
      \begin{exampleblock}{}
        $\frac{x_{1}+x_{2}+x_{3}}{3}\geq \sqrt[3]{x_{1}x_{2}x_{3}}$
      \end{exampleblock}
      \begin{exampleblock}{}
        $n=(\underbrace{1\cdots 1}_{k\text{ of 1's}})_2=2^{k+1}-1$
      \end{exampleblock}
      \begin{exampleblock}{}
        $\nabla f (P)= \left.\left(\frac{\partial f}{\partial x},\frac{\partial f}{\partial y},\frac{\partial f}{\partial z}\right)\right|_{P}$
      \end{exampleblock}
    \end{column}
    \begin{column}{0.67\textwidth}
      \begin{exampleblock}{}
        \begin{equation*}
          \int_{a}^b f(x)\,\mathrm{d}x=\lim_{|P|\rightarrow 0}\sum_{i=1}^n f(\xi_i)\Delta x_i
        \end{equation*}
      \end{exampleblock}
      \begin{exampleblock}{}
        \begin{equation}
          T(n) = \begin{cases}
            \mathcal{O}(n^d),&\textrm{if } d>\log_b a, \\
            \mathcal{O}(n^d\log n), &\textrm{if } d=\log_b a,\\
            \mathcal{O}(n^{\log_b a}), &\textrm{if } d<\log_b a.
          \end{cases}
        \end{equation}
      \end{exampleblock}
      \begin{exampleblock}{}
        \begin{align}
          Q^{T}A&=R \\
          \begin{pmatrix}
            q_1^T \\ q_2^T \\ q_3^T
          \end{pmatrix}
          \begin{pmatrix}
            a_1 & a_2 & a_3
          \end{pmatrix}
          &=R
        \end{align}
      \end{exampleblock}
    \end{column}
  \end{columns}
\end{frame}

%更深入地讲解 mathtools, unicode-math, siunix

\section{引用}
\begin{frame}[fragile]
  \frametitle{交叉引用}
  \only<1>{
    正如之前所提到的,\LaTeX{} 中使用 \cmd{label} 标记,然后可以使用 \cmd{ref} 来引用这个标记。 \cmd{label} 可以放在使用计数器的对象之后。
  }
  \only<2>{
    为了使得对公式编号的引用带有括号,推荐使用 \hologo{AmS}math 宏包中的 \cmd{eqref} 命令。对于多行公式环境,每一个换行符前都可以添加一个 \cmd{label} 用于引用该行公式。
  }
  \begin{columns}
    \begin{column}{0.5\textwidth}
      \begin{codeblock}[]{图}
\begin{figure}
|\only<1>{\highlightline}|  \caption{||示例}\label{fig:example}
\end{figure}
      \end{codeblock}
      \begin{codeblock}[]{表}
\begin{table}
|\only<1>{\highlightline}|  \caption{||示例}\label{tab:example}
\end{table}
      \end{codeblock}
    \end{column}
    \begin{column}{0.5\textwidth}
\begin{codeblock}[]{目次}
|\only<1>{\highlightline}|\section{||示例}\label{sec:example}
\end{codeblock}

\begin{codeblock}[]{公式}
\begin{equation}
  a = b + c
|\only<1>{\highlightline}|\label{eq:example}
\end{equation}
|\only<2>{\highlightline}|如公式 \eqref{eq:example} 所示,
\end{codeblock}
    \end{column}
  \end{columns}
\end{frame}

\begin{frame}[fragile]
  \frametitle{文献引用}
  \LaTeX{} 管理参考文献可以采用专用数据库文件 \texttt{.bib},很多的文献管理文件比如 EndNote \link{https://lic.sjtu.edu.cn/Default/softshow/tag/MDAwMDAwMDAwMLGImKE}, Zotero \link{https://www.zotero.org/}, JabRef \link{https://www.jabref.org/} 都可以直接导出这种格式的文件用于 \LaTeX{} 论文中的引用。一般不需要手写数据库文件,你只需要注意每一个文献会在数据库中有一个主键,这个类似于上文的 \cmd{label} 标记,只是要引用数据库中的文献需要使用 \cmd{cite} 命令。
  
  \begin{codeblock}[]{ref.bib}
|\highlightline|@phdthesis{devoftech,|\hfill\alert{\% 类型为博士论文,主键为\texttt{devoftech}}|
  title={||新时期我国信息技术产业的发展},
  author={||江泽民},
  year={2008}
}
  \end{codeblock}
\end{frame}

\begin{frame}
  \frametitle{文献引用}
  而让 \LaTeX{} 处理 \texttt{.bib} 数据库文件与引用有两种工作流。你可能需要查清楚如何在编辑器中设置对应的工作流,或者采用后面所提到的高级编译方式 \texttt{latexmk}。
  \begin{columns}
    \begin{column}{0.5\textwidth}
      \begin{block}{\hologo{BibTeX} + \pkg{gbt7714}}
        一般期刊提交使用这种方法,\pkg{natbib} 宏包将提供命令 \cmd{citet}(文本引用) 和 \cmd{citep}(括号引用)。中文引用可以直接使用 \pkg{gbt7714} 宏包,但是角模式和正文模式不能混用。
      \end{block}
    \end{column}
    \begin{column}{0.5\textwidth}
      \begin{block}{\hologo{biber} + \pkg{biblatex}}
        这是更容易自定义的方法,与 \hologo{BibTeX} 的运作方式稍有不同。\pkg{biblatex} 提供了更加智能的引用命令。而中文引用可以使用 \pkg{biblatex} 宏包的样式 \pkg{gb7714-2015},使用该样式需要使用 \hologo{XeLaTeX} 编译。
      \end{block}
    \end{column}
  \end{columns}
\end{frame}

\begin{frame}[fragile]
  \frametitle{文献引用}
  \begin{columns}
    \begin{column}{0.5\textwidth}
      \begin{codeblock}[]{\hologo{BibTeX} + \pkg{gbt7714}}
\documentclass{ctexart}
\usepackage{gbt7714}
\bibliographystyle{gbt7714-numerial}
% \citestyle{numbers}  % 正文模式
\begin{document}
  ||他指出了...\cite{devoftech}
  \bibliography{ref}
\end{document}
      \end{codeblock}
    \end{column}
    \begin{column}{0.5\textwidth}
      \begin{codeblock}[]{\hologo{biber} + \pkg{biblatex}}
\documentclass{ctexart}
\usepackage[backend=biber,style=gb7714-2015]{biblatex}
\addbibresource{ref.bib}
\begin{document}
  ||他在文献 \parencite{devoftech}
  ||指出了...\cite{devoftech}
  \printbibliography
\end{document}
      \end{codeblock}
    \end{column}
  \end{columns}
\end{frame}

\begin{frame}
  \frametitle{文献引用}
  \begin{columns}
    \begin{column}{0.5\textwidth}
      \includepdflarge{bibtex}
    \end{column}
    \begin{column}{0.5\textwidth}
      \includepdflarge{biblatex}
    \end{column}
  \end{columns}
\end{frame}

} % End of customized shaded number logo

|\only<3>{\highlightline}|  % !TeX root = ..\..\latex-talk.tex

\part{SJTUThesis}

\begin{frame}
  \frametitle{简介}
  \begin{columns}
    \begin{column}{0.6\textwidth}
      \begin{itemize}
        \item 最早由韦建文于 2009 年 11 月发布 0.1a 版,2018 年起由 SJTUG 接手维护
        \item 最新版:\SJTUThesisVersion{} (\SJTUThesisDate)
        \item 支持本科、硕士、博士学位论文以及课程论文的排版
      \end{itemize}
    \end{column}
    \begin{column}{0.4\textwidth}
      \begin{exampleblock}{}
        \begin{minipage}[c]{1cm}
          \includegraphics[width=0.8cm]{\getcontribpath{sjtug}{vi/sjtug}}
        \end{minipage}
        \begin{minipage}[c]{2cm}
          \href{https://github.com/sjtug}{sjtug}/\href{https://github.com/sjtug/SJTUThesis}{SJTUThesis}
        \end{minipage}
      \end{exampleblock}
      \vspace{-8pt}
      \begin{block}{}
        \scriptsize
        上海交通大学 \hologo{XeLaTeX} 学位论文及课程论文模板 | Shanghai Jiao Tong University \hologo{XeLaTeX} Thesis Template
      \end{block}
      \vspace{-8pt}
      \begin{alertblock}{}
        \scriptsize
        \begin{tabular}{cl}
          \faStar & 2.4k \\
          \faEye & 55 \\
          \faCodeBranch & 701 \\
        \end{tabular}
      \end{alertblock}
    \end{column}
  \end{columns}
\end{frame}

\begin{frame}
  \frametitle{下载与编译}
  \alert{下载} 推荐安装 Git \link{https://git-scm.com/} 后,克隆 SJTUG 镜像仓库
  \begin{exampleblock}{\faGit*}
    \ttfamily\small
    git clone https://mirror.sjtu.edu.cn/git/SJTUThesis.git/
  \end{exampleblock}

  \alert{编译} 推荐使用 \pkg{latexmk} 编译\footnote{\hologo{MiKTeX} 用户需要手动安装 Perl 解释器 \link{https://www.perl.org/get.html} 才能使用 \pkg{latexmk}。},在不能够利用自带的 \texttt{.latexmkrc} 配置文件的情况下,需要查清楚在对应的编辑器中如何使用 \hologo{XeLaTeX} + \hologo{biber} 编译 \link{https://github.com/sjtug/SJTUThesis/blob/master/README.md}。
  \begin{exampleblock}{\faTerminal}
    \ttfamily\small
    latexmk -xelatex main
  \end{exampleblock}

  Overleaf 用户可以下载压缩包后,上传并采用 \hologo{XeLaTeX} 编译方式。
\end{frame}

\begin{frame}
  \frametitle{手动编译}
  \alert{第一次编译失败} 如果没有办法通过通常方式编译成功,请尝试使用文件夹内附带 \faLinux{}\,\faApple{} \texttt{Makefile} 和 \faWindows{} \texttt{Compile.bat} 进行编译。

  \alert{统计字数} 编写过程中也可以使用对应的命令调用 \TeX{}count 来统计正文字数。
  \begin{columns}
    \begin{column}{0.38\textwidth}
      \begin{exampleblock}{\faLinux{}\,\faApple}
        \ttfamily
        make all\\
        make clean\\
        make cleanall\\
        make wordcount
      \end{exampleblock}
    \end{column}
    \begin{column}{0.38\textwidth}
      \begin{exampleblock}{\faWindows}
        \ttfamily
        ./Compile.bat thesis\\
        ./Compile.bat clean\\
        ./Compile.bat cleanall\\
        ./Compile.bat wordcount
      \end{exampleblock}
    \end{column}
    \begin{column}{0.24\textwidth}
      \begin{block}{\faInfo}
        \ttfamily
        编译论文\\
        清理中间文件\\
        $\hookrightarrow +$删除论文\\
        统计字数
      \end{block}
    \end{column}
  \end{columns}
\end{frame}

\begin{frame}[label=compile]
  \frametitle{编译问题排查}
  \begin{columns}
    \begin{column}{0.33\textwidth}
      \begin{alertblock}{无法使用 \texttt{latexmk}\thesisissue{578}}
        \hologo{MiKTeX} 需要安装 Perl 解释器。
      \end{alertblock}  
      \begin{alertblock}{C\TeX{} 套装无法编译\thesisissue{446}}
        使用最新 \TeX{} 发行版。
      \end{alertblock}
      \begin{alertblock}{\hologo{pdfLaTeX} 无法编译\thesisissue{444}}
        请使用 \texttt{latexmk},或更改编辑器设置以 \hologo{XeLaTeX} 编译。
      \end{alertblock}
    \end{column}
    \begin{column}{0.33\textwidth}
      \begin{alertblock}{缺少字体\thesisissue{564} \thesisdiscuss{598}}
        更换字体集,或者安装对应字体。
      \end{alertblock}
      \begin{alertblock}{缺少汉字\thesisissue{533} \thesisdiscuss{617}}
        去除使用 fandol 字体集的设定。或者是安装字体后,改用 \texttt{fontset=adobe} 或 \texttt{fontset=founder}。
      \end{alertblock}
    \end{column}
    \begin{column}{0.33\textwidth}
      \begin{block}{\faInfoCircle{} README}
        不同编辑器的设置请首先参阅 README \link{https://github.com/sjtug/SJTUThesis/blob/master/README.md} 文档。
      \end{block}
      \begin{block}{\faBookOpen{} Wiki}
        其他编译问题推荐查阅 Wiki \link{https://github.com/sjtug/SJTUThesis/wiki} 的使用说明部分。
      \end{block}
    \end{column}
  \end{columns}
\end{frame}

\begin{frame}[fragile, label=covers]
  \begin{codeblock}[firstnumber=3]{main.tex}
|\alert{\% 载入 SJTUThesis 模版}|
\documentclass[|\only<1>{\highlight{type}}\only<2>{type}|=|\only<1>{bachelor}\only<2>{\highlight{bachelor}}|]{sjtuthesis}
  \end{codeblock}
  \begin{figure}
    \parbox{0.9\textwidth}{
      \begin{subfigure}{0.20\textwidth}
        \framebox{\includegraphics[width=\linewidth]{support/thesis/bachelor}}
        \caption{\only<1>{学士}\only<2>{\texttt{bachelor}}}
      \end{subfigure}\hfill
      \begin{subfigure}{0.20\textwidth}
        \framebox{\includegraphics[width=\linewidth]{support/thesis/master}}
        \caption{\only<1>{硕士}\only<2>{\texttt{master}}}
      \end{subfigure}\hfill
      \begin{subfigure}{0.20\textwidth}
        \framebox{\includegraphics[width=\linewidth]{support/thesis/doctor}}
        \caption{\only<1>{博士}\only<2>{\texttt{doctor}}}
      \end{subfigure}\hfill
      \begin{subfigure}{0.20\textwidth}
        \framebox{\includegraphics[width=\linewidth]{support/thesis/course}}
        \caption{\only<1>{课程}\only<2>{\texttt{course}}}
      \end{subfigure}
      \caption{论文类型示例\only<2>{ \texttt{type}}}
    }
  \end{figure}
\end{frame}

\begin{frame}[fragile]
  \frametitle{文档类选项}
  % \framesubtitle{\textbackslash{}documentclass\{sjtuthesis\}}
  \begin{columns}
    \begin{column}{0.45\textwidth}
      \includegraphics[page=10]{thesisdir}
    \end{column}
    \begin{column}{0.55\textwidth}
      \begin{table}[H]
        \caption{文档类选项}
        \footnotesize
        \begin{tabular}{>{\ttfamily}rll}
          \toprule
          选项 & 含义 & 相关 \\
          \midrule
          type= & 指定论文类型 & 第 \ref{covers} 页\\
          fontset= & 指定字体 & 第 \ref{compile} 页\\
          \midrule
          review & 开启盲审模式 & \thesisissue{195} \thesisissue{686} \\
          twoside & 双页模式 & \thesisissue{554} \\
          oneside & 单页模式 & \thesisissue{694} \\
          openright & 章从奇数页开始 & \thesisdiscuss{724} \\
          openany & 章从任意页开始 & \thesisissue{446} \\
          \bottomrule
        \end{tabular}
      \end{table}
    \end{column}
  \end{columns}
\end{frame}

\begin{frame}[fragile]
  \frametitle{基本配置}
  \framesubtitle{\textbackslash{}input\{setup\}}
  \begin{columns}
    \begin{column}{0.45\textwidth}
      \includegraphics[page=9]{thesisdir}
    \end{column}
    \begin{column}{0.55\textwidth}
      \begin{codeblock}[firstnumber=12]{main.tex}
|\highlightline<1>|% 论文基本配置,加载宏包等全局配置
|\highlightline<1>|\input{setup}

\begin{document}

%TC:ignore

|\highlightline<2>|% 标题页
|\highlightline<2>|\maketitle
      \end{codeblock}
      \visible<2>{
        \cmd{sjtusetup} 中的 \pkg{info} 将会修改封面的信息设置(见第 \ref{covers} 页)。
      }
    \end{column}
  \end{columns}
\end{frame}

\begin{frame}[fragile]
  \frametitle{基本配置}
  \framesubtitle{\textbackslash{}sjtusetup}
  \begin{columns}
    \begin{column}{0.45\textwidth}
      \includegraphics[page=1]{thesisdir}
    \end{column}
    \begin{column}{0.55\textwidth}
      \begin{codeblock}[firstnumber=3]{setup.tex}
\sjtusetup{
  info = {
    title    = {||上海交通大学学位论文 \LaTeX{} 模板示例文档},
    title*   = {A Sample for \LaTeX-based SJTU Thesis Template},
    author   = {||某\quad{}某},
    author* = {Mo Mo},
  },
  style = { header-logo-color = red, 
  },
  name = {
    publications = {||攻读学位期间完成的论文},
  },
}
      \end{codeblock}
    \end{column}
  \end{columns}
\end{frame}

\begin{frame}
  \frametitle{基本配置}
  \framesubtitle{\textbackslash{}sjtusetup}
  \begin{columns}
    \begin{column}{0.45\textwidth}
      \includegraphics[page=1]{thesisdir}
    \end{column}
    \begin{column}{0.55\textwidth}
      \begin{table}[H]
        \centering
        \caption{info 域}
        \footnotesize
        \begin{tabular}{lll} \toprule
          命令作用 & 中文对应选项 & 英文对应选项 \\ \midrule
          论文标题 & \texttt{title} & \texttt{title*} \\
          关键字列表 & \texttt{keywords} & \texttt{keywords*} \\
          作者姓名&  \texttt{author} &\texttt{author*}\\
          申请学位名称 & \texttt{degree}&\texttt{degree*}\\
          院系名称 & \texttt{department} & \texttt{department*}\\
          专业名称 & \texttt{major} & \texttt{major*}\\
          导师 & \texttt{supervisor} & \texttt{supervisor*}\\
          副导师 & \texttt{assisupervisor} & \texttt{assisupervisor*}\\
          日期 & \multicolumn{2}{c}{\texttt{date}}\\
          学号 & \multicolumn{2}{c}{\texttt{id}}\\ \bottomrule
          \end{tabular}
      \end{table}
    \end{column}
  \end{columns}
\end{frame}

\begin{frame}[fragile]
  \frametitle{版权页}
  \framesubtitle{\textbackslash{}copyrightpage}
  \begin{columns}
    \begin{column}{0.45\textwidth}
      \only<1>{
        \includegraphics[page=9]{thesisdir}
      }
      \only<2>{
        \includegraphics[page=2]{thesisdir}
      }
      \only<3>{
        \begin{figure}[H]
          \framebox{\includegraphics[page=2,width=0.4\linewidth]{bachelor}}
          \caption{版权页}
        \end{figure}
      }
    \end{column}
    \begin{column}{0.55\textwidth}
      \begin{codeblock}[firstnumber=22]{main.tex}
|\highlightline<1>|% 原创性声明及使用授权书
|\highlightline<1>|\copyrightpage
|\highlightline<2>|% 插入外置原创性声明及使用授权书
|\highlightline<2>|% \copyrightpage[scans/sample-copyright-old.pdf]
      \end{codeblock}
      \only<1>{
        \cmd{copyrightpages} 可以用于插入版权页。
      }
      \only<2>{
        \cmd{copyrightpages} 也接受一个可选参数,用于直接使用扫描件。
      }
    \end{column}
  \end{columns}
\end{frame}

\begin{frame}[fragile]
  \frametitle{前置部分}
  \framesubtitle{\textbackslash{}frontmatter}
  \begin{columns}
    \begin{column}{0.45\textwidth}
      \only<1>{
        \includegraphics[page=9]{thesisdir}
      }
      \only<2>{
        \includegraphics[page=3]{thesisdir}
      }
      \only<3>{
        \begin{figure}[H]
          \begin{subfigure}{0.45\textwidth}
            \framebox{\includegraphics[page=3,width=\linewidth]{bachelor}}
            \caption{中文}
          \end{subfigure}\hfill
          \begin{subfigure}{0.45\textwidth}
            \framebox{\includegraphics[page=4,width=\linewidth]{bachelor}}
            \caption{英文}
          \end{subfigure}
          \caption{摘要}
        \end{figure}
      }
      \only<4>{
        \begin{figure}[H]
          \begin{subfigure}{0.30\linewidth}
            \centering
            \framebox{\includegraphics[page=5,width=0.6\linewidth]{bachelor}}
            \caption{目录}
          \end{subfigure}
          \begin{subfigure}{0.30\linewidth}
            \centering
            \framebox{\includegraphics[page=6,width=0.6\linewidth]{bachelor}}
            \caption{插图}
          \end{subfigure}

          \begin{subfigure}{0.30\linewidth}
            \centering
            \framebox{\includegraphics[page=7,width=0.6\linewidth]{bachelor}}
            \caption{表格}
          \end{subfigure}
          \begin{subfigure}{0.30\linewidth}
            \centering
            \framebox{\includegraphics[page=8,width=0.6\linewidth]{bachelor}}
            \caption{算法}
          \end{subfigure}
          \caption{索引}
        \end{figure}
      }
      \only<5>{
        \includegraphics[page=4]{thesisdir}
      }
      \only<6>{
        \begin{figure}[H]
          \framebox{\includegraphics[page=9,width=0.5\linewidth]{bachelor}}
          \caption{符号对照表}
        \end{figure}
      }
    \end{column}
    \begin{column}{0.55\textwidth}
      \begin{codeblock}[firstnumber=30]{main.tex}
|\highlightline<2-3>|% 摘要
|\highlightline<2-3>|\input{contents/abstract}

|\highlightline<4>|% 目录
|\highlightline<4>|\tableofcontents
|\highlightline<4>|% 插图索引
|\highlightline<4>|\listoffigures*
|\highlightline<4>|% 表格索引
|\highlightline<4>|\listoftables*
|\highlightline<4>|% 算法索引
|\highlightline<4>|\listofalgorithms*

|\highlightline<5-6>|% 符号对照表
|\highlightline<5-6>|\input{contents/nomenclature}
      \end{codeblock}
    \end{column}
  \end{columns}
\end{frame}

\begin{frame}[fragile]
  \frametitle{主体部分}
  \framesubtitle{\textbackslash{}mainmatter}
  \begin{columns}
    \begin{column}{0.45\textwidth}
      \only<1>{
        \includegraphics[page=5]{thesisdir}
      }
      \only<2>{
        \begin{figure}[H]
          \begin{subfigure}{0.30\linewidth}
            \centering
            \framebox{\includegraphics[page=11,width=0.6\linewidth]{bachelor}}
            \caption{简介}
          \end{subfigure}
          \begin{subfigure}{0.30\linewidth}
            \centering
            \framebox{\includegraphics[page=13,width=0.6\linewidth]{bachelor}}
            \caption{数学}
          \end{subfigure}

          \begin{subfigure}{0.30\linewidth}
            \centering
            \framebox{\includegraphics[page=16,width=0.6\linewidth]{bachelor}}
            \caption{浮动体}
          \end{subfigure}
          \begin{subfigure}{0.30\linewidth}
            \centering
            \framebox{\includegraphics[page=22,width=0.6\linewidth]{bachelor}}
            \caption{总结}
          \end{subfigure}
          \caption{主体部分}
        \end{figure}
      }
    \end{column}
    \begin{column}{0.55\textwidth}
      \begin{codeblock}[firstnumber=47]{main.tex}
|\highlightline|% 正文内容
|\highlightline|% !TeX root = ../../../latex-talk.tex

\section{是什么}

\begin{frame}
  \frametitle{\TeX{}}
  \begin{columns}[c]
    \begin{column}{0.7\textwidth}
      \begin{center}
        \rmfamily\Huge
        \highlight[structure]{\TeX{}}
      \end{center}
      \begin{center}
        \parbox{0.75\textwidth}{
          \TeX{} 是由斯坦福大学教授高德纳
          (Donald E.~Knuth)于 1977 年开始开发的排版引擎。目前仍在更新,最新版本号为 3.141592653 \link{https://tug.org/TUGboat/tb42-1/tb130knuth-tuneup21.pdf}。
        }
      \end{center}
    \end{column}
    \begin{column}{0.3\textwidth}
      \includegraphics[width=.8\columnwidth]{support/images/Knuth.jpg}
    \end{column}
  \end{columns}
  \note{\emph{这一部分背景介绍大家可以了解一下,暂时跳过。}
  \LaTeX{} 这个词由两个部分组成,\hologo{La} 和 \TeX{}。那我们首先了解一下 \TeX{} 是什么。
  \TeX{} 是由斯坦福大学的教授高德纳于 1977 年开始开发的排版引擎,它已经有三十多年的历史了,
  目前仍在更新,版本号(3.141592653)将会趋近于 $\pi$ 的取值,高德纳最近还在给 \textsl{TUGBoat} 写稿子
  \link{https://tug.org/TUGboat/tb42-1/tb130knuth-tuneup21.pdf},
  关于 \TeX{} 今年又做了哪些改进。}
\end{frame}

\begin{frame}
  \frametitle{\LaTeX{}}
  \begin{columns}[c]
    \begin{column}{0.7\textwidth}
      \begin{center}
        \rmfamily\Huge
        \highlight[structure]{\LaTeX{}}
      \end{center}
      \begin{center}
        \parbox{0.75\textwidth}{
          \LaTeX{} 是最早在 1985 年由现就职于微软的 Leslie Lamport 开发的一种 \TeX{} \textbf{格式}\footnotemark,使用一些列宏和扩展宏包来简化 \TeX{} 的使用。现在由 \LaTeX{} Project 的成员维护。现在广泛使用的版本是 \LaTeXe{},最新的版本为 \LaTeX3(2020 年 10 月后默认内置)。
        }
      \end{center}
    \end{column}
    \begin{column}{0.3\textwidth}
      \includegraphics[width=.8\columnwidth]{support/images/Lamport.jpg}
    \end{column}
  \end{columns}
  \footnotetext{\hologo{ConTeXt} 也是一种 \TeX{} 格式 \link{https://www.contextgarden.net/}。}
  \note{\emph{这一部分的背景介绍大家可以了解一下,暂时跳过。}
  \LaTeX{} 是最早由现就职于微软的 Leslie Lamport 开发的一种 \TeX{} 格式(与其对标的是
  \hologo{ConTeXt}\link{https://www.contextgarden.net/}),主要也是为了简化 \TeX{} 的使用。
  现在主要由 \LaTeX{} 开发组维护,现在广泛使用的版本是 \LaTeXe{},最新的版本为 \LaTeX3,
  在 2020 年 10 月后默认内置,所以要尽可能使用较新的发行版,以充分发挥其功能。}
\end{frame}

\begin{frame}
  \frametitle{程序}
  \begin{columns}[c]
    \begin{column}{0.7\textwidth}
      \begin{center}
        \rmfamily\Huge
        \highlight[structure]{\hologo{pdfLaTeX}}
      \end{center}
      \begin{center}
        \parbox{0.7\textwidth}{
          \hologo{pdfLaTeX} 是为了编译一个 \LaTeX{} 文档而运行的程序。实际上底层在运行一个叫 \hologo{pdfTeX} 的引擎,并预装了对应的 \LaTeX{} \textbf{格式}。为了利用临时文件,可能就需要多次运行程序。
        }
      \end{center}
    \end{column}
    \begin{column}{0.3\textwidth}
      \begin{block}{}
        \ttfamily\small
        > \highlight{pdflatex} main.tex\\
        This is pdfTeX, Version 3.141592653-
        2.6-1.40.23 (MiKTeX 21.10)\\
        entering extended mode\\
        \highlight{LaTeX2e} <2021-11-15>\\
        \highlight{L3} programming layer <2021-11-22>
      \end{block}
    \end{column}
  \end{columns}
  \note{\hologo{pdfLaTeX} 是为了编译一个 \LaTeX{} 文档而运行的程序。}
\end{frame}

% \begin{frame}
%   \frametitle{引擎}
%   \begin{columns}[c]
%     \begin{column}{0.7\textwidth}
%       \begin{center}
%         \rmfamily\Huge
%         \highlight[structure!70]{pdf}\hologo{La}\highlight[structure!70]{\TeX{}}
%       \end{center}
%       \begin{center}
%         \parbox{0.7\textwidth}{
%           pdf\TeX{} 是编译 \TeX{} 文档(以 \texttt{.tex} 结尾)的\textbf{引擎}---可以理解 \TeX{} 指令的\textbf{程序}。
%         }
%       \end{center}
%     \end{column}
%     \begin{column}{0.3\textwidth}
%       \begin{block}{}
%         \ttfamily\small
%         > pdflatex main.tex\\
%         This is \highlight[structure!70]{pdfTeX}, Version 3.141592653-
%         2.6-1.40.23 (MiKTeX 21.10)
%         entering extended mode\\
%         LaTeX2e <2021-11-15>\\
%         L3 programming layer <2021-11-22>
%       \end{block}
%     \end{column}
%   \end{columns}
%   \note{实际上底层在运行一个叫 \hologo{pdfTeX} 的引擎,并预装了对应的 \LaTeX{} 格式。}
% \end{frame}

\begin{frame}[label={frame:engine}]
  \frametitle{程序}
  \begin{table}
    \caption{主流 \hologo{(La)TeX} 程序
    \footnote{(u)p\TeX{} 是日语最常用的引擎,生成 \texttt{.dvi},支持 Unicode。}\footnote{Ap\TeX{} \link{https://github.com/clerkma/ptex-ng} 具有底层 CJK 支持,内联 Ruby,Color Emoji。}}
    \footnotesize
    \begin{stampbox}
      \begin{tabular}{c>{\raggedright}*{3}{p{3.5cm}}}
        \alert{引擎}     & \hologo{pdfTeX}   & \hologo{XeTeX}   & \hologo{LuaTeX}   \\
        \alert{程序}     & \hologo{pdfLaTeX} & \hologo{XeLaTeX} & \hologo{LuaLaTeX} \\
        \alert{特点}     & 直接生成 PDF,支持 micro-typography  & 支持 Unicode、OpenType 与复杂文字编排 (CTL) & 支持 Unicode,内联 Lua,支持 OpenType \\
      \end{tabular}
    \end{stampbox}
  \end{table}

  \begin{center}
    \parbox{.9\textwidth}{
      \hologo{pdfLaTeX} 不支持 Unicode。为了排版中文,大部分情况下应当使用 \hologo{XeLaTeX},而 \hologo{LuaLaTeX} 速度相对较慢。\faWindows{} 可以在一些情况下使用 \hologo{pdfLaTeX}。
    }
  \end{center}
  \note{当然为了排版中文,已经不再推荐使用 \hologo{pdfLaTeX} 了,应该使用
  \hologo{XeLaTeX} 或者 \hologo{LuaLaTeX},当然后者的速度还是相对较慢,
  它们支持 Unicode 编码,并可以使用 OpenType 字体的全部功能。
  当然 \faWindows{} 平台下在某些追求速度的情况下,
  还是可以试着使用 \hologo{pdfLaTeX} 的。

  \hologo{LuaLaTeX} 理想情况下不慢,但是使用一些宏包后会破坏理想状态,
  也会因配置产生不同的结果,不同的操作系统在 I/O 速度上的不同也会导致不同的时间。

  \hologo{pdfLaTeX} 也支持,只不过需要先生成 tfm \TeX{} 字体度量文件,后续使用 \TeX{}
  自身的配置方法,只能使用 7 比特或 8 比特字体。}
\end{frame}

% \begin{frame}
%   \paragraph{\hologo{pdfLaTeX}} \TeX{} 和 \LaTeX{} 被广泛使用之前,它们只需内置支持欧洲语言即可。在 Unicode 出现之前,\LaTeX{} 提供了许多种\textbf{文件编码}来允许很多语言的文字以原生的方式输入,\hologo{pdfLaTeX} 也只需要使用 8 位文件编码和 8 位字体。
% \end{frame}


|\highlightline|\input{contents/math_and_citations}
|\highlightline|\input{contents/floats}
|\highlightline|\input{contents/summary}

%TC:ignore

% 参考文献
\printbibliography[heading=bibintoc]
      \end{codeblock}
    \end{column}
  \end{columns}
\end{frame}

\begin{frame}
  \frametitle{数学}
  \begin{itemize}
    \item 公式示例:\nolinkurl{contents/math_and_citations.tex}
    \item \SJTUThesis{} 定义了常用的数学环境(需要手工引入 \texttt{ntheorem} 宏包):
      \begin{table}[h]
        \centering
        \footnotesize
        \begin{tabular}{*{7}{l}}\toprule
          assumption  & axiom   & conjecture & corollary    & definition  & example   & exercise  \\
          假设        & 公理    & 猜想       & 推论         & 定义        & 例        & 练习      \\\midrule
          lemma       & problem & proof      & proposition  & remark      & solution  & theorem   \\
          引理        & 问题    & 证明       & 命题         & 注          & 解        & 定理      \\\bottomrule
        \end{tabular}
      \end{table}
      \item \SJTUThesis{} 可以通过 \texttt{unimath} 选项使用 \pkg{unicode-math} 进行数学输入,注意与传统方式的区别。\thesisissue{555}
  \end{itemize}
\end{frame}

\begin{frame}[fragile]
  \frametitle{参考文献}
  \begin{columns}
    \begin{column}{0.45\textwidth}
      \includegraphics[page=6]{thesisdir}
    \end{column}
    \begin{column}{0.55\textwidth}
      \begin{codeblock}[firstnumber=111,numbersep=2pt]{setup.tex}
% 使用 BibLaTeX 处理参考文献
%   biblatex-gb7714-2015 常用选项
%     gbnamefmt=lowercase     姓名大小写由输入信息确定
%     gbpub=false             禁用出版信息缺失处理
\usepackage[backend=biber,style=gb7714-2015]{biblatex}
% 文献表字体
% \renewcommand{\bibfont}{\zihao{-5}}
% 文献表条目间的间距
\setlength{\bibitemsep}{0pt}
|\highlightline|% 导入参考文献数据库
|\highlightline|\addbibresource{bibdata/thesis.bib}
      \end{codeblock}
    \end{column}
  \end{columns}
\end{frame}

\begin{frame}[fragile]
  \frametitle{附录}
  \framesubtitle{\textbackslash{}appendix}
  \begin{columns}
    \begin{column}{0.45\textwidth}
      \only<1>{
        \includegraphics[page=7]{thesisdir}
      }
      \only<2>{
        \begin{figure}[H]
          \begin{subfigure}{0.45\linewidth}
            \framebox{\includegraphics[width=\linewidth,page=24]{bachelor}}
            \caption{}
          \end{subfigure}\hfill
          \begin{subfigure}{0.45\textwidth}
            \framebox{\includegraphics[width=\linewidth,page=25]{bachelor}}
            \caption{}
          \end{subfigure}
          \caption{附录}
        \end{figure}
      }
    \end{column}
    \begin{column}{0.55\textwidth}
      \begin{codeblock}[firstnumber=61]{main.tex}
% 附录中图表不加入索引
\captionsetup{list=no}

% 附录内容
|\highlightline|\input{contents/app_maxwell_equations}
|\highlightline|\input{contents/app_flow_chart}
      \end{codeblock}
    \end{column}
  \end{columns}
\end{frame}

\begin{frame}[fragile]
  \frametitle{结尾部分}
  \framesubtitle{\textbackslash{}backmatter}
  \begin{columns}
    \begin{column}{0.45\textwidth}
      \only<1>{
        \includegraphics[page=8]{thesisdir}
      }
      \only<2>{
        \begin{figure}[H]
          \begin{subfigure}{0.30\linewidth}
            \centering
            \framebox{\includegraphics[page=26,width=0.6\linewidth]{bachelor}}
            \caption{致谢}
          \end{subfigure}
          \begin{subfigure}{0.30\linewidth}
            \centering
            \framebox{\includegraphics[page=27,width=0.6\linewidth]{bachelor}}
            \caption{成就}
          \end{subfigure}

          \begin{subfigure}{0.30\linewidth}
            \centering
            \framebox{\includegraphics[page=28,width=0.6\linewidth]{bachelor}}
            \caption{简历}
          \end{subfigure}
          \begin{subfigure}{0.30\linewidth}
            \centering
            \framebox{\includegraphics[page=29,width=0.6\linewidth]{bachelor}}
            \caption{大摘要*}
          \end{subfigure}
          \caption{结尾部分}
        \end{figure}
      }
    \end{column}
    \begin{column}{0.55\textwidth}
      \begin{codeblock}[firstnumber=76]{main.tex}
% 致谢
\input{contents/acknowledgements}

% 发表论文及科研成果
% 盲审论文中,发表论文及科研成果等仅以第几作者注明即可,不要出现作者或他人姓名
\input{contents/achievements}

% 简历
\input{contents/resume}

% 学士学位论文要求在最后有一个大摘要,单独编页码
\input{contents/digest}
      \end{codeblock}
    \end{column}
  \end{columns}
\end{frame}

\begin{frame}
  \frametitle{还有其他问题?}
  \begin{columns}
    \begin{column}{0.75\textwidth}
    \begin{itemize}
      \item[{\faComment*[regular]}] 日常模板或 \LaTeX{} 使用问题可以前往 Discussions \link{https://github.com/sjtug/SJTUThesis/discussions} 提问
      
      (解决后别忘了 \textcolor{green}{\faCheckCircle{} Mark as answer}
      \item[{\faDotCircle[regular]}] 如果是 \textsc{SJTUThesis} 项目本身的 bug 和 feature request
      
      可以通过 Issues \link{https://github.com/sjtug/SJTUThesis/issues} 反馈。
      \item[{\faCodeBranch}] 如果你有好点子,可以贡献代码
     
      向 \textsc{SJTU\TeX{}}(v1) \link{https://github.com/sjtug/SJTUTeX/tree/v1} 存储库发 PR,\par
      而后把解包结果同步到 \textsc{SJTUThesis}。
  
      \item[{\faTag}] 如果你对正在基于 \LaTeX3 开发的新版感兴趣,\par
      也欢迎向 \textsc{SJTU\TeX{}}(v2) \link{https://github.com/sjtug/SJTUTeX/tree/v2} 发 PR。
  
      \item[{\faQq}] 也欢迎在 QQ 群即时讨论。
    \end{itemize}
    \end{column}
    \begin{column}{0.25\textwidth}
      \includegraphics[height=0.7\textheight]{qq.jpg}
    \end{column}
  \end{columns}
\end{frame}
\end{document}
      \end{codeblock}
    \end{column}
  \end{columns}
\end{frame}

\begin{frame}[fragile]
  \frametitle{组织文档}
  \begin{columns}
    \begin{column}{0.4\textwidth}
      \begin{codeblock}[]{learnlatex.tex}
|\highlightline|\chapter{||学习 \LaTeX{}}
\section{||概念}
\subsection{\LaTeX{}}
\LaTeX{} 是一个用以排版高质量作品的文档准备系统。
      \end{codeblock}
      子文件中就不需要添加 \env{document} 环境了\footnotemark。
    \end{column}
    \begin{column}{0.6\textwidth}
      \begin{codeblock}[]{主文档}
|\highlightline|\documentclass{ctexrep}
\includeonly{learnlatex,sjtuthesis}
\begin{document}
  \tableofcontents
  % !TeX root = ..\..\latex-talk.tex

\part{学习 \LaTeX{}}
% FIXME: Part Page miniframe overflow
% FIXME: footnote fault numbering

\begin{frame}[plain]
  \vfil
  \begin{center}
    \href{https://learnlatex.org}{
      \rmfamily
      Learn\,\lower1ex\hbox{\Huge\LaTeX{}}.org
    }
  \end{center}
  \vfil
  \begin{center}
    \parbox{0.75\linewidth}{
      Learn\LaTeX{}.org\cite{learnlatex} 提供了解 \LaTeX{} 的 16 篇简短的教程,并包含了一些可以在线运行的示例,可以通过亲自动手查看实验效果。本部分主要参考由 C\TeX{}-org 提供的中文翻译版本 \link{https://github.com/CTeX-org/learnlatex.github.io/tree/zh-Hans/zh-Hans/}。
    }
  \end{center}
  \vfil
\end{frame}

{ % Start of shaded number logo

\newcommand{\shadedfont}[2][1pt]{
  % #1 (optional): shadow distance
  % #2: the text needed to be shaded
  \hbox{\rlap{\color{gray}\hskip#1#2}#2}
}
\newcounter{learnsec}
\setcounter{learnsec}{-1}
\newcommand{\updatelogo}{
  % update the logo corresponding to current counter.
  \stepcounter{learnsec}
  \logo{
    \raise.3ex\hbox{\tiny\insertsection}\shadedfont{\arabic{learnsec}}
  }
}
\let\oldsection=\section
\renewcommand{\section}[1]{\oldsection{#1}\updatelogo}

\section{是什么}
\begin{frame}
  \frametitle{\TeX{}}
  \begin{columns}[c]
    \begin{column}{0.7\textwidth}
      \begin{center}
        \rmfamily\Huge
        \hologo{La}\highlight[structure!70]{\TeX{}}
      \end{center}
      \begin{center}
        \parbox{0.75\textwidth}{
          \TeX{} 是由斯坦福大学教授高德纳
          (Donald E.~Knuth)于 1977 年开始开发的排版引擎。目前仍在更新,最新版本号为 3.141592653 \link{https://tug.org/TUGboat/tb42-1/tb130knuth-tuneup21.pdf}。
        }
      \end{center}
    \end{column}
    \begin{column}{0.3\textwidth}
      \includegraphics[width=.8\columnwidth]{Knuth.jpg}
    \end{column}
  \end{columns}
\end{frame}

\begin{frame}
  \frametitle{\LaTeX{}}
  \begin{columns}[c]
    \begin{column}{0.7\textwidth}
      \begin{center}
        \rmfamily\Huge
        \highlight[structure]{\LaTeX{}}
      \end{center}
      \begin{center}
        \parbox{0.75\textwidth}{
          \LaTeX{} 是最早在 1985 年由现就职于微软的 Leslie Lamport 开发的一种 \TeX{} \textbf{格式}\footnotemark,使用一些列宏和扩展宏包来简化 \TeX{} 的使用。现在由 \LaTeX{} Project 的成员维护。现在广泛使用的版本是 \LaTeXe{},最新的版本为 \LaTeX3(2020 年 10 月后默认内置)。
        }
      \end{center}
    \end{column}
    \begin{column}{0.3\textwidth}
      \includegraphics[width=.8\columnwidth]{Lamport.jpg}
    \end{column}
  \end{columns}
  \footnotetext{\hologo{ConTeXt} 也是一种 \TeX{} 格式 \link{https://www.contextgarden.net/}。}
\end{frame}

\begin{frame}
  \frametitle{程序}
  \begin{columns}[c]
    \begin{column}{0.7\textwidth}
      \begin{center}
        \rmfamily\Huge
        \highlight[structure]{\hologo{pdfLaTeX}}
      \end{center}
      \begin{center}
        \parbox{0.7\textwidth}{
          \hologo{pdfLaTeX} 是为了编译一个 \LaTeX{} 文档而运行的程序。实际上底层在运行一个叫 \hologo{pdfTeX} 的引擎,并预装了对应的 \LaTeX{} \textbf{格式}。为了利用临时文件,可能就需要多次运行程序。
        }
      \end{center}
    \end{column}
    \begin{column}{0.3\textwidth}
      \begin{block}{}
        \ttfamily\small
        > \highlight{pdflatex} main.tex\\
        This is pdfTeX, Version 3.141592653-
        2.6-1.40.23 (MiKTeX 21.10)\\
        entering extended mode\\
        \highlight{LaTeX2e} <2021-11-15>\\
        \highlight{L3} programming layer <2021-11-22>
      \end{block}
    \end{column}
  \end{columns}
\end{frame}

\begin{frame}
  \frametitle{引擎}
  \begin{columns}[c]
    \begin{column}{0.7\textwidth}
      \begin{center}
        \rmfamily\Huge
        \highlight[structure!70]{pdf}\hologo{La}\highlight[structure!70]{\TeX{}}
      \end{center}
      \begin{center}
        \parbox{0.7\textwidth}{
          pdf\TeX{} 是编译 \TeX{} 文档(以 \texttt{.tex} 结尾)的\textbf{引擎}---可以理解 \TeX{} 指令的\textbf{程序}。
        }
      \end{center}
    \end{column}
    \begin{column}{0.3\textwidth}
      \begin{block}{}
        \ttfamily\small
        > pdflatex main.tex\\
        This is \highlight[structure!70]{pdfTeX}, Version 3.141592653-
        2.6-1.40.23 (MiKTeX 21.10)
        entering extended mode\\
        LaTeX2e <2021-11-15>\\
        L3 programming layer <2021-11-22>
      \end{block}
    \end{column}
  \end{columns}
\end{frame}

\begin{frame}
  \frametitle{Unicode 引擎}
  \begin{table}
    \caption{主流 \hologo{(La)TeX} 程序
    \footnote{(u)p\TeX{} 是日语最常用的引擎,生成 \texttt{.dvi},支持 Unicode。}\footnote{Ap\TeX{} 具有底层 CJK 支持,内联 Ruby,Color Emoji。}}
    \footnotesize
    \begin{stampbox}
      \begin{tabular}{c>{\raggedright}*{3}{p{3.5cm}}}
        \alert{引擎}     & \hologo{pdfTeX}   & \hologo{XeTeX}   & \hologo{LuaTeX}   \\
        \alert{程序}     & \hologo{pdfLaTeX} & \hologo{XeLaTeX} & \hologo{LuaLaTeX} \\
        \alert{特点}     & 直接生成 PDF,支持 micro-typography  & 支持 Unicode、OpenType 与复杂文字编排 (CTL) & 支持 Unicode,内联 Lua,支持 OpenType \\
      \end{tabular}
    \end{stampbox}
  \end{table}

  \begin{center}
    \parbox{.9\textwidth}{
      \hologo{pdfLaTeX} 不支持 Unicode。为了排版中文,大部分情况下 \faApple{}\,\faLinux{} 应当使用 \hologo{XeLaTeX},而 \hologo{LuaLaTeX} 速度相对较慢。\faWindows{} 可以在一些情况下使用 \hologo{pdfLaTeX}。
    }
  \end{center}
\end{frame}

% \begin{frame}
%   \paragraph{\hologo{pdfLaTeX}} \TeX{} 和 \LaTeX{} 被广泛使用之前,它们只需内置支持欧洲语言即可。在 Unicode 出现之前,\LaTeX{} 提供了许多种\textbf{文件编码}来允许很多语言的文字以原生的方式输入,\hologo{pdfLaTeX} 也只需要使用 8 位文件编码和 8 位字体。
% \end{frame}

\section{跑起来}
\begin{frame}
  \frametitle{发行版}
  \begin{table}
    \caption{\hologo{TeX} 发行版}
    \footnotesize
    \begin{stampbox}
      \begin{tabular}{c>{\raggedright}*{3}{p{3.2cm}}}
        \alert{发行版}     & \hologo{MiKTeX} \link{https://mirrors.sjtug.sjtu.edu.cn/ctan/systems/win32/miktex/setup/windows-x64/basic-miktex-21.12-x64.exe}   & \TeX{} Live \link{https://mirrors.sjtug.sjtu.edu.cn/ctan/systems/texlive/tlnet/install-tl.zip}   & Mac\TeX{} \link{https://mirrors.sjtug.sjtu.edu.cn/ctan/systems/mac/mactex/mactex-20210328.pkg}  \\[2pt]
        \alert{特点}      &  只安装必要文件,依赖用时更新  &  所有平台均可使用,每年发布一次 & Mac 系统专用,对 \TeX{} Live 的进一步打包 \\
        \alert{推荐平台}  & \faWindows  & \faLinux &  \faApple \\
      \end{tabular}
    \end{stampbox}
  \end{table}
  \begin{center}
    \parbox{.9\textwidth}{
      要让 \LaTeX{} 跑起来,核心就是要有一套 \TeX{} 发行版,来获取让 \LaTeX{} 工作所需的一组程序和文件。参考《一份简短的关于 \LaTeX{} 安装的介绍》\link{https://mirrors.sjtug.sjtu.edu.cn/ctan/info/install-latex-guide-zh-cn/install-latex-guide-zh-cn.pdf} 安装想使用的发行版。推荐使用发行版的最新版本\footnote{老版本 Linux 系统的包管理器自带 \TeX{} Live 发行版可能不是最新的 \link{https://repology.org/project/texlive/versions},尽量使用镜像安装,并手动将相关环境变量添加到路径 \link{https://www.tug.org/texlive/doc/texlive-zh-cn/texlive-zh-cn.pdf}。},并使用国内镜像。
    }
  \end{center}
\end{frame}

\begin{frame}[plain]
  \hbox to \textwidth{
    \hfil
    \vbox to 3cm{
      \hbox{
        \resizebox{3cm}{!}{\includegraphics{\getcontribpath{sjtug}{vi/sjtug.pdf}}}
      }
    }
    \hfil
    \vbox to 3cm{
      \vfill
      \hbox{\Large\bfseries\color{cprimary} 稳定、快速、现代的镜像服务。}
      \vskip2pt
      \hbox{托管于华东教育网骨干节点上海交通大学。}
      \vfill
    }
    \hskip20pt
    \hfil
  }

  \begin{center}
    \parbox{0.8\textwidth}{
      推荐使用 SJTUG 软件镜像服务,离得近,下得快。
      
      \begin{description}
        \footnotesize
        \item[\TeX{} Live]  {\ttfamily tlmgr option repository https://mirrors.sjtug.sjtu.edu.cn/CTAN/systems/texlive/tlnet}
        \item[\hologo{MiKTeX}] 在 \hologo{MiKTeX} Console 中设置镜像源为 \url{https://mirrors.sjtug.sjtu.edu.cn}
      \end{description}
    }
  \end{center}
\end{frame}

\begin{frame}
  \frametitle{编辑器}
  \begin{table}
    \caption{开源编辑器推荐}
    \footnotesize
    \begin{stampbox}
      \begin{tabular}{c>{\raggedright}*{3}{p{3.5cm}}}
        \alert{编辑器}     & \begin{tabular}{c}Visual Studio Code\\ \LaTeX{} Workshop\end{tabular}  & \TeX{}studio & \TeX{}works \\[5pt]
        \alert{特点}      &  搭配 VS Code 使用非常方便,易扩展  & 可以使用大量的菜单选项输入代码块,用户友好 & 只提供基础的高亮与运行方法,发行版自带\footnote{Mac\TeX{} 打包的是 \TeX{}Shop 编辑器。} \\
      \end{tabular}
    \end{stampbox}
  \end{table}
  \begin{center}
    \parbox{.9\textwidth}{
      使用专为 \LaTeX{} 设计的编辑器将带来更多便利,因为它们往往会提供一键编译、内置 PDF 阅读器以及语法高亮等功能。几乎所有现代的 \LaTeX{} 编辑器都提供 Sync\TeX{} 这一强大的功能,以在 PDF 和 代码间对应跳转。
    }
  \end{center}
\end{frame}

\begin{frame}
  \frametitle{在线平台}
  \begin{table}
    \caption{在线协作平台推荐}
    \footnotesize
    \begin{stampbox}
      \begin{tabular}{c>{\raggedright}*{2}{p{4cm}}}
        \alert{在线平台}     & Overleaf \link{https://www.overleaf.com/}  & 交大 \LaTeX{} 助手 \link{https://latex.sjtu.edu.cn/} \\[2pt]
        \alert{特点}      & 最流行的在线平台,提供大量的模板,但国内访问慢 & 校内平台,隐私保护有保障,共享项目限制少 \\
      \end{tabular}
    \end{stampbox}
  \end{table}
  \begin{center}
    \parbox{.9\textwidth}{
      在线平台允许你直接在网页中编辑文档,无需本地安装即可在后台运行 \LaTeX{},并显示生成的 PDF。可以参照 Overleaf 官方文档学习如何使用在线平台 \link{https://www.overleaf.com/learn}。
    }
  \end{center}
\end{frame}

\section{基本结构}
\begin{frame}[fragile]%
  \frametitle{文档部件}
  \begin{columns}[c]
    \begin{column}{0.4\textwidth}
      \only<1>{
        \cmd{documentclass} 命令加载了\textbf{文档类}。\pkg{article} 是由 \LaTeX{}提供的用于排版短文档的基本文档类。
        \begin{description}
          \footnotesize
          \item[\pkg{article}] 不包含章的短文档
          \item[\pkg{report}] 含有章的单面印刷文档
          \item[\pkg{book}] 含有章的双面印刷文档
          \item[\pkg{beamer}] 制作幻灯片
        \end{description}
      }
      \only<2>{
        \env{document} 环境用于指示文档主体的范围。\LaTeX{} 还有其他的使用 \cmd{begin} 和 \cmd{end} 的搭配,我们称这些为\textbf{环境}。它们将用来设定局部格式命令\footnotemark。
      }
      \only<3>{
        \includepdflarge{enminimal}
      }
    \end{column}
    \begin{column}{0.6\textwidth}
      \begin{codeblock}[]{排版英文最简示例}
|\only<1>{\highlightline}|\documentclass{article}
|\only<2>{\highlightline}|\begin{document}
|\only<3>{\highlightline}|  Together for a Shared Future
|\only<2>{\highlightline}|\end{document}
      \end{codeblock}
    \end{column}
  \end{columns}
  \only<2>{\footnotetext{环境实际上是一个组,只不过通过语义化的形式预装了对应的格式命令。普通的组可以直接使用一对大括号之间的内容 \{$\cdots$\} 表示。}}
\end{frame}

\section{扩展}
\begin{frame}[fragile]%
  \frametitle{中文排版}
  \begin{columns}[c]
    \begin{column}{0.4\textwidth}
      \only<1>{
        \cmd{usepackage} 用于使用宏包以向 \LaTeX{} 添加或修改功能,需要在\textbf{导言区}调用。
        这里使用 \pkg{ctex} 宏集以获得中文支持。其调用底层因随不同的引擎而不同。
        {
          \footnotesize
          \begin{stampbox}
            \begin{tabular}{c*{3}{c}}
              \alert{引擎}     & \hologo{pdfTeX}   & \hologo{XeTeX}   & \hologo{LuaTeX}   \\
              \alert{程序}     & \hologo{pdfLaTeX} & \hologo{XeLaTeX} & \hologo{LuaLaTeX} \\
              \alert{宏包}     & CJK\footnotemark & xeCJK & luatexja \\
              \alert{封装}     & \multicolumn{3}{c}{ctex} \\
            \end{tabular}
          \end{stampbox}
        }
        \vspace{-1cm}
      }
      \only<2>{
        C\TeX{} 建议对于之前提到的常规文档类,最佳实践是使用该宏集提供的四种中文文档类,以对特定类型提供额外的中文排版适配。
        \begin{center}
          \begin{stampbox}
            \footnotesize
            \begin{tabular}{cc}
              \pkg{ctexart} & \pkg{ctexrep} \\
              \pkg{ctexbook} & \pkg{ctexbeamer} \\
            \end{tabular}
          \end{stampbox}
        \end{center}
      }
      \only<3>{
        \includepdflarge{cnminimal}
      }
      \only<4>{
        大部分情况下,你都不应当在 \LaTeX{} 中强制断行:你几乎只是想另起一段,或者是想在段落之间添加空行(使用 \pkg{parskip} 宏包就可实现)。
        只有\alert{很少的}情况下你需要使用 \textbackslash{}\textbackslash{} 来另起一行而不另起一段。
      }
    \end{column}
    \begin{column}{0.6\textwidth}
      \begin{codeblock}[]{排版中文\only<2->{最佳实践}}
|\only<2>{\highlightline}|\documentclass{|\only<1>{article}\only<2->{ctexart}|}
|\only<1>{\highlightline\textbackslash{}usepackage\{ctex\}\hfill\color{cprimary}\% 导言区}|
\begin{document}
|\only<3>{\highlightline}|    一起向未来
|\only<4>{\highlightline}|
  Together for a Shared Future
\end{document}
      \end{codeblock}
    \end{column}
  \end{columns}
  \only<1>{\footnotetext{ctex 在 \faApple\,\faLinux{} 上已经不可以使用 \hologo{pdfLaTeX} 编译,以及在 \faWindows{} 上使用该引擎也会变更自动间距调整等功能的默认行为。}}
\end{frame}

\section{设定格式}
\begin{frame}[fragile]%
  \frametitle{字体样式}
  \begin{columns}
    \begin{column}{0.4\textwidth}
      \only<1>{
        \includepdflarge{fontstyle}
      }
      \only<2>{
        可以使用显示样式设定命令对小段做加粗、斜体、等宽等等的处理。
        \begin{center}
          \footnotesize
          \begin{stampbox}
            \begin{tabular}{rl}
              \cmd{textrm} & \textrm{衬线} \\
              \cmd{textbf} & \textbf{加粗} \\
              \cmd{textit} & \kaishu 斜体 \\
              \cmd{texttt} & \texttt{等宽} \\
              \cmd{textsf} & \textsf{无衬线} \\
              \cmd{textsc} & \textsc{Small Caps} \\
              \cmd{textsl} & \textsl{Slanted} \\
            \end{tabular}
          \end{stampbox}
        \end{center}
      }
      \only<3>{
        与 Word 不同的是,\LaTeX{} 一般情况下并不需要使用上面的显式命令,而是采用逻辑标记的方法,
        比如 \cmd{emph} 可以强调文字,以及下面将要提到的目次命令(第 \ref{sectioning} 页)。
        这样可以统一管理格式。
      }
    \end{column}
    \begin{column}{0.6\textwidth}
      \begin{codeblock}[]{样式}
\documentclass{ctexart}
\begin{document}
|\only<2>{\highlightline}|  \textbf{||一起向未来}

|\only<3>{\highlightline}|  \emph{Together for a Shared Future}
\end{document}
      \end{codeblock}
    \end{column}
  \end{columns}
\end{frame}

\begin{frame}[fragile]%
  \frametitle{\only<1-2>{字体大小}\only<3>{字体样式}}
  \begin{columns}
    \begin{column}{0.4\textwidth}
      \only<1>{
        \includepdflarge{fontsize}
      }
      \only<2>{
        同样地,你也可以显式地设定字体大小,但是这种命令会更改行文设置,所以需要使用一个组来限定作用范围\footnotemark。
        \begin{center}
          \footnotesize
          \begin{stampbox}
            \begin{tabular}{rl}
              \cmd{tiny} & \tiny 极小 \\
              \cmd{scriptsize} & \scriptsize 抄本大小  \\
              \cmd{footnotesize} & \footnotesize 脚注大小 \\
              \cmd{small} & \small 小 \\
              \cmd{normalsize} & \normalsize 正常大小 \\
              \cmd{large} & \large 大 \\
              \cmd{huge} & \Huge 巨大 \\
            \end{tabular}
          \end{stampbox}
        \end{center}
      }
      \only<3>{
        也可以使用字体样式对应的更改字体设置的命令,这对于大段文段的设置而言也是很方便的。
        \begin{center}
          \footnotesize
          \begin{stampbox}
            \begin{tabular}{ll}
              \cmd{textrm} & \cmd{rmfamily}\\
              \cmd{texttt} & \cmd{ttfamily}\\
              \cmd{textsf} & \cmd{sffamily}\\
              \cmd{textbf} & \cmd{bfseries}\\
              \cmd{textit} & \cmd{itshape}\\
              \cmd{textsc} & \cmd{scshape}\\
              \cmd{textsl} & \cmd{slshape}\\
            \end{tabular}
          \end{stampbox}
        \end{center}
      }
    \end{column}
    \begin{column}{0.6\textwidth}
      \begin{codeblock}[]{大小}
\documentclass{ctexart}
\begin{document}
|\only<2>{\highlightline}|  {\bfseries\Large 一起向未来\par}
|\only<3>{\highlightline}|  {\itshape Together for a Shared Future}
\end{document}
      \end{codeblock}
    \end{column}
  \end{columns}
  \only<2>{\footnotetext{注意最后显式地使用 \cmd{par} 在改回大小前结束该段,否则会导致下一行的行间距异常!}}
\end{frame}

\section{逻辑结构}
\begin{frame}[fragile]
  \frametitle{列表}
  \begin{columns}
    \begin{column}{0.35\textwidth}
      \begin{codeblock}[]{无序列表}
\documentclass{ctexart}
\begin{document}
|\highlightline|  \begin{itemize}
    \item 第一项
    \item 第二项
    \item 第三项
|\highlightline|  \end{itemize}
\end{document}
      \end{codeblock}
    \end{column}
    \begin{column}{0.35\textwidth}
      \begin{codeblock}[]{有序列表}
\documentclass{ctexart}
\begin{document}
|\highlightline|  \begin{enumerate}
    \item 第一项
    \item 第二项
    \item 第三项
|\highlightline|  \end{enumerate}
\end{document}
      \end{codeblock}
    \end{column}
    \begin{column}{0.35\textwidth}
      \begin{codeblock}[]{描述列表}
\documentclass{ctexart}
\begin{document}
|\highlightline|  \begin{description}
    \item[||第一] 文本
    \item[||第二] 文本
    \item[||第三] 文本  
|\highlightline|  \end{description}
\end{document}
      \end{codeblock}
    \end{column}
  \end{columns}
\end{frame}

%更深的列表技巧,定理环境等

\begin{frame}
  \frametitle{列表}
  \begin{columns}
    \begin{column}{0.35\textwidth}
      \includepdflarge{unordered}
    \end{column}
    \begin{column}{0.35\textwidth}
      \includepdflarge{ordered}
    \end{column}
    \begin{column}{0.35\textwidth}
      \includepdflarge{description}
    \end{column}
  \end{columns}
\end{frame}

\begin{frame}[fragile,label=sectioning]%
  \frametitle{目次结构}
  \begin{columns}
    \begin{column}{0.4\textwidth}
      \LaTeX{} 可以使用目次命令将文档划分层级\footnotemark,并自动设定对应字体样式和大小。
      \begin{center}
        \begin{stampbox}
          \footnotesize
          \begin{tabular}{rll}
           命令 & 中文 & 层次 \\
           \cmd{chapter} & 章\footnotemark & \sout{0} \\
           \cmd{section} & 节 & 1 \\
           \cmd{subsection} & 小节 & 2 \\
           \cmd{subsubsection} & 小小节 & 3 \\
          \end{tabular}
        \end{stampbox}
      \end{center}
    \end{column}
    \begin{column}{0.6\textwidth}
      \begin{codeblock}[]{目次}
\documentclass{ctexart}
\begin{document}
|\highlightline|  \section{||概念}
|\highlightline|  \subsection{\LaTeX{}}
  \LaTeX{} 是一个用以排版高质量作品的文档准备系统。
\end{document}
      \end{codeblock}
    \end{column}
  \end{columns}
  \footnotetext{章这一级只在 \pkg{report} 和 \pkg{book} 文档类(包括对应的中文文档类)有定义。还有不常用的 \cmd{part} (0@\pkg{article}/-1@\pkg{report}\&\pkg{book}\&\pkg{beamer}) 以及更低层次的 \cmd{paragraph} (4) 与 \cmd{subparagraph} (5)。 }
\end{frame}

\begin{frame}[fragile]%
  \frametitle{组织文档}
  \begin{columns}
    \begin{column}{0.4\textwidth}
      \only<1>{
        \cmd{tableofcontents} 用来生成对于目次命令的目录。如果你想设定显示到哪个层级,在这个命令前使用 \cmd{setcounter\{tocdepth\}\{层次\}}
      }
      \only<2>{
        对于大型文档而言,使用多个文件管理源文件通常是更方便的。而 \cmd{include} 和 \cmd{input} 都以相对路径的方式插入其他 \TeX{} 文档。
        区别在于,\cmd{include} 命令会从新页开始并做一些内部调整,这基本上只对 \pkg{chapter} 这一级有用。而 \cmd{input} 会原样插入源代码。
      }
      \only<3>{
        但是 \cmd{include} 插入的文档可以使用 \cmd{includeonly} 管理当前要排印哪一部分的内容,利用所有章节辅助文件的同时,减少编译时间并专注于该部分的内容。
      }
    \end{column}
    \begin{column}{0.6\textwidth}
      \begin{codeblock}[]{主文档}
\documentclass{ctexrep}
|\only<3>{\highlightline}|\includeonly{learnlatex,sjtuthesis}
\begin{document}
|\only<1>{\highlightline}|  \tableofcontents
|\only<2-3>{\highlightline}|  % !TeX root = ..\..\latex-talk.tex

\part{学习 \LaTeX{}}
% FIXME: Part Page miniframe overflow
% FIXME: footnote fault numbering

\begin{frame}[plain]
  \vfil
  \begin{center}
    \href{https://learnlatex.org}{
      \rmfamily
      Learn\,\lower1ex\hbox{\Huge\LaTeX{}}.org
    }
  \end{center}
  \vfil
  \begin{center}
    \parbox{0.75\linewidth}{
      Learn\LaTeX{}.org\cite{learnlatex} 提供了解 \LaTeX{} 的 16 篇简短的教程,并包含了一些可以在线运行的示例,可以通过亲自动手查看实验效果。本部分主要参考由 C\TeX{}-org 提供的中文翻译版本 \link{https://github.com/CTeX-org/learnlatex.github.io/tree/zh-Hans/zh-Hans/}。
    }
  \end{center}
  \vfil
\end{frame}

{ % Start of shaded number logo

\newcommand{\shadedfont}[2][1pt]{
  % #1 (optional): shadow distance
  % #2: the text needed to be shaded
  \hbox{\rlap{\color{gray}\hskip#1#2}#2}
}
\newcounter{learnsec}
\setcounter{learnsec}{-1}
\newcommand{\updatelogo}{
  % update the logo corresponding to current counter.
  \stepcounter{learnsec}
  \logo{
    \raise.3ex\hbox{\tiny\insertsection}\shadedfont{\arabic{learnsec}}
  }
}
\let\oldsection=\section
\renewcommand{\section}[1]{\oldsection{#1}\updatelogo}

\section{是什么}
\begin{frame}
  \frametitle{\TeX{}}
  \begin{columns}[c]
    \begin{column}{0.7\textwidth}
      \begin{center}
        \rmfamily\Huge
        \hologo{La}\highlight[structure!70]{\TeX{}}
      \end{center}
      \begin{center}
        \parbox{0.75\textwidth}{
          \TeX{} 是由斯坦福大学教授高德纳
          (Donald E.~Knuth)于 1977 年开始开发的排版引擎。目前仍在更新,最新版本号为 3.141592653 \link{https://tug.org/TUGboat/tb42-1/tb130knuth-tuneup21.pdf}。
        }
      \end{center}
    \end{column}
    \begin{column}{0.3\textwidth}
      \includegraphics[width=.8\columnwidth]{Knuth.jpg}
    \end{column}
  \end{columns}
\end{frame}

\begin{frame}
  \frametitle{\LaTeX{}}
  \begin{columns}[c]
    \begin{column}{0.7\textwidth}
      \begin{center}
        \rmfamily\Huge
        \highlight[structure]{\LaTeX{}}
      \end{center}
      \begin{center}
        \parbox{0.75\textwidth}{
          \LaTeX{} 是最早在 1985 年由现就职于微软的 Leslie Lamport 开发的一种 \TeX{} \textbf{格式}\footnotemark,使用一些列宏和扩展宏包来简化 \TeX{} 的使用。现在由 \LaTeX{} Project 的成员维护。现在广泛使用的版本是 \LaTeXe{},最新的版本为 \LaTeX3(2020 年 10 月后默认内置)。
        }
      \end{center}
    \end{column}
    \begin{column}{0.3\textwidth}
      \includegraphics[width=.8\columnwidth]{Lamport.jpg}
    \end{column}
  \end{columns}
  \footnotetext{\hologo{ConTeXt} 也是一种 \TeX{} 格式 \link{https://www.contextgarden.net/}。}
\end{frame}

\begin{frame}
  \frametitle{程序}
  \begin{columns}[c]
    \begin{column}{0.7\textwidth}
      \begin{center}
        \rmfamily\Huge
        \highlight[structure]{\hologo{pdfLaTeX}}
      \end{center}
      \begin{center}
        \parbox{0.7\textwidth}{
          \hologo{pdfLaTeX} 是为了编译一个 \LaTeX{} 文档而运行的程序。实际上底层在运行一个叫 \hologo{pdfTeX} 的引擎,并预装了对应的 \LaTeX{} \textbf{格式}。为了利用临时文件,可能就需要多次运行程序。
        }
      \end{center}
    \end{column}
    \begin{column}{0.3\textwidth}
      \begin{block}{}
        \ttfamily\small
        > \highlight{pdflatex} main.tex\\
        This is pdfTeX, Version 3.141592653-
        2.6-1.40.23 (MiKTeX 21.10)\\
        entering extended mode\\
        \highlight{LaTeX2e} <2021-11-15>\\
        \highlight{L3} programming layer <2021-11-22>
      \end{block}
    \end{column}
  \end{columns}
\end{frame}

\begin{frame}
  \frametitle{引擎}
  \begin{columns}[c]
    \begin{column}{0.7\textwidth}
      \begin{center}
        \rmfamily\Huge
        \highlight[structure!70]{pdf}\hologo{La}\highlight[structure!70]{\TeX{}}
      \end{center}
      \begin{center}
        \parbox{0.7\textwidth}{
          pdf\TeX{} 是编译 \TeX{} 文档(以 \texttt{.tex} 结尾)的\textbf{引擎}---可以理解 \TeX{} 指令的\textbf{程序}。
        }
      \end{center}
    \end{column}
    \begin{column}{0.3\textwidth}
      \begin{block}{}
        \ttfamily\small
        > pdflatex main.tex\\
        This is \highlight[structure!70]{pdfTeX}, Version 3.141592653-
        2.6-1.40.23 (MiKTeX 21.10)
        entering extended mode\\
        LaTeX2e <2021-11-15>\\
        L3 programming layer <2021-11-22>
      \end{block}
    \end{column}
  \end{columns}
\end{frame}

\begin{frame}
  \frametitle{Unicode 引擎}
  \begin{table}
    \caption{主流 \hologo{(La)TeX} 程序
    \footnote{(u)p\TeX{} 是日语最常用的引擎,生成 \texttt{.dvi},支持 Unicode。}\footnote{Ap\TeX{} 具有底层 CJK 支持,内联 Ruby,Color Emoji。}}
    \footnotesize
    \begin{stampbox}
      \begin{tabular}{c>{\raggedright}*{3}{p{3.5cm}}}
        \alert{引擎}     & \hologo{pdfTeX}   & \hologo{XeTeX}   & \hologo{LuaTeX}   \\
        \alert{程序}     & \hologo{pdfLaTeX} & \hologo{XeLaTeX} & \hologo{LuaLaTeX} \\
        \alert{特点}     & 直接生成 PDF,支持 micro-typography  & 支持 Unicode、OpenType 与复杂文字编排 (CTL) & 支持 Unicode,内联 Lua,支持 OpenType \\
      \end{tabular}
    \end{stampbox}
  \end{table}

  \begin{center}
    \parbox{.9\textwidth}{
      \hologo{pdfLaTeX} 不支持 Unicode。为了排版中文,大部分情况下 \faApple{}\,\faLinux{} 应当使用 \hologo{XeLaTeX},而 \hologo{LuaLaTeX} 速度相对较慢。\faWindows{} 可以在一些情况下使用 \hologo{pdfLaTeX}。
    }
  \end{center}
\end{frame}

% \begin{frame}
%   \paragraph{\hologo{pdfLaTeX}} \TeX{} 和 \LaTeX{} 被广泛使用之前,它们只需内置支持欧洲语言即可。在 Unicode 出现之前,\LaTeX{} 提供了许多种\textbf{文件编码}来允许很多语言的文字以原生的方式输入,\hologo{pdfLaTeX} 也只需要使用 8 位文件编码和 8 位字体。
% \end{frame}

\section{跑起来}
\begin{frame}
  \frametitle{发行版}
  \begin{table}
    \caption{\hologo{TeX} 发行版}
    \footnotesize
    \begin{stampbox}
      \begin{tabular}{c>{\raggedright}*{3}{p{3.2cm}}}
        \alert{发行版}     & \hologo{MiKTeX} \link{https://mirrors.sjtug.sjtu.edu.cn/ctan/systems/win32/miktex/setup/windows-x64/basic-miktex-21.12-x64.exe}   & \TeX{} Live \link{https://mirrors.sjtug.sjtu.edu.cn/ctan/systems/texlive/tlnet/install-tl.zip}   & Mac\TeX{} \link{https://mirrors.sjtug.sjtu.edu.cn/ctan/systems/mac/mactex/mactex-20210328.pkg}  \\[2pt]
        \alert{特点}      &  只安装必要文件,依赖用时更新  &  所有平台均可使用,每年发布一次 & Mac 系统专用,对 \TeX{} Live 的进一步打包 \\
        \alert{推荐平台}  & \faWindows  & \faLinux &  \faApple \\
      \end{tabular}
    \end{stampbox}
  \end{table}
  \begin{center}
    \parbox{.9\textwidth}{
      要让 \LaTeX{} 跑起来,核心就是要有一套 \TeX{} 发行版,来获取让 \LaTeX{} 工作所需的一组程序和文件。参考《一份简短的关于 \LaTeX{} 安装的介绍》\link{https://mirrors.sjtug.sjtu.edu.cn/ctan/info/install-latex-guide-zh-cn/install-latex-guide-zh-cn.pdf} 安装想使用的发行版。推荐使用发行版的最新版本\footnote{老版本 Linux 系统的包管理器自带 \TeX{} Live 发行版可能不是最新的 \link{https://repology.org/project/texlive/versions},尽量使用镜像安装,并手动将相关环境变量添加到路径 \link{https://www.tug.org/texlive/doc/texlive-zh-cn/texlive-zh-cn.pdf}。},并使用国内镜像。
    }
  \end{center}
\end{frame}

\begin{frame}[plain]
  \hbox to \textwidth{
    \hfil
    \vbox to 3cm{
      \hbox{
        \resizebox{3cm}{!}{\includegraphics{\getcontribpath{sjtug}{vi/sjtug.pdf}}}
      }
    }
    \hfil
    \vbox to 3cm{
      \vfill
      \hbox{\Large\bfseries\color{cprimary} 稳定、快速、现代的镜像服务。}
      \vskip2pt
      \hbox{托管于华东教育网骨干节点上海交通大学。}
      \vfill
    }
    \hskip20pt
    \hfil
  }

  \begin{center}
    \parbox{0.8\textwidth}{
      推荐使用 SJTUG 软件镜像服务,离得近,下得快。
      
      \begin{description}
        \footnotesize
        \item[\TeX{} Live]  {\ttfamily tlmgr option repository https://mirrors.sjtug.sjtu.edu.cn/CTAN/systems/texlive/tlnet}
        \item[\hologo{MiKTeX}] 在 \hologo{MiKTeX} Console 中设置镜像源为 \url{https://mirrors.sjtug.sjtu.edu.cn}
      \end{description}
    }
  \end{center}
\end{frame}

\begin{frame}
  \frametitle{编辑器}
  \begin{table}
    \caption{开源编辑器推荐}
    \footnotesize
    \begin{stampbox}
      \begin{tabular}{c>{\raggedright}*{3}{p{3.5cm}}}
        \alert{编辑器}     & \begin{tabular}{c}Visual Studio Code\\ \LaTeX{} Workshop\end{tabular}  & \TeX{}studio & \TeX{}works \\[5pt]
        \alert{特点}      &  搭配 VS Code 使用非常方便,易扩展  & 可以使用大量的菜单选项输入代码块,用户友好 & 只提供基础的高亮与运行方法,发行版自带\footnote{Mac\TeX{} 打包的是 \TeX{}Shop 编辑器。} \\
      \end{tabular}
    \end{stampbox}
  \end{table}
  \begin{center}
    \parbox{.9\textwidth}{
      使用专为 \LaTeX{} 设计的编辑器将带来更多便利,因为它们往往会提供一键编译、内置 PDF 阅读器以及语法高亮等功能。几乎所有现代的 \LaTeX{} 编辑器都提供 Sync\TeX{} 这一强大的功能,以在 PDF 和 代码间对应跳转。
    }
  \end{center}
\end{frame}

\begin{frame}
  \frametitle{在线平台}
  \begin{table}
    \caption{在线协作平台推荐}
    \footnotesize
    \begin{stampbox}
      \begin{tabular}{c>{\raggedright}*{2}{p{4cm}}}
        \alert{在线平台}     & Overleaf \link{https://www.overleaf.com/}  & 交大 \LaTeX{} 助手 \link{https://latex.sjtu.edu.cn/} \\[2pt]
        \alert{特点}      & 最流行的在线平台,提供大量的模板,但国内访问慢 & 校内平台,隐私保护有保障,共享项目限制少 \\
      \end{tabular}
    \end{stampbox}
  \end{table}
  \begin{center}
    \parbox{.9\textwidth}{
      在线平台允许你直接在网页中编辑文档,无需本地安装即可在后台运行 \LaTeX{},并显示生成的 PDF。可以参照 Overleaf 官方文档学习如何使用在线平台 \link{https://www.overleaf.com/learn}。
    }
  \end{center}
\end{frame}

\section{基本结构}
\begin{frame}[fragile]%
  \frametitle{文档部件}
  \begin{columns}[c]
    \begin{column}{0.4\textwidth}
      \only<1>{
        \cmd{documentclass} 命令加载了\textbf{文档类}。\pkg{article} 是由 \LaTeX{}提供的用于排版短文档的基本文档类。
        \begin{description}
          \footnotesize
          \item[\pkg{article}] 不包含章的短文档
          \item[\pkg{report}] 含有章的单面印刷文档
          \item[\pkg{book}] 含有章的双面印刷文档
          \item[\pkg{beamer}] 制作幻灯片
        \end{description}
      }
      \only<2>{
        \env{document} 环境用于指示文档主体的范围。\LaTeX{} 还有其他的使用 \cmd{begin} 和 \cmd{end} 的搭配,我们称这些为\textbf{环境}。它们将用来设定局部格式命令\footnotemark。
      }
      \only<3>{
        \includepdflarge{enminimal}
      }
    \end{column}
    \begin{column}{0.6\textwidth}
      \begin{codeblock}[]{排版英文最简示例}
|\only<1>{\highlightline}|\documentclass{article}
|\only<2>{\highlightline}|\begin{document}
|\only<3>{\highlightline}|  Together for a Shared Future
|\only<2>{\highlightline}|\end{document}
      \end{codeblock}
    \end{column}
  \end{columns}
  \only<2>{\footnotetext{环境实际上是一个组,只不过通过语义化的形式预装了对应的格式命令。普通的组可以直接使用一对大括号之间的内容 \{$\cdots$\} 表示。}}
\end{frame}

\section{扩展}
\begin{frame}[fragile]%
  \frametitle{中文排版}
  \begin{columns}[c]
    \begin{column}{0.4\textwidth}
      \only<1>{
        \cmd{usepackage} 用于使用宏包以向 \LaTeX{} 添加或修改功能,需要在\textbf{导言区}调用。
        这里使用 \pkg{ctex} 宏集以获得中文支持。其调用底层因随不同的引擎而不同。
        {
          \footnotesize
          \begin{stampbox}
            \begin{tabular}{c*{3}{c}}
              \alert{引擎}     & \hologo{pdfTeX}   & \hologo{XeTeX}   & \hologo{LuaTeX}   \\
              \alert{程序}     & \hologo{pdfLaTeX} & \hologo{XeLaTeX} & \hologo{LuaLaTeX} \\
              \alert{宏包}     & CJK\footnotemark & xeCJK & luatexja \\
              \alert{封装}     & \multicolumn{3}{c}{ctex} \\
            \end{tabular}
          \end{stampbox}
        }
        \vspace{-1cm}
      }
      \only<2>{
        C\TeX{} 建议对于之前提到的常规文档类,最佳实践是使用该宏集提供的四种中文文档类,以对特定类型提供额外的中文排版适配。
        \begin{center}
          \begin{stampbox}
            \footnotesize
            \begin{tabular}{cc}
              \pkg{ctexart} & \pkg{ctexrep} \\
              \pkg{ctexbook} & \pkg{ctexbeamer} \\
            \end{tabular}
          \end{stampbox}
        \end{center}
      }
      \only<3>{
        \includepdflarge{cnminimal}
      }
      \only<4>{
        大部分情况下,你都不应当在 \LaTeX{} 中强制断行:你几乎只是想另起一段,或者是想在段落之间添加空行(使用 \pkg{parskip} 宏包就可实现)。
        只有\alert{很少的}情况下你需要使用 \textbackslash{}\textbackslash{} 来另起一行而不另起一段。
      }
    \end{column}
    \begin{column}{0.6\textwidth}
      \begin{codeblock}[]{排版中文\only<2->{最佳实践}}
|\only<2>{\highlightline}|\documentclass{|\only<1>{article}\only<2->{ctexart}|}
|\only<1>{\highlightline\textbackslash{}usepackage\{ctex\}\hfill\color{cprimary}\% 导言区}|
\begin{document}
|\only<3>{\highlightline}|    一起向未来
|\only<4>{\highlightline}|
  Together for a Shared Future
\end{document}
      \end{codeblock}
    \end{column}
  \end{columns}
  \only<1>{\footnotetext{ctex 在 \faApple\,\faLinux{} 上已经不可以使用 \hologo{pdfLaTeX} 编译,以及在 \faWindows{} 上使用该引擎也会变更自动间距调整等功能的默认行为。}}
\end{frame}

\section{设定格式}
\begin{frame}[fragile]%
  \frametitle{字体样式}
  \begin{columns}
    \begin{column}{0.4\textwidth}
      \only<1>{
        \includepdflarge{fontstyle}
      }
      \only<2>{
        可以使用显示样式设定命令对小段做加粗、斜体、等宽等等的处理。
        \begin{center}
          \footnotesize
          \begin{stampbox}
            \begin{tabular}{rl}
              \cmd{textrm} & \textrm{衬线} \\
              \cmd{textbf} & \textbf{加粗} \\
              \cmd{textit} & \kaishu 斜体 \\
              \cmd{texttt} & \texttt{等宽} \\
              \cmd{textsf} & \textsf{无衬线} \\
              \cmd{textsc} & \textsc{Small Caps} \\
              \cmd{textsl} & \textsl{Slanted} \\
            \end{tabular}
          \end{stampbox}
        \end{center}
      }
      \only<3>{
        与 Word 不同的是,\LaTeX{} 一般情况下并不需要使用上面的显式命令,而是采用逻辑标记的方法,
        比如 \cmd{emph} 可以强调文字,以及下面将要提到的目次命令(第 \ref{sectioning} 页)。
        这样可以统一管理格式。
      }
    \end{column}
    \begin{column}{0.6\textwidth}
      \begin{codeblock}[]{样式}
\documentclass{ctexart}
\begin{document}
|\only<2>{\highlightline}|  \textbf{||一起向未来}

|\only<3>{\highlightline}|  \emph{Together for a Shared Future}
\end{document}
      \end{codeblock}
    \end{column}
  \end{columns}
\end{frame}

\begin{frame}[fragile]%
  \frametitle{\only<1-2>{字体大小}\only<3>{字体样式}}
  \begin{columns}
    \begin{column}{0.4\textwidth}
      \only<1>{
        \includepdflarge{fontsize}
      }
      \only<2>{
        同样地,你也可以显式地设定字体大小,但是这种命令会更改行文设置,所以需要使用一个组来限定作用范围\footnotemark。
        \begin{center}
          \footnotesize
          \begin{stampbox}
            \begin{tabular}{rl}
              \cmd{tiny} & \tiny 极小 \\
              \cmd{scriptsize} & \scriptsize 抄本大小  \\
              \cmd{footnotesize} & \footnotesize 脚注大小 \\
              \cmd{small} & \small 小 \\
              \cmd{normalsize} & \normalsize 正常大小 \\
              \cmd{large} & \large 大 \\
              \cmd{huge} & \Huge 巨大 \\
            \end{tabular}
          \end{stampbox}
        \end{center}
      }
      \only<3>{
        也可以使用字体样式对应的更改字体设置的命令,这对于大段文段的设置而言也是很方便的。
        \begin{center}
          \footnotesize
          \begin{stampbox}
            \begin{tabular}{ll}
              \cmd{textrm} & \cmd{rmfamily}\\
              \cmd{texttt} & \cmd{ttfamily}\\
              \cmd{textsf} & \cmd{sffamily}\\
              \cmd{textbf} & \cmd{bfseries}\\
              \cmd{textit} & \cmd{itshape}\\
              \cmd{textsc} & \cmd{scshape}\\
              \cmd{textsl} & \cmd{slshape}\\
            \end{tabular}
          \end{stampbox}
        \end{center}
      }
    \end{column}
    \begin{column}{0.6\textwidth}
      \begin{codeblock}[]{大小}
\documentclass{ctexart}
\begin{document}
|\only<2>{\highlightline}|  {\bfseries\Large 一起向未来\par}
|\only<3>{\highlightline}|  {\itshape Together for a Shared Future}
\end{document}
      \end{codeblock}
    \end{column}
  \end{columns}
  \only<2>{\footnotetext{注意最后显式地使用 \cmd{par} 在改回大小前结束该段,否则会导致下一行的行间距异常!}}
\end{frame}

\section{逻辑结构}
\begin{frame}[fragile]
  \frametitle{列表}
  \begin{columns}
    \begin{column}{0.35\textwidth}
      \begin{codeblock}[]{无序列表}
\documentclass{ctexart}
\begin{document}
|\highlightline|  \begin{itemize}
    \item 第一项
    \item 第二项
    \item 第三项
|\highlightline|  \end{itemize}
\end{document}
      \end{codeblock}
    \end{column}
    \begin{column}{0.35\textwidth}
      \begin{codeblock}[]{有序列表}
\documentclass{ctexart}
\begin{document}
|\highlightline|  \begin{enumerate}
    \item 第一项
    \item 第二项
    \item 第三项
|\highlightline|  \end{enumerate}
\end{document}
      \end{codeblock}
    \end{column}
    \begin{column}{0.35\textwidth}
      \begin{codeblock}[]{描述列表}
\documentclass{ctexart}
\begin{document}
|\highlightline|  \begin{description}
    \item[||第一] 文本
    \item[||第二] 文本
    \item[||第三] 文本  
|\highlightline|  \end{description}
\end{document}
      \end{codeblock}
    \end{column}
  \end{columns}
\end{frame}

%更深的列表技巧,定理环境等

\begin{frame}
  \frametitle{列表}
  \begin{columns}
    \begin{column}{0.35\textwidth}
      \includepdflarge{unordered}
    \end{column}
    \begin{column}{0.35\textwidth}
      \includepdflarge{ordered}
    \end{column}
    \begin{column}{0.35\textwidth}
      \includepdflarge{description}
    \end{column}
  \end{columns}
\end{frame}

\begin{frame}[fragile,label=sectioning]%
  \frametitle{目次结构}
  \begin{columns}
    \begin{column}{0.4\textwidth}
      \LaTeX{} 可以使用目次命令将文档划分层级\footnotemark,并自动设定对应字体样式和大小。
      \begin{center}
        \begin{stampbox}
          \footnotesize
          \begin{tabular}{rll}
           命令 & 中文 & 层次 \\
           \cmd{chapter} & 章\footnotemark & \sout{0} \\
           \cmd{section} & 节 & 1 \\
           \cmd{subsection} & 小节 & 2 \\
           \cmd{subsubsection} & 小小节 & 3 \\
          \end{tabular}
        \end{stampbox}
      \end{center}
    \end{column}
    \begin{column}{0.6\textwidth}
      \begin{codeblock}[]{目次}
\documentclass{ctexart}
\begin{document}
|\highlightline|  \section{||概念}
|\highlightline|  \subsection{\LaTeX{}}
  \LaTeX{} 是一个用以排版高质量作品的文档准备系统。
\end{document}
      \end{codeblock}
    \end{column}
  \end{columns}
  \footnotetext{章这一级只在 \pkg{report} 和 \pkg{book} 文档类(包括对应的中文文档类)有定义。还有不常用的 \cmd{part} (0@\pkg{article}/-1@\pkg{report}\&\pkg{book}\&\pkg{beamer}) 以及更低层次的 \cmd{paragraph} (4) 与 \cmd{subparagraph} (5)。 }
\end{frame}

\begin{frame}[fragile]%
  \frametitle{组织文档}
  \begin{columns}
    \begin{column}{0.4\textwidth}
      \only<1>{
        \cmd{tableofcontents} 用来生成对于目次命令的目录。如果你想设定显示到哪个层级,在这个命令前使用 \cmd{setcounter\{tocdepth\}\{层次\}}
      }
      \only<2>{
        对于大型文档而言,使用多个文件管理源文件通常是更方便的。而 \cmd{include} 和 \cmd{input} 都以相对路径的方式插入其他 \TeX{} 文档。
        区别在于,\cmd{include} 命令会从新页开始并做一些内部调整,这基本上只对 \pkg{chapter} 这一级有用。而 \cmd{input} 会原样插入源代码。
      }
      \only<3>{
        但是 \cmd{include} 插入的文档可以使用 \cmd{includeonly} 管理当前要排印哪一部分的内容,利用所有章节辅助文件的同时,减少编译时间并专注于该部分的内容。
      }
    \end{column}
    \begin{column}{0.6\textwidth}
      \begin{codeblock}[]{主文档}
\documentclass{ctexrep}
|\only<3>{\highlightline}|\includeonly{learnlatex,sjtuthesis}
\begin{document}
|\only<1>{\highlightline}|  \tableofcontents
|\only<2-3>{\highlightline}|  \include{learnlatex}
|\only<3>{\highlightline}|  \include{sjtuthesis}
\end{document}
      \end{codeblock}
    \end{column}
  \end{columns}
\end{frame}

\begin{frame}[fragile]
  \frametitle{组织文档}
  \begin{columns}
    \begin{column}{0.4\textwidth}
      \begin{codeblock}[]{learnlatex.tex}
|\highlightline|\chapter{||学习 \LaTeX{}}
\section{||概念}
\subsection{\LaTeX{}}
\LaTeX{} 是一个用以排版高质量作品的文档准备系统。
      \end{codeblock}
      子文件中就不需要添加 \env{document} 环境了\footnotemark。
    \end{column}
    \begin{column}{0.6\textwidth}
      \begin{codeblock}[]{主文档}
|\highlightline|\documentclass{ctexrep}
\includeonly{learnlatex,sjtuthesis}
\begin{document}
  \tableofcontents
  \include{learnlatex}
  \include{sjtuthesis}
\end{document}
      \end{codeblock}
    \end{column}
  \end{columns}
  \footnotetext{如果想强制指定子文档的主文档,可以在文件第一行输入魔术命令:\texttt{\% !TeX root = main.tex}}
\end{frame}

\section{图}
\begin{frame}[fragile]%
  \frametitle{\temporal<5>{插图}{浮动体}{插图}}
  \begin{columns}
    \begin{column}{0.6\textwidth}
      \begin{codeblock}[]{插入单图\only<4->{最佳实践}}
\documentclass{ctexart}
|\only<2>{\highlightline}|\usepackage{graphicx}
|\only<2>{\highlightline}|\graphicspath{{figs/}{pics/}}
\begin{document}
|\only<5>{\highlightline}|\begin{figure}[ht]
|\only<6>{\highlightline}|  \centering
|\only<3>{\highlightline}|  \includegraphics[width=|\only<1-3>{4cm}\only<4->{0.4\textbackslash{}textwidth}|]{sjtug}
|\only<7>{\highlightline}|  \caption{SJTUG 徽标}\label{fig:sjtug}
|\only<5>{\highlightline}|\end{figure}
\end{document}
      \end{codeblock}
    \end{column}
    \begin{column}{0.4\textwidth}
      \only<1>{
        \includepdflarge{insertimage}
      }
      \only<2>{
        为了插入外部图片,需要使用 \pkg{graphicx} 宏包。之后在文档主体便可以使用 \cmd{includegraphics} 插入图片。导言区也可以加入 \cmd{graphicspath} 指定图片文件夹\footnotemark。
      }
      \only<3>{
        \cmd{includegraphics} 命令便以相对路径的方式插入图片,如果无同名图片,那么后缀名可以省略。可以使用可选参数指定插入的图片尺寸,最佳实践是使用 \cmd{textwidth} 或 \cmd{linewidth} 的相对值指定尺寸大小,以在未来可能的布局更改中保留一定的灵活性。
      }
      \only<4>{
        也可以通过键值对的方法设置图片的其他属性。
        \begin{center}
          \footnotesize
          \begin{stampbox}
            \begin{tabular}{rl}
              \pkg{width} & 宽度 \\
              \pkg{height} & 高度 \\
              \pkg{scale} & 缩放 \\
              \pkg{angle} & 角度 \\
            \end{tabular}
          \end{stampbox}
        \end{center}
      }
      \only<5>{
        \env{figure} 为一个浮动体环境(\env{table} 也是),可以将其移动到其他位置上以减少行文中的空白。可以添加可选参数以指定如何放置浮动体,最多可以使用四种位置描述符:
        \begin{center}
          \footnotesize
          \begin{stampbox}
            \begin{tabular}{cll}
              \pkg{h} & Here & 尽可能在这里 \\
              \pkg{t} & Top & 页面顶部 \\
              \pkg{b} & Bottom & 页面底部 \\
              \pkg{p} & Page & 浮动体专页 \\
            \end{tabular}
          \end{stampbox}
        \end{center}
        还可以只使用 \pkg{float} 宏包提供的 \pkg{H} 描述符以强制置于此处。
      }
      \only<6>{
        采用 \cmd{centering} 命令而不是 \env{center} 环境来水平居中图片。这将避免多余的纵向间距。
      }
      \only<7>{
        使用 \cmd{caption} 命令输入题注,如果这个命令写在插入图片前面,题注将会在上方(中文中一般对 \env{table} 环境这么做)。后面将会看到如何对留有标记(\cmd{label})的图片进行引用。
      }
    \end{column}
  \end{columns}
  \only<2>{\footnotetext{其命令参数每个为一个以 \texttt{/} 结尾的文件夹,每个文件夹需要使用大括号包裹起来。}}
\end{frame}

\begin{frame}[fragile]
  \begin{columns}
    \begin{column}{0.6\textwidth}
      \begin{codeblock}[]{插入双图}
\documentclass{ctexart}
\usepackage{graphicx}
\graphicspath{{figs/}{pics/}}
\begin{document}
  \begin{figure}[ht]
|\only<1>{\highlightline}|    \begin{minipage}{0.48\textwidth}
      \centering
      \includegraphics[height=2cm]{sjtug}
|\only<2>{\highlightline}|      \caption{SJTUG 徽标}\label{fig:sjtug}
|\only<1>{\highlightline}|    \end{minipage}\hfill
|\only<1>{\highlightline}|    \begin{minipage}{0.48\textwidth}
      \centering
      \includegraphics[height=2cm]{sjtugt}
|\only<2>{\highlightline}|      \caption{SJTUG||文字}\label{fig:sjtugt}
|\only<1>{\highlightline}|    \end{minipage}
  \end{figure}
\end{document}
      \end{codeblock}
    \end{column}
    \begin{column}{0.4\textwidth}
      \only<1>{
        在 \env{figure} 环境里使用 \env{minipage} 小页制作列盒子,以并排两图并分别编号,需要设定强制参数以指定列宽。两个小页之间添加 \cmd{hfill} 使两个小页两端对齐。
      }
      \only<2>{
        在每个小页内部分别使用 \cmd{caption} 添加标签。
      }
      \only<3>{
        \includepdflarge{doubleimages}
      }
    \end{column}
  \end{columns}
\end{frame}

\begin{frame}[fragile]%
  \begin{columns}
    \begin{column}{0.6\textwidth}
      \begin{codeblock}[]{}
\documentclass{ctexart}
\usepackage{graphicx}
|\highlightline|\usepackage{subcaption}
\graphicspath{{figs/}{pics/}}
\begin{document}
  \begin{figure}[ht]
|\highlightline|    \begin{subfigure}{0.48\textwidth}
      \centering
      \includegraphics[height=2cm]{sjtug}
      \caption{||徽标}
|\highlightline|    \end{subfigure}\hfill
|\highlightline|    \begin{subfigure}{0.48\textwidth}
      \centering
      \includegraphics[height=2cm]{sjtugt}
      \caption{||文字}
|\highlightline|    \end{subfigure}
    \caption{SJTUG}\label{fig:sjtug}
  \end{figure}
\end{document}
      \end{codeblock}
    \end{column}
    \begin{column}{0.4\textwidth}
      \includepdflarge{subfigures}\vspace{15pt}
      \pkg{subcaption} 宏包提供了 \env{subfigure} 环境(以及 \env{subtable}),可以用于以子图的形式插入,编号会增加一级。也可以为子图添加子集引用编号。
    \end{column}
  \end{columns}
\end{frame}

\section{表}
\begin{frame}[fragile]
  \frametitle{简单表格}
  \begin{columns}
    \begin{column}{0.6\textwidth}
      \begin{codeblock}[]{}
\documentclass{ctexart}
|\only<1-2>{\highlightline}|\usepackage{|\temporal<1>{array}{\highlight{array}}{array},\temporal<2>{booktabs}{\highlight{booktabs}}{booktabs}|}
\begin{document}
\begin{table}[ht]
  \centering
  \caption{||北京冬奥会吉祥物}
|\only<1>{\highlightline}|  \begin{tabular}{lp{3cm}}
|\only<2>{\highlightline}|    \toprule
|\only<3>{\highlightline}|吉祥物 & 描述                          \\
|\only<2>{\highlightline}|    \midrule
|\only<3>{\highlightline}|冰墩墩 & 2022 年北京冬季奥运会吉祥物  \\
|\only<3>{\highlightline}|雪容融 & 2022 年北京冬季残奥会吉祥物  \\
|\only<2>{\highlightline}|    \bottomrule
|\only<1>{\highlightline}|  \end{tabular}
\end{table}
\end{document}
      \end{codeblock}
    \end{column}
    \begin{column}{0.4\textwidth}
      \only<1>{
        使用 \env{tabular} 环境绘制表格。需要添加参数(称为\textbf{表格导言})以确定每一列的对齐方式。引入 \pkg{array} 宏包来使用更多的\textbf{引导符}。
        \begin{center}
          \footnotesize
          \begin{stampbox}
            \begin{tabular}{>{\ttfamily}ll}
              \alert{l} & 向左对齐 \\
              \alert{c} & 居中对齐 \\
              \alert{r} & 向右对齐 \\
              \alert{p\{3cm\}} & 固定列宽,两端对齐 \\
              \alert{m\{3cm\}} & \texttt{p} + 垂直居中对齐 \\
              \alert{>\{\textbackslash{}bfseries\}} & 后一列单元格前加命令 \\
              \alert{*\{3\}\{l\}} & 三个左对齐列 \\
            \end{tabular}
          \end{stampbox}
        \end{center}
      }
      \only<2>{
        \pkg{booktabs} 宏包提供了标准三线表格所需要的行分割线:\cmd{toprule},\cmd{midrule},\cmd{bottomrule}。也可以使用 \cmd{cmidrule\{1-2\}} 来部分地绘制行分割线。一般不推荐在专业表格中使用纵向分割线。
      }
      \only<3>{
        每行内容使用 \textbackslash\textbackslash{} 分隔开,每行中的单元格使用 \& 分隔开。
      }
      \only<4>{
        \includepdflarge{table}
      }
    \end{column}
  \end{columns}
\end{frame}

\begin{frame}[fragile]%
  \begin{columns}
    \begin{column}{0.6\textwidth}
      \begin{codeblock}[]{表头居中}
\documentclass{ctexart}
\usepackage{array,booktabs}
\begin{document}
\begin{table}[ht]
  \centering
  \caption{||北京冬奥会吉祥物}
  \begin{tabular}{lp{3cm}}
    \toprule
|\highlightline|\multicolumn{1}{c}{||吉祥物} &
|\highlightline|\multicolumn{1}{c}{||描述} \\
    \midrule
||冰墩墩 & 2022 年北京冬季奥运会吉祥物  \\
||雪容融 & 2022 年北京冬季残奥会吉祥物  \\
    \bottomrule
  \end{tabular}
\end{table}
\end{document}
      \end{codeblock}
    \end{column}
    \begin{column}{0.4\textwidth}
      \cmd{multicolumn} 命令不仅可以用于合并同行的单元格,还可以用于临时地屏蔽表格导言对该列的对齐方式定义。这里用于居中表头。
      \begin{center}
        \begin{stampbox}
          \parbox{0.85\linewidth}{
            \ttfamily\color{blue}\textbackslash{}multicolumn\{格数\}\{对齐方式\}\{内容\}
          }
        \end{stampbox}
      \end{center}
      跨页表格考虑使用 \pkg{longtable} 宏包。带标注的表格可以考虑使用 \pkg{threeparttable} 宏包。考虑使用在线工具生成表格代码 \link{https://www.tablesgenerator.com/latex_tables}。
    \end{column}
  \end{columns}
\end{frame}

\section{数学公式}
\begin{frame}
  \frametitle{数学模式}
  \begin{alertblock}{}
  输入数学公式是 \LaTeX{} 的绝对强项,很多常见的公式服务依赖于一些轻量级渲染引擎比如 MathJax, K\kern-.3ex\raise.4ex\hbox{\footnotesize A}\kern-.3ex\TeX{}。但是它们实际上使用的是 \LaTeX{} 语法变种,也就是说并没有使用 \LaTeX{} 后端。所以不要期望有完全一致的输出。
  \end{alertblock}
  
  为了更好的获得数学公式输入支持,请使用 \hologo{AmS}math 宏包。数学模式分为两种:
  \begin{description}
    \item[行内模式] 一般通过一对美元符号(\$$\cdots$\$)标记,可以使用编辑器内建的符号表输入数学符号,也可以使用在线工具手写识别 \link{https://detexify.kirelabs.org/classify.html}。
    \item[行间模式] 一般通过 \env{equation} 环境\footnote{这是有编号环境,其加星号的变种 \env{equation*} 用于生成无编号环境。}输入。如果需要使用多行公式,请使用 \env{align} 环境,并按照类似表格输入的方式,使用 \& 对齐符号,\textbackslash\textbackslash{} 换行。如果不想手动居中,可以考虑多行自动居中的 \env{gather} 和单个大型公式首尾两端对齐 \env{multline}。
  \end{description}
\end{frame}

\begin{frame}
  \frametitle{数学命令展示}
  \begin{columns}
    \begin{column}{0.33\textwidth}
      \begin{exampleblock}{}
        $PV=nRT$
      \end{exampleblock}
      \begin{exampleblock}{}
        $\sum_{i=1}^ki^2=\frac{n(n+1)(2n+1)}{6}$
      \end{exampleblock}
      \begin{exampleblock}{}
        $T(n) = aT\left(\left\lceil\frac{n}{b}\right\rceil\right) + \mathcal{O}(n^d)$
      \end{exampleblock}
      \begin{exampleblock}{}
        $\frac{x_{1}+x_{2}+x_{3}}{3}\geq \sqrt[3]{x_{1}x_{2}x_{3}}$
      \end{exampleblock}
      \begin{exampleblock}{}
        $n=(\underbrace{1\cdots 1}_{k\text{ of 1's}})_2=2^{k+1}-1$
      \end{exampleblock}
      \begin{exampleblock}{}
        $\nabla f (P)= \left.\left(\frac{\partial f}{\partial x},\frac{\partial f}{\partial y},\frac{\partial f}{\partial z}\right)\right|_{P}$
      \end{exampleblock}
    \end{column}
    \begin{column}{0.67\textwidth}
      \begin{exampleblock}{}
        \begin{equation*}
          \int_{a}^b f(x)\,\mathrm{d}x=\lim_{|P|\rightarrow 0}\sum_{i=1}^n f(\xi_i)\Delta x_i
        \end{equation*}
      \end{exampleblock}
      \begin{exampleblock}{}
        \begin{equation}
          T(n) = \begin{cases}
            \mathcal{O}(n^d),&\textrm{if } d>\log_b a, \\
            \mathcal{O}(n^d\log n), &\textrm{if } d=\log_b a,\\
            \mathcal{O}(n^{\log_b a}), &\textrm{if } d<\log_b a.
          \end{cases}
        \end{equation}
      \end{exampleblock}
      \begin{exampleblock}{}
        \begin{align}
          Q^{T}A&=R \\
          \begin{pmatrix}
            q_1^T \\ q_2^T \\ q_3^T
          \end{pmatrix}
          \begin{pmatrix}
            a_1 & a_2 & a_3
          \end{pmatrix}
          &=R
        \end{align}
      \end{exampleblock}
    \end{column}
  \end{columns}
\end{frame}

%更深入地讲解 mathtools, unicode-math, siunix

\section{引用}
\begin{frame}[fragile]
  \frametitle{交叉引用}
  \only<1>{
    正如之前所提到的,\LaTeX{} 中使用 \cmd{label} 标记,然后可以使用 \cmd{ref} 来引用这个标记。 \cmd{label} 可以放在使用计数器的对象之后。
  }
  \only<2>{
    为了使得对公式编号的引用带有括号,推荐使用 \hologo{AmS}math 宏包中的 \cmd{eqref} 命令。对于多行公式环境,每一个换行符前都可以添加一个 \cmd{label} 用于引用该行公式。
  }
  \begin{columns}
    \begin{column}{0.5\textwidth}
      \begin{codeblock}[]{图}
\begin{figure}
|\only<1>{\highlightline}|  \caption{||示例}\label{fig:example}
\end{figure}
      \end{codeblock}
      \begin{codeblock}[]{表}
\begin{table}
|\only<1>{\highlightline}|  \caption{||示例}\label{tab:example}
\end{table}
      \end{codeblock}
    \end{column}
    \begin{column}{0.5\textwidth}
\begin{codeblock}[]{目次}
|\only<1>{\highlightline}|\section{||示例}\label{sec:example}
\end{codeblock}

\begin{codeblock}[]{公式}
\begin{equation}
  a = b + c
|\only<1>{\highlightline}|\label{eq:example}
\end{equation}
|\only<2>{\highlightline}|如公式 \eqref{eq:example} 所示,
\end{codeblock}
    \end{column}
  \end{columns}
\end{frame}

\begin{frame}[fragile]
  \frametitle{文献引用}
  \LaTeX{} 管理参考文献可以采用专用数据库文件 \texttt{.bib},很多的文献管理文件比如 EndNote \link{https://lic.sjtu.edu.cn/Default/softshow/tag/MDAwMDAwMDAwMLGImKE}, Zotero \link{https://www.zotero.org/}, JabRef \link{https://www.jabref.org/} 都可以直接导出这种格式的文件用于 \LaTeX{} 论文中的引用。一般不需要手写数据库文件,你只需要注意每一个文献会在数据库中有一个主键,这个类似于上文的 \cmd{label} 标记,只是要引用数据库中的文献需要使用 \cmd{cite} 命令。
  
  \begin{codeblock}[]{ref.bib}
|\highlightline|@phdthesis{devoftech,|\hfill\alert{\% 类型为博士论文,主键为\texttt{devoftech}}|
  title={||新时期我国信息技术产业的发展},
  author={||江泽民},
  year={2008}
}
  \end{codeblock}
\end{frame}

\begin{frame}
  \frametitle{文献引用}
  而让 \LaTeX{} 处理 \texttt{.bib} 数据库文件与引用有两种工作流。你可能需要查清楚如何在编辑器中设置对应的工作流,或者采用后面所提到的高级编译方式 \texttt{latexmk}。
  \begin{columns}
    \begin{column}{0.5\textwidth}
      \begin{block}{\hologo{BibTeX} + \pkg{gbt7714}}
        一般期刊提交使用这种方法,\pkg{natbib} 宏包将提供命令 \cmd{citet}(文本引用) 和 \cmd{citep}(括号引用)。中文引用可以直接使用 \pkg{gbt7714} 宏包,但是角模式和正文模式不能混用。
      \end{block}
    \end{column}
    \begin{column}{0.5\textwidth}
      \begin{block}{\hologo{biber} + \pkg{biblatex}}
        这是更容易自定义的方法,与 \hologo{BibTeX} 的运作方式稍有不同。\pkg{biblatex} 提供了更加智能的引用命令。而中文引用可以使用 \pkg{biblatex} 宏包的样式 \pkg{gb7714-2015},使用该样式需要使用 \hologo{XeLaTeX} 编译。
      \end{block}
    \end{column}
  \end{columns}
\end{frame}

\begin{frame}[fragile]
  \frametitle{文献引用}
  \begin{columns}
    \begin{column}{0.5\textwidth}
      \begin{codeblock}[]{\hologo{BibTeX} + \pkg{gbt7714}}
\documentclass{ctexart}
\usepackage{gbt7714}
\bibliographystyle{gbt7714-numerial}
% \citestyle{numbers}  % 正文模式
\begin{document}
  ||他指出了...\cite{devoftech}
  \bibliography{ref}
\end{document}
      \end{codeblock}
    \end{column}
    \begin{column}{0.5\textwidth}
      \begin{codeblock}[]{\hologo{biber} + \pkg{biblatex}}
\documentclass{ctexart}
\usepackage[backend=biber,style=gb7714-2015]{biblatex}
\addbibresource{ref.bib}
\begin{document}
  ||他在文献 \parencite{devoftech}
  ||指出了...\cite{devoftech}
  \printbibliography
\end{document}
      \end{codeblock}
    \end{column}
  \end{columns}
\end{frame}

\begin{frame}
  \frametitle{文献引用}
  \begin{columns}
    \begin{column}{0.5\textwidth}
      \includepdflarge{bibtex}
    \end{column}
    \begin{column}{0.5\textwidth}
      \includepdflarge{biblatex}
    \end{column}
  \end{columns}
\end{frame}

} % End of customized shaded number logo

|\only<3>{\highlightline}|  % !TeX root = ..\..\latex-talk.tex

\part{SJTUThesis}

\begin{frame}
  \frametitle{简介}
  \begin{columns}
    \begin{column}{0.6\textwidth}
      \begin{itemize}
        \item 最早由韦建文于 2009 年 11 月发布 0.1a 版,2018 年起由 SJTUG 接手维护
        \item 最新版:\SJTUThesisVersion{} (\SJTUThesisDate)
        \item 支持本科、硕士、博士学位论文以及课程论文的排版
      \end{itemize}
    \end{column}
    \begin{column}{0.4\textwidth}
      \begin{exampleblock}{}
        \begin{minipage}[c]{1cm}
          \includegraphics[width=0.8cm]{\getcontribpath{sjtug}{vi/sjtug}}
        \end{minipage}
        \begin{minipage}[c]{2cm}
          \href{https://github.com/sjtug}{sjtug}/\href{https://github.com/sjtug/SJTUThesis}{SJTUThesis}
        \end{minipage}
      \end{exampleblock}
      \vspace{-8pt}
      \begin{block}{}
        \scriptsize
        上海交通大学 \hologo{XeLaTeX} 学位论文及课程论文模板 | Shanghai Jiao Tong University \hologo{XeLaTeX} Thesis Template
      \end{block}
      \vspace{-8pt}
      \begin{alertblock}{}
        \scriptsize
        \begin{tabular}{cl}
          \faStar & 2.4k \\
          \faEye & 55 \\
          \faCodeBranch & 701 \\
        \end{tabular}
      \end{alertblock}
    \end{column}
  \end{columns}
\end{frame}

\begin{frame}
  \frametitle{下载与编译}
  \alert{下载} 推荐安装 Git \link{https://git-scm.com/} 后,克隆 SJTUG 镜像仓库
  \begin{exampleblock}{\faGit*}
    \ttfamily\small
    git clone https://mirror.sjtu.edu.cn/git/SJTUThesis.git/
  \end{exampleblock}

  \alert{编译} 推荐使用 \pkg{latexmk} 编译\footnote{\hologo{MiKTeX} 用户需要手动安装 Perl 解释器 \link{https://www.perl.org/get.html} 才能使用 \pkg{latexmk}。},在不能够利用自带的 \texttt{.latexmkrc} 配置文件的情况下,需要查清楚在对应的编辑器中如何使用 \hologo{XeLaTeX} + \hologo{biber} 编译 \link{https://github.com/sjtug/SJTUThesis/blob/master/README.md}。
  \begin{exampleblock}{\faTerminal}
    \ttfamily\small
    latexmk -xelatex main
  \end{exampleblock}

  Overleaf 用户可以下载压缩包后,上传并采用 \hologo{XeLaTeX} 编译方式。
\end{frame}

\begin{frame}
  \frametitle{手动编译}
  \alert{第一次编译失败} 如果没有办法通过通常方式编译成功,请尝试使用文件夹内附带 \faLinux{}\,\faApple{} \texttt{Makefile} 和 \faWindows{} \texttt{Compile.bat} 进行编译。

  \alert{统计字数} 编写过程中也可以使用对应的命令调用 \TeX{}count 来统计正文字数。
  \begin{columns}
    \begin{column}{0.38\textwidth}
      \begin{exampleblock}{\faLinux{}\,\faApple}
        \ttfamily
        make all\\
        make clean\\
        make cleanall\\
        make wordcount
      \end{exampleblock}
    \end{column}
    \begin{column}{0.38\textwidth}
      \begin{exampleblock}{\faWindows}
        \ttfamily
        ./Compile.bat thesis\\
        ./Compile.bat clean\\
        ./Compile.bat cleanall\\
        ./Compile.bat wordcount
      \end{exampleblock}
    \end{column}
    \begin{column}{0.24\textwidth}
      \begin{block}{\faInfo}
        \ttfamily
        编译论文\\
        清理中间文件\\
        $\hookrightarrow +$删除论文\\
        统计字数
      \end{block}
    \end{column}
  \end{columns}
\end{frame}

\begin{frame}[label=compile]
  \frametitle{编译问题排查}
  \begin{columns}
    \begin{column}{0.33\textwidth}
      \begin{alertblock}{无法使用 \texttt{latexmk}\thesisissue{578}}
        \hologo{MiKTeX} 需要安装 Perl 解释器。
      \end{alertblock}  
      \begin{alertblock}{C\TeX{} 套装无法编译\thesisissue{446}}
        使用最新 \TeX{} 发行版。
      \end{alertblock}
      \begin{alertblock}{\hologo{pdfLaTeX} 无法编译\thesisissue{444}}
        请使用 \texttt{latexmk},或更改编辑器设置以 \hologo{XeLaTeX} 编译。
      \end{alertblock}
    \end{column}
    \begin{column}{0.33\textwidth}
      \begin{alertblock}{缺少字体\thesisissue{564} \thesisdiscuss{598}}
        更换字体集,或者安装对应字体。
      \end{alertblock}
      \begin{alertblock}{缺少汉字\thesisissue{533} \thesisdiscuss{617}}
        去除使用 fandol 字体集的设定。或者是安装字体后,改用 \texttt{fontset=adobe} 或 \texttt{fontset=founder}。
      \end{alertblock}
    \end{column}
    \begin{column}{0.33\textwidth}
      \begin{block}{\faInfoCircle{} README}
        不同编辑器的设置请首先参阅 README \link{https://github.com/sjtug/SJTUThesis/blob/master/README.md} 文档。
      \end{block}
      \begin{block}{\faBookOpen{} Wiki}
        其他编译问题推荐查阅 Wiki \link{https://github.com/sjtug/SJTUThesis/wiki} 的使用说明部分。
      \end{block}
    \end{column}
  \end{columns}
\end{frame}

\begin{frame}[fragile, label=covers]
  \begin{codeblock}[firstnumber=3]{main.tex}
|\alert{\% 载入 SJTUThesis 模版}|
\documentclass[|\only<1>{\highlight{type}}\only<2>{type}|=|\only<1>{bachelor}\only<2>{\highlight{bachelor}}|]{sjtuthesis}
  \end{codeblock}
  \begin{figure}
    \parbox{0.9\textwidth}{
      \begin{subfigure}{0.20\textwidth}
        \framebox{\includegraphics[width=\linewidth]{support/thesis/bachelor}}
        \caption{\only<1>{学士}\only<2>{\texttt{bachelor}}}
      \end{subfigure}\hfill
      \begin{subfigure}{0.20\textwidth}
        \framebox{\includegraphics[width=\linewidth]{support/thesis/master}}
        \caption{\only<1>{硕士}\only<2>{\texttt{master}}}
      \end{subfigure}\hfill
      \begin{subfigure}{0.20\textwidth}
        \framebox{\includegraphics[width=\linewidth]{support/thesis/doctor}}
        \caption{\only<1>{博士}\only<2>{\texttt{doctor}}}
      \end{subfigure}\hfill
      \begin{subfigure}{0.20\textwidth}
        \framebox{\includegraphics[width=\linewidth]{support/thesis/course}}
        \caption{\only<1>{课程}\only<2>{\texttt{course}}}
      \end{subfigure}
      \caption{论文类型示例\only<2>{ \texttt{type}}}
    }
  \end{figure}
\end{frame}

\begin{frame}[fragile]
  \frametitle{文档类选项}
  % \framesubtitle{\textbackslash{}documentclass\{sjtuthesis\}}
  \begin{columns}
    \begin{column}{0.45\textwidth}
      \includegraphics[page=10]{thesisdir}
    \end{column}
    \begin{column}{0.55\textwidth}
      \begin{table}[H]
        \caption{文档类选项}
        \footnotesize
        \begin{tabular}{>{\ttfamily}rll}
          \toprule
          选项 & 含义 & 相关 \\
          \midrule
          type= & 指定论文类型 & 第 \ref{covers} 页\\
          fontset= & 指定字体 & 第 \ref{compile} 页\\
          \midrule
          review & 开启盲审模式 & \thesisissue{195} \thesisissue{686} \\
          twoside & 双页模式 & \thesisissue{554} \\
          oneside & 单页模式 & \thesisissue{694} \\
          openright & 章从奇数页开始 & \thesisdiscuss{724} \\
          openany & 章从任意页开始 & \thesisissue{446} \\
          \bottomrule
        \end{tabular}
      \end{table}
    \end{column}
  \end{columns}
\end{frame}

\begin{frame}[fragile]
  \frametitle{基本配置}
  \framesubtitle{\textbackslash{}input\{setup\}}
  \begin{columns}
    \begin{column}{0.45\textwidth}
      \includegraphics[page=9]{thesisdir}
    \end{column}
    \begin{column}{0.55\textwidth}
      \begin{codeblock}[firstnumber=12]{main.tex}
|\highlightline<1>|% 论文基本配置,加载宏包等全局配置
|\highlightline<1>|\input{setup}

\begin{document}

%TC:ignore

|\highlightline<2>|% 标题页
|\highlightline<2>|\maketitle
      \end{codeblock}
      \visible<2>{
        \cmd{sjtusetup} 中的 \pkg{info} 将会修改封面的信息设置(见第 \ref{covers} 页)。
      }
    \end{column}
  \end{columns}
\end{frame}

\begin{frame}[fragile]
  \frametitle{基本配置}
  \framesubtitle{\textbackslash{}sjtusetup}
  \begin{columns}
    \begin{column}{0.45\textwidth}
      \includegraphics[page=1]{thesisdir}
    \end{column}
    \begin{column}{0.55\textwidth}
      \begin{codeblock}[firstnumber=3]{setup.tex}
\sjtusetup{
  info = {
    title    = {||上海交通大学学位论文 \LaTeX{} 模板示例文档},
    title*   = {A Sample for \LaTeX-based SJTU Thesis Template},
    author   = {||某\quad{}某},
    author* = {Mo Mo},
  },
  style = { header-logo-color = red, 
  },
  name = {
    publications = {||攻读学位期间完成的论文},
  },
}
      \end{codeblock}
    \end{column}
  \end{columns}
\end{frame}

\begin{frame}
  \frametitle{基本配置}
  \framesubtitle{\textbackslash{}sjtusetup}
  \begin{columns}
    \begin{column}{0.45\textwidth}
      \includegraphics[page=1]{thesisdir}
    \end{column}
    \begin{column}{0.55\textwidth}
      \begin{table}[H]
        \centering
        \caption{info 域}
        \footnotesize
        \begin{tabular}{lll} \toprule
          命令作用 & 中文对应选项 & 英文对应选项 \\ \midrule
          论文标题 & \texttt{title} & \texttt{title*} \\
          关键字列表 & \texttt{keywords} & \texttt{keywords*} \\
          作者姓名&  \texttt{author} &\texttt{author*}\\
          申请学位名称 & \texttt{degree}&\texttt{degree*}\\
          院系名称 & \texttt{department} & \texttt{department*}\\
          专业名称 & \texttt{major} & \texttt{major*}\\
          导师 & \texttt{supervisor} & \texttt{supervisor*}\\
          副导师 & \texttt{assisupervisor} & \texttt{assisupervisor*}\\
          日期 & \multicolumn{2}{c}{\texttt{date}}\\
          学号 & \multicolumn{2}{c}{\texttt{id}}\\ \bottomrule
          \end{tabular}
      \end{table}
    \end{column}
  \end{columns}
\end{frame}

\begin{frame}[fragile]
  \frametitle{版权页}
  \framesubtitle{\textbackslash{}copyrightpage}
  \begin{columns}
    \begin{column}{0.45\textwidth}
      \only<1>{
        \includegraphics[page=9]{thesisdir}
      }
      \only<2>{
        \includegraphics[page=2]{thesisdir}
      }
      \only<3>{
        \begin{figure}[H]
          \framebox{\includegraphics[page=2,width=0.4\linewidth]{bachelor}}
          \caption{版权页}
        \end{figure}
      }
    \end{column}
    \begin{column}{0.55\textwidth}
      \begin{codeblock}[firstnumber=22]{main.tex}
|\highlightline<1>|% 原创性声明及使用授权书
|\highlightline<1>|\copyrightpage
|\highlightline<2>|% 插入外置原创性声明及使用授权书
|\highlightline<2>|% \copyrightpage[scans/sample-copyright-old.pdf]
      \end{codeblock}
      \only<1>{
        \cmd{copyrightpages} 可以用于插入版权页。
      }
      \only<2>{
        \cmd{copyrightpages} 也接受一个可选参数,用于直接使用扫描件。
      }
    \end{column}
  \end{columns}
\end{frame}

\begin{frame}[fragile]
  \frametitle{前置部分}
  \framesubtitle{\textbackslash{}frontmatter}
  \begin{columns}
    \begin{column}{0.45\textwidth}
      \only<1>{
        \includegraphics[page=9]{thesisdir}
      }
      \only<2>{
        \includegraphics[page=3]{thesisdir}
      }
      \only<3>{
        \begin{figure}[H]
          \begin{subfigure}{0.45\textwidth}
            \framebox{\includegraphics[page=3,width=\linewidth]{bachelor}}
            \caption{中文}
          \end{subfigure}\hfill
          \begin{subfigure}{0.45\textwidth}
            \framebox{\includegraphics[page=4,width=\linewidth]{bachelor}}
            \caption{英文}
          \end{subfigure}
          \caption{摘要}
        \end{figure}
      }
      \only<4>{
        \begin{figure}[H]
          \begin{subfigure}{0.30\linewidth}
            \centering
            \framebox{\includegraphics[page=5,width=0.6\linewidth]{bachelor}}
            \caption{目录}
          \end{subfigure}
          \begin{subfigure}{0.30\linewidth}
            \centering
            \framebox{\includegraphics[page=6,width=0.6\linewidth]{bachelor}}
            \caption{插图}
          \end{subfigure}

          \begin{subfigure}{0.30\linewidth}
            \centering
            \framebox{\includegraphics[page=7,width=0.6\linewidth]{bachelor}}
            \caption{表格}
          \end{subfigure}
          \begin{subfigure}{0.30\linewidth}
            \centering
            \framebox{\includegraphics[page=8,width=0.6\linewidth]{bachelor}}
            \caption{算法}
          \end{subfigure}
          \caption{索引}
        \end{figure}
      }
      \only<5>{
        \includegraphics[page=4]{thesisdir}
      }
      \only<6>{
        \begin{figure}[H]
          \framebox{\includegraphics[page=9,width=0.5\linewidth]{bachelor}}
          \caption{符号对照表}
        \end{figure}
      }
    \end{column}
    \begin{column}{0.55\textwidth}
      \begin{codeblock}[firstnumber=30]{main.tex}
|\highlightline<2-3>|% 摘要
|\highlightline<2-3>|\input{contents/abstract}

|\highlightline<4>|% 目录
|\highlightline<4>|\tableofcontents
|\highlightline<4>|% 插图索引
|\highlightline<4>|\listoffigures*
|\highlightline<4>|% 表格索引
|\highlightline<4>|\listoftables*
|\highlightline<4>|% 算法索引
|\highlightline<4>|\listofalgorithms*

|\highlightline<5-6>|% 符号对照表
|\highlightline<5-6>|\input{contents/nomenclature}
      \end{codeblock}
    \end{column}
  \end{columns}
\end{frame}

\begin{frame}[fragile]
  \frametitle{主体部分}
  \framesubtitle{\textbackslash{}mainmatter}
  \begin{columns}
    \begin{column}{0.45\textwidth}
      \only<1>{
        \includegraphics[page=5]{thesisdir}
      }
      \only<2>{
        \begin{figure}[H]
          \begin{subfigure}{0.30\linewidth}
            \centering
            \framebox{\includegraphics[page=11,width=0.6\linewidth]{bachelor}}
            \caption{简介}
          \end{subfigure}
          \begin{subfigure}{0.30\linewidth}
            \centering
            \framebox{\includegraphics[page=13,width=0.6\linewidth]{bachelor}}
            \caption{数学}
          \end{subfigure}

          \begin{subfigure}{0.30\linewidth}
            \centering
            \framebox{\includegraphics[page=16,width=0.6\linewidth]{bachelor}}
            \caption{浮动体}
          \end{subfigure}
          \begin{subfigure}{0.30\linewidth}
            \centering
            \framebox{\includegraphics[page=22,width=0.6\linewidth]{bachelor}}
            \caption{总结}
          \end{subfigure}
          \caption{主体部分}
        \end{figure}
      }
    \end{column}
    \begin{column}{0.55\textwidth}
      \begin{codeblock}[firstnumber=47]{main.tex}
|\highlightline|% 正文内容
|\highlightline|\input{contents/intro}
|\highlightline|\input{contents/math_and_citations}
|\highlightline|\input{contents/floats}
|\highlightline|\input{contents/summary}

%TC:ignore

% 参考文献
\printbibliography[heading=bibintoc]
      \end{codeblock}
    \end{column}
  \end{columns}
\end{frame}

\begin{frame}
  \frametitle{数学}
  \begin{itemize}
    \item 公式示例:\nolinkurl{contents/math_and_citations.tex}
    \item \SJTUThesis{} 定义了常用的数学环境(需要手工引入 \texttt{ntheorem} 宏包):
      \begin{table}[h]
        \centering
        \footnotesize
        \begin{tabular}{*{7}{l}}\toprule
          assumption  & axiom   & conjecture & corollary    & definition  & example   & exercise  \\
          假设        & 公理    & 猜想       & 推论         & 定义        & 例        & 练习      \\\midrule
          lemma       & problem & proof      & proposition  & remark      & solution  & theorem   \\
          引理        & 问题    & 证明       & 命题         & 注          & 解        & 定理      \\\bottomrule
        \end{tabular}
      \end{table}
      \item \SJTUThesis{} 可以通过 \texttt{unimath} 选项使用 \pkg{unicode-math} 进行数学输入,注意与传统方式的区别。\thesisissue{555}
  \end{itemize}
\end{frame}

\begin{frame}[fragile]
  \frametitle{参考文献}
  \begin{columns}
    \begin{column}{0.45\textwidth}
      \includegraphics[page=6]{thesisdir}
    \end{column}
    \begin{column}{0.55\textwidth}
      \begin{codeblock}[firstnumber=111,numbersep=2pt]{setup.tex}
% 使用 BibLaTeX 处理参考文献
%   biblatex-gb7714-2015 常用选项
%     gbnamefmt=lowercase     姓名大小写由输入信息确定
%     gbpub=false             禁用出版信息缺失处理
\usepackage[backend=biber,style=gb7714-2015]{biblatex}
% 文献表字体
% \renewcommand{\bibfont}{\zihao{-5}}
% 文献表条目间的间距
\setlength{\bibitemsep}{0pt}
|\highlightline|% 导入参考文献数据库
|\highlightline|\addbibresource{bibdata/thesis.bib}
      \end{codeblock}
    \end{column}
  \end{columns}
\end{frame}

\begin{frame}[fragile]
  \frametitle{附录}
  \framesubtitle{\textbackslash{}appendix}
  \begin{columns}
    \begin{column}{0.45\textwidth}
      \only<1>{
        \includegraphics[page=7]{thesisdir}
      }
      \only<2>{
        \begin{figure}[H]
          \begin{subfigure}{0.45\linewidth}
            \framebox{\includegraphics[width=\linewidth,page=24]{bachelor}}
            \caption{}
          \end{subfigure}\hfill
          \begin{subfigure}{0.45\textwidth}
            \framebox{\includegraphics[width=\linewidth,page=25]{bachelor}}
            \caption{}
          \end{subfigure}
          \caption{附录}
        \end{figure}
      }
    \end{column}
    \begin{column}{0.55\textwidth}
      \begin{codeblock}[firstnumber=61]{main.tex}
% 附录中图表不加入索引
\captionsetup{list=no}

% 附录内容
|\highlightline|\input{contents/app_maxwell_equations}
|\highlightline|\input{contents/app_flow_chart}
      \end{codeblock}
    \end{column}
  \end{columns}
\end{frame}

\begin{frame}[fragile]
  \frametitle{结尾部分}
  \framesubtitle{\textbackslash{}backmatter}
  \begin{columns}
    \begin{column}{0.45\textwidth}
      \only<1>{
        \includegraphics[page=8]{thesisdir}
      }
      \only<2>{
        \begin{figure}[H]
          \begin{subfigure}{0.30\linewidth}
            \centering
            \framebox{\includegraphics[page=26,width=0.6\linewidth]{bachelor}}
            \caption{致谢}
          \end{subfigure}
          \begin{subfigure}{0.30\linewidth}
            \centering
            \framebox{\includegraphics[page=27,width=0.6\linewidth]{bachelor}}
            \caption{成就}
          \end{subfigure}

          \begin{subfigure}{0.30\linewidth}
            \centering
            \framebox{\includegraphics[page=28,width=0.6\linewidth]{bachelor}}
            \caption{简历}
          \end{subfigure}
          \begin{subfigure}{0.30\linewidth}
            \centering
            \framebox{\includegraphics[page=29,width=0.6\linewidth]{bachelor}}
            \caption{大摘要*}
          \end{subfigure}
          \caption{结尾部分}
        \end{figure}
      }
    \end{column}
    \begin{column}{0.55\textwidth}
      \begin{codeblock}[firstnumber=76]{main.tex}
% 致谢
\input{contents/acknowledgements}

% 发表论文及科研成果
% 盲审论文中,发表论文及科研成果等仅以第几作者注明即可,不要出现作者或他人姓名
\input{contents/achievements}

% 简历
\input{contents/resume}

% 学士学位论文要求在最后有一个大摘要,单独编页码
\input{contents/digest}
      \end{codeblock}
    \end{column}
  \end{columns}
\end{frame}

\begin{frame}
  \frametitle{还有其他问题?}
  \begin{columns}
    \begin{column}{0.75\textwidth}
    \begin{itemize}
      \item[{\faComment*[regular]}] 日常模板或 \LaTeX{} 使用问题可以前往 Discussions \link{https://github.com/sjtug/SJTUThesis/discussions} 提问
      
      (解决后别忘了 \textcolor{green}{\faCheckCircle{} Mark as answer}
      \item[{\faDotCircle[regular]}] 如果是 \textsc{SJTUThesis} 项目本身的 bug 和 feature request
      
      可以通过 Issues \link{https://github.com/sjtug/SJTUThesis/issues} 反馈。
      \item[{\faCodeBranch}] 如果你有好点子,可以贡献代码
     
      向 \textsc{SJTU\TeX{}}(v1) \link{https://github.com/sjtug/SJTUTeX/tree/v1} 存储库发 PR,\par
      而后把解包结果同步到 \textsc{SJTUThesis}。
  
      \item[{\faTag}] 如果你对正在基于 \LaTeX3 开发的新版感兴趣,\par
      也欢迎向 \textsc{SJTU\TeX{}}(v2) \link{https://github.com/sjtug/SJTUTeX/tree/v2} 发 PR。
  
      \item[{\faQq}] 也欢迎在 QQ 群即时讨论。
    \end{itemize}
    \end{column}
    \begin{column}{0.25\textwidth}
      \includegraphics[height=0.7\textheight]{qq.jpg}
    \end{column}
  \end{columns}
\end{frame}
\end{document}
      \end{codeblock}
    \end{column}
  \end{columns}
\end{frame}

\begin{frame}[fragile]
  \frametitle{组织文档}
  \begin{columns}
    \begin{column}{0.4\textwidth}
      \begin{codeblock}[]{learnlatex.tex}
|\highlightline|\chapter{||学习 \LaTeX{}}
\section{||概念}
\subsection{\LaTeX{}}
\LaTeX{} 是一个用以排版高质量作品的文档准备系统。
      \end{codeblock}
      子文件中就不需要添加 \env{document} 环境了\footnotemark。
    \end{column}
    \begin{column}{0.6\textwidth}
      \begin{codeblock}[]{主文档}
|\highlightline|\documentclass{ctexrep}
\includeonly{learnlatex,sjtuthesis}
\begin{document}
  \tableofcontents
  % !TeX root = ..\..\latex-talk.tex

\part{学习 \LaTeX{}}
% FIXME: Part Page miniframe overflow
% FIXME: footnote fault numbering

\begin{frame}[plain]
  \vfil
  \begin{center}
    \href{https://learnlatex.org}{
      \rmfamily
      Learn\,\lower1ex\hbox{\Huge\LaTeX{}}.org
    }
  \end{center}
  \vfil
  \begin{center}
    \parbox{0.75\linewidth}{
      Learn\LaTeX{}.org\cite{learnlatex} 提供了解 \LaTeX{} 的 16 篇简短的教程,并包含了一些可以在线运行的示例,可以通过亲自动手查看实验效果。本部分主要参考由 C\TeX{}-org 提供的中文翻译版本 \link{https://github.com/CTeX-org/learnlatex.github.io/tree/zh-Hans/zh-Hans/}。
    }
  \end{center}
  \vfil
\end{frame}

{ % Start of shaded number logo

\newcommand{\shadedfont}[2][1pt]{
  % #1 (optional): shadow distance
  % #2: the text needed to be shaded
  \hbox{\rlap{\color{gray}\hskip#1#2}#2}
}
\newcounter{learnsec}
\setcounter{learnsec}{-1}
\newcommand{\updatelogo}{
  % update the logo corresponding to current counter.
  \stepcounter{learnsec}
  \logo{
    \raise.3ex\hbox{\tiny\insertsection}\shadedfont{\arabic{learnsec}}
  }
}
\let\oldsection=\section
\renewcommand{\section}[1]{\oldsection{#1}\updatelogo}

\section{是什么}
\begin{frame}
  \frametitle{\TeX{}}
  \begin{columns}[c]
    \begin{column}{0.7\textwidth}
      \begin{center}
        \rmfamily\Huge
        \hologo{La}\highlight[structure!70]{\TeX{}}
      \end{center}
      \begin{center}
        \parbox{0.75\textwidth}{
          \TeX{} 是由斯坦福大学教授高德纳
          (Donald E.~Knuth)于 1977 年开始开发的排版引擎。目前仍在更新,最新版本号为 3.141592653 \link{https://tug.org/TUGboat/tb42-1/tb130knuth-tuneup21.pdf}。
        }
      \end{center}
    \end{column}
    \begin{column}{0.3\textwidth}
      \includegraphics[width=.8\columnwidth]{Knuth.jpg}
    \end{column}
  \end{columns}
\end{frame}

\begin{frame}
  \frametitle{\LaTeX{}}
  \begin{columns}[c]
    \begin{column}{0.7\textwidth}
      \begin{center}
        \rmfamily\Huge
        \highlight[structure]{\LaTeX{}}
      \end{center}
      \begin{center}
        \parbox{0.75\textwidth}{
          \LaTeX{} 是最早在 1985 年由现就职于微软的 Leslie Lamport 开发的一种 \TeX{} \textbf{格式}\footnotemark,使用一些列宏和扩展宏包来简化 \TeX{} 的使用。现在由 \LaTeX{} Project 的成员维护。现在广泛使用的版本是 \LaTeXe{},最新的版本为 \LaTeX3(2020 年 10 月后默认内置)。
        }
      \end{center}
    \end{column}
    \begin{column}{0.3\textwidth}
      \includegraphics[width=.8\columnwidth]{Lamport.jpg}
    \end{column}
  \end{columns}
  \footnotetext{\hologo{ConTeXt} 也是一种 \TeX{} 格式 \link{https://www.contextgarden.net/}。}
\end{frame}

\begin{frame}
  \frametitle{程序}
  \begin{columns}[c]
    \begin{column}{0.7\textwidth}
      \begin{center}
        \rmfamily\Huge
        \highlight[structure]{\hologo{pdfLaTeX}}
      \end{center}
      \begin{center}
        \parbox{0.7\textwidth}{
          \hologo{pdfLaTeX} 是为了编译一个 \LaTeX{} 文档而运行的程序。实际上底层在运行一个叫 \hologo{pdfTeX} 的引擎,并预装了对应的 \LaTeX{} \textbf{格式}。为了利用临时文件,可能就需要多次运行程序。
        }
      \end{center}
    \end{column}
    \begin{column}{0.3\textwidth}
      \begin{block}{}
        \ttfamily\small
        > \highlight{pdflatex} main.tex\\
        This is pdfTeX, Version 3.141592653-
        2.6-1.40.23 (MiKTeX 21.10)\\
        entering extended mode\\
        \highlight{LaTeX2e} <2021-11-15>\\
        \highlight{L3} programming layer <2021-11-22>
      \end{block}
    \end{column}
  \end{columns}
\end{frame}

\begin{frame}
  \frametitle{引擎}
  \begin{columns}[c]
    \begin{column}{0.7\textwidth}
      \begin{center}
        \rmfamily\Huge
        \highlight[structure!70]{pdf}\hologo{La}\highlight[structure!70]{\TeX{}}
      \end{center}
      \begin{center}
        \parbox{0.7\textwidth}{
          pdf\TeX{} 是编译 \TeX{} 文档(以 \texttt{.tex} 结尾)的\textbf{引擎}---可以理解 \TeX{} 指令的\textbf{程序}。
        }
      \end{center}
    \end{column}
    \begin{column}{0.3\textwidth}
      \begin{block}{}
        \ttfamily\small
        > pdflatex main.tex\\
        This is \highlight[structure!70]{pdfTeX}, Version 3.141592653-
        2.6-1.40.23 (MiKTeX 21.10)
        entering extended mode\\
        LaTeX2e <2021-11-15>\\
        L3 programming layer <2021-11-22>
      \end{block}
    \end{column}
  \end{columns}
\end{frame}

\begin{frame}
  \frametitle{Unicode 引擎}
  \begin{table}
    \caption{主流 \hologo{(La)TeX} 程序
    \footnote{(u)p\TeX{} 是日语最常用的引擎,生成 \texttt{.dvi},支持 Unicode。}\footnote{Ap\TeX{} 具有底层 CJK 支持,内联 Ruby,Color Emoji。}}
    \footnotesize
    \begin{stampbox}
      \begin{tabular}{c>{\raggedright}*{3}{p{3.5cm}}}
        \alert{引擎}     & \hologo{pdfTeX}   & \hologo{XeTeX}   & \hologo{LuaTeX}   \\
        \alert{程序}     & \hologo{pdfLaTeX} & \hologo{XeLaTeX} & \hologo{LuaLaTeX} \\
        \alert{特点}     & 直接生成 PDF,支持 micro-typography  & 支持 Unicode、OpenType 与复杂文字编排 (CTL) & 支持 Unicode,内联 Lua,支持 OpenType \\
      \end{tabular}
    \end{stampbox}
  \end{table}

  \begin{center}
    \parbox{.9\textwidth}{
      \hologo{pdfLaTeX} 不支持 Unicode。为了排版中文,大部分情况下 \faApple{}\,\faLinux{} 应当使用 \hologo{XeLaTeX},而 \hologo{LuaLaTeX} 速度相对较慢。\faWindows{} 可以在一些情况下使用 \hologo{pdfLaTeX}。
    }
  \end{center}
\end{frame}

% \begin{frame}
%   \paragraph{\hologo{pdfLaTeX}} \TeX{} 和 \LaTeX{} 被广泛使用之前,它们只需内置支持欧洲语言即可。在 Unicode 出现之前,\LaTeX{} 提供了许多种\textbf{文件编码}来允许很多语言的文字以原生的方式输入,\hologo{pdfLaTeX} 也只需要使用 8 位文件编码和 8 位字体。
% \end{frame}

\section{跑起来}
\begin{frame}
  \frametitle{发行版}
  \begin{table}
    \caption{\hologo{TeX} 发行版}
    \footnotesize
    \begin{stampbox}
      \begin{tabular}{c>{\raggedright}*{3}{p{3.2cm}}}
        \alert{发行版}     & \hologo{MiKTeX} \link{https://mirrors.sjtug.sjtu.edu.cn/ctan/systems/win32/miktex/setup/windows-x64/basic-miktex-21.12-x64.exe}   & \TeX{} Live \link{https://mirrors.sjtug.sjtu.edu.cn/ctan/systems/texlive/tlnet/install-tl.zip}   & Mac\TeX{} \link{https://mirrors.sjtug.sjtu.edu.cn/ctan/systems/mac/mactex/mactex-20210328.pkg}  \\[2pt]
        \alert{特点}      &  只安装必要文件,依赖用时更新  &  所有平台均可使用,每年发布一次 & Mac 系统专用,对 \TeX{} Live 的进一步打包 \\
        \alert{推荐平台}  & \faWindows  & \faLinux &  \faApple \\
      \end{tabular}
    \end{stampbox}
  \end{table}
  \begin{center}
    \parbox{.9\textwidth}{
      要让 \LaTeX{} 跑起来,核心就是要有一套 \TeX{} 发行版,来获取让 \LaTeX{} 工作所需的一组程序和文件。参考《一份简短的关于 \LaTeX{} 安装的介绍》\link{https://mirrors.sjtug.sjtu.edu.cn/ctan/info/install-latex-guide-zh-cn/install-latex-guide-zh-cn.pdf} 安装想使用的发行版。推荐使用发行版的最新版本\footnote{老版本 Linux 系统的包管理器自带 \TeX{} Live 发行版可能不是最新的 \link{https://repology.org/project/texlive/versions},尽量使用镜像安装,并手动将相关环境变量添加到路径 \link{https://www.tug.org/texlive/doc/texlive-zh-cn/texlive-zh-cn.pdf}。},并使用国内镜像。
    }
  \end{center}
\end{frame}

\begin{frame}[plain]
  \hbox to \textwidth{
    \hfil
    \vbox to 3cm{
      \hbox{
        \resizebox{3cm}{!}{\includegraphics{\getcontribpath{sjtug}{vi/sjtug.pdf}}}
      }
    }
    \hfil
    \vbox to 3cm{
      \vfill
      \hbox{\Large\bfseries\color{cprimary} 稳定、快速、现代的镜像服务。}
      \vskip2pt
      \hbox{托管于华东教育网骨干节点上海交通大学。}
      \vfill
    }
    \hskip20pt
    \hfil
  }

  \begin{center}
    \parbox{0.8\textwidth}{
      推荐使用 SJTUG 软件镜像服务,离得近,下得快。
      
      \begin{description}
        \footnotesize
        \item[\TeX{} Live]  {\ttfamily tlmgr option repository https://mirrors.sjtug.sjtu.edu.cn/CTAN/systems/texlive/tlnet}
        \item[\hologo{MiKTeX}] 在 \hologo{MiKTeX} Console 中设置镜像源为 \url{https://mirrors.sjtug.sjtu.edu.cn}
      \end{description}
    }
  \end{center}
\end{frame}

\begin{frame}
  \frametitle{编辑器}
  \begin{table}
    \caption{开源编辑器推荐}
    \footnotesize
    \begin{stampbox}
      \begin{tabular}{c>{\raggedright}*{3}{p{3.5cm}}}
        \alert{编辑器}     & \begin{tabular}{c}Visual Studio Code\\ \LaTeX{} Workshop\end{tabular}  & \TeX{}studio & \TeX{}works \\[5pt]
        \alert{特点}      &  搭配 VS Code 使用非常方便,易扩展  & 可以使用大量的菜单选项输入代码块,用户友好 & 只提供基础的高亮与运行方法,发行版自带\footnote{Mac\TeX{} 打包的是 \TeX{}Shop 编辑器。} \\
      \end{tabular}
    \end{stampbox}
  \end{table}
  \begin{center}
    \parbox{.9\textwidth}{
      使用专为 \LaTeX{} 设计的编辑器将带来更多便利,因为它们往往会提供一键编译、内置 PDF 阅读器以及语法高亮等功能。几乎所有现代的 \LaTeX{} 编辑器都提供 Sync\TeX{} 这一强大的功能,以在 PDF 和 代码间对应跳转。
    }
  \end{center}
\end{frame}

\begin{frame}
  \frametitle{在线平台}
  \begin{table}
    \caption{在线协作平台推荐}
    \footnotesize
    \begin{stampbox}
      \begin{tabular}{c>{\raggedright}*{2}{p{4cm}}}
        \alert{在线平台}     & Overleaf \link{https://www.overleaf.com/}  & 交大 \LaTeX{} 助手 \link{https://latex.sjtu.edu.cn/} \\[2pt]
        \alert{特点}      & 最流行的在线平台,提供大量的模板,但国内访问慢 & 校内平台,隐私保护有保障,共享项目限制少 \\
      \end{tabular}
    \end{stampbox}
  \end{table}
  \begin{center}
    \parbox{.9\textwidth}{
      在线平台允许你直接在网页中编辑文档,无需本地安装即可在后台运行 \LaTeX{},并显示生成的 PDF。可以参照 Overleaf 官方文档学习如何使用在线平台 \link{https://www.overleaf.com/learn}。
    }
  \end{center}
\end{frame}

\section{基本结构}
\begin{frame}[fragile]%
  \frametitle{文档部件}
  \begin{columns}[c]
    \begin{column}{0.4\textwidth}
      \only<1>{
        \cmd{documentclass} 命令加载了\textbf{文档类}。\pkg{article} 是由 \LaTeX{}提供的用于排版短文档的基本文档类。
        \begin{description}
          \footnotesize
          \item[\pkg{article}] 不包含章的短文档
          \item[\pkg{report}] 含有章的单面印刷文档
          \item[\pkg{book}] 含有章的双面印刷文档
          \item[\pkg{beamer}] 制作幻灯片
        \end{description}
      }
      \only<2>{
        \env{document} 环境用于指示文档主体的范围。\LaTeX{} 还有其他的使用 \cmd{begin} 和 \cmd{end} 的搭配,我们称这些为\textbf{环境}。它们将用来设定局部格式命令\footnotemark。
      }
      \only<3>{
        \includepdflarge{enminimal}
      }
    \end{column}
    \begin{column}{0.6\textwidth}
      \begin{codeblock}[]{排版英文最简示例}
|\only<1>{\highlightline}|\documentclass{article}
|\only<2>{\highlightline}|\begin{document}
|\only<3>{\highlightline}|  Together for a Shared Future
|\only<2>{\highlightline}|\end{document}
      \end{codeblock}
    \end{column}
  \end{columns}
  \only<2>{\footnotetext{环境实际上是一个组,只不过通过语义化的形式预装了对应的格式命令。普通的组可以直接使用一对大括号之间的内容 \{$\cdots$\} 表示。}}
\end{frame}

\section{扩展}
\begin{frame}[fragile]%
  \frametitle{中文排版}
  \begin{columns}[c]
    \begin{column}{0.4\textwidth}
      \only<1>{
        \cmd{usepackage} 用于使用宏包以向 \LaTeX{} 添加或修改功能,需要在\textbf{导言区}调用。
        这里使用 \pkg{ctex} 宏集以获得中文支持。其调用底层因随不同的引擎而不同。
        {
          \footnotesize
          \begin{stampbox}
            \begin{tabular}{c*{3}{c}}
              \alert{引擎}     & \hologo{pdfTeX}   & \hologo{XeTeX}   & \hologo{LuaTeX}   \\
              \alert{程序}     & \hologo{pdfLaTeX} & \hologo{XeLaTeX} & \hologo{LuaLaTeX} \\
              \alert{宏包}     & CJK\footnotemark & xeCJK & luatexja \\
              \alert{封装}     & \multicolumn{3}{c}{ctex} \\
            \end{tabular}
          \end{stampbox}
        }
        \vspace{-1cm}
      }
      \only<2>{
        C\TeX{} 建议对于之前提到的常规文档类,最佳实践是使用该宏集提供的四种中文文档类,以对特定类型提供额外的中文排版适配。
        \begin{center}
          \begin{stampbox}
            \footnotesize
            \begin{tabular}{cc}
              \pkg{ctexart} & \pkg{ctexrep} \\
              \pkg{ctexbook} & \pkg{ctexbeamer} \\
            \end{tabular}
          \end{stampbox}
        \end{center}
      }
      \only<3>{
        \includepdflarge{cnminimal}
      }
      \only<4>{
        大部分情况下,你都不应当在 \LaTeX{} 中强制断行:你几乎只是想另起一段,或者是想在段落之间添加空行(使用 \pkg{parskip} 宏包就可实现)。
        只有\alert{很少的}情况下你需要使用 \textbackslash{}\textbackslash{} 来另起一行而不另起一段。
      }
    \end{column}
    \begin{column}{0.6\textwidth}
      \begin{codeblock}[]{排版中文\only<2->{最佳实践}}
|\only<2>{\highlightline}|\documentclass{|\only<1>{article}\only<2->{ctexart}|}
|\only<1>{\highlightline\textbackslash{}usepackage\{ctex\}\hfill\color{cprimary}\% 导言区}|
\begin{document}
|\only<3>{\highlightline}|    一起向未来
|\only<4>{\highlightline}|
  Together for a Shared Future
\end{document}
      \end{codeblock}
    \end{column}
  \end{columns}
  \only<1>{\footnotetext{ctex 在 \faApple\,\faLinux{} 上已经不可以使用 \hologo{pdfLaTeX} 编译,以及在 \faWindows{} 上使用该引擎也会变更自动间距调整等功能的默认行为。}}
\end{frame}

\section{设定格式}
\begin{frame}[fragile]%
  \frametitle{字体样式}
  \begin{columns}
    \begin{column}{0.4\textwidth}
      \only<1>{
        \includepdflarge{fontstyle}
      }
      \only<2>{
        可以使用显示样式设定命令对小段做加粗、斜体、等宽等等的处理。
        \begin{center}
          \footnotesize
          \begin{stampbox}
            \begin{tabular}{rl}
              \cmd{textrm} & \textrm{衬线} \\
              \cmd{textbf} & \textbf{加粗} \\
              \cmd{textit} & \kaishu 斜体 \\
              \cmd{texttt} & \texttt{等宽} \\
              \cmd{textsf} & \textsf{无衬线} \\
              \cmd{textsc} & \textsc{Small Caps} \\
              \cmd{textsl} & \textsl{Slanted} \\
            \end{tabular}
          \end{stampbox}
        \end{center}
      }
      \only<3>{
        与 Word 不同的是,\LaTeX{} 一般情况下并不需要使用上面的显式命令,而是采用逻辑标记的方法,
        比如 \cmd{emph} 可以强调文字,以及下面将要提到的目次命令(第 \ref{sectioning} 页)。
        这样可以统一管理格式。
      }
    \end{column}
    \begin{column}{0.6\textwidth}
      \begin{codeblock}[]{样式}
\documentclass{ctexart}
\begin{document}
|\only<2>{\highlightline}|  \textbf{||一起向未来}

|\only<3>{\highlightline}|  \emph{Together for a Shared Future}
\end{document}
      \end{codeblock}
    \end{column}
  \end{columns}
\end{frame}

\begin{frame}[fragile]%
  \frametitle{\only<1-2>{字体大小}\only<3>{字体样式}}
  \begin{columns}
    \begin{column}{0.4\textwidth}
      \only<1>{
        \includepdflarge{fontsize}
      }
      \only<2>{
        同样地,你也可以显式地设定字体大小,但是这种命令会更改行文设置,所以需要使用一个组来限定作用范围\footnotemark。
        \begin{center}
          \footnotesize
          \begin{stampbox}
            \begin{tabular}{rl}
              \cmd{tiny} & \tiny 极小 \\
              \cmd{scriptsize} & \scriptsize 抄本大小  \\
              \cmd{footnotesize} & \footnotesize 脚注大小 \\
              \cmd{small} & \small 小 \\
              \cmd{normalsize} & \normalsize 正常大小 \\
              \cmd{large} & \large 大 \\
              \cmd{huge} & \Huge 巨大 \\
            \end{tabular}
          \end{stampbox}
        \end{center}
      }
      \only<3>{
        也可以使用字体样式对应的更改字体设置的命令,这对于大段文段的设置而言也是很方便的。
        \begin{center}
          \footnotesize
          \begin{stampbox}
            \begin{tabular}{ll}
              \cmd{textrm} & \cmd{rmfamily}\\
              \cmd{texttt} & \cmd{ttfamily}\\
              \cmd{textsf} & \cmd{sffamily}\\
              \cmd{textbf} & \cmd{bfseries}\\
              \cmd{textit} & \cmd{itshape}\\
              \cmd{textsc} & \cmd{scshape}\\
              \cmd{textsl} & \cmd{slshape}\\
            \end{tabular}
          \end{stampbox}
        \end{center}
      }
    \end{column}
    \begin{column}{0.6\textwidth}
      \begin{codeblock}[]{大小}
\documentclass{ctexart}
\begin{document}
|\only<2>{\highlightline}|  {\bfseries\Large 一起向未来\par}
|\only<3>{\highlightline}|  {\itshape Together for a Shared Future}
\end{document}
      \end{codeblock}
    \end{column}
  \end{columns}
  \only<2>{\footnotetext{注意最后显式地使用 \cmd{par} 在改回大小前结束该段,否则会导致下一行的行间距异常!}}
\end{frame}

\section{逻辑结构}
\begin{frame}[fragile]
  \frametitle{列表}
  \begin{columns}
    \begin{column}{0.35\textwidth}
      \begin{codeblock}[]{无序列表}
\documentclass{ctexart}
\begin{document}
|\highlightline|  \begin{itemize}
    \item 第一项
    \item 第二项
    \item 第三项
|\highlightline|  \end{itemize}
\end{document}
      \end{codeblock}
    \end{column}
    \begin{column}{0.35\textwidth}
      \begin{codeblock}[]{有序列表}
\documentclass{ctexart}
\begin{document}
|\highlightline|  \begin{enumerate}
    \item 第一项
    \item 第二项
    \item 第三项
|\highlightline|  \end{enumerate}
\end{document}
      \end{codeblock}
    \end{column}
    \begin{column}{0.35\textwidth}
      \begin{codeblock}[]{描述列表}
\documentclass{ctexart}
\begin{document}
|\highlightline|  \begin{description}
    \item[||第一] 文本
    \item[||第二] 文本
    \item[||第三] 文本  
|\highlightline|  \end{description}
\end{document}
      \end{codeblock}
    \end{column}
  \end{columns}
\end{frame}

%更深的列表技巧,定理环境等

\begin{frame}
  \frametitle{列表}
  \begin{columns}
    \begin{column}{0.35\textwidth}
      \includepdflarge{unordered}
    \end{column}
    \begin{column}{0.35\textwidth}
      \includepdflarge{ordered}
    \end{column}
    \begin{column}{0.35\textwidth}
      \includepdflarge{description}
    \end{column}
  \end{columns}
\end{frame}

\begin{frame}[fragile,label=sectioning]%
  \frametitle{目次结构}
  \begin{columns}
    \begin{column}{0.4\textwidth}
      \LaTeX{} 可以使用目次命令将文档划分层级\footnotemark,并自动设定对应字体样式和大小。
      \begin{center}
        \begin{stampbox}
          \footnotesize
          \begin{tabular}{rll}
           命令 & 中文 & 层次 \\
           \cmd{chapter} & 章\footnotemark & \sout{0} \\
           \cmd{section} & 节 & 1 \\
           \cmd{subsection} & 小节 & 2 \\
           \cmd{subsubsection} & 小小节 & 3 \\
          \end{tabular}
        \end{stampbox}
      \end{center}
    \end{column}
    \begin{column}{0.6\textwidth}
      \begin{codeblock}[]{目次}
\documentclass{ctexart}
\begin{document}
|\highlightline|  \section{||概念}
|\highlightline|  \subsection{\LaTeX{}}
  \LaTeX{} 是一个用以排版高质量作品的文档准备系统。
\end{document}
      \end{codeblock}
    \end{column}
  \end{columns}
  \footnotetext{章这一级只在 \pkg{report} 和 \pkg{book} 文档类(包括对应的中文文档类)有定义。还有不常用的 \cmd{part} (0@\pkg{article}/-1@\pkg{report}\&\pkg{book}\&\pkg{beamer}) 以及更低层次的 \cmd{paragraph} (4) 与 \cmd{subparagraph} (5)。 }
\end{frame}

\begin{frame}[fragile]%
  \frametitle{组织文档}
  \begin{columns}
    \begin{column}{0.4\textwidth}
      \only<1>{
        \cmd{tableofcontents} 用来生成对于目次命令的目录。如果你想设定显示到哪个层级,在这个命令前使用 \cmd{setcounter\{tocdepth\}\{层次\}}
      }
      \only<2>{
        对于大型文档而言,使用多个文件管理源文件通常是更方便的。而 \cmd{include} 和 \cmd{input} 都以相对路径的方式插入其他 \TeX{} 文档。
        区别在于,\cmd{include} 命令会从新页开始并做一些内部调整,这基本上只对 \pkg{chapter} 这一级有用。而 \cmd{input} 会原样插入源代码。
      }
      \only<3>{
        但是 \cmd{include} 插入的文档可以使用 \cmd{includeonly} 管理当前要排印哪一部分的内容,利用所有章节辅助文件的同时,减少编译时间并专注于该部分的内容。
      }
    \end{column}
    \begin{column}{0.6\textwidth}
      \begin{codeblock}[]{主文档}
\documentclass{ctexrep}
|\only<3>{\highlightline}|\includeonly{learnlatex,sjtuthesis}
\begin{document}
|\only<1>{\highlightline}|  \tableofcontents
|\only<2-3>{\highlightline}|  \include{learnlatex}
|\only<3>{\highlightline}|  \include{sjtuthesis}
\end{document}
      \end{codeblock}
    \end{column}
  \end{columns}
\end{frame}

\begin{frame}[fragile]
  \frametitle{组织文档}
  \begin{columns}
    \begin{column}{0.4\textwidth}
      \begin{codeblock}[]{learnlatex.tex}
|\highlightline|\chapter{||学习 \LaTeX{}}
\section{||概念}
\subsection{\LaTeX{}}
\LaTeX{} 是一个用以排版高质量作品的文档准备系统。
      \end{codeblock}
      子文件中就不需要添加 \env{document} 环境了\footnotemark。
    \end{column}
    \begin{column}{0.6\textwidth}
      \begin{codeblock}[]{主文档}
|\highlightline|\documentclass{ctexrep}
\includeonly{learnlatex,sjtuthesis}
\begin{document}
  \tableofcontents
  \include{learnlatex}
  \include{sjtuthesis}
\end{document}
      \end{codeblock}
    \end{column}
  \end{columns}
  \footnotetext{如果想强制指定子文档的主文档,可以在文件第一行输入魔术命令:\texttt{\% !TeX root = main.tex}}
\end{frame}

\section{图}
\begin{frame}[fragile]%
  \frametitle{\temporal<5>{插图}{浮动体}{插图}}
  \begin{columns}
    \begin{column}{0.6\textwidth}
      \begin{codeblock}[]{插入单图\only<4->{最佳实践}}
\documentclass{ctexart}
|\only<2>{\highlightline}|\usepackage{graphicx}
|\only<2>{\highlightline}|\graphicspath{{figs/}{pics/}}
\begin{document}
|\only<5>{\highlightline}|\begin{figure}[ht]
|\only<6>{\highlightline}|  \centering
|\only<3>{\highlightline}|  \includegraphics[width=|\only<1-3>{4cm}\only<4->{0.4\textbackslash{}textwidth}|]{sjtug}
|\only<7>{\highlightline}|  \caption{SJTUG 徽标}\label{fig:sjtug}
|\only<5>{\highlightline}|\end{figure}
\end{document}
      \end{codeblock}
    \end{column}
    \begin{column}{0.4\textwidth}
      \only<1>{
        \includepdflarge{insertimage}
      }
      \only<2>{
        为了插入外部图片,需要使用 \pkg{graphicx} 宏包。之后在文档主体便可以使用 \cmd{includegraphics} 插入图片。导言区也可以加入 \cmd{graphicspath} 指定图片文件夹\footnotemark。
      }
      \only<3>{
        \cmd{includegraphics} 命令便以相对路径的方式插入图片,如果无同名图片,那么后缀名可以省略。可以使用可选参数指定插入的图片尺寸,最佳实践是使用 \cmd{textwidth} 或 \cmd{linewidth} 的相对值指定尺寸大小,以在未来可能的布局更改中保留一定的灵活性。
      }
      \only<4>{
        也可以通过键值对的方法设置图片的其他属性。
        \begin{center}
          \footnotesize
          \begin{stampbox}
            \begin{tabular}{rl}
              \pkg{width} & 宽度 \\
              \pkg{height} & 高度 \\
              \pkg{scale} & 缩放 \\
              \pkg{angle} & 角度 \\
            \end{tabular}
          \end{stampbox}
        \end{center}
      }
      \only<5>{
        \env{figure} 为一个浮动体环境(\env{table} 也是),可以将其移动到其他位置上以减少行文中的空白。可以添加可选参数以指定如何放置浮动体,最多可以使用四种位置描述符:
        \begin{center}
          \footnotesize
          \begin{stampbox}
            \begin{tabular}{cll}
              \pkg{h} & Here & 尽可能在这里 \\
              \pkg{t} & Top & 页面顶部 \\
              \pkg{b} & Bottom & 页面底部 \\
              \pkg{p} & Page & 浮动体专页 \\
            \end{tabular}
          \end{stampbox}
        \end{center}
        还可以只使用 \pkg{float} 宏包提供的 \pkg{H} 描述符以强制置于此处。
      }
      \only<6>{
        采用 \cmd{centering} 命令而不是 \env{center} 环境来水平居中图片。这将避免多余的纵向间距。
      }
      \only<7>{
        使用 \cmd{caption} 命令输入题注,如果这个命令写在插入图片前面,题注将会在上方(中文中一般对 \env{table} 环境这么做)。后面将会看到如何对留有标记(\cmd{label})的图片进行引用。
      }
    \end{column}
  \end{columns}
  \only<2>{\footnotetext{其命令参数每个为一个以 \texttt{/} 结尾的文件夹,每个文件夹需要使用大括号包裹起来。}}
\end{frame}

\begin{frame}[fragile]
  \begin{columns}
    \begin{column}{0.6\textwidth}
      \begin{codeblock}[]{插入双图}
\documentclass{ctexart}
\usepackage{graphicx}
\graphicspath{{figs/}{pics/}}
\begin{document}
  \begin{figure}[ht]
|\only<1>{\highlightline}|    \begin{minipage}{0.48\textwidth}
      \centering
      \includegraphics[height=2cm]{sjtug}
|\only<2>{\highlightline}|      \caption{SJTUG 徽标}\label{fig:sjtug}
|\only<1>{\highlightline}|    \end{minipage}\hfill
|\only<1>{\highlightline}|    \begin{minipage}{0.48\textwidth}
      \centering
      \includegraphics[height=2cm]{sjtugt}
|\only<2>{\highlightline}|      \caption{SJTUG||文字}\label{fig:sjtugt}
|\only<1>{\highlightline}|    \end{minipage}
  \end{figure}
\end{document}
      \end{codeblock}
    \end{column}
    \begin{column}{0.4\textwidth}
      \only<1>{
        在 \env{figure} 环境里使用 \env{minipage} 小页制作列盒子,以并排两图并分别编号,需要设定强制参数以指定列宽。两个小页之间添加 \cmd{hfill} 使两个小页两端对齐。
      }
      \only<2>{
        在每个小页内部分别使用 \cmd{caption} 添加标签。
      }
      \only<3>{
        \includepdflarge{doubleimages}
      }
    \end{column}
  \end{columns}
\end{frame}

\begin{frame}[fragile]%
  \begin{columns}
    \begin{column}{0.6\textwidth}
      \begin{codeblock}[]{}
\documentclass{ctexart}
\usepackage{graphicx}
|\highlightline|\usepackage{subcaption}
\graphicspath{{figs/}{pics/}}
\begin{document}
  \begin{figure}[ht]
|\highlightline|    \begin{subfigure}{0.48\textwidth}
      \centering
      \includegraphics[height=2cm]{sjtug}
      \caption{||徽标}
|\highlightline|    \end{subfigure}\hfill
|\highlightline|    \begin{subfigure}{0.48\textwidth}
      \centering
      \includegraphics[height=2cm]{sjtugt}
      \caption{||文字}
|\highlightline|    \end{subfigure}
    \caption{SJTUG}\label{fig:sjtug}
  \end{figure}
\end{document}
      \end{codeblock}
    \end{column}
    \begin{column}{0.4\textwidth}
      \includepdflarge{subfigures}\vspace{15pt}
      \pkg{subcaption} 宏包提供了 \env{subfigure} 环境(以及 \env{subtable}),可以用于以子图的形式插入,编号会增加一级。也可以为子图添加子集引用编号。
    \end{column}
  \end{columns}
\end{frame}

\section{表}
\begin{frame}[fragile]
  \frametitle{简单表格}
  \begin{columns}
    \begin{column}{0.6\textwidth}
      \begin{codeblock}[]{}
\documentclass{ctexart}
|\only<1-2>{\highlightline}|\usepackage{|\temporal<1>{array}{\highlight{array}}{array},\temporal<2>{booktabs}{\highlight{booktabs}}{booktabs}|}
\begin{document}
\begin{table}[ht]
  \centering
  \caption{||北京冬奥会吉祥物}
|\only<1>{\highlightline}|  \begin{tabular}{lp{3cm}}
|\only<2>{\highlightline}|    \toprule
|\only<3>{\highlightline}|吉祥物 & 描述                          \\
|\only<2>{\highlightline}|    \midrule
|\only<3>{\highlightline}|冰墩墩 & 2022 年北京冬季奥运会吉祥物  \\
|\only<3>{\highlightline}|雪容融 & 2022 年北京冬季残奥会吉祥物  \\
|\only<2>{\highlightline}|    \bottomrule
|\only<1>{\highlightline}|  \end{tabular}
\end{table}
\end{document}
      \end{codeblock}
    \end{column}
    \begin{column}{0.4\textwidth}
      \only<1>{
        使用 \env{tabular} 环境绘制表格。需要添加参数(称为\textbf{表格导言})以确定每一列的对齐方式。引入 \pkg{array} 宏包来使用更多的\textbf{引导符}。
        \begin{center}
          \footnotesize
          \begin{stampbox}
            \begin{tabular}{>{\ttfamily}ll}
              \alert{l} & 向左对齐 \\
              \alert{c} & 居中对齐 \\
              \alert{r} & 向右对齐 \\
              \alert{p\{3cm\}} & 固定列宽,两端对齐 \\
              \alert{m\{3cm\}} & \texttt{p} + 垂直居中对齐 \\
              \alert{>\{\textbackslash{}bfseries\}} & 后一列单元格前加命令 \\
              \alert{*\{3\}\{l\}} & 三个左对齐列 \\
            \end{tabular}
          \end{stampbox}
        \end{center}
      }
      \only<2>{
        \pkg{booktabs} 宏包提供了标准三线表格所需要的行分割线:\cmd{toprule},\cmd{midrule},\cmd{bottomrule}。也可以使用 \cmd{cmidrule\{1-2\}} 来部分地绘制行分割线。一般不推荐在专业表格中使用纵向分割线。
      }
      \only<3>{
        每行内容使用 \textbackslash\textbackslash{} 分隔开,每行中的单元格使用 \& 分隔开。
      }
      \only<4>{
        \includepdflarge{table}
      }
    \end{column}
  \end{columns}
\end{frame}

\begin{frame}[fragile]%
  \begin{columns}
    \begin{column}{0.6\textwidth}
      \begin{codeblock}[]{表头居中}
\documentclass{ctexart}
\usepackage{array,booktabs}
\begin{document}
\begin{table}[ht]
  \centering
  \caption{||北京冬奥会吉祥物}
  \begin{tabular}{lp{3cm}}
    \toprule
|\highlightline|\multicolumn{1}{c}{||吉祥物} &
|\highlightline|\multicolumn{1}{c}{||描述} \\
    \midrule
||冰墩墩 & 2022 年北京冬季奥运会吉祥物  \\
||雪容融 & 2022 年北京冬季残奥会吉祥物  \\
    \bottomrule
  \end{tabular}
\end{table}
\end{document}
      \end{codeblock}
    \end{column}
    \begin{column}{0.4\textwidth}
      \cmd{multicolumn} 命令不仅可以用于合并同行的单元格,还可以用于临时地屏蔽表格导言对该列的对齐方式定义。这里用于居中表头。
      \begin{center}
        \begin{stampbox}
          \parbox{0.85\linewidth}{
            \ttfamily\color{blue}\textbackslash{}multicolumn\{格数\}\{对齐方式\}\{内容\}
          }
        \end{stampbox}
      \end{center}
      跨页表格考虑使用 \pkg{longtable} 宏包。带标注的表格可以考虑使用 \pkg{threeparttable} 宏包。考虑使用在线工具生成表格代码 \link{https://www.tablesgenerator.com/latex_tables}。
    \end{column}
  \end{columns}
\end{frame}

\section{数学公式}
\begin{frame}
  \frametitle{数学模式}
  \begin{alertblock}{}
  输入数学公式是 \LaTeX{} 的绝对强项,很多常见的公式服务依赖于一些轻量级渲染引擎比如 MathJax, K\kern-.3ex\raise.4ex\hbox{\footnotesize A}\kern-.3ex\TeX{}。但是它们实际上使用的是 \LaTeX{} 语法变种,也就是说并没有使用 \LaTeX{} 后端。所以不要期望有完全一致的输出。
  \end{alertblock}
  
  为了更好的获得数学公式输入支持,请使用 \hologo{AmS}math 宏包。数学模式分为两种:
  \begin{description}
    \item[行内模式] 一般通过一对美元符号(\$$\cdots$\$)标记,可以使用编辑器内建的符号表输入数学符号,也可以使用在线工具手写识别 \link{https://detexify.kirelabs.org/classify.html}。
    \item[行间模式] 一般通过 \env{equation} 环境\footnote{这是有编号环境,其加星号的变种 \env{equation*} 用于生成无编号环境。}输入。如果需要使用多行公式,请使用 \env{align} 环境,并按照类似表格输入的方式,使用 \& 对齐符号,\textbackslash\textbackslash{} 换行。如果不想手动居中,可以考虑多行自动居中的 \env{gather} 和单个大型公式首尾两端对齐 \env{multline}。
  \end{description}
\end{frame}

\begin{frame}
  \frametitle{数学命令展示}
  \begin{columns}
    \begin{column}{0.33\textwidth}
      \begin{exampleblock}{}
        $PV=nRT$
      \end{exampleblock}
      \begin{exampleblock}{}
        $\sum_{i=1}^ki^2=\frac{n(n+1)(2n+1)}{6}$
      \end{exampleblock}
      \begin{exampleblock}{}
        $T(n) = aT\left(\left\lceil\frac{n}{b}\right\rceil\right) + \mathcal{O}(n^d)$
      \end{exampleblock}
      \begin{exampleblock}{}
        $\frac{x_{1}+x_{2}+x_{3}}{3}\geq \sqrt[3]{x_{1}x_{2}x_{3}}$
      \end{exampleblock}
      \begin{exampleblock}{}
        $n=(\underbrace{1\cdots 1}_{k\text{ of 1's}})_2=2^{k+1}-1$
      \end{exampleblock}
      \begin{exampleblock}{}
        $\nabla f (P)= \left.\left(\frac{\partial f}{\partial x},\frac{\partial f}{\partial y},\frac{\partial f}{\partial z}\right)\right|_{P}$
      \end{exampleblock}
    \end{column}
    \begin{column}{0.67\textwidth}
      \begin{exampleblock}{}
        \begin{equation*}
          \int_{a}^b f(x)\,\mathrm{d}x=\lim_{|P|\rightarrow 0}\sum_{i=1}^n f(\xi_i)\Delta x_i
        \end{equation*}
      \end{exampleblock}
      \begin{exampleblock}{}
        \begin{equation}
          T(n) = \begin{cases}
            \mathcal{O}(n^d),&\textrm{if } d>\log_b a, \\
            \mathcal{O}(n^d\log n), &\textrm{if } d=\log_b a,\\
            \mathcal{O}(n^{\log_b a}), &\textrm{if } d<\log_b a.
          \end{cases}
        \end{equation}
      \end{exampleblock}
      \begin{exampleblock}{}
        \begin{align}
          Q^{T}A&=R \\
          \begin{pmatrix}
            q_1^T \\ q_2^T \\ q_3^T
          \end{pmatrix}
          \begin{pmatrix}
            a_1 & a_2 & a_3
          \end{pmatrix}
          &=R
        \end{align}
      \end{exampleblock}
    \end{column}
  \end{columns}
\end{frame}

%更深入地讲解 mathtools, unicode-math, siunix

\section{引用}
\begin{frame}[fragile]
  \frametitle{交叉引用}
  \only<1>{
    正如之前所提到的,\LaTeX{} 中使用 \cmd{label} 标记,然后可以使用 \cmd{ref} 来引用这个标记。 \cmd{label} 可以放在使用计数器的对象之后。
  }
  \only<2>{
    为了使得对公式编号的引用带有括号,推荐使用 \hologo{AmS}math 宏包中的 \cmd{eqref} 命令。对于多行公式环境,每一个换行符前都可以添加一个 \cmd{label} 用于引用该行公式。
  }
  \begin{columns}
    \begin{column}{0.5\textwidth}
      \begin{codeblock}[]{图}
\begin{figure}
|\only<1>{\highlightline}|  \caption{||示例}\label{fig:example}
\end{figure}
      \end{codeblock}
      \begin{codeblock}[]{表}
\begin{table}
|\only<1>{\highlightline}|  \caption{||示例}\label{tab:example}
\end{table}
      \end{codeblock}
    \end{column}
    \begin{column}{0.5\textwidth}
\begin{codeblock}[]{目次}
|\only<1>{\highlightline}|\section{||示例}\label{sec:example}
\end{codeblock}

\begin{codeblock}[]{公式}
\begin{equation}
  a = b + c
|\only<1>{\highlightline}|\label{eq:example}
\end{equation}
|\only<2>{\highlightline}|如公式 \eqref{eq:example} 所示,
\end{codeblock}
    \end{column}
  \end{columns}
\end{frame}

\begin{frame}[fragile]
  \frametitle{文献引用}
  \LaTeX{} 管理参考文献可以采用专用数据库文件 \texttt{.bib},很多的文献管理文件比如 EndNote \link{https://lic.sjtu.edu.cn/Default/softshow/tag/MDAwMDAwMDAwMLGImKE}, Zotero \link{https://www.zotero.org/}, JabRef \link{https://www.jabref.org/} 都可以直接导出这种格式的文件用于 \LaTeX{} 论文中的引用。一般不需要手写数据库文件,你只需要注意每一个文献会在数据库中有一个主键,这个类似于上文的 \cmd{label} 标记,只是要引用数据库中的文献需要使用 \cmd{cite} 命令。
  
  \begin{codeblock}[]{ref.bib}
|\highlightline|@phdthesis{devoftech,|\hfill\alert{\% 类型为博士论文,主键为\texttt{devoftech}}|
  title={||新时期我国信息技术产业的发展},
  author={||江泽民},
  year={2008}
}
  \end{codeblock}
\end{frame}

\begin{frame}
  \frametitle{文献引用}
  而让 \LaTeX{} 处理 \texttt{.bib} 数据库文件与引用有两种工作流。你可能需要查清楚如何在编辑器中设置对应的工作流,或者采用后面所提到的高级编译方式 \texttt{latexmk}。
  \begin{columns}
    \begin{column}{0.5\textwidth}
      \begin{block}{\hologo{BibTeX} + \pkg{gbt7714}}
        一般期刊提交使用这种方法,\pkg{natbib} 宏包将提供命令 \cmd{citet}(文本引用) 和 \cmd{citep}(括号引用)。中文引用可以直接使用 \pkg{gbt7714} 宏包,但是角模式和正文模式不能混用。
      \end{block}
    \end{column}
    \begin{column}{0.5\textwidth}
      \begin{block}{\hologo{biber} + \pkg{biblatex}}
        这是更容易自定义的方法,与 \hologo{BibTeX} 的运作方式稍有不同。\pkg{biblatex} 提供了更加智能的引用命令。而中文引用可以使用 \pkg{biblatex} 宏包的样式 \pkg{gb7714-2015},使用该样式需要使用 \hologo{XeLaTeX} 编译。
      \end{block}
    \end{column}
  \end{columns}
\end{frame}

\begin{frame}[fragile]
  \frametitle{文献引用}
  \begin{columns}
    \begin{column}{0.5\textwidth}
      \begin{codeblock}[]{\hologo{BibTeX} + \pkg{gbt7714}}
\documentclass{ctexart}
\usepackage{gbt7714}
\bibliographystyle{gbt7714-numerial}
% \citestyle{numbers}  % 正文模式
\begin{document}
  ||他指出了...\cite{devoftech}
  \bibliography{ref}
\end{document}
      \end{codeblock}
    \end{column}
    \begin{column}{0.5\textwidth}
      \begin{codeblock}[]{\hologo{biber} + \pkg{biblatex}}
\documentclass{ctexart}
\usepackage[backend=biber,style=gb7714-2015]{biblatex}
\addbibresource{ref.bib}
\begin{document}
  ||他在文献 \parencite{devoftech}
  ||指出了...\cite{devoftech}
  \printbibliography
\end{document}
      \end{codeblock}
    \end{column}
  \end{columns}
\end{frame}

\begin{frame}
  \frametitle{文献引用}
  \begin{columns}
    \begin{column}{0.5\textwidth}
      \includepdflarge{bibtex}
    \end{column}
    \begin{column}{0.5\textwidth}
      \includepdflarge{biblatex}
    \end{column}
  \end{columns}
\end{frame}

} % End of customized shaded number logo

  % !TeX root = ..\..\latex-talk.tex

\part{SJTUThesis}

\begin{frame}
  \frametitle{简介}
  \begin{columns}
    \begin{column}{0.6\textwidth}
      \begin{itemize}
        \item 最早由韦建文于 2009 年 11 月发布 0.1a 版,2018 年起由 SJTUG 接手维护
        \item 最新版:\SJTUThesisVersion{} (\SJTUThesisDate)
        \item 支持本科、硕士、博士学位论文以及课程论文的排版
      \end{itemize}
    \end{column}
    \begin{column}{0.4\textwidth}
      \begin{exampleblock}{}
        \begin{minipage}[c]{1cm}
          \includegraphics[width=0.8cm]{\getcontribpath{sjtug}{vi/sjtug}}
        \end{minipage}
        \begin{minipage}[c]{2cm}
          \href{https://github.com/sjtug}{sjtug}/\href{https://github.com/sjtug/SJTUThesis}{SJTUThesis}
        \end{minipage}
      \end{exampleblock}
      \vspace{-8pt}
      \begin{block}{}
        \scriptsize
        上海交通大学 \hologo{XeLaTeX} 学位论文及课程论文模板 | Shanghai Jiao Tong University \hologo{XeLaTeX} Thesis Template
      \end{block}
      \vspace{-8pt}
      \begin{alertblock}{}
        \scriptsize
        \begin{tabular}{cl}
          \faStar & 2.4k \\
          \faEye & 55 \\
          \faCodeBranch & 701 \\
        \end{tabular}
      \end{alertblock}
    \end{column}
  \end{columns}
\end{frame}

\begin{frame}
  \frametitle{下载与编译}
  \alert{下载} 推荐安装 Git \link{https://git-scm.com/} 后,克隆 SJTUG 镜像仓库
  \begin{exampleblock}{\faGit*}
    \ttfamily\small
    git clone https://mirror.sjtu.edu.cn/git/SJTUThesis.git/
  \end{exampleblock}

  \alert{编译} 推荐使用 \pkg{latexmk} 编译\footnote{\hologo{MiKTeX} 用户需要手动安装 Perl 解释器 \link{https://www.perl.org/get.html} 才能使用 \pkg{latexmk}。},在不能够利用自带的 \texttt{.latexmkrc} 配置文件的情况下,需要查清楚在对应的编辑器中如何使用 \hologo{XeLaTeX} + \hologo{biber} 编译 \link{https://github.com/sjtug/SJTUThesis/blob/master/README.md}。
  \begin{exampleblock}{\faTerminal}
    \ttfamily\small
    latexmk -xelatex main
  \end{exampleblock}

  Overleaf 用户可以下载压缩包后,上传并采用 \hologo{XeLaTeX} 编译方式。
\end{frame}

\begin{frame}
  \frametitle{手动编译}
  \alert{第一次编译失败} 如果没有办法通过通常方式编译成功,请尝试使用文件夹内附带 \faLinux{}\,\faApple{} \texttt{Makefile} 和 \faWindows{} \texttt{Compile.bat} 进行编译。

  \alert{统计字数} 编写过程中也可以使用对应的命令调用 \TeX{}count 来统计正文字数。
  \begin{columns}
    \begin{column}{0.38\textwidth}
      \begin{exampleblock}{\faLinux{}\,\faApple}
        \ttfamily
        make all\\
        make clean\\
        make cleanall\\
        make wordcount
      \end{exampleblock}
    \end{column}
    \begin{column}{0.38\textwidth}
      \begin{exampleblock}{\faWindows}
        \ttfamily
        ./Compile.bat thesis\\
        ./Compile.bat clean\\
        ./Compile.bat cleanall\\
        ./Compile.bat wordcount
      \end{exampleblock}
    \end{column}
    \begin{column}{0.24\textwidth}
      \begin{block}{\faInfo}
        \ttfamily
        编译论文\\
        清理中间文件\\
        $\hookrightarrow +$删除论文\\
        统计字数
      \end{block}
    \end{column}
  \end{columns}
\end{frame}

\begin{frame}[label=compile]
  \frametitle{编译问题排查}
  \begin{columns}
    \begin{column}{0.33\textwidth}
      \begin{alertblock}{无法使用 \texttt{latexmk}\thesisissue{578}}
        \hologo{MiKTeX} 需要安装 Perl 解释器。
      \end{alertblock}  
      \begin{alertblock}{C\TeX{} 套装无法编译\thesisissue{446}}
        使用最新 \TeX{} 发行版。
      \end{alertblock}
      \begin{alertblock}{\hologo{pdfLaTeX} 无法编译\thesisissue{444}}
        请使用 \texttt{latexmk},或更改编辑器设置以 \hologo{XeLaTeX} 编译。
      \end{alertblock}
    \end{column}
    \begin{column}{0.33\textwidth}
      \begin{alertblock}{缺少字体\thesisissue{564} \thesisdiscuss{598}}
        更换字体集,或者安装对应字体。
      \end{alertblock}
      \begin{alertblock}{缺少汉字\thesisissue{533} \thesisdiscuss{617}}
        去除使用 fandol 字体集的设定。或者是安装字体后,改用 \texttt{fontset=adobe} 或 \texttt{fontset=founder}。
      \end{alertblock}
    \end{column}
    \begin{column}{0.33\textwidth}
      \begin{block}{\faInfoCircle{} README}
        不同编辑器的设置请首先参阅 README \link{https://github.com/sjtug/SJTUThesis/blob/master/README.md} 文档。
      \end{block}
      \begin{block}{\faBookOpen{} Wiki}
        其他编译问题推荐查阅 Wiki \link{https://github.com/sjtug/SJTUThesis/wiki} 的使用说明部分。
      \end{block}
    \end{column}
  \end{columns}
\end{frame}

\begin{frame}[fragile, label=covers]
  \begin{codeblock}[firstnumber=3]{main.tex}
|\alert{\% 载入 SJTUThesis 模版}|
\documentclass[|\only<1>{\highlight{type}}\only<2>{type}|=|\only<1>{bachelor}\only<2>{\highlight{bachelor}}|]{sjtuthesis}
  \end{codeblock}
  \begin{figure}
    \parbox{0.9\textwidth}{
      \begin{subfigure}{0.20\textwidth}
        \framebox{\includegraphics[width=\linewidth]{support/thesis/bachelor}}
        \caption{\only<1>{学士}\only<2>{\texttt{bachelor}}}
      \end{subfigure}\hfill
      \begin{subfigure}{0.20\textwidth}
        \framebox{\includegraphics[width=\linewidth]{support/thesis/master}}
        \caption{\only<1>{硕士}\only<2>{\texttt{master}}}
      \end{subfigure}\hfill
      \begin{subfigure}{0.20\textwidth}
        \framebox{\includegraphics[width=\linewidth]{support/thesis/doctor}}
        \caption{\only<1>{博士}\only<2>{\texttt{doctor}}}
      \end{subfigure}\hfill
      \begin{subfigure}{0.20\textwidth}
        \framebox{\includegraphics[width=\linewidth]{support/thesis/course}}
        \caption{\only<1>{课程}\only<2>{\texttt{course}}}
      \end{subfigure}
      \caption{论文类型示例\only<2>{ \texttt{type}}}
    }
  \end{figure}
\end{frame}

\begin{frame}[fragile]
  \frametitle{文档类选项}
  % \framesubtitle{\textbackslash{}documentclass\{sjtuthesis\}}
  \begin{columns}
    \begin{column}{0.45\textwidth}
      \includegraphics[page=10]{thesisdir}
    \end{column}
    \begin{column}{0.55\textwidth}
      \begin{table}[H]
        \caption{文档类选项}
        \footnotesize
        \begin{tabular}{>{\ttfamily}rll}
          \toprule
          选项 & 含义 & 相关 \\
          \midrule
          type= & 指定论文类型 & 第 \ref{covers} 页\\
          fontset= & 指定字体 & 第 \ref{compile} 页\\
          \midrule
          review & 开启盲审模式 & \thesisissue{195} \thesisissue{686} \\
          twoside & 双页模式 & \thesisissue{554} \\
          oneside & 单页模式 & \thesisissue{694} \\
          openright & 章从奇数页开始 & \thesisdiscuss{724} \\
          openany & 章从任意页开始 & \thesisissue{446} \\
          \bottomrule
        \end{tabular}
      \end{table}
    \end{column}
  \end{columns}
\end{frame}

\begin{frame}[fragile]
  \frametitle{基本配置}
  \framesubtitle{\textbackslash{}input\{setup\}}
  \begin{columns}
    \begin{column}{0.45\textwidth}
      \includegraphics[page=9]{thesisdir}
    \end{column}
    \begin{column}{0.55\textwidth}
      \begin{codeblock}[firstnumber=12]{main.tex}
|\highlightline<1>|% 论文基本配置,加载宏包等全局配置
|\highlightline<1>|\input{setup}

\begin{document}

%TC:ignore

|\highlightline<2>|% 标题页
|\highlightline<2>|\maketitle
      \end{codeblock}
      \visible<2>{
        \cmd{sjtusetup} 中的 \pkg{info} 将会修改封面的信息设置(见第 \ref{covers} 页)。
      }
    \end{column}
  \end{columns}
\end{frame}

\begin{frame}[fragile]
  \frametitle{基本配置}
  \framesubtitle{\textbackslash{}sjtusetup}
  \begin{columns}
    \begin{column}{0.45\textwidth}
      \includegraphics[page=1]{thesisdir}
    \end{column}
    \begin{column}{0.55\textwidth}
      \begin{codeblock}[firstnumber=3]{setup.tex}
\sjtusetup{
  info = {
    title    = {||上海交通大学学位论文 \LaTeX{} 模板示例文档},
    title*   = {A Sample for \LaTeX-based SJTU Thesis Template},
    author   = {||某\quad{}某},
    author* = {Mo Mo},
  },
  style = { header-logo-color = red, 
  },
  name = {
    publications = {||攻读学位期间完成的论文},
  },
}
      \end{codeblock}
    \end{column}
  \end{columns}
\end{frame}

\begin{frame}
  \frametitle{基本配置}
  \framesubtitle{\textbackslash{}sjtusetup}
  \begin{columns}
    \begin{column}{0.45\textwidth}
      \includegraphics[page=1]{thesisdir}
    \end{column}
    \begin{column}{0.55\textwidth}
      \begin{table}[H]
        \centering
        \caption{info 域}
        \footnotesize
        \begin{tabular}{lll} \toprule
          命令作用 & 中文对应选项 & 英文对应选项 \\ \midrule
          论文标题 & \texttt{title} & \texttt{title*} \\
          关键字列表 & \texttt{keywords} & \texttt{keywords*} \\
          作者姓名&  \texttt{author} &\texttt{author*}\\
          申请学位名称 & \texttt{degree}&\texttt{degree*}\\
          院系名称 & \texttt{department} & \texttt{department*}\\
          专业名称 & \texttt{major} & \texttt{major*}\\
          导师 & \texttt{supervisor} & \texttt{supervisor*}\\
          副导师 & \texttt{assisupervisor} & \texttt{assisupervisor*}\\
          日期 & \multicolumn{2}{c}{\texttt{date}}\\
          学号 & \multicolumn{2}{c}{\texttt{id}}\\ \bottomrule
          \end{tabular}
      \end{table}
    \end{column}
  \end{columns}
\end{frame}

\begin{frame}[fragile]
  \frametitle{版权页}
  \framesubtitle{\textbackslash{}copyrightpage}
  \begin{columns}
    \begin{column}{0.45\textwidth}
      \only<1>{
        \includegraphics[page=9]{thesisdir}
      }
      \only<2>{
        \includegraphics[page=2]{thesisdir}
      }
      \only<3>{
        \begin{figure}[H]
          \framebox{\includegraphics[page=2,width=0.4\linewidth]{bachelor}}
          \caption{版权页}
        \end{figure}
      }
    \end{column}
    \begin{column}{0.55\textwidth}
      \begin{codeblock}[firstnumber=22]{main.tex}
|\highlightline<1>|% 原创性声明及使用授权书
|\highlightline<1>|\copyrightpage
|\highlightline<2>|% 插入外置原创性声明及使用授权书
|\highlightline<2>|% \copyrightpage[scans/sample-copyright-old.pdf]
      \end{codeblock}
      \only<1>{
        \cmd{copyrightpages} 可以用于插入版权页。
      }
      \only<2>{
        \cmd{copyrightpages} 也接受一个可选参数,用于直接使用扫描件。
      }
    \end{column}
  \end{columns}
\end{frame}

\begin{frame}[fragile]
  \frametitle{前置部分}
  \framesubtitle{\textbackslash{}frontmatter}
  \begin{columns}
    \begin{column}{0.45\textwidth}
      \only<1>{
        \includegraphics[page=9]{thesisdir}
      }
      \only<2>{
        \includegraphics[page=3]{thesisdir}
      }
      \only<3>{
        \begin{figure}[H]
          \begin{subfigure}{0.45\textwidth}
            \framebox{\includegraphics[page=3,width=\linewidth]{bachelor}}
            \caption{中文}
          \end{subfigure}\hfill
          \begin{subfigure}{0.45\textwidth}
            \framebox{\includegraphics[page=4,width=\linewidth]{bachelor}}
            \caption{英文}
          \end{subfigure}
          \caption{摘要}
        \end{figure}
      }
      \only<4>{
        \begin{figure}[H]
          \begin{subfigure}{0.30\linewidth}
            \centering
            \framebox{\includegraphics[page=5,width=0.6\linewidth]{bachelor}}
            \caption{目录}
          \end{subfigure}
          \begin{subfigure}{0.30\linewidth}
            \centering
            \framebox{\includegraphics[page=6,width=0.6\linewidth]{bachelor}}
            \caption{插图}
          \end{subfigure}

          \begin{subfigure}{0.30\linewidth}
            \centering
            \framebox{\includegraphics[page=7,width=0.6\linewidth]{bachelor}}
            \caption{表格}
          \end{subfigure}
          \begin{subfigure}{0.30\linewidth}
            \centering
            \framebox{\includegraphics[page=8,width=0.6\linewidth]{bachelor}}
            \caption{算法}
          \end{subfigure}
          \caption{索引}
        \end{figure}
      }
      \only<5>{
        \includegraphics[page=4]{thesisdir}
      }
      \only<6>{
        \begin{figure}[H]
          \framebox{\includegraphics[page=9,width=0.5\linewidth]{bachelor}}
          \caption{符号对照表}
        \end{figure}
      }
    \end{column}
    \begin{column}{0.55\textwidth}
      \begin{codeblock}[firstnumber=30]{main.tex}
|\highlightline<2-3>|% 摘要
|\highlightline<2-3>|\input{contents/abstract}

|\highlightline<4>|% 目录
|\highlightline<4>|\tableofcontents
|\highlightline<4>|% 插图索引
|\highlightline<4>|\listoffigures*
|\highlightline<4>|% 表格索引
|\highlightline<4>|\listoftables*
|\highlightline<4>|% 算法索引
|\highlightline<4>|\listofalgorithms*

|\highlightline<5-6>|% 符号对照表
|\highlightline<5-6>|\input{contents/nomenclature}
      \end{codeblock}
    \end{column}
  \end{columns}
\end{frame}

\begin{frame}[fragile]
  \frametitle{主体部分}
  \framesubtitle{\textbackslash{}mainmatter}
  \begin{columns}
    \begin{column}{0.45\textwidth}
      \only<1>{
        \includegraphics[page=5]{thesisdir}
      }
      \only<2>{
        \begin{figure}[H]
          \begin{subfigure}{0.30\linewidth}
            \centering
            \framebox{\includegraphics[page=11,width=0.6\linewidth]{bachelor}}
            \caption{简介}
          \end{subfigure}
          \begin{subfigure}{0.30\linewidth}
            \centering
            \framebox{\includegraphics[page=13,width=0.6\linewidth]{bachelor}}
            \caption{数学}
          \end{subfigure}

          \begin{subfigure}{0.30\linewidth}
            \centering
            \framebox{\includegraphics[page=16,width=0.6\linewidth]{bachelor}}
            \caption{浮动体}
          \end{subfigure}
          \begin{subfigure}{0.30\linewidth}
            \centering
            \framebox{\includegraphics[page=22,width=0.6\linewidth]{bachelor}}
            \caption{总结}
          \end{subfigure}
          \caption{主体部分}
        \end{figure}
      }
    \end{column}
    \begin{column}{0.55\textwidth}
      \begin{codeblock}[firstnumber=47]{main.tex}
|\highlightline|% 正文内容
|\highlightline|\input{contents/intro}
|\highlightline|\input{contents/math_and_citations}
|\highlightline|\input{contents/floats}
|\highlightline|\input{contents/summary}

%TC:ignore

% 参考文献
\printbibliography[heading=bibintoc]
      \end{codeblock}
    \end{column}
  \end{columns}
\end{frame}

\begin{frame}
  \frametitle{数学}
  \begin{itemize}
    \item 公式示例:\nolinkurl{contents/math_and_citations.tex}
    \item \SJTUThesis{} 定义了常用的数学环境(需要手工引入 \texttt{ntheorem} 宏包):
      \begin{table}[h]
        \centering
        \footnotesize
        \begin{tabular}{*{7}{l}}\toprule
          assumption  & axiom   & conjecture & corollary    & definition  & example   & exercise  \\
          假设        & 公理    & 猜想       & 推论         & 定义        & 例        & 练习      \\\midrule
          lemma       & problem & proof      & proposition  & remark      & solution  & theorem   \\
          引理        & 问题    & 证明       & 命题         & 注          & 解        & 定理      \\\bottomrule
        \end{tabular}
      \end{table}
      \item \SJTUThesis{} 可以通过 \texttt{unimath} 选项使用 \pkg{unicode-math} 进行数学输入,注意与传统方式的区别。\thesisissue{555}
  \end{itemize}
\end{frame}

\begin{frame}[fragile]
  \frametitle{参考文献}
  \begin{columns}
    \begin{column}{0.45\textwidth}
      \includegraphics[page=6]{thesisdir}
    \end{column}
    \begin{column}{0.55\textwidth}
      \begin{codeblock}[firstnumber=111,numbersep=2pt]{setup.tex}
% 使用 BibLaTeX 处理参考文献
%   biblatex-gb7714-2015 常用选项
%     gbnamefmt=lowercase     姓名大小写由输入信息确定
%     gbpub=false             禁用出版信息缺失处理
\usepackage[backend=biber,style=gb7714-2015]{biblatex}
% 文献表字体
% \renewcommand{\bibfont}{\zihao{-5}}
% 文献表条目间的间距
\setlength{\bibitemsep}{0pt}
|\highlightline|% 导入参考文献数据库
|\highlightline|\addbibresource{bibdata/thesis.bib}
      \end{codeblock}
    \end{column}
  \end{columns}
\end{frame}

\begin{frame}[fragile]
  \frametitle{附录}
  \framesubtitle{\textbackslash{}appendix}
  \begin{columns}
    \begin{column}{0.45\textwidth}
      \only<1>{
        \includegraphics[page=7]{thesisdir}
      }
      \only<2>{
        \begin{figure}[H]
          \begin{subfigure}{0.45\linewidth}
            \framebox{\includegraphics[width=\linewidth,page=24]{bachelor}}
            \caption{}
          \end{subfigure}\hfill
          \begin{subfigure}{0.45\textwidth}
            \framebox{\includegraphics[width=\linewidth,page=25]{bachelor}}
            \caption{}
          \end{subfigure}
          \caption{附录}
        \end{figure}
      }
    \end{column}
    \begin{column}{0.55\textwidth}
      \begin{codeblock}[firstnumber=61]{main.tex}
% 附录中图表不加入索引
\captionsetup{list=no}

% 附录内容
|\highlightline|\input{contents/app_maxwell_equations}
|\highlightline|\input{contents/app_flow_chart}
      \end{codeblock}
    \end{column}
  \end{columns}
\end{frame}

\begin{frame}[fragile]
  \frametitle{结尾部分}
  \framesubtitle{\textbackslash{}backmatter}
  \begin{columns}
    \begin{column}{0.45\textwidth}
      \only<1>{
        \includegraphics[page=8]{thesisdir}
      }
      \only<2>{
        \begin{figure}[H]
          \begin{subfigure}{0.30\linewidth}
            \centering
            \framebox{\includegraphics[page=26,width=0.6\linewidth]{bachelor}}
            \caption{致谢}
          \end{subfigure}
          \begin{subfigure}{0.30\linewidth}
            \centering
            \framebox{\includegraphics[page=27,width=0.6\linewidth]{bachelor}}
            \caption{成就}
          \end{subfigure}

          \begin{subfigure}{0.30\linewidth}
            \centering
            \framebox{\includegraphics[page=28,width=0.6\linewidth]{bachelor}}
            \caption{简历}
          \end{subfigure}
          \begin{subfigure}{0.30\linewidth}
            \centering
            \framebox{\includegraphics[page=29,width=0.6\linewidth]{bachelor}}
            \caption{大摘要*}
          \end{subfigure}
          \caption{结尾部分}
        \end{figure}
      }
    \end{column}
    \begin{column}{0.55\textwidth}
      \begin{codeblock}[firstnumber=76]{main.tex}
% 致谢
\input{contents/acknowledgements}

% 发表论文及科研成果
% 盲审论文中,发表论文及科研成果等仅以第几作者注明即可,不要出现作者或他人姓名
\input{contents/achievements}

% 简历
\input{contents/resume}

% 学士学位论文要求在最后有一个大摘要,单独编页码
\input{contents/digest}
      \end{codeblock}
    \end{column}
  \end{columns}
\end{frame}

\begin{frame}
  \frametitle{还有其他问题?}
  \begin{columns}
    \begin{column}{0.75\textwidth}
    \begin{itemize}
      \item[{\faComment*[regular]}] 日常模板或 \LaTeX{} 使用问题可以前往 Discussions \link{https://github.com/sjtug/SJTUThesis/discussions} 提问
      
      (解决后别忘了 \textcolor{green}{\faCheckCircle{} Mark as answer}
      \item[{\faDotCircle[regular]}] 如果是 \textsc{SJTUThesis} 项目本身的 bug 和 feature request
      
      可以通过 Issues \link{https://github.com/sjtug/SJTUThesis/issues} 反馈。
      \item[{\faCodeBranch}] 如果你有好点子,可以贡献代码
     
      向 \textsc{SJTU\TeX{}}(v1) \link{https://github.com/sjtug/SJTUTeX/tree/v1} 存储库发 PR,\par
      而后把解包结果同步到 \textsc{SJTUThesis}。
  
      \item[{\faTag}] 如果你对正在基于 \LaTeX3 开发的新版感兴趣,\par
      也欢迎向 \textsc{SJTU\TeX{}}(v2) \link{https://github.com/sjtug/SJTUTeX/tree/v2} 发 PR。
  
      \item[{\faQq}] 也欢迎在 QQ 群即时讨论。
    \end{itemize}
    \end{column}
    \begin{column}{0.25\textwidth}
      \includegraphics[height=0.7\textheight]{qq.jpg}
    \end{column}
  \end{columns}
\end{frame}
\end{document}
      \end{codeblock}
    \end{column}
  \end{columns}
  \footnotetext{如果想强制指定子文档的主文档,可以在文件第一行输入魔术命令:\texttt{\% !TeX root = main.tex}}
\end{frame}

\section{图}
\begin{frame}[fragile]%
  \frametitle{\temporal<5>{插图}{浮动体}{插图}}
  \begin{columns}
    \begin{column}{0.6\textwidth}
      \begin{codeblock}[]{插入单图\only<4->{最佳实践}}
\documentclass{ctexart}
|\only<2>{\highlightline}|\usepackage{graphicx}
|\only<2>{\highlightline}|\graphicspath{{figs/}{pics/}}
\begin{document}
|\only<5>{\highlightline}|\begin{figure}[ht]
|\only<6>{\highlightline}|  \centering
|\only<3>{\highlightline}|  \includegraphics[width=|\only<1-3>{4cm}\only<4->{0.4\textbackslash{}textwidth}|]{sjtug}
|\only<7>{\highlightline}|  \caption{SJTUG 徽标}\label{fig:sjtug}
|\only<5>{\highlightline}|\end{figure}
\end{document}
      \end{codeblock}
    \end{column}
    \begin{column}{0.4\textwidth}
      \only<1>{
        \includepdflarge{insertimage}
      }
      \only<2>{
        为了插入外部图片,需要使用 \pkg{graphicx} 宏包。之后在文档主体便可以使用 \cmd{includegraphics} 插入图片。导言区也可以加入 \cmd{graphicspath} 指定图片文件夹\footnotemark。
      }
      \only<3>{
        \cmd{includegraphics} 命令便以相对路径的方式插入图片,如果无同名图片,那么后缀名可以省略。可以使用可选参数指定插入的图片尺寸,最佳实践是使用 \cmd{textwidth} 或 \cmd{linewidth} 的相对值指定尺寸大小,以在未来可能的布局更改中保留一定的灵活性。
      }
      \only<4>{
        也可以通过键值对的方法设置图片的其他属性。
        \begin{center}
          \footnotesize
          \begin{stampbox}
            \begin{tabular}{rl}
              \pkg{width} & 宽度 \\
              \pkg{height} & 高度 \\
              \pkg{scale} & 缩放 \\
              \pkg{angle} & 角度 \\
            \end{tabular}
          \end{stampbox}
        \end{center}
      }
      \only<5>{
        \env{figure} 为一个浮动体环境(\env{table} 也是),可以将其移动到其他位置上以减少行文中的空白。可以添加可选参数以指定如何放置浮动体,最多可以使用四种位置描述符:
        \begin{center}
          \footnotesize
          \begin{stampbox}
            \begin{tabular}{cll}
              \pkg{h} & Here & 尽可能在这里 \\
              \pkg{t} & Top & 页面顶部 \\
              \pkg{b} & Bottom & 页面底部 \\
              \pkg{p} & Page & 浮动体专页 \\
            \end{tabular}
          \end{stampbox}
        \end{center}
        还可以只使用 \pkg{float} 宏包提供的 \pkg{H} 描述符以强制置于此处。
      }
      \only<6>{
        采用 \cmd{centering} 命令而不是 \env{center} 环境来水平居中图片。这将避免多余的纵向间距。
      }
      \only<7>{
        使用 \cmd{caption} 命令输入题注,如果这个命令写在插入图片前面,题注将会在上方(中文中一般对 \env{table} 环境这么做)。后面将会看到如何对留有标记(\cmd{label})的图片进行引用。
      }
    \end{column}
  \end{columns}
  \only<2>{\footnotetext{其命令参数每个为一个以 \texttt{/} 结尾的文件夹,每个文件夹需要使用大括号包裹起来。}}
\end{frame}

\begin{frame}[fragile]
  \begin{columns}
    \begin{column}{0.6\textwidth}
      \begin{codeblock}[]{插入双图}
\documentclass{ctexart}
\usepackage{graphicx}
\graphicspath{{figs/}{pics/}}
\begin{document}
  \begin{figure}[ht]
|\only<1>{\highlightline}|    \begin{minipage}{0.48\textwidth}
      \centering
      \includegraphics[height=2cm]{sjtug}
|\only<2>{\highlightline}|      \caption{SJTUG 徽标}\label{fig:sjtug}
|\only<1>{\highlightline}|    \end{minipage}\hfill
|\only<1>{\highlightline}|    \begin{minipage}{0.48\textwidth}
      \centering
      \includegraphics[height=2cm]{sjtugt}
|\only<2>{\highlightline}|      \caption{SJTUG||文字}\label{fig:sjtugt}
|\only<1>{\highlightline}|    \end{minipage}
  \end{figure}
\end{document}
      \end{codeblock}
    \end{column}
    \begin{column}{0.4\textwidth}
      \only<1>{
        在 \env{figure} 环境里使用 \env{minipage} 小页制作列盒子,以并排两图并分别编号,需要设定强制参数以指定列宽。两个小页之间添加 \cmd{hfill} 使两个小页两端对齐。
      }
      \only<2>{
        在每个小页内部分别使用 \cmd{caption} 添加标签。
      }
      \only<3>{
        \includepdflarge{doubleimages}
      }
    \end{column}
  \end{columns}
\end{frame}

\begin{frame}[fragile]%
  \begin{columns}
    \begin{column}{0.6\textwidth}
      \begin{codeblock}[]{}
\documentclass{ctexart}
\usepackage{graphicx}
|\highlightline|\usepackage{subcaption}
\graphicspath{{figs/}{pics/}}
\begin{document}
  \begin{figure}[ht]
|\highlightline|    \begin{subfigure}{0.48\textwidth}
      \centering
      \includegraphics[height=2cm]{sjtug}
      \caption{||徽标}
|\highlightline|    \end{subfigure}\hfill
|\highlightline|    \begin{subfigure}{0.48\textwidth}
      \centering
      \includegraphics[height=2cm]{sjtugt}
      \caption{||文字}
|\highlightline|    \end{subfigure}
    \caption{SJTUG}\label{fig:sjtug}
  \end{figure}
\end{document}
      \end{codeblock}
    \end{column}
    \begin{column}{0.4\textwidth}
      \includepdflarge{subfigures}\vspace{15pt}
      \pkg{subcaption} 宏包提供了 \env{subfigure} 环境(以及 \env{subtable}),可以用于以子图的形式插入,编号会增加一级。也可以为子图添加子集引用编号。
    \end{column}
  \end{columns}
\end{frame}

\section{表}
\begin{frame}[fragile]
  \frametitle{简单表格}
  \begin{columns}
    \begin{column}{0.6\textwidth}
      \begin{codeblock}[]{}
\documentclass{ctexart}
|\only<1-2>{\highlightline}|\usepackage{|\temporal<1>{array}{\highlight{array}}{array},\temporal<2>{booktabs}{\highlight{booktabs}}{booktabs}|}
\begin{document}
\begin{table}[ht]
  \centering
  \caption{||北京冬奥会吉祥物}
|\only<1>{\highlightline}|  \begin{tabular}{lp{3cm}}
|\only<2>{\highlightline}|    \toprule
|\only<3>{\highlightline}|吉祥物 & 描述                          \\
|\only<2>{\highlightline}|    \midrule
|\only<3>{\highlightline}|冰墩墩 & 2022 年北京冬季奥运会吉祥物  \\
|\only<3>{\highlightline}|雪容融 & 2022 年北京冬季残奥会吉祥物  \\
|\only<2>{\highlightline}|    \bottomrule
|\only<1>{\highlightline}|  \end{tabular}
\end{table}
\end{document}
      \end{codeblock}
    \end{column}
    \begin{column}{0.4\textwidth}
      \only<1>{
        使用 \env{tabular} 环境绘制表格。需要添加参数(称为\textbf{表格导言})以确定每一列的对齐方式。引入 \pkg{array} 宏包来使用更多的\textbf{引导符}。
        \begin{center}
          \footnotesize
          \begin{stampbox}
            \begin{tabular}{>{\ttfamily}ll}
              \alert{l} & 向左对齐 \\
              \alert{c} & 居中对齐 \\
              \alert{r} & 向右对齐 \\
              \alert{p\{3cm\}} & 固定列宽,两端对齐 \\
              \alert{m\{3cm\}} & \texttt{p} + 垂直居中对齐 \\
              \alert{>\{\textbackslash{}bfseries\}} & 后一列单元格前加命令 \\
              \alert{*\{3\}\{l\}} & 三个左对齐列 \\
            \end{tabular}
          \end{stampbox}
        \end{center}
      }
      \only<2>{
        \pkg{booktabs} 宏包提供了标准三线表格所需要的行分割线:\cmd{toprule},\cmd{midrule},\cmd{bottomrule}。也可以使用 \cmd{cmidrule\{1-2\}} 来部分地绘制行分割线。一般不推荐在专业表格中使用纵向分割线。
      }
      \only<3>{
        每行内容使用 \textbackslash\textbackslash{} 分隔开,每行中的单元格使用 \& 分隔开。
      }
      \only<4>{
        \includepdflarge{table}
      }
    \end{column}
  \end{columns}
\end{frame}

\begin{frame}[fragile]%
  \begin{columns}
    \begin{column}{0.6\textwidth}
      \begin{codeblock}[]{表头居中}
\documentclass{ctexart}
\usepackage{array,booktabs}
\begin{document}
\begin{table}[ht]
  \centering
  \caption{||北京冬奥会吉祥物}
  \begin{tabular}{lp{3cm}}
    \toprule
|\highlightline|\multicolumn{1}{c}{||吉祥物} &
|\highlightline|\multicolumn{1}{c}{||描述} \\
    \midrule
||冰墩墩 & 2022 年北京冬季奥运会吉祥物  \\
||雪容融 & 2022 年北京冬季残奥会吉祥物  \\
    \bottomrule
  \end{tabular}
\end{table}
\end{document}
      \end{codeblock}
    \end{column}
    \begin{column}{0.4\textwidth}
      \cmd{multicolumn} 命令不仅可以用于合并同行的单元格,还可以用于临时地屏蔽表格导言对该列的对齐方式定义。这里用于居中表头。
      \begin{center}
        \begin{stampbox}
          \parbox{0.85\linewidth}{
            \ttfamily\color{blue}\textbackslash{}multicolumn\{格数\}\{对齐方式\}\{内容\}
          }
        \end{stampbox}
      \end{center}
      跨页表格考虑使用 \pkg{longtable} 宏包。带标注的表格可以考虑使用 \pkg{threeparttable} 宏包。考虑使用在线工具生成表格代码 \link{https://www.tablesgenerator.com/latex_tables}。
    \end{column}
  \end{columns}
\end{frame}

\section{数学公式}
\begin{frame}
  \frametitle{数学模式}
  \begin{alertblock}{}
  输入数学公式是 \LaTeX{} 的绝对强项,很多常见的公式服务依赖于一些轻量级渲染引擎比如 MathJax, K\kern-.3ex\raise.4ex\hbox{\footnotesize A}\kern-.3ex\TeX{}。但是它们实际上使用的是 \LaTeX{} 语法变种,也就是说并没有使用 \LaTeX{} 后端。所以不要期望有完全一致的输出。
  \end{alertblock}
  
  为了更好的获得数学公式输入支持,请使用 \hologo{AmS}math 宏包。数学模式分为两种:
  \begin{description}
    \item[行内模式] 一般通过一对美元符号(\$$\cdots$\$)标记,可以使用编辑器内建的符号表输入数学符号,也可以使用在线工具手写识别 \link{https://detexify.kirelabs.org/classify.html}。
    \item[行间模式] 一般通过 \env{equation} 环境\footnote{这是有编号环境,其加星号的变种 \env{equation*} 用于生成无编号环境。}输入。如果需要使用多行公式,请使用 \env{align} 环境,并按照类似表格输入的方式,使用 \& 对齐符号,\textbackslash\textbackslash{} 换行。如果不想手动居中,可以考虑多行自动居中的 \env{gather} 和单个大型公式首尾两端对齐 \env{multline}。
  \end{description}
\end{frame}

\begin{frame}
  \frametitle{数学命令展示}
  \begin{columns}
    \begin{column}{0.33\textwidth}
      \begin{exampleblock}{}
        $PV=nRT$
      \end{exampleblock}
      \begin{exampleblock}{}
        $\sum_{i=1}^ki^2=\frac{n(n+1)(2n+1)}{6}$
      \end{exampleblock}
      \begin{exampleblock}{}
        $T(n) = aT\left(\left\lceil\frac{n}{b}\right\rceil\right) + \mathcal{O}(n^d)$
      \end{exampleblock}
      \begin{exampleblock}{}
        $\frac{x_{1}+x_{2}+x_{3}}{3}\geq \sqrt[3]{x_{1}x_{2}x_{3}}$
      \end{exampleblock}
      \begin{exampleblock}{}
        $n=(\underbrace{1\cdots 1}_{k\text{ of 1's}})_2=2^{k+1}-1$
      \end{exampleblock}
      \begin{exampleblock}{}
        $\nabla f (P)= \left.\left(\frac{\partial f}{\partial x},\frac{\partial f}{\partial y},\frac{\partial f}{\partial z}\right)\right|_{P}$
      \end{exampleblock}
    \end{column}
    \begin{column}{0.67\textwidth}
      \begin{exampleblock}{}
        \begin{equation*}
          \int_{a}^b f(x)\,\mathrm{d}x=\lim_{|P|\rightarrow 0}\sum_{i=1}^n f(\xi_i)\Delta x_i
        \end{equation*}
      \end{exampleblock}
      \begin{exampleblock}{}
        \begin{equation}
          T(n) = \begin{cases}
            \mathcal{O}(n^d),&\textrm{if } d>\log_b a, \\
            \mathcal{O}(n^d\log n), &\textrm{if } d=\log_b a,\\
            \mathcal{O}(n^{\log_b a}), &\textrm{if } d<\log_b a.
          \end{cases}
        \end{equation}
      \end{exampleblock}
      \begin{exampleblock}{}
        \begin{align}
          Q^{T}A&=R \\
          \begin{pmatrix}
            q_1^T \\ q_2^T \\ q_3^T
          \end{pmatrix}
          \begin{pmatrix}
            a_1 & a_2 & a_3
          \end{pmatrix}
          &=R
        \end{align}
      \end{exampleblock}
    \end{column}
  \end{columns}
\end{frame}

%更深入地讲解 mathtools, unicode-math, siunix

\section{引用}
\begin{frame}[fragile]
  \frametitle{交叉引用}
  \only<1>{
    正如之前所提到的,\LaTeX{} 中使用 \cmd{label} 标记,然后可以使用 \cmd{ref} 来引用这个标记。 \cmd{label} 可以放在使用计数器的对象之后。
  }
  \only<2>{
    为了使得对公式编号的引用带有括号,推荐使用 \hologo{AmS}math 宏包中的 \cmd{eqref} 命令。对于多行公式环境,每一个换行符前都可以添加一个 \cmd{label} 用于引用该行公式。
  }
  \begin{columns}
    \begin{column}{0.5\textwidth}
      \begin{codeblock}[]{图}
\begin{figure}
|\only<1>{\highlightline}|  \caption{||示例}\label{fig:example}
\end{figure}
      \end{codeblock}
      \begin{codeblock}[]{表}
\begin{table}
|\only<1>{\highlightline}|  \caption{||示例}\label{tab:example}
\end{table}
      \end{codeblock}
    \end{column}
    \begin{column}{0.5\textwidth}
\begin{codeblock}[]{目次}
|\only<1>{\highlightline}|\section{||示例}\label{sec:example}
\end{codeblock}

\begin{codeblock}[]{公式}
\begin{equation}
  a = b + c
|\only<1>{\highlightline}|\label{eq:example}
\end{equation}
|\only<2>{\highlightline}|如公式 \eqref{eq:example} 所示,
\end{codeblock}
    \end{column}
  \end{columns}
\end{frame}

\begin{frame}[fragile]
  \frametitle{文献引用}
  \LaTeX{} 管理参考文献可以采用专用数据库文件 \texttt{.bib},很多的文献管理文件比如 EndNote \link{https://lic.sjtu.edu.cn/Default/softshow/tag/MDAwMDAwMDAwMLGImKE}, Zotero \link{https://www.zotero.org/}, JabRef \link{https://www.jabref.org/} 都可以直接导出这种格式的文件用于 \LaTeX{} 论文中的引用。一般不需要手写数据库文件,你只需要注意每一个文献会在数据库中有一个主键,这个类似于上文的 \cmd{label} 标记,只是要引用数据库中的文献需要使用 \cmd{cite} 命令。
  
  \begin{codeblock}[]{ref.bib}
|\highlightline|@phdthesis{devoftech,|\hfill\alert{\% 类型为博士论文,主键为\texttt{devoftech}}|
  title={||新时期我国信息技术产业的发展},
  author={||江泽民},
  year={2008}
}
  \end{codeblock}
\end{frame}

\begin{frame}
  \frametitle{文献引用}
  而让 \LaTeX{} 处理 \texttt{.bib} 数据库文件与引用有两种工作流。你可能需要查清楚如何在编辑器中设置对应的工作流,或者采用后面所提到的高级编译方式 \texttt{latexmk}。
  \begin{columns}
    \begin{column}{0.5\textwidth}
      \begin{block}{\hologo{BibTeX} + \pkg{gbt7714}}
        一般期刊提交使用这种方法,\pkg{natbib} 宏包将提供命令 \cmd{citet}(文本引用) 和 \cmd{citep}(括号引用)。中文引用可以直接使用 \pkg{gbt7714} 宏包,但是角模式和正文模式不能混用。
      \end{block}
    \end{column}
    \begin{column}{0.5\textwidth}
      \begin{block}{\hologo{biber} + \pkg{biblatex}}
        这是更容易自定义的方法,与 \hologo{BibTeX} 的运作方式稍有不同。\pkg{biblatex} 提供了更加智能的引用命令。而中文引用可以使用 \pkg{biblatex} 宏包的样式 \pkg{gb7714-2015},使用该样式需要使用 \hologo{XeLaTeX} 编译。
      \end{block}
    \end{column}
  \end{columns}
\end{frame}

\begin{frame}[fragile]
  \frametitle{文献引用}
  \begin{columns}
    \begin{column}{0.5\textwidth}
      \begin{codeblock}[]{\hologo{BibTeX} + \pkg{gbt7714}}
\documentclass{ctexart}
\usepackage{gbt7714}
\bibliographystyle{gbt7714-numerial}
% \citestyle{numbers}  % 正文模式
\begin{document}
  ||他指出了...\cite{devoftech}
  \bibliography{ref}
\end{document}
      \end{codeblock}
    \end{column}
    \begin{column}{0.5\textwidth}
      \begin{codeblock}[]{\hologo{biber} + \pkg{biblatex}}
\documentclass{ctexart}
\usepackage[backend=biber,style=gb7714-2015]{biblatex}
\addbibresource{ref.bib}
\begin{document}
  ||他在文献 \parencite{devoftech}
  ||指出了...\cite{devoftech}
  \printbibliography
\end{document}
      \end{codeblock}
    \end{column}
  \end{columns}
\end{frame}

\begin{frame}
  \frametitle{文献引用}
  \begin{columns}
    \begin{column}{0.5\textwidth}
      \includepdflarge{bibtex}
    \end{column}
    \begin{column}{0.5\textwidth}
      \includepdflarge{biblatex}
    \end{column}
  \end{columns}
\end{frame}

} % End of customized shaded number logo

  % !TeX root = ..\..\latex-talk.tex

\part{SJTUThesis}

\begin{frame}
  \frametitle{简介}
  \begin{columns}
    \begin{column}{0.6\textwidth}
      \begin{itemize}
        \item 最早由韦建文于 2009 年 11 月发布 0.1a 版,2018 年起由 SJTUG 接手维护
        \item 最新版:\SJTUThesisVersion{} (\SJTUThesisDate)
        \item 支持本科、硕士、博士学位论文以及课程论文的排版
      \end{itemize}
    \end{column}
    \begin{column}{0.4\textwidth}
      \begin{exampleblock}{}
        \begin{minipage}[c]{1cm}
          \includegraphics[width=0.8cm]{\getcontribpath{sjtug}{vi/sjtug}}
        \end{minipage}
        \begin{minipage}[c]{2cm}
          \href{https://github.com/sjtug}{sjtug}/\href{https://github.com/sjtug/SJTUThesis}{SJTUThesis}
        \end{minipage}
      \end{exampleblock}
      \vspace{-8pt}
      \begin{block}{}
        \scriptsize
        上海交通大学 \hologo{XeLaTeX} 学位论文及课程论文模板 | Shanghai Jiao Tong University \hologo{XeLaTeX} Thesis Template
      \end{block}
      \vspace{-8pt}
      \begin{alertblock}{}
        \scriptsize
        \begin{tabular}{cl}
          \faStar & 2.4k \\
          \faEye & 55 \\
          \faCodeBranch & 701 \\
        \end{tabular}
      \end{alertblock}
    \end{column}
  \end{columns}
\end{frame}

\begin{frame}
  \frametitle{下载与编译}
  \alert{下载} 推荐安装 Git \link{https://git-scm.com/} 后,克隆 SJTUG 镜像仓库
  \begin{exampleblock}{\faGit*}
    \ttfamily\small
    git clone https://mirror.sjtu.edu.cn/git/SJTUThesis.git/
  \end{exampleblock}

  \alert{编译} 推荐使用 \pkg{latexmk} 编译\footnote{\hologo{MiKTeX} 用户需要手动安装 Perl 解释器 \link{https://www.perl.org/get.html} 才能使用 \pkg{latexmk}。},在不能够利用自带的 \texttt{.latexmkrc} 配置文件的情况下,需要查清楚在对应的编辑器中如何使用 \hologo{XeLaTeX} + \hologo{biber} 编译 \link{https://github.com/sjtug/SJTUThesis/blob/master/README.md}。
  \begin{exampleblock}{\faTerminal}
    \ttfamily\small
    latexmk -xelatex main
  \end{exampleblock}

  Overleaf 用户可以下载压缩包后,上传并采用 \hologo{XeLaTeX} 编译方式。
\end{frame}

\begin{frame}
  \frametitle{手动编译}
  \alert{第一次编译失败} 如果没有办法通过通常方式编译成功,请尝试使用文件夹内附带 \faLinux{}\,\faApple{} \texttt{Makefile} 和 \faWindows{} \texttt{Compile.bat} 进行编译。

  \alert{统计字数} 编写过程中也可以使用对应的命令调用 \TeX{}count 来统计正文字数。
  \begin{columns}
    \begin{column}{0.38\textwidth}
      \begin{exampleblock}{\faLinux{}\,\faApple}
        \ttfamily
        make all\\
        make clean\\
        make cleanall\\
        make wordcount
      \end{exampleblock}
    \end{column}
    \begin{column}{0.38\textwidth}
      \begin{exampleblock}{\faWindows}
        \ttfamily
        ./Compile.bat thesis\\
        ./Compile.bat clean\\
        ./Compile.bat cleanall\\
        ./Compile.bat wordcount
      \end{exampleblock}
    \end{column}
    \begin{column}{0.24\textwidth}
      \begin{block}{\faInfo}
        \ttfamily
        编译论文\\
        清理中间文件\\
        $\hookrightarrow +$删除论文\\
        统计字数
      \end{block}
    \end{column}
  \end{columns}
\end{frame}

\begin{frame}[label=compile]
  \frametitle{编译问题排查}
  \begin{columns}
    \begin{column}{0.33\textwidth}
      \begin{alertblock}{无法使用 \texttt{latexmk}\thesisissue{578}}
        \hologo{MiKTeX} 需要安装 Perl 解释器。
      \end{alertblock}  
      \begin{alertblock}{C\TeX{} 套装无法编译\thesisissue{446}}
        使用最新 \TeX{} 发行版。
      \end{alertblock}
      \begin{alertblock}{\hologo{pdfLaTeX} 无法编译\thesisissue{444}}
        请使用 \texttt{latexmk},或更改编辑器设置以 \hologo{XeLaTeX} 编译。
      \end{alertblock}
    \end{column}
    \begin{column}{0.33\textwidth}
      \begin{alertblock}{缺少字体\thesisissue{564} \thesisdiscuss{598}}
        更换字体集,或者安装对应字体。
      \end{alertblock}
      \begin{alertblock}{缺少汉字\thesisissue{533} \thesisdiscuss{617}}
        去除使用 fandol 字体集的设定。或者是安装字体后,改用 \texttt{fontset=adobe} 或 \texttt{fontset=founder}。
      \end{alertblock}
    \end{column}
    \begin{column}{0.33\textwidth}
      \begin{block}{\faInfoCircle{} README}
        不同编辑器的设置请首先参阅 README \link{https://github.com/sjtug/SJTUThesis/blob/master/README.md} 文档。
      \end{block}
      \begin{block}{\faBookOpen{} Wiki}
        其他编译问题推荐查阅 Wiki \link{https://github.com/sjtug/SJTUThesis/wiki} 的使用说明部分。
      \end{block}
    \end{column}
  \end{columns}
\end{frame}

\begin{frame}[fragile, label=covers]
  \begin{codeblock}[firstnumber=3]{main.tex}
|\alert{\% 载入 SJTUThesis 模版}|
\documentclass[|\only<1>{\highlight{type}}\only<2>{type}|=|\only<1>{bachelor}\only<2>{\highlight{bachelor}}|]{sjtuthesis}
  \end{codeblock}
  \begin{figure}
    \parbox{0.9\textwidth}{
      \begin{subfigure}{0.20\textwidth}
        \framebox{\includegraphics[width=\linewidth]{support/thesis/bachelor}}
        \caption{\only<1>{学士}\only<2>{\texttt{bachelor}}}
      \end{subfigure}\hfill
      \begin{subfigure}{0.20\textwidth}
        \framebox{\includegraphics[width=\linewidth]{support/thesis/master}}
        \caption{\only<1>{硕士}\only<2>{\texttt{master}}}
      \end{subfigure}\hfill
      \begin{subfigure}{0.20\textwidth}
        \framebox{\includegraphics[width=\linewidth]{support/thesis/doctor}}
        \caption{\only<1>{博士}\only<2>{\texttt{doctor}}}
      \end{subfigure}\hfill
      \begin{subfigure}{0.20\textwidth}
        \framebox{\includegraphics[width=\linewidth]{support/thesis/course}}
        \caption{\only<1>{课程}\only<2>{\texttt{course}}}
      \end{subfigure}
      \caption{论文类型示例\only<2>{ \texttt{type}}}
    }
  \end{figure}
\end{frame}

\begin{frame}[fragile]
  \frametitle{文档类选项}
  % \framesubtitle{\textbackslash{}documentclass\{sjtuthesis\}}
  \begin{columns}
    \begin{column}{0.45\textwidth}
      \includegraphics[page=10]{thesisdir}
    \end{column}
    \begin{column}{0.55\textwidth}
      \begin{table}[H]
        \caption{文档类选项}
        \footnotesize
        \begin{tabular}{>{\ttfamily}rll}
          \toprule
          选项 & 含义 & 相关 \\
          \midrule
          type= & 指定论文类型 & 第 \ref{covers} 页\\
          fontset= & 指定字体 & 第 \ref{compile} 页\\
          \midrule
          review & 开启盲审模式 & \thesisissue{195} \thesisissue{686} \\
          twoside & 双页模式 & \thesisissue{554} \\
          oneside & 单页模式 & \thesisissue{694} \\
          openright & 章从奇数页开始 & \thesisdiscuss{724} \\
          openany & 章从任意页开始 & \thesisissue{446} \\
          \bottomrule
        \end{tabular}
      \end{table}
    \end{column}
  \end{columns}
\end{frame}

\begin{frame}[fragile]
  \frametitle{基本配置}
  \framesubtitle{\textbackslash{}input\{setup\}}
  \begin{columns}
    \begin{column}{0.45\textwidth}
      \includegraphics[page=9]{thesisdir}
    \end{column}
    \begin{column}{0.55\textwidth}
      \begin{codeblock}[firstnumber=12]{main.tex}
|\highlightline<1>|% 论文基本配置,加载宏包等全局配置
|\highlightline<1>|\input{setup}

\begin{document}

%TC:ignore

|\highlightline<2>|% 标题页
|\highlightline<2>|\maketitle
      \end{codeblock}
      \visible<2>{
        \cmd{sjtusetup} 中的 \pkg{info} 将会修改封面的信息设置(见第 \ref{covers} 页)。
      }
    \end{column}
  \end{columns}
\end{frame}

\begin{frame}[fragile]
  \frametitle{基本配置}
  \framesubtitle{\textbackslash{}sjtusetup}
  \begin{columns}
    \begin{column}{0.45\textwidth}
      \includegraphics[page=1]{thesisdir}
    \end{column}
    \begin{column}{0.55\textwidth}
      \begin{codeblock}[firstnumber=3]{setup.tex}
\sjtusetup{
  info = {
    title    = {||上海交通大学学位论文 \LaTeX{} 模板示例文档},
    title*   = {A Sample for \LaTeX-based SJTU Thesis Template},
    author   = {||某\quad{}某},
    author* = {Mo Mo},
  },
  style = { header-logo-color = red, 
  },
  name = {
    publications = {||攻读学位期间完成的论文},
  },
}
      \end{codeblock}
    \end{column}
  \end{columns}
\end{frame}

\begin{frame}
  \frametitle{基本配置}
  \framesubtitle{\textbackslash{}sjtusetup}
  \begin{columns}
    \begin{column}{0.45\textwidth}
      \includegraphics[page=1]{thesisdir}
    \end{column}
    \begin{column}{0.55\textwidth}
      \begin{table}[H]
        \centering
        \caption{info 域}
        \footnotesize
        \begin{tabular}{lll} \toprule
          命令作用 & 中文对应选项 & 英文对应选项 \\ \midrule
          论文标题 & \texttt{title} & \texttt{title*} \\
          关键字列表 & \texttt{keywords} & \texttt{keywords*} \\
          作者姓名&  \texttt{author} &\texttt{author*}\\
          申请学位名称 & \texttt{degree}&\texttt{degree*}\\
          院系名称 & \texttt{department} & \texttt{department*}\\
          专业名称 & \texttt{major} & \texttt{major*}\\
          导师 & \texttt{supervisor} & \texttt{supervisor*}\\
          副导师 & \texttt{assisupervisor} & \texttt{assisupervisor*}\\
          日期 & \multicolumn{2}{c}{\texttt{date}}\\
          学号 & \multicolumn{2}{c}{\texttt{id}}\\ \bottomrule
          \end{tabular}
      \end{table}
    \end{column}
  \end{columns}
\end{frame}

\begin{frame}[fragile]
  \frametitle{版权页}
  \framesubtitle{\textbackslash{}copyrightpage}
  \begin{columns}
    \begin{column}{0.45\textwidth}
      \only<1>{
        \includegraphics[page=9]{thesisdir}
      }
      \only<2>{
        \includegraphics[page=2]{thesisdir}
      }
      \only<3>{
        \begin{figure}[H]
          \framebox{\includegraphics[page=2,width=0.4\linewidth]{bachelor}}
          \caption{版权页}
        \end{figure}
      }
    \end{column}
    \begin{column}{0.55\textwidth}
      \begin{codeblock}[firstnumber=22]{main.tex}
|\highlightline<1>|% 原创性声明及使用授权书
|\highlightline<1>|\copyrightpage
|\highlightline<2>|% 插入外置原创性声明及使用授权书
|\highlightline<2>|% \copyrightpage[scans/sample-copyright-old.pdf]
      \end{codeblock}
      \only<1>{
        \cmd{copyrightpages} 可以用于插入版权页。
      }
      \only<2>{
        \cmd{copyrightpages} 也接受一个可选参数,用于直接使用扫描件。
      }
    \end{column}
  \end{columns}
\end{frame}

\begin{frame}[fragile]
  \frametitle{前置部分}
  \framesubtitle{\textbackslash{}frontmatter}
  \begin{columns}
    \begin{column}{0.45\textwidth}
      \only<1>{
        \includegraphics[page=9]{thesisdir}
      }
      \only<2>{
        \includegraphics[page=3]{thesisdir}
      }
      \only<3>{
        \begin{figure}[H]
          \begin{subfigure}{0.45\textwidth}
            \framebox{\includegraphics[page=3,width=\linewidth]{bachelor}}
            \caption{中文}
          \end{subfigure}\hfill
          \begin{subfigure}{0.45\textwidth}
            \framebox{\includegraphics[page=4,width=\linewidth]{bachelor}}
            \caption{英文}
          \end{subfigure}
          \caption{摘要}
        \end{figure}
      }
      \only<4>{
        \begin{figure}[H]
          \begin{subfigure}{0.30\linewidth}
            \centering
            \framebox{\includegraphics[page=5,width=0.6\linewidth]{bachelor}}
            \caption{目录}
          \end{subfigure}
          \begin{subfigure}{0.30\linewidth}
            \centering
            \framebox{\includegraphics[page=6,width=0.6\linewidth]{bachelor}}
            \caption{插图}
          \end{subfigure}

          \begin{subfigure}{0.30\linewidth}
            \centering
            \framebox{\includegraphics[page=7,width=0.6\linewidth]{bachelor}}
            \caption{表格}
          \end{subfigure}
          \begin{subfigure}{0.30\linewidth}
            \centering
            \framebox{\includegraphics[page=8,width=0.6\linewidth]{bachelor}}
            \caption{算法}
          \end{subfigure}
          \caption{索引}
        \end{figure}
      }
      \only<5>{
        \includegraphics[page=4]{thesisdir}
      }
      \only<6>{
        \begin{figure}[H]
          \framebox{\includegraphics[page=9,width=0.5\linewidth]{bachelor}}
          \caption{符号对照表}
        \end{figure}
      }
    \end{column}
    \begin{column}{0.55\textwidth}
      \begin{codeblock}[firstnumber=30]{main.tex}
|\highlightline<2-3>|% 摘要
|\highlightline<2-3>|\input{contents/abstract}

|\highlightline<4>|% 目录
|\highlightline<4>|\tableofcontents
|\highlightline<4>|% 插图索引
|\highlightline<4>|\listoffigures*
|\highlightline<4>|% 表格索引
|\highlightline<4>|\listoftables*
|\highlightline<4>|% 算法索引
|\highlightline<4>|\listofalgorithms*

|\highlightline<5-6>|% 符号对照表
|\highlightline<5-6>|\input{contents/nomenclature}
      \end{codeblock}
    \end{column}
  \end{columns}
\end{frame}

\begin{frame}[fragile]
  \frametitle{主体部分}
  \framesubtitle{\textbackslash{}mainmatter}
  \begin{columns}
    \begin{column}{0.45\textwidth}
      \only<1>{
        \includegraphics[page=5]{thesisdir}
      }
      \only<2>{
        \begin{figure}[H]
          \begin{subfigure}{0.30\linewidth}
            \centering
            \framebox{\includegraphics[page=11,width=0.6\linewidth]{bachelor}}
            \caption{简介}
          \end{subfigure}
          \begin{subfigure}{0.30\linewidth}
            \centering
            \framebox{\includegraphics[page=13,width=0.6\linewidth]{bachelor}}
            \caption{数学}
          \end{subfigure}

          \begin{subfigure}{0.30\linewidth}
            \centering
            \framebox{\includegraphics[page=16,width=0.6\linewidth]{bachelor}}
            \caption{浮动体}
          \end{subfigure}
          \begin{subfigure}{0.30\linewidth}
            \centering
            \framebox{\includegraphics[page=22,width=0.6\linewidth]{bachelor}}
            \caption{总结}
          \end{subfigure}
          \caption{主体部分}
        \end{figure}
      }
    \end{column}
    \begin{column}{0.55\textwidth}
      \begin{codeblock}[firstnumber=47]{main.tex}
|\highlightline|% 正文内容
|\highlightline|% !TeX root = ../../../latex-talk.tex

\section{是什么}

\begin{frame}
  \frametitle{\TeX{}}
  \begin{columns}[c]
    \begin{column}{0.7\textwidth}
      \begin{center}
        \rmfamily\Huge
        \highlight[structure]{\TeX{}}
      \end{center}
      \begin{center}
        \parbox{0.75\textwidth}{
          \TeX{} 是由斯坦福大学教授高德纳
          (Donald E.~Knuth)于 1977 年开始开发的排版引擎。目前仍在更新,最新版本号为 3.141592653 \link{https://tug.org/TUGboat/tb42-1/tb130knuth-tuneup21.pdf}。
        }
      \end{center}
    \end{column}
    \begin{column}{0.3\textwidth}
      \includegraphics[width=.8\columnwidth]{support/images/Knuth.jpg}
    \end{column}
  \end{columns}
  \note{\emph{这一部分背景介绍大家可以了解一下,暂时跳过。}
  \LaTeX{} 这个词由两个部分组成,\hologo{La} 和 \TeX{}。那我们首先了解一下 \TeX{} 是什么。
  \TeX{} 是由斯坦福大学的教授高德纳于 1977 年开始开发的排版引擎,它已经有三十多年的历史了,
  目前仍在更新,版本号(3.141592653)将会趋近于 $\pi$ 的取值,高德纳最近还在给 \textsl{TUGBoat} 写稿子
  \link{https://tug.org/TUGboat/tb42-1/tb130knuth-tuneup21.pdf},
  关于 \TeX{} 今年又做了哪些改进。}
\end{frame}

\begin{frame}
  \frametitle{\LaTeX{}}
  \begin{columns}[c]
    \begin{column}{0.7\textwidth}
      \begin{center}
        \rmfamily\Huge
        \highlight[structure]{\LaTeX{}}
      \end{center}
      \begin{center}
        \parbox{0.75\textwidth}{
          \LaTeX{} 是最早在 1985 年由现就职于微软的 Leslie Lamport 开发的一种 \TeX{} \textbf{格式}\footnotemark,使用一些列宏和扩展宏包来简化 \TeX{} 的使用。现在由 \LaTeX{} Project 的成员维护。现在广泛使用的版本是 \LaTeXe{},最新的版本为 \LaTeX3(2020 年 10 月后默认内置)。
        }
      \end{center}
    \end{column}
    \begin{column}{0.3\textwidth}
      \includegraphics[width=.8\columnwidth]{support/images/Lamport.jpg}
    \end{column}
  \end{columns}
  \footnotetext{\hologo{ConTeXt} 也是一种 \TeX{} 格式 \link{https://www.contextgarden.net/}。}
  \note{\emph{这一部分的背景介绍大家可以了解一下,暂时跳过。}
  \LaTeX{} 是最早由现就职于微软的 Leslie Lamport 开发的一种 \TeX{} 格式(与其对标的是
  \hologo{ConTeXt}\link{https://www.contextgarden.net/}),主要也是为了简化 \TeX{} 的使用。
  现在主要由 \LaTeX{} 开发组维护,现在广泛使用的版本是 \LaTeXe{},最新的版本为 \LaTeX3,
  在 2020 年 10 月后默认内置,所以要尽可能使用较新的发行版,以充分发挥其功能。}
\end{frame}

\begin{frame}
  \frametitle{程序}
  \begin{columns}[c]
    \begin{column}{0.7\textwidth}
      \begin{center}
        \rmfamily\Huge
        \highlight[structure]{\hologo{pdfLaTeX}}
      \end{center}
      \begin{center}
        \parbox{0.7\textwidth}{
          \hologo{pdfLaTeX} 是为了编译一个 \LaTeX{} 文档而运行的程序。实际上底层在运行一个叫 \hologo{pdfTeX} 的引擎,并预装了对应的 \LaTeX{} \textbf{格式}。为了利用临时文件,可能就需要多次运行程序。
        }
      \end{center}
    \end{column}
    \begin{column}{0.3\textwidth}
      \begin{block}{}
        \ttfamily\small
        > \highlight{pdflatex} main.tex\\
        This is pdfTeX, Version 3.141592653-
        2.6-1.40.23 (MiKTeX 21.10)\\
        entering extended mode\\
        \highlight{LaTeX2e} <2021-11-15>\\
        \highlight{L3} programming layer <2021-11-22>
      \end{block}
    \end{column}
  \end{columns}
  \note{\hologo{pdfLaTeX} 是为了编译一个 \LaTeX{} 文档而运行的程序。}
\end{frame}

% \begin{frame}
%   \frametitle{引擎}
%   \begin{columns}[c]
%     \begin{column}{0.7\textwidth}
%       \begin{center}
%         \rmfamily\Huge
%         \highlight[structure!70]{pdf}\hologo{La}\highlight[structure!70]{\TeX{}}
%       \end{center}
%       \begin{center}
%         \parbox{0.7\textwidth}{
%           pdf\TeX{} 是编译 \TeX{} 文档(以 \texttt{.tex} 结尾)的\textbf{引擎}---可以理解 \TeX{} 指令的\textbf{程序}。
%         }
%       \end{center}
%     \end{column}
%     \begin{column}{0.3\textwidth}
%       \begin{block}{}
%         \ttfamily\small
%         > pdflatex main.tex\\
%         This is \highlight[structure!70]{pdfTeX}, Version 3.141592653-
%         2.6-1.40.23 (MiKTeX 21.10)
%         entering extended mode\\
%         LaTeX2e <2021-11-15>\\
%         L3 programming layer <2021-11-22>
%       \end{block}
%     \end{column}
%   \end{columns}
%   \note{实际上底层在运行一个叫 \hologo{pdfTeX} 的引擎,并预装了对应的 \LaTeX{} 格式。}
% \end{frame}

\begin{frame}[label={frame:engine}]
  \frametitle{程序}
  \begin{table}
    \caption{主流 \hologo{(La)TeX} 程序
    \footnote{(u)p\TeX{} 是日语最常用的引擎,生成 \texttt{.dvi},支持 Unicode。}\footnote{Ap\TeX{} \link{https://github.com/clerkma/ptex-ng} 具有底层 CJK 支持,内联 Ruby,Color Emoji。}}
    \footnotesize
    \begin{stampbox}
      \begin{tabular}{c>{\raggedright}*{3}{p{3.5cm}}}
        \alert{引擎}     & \hologo{pdfTeX}   & \hologo{XeTeX}   & \hologo{LuaTeX}   \\
        \alert{程序}     & \hologo{pdfLaTeX} & \hologo{XeLaTeX} & \hologo{LuaLaTeX} \\
        \alert{特点}     & 直接生成 PDF,支持 micro-typography  & 支持 Unicode、OpenType 与复杂文字编排 (CTL) & 支持 Unicode,内联 Lua,支持 OpenType \\
      \end{tabular}
    \end{stampbox}
  \end{table}

  \begin{center}
    \parbox{.9\textwidth}{
      \hologo{pdfLaTeX} 不支持 Unicode。为了排版中文,大部分情况下应当使用 \hologo{XeLaTeX},而 \hologo{LuaLaTeX} 速度相对较慢。\faWindows{} 可以在一些情况下使用 \hologo{pdfLaTeX}。
    }
  \end{center}
  \note{当然为了排版中文,已经不再推荐使用 \hologo{pdfLaTeX} 了,应该使用
  \hologo{XeLaTeX} 或者 \hologo{LuaLaTeX},当然后者的速度还是相对较慢,
  它们支持 Unicode 编码,并可以使用 OpenType 字体的全部功能。
  当然 \faWindows{} 平台下在某些追求速度的情况下,
  还是可以试着使用 \hologo{pdfLaTeX} 的。

  \hologo{LuaLaTeX} 理想情况下不慢,但是使用一些宏包后会破坏理想状态,
  也会因配置产生不同的结果,不同的操作系统在 I/O 速度上的不同也会导致不同的时间。

  \hologo{pdfLaTeX} 也支持,只不过需要先生成 tfm \TeX{} 字体度量文件,后续使用 \TeX{}
  自身的配置方法,只能使用 7 比特或 8 比特字体。}
\end{frame}

% \begin{frame}
%   \paragraph{\hologo{pdfLaTeX}} \TeX{} 和 \LaTeX{} 被广泛使用之前,它们只需内置支持欧洲语言即可。在 Unicode 出现之前,\LaTeX{} 提供了许多种\textbf{文件编码}来允许很多语言的文字以原生的方式输入,\hologo{pdfLaTeX} 也只需要使用 8 位文件编码和 8 位字体。
% \end{frame}


|\highlightline|\input{contents/math_and_citations}
|\highlightline|\input{contents/floats}
|\highlightline|\input{contents/summary}

%TC:ignore

% 参考文献
\printbibliography[heading=bibintoc]
      \end{codeblock}
    \end{column}
  \end{columns}
\end{frame}

\begin{frame}
  \frametitle{数学}
  \begin{itemize}
    \item 公式示例:\nolinkurl{contents/math_and_citations.tex}
    \item \SJTUThesis{} 定义了常用的数学环境(需要手工引入 \texttt{ntheorem} 宏包):
      \begin{table}[h]
        \centering
        \footnotesize
        \begin{tabular}{*{7}{l}}\toprule
          assumption  & axiom   & conjecture & corollary    & definition  & example   & exercise  \\
          假设        & 公理    & 猜想       & 推论         & 定义        & 例        & 练习      \\\midrule
          lemma       & problem & proof      & proposition  & remark      & solution  & theorem   \\
          引理        & 问题    & 证明       & 命题         & 注          & 解        & 定理      \\\bottomrule
        \end{tabular}
      \end{table}
      \item \SJTUThesis{} 可以通过 \texttt{unimath} 选项使用 \pkg{unicode-math} 进行数学输入,注意与传统方式的区别。\thesisissue{555}
  \end{itemize}
\end{frame}

\begin{frame}[fragile]
  \frametitle{参考文献}
  \begin{columns}
    \begin{column}{0.45\textwidth}
      \includegraphics[page=6]{thesisdir}
    \end{column}
    \begin{column}{0.55\textwidth}
      \begin{codeblock}[firstnumber=111,numbersep=2pt]{setup.tex}
% 使用 BibLaTeX 处理参考文献
%   biblatex-gb7714-2015 常用选项
%     gbnamefmt=lowercase     姓名大小写由输入信息确定
%     gbpub=false             禁用出版信息缺失处理
\usepackage[backend=biber,style=gb7714-2015]{biblatex}
% 文献表字体
% \renewcommand{\bibfont}{\zihao{-5}}
% 文献表条目间的间距
\setlength{\bibitemsep}{0pt}
|\highlightline|% 导入参考文献数据库
|\highlightline|\addbibresource{bibdata/thesis.bib}
      \end{codeblock}
    \end{column}
  \end{columns}
\end{frame}

\begin{frame}[fragile]
  \frametitle{附录}
  \framesubtitle{\textbackslash{}appendix}
  \begin{columns}
    \begin{column}{0.45\textwidth}
      \only<1>{
        \includegraphics[page=7]{thesisdir}
      }
      \only<2>{
        \begin{figure}[H]
          \begin{subfigure}{0.45\linewidth}
            \framebox{\includegraphics[width=\linewidth,page=24]{bachelor}}
            \caption{}
          \end{subfigure}\hfill
          \begin{subfigure}{0.45\textwidth}
            \framebox{\includegraphics[width=\linewidth,page=25]{bachelor}}
            \caption{}
          \end{subfigure}
          \caption{附录}
        \end{figure}
      }
    \end{column}
    \begin{column}{0.55\textwidth}
      \begin{codeblock}[firstnumber=61]{main.tex}
% 附录中图表不加入索引
\captionsetup{list=no}

% 附录内容
|\highlightline|\input{contents/app_maxwell_equations}
|\highlightline|\input{contents/app_flow_chart}
      \end{codeblock}
    \end{column}
  \end{columns}
\end{frame}

\begin{frame}[fragile]
  \frametitle{结尾部分}
  \framesubtitle{\textbackslash{}backmatter}
  \begin{columns}
    \begin{column}{0.45\textwidth}
      \only<1>{
        \includegraphics[page=8]{thesisdir}
      }
      \only<2>{
        \begin{figure}[H]
          \begin{subfigure}{0.30\linewidth}
            \centering
            \framebox{\includegraphics[page=26,width=0.6\linewidth]{bachelor}}
            \caption{致谢}
          \end{subfigure}
          \begin{subfigure}{0.30\linewidth}
            \centering
            \framebox{\includegraphics[page=27,width=0.6\linewidth]{bachelor}}
            \caption{成就}
          \end{subfigure}

          \begin{subfigure}{0.30\linewidth}
            \centering
            \framebox{\includegraphics[page=28,width=0.6\linewidth]{bachelor}}
            \caption{简历}
          \end{subfigure}
          \begin{subfigure}{0.30\linewidth}
            \centering
            \framebox{\includegraphics[page=29,width=0.6\linewidth]{bachelor}}
            \caption{大摘要*}
          \end{subfigure}
          \caption{结尾部分}
        \end{figure}
      }
    \end{column}
    \begin{column}{0.55\textwidth}
      \begin{codeblock}[firstnumber=76]{main.tex}
% 致谢
\input{contents/acknowledgements}

% 发表论文及科研成果
% 盲审论文中,发表论文及科研成果等仅以第几作者注明即可,不要出现作者或他人姓名
\input{contents/achievements}

% 简历
\input{contents/resume}

% 学士学位论文要求在最后有一个大摘要,单独编页码
\input{contents/digest}
      \end{codeblock}
    \end{column}
  \end{columns}
\end{frame}

\begin{frame}
  \frametitle{还有其他问题?}
  \begin{columns}
    \begin{column}{0.75\textwidth}
    \begin{itemize}
      \item[{\faComment*[regular]}] 日常模板或 \LaTeX{} 使用问题可以前往 Discussions \link{https://github.com/sjtug/SJTUThesis/discussions} 提问
      
      (解决后别忘了 \textcolor{green}{\faCheckCircle{} Mark as answer}
      \item[{\faDotCircle[regular]}] 如果是 \textsc{SJTUThesis} 项目本身的 bug 和 feature request
      
      可以通过 Issues \link{https://github.com/sjtug/SJTUThesis/issues} 反馈。
      \item[{\faCodeBranch}] 如果你有好点子,可以贡献代码
     
      向 \textsc{SJTU\TeX{}}(v1) \link{https://github.com/sjtug/SJTUTeX/tree/v1} 存储库发 PR,\par
      而后把解包结果同步到 \textsc{SJTUThesis}。
  
      \item[{\faTag}] 如果你对正在基于 \LaTeX3 开发的新版感兴趣,\par
      也欢迎向 \textsc{SJTU\TeX{}}(v2) \link{https://github.com/sjtug/SJTUTeX/tree/v2} 发 PR。
  
      \item[{\faQq}] 也欢迎在 QQ 群即时讨论。
    \end{itemize}
    \end{column}
    \begin{column}{0.25\textwidth}
      \includegraphics[height=0.7\textheight]{qq.jpg}
    \end{column}
  \end{columns}
\end{frame}
\end{document}
      \end{codeblock}
    \end{column}
  \end{columns}
  \footnotetext{如果想强制指定子文档的主文档,可以在文件第一行输入魔术命令:\texttt{\% !TeX root = main.tex}}
\end{frame}

\section{图}
\begin{frame}[fragile]%
  \frametitle{\temporal<5>{插图}{浮动体}{插图}}
  \begin{columns}
    \begin{column}{0.6\textwidth}
      \begin{codeblock}[]{插入单图\only<4->{最佳实践}}
\documentclass{ctexart}
|\only<2>{\highlightline}|\usepackage{graphicx}
|\only<2>{\highlightline}|\graphicspath{{figs/}{pics/}}
\begin{document}
|\only<5>{\highlightline}|\begin{figure}[ht]
|\only<6>{\highlightline}|  \centering
|\only<3>{\highlightline}|  \includegraphics[width=|\only<1-3>{4cm}\only<4->{0.4\textbackslash{}textwidth}|]{sjtug}
|\only<7>{\highlightline}|  \caption{SJTUG 徽标}\label{fig:sjtug}
|\only<5>{\highlightline}|\end{figure}
\end{document}
      \end{codeblock}
    \end{column}
    \begin{column}{0.4\textwidth}
      \only<1>{
        \includepdflarge{insertimage}
      }
      \only<2>{
        为了插入外部图片,需要使用 \pkg{graphicx} 宏包。之后在文档主体便可以使用 \cmd{includegraphics} 插入图片。导言区也可以加入 \cmd{graphicspath} 指定图片文件夹\footnotemark。
      }
      \only<3>{
        \cmd{includegraphics} 命令便以相对路径的方式插入图片,如果无同名图片,那么后缀名可以省略。可以使用可选参数指定插入的图片尺寸,最佳实践是使用 \cmd{textwidth} 或 \cmd{linewidth} 的相对值指定尺寸大小,以在未来可能的布局更改中保留一定的灵活性。
      }
      \only<4>{
        也可以通过键值对的方法设置图片的其他属性。
        \begin{center}
          \footnotesize
          \begin{stampbox}
            \begin{tabular}{rl}
              \pkg{width} & 宽度 \\
              \pkg{height} & 高度 \\
              \pkg{scale} & 缩放 \\
              \pkg{angle} & 角度 \\
            \end{tabular}
          \end{stampbox}
        \end{center}
      }
      \only<5>{
        \env{figure} 为一个浮动体环境(\env{table} 也是),可以将其移动到其他位置上以减少行文中的空白。可以添加可选参数以指定如何放置浮动体,最多可以使用四种位置描述符:
        \begin{center}
          \footnotesize
          \begin{stampbox}
            \begin{tabular}{cll}
              \pkg{h} & Here & 尽可能在这里 \\
              \pkg{t} & Top & 页面顶部 \\
              \pkg{b} & Bottom & 页面底部 \\
              \pkg{p} & Page & 浮动体专页 \\
            \end{tabular}
          \end{stampbox}
        \end{center}
        还可以只使用 \pkg{float} 宏包提供的 \pkg{H} 描述符以强制置于此处。
      }
      \only<6>{
        采用 \cmd{centering} 命令而不是 \env{center} 环境来水平居中图片。这将避免多余的纵向间距。
      }
      \only<7>{
        使用 \cmd{caption} 命令输入题注,如果这个命令写在插入图片前面,题注将会在上方(中文中一般对 \env{table} 环境这么做)。后面将会看到如何对留有标记(\cmd{label})的图片进行引用。
      }
    \end{column}
  \end{columns}
  \only<2>{\footnotetext{其命令参数每个为一个以 \texttt{/} 结尾的文件夹,每个文件夹需要使用大括号包裹起来。}}
\end{frame}

\begin{frame}[fragile]
  \begin{columns}
    \begin{column}{0.6\textwidth}
      \begin{codeblock}[]{插入双图}
\documentclass{ctexart}
\usepackage{graphicx}
\graphicspath{{figs/}{pics/}}
\begin{document}
  \begin{figure}[ht]
|\only<1>{\highlightline}|    \begin{minipage}{0.48\textwidth}
      \centering
      \includegraphics[height=2cm]{sjtug}
|\only<2>{\highlightline}|      \caption{SJTUG 徽标}\label{fig:sjtug}
|\only<1>{\highlightline}|    \end{minipage}\hfill
|\only<1>{\highlightline}|    \begin{minipage}{0.48\textwidth}
      \centering
      \includegraphics[height=2cm]{sjtugt}
|\only<2>{\highlightline}|      \caption{SJTUG||文字}\label{fig:sjtugt}
|\only<1>{\highlightline}|    \end{minipage}
  \end{figure}
\end{document}
      \end{codeblock}
    \end{column}
    \begin{column}{0.4\textwidth}
      \only<1>{
        在 \env{figure} 环境里使用 \env{minipage} 小页制作列盒子,以并排两图并分别编号,需要设定强制参数以指定列宽。两个小页之间添加 \cmd{hfill} 使两个小页两端对齐。
      }
      \only<2>{
        在每个小页内部分别使用 \cmd{caption} 添加标签。
      }
      \only<3>{
        \includepdflarge{doubleimages}
      }
    \end{column}
  \end{columns}
\end{frame}

\begin{frame}[fragile]%
  \begin{columns}
    \begin{column}{0.6\textwidth}
      \begin{codeblock}[]{}
\documentclass{ctexart}
\usepackage{graphicx}
|\highlightline|\usepackage{subcaption}
\graphicspath{{figs/}{pics/}}
\begin{document}
  \begin{figure}[ht]
|\highlightline|    \begin{subfigure}{0.48\textwidth}
      \centering
      \includegraphics[height=2cm]{sjtug}
      \caption{||徽标}
|\highlightline|    \end{subfigure}\hfill
|\highlightline|    \begin{subfigure}{0.48\textwidth}
      \centering
      \includegraphics[height=2cm]{sjtugt}
      \caption{||文字}
|\highlightline|    \end{subfigure}
    \caption{SJTUG}\label{fig:sjtug}
  \end{figure}
\end{document}
      \end{codeblock}
    \end{column}
    \begin{column}{0.4\textwidth}
      \includepdflarge{subfigures}\vspace{15pt}
      \pkg{subcaption} 宏包提供了 \env{subfigure} 环境(以及 \env{subtable}),可以用于以子图的形式插入,编号会增加一级。也可以为子图添加子集引用编号。
    \end{column}
  \end{columns}
\end{frame}

\section{表}
\begin{frame}[fragile]
  \frametitle{简单表格}
  \begin{columns}
    \begin{column}{0.6\textwidth}
      \begin{codeblock}[]{}
\documentclass{ctexart}
|\only<1-2>{\highlightline}|\usepackage{|\temporal<1>{array}{\highlight{array}}{array},\temporal<2>{booktabs}{\highlight{booktabs}}{booktabs}|}
\begin{document}
\begin{table}[ht]
  \centering
  \caption{||北京冬奥会吉祥物}
|\only<1>{\highlightline}|  \begin{tabular}{lp{3cm}}
|\only<2>{\highlightline}|    \toprule
|\only<3>{\highlightline}|吉祥物 & 描述                          \\
|\only<2>{\highlightline}|    \midrule
|\only<3>{\highlightline}|冰墩墩 & 2022 年北京冬季奥运会吉祥物  \\
|\only<3>{\highlightline}|雪容融 & 2022 年北京冬季残奥会吉祥物  \\
|\only<2>{\highlightline}|    \bottomrule
|\only<1>{\highlightline}|  \end{tabular}
\end{table}
\end{document}
      \end{codeblock}
    \end{column}
    \begin{column}{0.4\textwidth}
      \only<1>{
        使用 \env{tabular} 环境绘制表格。需要添加参数(称为\textbf{表格导言})以确定每一列的对齐方式。引入 \pkg{array} 宏包来使用更多的\textbf{引导符}。
        \begin{center}
          \footnotesize
          \begin{stampbox}
            \begin{tabular}{>{\ttfamily}ll}
              \alert{l} & 向左对齐 \\
              \alert{c} & 居中对齐 \\
              \alert{r} & 向右对齐 \\
              \alert{p\{3cm\}} & 固定列宽,两端对齐 \\
              \alert{m\{3cm\}} & \texttt{p} + 垂直居中对齐 \\
              \alert{>\{\textbackslash{}bfseries\}} & 后一列单元格前加命令 \\
              \alert{*\{3\}\{l\}} & 三个左对齐列 \\
            \end{tabular}
          \end{stampbox}
        \end{center}
      }
      \only<2>{
        \pkg{booktabs} 宏包提供了标准三线表格所需要的行分割线:\cmd{toprule},\cmd{midrule},\cmd{bottomrule}。也可以使用 \cmd{cmidrule\{1-2\}} 来部分地绘制行分割线。一般不推荐在专业表格中使用纵向分割线。
      }
      \only<3>{
        每行内容使用 \textbackslash\textbackslash{} 分隔开,每行中的单元格使用 \& 分隔开。
      }
      \only<4>{
        \includepdflarge{table}
      }
    \end{column}
  \end{columns}
\end{frame}

\begin{frame}[fragile]%
  \begin{columns}
    \begin{column}{0.6\textwidth}
      \begin{codeblock}[]{表头居中}
\documentclass{ctexart}
\usepackage{array,booktabs}
\begin{document}
\begin{table}[ht]
  \centering
  \caption{||北京冬奥会吉祥物}
  \begin{tabular}{lp{3cm}}
    \toprule
|\highlightline|\multicolumn{1}{c}{||吉祥物} &
|\highlightline|\multicolumn{1}{c}{||描述} \\
    \midrule
||冰墩墩 & 2022 年北京冬季奥运会吉祥物  \\
||雪容融 & 2022 年北京冬季残奥会吉祥物  \\
    \bottomrule
  \end{tabular}
\end{table}
\end{document}
      \end{codeblock}
    \end{column}
    \begin{column}{0.4\textwidth}
      \cmd{multicolumn} 命令不仅可以用于合并同行的单元格,还可以用于临时地屏蔽表格导言对该列的对齐方式定义。这里用于居中表头。
      \begin{center}
        \begin{stampbox}
          \parbox{0.85\linewidth}{
            \ttfamily\color{blue}\textbackslash{}multicolumn\{格数\}\{对齐方式\}\{内容\}
          }
        \end{stampbox}
      \end{center}
      跨页表格考虑使用 \pkg{longtable} 宏包。带标注的表格可以考虑使用 \pkg{threeparttable} 宏包。考虑使用在线工具生成表格代码 \link{https://www.tablesgenerator.com/latex_tables}。
    \end{column}
  \end{columns}
\end{frame}

\section{数学公式}
\begin{frame}
  \frametitle{数学模式}
  \begin{alertblock}{}
  输入数学公式是 \LaTeX{} 的绝对强项,很多常见的公式服务依赖于一些轻量级渲染引擎比如 MathJax, K\kern-.3ex\raise.4ex\hbox{\footnotesize A}\kern-.3ex\TeX{}。但是它们实际上使用的是 \LaTeX{} 语法变种,也就是说并没有使用 \LaTeX{} 后端。所以不要期望有完全一致的输出。
  \end{alertblock}
  
  为了更好的获得数学公式输入支持,请使用 \hologo{AmS}math 宏包。数学模式分为两种:
  \begin{description}
    \item[行内模式] 一般通过一对美元符号(\$$\cdots$\$)标记,可以使用编辑器内建的符号表输入数学符号,也可以使用在线工具手写识别 \link{https://detexify.kirelabs.org/classify.html}。
    \item[行间模式] 一般通过 \env{equation} 环境\footnote{这是有编号环境,其加星号的变种 \env{equation*} 用于生成无编号环境。}输入。如果需要使用多行公式,请使用 \env{align} 环境,并按照类似表格输入的方式,使用 \& 对齐符号,\textbackslash\textbackslash{} 换行。如果不想手动居中,可以考虑多行自动居中的 \env{gather} 和单个大型公式首尾两端对齐 \env{multline}。
  \end{description}
\end{frame}

\begin{frame}
  \frametitle{数学命令展示}
  \begin{columns}
    \begin{column}{0.33\textwidth}
      \begin{exampleblock}{}
        $PV=nRT$
      \end{exampleblock}
      \begin{exampleblock}{}
        $\sum_{i=1}^ki^2=\frac{n(n+1)(2n+1)}{6}$
      \end{exampleblock}
      \begin{exampleblock}{}
        $T(n) = aT\left(\left\lceil\frac{n}{b}\right\rceil\right) + \mathcal{O}(n^d)$
      \end{exampleblock}
      \begin{exampleblock}{}
        $\frac{x_{1}+x_{2}+x_{3}}{3}\geq \sqrt[3]{x_{1}x_{2}x_{3}}$
      \end{exampleblock}
      \begin{exampleblock}{}
        $n=(\underbrace{1\cdots 1}_{k\text{ of 1's}})_2=2^{k+1}-1$
      \end{exampleblock}
      \begin{exampleblock}{}
        $\nabla f (P)= \left.\left(\frac{\partial f}{\partial x},\frac{\partial f}{\partial y},\frac{\partial f}{\partial z}\right)\right|_{P}$
      \end{exampleblock}
    \end{column}
    \begin{column}{0.67\textwidth}
      \begin{exampleblock}{}
        \begin{equation*}
          \int_{a}^b f(x)\,\mathrm{d}x=\lim_{|P|\rightarrow 0}\sum_{i=1}^n f(\xi_i)\Delta x_i
        \end{equation*}
      \end{exampleblock}
      \begin{exampleblock}{}
        \begin{equation}
          T(n) = \begin{cases}
            \mathcal{O}(n^d),&\textrm{if } d>\log_b a, \\
            \mathcal{O}(n^d\log n), &\textrm{if } d=\log_b a,\\
            \mathcal{O}(n^{\log_b a}), &\textrm{if } d<\log_b a.
          \end{cases}
        \end{equation}
      \end{exampleblock}
      \begin{exampleblock}{}
        \begin{align}
          Q^{T}A&=R \\
          \begin{pmatrix}
            q_1^T \\ q_2^T \\ q_3^T
          \end{pmatrix}
          \begin{pmatrix}
            a_1 & a_2 & a_3
          \end{pmatrix}
          &=R
        \end{align}
      \end{exampleblock}
    \end{column}
  \end{columns}
\end{frame}

%更深入地讲解 mathtools, unicode-math, siunix

\section{引用}
\begin{frame}[fragile]
  \frametitle{交叉引用}
  \only<1>{
    正如之前所提到的,\LaTeX{} 中使用 \cmd{label} 标记,然后可以使用 \cmd{ref} 来引用这个标记。 \cmd{label} 可以放在使用计数器的对象之后。
  }
  \only<2>{
    为了使得对公式编号的引用带有括号,推荐使用 \hologo{AmS}math 宏包中的 \cmd{eqref} 命令。对于多行公式环境,每一个换行符前都可以添加一个 \cmd{label} 用于引用该行公式。
  }
  \begin{columns}
    \begin{column}{0.5\textwidth}
      \begin{codeblock}[]{图}
\begin{figure}
|\only<1>{\highlightline}|  \caption{||示例}\label{fig:example}
\end{figure}
      \end{codeblock}
      \begin{codeblock}[]{表}
\begin{table}
|\only<1>{\highlightline}|  \caption{||示例}\label{tab:example}
\end{table}
      \end{codeblock}
    \end{column}
    \begin{column}{0.5\textwidth}
\begin{codeblock}[]{目次}
|\only<1>{\highlightline}|\section{||示例}\label{sec:example}
\end{codeblock}

\begin{codeblock}[]{公式}
\begin{equation}
  a = b + c
|\only<1>{\highlightline}|\label{eq:example}
\end{equation}
|\only<2>{\highlightline}|如公式 \eqref{eq:example} 所示,
\end{codeblock}
    \end{column}
  \end{columns}
\end{frame}

\begin{frame}[fragile]
  \frametitle{文献引用}
  \LaTeX{} 管理参考文献可以采用专用数据库文件 \texttt{.bib},很多的文献管理文件比如 EndNote \link{https://lic.sjtu.edu.cn/Default/softshow/tag/MDAwMDAwMDAwMLGImKE}, Zotero \link{https://www.zotero.org/}, JabRef \link{https://www.jabref.org/} 都可以直接导出这种格式的文件用于 \LaTeX{} 论文中的引用。一般不需要手写数据库文件,你只需要注意每一个文献会在数据库中有一个主键,这个类似于上文的 \cmd{label} 标记,只是要引用数据库中的文献需要使用 \cmd{cite} 命令。
  
  \begin{codeblock}[]{ref.bib}
|\highlightline|@phdthesis{devoftech,|\hfill\alert{\% 类型为博士论文,主键为\texttt{devoftech}}|
  title={||新时期我国信息技术产业的发展},
  author={||江泽民},
  year={2008}
}
  \end{codeblock}
\end{frame}

\begin{frame}
  \frametitle{文献引用}
  而让 \LaTeX{} 处理 \texttt{.bib} 数据库文件与引用有两种工作流。你可能需要查清楚如何在编辑器中设置对应的工作流,或者采用后面所提到的高级编译方式 \texttt{latexmk}。
  \begin{columns}
    \begin{column}{0.5\textwidth}
      \begin{block}{\hologo{BibTeX} + \pkg{gbt7714}}
        一般期刊提交使用这种方法,\pkg{natbib} 宏包将提供命令 \cmd{citet}(文本引用) 和 \cmd{citep}(括号引用)。中文引用可以直接使用 \pkg{gbt7714} 宏包,但是角模式和正文模式不能混用。
      \end{block}
    \end{column}
    \begin{column}{0.5\textwidth}
      \begin{block}{\hologo{biber} + \pkg{biblatex}}
        这是更容易自定义的方法,与 \hologo{BibTeX} 的运作方式稍有不同。\pkg{biblatex} 提供了更加智能的引用命令。而中文引用可以使用 \pkg{biblatex} 宏包的样式 \pkg{gb7714-2015},使用该样式需要使用 \hologo{XeLaTeX} 编译。
      \end{block}
    \end{column}
  \end{columns}
\end{frame}

\begin{frame}[fragile]
  \frametitle{文献引用}
  \begin{columns}
    \begin{column}{0.5\textwidth}
      \begin{codeblock}[]{\hologo{BibTeX} + \pkg{gbt7714}}
\documentclass{ctexart}
\usepackage{gbt7714}
\bibliographystyle{gbt7714-numerial}
% \citestyle{numbers}  % 正文模式
\begin{document}
  ||他指出了...\cite{devoftech}
  \bibliography{ref}
\end{document}
      \end{codeblock}
    \end{column}
    \begin{column}{0.5\textwidth}
      \begin{codeblock}[]{\hologo{biber} + \pkg{biblatex}}
\documentclass{ctexart}
\usepackage[backend=biber,style=gb7714-2015]{biblatex}
\addbibresource{ref.bib}
\begin{document}
  ||他在文献 \parencite{devoftech}
  ||指出了...\cite{devoftech}
  \printbibliography
\end{document}
      \end{codeblock}
    \end{column}
  \end{columns}
\end{frame}

\begin{frame}
  \frametitle{文献引用}
  \begin{columns}
    \begin{column}{0.5\textwidth}
      \includepdflarge{bibtex}
    \end{column}
    \begin{column}{0.5\textwidth}
      \includepdflarge{biblatex}
    \end{column}
  \end{columns}
\end{frame}

} % End of customized shaded number logo

|\highlightline<3>|  % !TeX root = ..\..\latex-talk.tex

\part{SJTUThesis}

\begin{frame}
  \frametitle{简介}
  \begin{columns}
    \begin{column}{0.6\textwidth}
      \begin{itemize}
        \item 最早由韦建文于 2009 年 11 月发布 0.1a 版,2018 年起由 SJTUG 接手维护
        \item 最新版:\SJTUThesisVersion{} (\SJTUThesisDate)
        \item 支持本科、硕士、博士学位论文以及课程论文的排版
      \end{itemize}
    \end{column}
    \begin{column}{0.4\textwidth}
      \begin{exampleblock}{}
        \begin{minipage}[c]{1cm}
          \includegraphics[width=0.8cm]{\getcontribpath{sjtug}{vi/sjtug}}
        \end{minipage}
        \begin{minipage}[c]{2cm}
          \href{https://github.com/sjtug}{sjtug}/\href{https://github.com/sjtug/SJTUThesis}{SJTUThesis}
        \end{minipage}
      \end{exampleblock}
      \vspace{-8pt}
      \begin{block}{}
        \scriptsize
        上海交通大学 \hologo{XeLaTeX} 学位论文及课程论文模板 | Shanghai Jiao Tong University \hologo{XeLaTeX} Thesis Template
      \end{block}
      \vspace{-8pt}
      \begin{alertblock}{}
        \scriptsize
        \begin{tabular}{cl}
          \faStar & 2.4k \\
          \faEye & 55 \\
          \faCodeBranch & 701 \\
        \end{tabular}
      \end{alertblock}
    \end{column}
  \end{columns}
\end{frame}

\begin{frame}
  \frametitle{下载与编译}
  \alert{下载} 推荐安装 Git \link{https://git-scm.com/} 后,克隆 SJTUG 镜像仓库
  \begin{exampleblock}{\faGit*}
    \ttfamily\small
    git clone https://mirror.sjtu.edu.cn/git/SJTUThesis.git/
  \end{exampleblock}

  \alert{编译} 推荐使用 \pkg{latexmk} 编译\footnote{\hologo{MiKTeX} 用户需要手动安装 Perl 解释器 \link{https://www.perl.org/get.html} 才能使用 \pkg{latexmk}。},在不能够利用自带的 \texttt{.latexmkrc} 配置文件的情况下,需要查清楚在对应的编辑器中如何使用 \hologo{XeLaTeX} + \hologo{biber} 编译 \link{https://github.com/sjtug/SJTUThesis/blob/master/README.md}。
  \begin{exampleblock}{\faTerminal}
    \ttfamily\small
    latexmk -xelatex main
  \end{exampleblock}

  Overleaf 用户可以下载压缩包后,上传并采用 \hologo{XeLaTeX} 编译方式。
\end{frame}

\begin{frame}
  \frametitle{手动编译}
  \alert{第一次编译失败} 如果没有办法通过通常方式编译成功,请尝试使用文件夹内附带 \faLinux{}\,\faApple{} \texttt{Makefile} 和 \faWindows{} \texttt{Compile.bat} 进行编译。

  \alert{统计字数} 编写过程中也可以使用对应的命令调用 \TeX{}count 来统计正文字数。
  \begin{columns}
    \begin{column}{0.38\textwidth}
      \begin{exampleblock}{\faLinux{}\,\faApple}
        \ttfamily
        make all\\
        make clean\\
        make cleanall\\
        make wordcount
      \end{exampleblock}
    \end{column}
    \begin{column}{0.38\textwidth}
      \begin{exampleblock}{\faWindows}
        \ttfamily
        ./Compile.bat thesis\\
        ./Compile.bat clean\\
        ./Compile.bat cleanall\\
        ./Compile.bat wordcount
      \end{exampleblock}
    \end{column}
    \begin{column}{0.24\textwidth}
      \begin{block}{\faInfo}
        \ttfamily
        编译论文\\
        清理中间文件\\
        $\hookrightarrow +$删除论文\\
        统计字数
      \end{block}
    \end{column}
  \end{columns}
\end{frame}

\begin{frame}[label=compile]
  \frametitle{编译问题排查}
  \begin{columns}
    \begin{column}{0.33\textwidth}
      \begin{alertblock}{无法使用 \texttt{latexmk}\thesisissue{578}}
        \hologo{MiKTeX} 需要安装 Perl 解释器。
      \end{alertblock}  
      \begin{alertblock}{C\TeX{} 套装无法编译\thesisissue{446}}
        使用最新 \TeX{} 发行版。
      \end{alertblock}
      \begin{alertblock}{\hologo{pdfLaTeX} 无法编译\thesisissue{444}}
        请使用 \texttt{latexmk},或更改编辑器设置以 \hologo{XeLaTeX} 编译。
      \end{alertblock}
    \end{column}
    \begin{column}{0.33\textwidth}
      \begin{alertblock}{缺少字体\thesisissue{564} \thesisdiscuss{598}}
        更换字体集,或者安装对应字体。
      \end{alertblock}
      \begin{alertblock}{缺少汉字\thesisissue{533} \thesisdiscuss{617}}
        去除使用 fandol 字体集的设定。或者是安装字体后,改用 \texttt{fontset=adobe} 或 \texttt{fontset=founder}。
      \end{alertblock}
    \end{column}
    \begin{column}{0.33\textwidth}
      \begin{block}{\faInfoCircle{} README}
        不同编辑器的设置请首先参阅 README \link{https://github.com/sjtug/SJTUThesis/blob/master/README.md} 文档。
      \end{block}
      \begin{block}{\faBookOpen{} Wiki}
        其他编译问题推荐查阅 Wiki \link{https://github.com/sjtug/SJTUThesis/wiki} 的使用说明部分。
      \end{block}
    \end{column}
  \end{columns}
\end{frame}

\begin{frame}[fragile, label=covers]
  \begin{codeblock}[firstnumber=3]{main.tex}
|\alert{\% 载入 SJTUThesis 模版}|
\documentclass[|\only<1>{\highlight{type}}\only<2>{type}|=|\only<1>{bachelor}\only<2>{\highlight{bachelor}}|]{sjtuthesis}
  \end{codeblock}
  \begin{figure}
    \parbox{0.9\textwidth}{
      \begin{subfigure}{0.20\textwidth}
        \framebox{\includegraphics[width=\linewidth]{support/thesis/bachelor}}
        \caption{\only<1>{学士}\only<2>{\texttt{bachelor}}}
      \end{subfigure}\hfill
      \begin{subfigure}{0.20\textwidth}
        \framebox{\includegraphics[width=\linewidth]{support/thesis/master}}
        \caption{\only<1>{硕士}\only<2>{\texttt{master}}}
      \end{subfigure}\hfill
      \begin{subfigure}{0.20\textwidth}
        \framebox{\includegraphics[width=\linewidth]{support/thesis/doctor}}
        \caption{\only<1>{博士}\only<2>{\texttt{doctor}}}
      \end{subfigure}\hfill
      \begin{subfigure}{0.20\textwidth}
        \framebox{\includegraphics[width=\linewidth]{support/thesis/course}}
        \caption{\only<1>{课程}\only<2>{\texttt{course}}}
      \end{subfigure}
      \caption{论文类型示例\only<2>{ \texttt{type}}}
    }
  \end{figure}
\end{frame}

\begin{frame}[fragile]
  \frametitle{文档类选项}
  % \framesubtitle{\textbackslash{}documentclass\{sjtuthesis\}}
  \begin{columns}
    \begin{column}{0.45\textwidth}
      \includegraphics[page=10]{thesisdir}
    \end{column}
    \begin{column}{0.55\textwidth}
      \begin{table}[H]
        \caption{文档类选项}
        \footnotesize
        \begin{tabular}{>{\ttfamily}rll}
          \toprule
          选项 & 含义 & 相关 \\
          \midrule
          type= & 指定论文类型 & 第 \ref{covers} 页\\
          fontset= & 指定字体 & 第 \ref{compile} 页\\
          \midrule
          review & 开启盲审模式 & \thesisissue{195} \thesisissue{686} \\
          twoside & 双页模式 & \thesisissue{554} \\
          oneside & 单页模式 & \thesisissue{694} \\
          openright & 章从奇数页开始 & \thesisdiscuss{724} \\
          openany & 章从任意页开始 & \thesisissue{446} \\
          \bottomrule
        \end{tabular}
      \end{table}
    \end{column}
  \end{columns}
\end{frame}

\begin{frame}[fragile]
  \frametitle{基本配置}
  \framesubtitle{\textbackslash{}input\{setup\}}
  \begin{columns}
    \begin{column}{0.45\textwidth}
      \includegraphics[page=9]{thesisdir}
    \end{column}
    \begin{column}{0.55\textwidth}
      \begin{codeblock}[firstnumber=12]{main.tex}
|\highlightline<1>|% 论文基本配置,加载宏包等全局配置
|\highlightline<1>|\input{setup}

\begin{document}

%TC:ignore

|\highlightline<2>|% 标题页
|\highlightline<2>|\maketitle
      \end{codeblock}
      \visible<2>{
        \cmd{sjtusetup} 中的 \pkg{info} 将会修改封面的信息设置(见第 \ref{covers} 页)。
      }
    \end{column}
  \end{columns}
\end{frame}

\begin{frame}[fragile]
  \frametitle{基本配置}
  \framesubtitle{\textbackslash{}sjtusetup}
  \begin{columns}
    \begin{column}{0.45\textwidth}
      \includegraphics[page=1]{thesisdir}
    \end{column}
    \begin{column}{0.55\textwidth}
      \begin{codeblock}[firstnumber=3]{setup.tex}
\sjtusetup{
  info = {
    title    = {||上海交通大学学位论文 \LaTeX{} 模板示例文档},
    title*   = {A Sample for \LaTeX-based SJTU Thesis Template},
    author   = {||某\quad{}某},
    author* = {Mo Mo},
  },
  style = { header-logo-color = red, 
  },
  name = {
    publications = {||攻读学位期间完成的论文},
  },
}
      \end{codeblock}
    \end{column}
  \end{columns}
\end{frame}

\begin{frame}
  \frametitle{基本配置}
  \framesubtitle{\textbackslash{}sjtusetup}
  \begin{columns}
    \begin{column}{0.45\textwidth}
      \includegraphics[page=1]{thesisdir}
    \end{column}
    \begin{column}{0.55\textwidth}
      \begin{table}[H]
        \centering
        \caption{info 域}
        \footnotesize
        \begin{tabular}{lll} \toprule
          命令作用 & 中文对应选项 & 英文对应选项 \\ \midrule
          论文标题 & \texttt{title} & \texttt{title*} \\
          关键字列表 & \texttt{keywords} & \texttt{keywords*} \\
          作者姓名&  \texttt{author} &\texttt{author*}\\
          申请学位名称 & \texttt{degree}&\texttt{degree*}\\
          院系名称 & \texttt{department} & \texttt{department*}\\
          专业名称 & \texttt{major} & \texttt{major*}\\
          导师 & \texttt{supervisor} & \texttt{supervisor*}\\
          副导师 & \texttt{assisupervisor} & \texttt{assisupervisor*}\\
          日期 & \multicolumn{2}{c}{\texttt{date}}\\
          学号 & \multicolumn{2}{c}{\texttt{id}}\\ \bottomrule
          \end{tabular}
      \end{table}
    \end{column}
  \end{columns}
\end{frame}

\begin{frame}[fragile]
  \frametitle{版权页}
  \framesubtitle{\textbackslash{}copyrightpage}
  \begin{columns}
    \begin{column}{0.45\textwidth}
      \only<1>{
        \includegraphics[page=9]{thesisdir}
      }
      \only<2>{
        \includegraphics[page=2]{thesisdir}
      }
      \only<3>{
        \begin{figure}[H]
          \framebox{\includegraphics[page=2,width=0.4\linewidth]{bachelor}}
          \caption{版权页}
        \end{figure}
      }
    \end{column}
    \begin{column}{0.55\textwidth}
      \begin{codeblock}[firstnumber=22]{main.tex}
|\highlightline<1>|% 原创性声明及使用授权书
|\highlightline<1>|\copyrightpage
|\highlightline<2>|% 插入外置原创性声明及使用授权书
|\highlightline<2>|% \copyrightpage[scans/sample-copyright-old.pdf]
      \end{codeblock}
      \only<1>{
        \cmd{copyrightpages} 可以用于插入版权页。
      }
      \only<2>{
        \cmd{copyrightpages} 也接受一个可选参数,用于直接使用扫描件。
      }
    \end{column}
  \end{columns}
\end{frame}

\begin{frame}[fragile]
  \frametitle{前置部分}
  \framesubtitle{\textbackslash{}frontmatter}
  \begin{columns}
    \begin{column}{0.45\textwidth}
      \only<1>{
        \includegraphics[page=9]{thesisdir}
      }
      \only<2>{
        \includegraphics[page=3]{thesisdir}
      }
      \only<3>{
        \begin{figure}[H]
          \begin{subfigure}{0.45\textwidth}
            \framebox{\includegraphics[page=3,width=\linewidth]{bachelor}}
            \caption{中文}
          \end{subfigure}\hfill
          \begin{subfigure}{0.45\textwidth}
            \framebox{\includegraphics[page=4,width=\linewidth]{bachelor}}
            \caption{英文}
          \end{subfigure}
          \caption{摘要}
        \end{figure}
      }
      \only<4>{
        \begin{figure}[H]
          \begin{subfigure}{0.30\linewidth}
            \centering
            \framebox{\includegraphics[page=5,width=0.6\linewidth]{bachelor}}
            \caption{目录}
          \end{subfigure}
          \begin{subfigure}{0.30\linewidth}
            \centering
            \framebox{\includegraphics[page=6,width=0.6\linewidth]{bachelor}}
            \caption{插图}
          \end{subfigure}

          \begin{subfigure}{0.30\linewidth}
            \centering
            \framebox{\includegraphics[page=7,width=0.6\linewidth]{bachelor}}
            \caption{表格}
          \end{subfigure}
          \begin{subfigure}{0.30\linewidth}
            \centering
            \framebox{\includegraphics[page=8,width=0.6\linewidth]{bachelor}}
            \caption{算法}
          \end{subfigure}
          \caption{索引}
        \end{figure}
      }
      \only<5>{
        \includegraphics[page=4]{thesisdir}
      }
      \only<6>{
        \begin{figure}[H]
          \framebox{\includegraphics[page=9,width=0.5\linewidth]{bachelor}}
          \caption{符号对照表}
        \end{figure}
      }
    \end{column}
    \begin{column}{0.55\textwidth}
      \begin{codeblock}[firstnumber=30]{main.tex}
|\highlightline<2-3>|% 摘要
|\highlightline<2-3>|\input{contents/abstract}

|\highlightline<4>|% 目录
|\highlightline<4>|\tableofcontents
|\highlightline<4>|% 插图索引
|\highlightline<4>|\listoffigures*
|\highlightline<4>|% 表格索引
|\highlightline<4>|\listoftables*
|\highlightline<4>|% 算法索引
|\highlightline<4>|\listofalgorithms*

|\highlightline<5-6>|% 符号对照表
|\highlightline<5-6>|\input{contents/nomenclature}
      \end{codeblock}
    \end{column}
  \end{columns}
\end{frame}

\begin{frame}[fragile]
  \frametitle{主体部分}
  \framesubtitle{\textbackslash{}mainmatter}
  \begin{columns}
    \begin{column}{0.45\textwidth}
      \only<1>{
        \includegraphics[page=5]{thesisdir}
      }
      \only<2>{
        \begin{figure}[H]
          \begin{subfigure}{0.30\linewidth}
            \centering
            \framebox{\includegraphics[page=11,width=0.6\linewidth]{bachelor}}
            \caption{简介}
          \end{subfigure}
          \begin{subfigure}{0.30\linewidth}
            \centering
            \framebox{\includegraphics[page=13,width=0.6\linewidth]{bachelor}}
            \caption{数学}
          \end{subfigure}

          \begin{subfigure}{0.30\linewidth}
            \centering
            \framebox{\includegraphics[page=16,width=0.6\linewidth]{bachelor}}
            \caption{浮动体}
          \end{subfigure}
          \begin{subfigure}{0.30\linewidth}
            \centering
            \framebox{\includegraphics[page=22,width=0.6\linewidth]{bachelor}}
            \caption{总结}
          \end{subfigure}
          \caption{主体部分}
        \end{figure}
      }
    \end{column}
    \begin{column}{0.55\textwidth}
      \begin{codeblock}[firstnumber=47]{main.tex}
|\highlightline|% 正文内容
|\highlightline|% !TeX root = ../../../latex-talk.tex

\section{是什么}

\begin{frame}
  \frametitle{\TeX{}}
  \begin{columns}[c]
    \begin{column}{0.7\textwidth}
      \begin{center}
        \rmfamily\Huge
        \highlight[structure]{\TeX{}}
      \end{center}
      \begin{center}
        \parbox{0.75\textwidth}{
          \TeX{} 是由斯坦福大学教授高德纳
          (Donald E.~Knuth)于 1977 年开始开发的排版引擎。目前仍在更新,最新版本号为 3.141592653 \link{https://tug.org/TUGboat/tb42-1/tb130knuth-tuneup21.pdf}。
        }
      \end{center}
    \end{column}
    \begin{column}{0.3\textwidth}
      \includegraphics[width=.8\columnwidth]{support/images/Knuth.jpg}
    \end{column}
  \end{columns}
  \note{\emph{这一部分背景介绍大家可以了解一下,暂时跳过。}
  \LaTeX{} 这个词由两个部分组成,\hologo{La} 和 \TeX{}。那我们首先了解一下 \TeX{} 是什么。
  \TeX{} 是由斯坦福大学的教授高德纳于 1977 年开始开发的排版引擎,它已经有三十多年的历史了,
  目前仍在更新,版本号(3.141592653)将会趋近于 $\pi$ 的取值,高德纳最近还在给 \textsl{TUGBoat} 写稿子
  \link{https://tug.org/TUGboat/tb42-1/tb130knuth-tuneup21.pdf},
  关于 \TeX{} 今年又做了哪些改进。}
\end{frame}

\begin{frame}
  \frametitle{\LaTeX{}}
  \begin{columns}[c]
    \begin{column}{0.7\textwidth}
      \begin{center}
        \rmfamily\Huge
        \highlight[structure]{\LaTeX{}}
      \end{center}
      \begin{center}
        \parbox{0.75\textwidth}{
          \LaTeX{} 是最早在 1985 年由现就职于微软的 Leslie Lamport 开发的一种 \TeX{} \textbf{格式}\footnotemark,使用一些列宏和扩展宏包来简化 \TeX{} 的使用。现在由 \LaTeX{} Project 的成员维护。现在广泛使用的版本是 \LaTeXe{},最新的版本为 \LaTeX3(2020 年 10 月后默认内置)。
        }
      \end{center}
    \end{column}
    \begin{column}{0.3\textwidth}
      \includegraphics[width=.8\columnwidth]{support/images/Lamport.jpg}
    \end{column}
  \end{columns}
  \footnotetext{\hologo{ConTeXt} 也是一种 \TeX{} 格式 \link{https://www.contextgarden.net/}。}
  \note{\emph{这一部分的背景介绍大家可以了解一下,暂时跳过。}
  \LaTeX{} 是最早由现就职于微软的 Leslie Lamport 开发的一种 \TeX{} 格式(与其对标的是
  \hologo{ConTeXt}\link{https://www.contextgarden.net/}),主要也是为了简化 \TeX{} 的使用。
  现在主要由 \LaTeX{} 开发组维护,现在广泛使用的版本是 \LaTeXe{},最新的版本为 \LaTeX3,
  在 2020 年 10 月后默认内置,所以要尽可能使用较新的发行版,以充分发挥其功能。}
\end{frame}

\begin{frame}
  \frametitle{程序}
  \begin{columns}[c]
    \begin{column}{0.7\textwidth}
      \begin{center}
        \rmfamily\Huge
        \highlight[structure]{\hologo{pdfLaTeX}}
      \end{center}
      \begin{center}
        \parbox{0.7\textwidth}{
          \hologo{pdfLaTeX} 是为了编译一个 \LaTeX{} 文档而运行的程序。实际上底层在运行一个叫 \hologo{pdfTeX} 的引擎,并预装了对应的 \LaTeX{} \textbf{格式}。为了利用临时文件,可能就需要多次运行程序。
        }
      \end{center}
    \end{column}
    \begin{column}{0.3\textwidth}
      \begin{block}{}
        \ttfamily\small
        > \highlight{pdflatex} main.tex\\
        This is pdfTeX, Version 3.141592653-
        2.6-1.40.23 (MiKTeX 21.10)\\
        entering extended mode\\
        \highlight{LaTeX2e} <2021-11-15>\\
        \highlight{L3} programming layer <2021-11-22>
      \end{block}
    \end{column}
  \end{columns}
  \note{\hologo{pdfLaTeX} 是为了编译一个 \LaTeX{} 文档而运行的程序。}
\end{frame}

% \begin{frame}
%   \frametitle{引擎}
%   \begin{columns}[c]
%     \begin{column}{0.7\textwidth}
%       \begin{center}
%         \rmfamily\Huge
%         \highlight[structure!70]{pdf}\hologo{La}\highlight[structure!70]{\TeX{}}
%       \end{center}
%       \begin{center}
%         \parbox{0.7\textwidth}{
%           pdf\TeX{} 是编译 \TeX{} 文档(以 \texttt{.tex} 结尾)的\textbf{引擎}---可以理解 \TeX{} 指令的\textbf{程序}。
%         }
%       \end{center}
%     \end{column}
%     \begin{column}{0.3\textwidth}
%       \begin{block}{}
%         \ttfamily\small
%         > pdflatex main.tex\\
%         This is \highlight[structure!70]{pdfTeX}, Version 3.141592653-
%         2.6-1.40.23 (MiKTeX 21.10)
%         entering extended mode\\
%         LaTeX2e <2021-11-15>\\
%         L3 programming layer <2021-11-22>
%       \end{block}
%     \end{column}
%   \end{columns}
%   \note{实际上底层在运行一个叫 \hologo{pdfTeX} 的引擎,并预装了对应的 \LaTeX{} 格式。}
% \end{frame}

\begin{frame}[label={frame:engine}]
  \frametitle{程序}
  \begin{table}
    \caption{主流 \hologo{(La)TeX} 程序
    \footnote{(u)p\TeX{} 是日语最常用的引擎,生成 \texttt{.dvi},支持 Unicode。}\footnote{Ap\TeX{} \link{https://github.com/clerkma/ptex-ng} 具有底层 CJK 支持,内联 Ruby,Color Emoji。}}
    \footnotesize
    \begin{stampbox}
      \begin{tabular}{c>{\raggedright}*{3}{p{3.5cm}}}
        \alert{引擎}     & \hologo{pdfTeX}   & \hologo{XeTeX}   & \hologo{LuaTeX}   \\
        \alert{程序}     & \hologo{pdfLaTeX} & \hologo{XeLaTeX} & \hologo{LuaLaTeX} \\
        \alert{特点}     & 直接生成 PDF,支持 micro-typography  & 支持 Unicode、OpenType 与复杂文字编排 (CTL) & 支持 Unicode,内联 Lua,支持 OpenType \\
      \end{tabular}
    \end{stampbox}
  \end{table}

  \begin{center}
    \parbox{.9\textwidth}{
      \hologo{pdfLaTeX} 不支持 Unicode。为了排版中文,大部分情况下应当使用 \hologo{XeLaTeX},而 \hologo{LuaLaTeX} 速度相对较慢。\faWindows{} 可以在一些情况下使用 \hologo{pdfLaTeX}。
    }
  \end{center}
  \note{当然为了排版中文,已经不再推荐使用 \hologo{pdfLaTeX} 了,应该使用
  \hologo{XeLaTeX} 或者 \hologo{LuaLaTeX},当然后者的速度还是相对较慢,
  它们支持 Unicode 编码,并可以使用 OpenType 字体的全部功能。
  当然 \faWindows{} 平台下在某些追求速度的情况下,
  还是可以试着使用 \hologo{pdfLaTeX} 的。

  \hologo{LuaLaTeX} 理想情况下不慢,但是使用一些宏包后会破坏理想状态,
  也会因配置产生不同的结果,不同的操作系统在 I/O 速度上的不同也会导致不同的时间。

  \hologo{pdfLaTeX} 也支持,只不过需要先生成 tfm \TeX{} 字体度量文件,后续使用 \TeX{}
  自身的配置方法,只能使用 7 比特或 8 比特字体。}
\end{frame}

% \begin{frame}
%   \paragraph{\hologo{pdfLaTeX}} \TeX{} 和 \LaTeX{} 被广泛使用之前,它们只需内置支持欧洲语言即可。在 Unicode 出现之前,\LaTeX{} 提供了许多种\textbf{文件编码}来允许很多语言的文字以原生的方式输入,\hologo{pdfLaTeX} 也只需要使用 8 位文件编码和 8 位字体。
% \end{frame}


|\highlightline|\input{contents/math_and_citations}
|\highlightline|\input{contents/floats}
|\highlightline|\input{contents/summary}

%TC:ignore

% 参考文献
\printbibliography[heading=bibintoc]
      \end{codeblock}
    \end{column}
  \end{columns}
\end{frame}

\begin{frame}
  \frametitle{数学}
  \begin{itemize}
    \item 公式示例:\nolinkurl{contents/math_and_citations.tex}
    \item \SJTUThesis{} 定义了常用的数学环境(需要手工引入 \texttt{ntheorem} 宏包):
      \begin{table}[h]
        \centering
        \footnotesize
        \begin{tabular}{*{7}{l}}\toprule
          assumption  & axiom   & conjecture & corollary    & definition  & example   & exercise  \\
          假设        & 公理    & 猜想       & 推论         & 定义        & 例        & 练习      \\\midrule
          lemma       & problem & proof      & proposition  & remark      & solution  & theorem   \\
          引理        & 问题    & 证明       & 命题         & 注          & 解        & 定理      \\\bottomrule
        \end{tabular}
      \end{table}
      \item \SJTUThesis{} 可以通过 \texttt{unimath} 选项使用 \pkg{unicode-math} 进行数学输入,注意与传统方式的区别。\thesisissue{555}
  \end{itemize}
\end{frame}

\begin{frame}[fragile]
  \frametitle{参考文献}
  \begin{columns}
    \begin{column}{0.45\textwidth}
      \includegraphics[page=6]{thesisdir}
    \end{column}
    \begin{column}{0.55\textwidth}
      \begin{codeblock}[firstnumber=111,numbersep=2pt]{setup.tex}
% 使用 BibLaTeX 处理参考文献
%   biblatex-gb7714-2015 常用选项
%     gbnamefmt=lowercase     姓名大小写由输入信息确定
%     gbpub=false             禁用出版信息缺失处理
\usepackage[backend=biber,style=gb7714-2015]{biblatex}
% 文献表字体
% \renewcommand{\bibfont}{\zihao{-5}}
% 文献表条目间的间距
\setlength{\bibitemsep}{0pt}
|\highlightline|% 导入参考文献数据库
|\highlightline|\addbibresource{bibdata/thesis.bib}
      \end{codeblock}
    \end{column}
  \end{columns}
\end{frame}

\begin{frame}[fragile]
  \frametitle{附录}
  \framesubtitle{\textbackslash{}appendix}
  \begin{columns}
    \begin{column}{0.45\textwidth}
      \only<1>{
        \includegraphics[page=7]{thesisdir}
      }
      \only<2>{
        \begin{figure}[H]
          \begin{subfigure}{0.45\linewidth}
            \framebox{\includegraphics[width=\linewidth,page=24]{bachelor}}
            \caption{}
          \end{subfigure}\hfill
          \begin{subfigure}{0.45\textwidth}
            \framebox{\includegraphics[width=\linewidth,page=25]{bachelor}}
            \caption{}
          \end{subfigure}
          \caption{附录}
        \end{figure}
      }
    \end{column}
    \begin{column}{0.55\textwidth}
      \begin{codeblock}[firstnumber=61]{main.tex}
% 附录中图表不加入索引
\captionsetup{list=no}

% 附录内容
|\highlightline|\input{contents/app_maxwell_equations}
|\highlightline|\input{contents/app_flow_chart}
      \end{codeblock}
    \end{column}
  \end{columns}
\end{frame}

\begin{frame}[fragile]
  \frametitle{结尾部分}
  \framesubtitle{\textbackslash{}backmatter}
  \begin{columns}
    \begin{column}{0.45\textwidth}
      \only<1>{
        \includegraphics[page=8]{thesisdir}
      }
      \only<2>{
        \begin{figure}[H]
          \begin{subfigure}{0.30\linewidth}
            \centering
            \framebox{\includegraphics[page=26,width=0.6\linewidth]{bachelor}}
            \caption{致谢}
          \end{subfigure}
          \begin{subfigure}{0.30\linewidth}
            \centering
            \framebox{\includegraphics[page=27,width=0.6\linewidth]{bachelor}}
            \caption{成就}
          \end{subfigure}

          \begin{subfigure}{0.30\linewidth}
            \centering
            \framebox{\includegraphics[page=28,width=0.6\linewidth]{bachelor}}
            \caption{简历}
          \end{subfigure}
          \begin{subfigure}{0.30\linewidth}
            \centering
            \framebox{\includegraphics[page=29,width=0.6\linewidth]{bachelor}}
            \caption{大摘要*}
          \end{subfigure}
          \caption{结尾部分}
        \end{figure}
      }
    \end{column}
    \begin{column}{0.55\textwidth}
      \begin{codeblock}[firstnumber=76]{main.tex}
% 致谢
\input{contents/acknowledgements}

% 发表论文及科研成果
% 盲审论文中,发表论文及科研成果等仅以第几作者注明即可,不要出现作者或他人姓名
\input{contents/achievements}

% 简历
\input{contents/resume}

% 学士学位论文要求在最后有一个大摘要,单独编页码
\input{contents/digest}
      \end{codeblock}
    \end{column}
  \end{columns}
\end{frame}

\begin{frame}
  \frametitle{还有其他问题?}
  \begin{columns}
    \begin{column}{0.75\textwidth}
    \begin{itemize}
      \item[{\faComment*[regular]}] 日常模板或 \LaTeX{} 使用问题可以前往 Discussions \link{https://github.com/sjtug/SJTUThesis/discussions} 提问
      
      (解决后别忘了 \textcolor{green}{\faCheckCircle{} Mark as answer}
      \item[{\faDotCircle[regular]}] 如果是 \textsc{SJTUThesis} 项目本身的 bug 和 feature request
      
      可以通过 Issues \link{https://github.com/sjtug/SJTUThesis/issues} 反馈。
      \item[{\faCodeBranch}] 如果你有好点子,可以贡献代码
     
      向 \textsc{SJTU\TeX{}}(v1) \link{https://github.com/sjtug/SJTUTeX/tree/v1} 存储库发 PR,\par
      而后把解包结果同步到 \textsc{SJTUThesis}。
  
      \item[{\faTag}] 如果你对正在基于 \LaTeX3 开发的新版感兴趣,\par
      也欢迎向 \textsc{SJTU\TeX{}}(v2) \link{https://github.com/sjtug/SJTUTeX/tree/v2} 发 PR。
  
      \item[{\faQq}] 也欢迎在 QQ 群即时讨论。
    \end{itemize}
    \end{column}
    \begin{column}{0.25\textwidth}
      \includegraphics[height=0.7\textheight]{qq.jpg}
    \end{column}
  \end{columns}
\end{frame}
\end{document}
      \end{codeblock}
    \end{column}
  \end{columns}
  \note<3>{当然,如果需要不换页插入源代码就不用 \cmd{include} 了,
  因为这最主要的好处在于能够在组建大型文档的时候,得到当前页码、编号的进度信息。
  在插入小部件时,还是推荐使用 \cmd{input},这个命令不会额外地产生 \texttt{.aux} 文件,
  对于 I/O 反应慢的(说的就是 \faWindows{})比较友好。}
\end{frame}

\begin{frame}[fragile]
  \frametitle{组织文档}
  \begin{columns}
    \begin{column}{0.4\textwidth}
      \begin{codeblock}[]{learnlatex.tex}
|\highlightline|\chapter{|\phantom{}|学习 \LaTeX{}}
\section{|\phantom{}|概念}
\subsection{\LaTeX{}}
\LaTeX{} 是一个用以排版高质量作品的文档准备系统。
      \end{codeblock}
      子文件中就不需要添加 \env{document} 环境了\footnotemark。
    \end{column}
    \begin{column}{0.6\textwidth}
      \begin{codeblock}[]{主文档 main.tex}
|\highlightline|\documentclass{ctexrep}
\includeonly{learnlatex,sjtuthesis}
\begin{document}
  \tableofcontents
  % !TeX root = ..\..\latex-talk.tex

\part{学习 \LaTeX{}}
% FIXME: Part Page miniframe overflow
% FIXME: footnote fault numbering

\begin{frame}[plain]
  \vfil
  \begin{center}
    \href{https://learnlatex.org}{
      \rmfamily
      Learn\,\lower1ex\hbox{\Huge\LaTeX{}}.org
    }
  \end{center}
  \vfil
  \begin{center}
    \parbox{0.75\linewidth}{
      Learn\LaTeX{}.org\cite{learnlatex} 提供了解 \LaTeX{} 的 16 篇简短的教程,并包含了一些可以在线运行的示例,可以通过亲自动手查看实验效果。本部分主要参考由 C\TeX{}-org 提供的中文翻译版本 \link{https://github.com/CTeX-org/learnlatex.github.io/tree/zh-Hans/zh-Hans/}。
    }
  \end{center}
  \vfil
\end{frame}

{ % Start of shaded number logo

\newcommand{\shadedfont}[2][1pt]{
  % #1 (optional): shadow distance
  % #2: the text needed to be shaded
  \hbox{\rlap{\color{gray}\hskip#1#2}#2}
}
\newcounter{learnsec}
\setcounter{learnsec}{-1}
\newcommand{\updatelogo}{
  % update the logo corresponding to current counter.
  \stepcounter{learnsec}
  \logo{
    \raise.3ex\hbox{\tiny\insertsection}\shadedfont{\arabic{learnsec}}
  }
}
\let\oldsection=\section
\renewcommand{\section}[1]{\oldsection{#1}\updatelogo}

\section{是什么}
\begin{frame}
  \frametitle{\TeX{}}
  \begin{columns}[c]
    \begin{column}{0.7\textwidth}
      \begin{center}
        \rmfamily\Huge
        \hologo{La}\highlight[structure!70]{\TeX{}}
      \end{center}
      \begin{center}
        \parbox{0.75\textwidth}{
          \TeX{} 是由斯坦福大学教授高德纳
          (Donald E.~Knuth)于 1977 年开始开发的排版引擎。目前仍在更新,最新版本号为 3.141592653 \link{https://tug.org/TUGboat/tb42-1/tb130knuth-tuneup21.pdf}。
        }
      \end{center}
    \end{column}
    \begin{column}{0.3\textwidth}
      \includegraphics[width=.8\columnwidth]{Knuth.jpg}
    \end{column}
  \end{columns}
\end{frame}

\begin{frame}
  \frametitle{\LaTeX{}}
  \begin{columns}[c]
    \begin{column}{0.7\textwidth}
      \begin{center}
        \rmfamily\Huge
        \highlight[structure]{\LaTeX{}}
      \end{center}
      \begin{center}
        \parbox{0.75\textwidth}{
          \LaTeX{} 是最早在 1985 年由现就职于微软的 Leslie Lamport 开发的一种 \TeX{} \textbf{格式}\footnotemark,使用一些列宏和扩展宏包来简化 \TeX{} 的使用。现在由 \LaTeX{} Project 的成员维护。现在广泛使用的版本是 \LaTeXe{},最新的版本为 \LaTeX3(2020 年 10 月后默认内置)。
        }
      \end{center}
    \end{column}
    \begin{column}{0.3\textwidth}
      \includegraphics[width=.8\columnwidth]{Lamport.jpg}
    \end{column}
  \end{columns}
  \footnotetext{\hologo{ConTeXt} 也是一种 \TeX{} 格式 \link{https://www.contextgarden.net/}。}
\end{frame}

\begin{frame}
  \frametitle{程序}
  \begin{columns}[c]
    \begin{column}{0.7\textwidth}
      \begin{center}
        \rmfamily\Huge
        \highlight[structure]{\hologo{pdfLaTeX}}
      \end{center}
      \begin{center}
        \parbox{0.7\textwidth}{
          \hologo{pdfLaTeX} 是为了编译一个 \LaTeX{} 文档而运行的程序。实际上底层在运行一个叫 \hologo{pdfTeX} 的引擎,并预装了对应的 \LaTeX{} \textbf{格式}。为了利用临时文件,可能就需要多次运行程序。
        }
      \end{center}
    \end{column}
    \begin{column}{0.3\textwidth}
      \begin{block}{}
        \ttfamily\small
        > \highlight{pdflatex} main.tex\\
        This is pdfTeX, Version 3.141592653-
        2.6-1.40.23 (MiKTeX 21.10)\\
        entering extended mode\\
        \highlight{LaTeX2e} <2021-11-15>\\
        \highlight{L3} programming layer <2021-11-22>
      \end{block}
    \end{column}
  \end{columns}
\end{frame}

\begin{frame}
  \frametitle{引擎}
  \begin{columns}[c]
    \begin{column}{0.7\textwidth}
      \begin{center}
        \rmfamily\Huge
        \highlight[structure!70]{pdf}\hologo{La}\highlight[structure!70]{\TeX{}}
      \end{center}
      \begin{center}
        \parbox{0.7\textwidth}{
          pdf\TeX{} 是编译 \TeX{} 文档(以 \texttt{.tex} 结尾)的\textbf{引擎}---可以理解 \TeX{} 指令的\textbf{程序}。
        }
      \end{center}
    \end{column}
    \begin{column}{0.3\textwidth}
      \begin{block}{}
        \ttfamily\small
        > pdflatex main.tex\\
        This is \highlight[structure!70]{pdfTeX}, Version 3.141592653-
        2.6-1.40.23 (MiKTeX 21.10)
        entering extended mode\\
        LaTeX2e <2021-11-15>\\
        L3 programming layer <2021-11-22>
      \end{block}
    \end{column}
  \end{columns}
\end{frame}

\begin{frame}
  \frametitle{Unicode 引擎}
  \begin{table}
    \caption{主流 \hologo{(La)TeX} 程序
    \footnote{(u)p\TeX{} 是日语最常用的引擎,生成 \texttt{.dvi},支持 Unicode。}\footnote{Ap\TeX{} 具有底层 CJK 支持,内联 Ruby,Color Emoji。}}
    \footnotesize
    \begin{stampbox}
      \begin{tabular}{c>{\raggedright}*{3}{p{3.5cm}}}
        \alert{引擎}     & \hologo{pdfTeX}   & \hologo{XeTeX}   & \hologo{LuaTeX}   \\
        \alert{程序}     & \hologo{pdfLaTeX} & \hologo{XeLaTeX} & \hologo{LuaLaTeX} \\
        \alert{特点}     & 直接生成 PDF,支持 micro-typography  & 支持 Unicode、OpenType 与复杂文字编排 (CTL) & 支持 Unicode,内联 Lua,支持 OpenType \\
      \end{tabular}
    \end{stampbox}
  \end{table}

  \begin{center}
    \parbox{.9\textwidth}{
      \hologo{pdfLaTeX} 不支持 Unicode。为了排版中文,大部分情况下 \faApple{}\,\faLinux{} 应当使用 \hologo{XeLaTeX},而 \hologo{LuaLaTeX} 速度相对较慢。\faWindows{} 可以在一些情况下使用 \hologo{pdfLaTeX}。
    }
  \end{center}
\end{frame}

% \begin{frame}
%   \paragraph{\hologo{pdfLaTeX}} \TeX{} 和 \LaTeX{} 被广泛使用之前,它们只需内置支持欧洲语言即可。在 Unicode 出现之前,\LaTeX{} 提供了许多种\textbf{文件编码}来允许很多语言的文字以原生的方式输入,\hologo{pdfLaTeX} 也只需要使用 8 位文件编码和 8 位字体。
% \end{frame}

\section{跑起来}
\begin{frame}
  \frametitle{发行版}
  \begin{table}
    \caption{\hologo{TeX} 发行版}
    \footnotesize
    \begin{stampbox}
      \begin{tabular}{c>{\raggedright}*{3}{p{3.2cm}}}
        \alert{发行版}     & \hologo{MiKTeX} \link{https://mirrors.sjtug.sjtu.edu.cn/ctan/systems/win32/miktex/setup/windows-x64/basic-miktex-21.12-x64.exe}   & \TeX{} Live \link{https://mirrors.sjtug.sjtu.edu.cn/ctan/systems/texlive/tlnet/install-tl.zip}   & Mac\TeX{} \link{https://mirrors.sjtug.sjtu.edu.cn/ctan/systems/mac/mactex/mactex-20210328.pkg}  \\[2pt]
        \alert{特点}      &  只安装必要文件,依赖用时更新  &  所有平台均可使用,每年发布一次 & Mac 系统专用,对 \TeX{} Live 的进一步打包 \\
        \alert{推荐平台}  & \faWindows  & \faLinux &  \faApple \\
      \end{tabular}
    \end{stampbox}
  \end{table}
  \begin{center}
    \parbox{.9\textwidth}{
      要让 \LaTeX{} 跑起来,核心就是要有一套 \TeX{} 发行版,来获取让 \LaTeX{} 工作所需的一组程序和文件。参考《一份简短的关于 \LaTeX{} 安装的介绍》\link{https://mirrors.sjtug.sjtu.edu.cn/ctan/info/install-latex-guide-zh-cn/install-latex-guide-zh-cn.pdf} 安装想使用的发行版。推荐使用发行版的最新版本\footnote{老版本 Linux 系统的包管理器自带 \TeX{} Live 发行版可能不是最新的 \link{https://repology.org/project/texlive/versions},尽量使用镜像安装,并手动将相关环境变量添加到路径 \link{https://www.tug.org/texlive/doc/texlive-zh-cn/texlive-zh-cn.pdf}。},并使用国内镜像。
    }
  \end{center}
\end{frame}

\begin{frame}[plain]
  \hbox to \textwidth{
    \hfil
    \vbox to 3cm{
      \hbox{
        \resizebox{3cm}{!}{\includegraphics{\getcontribpath{sjtug}{vi/sjtug.pdf}}}
      }
    }
    \hfil
    \vbox to 3cm{
      \vfill
      \hbox{\Large\bfseries\color{cprimary} 稳定、快速、现代的镜像服务。}
      \vskip2pt
      \hbox{托管于华东教育网骨干节点上海交通大学。}
      \vfill
    }
    \hskip20pt
    \hfil
  }

  \begin{center}
    \parbox{0.8\textwidth}{
      推荐使用 SJTUG 软件镜像服务,离得近,下得快。
      
      \begin{description}
        \footnotesize
        \item[\TeX{} Live]  {\ttfamily tlmgr option repository https://mirrors.sjtug.sjtu.edu.cn/CTAN/systems/texlive/tlnet}
        \item[\hologo{MiKTeX}] 在 \hologo{MiKTeX} Console 中设置镜像源为 \url{https://mirrors.sjtug.sjtu.edu.cn}
      \end{description}
    }
  \end{center}
\end{frame}

\begin{frame}
  \frametitle{编辑器}
  \begin{table}
    \caption{开源编辑器推荐}
    \footnotesize
    \begin{stampbox}
      \begin{tabular}{c>{\raggedright}*{3}{p{3.5cm}}}
        \alert{编辑器}     & \begin{tabular}{c}Visual Studio Code\\ \LaTeX{} Workshop\end{tabular}  & \TeX{}studio & \TeX{}works \\[5pt]
        \alert{特点}      &  搭配 VS Code 使用非常方便,易扩展  & 可以使用大量的菜单选项输入代码块,用户友好 & 只提供基础的高亮与运行方法,发行版自带\footnote{Mac\TeX{} 打包的是 \TeX{}Shop 编辑器。} \\
      \end{tabular}
    \end{stampbox}
  \end{table}
  \begin{center}
    \parbox{.9\textwidth}{
      使用专为 \LaTeX{} 设计的编辑器将带来更多便利,因为它们往往会提供一键编译、内置 PDF 阅读器以及语法高亮等功能。几乎所有现代的 \LaTeX{} 编辑器都提供 Sync\TeX{} 这一强大的功能,以在 PDF 和 代码间对应跳转。
    }
  \end{center}
\end{frame}

\begin{frame}
  \frametitle{在线平台}
  \begin{table}
    \caption{在线协作平台推荐}
    \footnotesize
    \begin{stampbox}
      \begin{tabular}{c>{\raggedright}*{2}{p{4cm}}}
        \alert{在线平台}     & Overleaf \link{https://www.overleaf.com/}  & 交大 \LaTeX{} 助手 \link{https://latex.sjtu.edu.cn/} \\[2pt]
        \alert{特点}      & 最流行的在线平台,提供大量的模板,但国内访问慢 & 校内平台,隐私保护有保障,共享项目限制少 \\
      \end{tabular}
    \end{stampbox}
  \end{table}
  \begin{center}
    \parbox{.9\textwidth}{
      在线平台允许你直接在网页中编辑文档,无需本地安装即可在后台运行 \LaTeX{},并显示生成的 PDF。可以参照 Overleaf 官方文档学习如何使用在线平台 \link{https://www.overleaf.com/learn}。
    }
  \end{center}
\end{frame}

\section{基本结构}
\begin{frame}[fragile]%
  \frametitle{文档部件}
  \begin{columns}[c]
    \begin{column}{0.4\textwidth}
      \only<1>{
        \cmd{documentclass} 命令加载了\textbf{文档类}。\pkg{article} 是由 \LaTeX{}提供的用于排版短文档的基本文档类。
        \begin{description}
          \footnotesize
          \item[\pkg{article}] 不包含章的短文档
          \item[\pkg{report}] 含有章的单面印刷文档
          \item[\pkg{book}] 含有章的双面印刷文档
          \item[\pkg{beamer}] 制作幻灯片
        \end{description}
      }
      \only<2>{
        \env{document} 环境用于指示文档主体的范围。\LaTeX{} 还有其他的使用 \cmd{begin} 和 \cmd{end} 的搭配,我们称这些为\textbf{环境}。它们将用来设定局部格式命令\footnotemark。
      }
      \only<3>{
        \includepdflarge{enminimal}
      }
    \end{column}
    \begin{column}{0.6\textwidth}
      \begin{codeblock}[]{排版英文最简示例}
|\only<1>{\highlightline}|\documentclass{article}
|\only<2>{\highlightline}|\begin{document}
|\only<3>{\highlightline}|  Together for a Shared Future
|\only<2>{\highlightline}|\end{document}
      \end{codeblock}
    \end{column}
  \end{columns}
  \only<2>{\footnotetext{环境实际上是一个组,只不过通过语义化的形式预装了对应的格式命令。普通的组可以直接使用一对大括号之间的内容 \{$\cdots$\} 表示。}}
\end{frame}

\section{扩展}
\begin{frame}[fragile]%
  \frametitle{中文排版}
  \begin{columns}[c]
    \begin{column}{0.4\textwidth}
      \only<1>{
        \cmd{usepackage} 用于使用宏包以向 \LaTeX{} 添加或修改功能,需要在\textbf{导言区}调用。
        这里使用 \pkg{ctex} 宏集以获得中文支持。其调用底层因随不同的引擎而不同。
        {
          \footnotesize
          \begin{stampbox}
            \begin{tabular}{c*{3}{c}}
              \alert{引擎}     & \hologo{pdfTeX}   & \hologo{XeTeX}   & \hologo{LuaTeX}   \\
              \alert{程序}     & \hologo{pdfLaTeX} & \hologo{XeLaTeX} & \hologo{LuaLaTeX} \\
              \alert{宏包}     & CJK\footnotemark & xeCJK & luatexja \\
              \alert{封装}     & \multicolumn{3}{c}{ctex} \\
            \end{tabular}
          \end{stampbox}
        }
        \vspace{-1cm}
      }
      \only<2>{
        C\TeX{} 建议对于之前提到的常规文档类,最佳实践是使用该宏集提供的四种中文文档类,以对特定类型提供额外的中文排版适配。
        \begin{center}
          \begin{stampbox}
            \footnotesize
            \begin{tabular}{cc}
              \pkg{ctexart} & \pkg{ctexrep} \\
              \pkg{ctexbook} & \pkg{ctexbeamer} \\
            \end{tabular}
          \end{stampbox}
        \end{center}
      }
      \only<3>{
        \includepdflarge{cnminimal}
      }
      \only<4>{
        大部分情况下,你都不应当在 \LaTeX{} 中强制断行:你几乎只是想另起一段,或者是想在段落之间添加空行(使用 \pkg{parskip} 宏包就可实现)。
        只有\alert{很少的}情况下你需要使用 \textbackslash{}\textbackslash{} 来另起一行而不另起一段。
      }
    \end{column}
    \begin{column}{0.6\textwidth}
      \begin{codeblock}[]{排版中文\only<2->{最佳实践}}
|\only<2>{\highlightline}|\documentclass{|\only<1>{article}\only<2->{ctexart}|}
|\only<1>{\highlightline\textbackslash{}usepackage\{ctex\}\hfill\color{cprimary}\% 导言区}|
\begin{document}
|\only<3>{\highlightline}|    一起向未来
|\only<4>{\highlightline}|
  Together for a Shared Future
\end{document}
      \end{codeblock}
    \end{column}
  \end{columns}
  \only<1>{\footnotetext{ctex 在 \faApple\,\faLinux{} 上已经不可以使用 \hologo{pdfLaTeX} 编译,以及在 \faWindows{} 上使用该引擎也会变更自动间距调整等功能的默认行为。}}
\end{frame}

\section{设定格式}
\begin{frame}[fragile]%
  \frametitle{字体样式}
  \begin{columns}
    \begin{column}{0.4\textwidth}
      \only<1>{
        \includepdflarge{fontstyle}
      }
      \only<2>{
        可以使用显示样式设定命令对小段做加粗、斜体、等宽等等的处理。
        \begin{center}
          \footnotesize
          \begin{stampbox}
            \begin{tabular}{rl}
              \cmd{textrm} & \textrm{衬线} \\
              \cmd{textbf} & \textbf{加粗} \\
              \cmd{textit} & \kaishu 斜体 \\
              \cmd{texttt} & \texttt{等宽} \\
              \cmd{textsf} & \textsf{无衬线} \\
              \cmd{textsc} & \textsc{Small Caps} \\
              \cmd{textsl} & \textsl{Slanted} \\
            \end{tabular}
          \end{stampbox}
        \end{center}
      }
      \only<3>{
        与 Word 不同的是,\LaTeX{} 一般情况下并不需要使用上面的显式命令,而是采用逻辑标记的方法,
        比如 \cmd{emph} 可以强调文字,以及下面将要提到的目次命令(第 \ref{sectioning} 页)。
        这样可以统一管理格式。
      }
    \end{column}
    \begin{column}{0.6\textwidth}
      \begin{codeblock}[]{样式}
\documentclass{ctexart}
\begin{document}
|\only<2>{\highlightline}|  \textbf{||一起向未来}

|\only<3>{\highlightline}|  \emph{Together for a Shared Future}
\end{document}
      \end{codeblock}
    \end{column}
  \end{columns}
\end{frame}

\begin{frame}[fragile]%
  \frametitle{\only<1-2>{字体大小}\only<3>{字体样式}}
  \begin{columns}
    \begin{column}{0.4\textwidth}
      \only<1>{
        \includepdflarge{fontsize}
      }
      \only<2>{
        同样地,你也可以显式地设定字体大小,但是这种命令会更改行文设置,所以需要使用一个组来限定作用范围\footnotemark。
        \begin{center}
          \footnotesize
          \begin{stampbox}
            \begin{tabular}{rl}
              \cmd{tiny} & \tiny 极小 \\
              \cmd{scriptsize} & \scriptsize 抄本大小  \\
              \cmd{footnotesize} & \footnotesize 脚注大小 \\
              \cmd{small} & \small 小 \\
              \cmd{normalsize} & \normalsize 正常大小 \\
              \cmd{large} & \large 大 \\
              \cmd{huge} & \Huge 巨大 \\
            \end{tabular}
          \end{stampbox}
        \end{center}
      }
      \only<3>{
        也可以使用字体样式对应的更改字体设置的命令,这对于大段文段的设置而言也是很方便的。
        \begin{center}
          \footnotesize
          \begin{stampbox}
            \begin{tabular}{ll}
              \cmd{textrm} & \cmd{rmfamily}\\
              \cmd{texttt} & \cmd{ttfamily}\\
              \cmd{textsf} & \cmd{sffamily}\\
              \cmd{textbf} & \cmd{bfseries}\\
              \cmd{textit} & \cmd{itshape}\\
              \cmd{textsc} & \cmd{scshape}\\
              \cmd{textsl} & \cmd{slshape}\\
            \end{tabular}
          \end{stampbox}
        \end{center}
      }
    \end{column}
    \begin{column}{0.6\textwidth}
      \begin{codeblock}[]{大小}
\documentclass{ctexart}
\begin{document}
|\only<2>{\highlightline}|  {\bfseries\Large 一起向未来\par}
|\only<3>{\highlightline}|  {\itshape Together for a Shared Future}
\end{document}
      \end{codeblock}
    \end{column}
  \end{columns}
  \only<2>{\footnotetext{注意最后显式地使用 \cmd{par} 在改回大小前结束该段,否则会导致下一行的行间距异常!}}
\end{frame}

\section{逻辑结构}
\begin{frame}[fragile]
  \frametitle{列表}
  \begin{columns}
    \begin{column}{0.35\textwidth}
      \begin{codeblock}[]{无序列表}
\documentclass{ctexart}
\begin{document}
|\highlightline|  \begin{itemize}
    \item 第一项
    \item 第二项
    \item 第三项
|\highlightline|  \end{itemize}
\end{document}
      \end{codeblock}
    \end{column}
    \begin{column}{0.35\textwidth}
      \begin{codeblock}[]{有序列表}
\documentclass{ctexart}
\begin{document}
|\highlightline|  \begin{enumerate}
    \item 第一项
    \item 第二项
    \item 第三项
|\highlightline|  \end{enumerate}
\end{document}
      \end{codeblock}
    \end{column}
    \begin{column}{0.35\textwidth}
      \begin{codeblock}[]{描述列表}
\documentclass{ctexart}
\begin{document}
|\highlightline|  \begin{description}
    \item[||第一] 文本
    \item[||第二] 文本
    \item[||第三] 文本  
|\highlightline|  \end{description}
\end{document}
      \end{codeblock}
    \end{column}
  \end{columns}
\end{frame}

%更深的列表技巧,定理环境等

\begin{frame}
  \frametitle{列表}
  \begin{columns}
    \begin{column}{0.35\textwidth}
      \includepdflarge{unordered}
    \end{column}
    \begin{column}{0.35\textwidth}
      \includepdflarge{ordered}
    \end{column}
    \begin{column}{0.35\textwidth}
      \includepdflarge{description}
    \end{column}
  \end{columns}
\end{frame}

\begin{frame}[fragile,label=sectioning]%
  \frametitle{目次结构}
  \begin{columns}
    \begin{column}{0.4\textwidth}
      \LaTeX{} 可以使用目次命令将文档划分层级\footnotemark,并自动设定对应字体样式和大小。
      \begin{center}
        \begin{stampbox}
          \footnotesize
          \begin{tabular}{rll}
           命令 & 中文 & 层次 \\
           \cmd{chapter} & 章\footnotemark & \sout{0} \\
           \cmd{section} & 节 & 1 \\
           \cmd{subsection} & 小节 & 2 \\
           \cmd{subsubsection} & 小小节 & 3 \\
          \end{tabular}
        \end{stampbox}
      \end{center}
    \end{column}
    \begin{column}{0.6\textwidth}
      \begin{codeblock}[]{目次}
\documentclass{ctexart}
\begin{document}
|\highlightline|  \section{||概念}
|\highlightline|  \subsection{\LaTeX{}}
  \LaTeX{} 是一个用以排版高质量作品的文档准备系统。
\end{document}
      \end{codeblock}
    \end{column}
  \end{columns}
  \footnotetext{章这一级只在 \pkg{report} 和 \pkg{book} 文档类(包括对应的中文文档类)有定义。还有不常用的 \cmd{part} (0@\pkg{article}/-1@\pkg{report}\&\pkg{book}\&\pkg{beamer}) 以及更低层次的 \cmd{paragraph} (4) 与 \cmd{subparagraph} (5)。 }
\end{frame}

\begin{frame}[fragile]%
  \frametitle{组织文档}
  \begin{columns}
    \begin{column}{0.4\textwidth}
      \only<1>{
        \cmd{tableofcontents} 用来生成对于目次命令的目录。如果你想设定显示到哪个层级,在这个命令前使用 \cmd{setcounter\{tocdepth\}\{层次\}}
      }
      \only<2>{
        对于大型文档而言,使用多个文件管理源文件通常是更方便的。而 \cmd{include} 和 \cmd{input} 都以相对路径的方式插入其他 \TeX{} 文档。
        区别在于,\cmd{include} 命令会从新页开始并做一些内部调整,这基本上只对 \pkg{chapter} 这一级有用。而 \cmd{input} 会原样插入源代码。
      }
      \only<3>{
        但是 \cmd{include} 插入的文档可以使用 \cmd{includeonly} 管理当前要排印哪一部分的内容,利用所有章节辅助文件的同时,减少编译时间并专注于该部分的内容。
      }
    \end{column}
    \begin{column}{0.6\textwidth}
      \begin{codeblock}[]{主文档}
\documentclass{ctexrep}
|\only<3>{\highlightline}|\includeonly{learnlatex,sjtuthesis}
\begin{document}
|\only<1>{\highlightline}|  \tableofcontents
|\only<2-3>{\highlightline}|  % !TeX root = ..\..\latex-talk.tex

\part{学习 \LaTeX{}}
% FIXME: Part Page miniframe overflow
% FIXME: footnote fault numbering

\begin{frame}[plain]
  \vfil
  \begin{center}
    \href{https://learnlatex.org}{
      \rmfamily
      Learn\,\lower1ex\hbox{\Huge\LaTeX{}}.org
    }
  \end{center}
  \vfil
  \begin{center}
    \parbox{0.75\linewidth}{
      Learn\LaTeX{}.org\cite{learnlatex} 提供了解 \LaTeX{} 的 16 篇简短的教程,并包含了一些可以在线运行的示例,可以通过亲自动手查看实验效果。本部分主要参考由 C\TeX{}-org 提供的中文翻译版本 \link{https://github.com/CTeX-org/learnlatex.github.io/tree/zh-Hans/zh-Hans/}。
    }
  \end{center}
  \vfil
\end{frame}

{ % Start of shaded number logo

\newcommand{\shadedfont}[2][1pt]{
  % #1 (optional): shadow distance
  % #2: the text needed to be shaded
  \hbox{\rlap{\color{gray}\hskip#1#2}#2}
}
\newcounter{learnsec}
\setcounter{learnsec}{-1}
\newcommand{\updatelogo}{
  % update the logo corresponding to current counter.
  \stepcounter{learnsec}
  \logo{
    \raise.3ex\hbox{\tiny\insertsection}\shadedfont{\arabic{learnsec}}
  }
}
\let\oldsection=\section
\renewcommand{\section}[1]{\oldsection{#1}\updatelogo}

\section{是什么}
\begin{frame}
  \frametitle{\TeX{}}
  \begin{columns}[c]
    \begin{column}{0.7\textwidth}
      \begin{center}
        \rmfamily\Huge
        \hologo{La}\highlight[structure!70]{\TeX{}}
      \end{center}
      \begin{center}
        \parbox{0.75\textwidth}{
          \TeX{} 是由斯坦福大学教授高德纳
          (Donald E.~Knuth)于 1977 年开始开发的排版引擎。目前仍在更新,最新版本号为 3.141592653 \link{https://tug.org/TUGboat/tb42-1/tb130knuth-tuneup21.pdf}。
        }
      \end{center}
    \end{column}
    \begin{column}{0.3\textwidth}
      \includegraphics[width=.8\columnwidth]{Knuth.jpg}
    \end{column}
  \end{columns}
\end{frame}

\begin{frame}
  \frametitle{\LaTeX{}}
  \begin{columns}[c]
    \begin{column}{0.7\textwidth}
      \begin{center}
        \rmfamily\Huge
        \highlight[structure]{\LaTeX{}}
      \end{center}
      \begin{center}
        \parbox{0.75\textwidth}{
          \LaTeX{} 是最早在 1985 年由现就职于微软的 Leslie Lamport 开发的一种 \TeX{} \textbf{格式}\footnotemark,使用一些列宏和扩展宏包来简化 \TeX{} 的使用。现在由 \LaTeX{} Project 的成员维护。现在广泛使用的版本是 \LaTeXe{},最新的版本为 \LaTeX3(2020 年 10 月后默认内置)。
        }
      \end{center}
    \end{column}
    \begin{column}{0.3\textwidth}
      \includegraphics[width=.8\columnwidth]{Lamport.jpg}
    \end{column}
  \end{columns}
  \footnotetext{\hologo{ConTeXt} 也是一种 \TeX{} 格式 \link{https://www.contextgarden.net/}。}
\end{frame}

\begin{frame}
  \frametitle{程序}
  \begin{columns}[c]
    \begin{column}{0.7\textwidth}
      \begin{center}
        \rmfamily\Huge
        \highlight[structure]{\hologo{pdfLaTeX}}
      \end{center}
      \begin{center}
        \parbox{0.7\textwidth}{
          \hologo{pdfLaTeX} 是为了编译一个 \LaTeX{} 文档而运行的程序。实际上底层在运行一个叫 \hologo{pdfTeX} 的引擎,并预装了对应的 \LaTeX{} \textbf{格式}。为了利用临时文件,可能就需要多次运行程序。
        }
      \end{center}
    \end{column}
    \begin{column}{0.3\textwidth}
      \begin{block}{}
        \ttfamily\small
        > \highlight{pdflatex} main.tex\\
        This is pdfTeX, Version 3.141592653-
        2.6-1.40.23 (MiKTeX 21.10)\\
        entering extended mode\\
        \highlight{LaTeX2e} <2021-11-15>\\
        \highlight{L3} programming layer <2021-11-22>
      \end{block}
    \end{column}
  \end{columns}
\end{frame}

\begin{frame}
  \frametitle{引擎}
  \begin{columns}[c]
    \begin{column}{0.7\textwidth}
      \begin{center}
        \rmfamily\Huge
        \highlight[structure!70]{pdf}\hologo{La}\highlight[structure!70]{\TeX{}}
      \end{center}
      \begin{center}
        \parbox{0.7\textwidth}{
          pdf\TeX{} 是编译 \TeX{} 文档(以 \texttt{.tex} 结尾)的\textbf{引擎}---可以理解 \TeX{} 指令的\textbf{程序}。
        }
      \end{center}
    \end{column}
    \begin{column}{0.3\textwidth}
      \begin{block}{}
        \ttfamily\small
        > pdflatex main.tex\\
        This is \highlight[structure!70]{pdfTeX}, Version 3.141592653-
        2.6-1.40.23 (MiKTeX 21.10)
        entering extended mode\\
        LaTeX2e <2021-11-15>\\
        L3 programming layer <2021-11-22>
      \end{block}
    \end{column}
  \end{columns}
\end{frame}

\begin{frame}
  \frametitle{Unicode 引擎}
  \begin{table}
    \caption{主流 \hologo{(La)TeX} 程序
    \footnote{(u)p\TeX{} 是日语最常用的引擎,生成 \texttt{.dvi},支持 Unicode。}\footnote{Ap\TeX{} 具有底层 CJK 支持,内联 Ruby,Color Emoji。}}
    \footnotesize
    \begin{stampbox}
      \begin{tabular}{c>{\raggedright}*{3}{p{3.5cm}}}
        \alert{引擎}     & \hologo{pdfTeX}   & \hologo{XeTeX}   & \hologo{LuaTeX}   \\
        \alert{程序}     & \hologo{pdfLaTeX} & \hologo{XeLaTeX} & \hologo{LuaLaTeX} \\
        \alert{特点}     & 直接生成 PDF,支持 micro-typography  & 支持 Unicode、OpenType 与复杂文字编排 (CTL) & 支持 Unicode,内联 Lua,支持 OpenType \\
      \end{tabular}
    \end{stampbox}
  \end{table}

  \begin{center}
    \parbox{.9\textwidth}{
      \hologo{pdfLaTeX} 不支持 Unicode。为了排版中文,大部分情况下 \faApple{}\,\faLinux{} 应当使用 \hologo{XeLaTeX},而 \hologo{LuaLaTeX} 速度相对较慢。\faWindows{} 可以在一些情况下使用 \hologo{pdfLaTeX}。
    }
  \end{center}
\end{frame}

% \begin{frame}
%   \paragraph{\hologo{pdfLaTeX}} \TeX{} 和 \LaTeX{} 被广泛使用之前,它们只需内置支持欧洲语言即可。在 Unicode 出现之前,\LaTeX{} 提供了许多种\textbf{文件编码}来允许很多语言的文字以原生的方式输入,\hologo{pdfLaTeX} 也只需要使用 8 位文件编码和 8 位字体。
% \end{frame}

\section{跑起来}
\begin{frame}
  \frametitle{发行版}
  \begin{table}
    \caption{\hologo{TeX} 发行版}
    \footnotesize
    \begin{stampbox}
      \begin{tabular}{c>{\raggedright}*{3}{p{3.2cm}}}
        \alert{发行版}     & \hologo{MiKTeX} \link{https://mirrors.sjtug.sjtu.edu.cn/ctan/systems/win32/miktex/setup/windows-x64/basic-miktex-21.12-x64.exe}   & \TeX{} Live \link{https://mirrors.sjtug.sjtu.edu.cn/ctan/systems/texlive/tlnet/install-tl.zip}   & Mac\TeX{} \link{https://mirrors.sjtug.sjtu.edu.cn/ctan/systems/mac/mactex/mactex-20210328.pkg}  \\[2pt]
        \alert{特点}      &  只安装必要文件,依赖用时更新  &  所有平台均可使用,每年发布一次 & Mac 系统专用,对 \TeX{} Live 的进一步打包 \\
        \alert{推荐平台}  & \faWindows  & \faLinux &  \faApple \\
      \end{tabular}
    \end{stampbox}
  \end{table}
  \begin{center}
    \parbox{.9\textwidth}{
      要让 \LaTeX{} 跑起来,核心就是要有一套 \TeX{} 发行版,来获取让 \LaTeX{} 工作所需的一组程序和文件。参考《一份简短的关于 \LaTeX{} 安装的介绍》\link{https://mirrors.sjtug.sjtu.edu.cn/ctan/info/install-latex-guide-zh-cn/install-latex-guide-zh-cn.pdf} 安装想使用的发行版。推荐使用发行版的最新版本\footnote{老版本 Linux 系统的包管理器自带 \TeX{} Live 发行版可能不是最新的 \link{https://repology.org/project/texlive/versions},尽量使用镜像安装,并手动将相关环境变量添加到路径 \link{https://www.tug.org/texlive/doc/texlive-zh-cn/texlive-zh-cn.pdf}。},并使用国内镜像。
    }
  \end{center}
\end{frame}

\begin{frame}[plain]
  \hbox to \textwidth{
    \hfil
    \vbox to 3cm{
      \hbox{
        \resizebox{3cm}{!}{\includegraphics{\getcontribpath{sjtug}{vi/sjtug.pdf}}}
      }
    }
    \hfil
    \vbox to 3cm{
      \vfill
      \hbox{\Large\bfseries\color{cprimary} 稳定、快速、现代的镜像服务。}
      \vskip2pt
      \hbox{托管于华东教育网骨干节点上海交通大学。}
      \vfill
    }
    \hskip20pt
    \hfil
  }

  \begin{center}
    \parbox{0.8\textwidth}{
      推荐使用 SJTUG 软件镜像服务,离得近,下得快。
      
      \begin{description}
        \footnotesize
        \item[\TeX{} Live]  {\ttfamily tlmgr option repository https://mirrors.sjtug.sjtu.edu.cn/CTAN/systems/texlive/tlnet}
        \item[\hologo{MiKTeX}] 在 \hologo{MiKTeX} Console 中设置镜像源为 \url{https://mirrors.sjtug.sjtu.edu.cn}
      \end{description}
    }
  \end{center}
\end{frame}

\begin{frame}
  \frametitle{编辑器}
  \begin{table}
    \caption{开源编辑器推荐}
    \footnotesize
    \begin{stampbox}
      \begin{tabular}{c>{\raggedright}*{3}{p{3.5cm}}}
        \alert{编辑器}     & \begin{tabular}{c}Visual Studio Code\\ \LaTeX{} Workshop\end{tabular}  & \TeX{}studio & \TeX{}works \\[5pt]
        \alert{特点}      &  搭配 VS Code 使用非常方便,易扩展  & 可以使用大量的菜单选项输入代码块,用户友好 & 只提供基础的高亮与运行方法,发行版自带\footnote{Mac\TeX{} 打包的是 \TeX{}Shop 编辑器。} \\
      \end{tabular}
    \end{stampbox}
  \end{table}
  \begin{center}
    \parbox{.9\textwidth}{
      使用专为 \LaTeX{} 设计的编辑器将带来更多便利,因为它们往往会提供一键编译、内置 PDF 阅读器以及语法高亮等功能。几乎所有现代的 \LaTeX{} 编辑器都提供 Sync\TeX{} 这一强大的功能,以在 PDF 和 代码间对应跳转。
    }
  \end{center}
\end{frame}

\begin{frame}
  \frametitle{在线平台}
  \begin{table}
    \caption{在线协作平台推荐}
    \footnotesize
    \begin{stampbox}
      \begin{tabular}{c>{\raggedright}*{2}{p{4cm}}}
        \alert{在线平台}     & Overleaf \link{https://www.overleaf.com/}  & 交大 \LaTeX{} 助手 \link{https://latex.sjtu.edu.cn/} \\[2pt]
        \alert{特点}      & 最流行的在线平台,提供大量的模板,但国内访问慢 & 校内平台,隐私保护有保障,共享项目限制少 \\
      \end{tabular}
    \end{stampbox}
  \end{table}
  \begin{center}
    \parbox{.9\textwidth}{
      在线平台允许你直接在网页中编辑文档,无需本地安装即可在后台运行 \LaTeX{},并显示生成的 PDF。可以参照 Overleaf 官方文档学习如何使用在线平台 \link{https://www.overleaf.com/learn}。
    }
  \end{center}
\end{frame}

\section{基本结构}
\begin{frame}[fragile]%
  \frametitle{文档部件}
  \begin{columns}[c]
    \begin{column}{0.4\textwidth}
      \only<1>{
        \cmd{documentclass} 命令加载了\textbf{文档类}。\pkg{article} 是由 \LaTeX{}提供的用于排版短文档的基本文档类。
        \begin{description}
          \footnotesize
          \item[\pkg{article}] 不包含章的短文档
          \item[\pkg{report}] 含有章的单面印刷文档
          \item[\pkg{book}] 含有章的双面印刷文档
          \item[\pkg{beamer}] 制作幻灯片
        \end{description}
      }
      \only<2>{
        \env{document} 环境用于指示文档主体的范围。\LaTeX{} 还有其他的使用 \cmd{begin} 和 \cmd{end} 的搭配,我们称这些为\textbf{环境}。它们将用来设定局部格式命令\footnotemark。
      }
      \only<3>{
        \includepdflarge{enminimal}
      }
    \end{column}
    \begin{column}{0.6\textwidth}
      \begin{codeblock}[]{排版英文最简示例}
|\only<1>{\highlightline}|\documentclass{article}
|\only<2>{\highlightline}|\begin{document}
|\only<3>{\highlightline}|  Together for a Shared Future
|\only<2>{\highlightline}|\end{document}
      \end{codeblock}
    \end{column}
  \end{columns}
  \only<2>{\footnotetext{环境实际上是一个组,只不过通过语义化的形式预装了对应的格式命令。普通的组可以直接使用一对大括号之间的内容 \{$\cdots$\} 表示。}}
\end{frame}

\section{扩展}
\begin{frame}[fragile]%
  \frametitle{中文排版}
  \begin{columns}[c]
    \begin{column}{0.4\textwidth}
      \only<1>{
        \cmd{usepackage} 用于使用宏包以向 \LaTeX{} 添加或修改功能,需要在\textbf{导言区}调用。
        这里使用 \pkg{ctex} 宏集以获得中文支持。其调用底层因随不同的引擎而不同。
        {
          \footnotesize
          \begin{stampbox}
            \begin{tabular}{c*{3}{c}}
              \alert{引擎}     & \hologo{pdfTeX}   & \hologo{XeTeX}   & \hologo{LuaTeX}   \\
              \alert{程序}     & \hologo{pdfLaTeX} & \hologo{XeLaTeX} & \hologo{LuaLaTeX} \\
              \alert{宏包}     & CJK\footnotemark & xeCJK & luatexja \\
              \alert{封装}     & \multicolumn{3}{c}{ctex} \\
            \end{tabular}
          \end{stampbox}
        }
        \vspace{-1cm}
      }
      \only<2>{
        C\TeX{} 建议对于之前提到的常规文档类,最佳实践是使用该宏集提供的四种中文文档类,以对特定类型提供额外的中文排版适配。
        \begin{center}
          \begin{stampbox}
            \footnotesize
            \begin{tabular}{cc}
              \pkg{ctexart} & \pkg{ctexrep} \\
              \pkg{ctexbook} & \pkg{ctexbeamer} \\
            \end{tabular}
          \end{stampbox}
        \end{center}
      }
      \only<3>{
        \includepdflarge{cnminimal}
      }
      \only<4>{
        大部分情况下,你都不应当在 \LaTeX{} 中强制断行:你几乎只是想另起一段,或者是想在段落之间添加空行(使用 \pkg{parskip} 宏包就可实现)。
        只有\alert{很少的}情况下你需要使用 \textbackslash{}\textbackslash{} 来另起一行而不另起一段。
      }
    \end{column}
    \begin{column}{0.6\textwidth}
      \begin{codeblock}[]{排版中文\only<2->{最佳实践}}
|\only<2>{\highlightline}|\documentclass{|\only<1>{article}\only<2->{ctexart}|}
|\only<1>{\highlightline\textbackslash{}usepackage\{ctex\}\hfill\color{cprimary}\% 导言区}|
\begin{document}
|\only<3>{\highlightline}|    一起向未来
|\only<4>{\highlightline}|
  Together for a Shared Future
\end{document}
      \end{codeblock}
    \end{column}
  \end{columns}
  \only<1>{\footnotetext{ctex 在 \faApple\,\faLinux{} 上已经不可以使用 \hologo{pdfLaTeX} 编译,以及在 \faWindows{} 上使用该引擎也会变更自动间距调整等功能的默认行为。}}
\end{frame}

\section{设定格式}
\begin{frame}[fragile]%
  \frametitle{字体样式}
  \begin{columns}
    \begin{column}{0.4\textwidth}
      \only<1>{
        \includepdflarge{fontstyle}
      }
      \only<2>{
        可以使用显示样式设定命令对小段做加粗、斜体、等宽等等的处理。
        \begin{center}
          \footnotesize
          \begin{stampbox}
            \begin{tabular}{rl}
              \cmd{textrm} & \textrm{衬线} \\
              \cmd{textbf} & \textbf{加粗} \\
              \cmd{textit} & \kaishu 斜体 \\
              \cmd{texttt} & \texttt{等宽} \\
              \cmd{textsf} & \textsf{无衬线} \\
              \cmd{textsc} & \textsc{Small Caps} \\
              \cmd{textsl} & \textsl{Slanted} \\
            \end{tabular}
          \end{stampbox}
        \end{center}
      }
      \only<3>{
        与 Word 不同的是,\LaTeX{} 一般情况下并不需要使用上面的显式命令,而是采用逻辑标记的方法,
        比如 \cmd{emph} 可以强调文字,以及下面将要提到的目次命令(第 \ref{sectioning} 页)。
        这样可以统一管理格式。
      }
    \end{column}
    \begin{column}{0.6\textwidth}
      \begin{codeblock}[]{样式}
\documentclass{ctexart}
\begin{document}
|\only<2>{\highlightline}|  \textbf{||一起向未来}

|\only<3>{\highlightline}|  \emph{Together for a Shared Future}
\end{document}
      \end{codeblock}
    \end{column}
  \end{columns}
\end{frame}

\begin{frame}[fragile]%
  \frametitle{\only<1-2>{字体大小}\only<3>{字体样式}}
  \begin{columns}
    \begin{column}{0.4\textwidth}
      \only<1>{
        \includepdflarge{fontsize}
      }
      \only<2>{
        同样地,你也可以显式地设定字体大小,但是这种命令会更改行文设置,所以需要使用一个组来限定作用范围\footnotemark。
        \begin{center}
          \footnotesize
          \begin{stampbox}
            \begin{tabular}{rl}
              \cmd{tiny} & \tiny 极小 \\
              \cmd{scriptsize} & \scriptsize 抄本大小  \\
              \cmd{footnotesize} & \footnotesize 脚注大小 \\
              \cmd{small} & \small 小 \\
              \cmd{normalsize} & \normalsize 正常大小 \\
              \cmd{large} & \large 大 \\
              \cmd{huge} & \Huge 巨大 \\
            \end{tabular}
          \end{stampbox}
        \end{center}
      }
      \only<3>{
        也可以使用字体样式对应的更改字体设置的命令,这对于大段文段的设置而言也是很方便的。
        \begin{center}
          \footnotesize
          \begin{stampbox}
            \begin{tabular}{ll}
              \cmd{textrm} & \cmd{rmfamily}\\
              \cmd{texttt} & \cmd{ttfamily}\\
              \cmd{textsf} & \cmd{sffamily}\\
              \cmd{textbf} & \cmd{bfseries}\\
              \cmd{textit} & \cmd{itshape}\\
              \cmd{textsc} & \cmd{scshape}\\
              \cmd{textsl} & \cmd{slshape}\\
            \end{tabular}
          \end{stampbox}
        \end{center}
      }
    \end{column}
    \begin{column}{0.6\textwidth}
      \begin{codeblock}[]{大小}
\documentclass{ctexart}
\begin{document}
|\only<2>{\highlightline}|  {\bfseries\Large 一起向未来\par}
|\only<3>{\highlightline}|  {\itshape Together for a Shared Future}
\end{document}
      \end{codeblock}
    \end{column}
  \end{columns}
  \only<2>{\footnotetext{注意最后显式地使用 \cmd{par} 在改回大小前结束该段,否则会导致下一行的行间距异常!}}
\end{frame}

\section{逻辑结构}
\begin{frame}[fragile]
  \frametitle{列表}
  \begin{columns}
    \begin{column}{0.35\textwidth}
      \begin{codeblock}[]{无序列表}
\documentclass{ctexart}
\begin{document}
|\highlightline|  \begin{itemize}
    \item 第一项
    \item 第二项
    \item 第三项
|\highlightline|  \end{itemize}
\end{document}
      \end{codeblock}
    \end{column}
    \begin{column}{0.35\textwidth}
      \begin{codeblock}[]{有序列表}
\documentclass{ctexart}
\begin{document}
|\highlightline|  \begin{enumerate}
    \item 第一项
    \item 第二项
    \item 第三项
|\highlightline|  \end{enumerate}
\end{document}
      \end{codeblock}
    \end{column}
    \begin{column}{0.35\textwidth}
      \begin{codeblock}[]{描述列表}
\documentclass{ctexart}
\begin{document}
|\highlightline|  \begin{description}
    \item[||第一] 文本
    \item[||第二] 文本
    \item[||第三] 文本  
|\highlightline|  \end{description}
\end{document}
      \end{codeblock}
    \end{column}
  \end{columns}
\end{frame}

%更深的列表技巧,定理环境等

\begin{frame}
  \frametitle{列表}
  \begin{columns}
    \begin{column}{0.35\textwidth}
      \includepdflarge{unordered}
    \end{column}
    \begin{column}{0.35\textwidth}
      \includepdflarge{ordered}
    \end{column}
    \begin{column}{0.35\textwidth}
      \includepdflarge{description}
    \end{column}
  \end{columns}
\end{frame}

\begin{frame}[fragile,label=sectioning]%
  \frametitle{目次结构}
  \begin{columns}
    \begin{column}{0.4\textwidth}
      \LaTeX{} 可以使用目次命令将文档划分层级\footnotemark,并自动设定对应字体样式和大小。
      \begin{center}
        \begin{stampbox}
          \footnotesize
          \begin{tabular}{rll}
           命令 & 中文 & 层次 \\
           \cmd{chapter} & 章\footnotemark & \sout{0} \\
           \cmd{section} & 节 & 1 \\
           \cmd{subsection} & 小节 & 2 \\
           \cmd{subsubsection} & 小小节 & 3 \\
          \end{tabular}
        \end{stampbox}
      \end{center}
    \end{column}
    \begin{column}{0.6\textwidth}
      \begin{codeblock}[]{目次}
\documentclass{ctexart}
\begin{document}
|\highlightline|  \section{||概念}
|\highlightline|  \subsection{\LaTeX{}}
  \LaTeX{} 是一个用以排版高质量作品的文档准备系统。
\end{document}
      \end{codeblock}
    \end{column}
  \end{columns}
  \footnotetext{章这一级只在 \pkg{report} 和 \pkg{book} 文档类(包括对应的中文文档类)有定义。还有不常用的 \cmd{part} (0@\pkg{article}/-1@\pkg{report}\&\pkg{book}\&\pkg{beamer}) 以及更低层次的 \cmd{paragraph} (4) 与 \cmd{subparagraph} (5)。 }
\end{frame}

\begin{frame}[fragile]%
  \frametitle{组织文档}
  \begin{columns}
    \begin{column}{0.4\textwidth}
      \only<1>{
        \cmd{tableofcontents} 用来生成对于目次命令的目录。如果你想设定显示到哪个层级,在这个命令前使用 \cmd{setcounter\{tocdepth\}\{层次\}}
      }
      \only<2>{
        对于大型文档而言,使用多个文件管理源文件通常是更方便的。而 \cmd{include} 和 \cmd{input} 都以相对路径的方式插入其他 \TeX{} 文档。
        区别在于,\cmd{include} 命令会从新页开始并做一些内部调整,这基本上只对 \pkg{chapter} 这一级有用。而 \cmd{input} 会原样插入源代码。
      }
      \only<3>{
        但是 \cmd{include} 插入的文档可以使用 \cmd{includeonly} 管理当前要排印哪一部分的内容,利用所有章节辅助文件的同时,减少编译时间并专注于该部分的内容。
      }
    \end{column}
    \begin{column}{0.6\textwidth}
      \begin{codeblock}[]{主文档}
\documentclass{ctexrep}
|\only<3>{\highlightline}|\includeonly{learnlatex,sjtuthesis}
\begin{document}
|\only<1>{\highlightline}|  \tableofcontents
|\only<2-3>{\highlightline}|  % !TeX root = ..\..\latex-talk.tex

\part{学习 \LaTeX{}}
% FIXME: Part Page miniframe overflow
% FIXME: footnote fault numbering

\begin{frame}[plain]
  \vfil
  \begin{center}
    \href{https://learnlatex.org}{
      \rmfamily
      Learn\,\lower1ex\hbox{\Huge\LaTeX{}}.org
    }
  \end{center}
  \vfil
  \begin{center}
    \parbox{0.75\linewidth}{
      Learn\LaTeX{}.org\cite{learnlatex} 提供了解 \LaTeX{} 的 16 篇简短的教程,并包含了一些可以在线运行的示例,可以通过亲自动手查看实验效果。本部分主要参考由 C\TeX{}-org 提供的中文翻译版本 \link{https://github.com/CTeX-org/learnlatex.github.io/tree/zh-Hans/zh-Hans/}。
    }
  \end{center}
  \vfil
\end{frame}

{ % Start of shaded number logo

\newcommand{\shadedfont}[2][1pt]{
  % #1 (optional): shadow distance
  % #2: the text needed to be shaded
  \hbox{\rlap{\color{gray}\hskip#1#2}#2}
}
\newcounter{learnsec}
\setcounter{learnsec}{-1}
\newcommand{\updatelogo}{
  % update the logo corresponding to current counter.
  \stepcounter{learnsec}
  \logo{
    \raise.3ex\hbox{\tiny\insertsection}\shadedfont{\arabic{learnsec}}
  }
}
\let\oldsection=\section
\renewcommand{\section}[1]{\oldsection{#1}\updatelogo}

\section{是什么}
\begin{frame}
  \frametitle{\TeX{}}
  \begin{columns}[c]
    \begin{column}{0.7\textwidth}
      \begin{center}
        \rmfamily\Huge
        \hologo{La}\highlight[structure!70]{\TeX{}}
      \end{center}
      \begin{center}
        \parbox{0.75\textwidth}{
          \TeX{} 是由斯坦福大学教授高德纳
          (Donald E.~Knuth)于 1977 年开始开发的排版引擎。目前仍在更新,最新版本号为 3.141592653 \link{https://tug.org/TUGboat/tb42-1/tb130knuth-tuneup21.pdf}。
        }
      \end{center}
    \end{column}
    \begin{column}{0.3\textwidth}
      \includegraphics[width=.8\columnwidth]{Knuth.jpg}
    \end{column}
  \end{columns}
\end{frame}

\begin{frame}
  \frametitle{\LaTeX{}}
  \begin{columns}[c]
    \begin{column}{0.7\textwidth}
      \begin{center}
        \rmfamily\Huge
        \highlight[structure]{\LaTeX{}}
      \end{center}
      \begin{center}
        \parbox{0.75\textwidth}{
          \LaTeX{} 是最早在 1985 年由现就职于微软的 Leslie Lamport 开发的一种 \TeX{} \textbf{格式}\footnotemark,使用一些列宏和扩展宏包来简化 \TeX{} 的使用。现在由 \LaTeX{} Project 的成员维护。现在广泛使用的版本是 \LaTeXe{},最新的版本为 \LaTeX3(2020 年 10 月后默认内置)。
        }
      \end{center}
    \end{column}
    \begin{column}{0.3\textwidth}
      \includegraphics[width=.8\columnwidth]{Lamport.jpg}
    \end{column}
  \end{columns}
  \footnotetext{\hologo{ConTeXt} 也是一种 \TeX{} 格式 \link{https://www.contextgarden.net/}。}
\end{frame}

\begin{frame}
  \frametitle{程序}
  \begin{columns}[c]
    \begin{column}{0.7\textwidth}
      \begin{center}
        \rmfamily\Huge
        \highlight[structure]{\hologo{pdfLaTeX}}
      \end{center}
      \begin{center}
        \parbox{0.7\textwidth}{
          \hologo{pdfLaTeX} 是为了编译一个 \LaTeX{} 文档而运行的程序。实际上底层在运行一个叫 \hologo{pdfTeX} 的引擎,并预装了对应的 \LaTeX{} \textbf{格式}。为了利用临时文件,可能就需要多次运行程序。
        }
      \end{center}
    \end{column}
    \begin{column}{0.3\textwidth}
      \begin{block}{}
        \ttfamily\small
        > \highlight{pdflatex} main.tex\\
        This is pdfTeX, Version 3.141592653-
        2.6-1.40.23 (MiKTeX 21.10)\\
        entering extended mode\\
        \highlight{LaTeX2e} <2021-11-15>\\
        \highlight{L3} programming layer <2021-11-22>
      \end{block}
    \end{column}
  \end{columns}
\end{frame}

\begin{frame}
  \frametitle{引擎}
  \begin{columns}[c]
    \begin{column}{0.7\textwidth}
      \begin{center}
        \rmfamily\Huge
        \highlight[structure!70]{pdf}\hologo{La}\highlight[structure!70]{\TeX{}}
      \end{center}
      \begin{center}
        \parbox{0.7\textwidth}{
          pdf\TeX{} 是编译 \TeX{} 文档(以 \texttt{.tex} 结尾)的\textbf{引擎}---可以理解 \TeX{} 指令的\textbf{程序}。
        }
      \end{center}
    \end{column}
    \begin{column}{0.3\textwidth}
      \begin{block}{}
        \ttfamily\small
        > pdflatex main.tex\\
        This is \highlight[structure!70]{pdfTeX}, Version 3.141592653-
        2.6-1.40.23 (MiKTeX 21.10)
        entering extended mode\\
        LaTeX2e <2021-11-15>\\
        L3 programming layer <2021-11-22>
      \end{block}
    \end{column}
  \end{columns}
\end{frame}

\begin{frame}
  \frametitle{Unicode 引擎}
  \begin{table}
    \caption{主流 \hologo{(La)TeX} 程序
    \footnote{(u)p\TeX{} 是日语最常用的引擎,生成 \texttt{.dvi},支持 Unicode。}\footnote{Ap\TeX{} 具有底层 CJK 支持,内联 Ruby,Color Emoji。}}
    \footnotesize
    \begin{stampbox}
      \begin{tabular}{c>{\raggedright}*{3}{p{3.5cm}}}
        \alert{引擎}     & \hologo{pdfTeX}   & \hologo{XeTeX}   & \hologo{LuaTeX}   \\
        \alert{程序}     & \hologo{pdfLaTeX} & \hologo{XeLaTeX} & \hologo{LuaLaTeX} \\
        \alert{特点}     & 直接生成 PDF,支持 micro-typography  & 支持 Unicode、OpenType 与复杂文字编排 (CTL) & 支持 Unicode,内联 Lua,支持 OpenType \\
      \end{tabular}
    \end{stampbox}
  \end{table}

  \begin{center}
    \parbox{.9\textwidth}{
      \hologo{pdfLaTeX} 不支持 Unicode。为了排版中文,大部分情况下 \faApple{}\,\faLinux{} 应当使用 \hologo{XeLaTeX},而 \hologo{LuaLaTeX} 速度相对较慢。\faWindows{} 可以在一些情况下使用 \hologo{pdfLaTeX}。
    }
  \end{center}
\end{frame}

% \begin{frame}
%   \paragraph{\hologo{pdfLaTeX}} \TeX{} 和 \LaTeX{} 被广泛使用之前,它们只需内置支持欧洲语言即可。在 Unicode 出现之前,\LaTeX{} 提供了许多种\textbf{文件编码}来允许很多语言的文字以原生的方式输入,\hologo{pdfLaTeX} 也只需要使用 8 位文件编码和 8 位字体。
% \end{frame}

\section{跑起来}
\begin{frame}
  \frametitle{发行版}
  \begin{table}
    \caption{\hologo{TeX} 发行版}
    \footnotesize
    \begin{stampbox}
      \begin{tabular}{c>{\raggedright}*{3}{p{3.2cm}}}
        \alert{发行版}     & \hologo{MiKTeX} \link{https://mirrors.sjtug.sjtu.edu.cn/ctan/systems/win32/miktex/setup/windows-x64/basic-miktex-21.12-x64.exe}   & \TeX{} Live \link{https://mirrors.sjtug.sjtu.edu.cn/ctan/systems/texlive/tlnet/install-tl.zip}   & Mac\TeX{} \link{https://mirrors.sjtug.sjtu.edu.cn/ctan/systems/mac/mactex/mactex-20210328.pkg}  \\[2pt]
        \alert{特点}      &  只安装必要文件,依赖用时更新  &  所有平台均可使用,每年发布一次 & Mac 系统专用,对 \TeX{} Live 的进一步打包 \\
        \alert{推荐平台}  & \faWindows  & \faLinux &  \faApple \\
      \end{tabular}
    \end{stampbox}
  \end{table}
  \begin{center}
    \parbox{.9\textwidth}{
      要让 \LaTeX{} 跑起来,核心就是要有一套 \TeX{} 发行版,来获取让 \LaTeX{} 工作所需的一组程序和文件。参考《一份简短的关于 \LaTeX{} 安装的介绍》\link{https://mirrors.sjtug.sjtu.edu.cn/ctan/info/install-latex-guide-zh-cn/install-latex-guide-zh-cn.pdf} 安装想使用的发行版。推荐使用发行版的最新版本\footnote{老版本 Linux 系统的包管理器自带 \TeX{} Live 发行版可能不是最新的 \link{https://repology.org/project/texlive/versions},尽量使用镜像安装,并手动将相关环境变量添加到路径 \link{https://www.tug.org/texlive/doc/texlive-zh-cn/texlive-zh-cn.pdf}。},并使用国内镜像。
    }
  \end{center}
\end{frame}

\begin{frame}[plain]
  \hbox to \textwidth{
    \hfil
    \vbox to 3cm{
      \hbox{
        \resizebox{3cm}{!}{\includegraphics{\getcontribpath{sjtug}{vi/sjtug.pdf}}}
      }
    }
    \hfil
    \vbox to 3cm{
      \vfill
      \hbox{\Large\bfseries\color{cprimary} 稳定、快速、现代的镜像服务。}
      \vskip2pt
      \hbox{托管于华东教育网骨干节点上海交通大学。}
      \vfill
    }
    \hskip20pt
    \hfil
  }

  \begin{center}
    \parbox{0.8\textwidth}{
      推荐使用 SJTUG 软件镜像服务,离得近,下得快。
      
      \begin{description}
        \footnotesize
        \item[\TeX{} Live]  {\ttfamily tlmgr option repository https://mirrors.sjtug.sjtu.edu.cn/CTAN/systems/texlive/tlnet}
        \item[\hologo{MiKTeX}] 在 \hologo{MiKTeX} Console 中设置镜像源为 \url{https://mirrors.sjtug.sjtu.edu.cn}
      \end{description}
    }
  \end{center}
\end{frame}

\begin{frame}
  \frametitle{编辑器}
  \begin{table}
    \caption{开源编辑器推荐}
    \footnotesize
    \begin{stampbox}
      \begin{tabular}{c>{\raggedright}*{3}{p{3.5cm}}}
        \alert{编辑器}     & \begin{tabular}{c}Visual Studio Code\\ \LaTeX{} Workshop\end{tabular}  & \TeX{}studio & \TeX{}works \\[5pt]
        \alert{特点}      &  搭配 VS Code 使用非常方便,易扩展  & 可以使用大量的菜单选项输入代码块,用户友好 & 只提供基础的高亮与运行方法,发行版自带\footnote{Mac\TeX{} 打包的是 \TeX{}Shop 编辑器。} \\
      \end{tabular}
    \end{stampbox}
  \end{table}
  \begin{center}
    \parbox{.9\textwidth}{
      使用专为 \LaTeX{} 设计的编辑器将带来更多便利,因为它们往往会提供一键编译、内置 PDF 阅读器以及语法高亮等功能。几乎所有现代的 \LaTeX{} 编辑器都提供 Sync\TeX{} 这一强大的功能,以在 PDF 和 代码间对应跳转。
    }
  \end{center}
\end{frame}

\begin{frame}
  \frametitle{在线平台}
  \begin{table}
    \caption{在线协作平台推荐}
    \footnotesize
    \begin{stampbox}
      \begin{tabular}{c>{\raggedright}*{2}{p{4cm}}}
        \alert{在线平台}     & Overleaf \link{https://www.overleaf.com/}  & 交大 \LaTeX{} 助手 \link{https://latex.sjtu.edu.cn/} \\[2pt]
        \alert{特点}      & 最流行的在线平台,提供大量的模板,但国内访问慢 & 校内平台,隐私保护有保障,共享项目限制少 \\
      \end{tabular}
    \end{stampbox}
  \end{table}
  \begin{center}
    \parbox{.9\textwidth}{
      在线平台允许你直接在网页中编辑文档,无需本地安装即可在后台运行 \LaTeX{},并显示生成的 PDF。可以参照 Overleaf 官方文档学习如何使用在线平台 \link{https://www.overleaf.com/learn}。
    }
  \end{center}
\end{frame}

\section{基本结构}
\begin{frame}[fragile]%
  \frametitle{文档部件}
  \begin{columns}[c]
    \begin{column}{0.4\textwidth}
      \only<1>{
        \cmd{documentclass} 命令加载了\textbf{文档类}。\pkg{article} 是由 \LaTeX{}提供的用于排版短文档的基本文档类。
        \begin{description}
          \footnotesize
          \item[\pkg{article}] 不包含章的短文档
          \item[\pkg{report}] 含有章的单面印刷文档
          \item[\pkg{book}] 含有章的双面印刷文档
          \item[\pkg{beamer}] 制作幻灯片
        \end{description}
      }
      \only<2>{
        \env{document} 环境用于指示文档主体的范围。\LaTeX{} 还有其他的使用 \cmd{begin} 和 \cmd{end} 的搭配,我们称这些为\textbf{环境}。它们将用来设定局部格式命令\footnotemark。
      }
      \only<3>{
        \includepdflarge{enminimal}
      }
    \end{column}
    \begin{column}{0.6\textwidth}
      \begin{codeblock}[]{排版英文最简示例}
|\only<1>{\highlightline}|\documentclass{article}
|\only<2>{\highlightline}|\begin{document}
|\only<3>{\highlightline}|  Together for a Shared Future
|\only<2>{\highlightline}|\end{document}
      \end{codeblock}
    \end{column}
  \end{columns}
  \only<2>{\footnotetext{环境实际上是一个组,只不过通过语义化的形式预装了对应的格式命令。普通的组可以直接使用一对大括号之间的内容 \{$\cdots$\} 表示。}}
\end{frame}

\section{扩展}
\begin{frame}[fragile]%
  \frametitle{中文排版}
  \begin{columns}[c]
    \begin{column}{0.4\textwidth}
      \only<1>{
        \cmd{usepackage} 用于使用宏包以向 \LaTeX{} 添加或修改功能,需要在\textbf{导言区}调用。
        这里使用 \pkg{ctex} 宏集以获得中文支持。其调用底层因随不同的引擎而不同。
        {
          \footnotesize
          \begin{stampbox}
            \begin{tabular}{c*{3}{c}}
              \alert{引擎}     & \hologo{pdfTeX}   & \hologo{XeTeX}   & \hologo{LuaTeX}   \\
              \alert{程序}     & \hologo{pdfLaTeX} & \hologo{XeLaTeX} & \hologo{LuaLaTeX} \\
              \alert{宏包}     & CJK\footnotemark & xeCJK & luatexja \\
              \alert{封装}     & \multicolumn{3}{c}{ctex} \\
            \end{tabular}
          \end{stampbox}
        }
        \vspace{-1cm}
      }
      \only<2>{
        C\TeX{} 建议对于之前提到的常规文档类,最佳实践是使用该宏集提供的四种中文文档类,以对特定类型提供额外的中文排版适配。
        \begin{center}
          \begin{stampbox}
            \footnotesize
            \begin{tabular}{cc}
              \pkg{ctexart} & \pkg{ctexrep} \\
              \pkg{ctexbook} & \pkg{ctexbeamer} \\
            \end{tabular}
          \end{stampbox}
        \end{center}
      }
      \only<3>{
        \includepdflarge{cnminimal}
      }
      \only<4>{
        大部分情况下,你都不应当在 \LaTeX{} 中强制断行:你几乎只是想另起一段,或者是想在段落之间添加空行(使用 \pkg{parskip} 宏包就可实现)。
        只有\alert{很少的}情况下你需要使用 \textbackslash{}\textbackslash{} 来另起一行而不另起一段。
      }
    \end{column}
    \begin{column}{0.6\textwidth}
      \begin{codeblock}[]{排版中文\only<2->{最佳实践}}
|\only<2>{\highlightline}|\documentclass{|\only<1>{article}\only<2->{ctexart}|}
|\only<1>{\highlightline\textbackslash{}usepackage\{ctex\}\hfill\color{cprimary}\% 导言区}|
\begin{document}
|\only<3>{\highlightline}|    一起向未来
|\only<4>{\highlightline}|
  Together for a Shared Future
\end{document}
      \end{codeblock}
    \end{column}
  \end{columns}
  \only<1>{\footnotetext{ctex 在 \faApple\,\faLinux{} 上已经不可以使用 \hologo{pdfLaTeX} 编译,以及在 \faWindows{} 上使用该引擎也会变更自动间距调整等功能的默认行为。}}
\end{frame}

\section{设定格式}
\begin{frame}[fragile]%
  \frametitle{字体样式}
  \begin{columns}
    \begin{column}{0.4\textwidth}
      \only<1>{
        \includepdflarge{fontstyle}
      }
      \only<2>{
        可以使用显示样式设定命令对小段做加粗、斜体、等宽等等的处理。
        \begin{center}
          \footnotesize
          \begin{stampbox}
            \begin{tabular}{rl}
              \cmd{textrm} & \textrm{衬线} \\
              \cmd{textbf} & \textbf{加粗} \\
              \cmd{textit} & \kaishu 斜体 \\
              \cmd{texttt} & \texttt{等宽} \\
              \cmd{textsf} & \textsf{无衬线} \\
              \cmd{textsc} & \textsc{Small Caps} \\
              \cmd{textsl} & \textsl{Slanted} \\
            \end{tabular}
          \end{stampbox}
        \end{center}
      }
      \only<3>{
        与 Word 不同的是,\LaTeX{} 一般情况下并不需要使用上面的显式命令,而是采用逻辑标记的方法,
        比如 \cmd{emph} 可以强调文字,以及下面将要提到的目次命令(第 \ref{sectioning} 页)。
        这样可以统一管理格式。
      }
    \end{column}
    \begin{column}{0.6\textwidth}
      \begin{codeblock}[]{样式}
\documentclass{ctexart}
\begin{document}
|\only<2>{\highlightline}|  \textbf{||一起向未来}

|\only<3>{\highlightline}|  \emph{Together for a Shared Future}
\end{document}
      \end{codeblock}
    \end{column}
  \end{columns}
\end{frame}

\begin{frame}[fragile]%
  \frametitle{\only<1-2>{字体大小}\only<3>{字体样式}}
  \begin{columns}
    \begin{column}{0.4\textwidth}
      \only<1>{
        \includepdflarge{fontsize}
      }
      \only<2>{
        同样地,你也可以显式地设定字体大小,但是这种命令会更改行文设置,所以需要使用一个组来限定作用范围\footnotemark。
        \begin{center}
          \footnotesize
          \begin{stampbox}
            \begin{tabular}{rl}
              \cmd{tiny} & \tiny 极小 \\
              \cmd{scriptsize} & \scriptsize 抄本大小  \\
              \cmd{footnotesize} & \footnotesize 脚注大小 \\
              \cmd{small} & \small 小 \\
              \cmd{normalsize} & \normalsize 正常大小 \\
              \cmd{large} & \large 大 \\
              \cmd{huge} & \Huge 巨大 \\
            \end{tabular}
          \end{stampbox}
        \end{center}
      }
      \only<3>{
        也可以使用字体样式对应的更改字体设置的命令,这对于大段文段的设置而言也是很方便的。
        \begin{center}
          \footnotesize
          \begin{stampbox}
            \begin{tabular}{ll}
              \cmd{textrm} & \cmd{rmfamily}\\
              \cmd{texttt} & \cmd{ttfamily}\\
              \cmd{textsf} & \cmd{sffamily}\\
              \cmd{textbf} & \cmd{bfseries}\\
              \cmd{textit} & \cmd{itshape}\\
              \cmd{textsc} & \cmd{scshape}\\
              \cmd{textsl} & \cmd{slshape}\\
            \end{tabular}
          \end{stampbox}
        \end{center}
      }
    \end{column}
    \begin{column}{0.6\textwidth}
      \begin{codeblock}[]{大小}
\documentclass{ctexart}
\begin{document}
|\only<2>{\highlightline}|  {\bfseries\Large 一起向未来\par}
|\only<3>{\highlightline}|  {\itshape Together for a Shared Future}
\end{document}
      \end{codeblock}
    \end{column}
  \end{columns}
  \only<2>{\footnotetext{注意最后显式地使用 \cmd{par} 在改回大小前结束该段,否则会导致下一行的行间距异常!}}
\end{frame}

\section{逻辑结构}
\begin{frame}[fragile]
  \frametitle{列表}
  \begin{columns}
    \begin{column}{0.35\textwidth}
      \begin{codeblock}[]{无序列表}
\documentclass{ctexart}
\begin{document}
|\highlightline|  \begin{itemize}
    \item 第一项
    \item 第二项
    \item 第三项
|\highlightline|  \end{itemize}
\end{document}
      \end{codeblock}
    \end{column}
    \begin{column}{0.35\textwidth}
      \begin{codeblock}[]{有序列表}
\documentclass{ctexart}
\begin{document}
|\highlightline|  \begin{enumerate}
    \item 第一项
    \item 第二项
    \item 第三项
|\highlightline|  \end{enumerate}
\end{document}
      \end{codeblock}
    \end{column}
    \begin{column}{0.35\textwidth}
      \begin{codeblock}[]{描述列表}
\documentclass{ctexart}
\begin{document}
|\highlightline|  \begin{description}
    \item[||第一] 文本
    \item[||第二] 文本
    \item[||第三] 文本  
|\highlightline|  \end{description}
\end{document}
      \end{codeblock}
    \end{column}
  \end{columns}
\end{frame}

%更深的列表技巧,定理环境等

\begin{frame}
  \frametitle{列表}
  \begin{columns}
    \begin{column}{0.35\textwidth}
      \includepdflarge{unordered}
    \end{column}
    \begin{column}{0.35\textwidth}
      \includepdflarge{ordered}
    \end{column}
    \begin{column}{0.35\textwidth}
      \includepdflarge{description}
    \end{column}
  \end{columns}
\end{frame}

\begin{frame}[fragile,label=sectioning]%
  \frametitle{目次结构}
  \begin{columns}
    \begin{column}{0.4\textwidth}
      \LaTeX{} 可以使用目次命令将文档划分层级\footnotemark,并自动设定对应字体样式和大小。
      \begin{center}
        \begin{stampbox}
          \footnotesize
          \begin{tabular}{rll}
           命令 & 中文 & 层次 \\
           \cmd{chapter} & 章\footnotemark & \sout{0} \\
           \cmd{section} & 节 & 1 \\
           \cmd{subsection} & 小节 & 2 \\
           \cmd{subsubsection} & 小小节 & 3 \\
          \end{tabular}
        \end{stampbox}
      \end{center}
    \end{column}
    \begin{column}{0.6\textwidth}
      \begin{codeblock}[]{目次}
\documentclass{ctexart}
\begin{document}
|\highlightline|  \section{||概念}
|\highlightline|  \subsection{\LaTeX{}}
  \LaTeX{} 是一个用以排版高质量作品的文档准备系统。
\end{document}
      \end{codeblock}
    \end{column}
  \end{columns}
  \footnotetext{章这一级只在 \pkg{report} 和 \pkg{book} 文档类(包括对应的中文文档类)有定义。还有不常用的 \cmd{part} (0@\pkg{article}/-1@\pkg{report}\&\pkg{book}\&\pkg{beamer}) 以及更低层次的 \cmd{paragraph} (4) 与 \cmd{subparagraph} (5)。 }
\end{frame}

\begin{frame}[fragile]%
  \frametitle{组织文档}
  \begin{columns}
    \begin{column}{0.4\textwidth}
      \only<1>{
        \cmd{tableofcontents} 用来生成对于目次命令的目录。如果你想设定显示到哪个层级,在这个命令前使用 \cmd{setcounter\{tocdepth\}\{层次\}}
      }
      \only<2>{
        对于大型文档而言,使用多个文件管理源文件通常是更方便的。而 \cmd{include} 和 \cmd{input} 都以相对路径的方式插入其他 \TeX{} 文档。
        区别在于,\cmd{include} 命令会从新页开始并做一些内部调整,这基本上只对 \pkg{chapter} 这一级有用。而 \cmd{input} 会原样插入源代码。
      }
      \only<3>{
        但是 \cmd{include} 插入的文档可以使用 \cmd{includeonly} 管理当前要排印哪一部分的内容,利用所有章节辅助文件的同时,减少编译时间并专注于该部分的内容。
      }
    \end{column}
    \begin{column}{0.6\textwidth}
      \begin{codeblock}[]{主文档}
\documentclass{ctexrep}
|\only<3>{\highlightline}|\includeonly{learnlatex,sjtuthesis}
\begin{document}
|\only<1>{\highlightline}|  \tableofcontents
|\only<2-3>{\highlightline}|  \include{learnlatex}
|\only<3>{\highlightline}|  \include{sjtuthesis}
\end{document}
      \end{codeblock}
    \end{column}
  \end{columns}
\end{frame}

\begin{frame}[fragile]
  \frametitle{组织文档}
  \begin{columns}
    \begin{column}{0.4\textwidth}
      \begin{codeblock}[]{learnlatex.tex}
|\highlightline|\chapter{||学习 \LaTeX{}}
\section{||概念}
\subsection{\LaTeX{}}
\LaTeX{} 是一个用以排版高质量作品的文档准备系统。
      \end{codeblock}
      子文件中就不需要添加 \env{document} 环境了\footnotemark。
    \end{column}
    \begin{column}{0.6\textwidth}
      \begin{codeblock}[]{主文档}
|\highlightline|\documentclass{ctexrep}
\includeonly{learnlatex,sjtuthesis}
\begin{document}
  \tableofcontents
  \include{learnlatex}
  \include{sjtuthesis}
\end{document}
      \end{codeblock}
    \end{column}
  \end{columns}
  \footnotetext{如果想强制指定子文档的主文档,可以在文件第一行输入魔术命令:\texttt{\% !TeX root = main.tex}}
\end{frame}

\section{图}
\begin{frame}[fragile]%
  \frametitle{\temporal<5>{插图}{浮动体}{插图}}
  \begin{columns}
    \begin{column}{0.6\textwidth}
      \begin{codeblock}[]{插入单图\only<4->{最佳实践}}
\documentclass{ctexart}
|\only<2>{\highlightline}|\usepackage{graphicx}
|\only<2>{\highlightline}|\graphicspath{{figs/}{pics/}}
\begin{document}
|\only<5>{\highlightline}|\begin{figure}[ht]
|\only<6>{\highlightline}|  \centering
|\only<3>{\highlightline}|  \includegraphics[width=|\only<1-3>{4cm}\only<4->{0.4\textbackslash{}textwidth}|]{sjtug}
|\only<7>{\highlightline}|  \caption{SJTUG 徽标}\label{fig:sjtug}
|\only<5>{\highlightline}|\end{figure}
\end{document}
      \end{codeblock}
    \end{column}
    \begin{column}{0.4\textwidth}
      \only<1>{
        \includepdflarge{insertimage}
      }
      \only<2>{
        为了插入外部图片,需要使用 \pkg{graphicx} 宏包。之后在文档主体便可以使用 \cmd{includegraphics} 插入图片。导言区也可以加入 \cmd{graphicspath} 指定图片文件夹\footnotemark。
      }
      \only<3>{
        \cmd{includegraphics} 命令便以相对路径的方式插入图片,如果无同名图片,那么后缀名可以省略。可以使用可选参数指定插入的图片尺寸,最佳实践是使用 \cmd{textwidth} 或 \cmd{linewidth} 的相对值指定尺寸大小,以在未来可能的布局更改中保留一定的灵活性。
      }
      \only<4>{
        也可以通过键值对的方法设置图片的其他属性。
        \begin{center}
          \footnotesize
          \begin{stampbox}
            \begin{tabular}{rl}
              \pkg{width} & 宽度 \\
              \pkg{height} & 高度 \\
              \pkg{scale} & 缩放 \\
              \pkg{angle} & 角度 \\
            \end{tabular}
          \end{stampbox}
        \end{center}
      }
      \only<5>{
        \env{figure} 为一个浮动体环境(\env{table} 也是),可以将其移动到其他位置上以减少行文中的空白。可以添加可选参数以指定如何放置浮动体,最多可以使用四种位置描述符:
        \begin{center}
          \footnotesize
          \begin{stampbox}
            \begin{tabular}{cll}
              \pkg{h} & Here & 尽可能在这里 \\
              \pkg{t} & Top & 页面顶部 \\
              \pkg{b} & Bottom & 页面底部 \\
              \pkg{p} & Page & 浮动体专页 \\
            \end{tabular}
          \end{stampbox}
        \end{center}
        还可以只使用 \pkg{float} 宏包提供的 \pkg{H} 描述符以强制置于此处。
      }
      \only<6>{
        采用 \cmd{centering} 命令而不是 \env{center} 环境来水平居中图片。这将避免多余的纵向间距。
      }
      \only<7>{
        使用 \cmd{caption} 命令输入题注,如果这个命令写在插入图片前面,题注将会在上方(中文中一般对 \env{table} 环境这么做)。后面将会看到如何对留有标记(\cmd{label})的图片进行引用。
      }
    \end{column}
  \end{columns}
  \only<2>{\footnotetext{其命令参数每个为一个以 \texttt{/} 结尾的文件夹,每个文件夹需要使用大括号包裹起来。}}
\end{frame}

\begin{frame}[fragile]
  \begin{columns}
    \begin{column}{0.6\textwidth}
      \begin{codeblock}[]{插入双图}
\documentclass{ctexart}
\usepackage{graphicx}
\graphicspath{{figs/}{pics/}}
\begin{document}
  \begin{figure}[ht]
|\only<1>{\highlightline}|    \begin{minipage}{0.48\textwidth}
      \centering
      \includegraphics[height=2cm]{sjtug}
|\only<2>{\highlightline}|      \caption{SJTUG 徽标}\label{fig:sjtug}
|\only<1>{\highlightline}|    \end{minipage}\hfill
|\only<1>{\highlightline}|    \begin{minipage}{0.48\textwidth}
      \centering
      \includegraphics[height=2cm]{sjtugt}
|\only<2>{\highlightline}|      \caption{SJTUG||文字}\label{fig:sjtugt}
|\only<1>{\highlightline}|    \end{minipage}
  \end{figure}
\end{document}
      \end{codeblock}
    \end{column}
    \begin{column}{0.4\textwidth}
      \only<1>{
        在 \env{figure} 环境里使用 \env{minipage} 小页制作列盒子,以并排两图并分别编号,需要设定强制参数以指定列宽。两个小页之间添加 \cmd{hfill} 使两个小页两端对齐。
      }
      \only<2>{
        在每个小页内部分别使用 \cmd{caption} 添加标签。
      }
      \only<3>{
        \includepdflarge{doubleimages}
      }
    \end{column}
  \end{columns}
\end{frame}

\begin{frame}[fragile]%
  \begin{columns}
    \begin{column}{0.6\textwidth}
      \begin{codeblock}[]{}
\documentclass{ctexart}
\usepackage{graphicx}
|\highlightline|\usepackage{subcaption}
\graphicspath{{figs/}{pics/}}
\begin{document}
  \begin{figure}[ht]
|\highlightline|    \begin{subfigure}{0.48\textwidth}
      \centering
      \includegraphics[height=2cm]{sjtug}
      \caption{||徽标}
|\highlightline|    \end{subfigure}\hfill
|\highlightline|    \begin{subfigure}{0.48\textwidth}
      \centering
      \includegraphics[height=2cm]{sjtugt}
      \caption{||文字}
|\highlightline|    \end{subfigure}
    \caption{SJTUG}\label{fig:sjtug}
  \end{figure}
\end{document}
      \end{codeblock}
    \end{column}
    \begin{column}{0.4\textwidth}
      \includepdflarge{subfigures}\vspace{15pt}
      \pkg{subcaption} 宏包提供了 \env{subfigure} 环境(以及 \env{subtable}),可以用于以子图的形式插入,编号会增加一级。也可以为子图添加子集引用编号。
    \end{column}
  \end{columns}
\end{frame}

\section{表}
\begin{frame}[fragile]
  \frametitle{简单表格}
  \begin{columns}
    \begin{column}{0.6\textwidth}
      \begin{codeblock}[]{}
\documentclass{ctexart}
|\only<1-2>{\highlightline}|\usepackage{|\temporal<1>{array}{\highlight{array}}{array},\temporal<2>{booktabs}{\highlight{booktabs}}{booktabs}|}
\begin{document}
\begin{table}[ht]
  \centering
  \caption{||北京冬奥会吉祥物}
|\only<1>{\highlightline}|  \begin{tabular}{lp{3cm}}
|\only<2>{\highlightline}|    \toprule
|\only<3>{\highlightline}|吉祥物 & 描述                          \\
|\only<2>{\highlightline}|    \midrule
|\only<3>{\highlightline}|冰墩墩 & 2022 年北京冬季奥运会吉祥物  \\
|\only<3>{\highlightline}|雪容融 & 2022 年北京冬季残奥会吉祥物  \\
|\only<2>{\highlightline}|    \bottomrule
|\only<1>{\highlightline}|  \end{tabular}
\end{table}
\end{document}
      \end{codeblock}
    \end{column}
    \begin{column}{0.4\textwidth}
      \only<1>{
        使用 \env{tabular} 环境绘制表格。需要添加参数(称为\textbf{表格导言})以确定每一列的对齐方式。引入 \pkg{array} 宏包来使用更多的\textbf{引导符}。
        \begin{center}
          \footnotesize
          \begin{stampbox}
            \begin{tabular}{>{\ttfamily}ll}
              \alert{l} & 向左对齐 \\
              \alert{c} & 居中对齐 \\
              \alert{r} & 向右对齐 \\
              \alert{p\{3cm\}} & 固定列宽,两端对齐 \\
              \alert{m\{3cm\}} & \texttt{p} + 垂直居中对齐 \\
              \alert{>\{\textbackslash{}bfseries\}} & 后一列单元格前加命令 \\
              \alert{*\{3\}\{l\}} & 三个左对齐列 \\
            \end{tabular}
          \end{stampbox}
        \end{center}
      }
      \only<2>{
        \pkg{booktabs} 宏包提供了标准三线表格所需要的行分割线:\cmd{toprule},\cmd{midrule},\cmd{bottomrule}。也可以使用 \cmd{cmidrule\{1-2\}} 来部分地绘制行分割线。一般不推荐在专业表格中使用纵向分割线。
      }
      \only<3>{
        每行内容使用 \textbackslash\textbackslash{} 分隔开,每行中的单元格使用 \& 分隔开。
      }
      \only<4>{
        \includepdflarge{table}
      }
    \end{column}
  \end{columns}
\end{frame}

\begin{frame}[fragile]%
  \begin{columns}
    \begin{column}{0.6\textwidth}
      \begin{codeblock}[]{表头居中}
\documentclass{ctexart}
\usepackage{array,booktabs}
\begin{document}
\begin{table}[ht]
  \centering
  \caption{||北京冬奥会吉祥物}
  \begin{tabular}{lp{3cm}}
    \toprule
|\highlightline|\multicolumn{1}{c}{||吉祥物} &
|\highlightline|\multicolumn{1}{c}{||描述} \\
    \midrule
||冰墩墩 & 2022 年北京冬季奥运会吉祥物  \\
||雪容融 & 2022 年北京冬季残奥会吉祥物  \\
    \bottomrule
  \end{tabular}
\end{table}
\end{document}
      \end{codeblock}
    \end{column}
    \begin{column}{0.4\textwidth}
      \cmd{multicolumn} 命令不仅可以用于合并同行的单元格,还可以用于临时地屏蔽表格导言对该列的对齐方式定义。这里用于居中表头。
      \begin{center}
        \begin{stampbox}
          \parbox{0.85\linewidth}{
            \ttfamily\color{blue}\textbackslash{}multicolumn\{格数\}\{对齐方式\}\{内容\}
          }
        \end{stampbox}
      \end{center}
      跨页表格考虑使用 \pkg{longtable} 宏包。带标注的表格可以考虑使用 \pkg{threeparttable} 宏包。考虑使用在线工具生成表格代码 \link{https://www.tablesgenerator.com/latex_tables}。
    \end{column}
  \end{columns}
\end{frame}

\section{数学公式}
\begin{frame}
  \frametitle{数学模式}
  \begin{alertblock}{}
  输入数学公式是 \LaTeX{} 的绝对强项,很多常见的公式服务依赖于一些轻量级渲染引擎比如 MathJax, K\kern-.3ex\raise.4ex\hbox{\footnotesize A}\kern-.3ex\TeX{}。但是它们实际上使用的是 \LaTeX{} 语法变种,也就是说并没有使用 \LaTeX{} 后端。所以不要期望有完全一致的输出。
  \end{alertblock}
  
  为了更好的获得数学公式输入支持,请使用 \hologo{AmS}math 宏包。数学模式分为两种:
  \begin{description}
    \item[行内模式] 一般通过一对美元符号(\$$\cdots$\$)标记,可以使用编辑器内建的符号表输入数学符号,也可以使用在线工具手写识别 \link{https://detexify.kirelabs.org/classify.html}。
    \item[行间模式] 一般通过 \env{equation} 环境\footnote{这是有编号环境,其加星号的变种 \env{equation*} 用于生成无编号环境。}输入。如果需要使用多行公式,请使用 \env{align} 环境,并按照类似表格输入的方式,使用 \& 对齐符号,\textbackslash\textbackslash{} 换行。如果不想手动居中,可以考虑多行自动居中的 \env{gather} 和单个大型公式首尾两端对齐 \env{multline}。
  \end{description}
\end{frame}

\begin{frame}
  \frametitle{数学命令展示}
  \begin{columns}
    \begin{column}{0.33\textwidth}
      \begin{exampleblock}{}
        $PV=nRT$
      \end{exampleblock}
      \begin{exampleblock}{}
        $\sum_{i=1}^ki^2=\frac{n(n+1)(2n+1)}{6}$
      \end{exampleblock}
      \begin{exampleblock}{}
        $T(n) = aT\left(\left\lceil\frac{n}{b}\right\rceil\right) + \mathcal{O}(n^d)$
      \end{exampleblock}
      \begin{exampleblock}{}
        $\frac{x_{1}+x_{2}+x_{3}}{3}\geq \sqrt[3]{x_{1}x_{2}x_{3}}$
      \end{exampleblock}
      \begin{exampleblock}{}
        $n=(\underbrace{1\cdots 1}_{k\text{ of 1's}})_2=2^{k+1}-1$
      \end{exampleblock}
      \begin{exampleblock}{}
        $\nabla f (P)= \left.\left(\frac{\partial f}{\partial x},\frac{\partial f}{\partial y},\frac{\partial f}{\partial z}\right)\right|_{P}$
      \end{exampleblock}
    \end{column}
    \begin{column}{0.67\textwidth}
      \begin{exampleblock}{}
        \begin{equation*}
          \int_{a}^b f(x)\,\mathrm{d}x=\lim_{|P|\rightarrow 0}\sum_{i=1}^n f(\xi_i)\Delta x_i
        \end{equation*}
      \end{exampleblock}
      \begin{exampleblock}{}
        \begin{equation}
          T(n) = \begin{cases}
            \mathcal{O}(n^d),&\textrm{if } d>\log_b a, \\
            \mathcal{O}(n^d\log n), &\textrm{if } d=\log_b a,\\
            \mathcal{O}(n^{\log_b a}), &\textrm{if } d<\log_b a.
          \end{cases}
        \end{equation}
      \end{exampleblock}
      \begin{exampleblock}{}
        \begin{align}
          Q^{T}A&=R \\
          \begin{pmatrix}
            q_1^T \\ q_2^T \\ q_3^T
          \end{pmatrix}
          \begin{pmatrix}
            a_1 & a_2 & a_3
          \end{pmatrix}
          &=R
        \end{align}
      \end{exampleblock}
    \end{column}
  \end{columns}
\end{frame}

%更深入地讲解 mathtools, unicode-math, siunix

\section{引用}
\begin{frame}[fragile]
  \frametitle{交叉引用}
  \only<1>{
    正如之前所提到的,\LaTeX{} 中使用 \cmd{label} 标记,然后可以使用 \cmd{ref} 来引用这个标记。 \cmd{label} 可以放在使用计数器的对象之后。
  }
  \only<2>{
    为了使得对公式编号的引用带有括号,推荐使用 \hologo{AmS}math 宏包中的 \cmd{eqref} 命令。对于多行公式环境,每一个换行符前都可以添加一个 \cmd{label} 用于引用该行公式。
  }
  \begin{columns}
    \begin{column}{0.5\textwidth}
      \begin{codeblock}[]{图}
\begin{figure}
|\only<1>{\highlightline}|  \caption{||示例}\label{fig:example}
\end{figure}
      \end{codeblock}
      \begin{codeblock}[]{表}
\begin{table}
|\only<1>{\highlightline}|  \caption{||示例}\label{tab:example}
\end{table}
      \end{codeblock}
    \end{column}
    \begin{column}{0.5\textwidth}
\begin{codeblock}[]{目次}
|\only<1>{\highlightline}|\section{||示例}\label{sec:example}
\end{codeblock}

\begin{codeblock}[]{公式}
\begin{equation}
  a = b + c
|\only<1>{\highlightline}|\label{eq:example}
\end{equation}
|\only<2>{\highlightline}|如公式 \eqref{eq:example} 所示,
\end{codeblock}
    \end{column}
  \end{columns}
\end{frame}

\begin{frame}[fragile]
  \frametitle{文献引用}
  \LaTeX{} 管理参考文献可以采用专用数据库文件 \texttt{.bib},很多的文献管理文件比如 EndNote \link{https://lic.sjtu.edu.cn/Default/softshow/tag/MDAwMDAwMDAwMLGImKE}, Zotero \link{https://www.zotero.org/}, JabRef \link{https://www.jabref.org/} 都可以直接导出这种格式的文件用于 \LaTeX{} 论文中的引用。一般不需要手写数据库文件,你只需要注意每一个文献会在数据库中有一个主键,这个类似于上文的 \cmd{label} 标记,只是要引用数据库中的文献需要使用 \cmd{cite} 命令。
  
  \begin{codeblock}[]{ref.bib}
|\highlightline|@phdthesis{devoftech,|\hfill\alert{\% 类型为博士论文,主键为\texttt{devoftech}}|
  title={||新时期我国信息技术产业的发展},
  author={||江泽民},
  year={2008}
}
  \end{codeblock}
\end{frame}

\begin{frame}
  \frametitle{文献引用}
  而让 \LaTeX{} 处理 \texttt{.bib} 数据库文件与引用有两种工作流。你可能需要查清楚如何在编辑器中设置对应的工作流,或者采用后面所提到的高级编译方式 \texttt{latexmk}。
  \begin{columns}
    \begin{column}{0.5\textwidth}
      \begin{block}{\hologo{BibTeX} + \pkg{gbt7714}}
        一般期刊提交使用这种方法,\pkg{natbib} 宏包将提供命令 \cmd{citet}(文本引用) 和 \cmd{citep}(括号引用)。中文引用可以直接使用 \pkg{gbt7714} 宏包,但是角模式和正文模式不能混用。
      \end{block}
    \end{column}
    \begin{column}{0.5\textwidth}
      \begin{block}{\hologo{biber} + \pkg{biblatex}}
        这是更容易自定义的方法,与 \hologo{BibTeX} 的运作方式稍有不同。\pkg{biblatex} 提供了更加智能的引用命令。而中文引用可以使用 \pkg{biblatex} 宏包的样式 \pkg{gb7714-2015},使用该样式需要使用 \hologo{XeLaTeX} 编译。
      \end{block}
    \end{column}
  \end{columns}
\end{frame}

\begin{frame}[fragile]
  \frametitle{文献引用}
  \begin{columns}
    \begin{column}{0.5\textwidth}
      \begin{codeblock}[]{\hologo{BibTeX} + \pkg{gbt7714}}
\documentclass{ctexart}
\usepackage{gbt7714}
\bibliographystyle{gbt7714-numerial}
% \citestyle{numbers}  % 正文模式
\begin{document}
  ||他指出了...\cite{devoftech}
  \bibliography{ref}
\end{document}
      \end{codeblock}
    \end{column}
    \begin{column}{0.5\textwidth}
      \begin{codeblock}[]{\hologo{biber} + \pkg{biblatex}}
\documentclass{ctexart}
\usepackage[backend=biber,style=gb7714-2015]{biblatex}
\addbibresource{ref.bib}
\begin{document}
  ||他在文献 \parencite{devoftech}
  ||指出了...\cite{devoftech}
  \printbibliography
\end{document}
      \end{codeblock}
    \end{column}
  \end{columns}
\end{frame}

\begin{frame}
  \frametitle{文献引用}
  \begin{columns}
    \begin{column}{0.5\textwidth}
      \includepdflarge{bibtex}
    \end{column}
    \begin{column}{0.5\textwidth}
      \includepdflarge{biblatex}
    \end{column}
  \end{columns}
\end{frame}

} % End of customized shaded number logo

|\only<3>{\highlightline}|  % !TeX root = ..\..\latex-talk.tex

\part{SJTUThesis}

\begin{frame}
  \frametitle{简介}
  \begin{columns}
    \begin{column}{0.6\textwidth}
      \begin{itemize}
        \item 最早由韦建文于 2009 年 11 月发布 0.1a 版,2018 年起由 SJTUG 接手维护
        \item 最新版:\SJTUThesisVersion{} (\SJTUThesisDate)
        \item 支持本科、硕士、博士学位论文以及课程论文的排版
      \end{itemize}
    \end{column}
    \begin{column}{0.4\textwidth}
      \begin{exampleblock}{}
        \begin{minipage}[c]{1cm}
          \includegraphics[width=0.8cm]{\getcontribpath{sjtug}{vi/sjtug}}
        \end{minipage}
        \begin{minipage}[c]{2cm}
          \href{https://github.com/sjtug}{sjtug}/\href{https://github.com/sjtug/SJTUThesis}{SJTUThesis}
        \end{minipage}
      \end{exampleblock}
      \vspace{-8pt}
      \begin{block}{}
        \scriptsize
        上海交通大学 \hologo{XeLaTeX} 学位论文及课程论文模板 | Shanghai Jiao Tong University \hologo{XeLaTeX} Thesis Template
      \end{block}
      \vspace{-8pt}
      \begin{alertblock}{}
        \scriptsize
        \begin{tabular}{cl}
          \faStar & 2.4k \\
          \faEye & 55 \\
          \faCodeBranch & 701 \\
        \end{tabular}
      \end{alertblock}
    \end{column}
  \end{columns}
\end{frame}

\begin{frame}
  \frametitle{下载与编译}
  \alert{下载} 推荐安装 Git \link{https://git-scm.com/} 后,克隆 SJTUG 镜像仓库
  \begin{exampleblock}{\faGit*}
    \ttfamily\small
    git clone https://mirror.sjtu.edu.cn/git/SJTUThesis.git/
  \end{exampleblock}

  \alert{编译} 推荐使用 \pkg{latexmk} 编译\footnote{\hologo{MiKTeX} 用户需要手动安装 Perl 解释器 \link{https://www.perl.org/get.html} 才能使用 \pkg{latexmk}。},在不能够利用自带的 \texttt{.latexmkrc} 配置文件的情况下,需要查清楚在对应的编辑器中如何使用 \hologo{XeLaTeX} + \hologo{biber} 编译 \link{https://github.com/sjtug/SJTUThesis/blob/master/README.md}。
  \begin{exampleblock}{\faTerminal}
    \ttfamily\small
    latexmk -xelatex main
  \end{exampleblock}

  Overleaf 用户可以下载压缩包后,上传并采用 \hologo{XeLaTeX} 编译方式。
\end{frame}

\begin{frame}
  \frametitle{手动编译}
  \alert{第一次编译失败} 如果没有办法通过通常方式编译成功,请尝试使用文件夹内附带 \faLinux{}\,\faApple{} \texttt{Makefile} 和 \faWindows{} \texttt{Compile.bat} 进行编译。

  \alert{统计字数} 编写过程中也可以使用对应的命令调用 \TeX{}count 来统计正文字数。
  \begin{columns}
    \begin{column}{0.38\textwidth}
      \begin{exampleblock}{\faLinux{}\,\faApple}
        \ttfamily
        make all\\
        make clean\\
        make cleanall\\
        make wordcount
      \end{exampleblock}
    \end{column}
    \begin{column}{0.38\textwidth}
      \begin{exampleblock}{\faWindows}
        \ttfamily
        ./Compile.bat thesis\\
        ./Compile.bat clean\\
        ./Compile.bat cleanall\\
        ./Compile.bat wordcount
      \end{exampleblock}
    \end{column}
    \begin{column}{0.24\textwidth}
      \begin{block}{\faInfo}
        \ttfamily
        编译论文\\
        清理中间文件\\
        $\hookrightarrow +$删除论文\\
        统计字数
      \end{block}
    \end{column}
  \end{columns}
\end{frame}

\begin{frame}[label=compile]
  \frametitle{编译问题排查}
  \begin{columns}
    \begin{column}{0.33\textwidth}
      \begin{alertblock}{无法使用 \texttt{latexmk}\thesisissue{578}}
        \hologo{MiKTeX} 需要安装 Perl 解释器。
      \end{alertblock}  
      \begin{alertblock}{C\TeX{} 套装无法编译\thesisissue{446}}
        使用最新 \TeX{} 发行版。
      \end{alertblock}
      \begin{alertblock}{\hologo{pdfLaTeX} 无法编译\thesisissue{444}}
        请使用 \texttt{latexmk},或更改编辑器设置以 \hologo{XeLaTeX} 编译。
      \end{alertblock}
    \end{column}
    \begin{column}{0.33\textwidth}
      \begin{alertblock}{缺少字体\thesisissue{564} \thesisdiscuss{598}}
        更换字体集,或者安装对应字体。
      \end{alertblock}
      \begin{alertblock}{缺少汉字\thesisissue{533} \thesisdiscuss{617}}
        去除使用 fandol 字体集的设定。或者是安装字体后,改用 \texttt{fontset=adobe} 或 \texttt{fontset=founder}。
      \end{alertblock}
    \end{column}
    \begin{column}{0.33\textwidth}
      \begin{block}{\faInfoCircle{} README}
        不同编辑器的设置请首先参阅 README \link{https://github.com/sjtug/SJTUThesis/blob/master/README.md} 文档。
      \end{block}
      \begin{block}{\faBookOpen{} Wiki}
        其他编译问题推荐查阅 Wiki \link{https://github.com/sjtug/SJTUThesis/wiki} 的使用说明部分。
      \end{block}
    \end{column}
  \end{columns}
\end{frame}

\begin{frame}[fragile, label=covers]
  \begin{codeblock}[firstnumber=3]{main.tex}
|\alert{\% 载入 SJTUThesis 模版}|
\documentclass[|\only<1>{\highlight{type}}\only<2>{type}|=|\only<1>{bachelor}\only<2>{\highlight{bachelor}}|]{sjtuthesis}
  \end{codeblock}
  \begin{figure}
    \parbox{0.9\textwidth}{
      \begin{subfigure}{0.20\textwidth}
        \framebox{\includegraphics[width=\linewidth]{support/thesis/bachelor}}
        \caption{\only<1>{学士}\only<2>{\texttt{bachelor}}}
      \end{subfigure}\hfill
      \begin{subfigure}{0.20\textwidth}
        \framebox{\includegraphics[width=\linewidth]{support/thesis/master}}
        \caption{\only<1>{硕士}\only<2>{\texttt{master}}}
      \end{subfigure}\hfill
      \begin{subfigure}{0.20\textwidth}
        \framebox{\includegraphics[width=\linewidth]{support/thesis/doctor}}
        \caption{\only<1>{博士}\only<2>{\texttt{doctor}}}
      \end{subfigure}\hfill
      \begin{subfigure}{0.20\textwidth}
        \framebox{\includegraphics[width=\linewidth]{support/thesis/course}}
        \caption{\only<1>{课程}\only<2>{\texttt{course}}}
      \end{subfigure}
      \caption{论文类型示例\only<2>{ \texttt{type}}}
    }
  \end{figure}
\end{frame}

\begin{frame}[fragile]
  \frametitle{文档类选项}
  % \framesubtitle{\textbackslash{}documentclass\{sjtuthesis\}}
  \begin{columns}
    \begin{column}{0.45\textwidth}
      \includegraphics[page=10]{thesisdir}
    \end{column}
    \begin{column}{0.55\textwidth}
      \begin{table}[H]
        \caption{文档类选项}
        \footnotesize
        \begin{tabular}{>{\ttfamily}rll}
          \toprule
          选项 & 含义 & 相关 \\
          \midrule
          type= & 指定论文类型 & 第 \ref{covers} 页\\
          fontset= & 指定字体 & 第 \ref{compile} 页\\
          \midrule
          review & 开启盲审模式 & \thesisissue{195} \thesisissue{686} \\
          twoside & 双页模式 & \thesisissue{554} \\
          oneside & 单页模式 & \thesisissue{694} \\
          openright & 章从奇数页开始 & \thesisdiscuss{724} \\
          openany & 章从任意页开始 & \thesisissue{446} \\
          \bottomrule
        \end{tabular}
      \end{table}
    \end{column}
  \end{columns}
\end{frame}

\begin{frame}[fragile]
  \frametitle{基本配置}
  \framesubtitle{\textbackslash{}input\{setup\}}
  \begin{columns}
    \begin{column}{0.45\textwidth}
      \includegraphics[page=9]{thesisdir}
    \end{column}
    \begin{column}{0.55\textwidth}
      \begin{codeblock}[firstnumber=12]{main.tex}
|\highlightline<1>|% 论文基本配置,加载宏包等全局配置
|\highlightline<1>|\input{setup}

\begin{document}

%TC:ignore

|\highlightline<2>|% 标题页
|\highlightline<2>|\maketitle
      \end{codeblock}
      \visible<2>{
        \cmd{sjtusetup} 中的 \pkg{info} 将会修改封面的信息设置(见第 \ref{covers} 页)。
      }
    \end{column}
  \end{columns}
\end{frame}

\begin{frame}[fragile]
  \frametitle{基本配置}
  \framesubtitle{\textbackslash{}sjtusetup}
  \begin{columns}
    \begin{column}{0.45\textwidth}
      \includegraphics[page=1]{thesisdir}
    \end{column}
    \begin{column}{0.55\textwidth}
      \begin{codeblock}[firstnumber=3]{setup.tex}
\sjtusetup{
  info = {
    title    = {||上海交通大学学位论文 \LaTeX{} 模板示例文档},
    title*   = {A Sample for \LaTeX-based SJTU Thesis Template},
    author   = {||某\quad{}某},
    author* = {Mo Mo},
  },
  style = { header-logo-color = red, 
  },
  name = {
    publications = {||攻读学位期间完成的论文},
  },
}
      \end{codeblock}
    \end{column}
  \end{columns}
\end{frame}

\begin{frame}
  \frametitle{基本配置}
  \framesubtitle{\textbackslash{}sjtusetup}
  \begin{columns}
    \begin{column}{0.45\textwidth}
      \includegraphics[page=1]{thesisdir}
    \end{column}
    \begin{column}{0.55\textwidth}
      \begin{table}[H]
        \centering
        \caption{info 域}
        \footnotesize
        \begin{tabular}{lll} \toprule
          命令作用 & 中文对应选项 & 英文对应选项 \\ \midrule
          论文标题 & \texttt{title} & \texttt{title*} \\
          关键字列表 & \texttt{keywords} & \texttt{keywords*} \\
          作者姓名&  \texttt{author} &\texttt{author*}\\
          申请学位名称 & \texttt{degree}&\texttt{degree*}\\
          院系名称 & \texttt{department} & \texttt{department*}\\
          专业名称 & \texttt{major} & \texttt{major*}\\
          导师 & \texttt{supervisor} & \texttt{supervisor*}\\
          副导师 & \texttt{assisupervisor} & \texttt{assisupervisor*}\\
          日期 & \multicolumn{2}{c}{\texttt{date}}\\
          学号 & \multicolumn{2}{c}{\texttt{id}}\\ \bottomrule
          \end{tabular}
      \end{table}
    \end{column}
  \end{columns}
\end{frame}

\begin{frame}[fragile]
  \frametitle{版权页}
  \framesubtitle{\textbackslash{}copyrightpage}
  \begin{columns}
    \begin{column}{0.45\textwidth}
      \only<1>{
        \includegraphics[page=9]{thesisdir}
      }
      \only<2>{
        \includegraphics[page=2]{thesisdir}
      }
      \only<3>{
        \begin{figure}[H]
          \framebox{\includegraphics[page=2,width=0.4\linewidth]{bachelor}}
          \caption{版权页}
        \end{figure}
      }
    \end{column}
    \begin{column}{0.55\textwidth}
      \begin{codeblock}[firstnumber=22]{main.tex}
|\highlightline<1>|% 原创性声明及使用授权书
|\highlightline<1>|\copyrightpage
|\highlightline<2>|% 插入外置原创性声明及使用授权书
|\highlightline<2>|% \copyrightpage[scans/sample-copyright-old.pdf]
      \end{codeblock}
      \only<1>{
        \cmd{copyrightpages} 可以用于插入版权页。
      }
      \only<2>{
        \cmd{copyrightpages} 也接受一个可选参数,用于直接使用扫描件。
      }
    \end{column}
  \end{columns}
\end{frame}

\begin{frame}[fragile]
  \frametitle{前置部分}
  \framesubtitle{\textbackslash{}frontmatter}
  \begin{columns}
    \begin{column}{0.45\textwidth}
      \only<1>{
        \includegraphics[page=9]{thesisdir}
      }
      \only<2>{
        \includegraphics[page=3]{thesisdir}
      }
      \only<3>{
        \begin{figure}[H]
          \begin{subfigure}{0.45\textwidth}
            \framebox{\includegraphics[page=3,width=\linewidth]{bachelor}}
            \caption{中文}
          \end{subfigure}\hfill
          \begin{subfigure}{0.45\textwidth}
            \framebox{\includegraphics[page=4,width=\linewidth]{bachelor}}
            \caption{英文}
          \end{subfigure}
          \caption{摘要}
        \end{figure}
      }
      \only<4>{
        \begin{figure}[H]
          \begin{subfigure}{0.30\linewidth}
            \centering
            \framebox{\includegraphics[page=5,width=0.6\linewidth]{bachelor}}
            \caption{目录}
          \end{subfigure}
          \begin{subfigure}{0.30\linewidth}
            \centering
            \framebox{\includegraphics[page=6,width=0.6\linewidth]{bachelor}}
            \caption{插图}
          \end{subfigure}

          \begin{subfigure}{0.30\linewidth}
            \centering
            \framebox{\includegraphics[page=7,width=0.6\linewidth]{bachelor}}
            \caption{表格}
          \end{subfigure}
          \begin{subfigure}{0.30\linewidth}
            \centering
            \framebox{\includegraphics[page=8,width=0.6\linewidth]{bachelor}}
            \caption{算法}
          \end{subfigure}
          \caption{索引}
        \end{figure}
      }
      \only<5>{
        \includegraphics[page=4]{thesisdir}
      }
      \only<6>{
        \begin{figure}[H]
          \framebox{\includegraphics[page=9,width=0.5\linewidth]{bachelor}}
          \caption{符号对照表}
        \end{figure}
      }
    \end{column}
    \begin{column}{0.55\textwidth}
      \begin{codeblock}[firstnumber=30]{main.tex}
|\highlightline<2-3>|% 摘要
|\highlightline<2-3>|\input{contents/abstract}

|\highlightline<4>|% 目录
|\highlightline<4>|\tableofcontents
|\highlightline<4>|% 插图索引
|\highlightline<4>|\listoffigures*
|\highlightline<4>|% 表格索引
|\highlightline<4>|\listoftables*
|\highlightline<4>|% 算法索引
|\highlightline<4>|\listofalgorithms*

|\highlightline<5-6>|% 符号对照表
|\highlightline<5-6>|\input{contents/nomenclature}
      \end{codeblock}
    \end{column}
  \end{columns}
\end{frame}

\begin{frame}[fragile]
  \frametitle{主体部分}
  \framesubtitle{\textbackslash{}mainmatter}
  \begin{columns}
    \begin{column}{0.45\textwidth}
      \only<1>{
        \includegraphics[page=5]{thesisdir}
      }
      \only<2>{
        \begin{figure}[H]
          \begin{subfigure}{0.30\linewidth}
            \centering
            \framebox{\includegraphics[page=11,width=0.6\linewidth]{bachelor}}
            \caption{简介}
          \end{subfigure}
          \begin{subfigure}{0.30\linewidth}
            \centering
            \framebox{\includegraphics[page=13,width=0.6\linewidth]{bachelor}}
            \caption{数学}
          \end{subfigure}

          \begin{subfigure}{0.30\linewidth}
            \centering
            \framebox{\includegraphics[page=16,width=0.6\linewidth]{bachelor}}
            \caption{浮动体}
          \end{subfigure}
          \begin{subfigure}{0.30\linewidth}
            \centering
            \framebox{\includegraphics[page=22,width=0.6\linewidth]{bachelor}}
            \caption{总结}
          \end{subfigure}
          \caption{主体部分}
        \end{figure}
      }
    \end{column}
    \begin{column}{0.55\textwidth}
      \begin{codeblock}[firstnumber=47]{main.tex}
|\highlightline|% 正文内容
|\highlightline|\input{contents/intro}
|\highlightline|\input{contents/math_and_citations}
|\highlightline|\input{contents/floats}
|\highlightline|\input{contents/summary}

%TC:ignore

% 参考文献
\printbibliography[heading=bibintoc]
      \end{codeblock}
    \end{column}
  \end{columns}
\end{frame}

\begin{frame}
  \frametitle{数学}
  \begin{itemize}
    \item 公式示例:\nolinkurl{contents/math_and_citations.tex}
    \item \SJTUThesis{} 定义了常用的数学环境(需要手工引入 \texttt{ntheorem} 宏包):
      \begin{table}[h]
        \centering
        \footnotesize
        \begin{tabular}{*{7}{l}}\toprule
          assumption  & axiom   & conjecture & corollary    & definition  & example   & exercise  \\
          假设        & 公理    & 猜想       & 推论         & 定义        & 例        & 练习      \\\midrule
          lemma       & problem & proof      & proposition  & remark      & solution  & theorem   \\
          引理        & 问题    & 证明       & 命题         & 注          & 解        & 定理      \\\bottomrule
        \end{tabular}
      \end{table}
      \item \SJTUThesis{} 可以通过 \texttt{unimath} 选项使用 \pkg{unicode-math} 进行数学输入,注意与传统方式的区别。\thesisissue{555}
  \end{itemize}
\end{frame}

\begin{frame}[fragile]
  \frametitle{参考文献}
  \begin{columns}
    \begin{column}{0.45\textwidth}
      \includegraphics[page=6]{thesisdir}
    \end{column}
    \begin{column}{0.55\textwidth}
      \begin{codeblock}[firstnumber=111,numbersep=2pt]{setup.tex}
% 使用 BibLaTeX 处理参考文献
%   biblatex-gb7714-2015 常用选项
%     gbnamefmt=lowercase     姓名大小写由输入信息确定
%     gbpub=false             禁用出版信息缺失处理
\usepackage[backend=biber,style=gb7714-2015]{biblatex}
% 文献表字体
% \renewcommand{\bibfont}{\zihao{-5}}
% 文献表条目间的间距
\setlength{\bibitemsep}{0pt}
|\highlightline|% 导入参考文献数据库
|\highlightline|\addbibresource{bibdata/thesis.bib}
      \end{codeblock}
    \end{column}
  \end{columns}
\end{frame}

\begin{frame}[fragile]
  \frametitle{附录}
  \framesubtitle{\textbackslash{}appendix}
  \begin{columns}
    \begin{column}{0.45\textwidth}
      \only<1>{
        \includegraphics[page=7]{thesisdir}
      }
      \only<2>{
        \begin{figure}[H]
          \begin{subfigure}{0.45\linewidth}
            \framebox{\includegraphics[width=\linewidth,page=24]{bachelor}}
            \caption{}
          \end{subfigure}\hfill
          \begin{subfigure}{0.45\textwidth}
            \framebox{\includegraphics[width=\linewidth,page=25]{bachelor}}
            \caption{}
          \end{subfigure}
          \caption{附录}
        \end{figure}
      }
    \end{column}
    \begin{column}{0.55\textwidth}
      \begin{codeblock}[firstnumber=61]{main.tex}
% 附录中图表不加入索引
\captionsetup{list=no}

% 附录内容
|\highlightline|\input{contents/app_maxwell_equations}
|\highlightline|\input{contents/app_flow_chart}
      \end{codeblock}
    \end{column}
  \end{columns}
\end{frame}

\begin{frame}[fragile]
  \frametitle{结尾部分}
  \framesubtitle{\textbackslash{}backmatter}
  \begin{columns}
    \begin{column}{0.45\textwidth}
      \only<1>{
        \includegraphics[page=8]{thesisdir}
      }
      \only<2>{
        \begin{figure}[H]
          \begin{subfigure}{0.30\linewidth}
            \centering
            \framebox{\includegraphics[page=26,width=0.6\linewidth]{bachelor}}
            \caption{致谢}
          \end{subfigure}
          \begin{subfigure}{0.30\linewidth}
            \centering
            \framebox{\includegraphics[page=27,width=0.6\linewidth]{bachelor}}
            \caption{成就}
          \end{subfigure}

          \begin{subfigure}{0.30\linewidth}
            \centering
            \framebox{\includegraphics[page=28,width=0.6\linewidth]{bachelor}}
            \caption{简历}
          \end{subfigure}
          \begin{subfigure}{0.30\linewidth}
            \centering
            \framebox{\includegraphics[page=29,width=0.6\linewidth]{bachelor}}
            \caption{大摘要*}
          \end{subfigure}
          \caption{结尾部分}
        \end{figure}
      }
    \end{column}
    \begin{column}{0.55\textwidth}
      \begin{codeblock}[firstnumber=76]{main.tex}
% 致谢
\input{contents/acknowledgements}

% 发表论文及科研成果
% 盲审论文中,发表论文及科研成果等仅以第几作者注明即可,不要出现作者或他人姓名
\input{contents/achievements}

% 简历
\input{contents/resume}

% 学士学位论文要求在最后有一个大摘要,单独编页码
\input{contents/digest}
      \end{codeblock}
    \end{column}
  \end{columns}
\end{frame}

\begin{frame}
  \frametitle{还有其他问题?}
  \begin{columns}
    \begin{column}{0.75\textwidth}
    \begin{itemize}
      \item[{\faComment*[regular]}] 日常模板或 \LaTeX{} 使用问题可以前往 Discussions \link{https://github.com/sjtug/SJTUThesis/discussions} 提问
      
      (解决后别忘了 \textcolor{green}{\faCheckCircle{} Mark as answer}
      \item[{\faDotCircle[regular]}] 如果是 \textsc{SJTUThesis} 项目本身的 bug 和 feature request
      
      可以通过 Issues \link{https://github.com/sjtug/SJTUThesis/issues} 反馈。
      \item[{\faCodeBranch}] 如果你有好点子,可以贡献代码
     
      向 \textsc{SJTU\TeX{}}(v1) \link{https://github.com/sjtug/SJTUTeX/tree/v1} 存储库发 PR,\par
      而后把解包结果同步到 \textsc{SJTUThesis}。
  
      \item[{\faTag}] 如果你对正在基于 \LaTeX3 开发的新版感兴趣,\par
      也欢迎向 \textsc{SJTU\TeX{}}(v2) \link{https://github.com/sjtug/SJTUTeX/tree/v2} 发 PR。
  
      \item[{\faQq}] 也欢迎在 QQ 群即时讨论。
    \end{itemize}
    \end{column}
    \begin{column}{0.25\textwidth}
      \includegraphics[height=0.7\textheight]{qq.jpg}
    \end{column}
  \end{columns}
\end{frame}
\end{document}
      \end{codeblock}
    \end{column}
  \end{columns}
\end{frame}

\begin{frame}[fragile]
  \frametitle{组织文档}
  \begin{columns}
    \begin{column}{0.4\textwidth}
      \begin{codeblock}[]{learnlatex.tex}
|\highlightline|\chapter{||学习 \LaTeX{}}
\section{||概念}
\subsection{\LaTeX{}}
\LaTeX{} 是一个用以排版高质量作品的文档准备系统。
      \end{codeblock}
      子文件中就不需要添加 \env{document} 环境了\footnotemark。
    \end{column}
    \begin{column}{0.6\textwidth}
      \begin{codeblock}[]{主文档}
|\highlightline|\documentclass{ctexrep}
\includeonly{learnlatex,sjtuthesis}
\begin{document}
  \tableofcontents
  % !TeX root = ..\..\latex-talk.tex

\part{学习 \LaTeX{}}
% FIXME: Part Page miniframe overflow
% FIXME: footnote fault numbering

\begin{frame}[plain]
  \vfil
  \begin{center}
    \href{https://learnlatex.org}{
      \rmfamily
      Learn\,\lower1ex\hbox{\Huge\LaTeX{}}.org
    }
  \end{center}
  \vfil
  \begin{center}
    \parbox{0.75\linewidth}{
      Learn\LaTeX{}.org\cite{learnlatex} 提供了解 \LaTeX{} 的 16 篇简短的教程,并包含了一些可以在线运行的示例,可以通过亲自动手查看实验效果。本部分主要参考由 C\TeX{}-org 提供的中文翻译版本 \link{https://github.com/CTeX-org/learnlatex.github.io/tree/zh-Hans/zh-Hans/}。
    }
  \end{center}
  \vfil
\end{frame}

{ % Start of shaded number logo

\newcommand{\shadedfont}[2][1pt]{
  % #1 (optional): shadow distance
  % #2: the text needed to be shaded
  \hbox{\rlap{\color{gray}\hskip#1#2}#2}
}
\newcounter{learnsec}
\setcounter{learnsec}{-1}
\newcommand{\updatelogo}{
  % update the logo corresponding to current counter.
  \stepcounter{learnsec}
  \logo{
    \raise.3ex\hbox{\tiny\insertsection}\shadedfont{\arabic{learnsec}}
  }
}
\let\oldsection=\section
\renewcommand{\section}[1]{\oldsection{#1}\updatelogo}

\section{是什么}
\begin{frame}
  \frametitle{\TeX{}}
  \begin{columns}[c]
    \begin{column}{0.7\textwidth}
      \begin{center}
        \rmfamily\Huge
        \hologo{La}\highlight[structure!70]{\TeX{}}
      \end{center}
      \begin{center}
        \parbox{0.75\textwidth}{
          \TeX{} 是由斯坦福大学教授高德纳
          (Donald E.~Knuth)于 1977 年开始开发的排版引擎。目前仍在更新,最新版本号为 3.141592653 \link{https://tug.org/TUGboat/tb42-1/tb130knuth-tuneup21.pdf}。
        }
      \end{center}
    \end{column}
    \begin{column}{0.3\textwidth}
      \includegraphics[width=.8\columnwidth]{Knuth.jpg}
    \end{column}
  \end{columns}
\end{frame}

\begin{frame}
  \frametitle{\LaTeX{}}
  \begin{columns}[c]
    \begin{column}{0.7\textwidth}
      \begin{center}
        \rmfamily\Huge
        \highlight[structure]{\LaTeX{}}
      \end{center}
      \begin{center}
        \parbox{0.75\textwidth}{
          \LaTeX{} 是最早在 1985 年由现就职于微软的 Leslie Lamport 开发的一种 \TeX{} \textbf{格式}\footnotemark,使用一些列宏和扩展宏包来简化 \TeX{} 的使用。现在由 \LaTeX{} Project 的成员维护。现在广泛使用的版本是 \LaTeXe{},最新的版本为 \LaTeX3(2020 年 10 月后默认内置)。
        }
      \end{center}
    \end{column}
    \begin{column}{0.3\textwidth}
      \includegraphics[width=.8\columnwidth]{Lamport.jpg}
    \end{column}
  \end{columns}
  \footnotetext{\hologo{ConTeXt} 也是一种 \TeX{} 格式 \link{https://www.contextgarden.net/}。}
\end{frame}

\begin{frame}
  \frametitle{程序}
  \begin{columns}[c]
    \begin{column}{0.7\textwidth}
      \begin{center}
        \rmfamily\Huge
        \highlight[structure]{\hologo{pdfLaTeX}}
      \end{center}
      \begin{center}
        \parbox{0.7\textwidth}{
          \hologo{pdfLaTeX} 是为了编译一个 \LaTeX{} 文档而运行的程序。实际上底层在运行一个叫 \hologo{pdfTeX} 的引擎,并预装了对应的 \LaTeX{} \textbf{格式}。为了利用临时文件,可能就需要多次运行程序。
        }
      \end{center}
    \end{column}
    \begin{column}{0.3\textwidth}
      \begin{block}{}
        \ttfamily\small
        > \highlight{pdflatex} main.tex\\
        This is pdfTeX, Version 3.141592653-
        2.6-1.40.23 (MiKTeX 21.10)\\
        entering extended mode\\
        \highlight{LaTeX2e} <2021-11-15>\\
        \highlight{L3} programming layer <2021-11-22>
      \end{block}
    \end{column}
  \end{columns}
\end{frame}

\begin{frame}
  \frametitle{引擎}
  \begin{columns}[c]
    \begin{column}{0.7\textwidth}
      \begin{center}
        \rmfamily\Huge
        \highlight[structure!70]{pdf}\hologo{La}\highlight[structure!70]{\TeX{}}
      \end{center}
      \begin{center}
        \parbox{0.7\textwidth}{
          pdf\TeX{} 是编译 \TeX{} 文档(以 \texttt{.tex} 结尾)的\textbf{引擎}---可以理解 \TeX{} 指令的\textbf{程序}。
        }
      \end{center}
    \end{column}
    \begin{column}{0.3\textwidth}
      \begin{block}{}
        \ttfamily\small
        > pdflatex main.tex\\
        This is \highlight[structure!70]{pdfTeX}, Version 3.141592653-
        2.6-1.40.23 (MiKTeX 21.10)
        entering extended mode\\
        LaTeX2e <2021-11-15>\\
        L3 programming layer <2021-11-22>
      \end{block}
    \end{column}
  \end{columns}
\end{frame}

\begin{frame}
  \frametitle{Unicode 引擎}
  \begin{table}
    \caption{主流 \hologo{(La)TeX} 程序
    \footnote{(u)p\TeX{} 是日语最常用的引擎,生成 \texttt{.dvi},支持 Unicode。}\footnote{Ap\TeX{} 具有底层 CJK 支持,内联 Ruby,Color Emoji。}}
    \footnotesize
    \begin{stampbox}
      \begin{tabular}{c>{\raggedright}*{3}{p{3.5cm}}}
        \alert{引擎}     & \hologo{pdfTeX}   & \hologo{XeTeX}   & \hologo{LuaTeX}   \\
        \alert{程序}     & \hologo{pdfLaTeX} & \hologo{XeLaTeX} & \hologo{LuaLaTeX} \\
        \alert{特点}     & 直接生成 PDF,支持 micro-typography  & 支持 Unicode、OpenType 与复杂文字编排 (CTL) & 支持 Unicode,内联 Lua,支持 OpenType \\
      \end{tabular}
    \end{stampbox}
  \end{table}

  \begin{center}
    \parbox{.9\textwidth}{
      \hologo{pdfLaTeX} 不支持 Unicode。为了排版中文,大部分情况下 \faApple{}\,\faLinux{} 应当使用 \hologo{XeLaTeX},而 \hologo{LuaLaTeX} 速度相对较慢。\faWindows{} 可以在一些情况下使用 \hologo{pdfLaTeX}。
    }
  \end{center}
\end{frame}

% \begin{frame}
%   \paragraph{\hologo{pdfLaTeX}} \TeX{} 和 \LaTeX{} 被广泛使用之前,它们只需内置支持欧洲语言即可。在 Unicode 出现之前,\LaTeX{} 提供了许多种\textbf{文件编码}来允许很多语言的文字以原生的方式输入,\hologo{pdfLaTeX} 也只需要使用 8 位文件编码和 8 位字体。
% \end{frame}

\section{跑起来}
\begin{frame}
  \frametitle{发行版}
  \begin{table}
    \caption{\hologo{TeX} 发行版}
    \footnotesize
    \begin{stampbox}
      \begin{tabular}{c>{\raggedright}*{3}{p{3.2cm}}}
        \alert{发行版}     & \hologo{MiKTeX} \link{https://mirrors.sjtug.sjtu.edu.cn/ctan/systems/win32/miktex/setup/windows-x64/basic-miktex-21.12-x64.exe}   & \TeX{} Live \link{https://mirrors.sjtug.sjtu.edu.cn/ctan/systems/texlive/tlnet/install-tl.zip}   & Mac\TeX{} \link{https://mirrors.sjtug.sjtu.edu.cn/ctan/systems/mac/mactex/mactex-20210328.pkg}  \\[2pt]
        \alert{特点}      &  只安装必要文件,依赖用时更新  &  所有平台均可使用,每年发布一次 & Mac 系统专用,对 \TeX{} Live 的进一步打包 \\
        \alert{推荐平台}  & \faWindows  & \faLinux &  \faApple \\
      \end{tabular}
    \end{stampbox}
  \end{table}
  \begin{center}
    \parbox{.9\textwidth}{
      要让 \LaTeX{} 跑起来,核心就是要有一套 \TeX{} 发行版,来获取让 \LaTeX{} 工作所需的一组程序和文件。参考《一份简短的关于 \LaTeX{} 安装的介绍》\link{https://mirrors.sjtug.sjtu.edu.cn/ctan/info/install-latex-guide-zh-cn/install-latex-guide-zh-cn.pdf} 安装想使用的发行版。推荐使用发行版的最新版本\footnote{老版本 Linux 系统的包管理器自带 \TeX{} Live 发行版可能不是最新的 \link{https://repology.org/project/texlive/versions},尽量使用镜像安装,并手动将相关环境变量添加到路径 \link{https://www.tug.org/texlive/doc/texlive-zh-cn/texlive-zh-cn.pdf}。},并使用国内镜像。
    }
  \end{center}
\end{frame}

\begin{frame}[plain]
  \hbox to \textwidth{
    \hfil
    \vbox to 3cm{
      \hbox{
        \resizebox{3cm}{!}{\includegraphics{\getcontribpath{sjtug}{vi/sjtug.pdf}}}
      }
    }
    \hfil
    \vbox to 3cm{
      \vfill
      \hbox{\Large\bfseries\color{cprimary} 稳定、快速、现代的镜像服务。}
      \vskip2pt
      \hbox{托管于华东教育网骨干节点上海交通大学。}
      \vfill
    }
    \hskip20pt
    \hfil
  }

  \begin{center}
    \parbox{0.8\textwidth}{
      推荐使用 SJTUG 软件镜像服务,离得近,下得快。
      
      \begin{description}
        \footnotesize
        \item[\TeX{} Live]  {\ttfamily tlmgr option repository https://mirrors.sjtug.sjtu.edu.cn/CTAN/systems/texlive/tlnet}
        \item[\hologo{MiKTeX}] 在 \hologo{MiKTeX} Console 中设置镜像源为 \url{https://mirrors.sjtug.sjtu.edu.cn}
      \end{description}
    }
  \end{center}
\end{frame}

\begin{frame}
  \frametitle{编辑器}
  \begin{table}
    \caption{开源编辑器推荐}
    \footnotesize
    \begin{stampbox}
      \begin{tabular}{c>{\raggedright}*{3}{p{3.5cm}}}
        \alert{编辑器}     & \begin{tabular}{c}Visual Studio Code\\ \LaTeX{} Workshop\end{tabular}  & \TeX{}studio & \TeX{}works \\[5pt]
        \alert{特点}      &  搭配 VS Code 使用非常方便,易扩展  & 可以使用大量的菜单选项输入代码块,用户友好 & 只提供基础的高亮与运行方法,发行版自带\footnote{Mac\TeX{} 打包的是 \TeX{}Shop 编辑器。} \\
      \end{tabular}
    \end{stampbox}
  \end{table}
  \begin{center}
    \parbox{.9\textwidth}{
      使用专为 \LaTeX{} 设计的编辑器将带来更多便利,因为它们往往会提供一键编译、内置 PDF 阅读器以及语法高亮等功能。几乎所有现代的 \LaTeX{} 编辑器都提供 Sync\TeX{} 这一强大的功能,以在 PDF 和 代码间对应跳转。
    }
  \end{center}
\end{frame}

\begin{frame}
  \frametitle{在线平台}
  \begin{table}
    \caption{在线协作平台推荐}
    \footnotesize
    \begin{stampbox}
      \begin{tabular}{c>{\raggedright}*{2}{p{4cm}}}
        \alert{在线平台}     & Overleaf \link{https://www.overleaf.com/}  & 交大 \LaTeX{} 助手 \link{https://latex.sjtu.edu.cn/} \\[2pt]
        \alert{特点}      & 最流行的在线平台,提供大量的模板,但国内访问慢 & 校内平台,隐私保护有保障,共享项目限制少 \\
      \end{tabular}
    \end{stampbox}
  \end{table}
  \begin{center}
    \parbox{.9\textwidth}{
      在线平台允许你直接在网页中编辑文档,无需本地安装即可在后台运行 \LaTeX{},并显示生成的 PDF。可以参照 Overleaf 官方文档学习如何使用在线平台 \link{https://www.overleaf.com/learn}。
    }
  \end{center}
\end{frame}

\section{基本结构}
\begin{frame}[fragile]%
  \frametitle{文档部件}
  \begin{columns}[c]
    \begin{column}{0.4\textwidth}
      \only<1>{
        \cmd{documentclass} 命令加载了\textbf{文档类}。\pkg{article} 是由 \LaTeX{}提供的用于排版短文档的基本文档类。
        \begin{description}
          \footnotesize
          \item[\pkg{article}] 不包含章的短文档
          \item[\pkg{report}] 含有章的单面印刷文档
          \item[\pkg{book}] 含有章的双面印刷文档
          \item[\pkg{beamer}] 制作幻灯片
        \end{description}
      }
      \only<2>{
        \env{document} 环境用于指示文档主体的范围。\LaTeX{} 还有其他的使用 \cmd{begin} 和 \cmd{end} 的搭配,我们称这些为\textbf{环境}。它们将用来设定局部格式命令\footnotemark。
      }
      \only<3>{
        \includepdflarge{enminimal}
      }
    \end{column}
    \begin{column}{0.6\textwidth}
      \begin{codeblock}[]{排版英文最简示例}
|\only<1>{\highlightline}|\documentclass{article}
|\only<2>{\highlightline}|\begin{document}
|\only<3>{\highlightline}|  Together for a Shared Future
|\only<2>{\highlightline}|\end{document}
      \end{codeblock}
    \end{column}
  \end{columns}
  \only<2>{\footnotetext{环境实际上是一个组,只不过通过语义化的形式预装了对应的格式命令。普通的组可以直接使用一对大括号之间的内容 \{$\cdots$\} 表示。}}
\end{frame}

\section{扩展}
\begin{frame}[fragile]%
  \frametitle{中文排版}
  \begin{columns}[c]
    \begin{column}{0.4\textwidth}
      \only<1>{
        \cmd{usepackage} 用于使用宏包以向 \LaTeX{} 添加或修改功能,需要在\textbf{导言区}调用。
        这里使用 \pkg{ctex} 宏集以获得中文支持。其调用底层因随不同的引擎而不同。
        {
          \footnotesize
          \begin{stampbox}
            \begin{tabular}{c*{3}{c}}
              \alert{引擎}     & \hologo{pdfTeX}   & \hologo{XeTeX}   & \hologo{LuaTeX}   \\
              \alert{程序}     & \hologo{pdfLaTeX} & \hologo{XeLaTeX} & \hologo{LuaLaTeX} \\
              \alert{宏包}     & CJK\footnotemark & xeCJK & luatexja \\
              \alert{封装}     & \multicolumn{3}{c}{ctex} \\
            \end{tabular}
          \end{stampbox}
        }
        \vspace{-1cm}
      }
      \only<2>{
        C\TeX{} 建议对于之前提到的常规文档类,最佳实践是使用该宏集提供的四种中文文档类,以对特定类型提供额外的中文排版适配。
        \begin{center}
          \begin{stampbox}
            \footnotesize
            \begin{tabular}{cc}
              \pkg{ctexart} & \pkg{ctexrep} \\
              \pkg{ctexbook} & \pkg{ctexbeamer} \\
            \end{tabular}
          \end{stampbox}
        \end{center}
      }
      \only<3>{
        \includepdflarge{cnminimal}
      }
      \only<4>{
        大部分情况下,你都不应当在 \LaTeX{} 中强制断行:你几乎只是想另起一段,或者是想在段落之间添加空行(使用 \pkg{parskip} 宏包就可实现)。
        只有\alert{很少的}情况下你需要使用 \textbackslash{}\textbackslash{} 来另起一行而不另起一段。
      }
    \end{column}
    \begin{column}{0.6\textwidth}
      \begin{codeblock}[]{排版中文\only<2->{最佳实践}}
|\only<2>{\highlightline}|\documentclass{|\only<1>{article}\only<2->{ctexart}|}
|\only<1>{\highlightline\textbackslash{}usepackage\{ctex\}\hfill\color{cprimary}\% 导言区}|
\begin{document}
|\only<3>{\highlightline}|    一起向未来
|\only<4>{\highlightline}|
  Together for a Shared Future
\end{document}
      \end{codeblock}
    \end{column}
  \end{columns}
  \only<1>{\footnotetext{ctex 在 \faApple\,\faLinux{} 上已经不可以使用 \hologo{pdfLaTeX} 编译,以及在 \faWindows{} 上使用该引擎也会变更自动间距调整等功能的默认行为。}}
\end{frame}

\section{设定格式}
\begin{frame}[fragile]%
  \frametitle{字体样式}
  \begin{columns}
    \begin{column}{0.4\textwidth}
      \only<1>{
        \includepdflarge{fontstyle}
      }
      \only<2>{
        可以使用显示样式设定命令对小段做加粗、斜体、等宽等等的处理。
        \begin{center}
          \footnotesize
          \begin{stampbox}
            \begin{tabular}{rl}
              \cmd{textrm} & \textrm{衬线} \\
              \cmd{textbf} & \textbf{加粗} \\
              \cmd{textit} & \kaishu 斜体 \\
              \cmd{texttt} & \texttt{等宽} \\
              \cmd{textsf} & \textsf{无衬线} \\
              \cmd{textsc} & \textsc{Small Caps} \\
              \cmd{textsl} & \textsl{Slanted} \\
            \end{tabular}
          \end{stampbox}
        \end{center}
      }
      \only<3>{
        与 Word 不同的是,\LaTeX{} 一般情况下并不需要使用上面的显式命令,而是采用逻辑标记的方法,
        比如 \cmd{emph} 可以强调文字,以及下面将要提到的目次命令(第 \ref{sectioning} 页)。
        这样可以统一管理格式。
      }
    \end{column}
    \begin{column}{0.6\textwidth}
      \begin{codeblock}[]{样式}
\documentclass{ctexart}
\begin{document}
|\only<2>{\highlightline}|  \textbf{||一起向未来}

|\only<3>{\highlightline}|  \emph{Together for a Shared Future}
\end{document}
      \end{codeblock}
    \end{column}
  \end{columns}
\end{frame}

\begin{frame}[fragile]%
  \frametitle{\only<1-2>{字体大小}\only<3>{字体样式}}
  \begin{columns}
    \begin{column}{0.4\textwidth}
      \only<1>{
        \includepdflarge{fontsize}
      }
      \only<2>{
        同样地,你也可以显式地设定字体大小,但是这种命令会更改行文设置,所以需要使用一个组来限定作用范围\footnotemark。
        \begin{center}
          \footnotesize
          \begin{stampbox}
            \begin{tabular}{rl}
              \cmd{tiny} & \tiny 极小 \\
              \cmd{scriptsize} & \scriptsize 抄本大小  \\
              \cmd{footnotesize} & \footnotesize 脚注大小 \\
              \cmd{small} & \small 小 \\
              \cmd{normalsize} & \normalsize 正常大小 \\
              \cmd{large} & \large 大 \\
              \cmd{huge} & \Huge 巨大 \\
            \end{tabular}
          \end{stampbox}
        \end{center}
      }
      \only<3>{
        也可以使用字体样式对应的更改字体设置的命令,这对于大段文段的设置而言也是很方便的。
        \begin{center}
          \footnotesize
          \begin{stampbox}
            \begin{tabular}{ll}
              \cmd{textrm} & \cmd{rmfamily}\\
              \cmd{texttt} & \cmd{ttfamily}\\
              \cmd{textsf} & \cmd{sffamily}\\
              \cmd{textbf} & \cmd{bfseries}\\
              \cmd{textit} & \cmd{itshape}\\
              \cmd{textsc} & \cmd{scshape}\\
              \cmd{textsl} & \cmd{slshape}\\
            \end{tabular}
          \end{stampbox}
        \end{center}
      }
    \end{column}
    \begin{column}{0.6\textwidth}
      \begin{codeblock}[]{大小}
\documentclass{ctexart}
\begin{document}
|\only<2>{\highlightline}|  {\bfseries\Large 一起向未来\par}
|\only<3>{\highlightline}|  {\itshape Together for a Shared Future}
\end{document}
      \end{codeblock}
    \end{column}
  \end{columns}
  \only<2>{\footnotetext{注意最后显式地使用 \cmd{par} 在改回大小前结束该段,否则会导致下一行的行间距异常!}}
\end{frame}

\section{逻辑结构}
\begin{frame}[fragile]
  \frametitle{列表}
  \begin{columns}
    \begin{column}{0.35\textwidth}
      \begin{codeblock}[]{无序列表}
\documentclass{ctexart}
\begin{document}
|\highlightline|  \begin{itemize}
    \item 第一项
    \item 第二项
    \item 第三项
|\highlightline|  \end{itemize}
\end{document}
      \end{codeblock}
    \end{column}
    \begin{column}{0.35\textwidth}
      \begin{codeblock}[]{有序列表}
\documentclass{ctexart}
\begin{document}
|\highlightline|  \begin{enumerate}
    \item 第一项
    \item 第二项
    \item 第三项
|\highlightline|  \end{enumerate}
\end{document}
      \end{codeblock}
    \end{column}
    \begin{column}{0.35\textwidth}
      \begin{codeblock}[]{描述列表}
\documentclass{ctexart}
\begin{document}
|\highlightline|  \begin{description}
    \item[||第一] 文本
    \item[||第二] 文本
    \item[||第三] 文本  
|\highlightline|  \end{description}
\end{document}
      \end{codeblock}
    \end{column}
  \end{columns}
\end{frame}

%更深的列表技巧,定理环境等

\begin{frame}
  \frametitle{列表}
  \begin{columns}
    \begin{column}{0.35\textwidth}
      \includepdflarge{unordered}
    \end{column}
    \begin{column}{0.35\textwidth}
      \includepdflarge{ordered}
    \end{column}
    \begin{column}{0.35\textwidth}
      \includepdflarge{description}
    \end{column}
  \end{columns}
\end{frame}

\begin{frame}[fragile,label=sectioning]%
  \frametitle{目次结构}
  \begin{columns}
    \begin{column}{0.4\textwidth}
      \LaTeX{} 可以使用目次命令将文档划分层级\footnotemark,并自动设定对应字体样式和大小。
      \begin{center}
        \begin{stampbox}
          \footnotesize
          \begin{tabular}{rll}
           命令 & 中文 & 层次 \\
           \cmd{chapter} & 章\footnotemark & \sout{0} \\
           \cmd{section} & 节 & 1 \\
           \cmd{subsection} & 小节 & 2 \\
           \cmd{subsubsection} & 小小节 & 3 \\
          \end{tabular}
        \end{stampbox}
      \end{center}
    \end{column}
    \begin{column}{0.6\textwidth}
      \begin{codeblock}[]{目次}
\documentclass{ctexart}
\begin{document}
|\highlightline|  \section{||概念}
|\highlightline|  \subsection{\LaTeX{}}
  \LaTeX{} 是一个用以排版高质量作品的文档准备系统。
\end{document}
      \end{codeblock}
    \end{column}
  \end{columns}
  \footnotetext{章这一级只在 \pkg{report} 和 \pkg{book} 文档类(包括对应的中文文档类)有定义。还有不常用的 \cmd{part} (0@\pkg{article}/-1@\pkg{report}\&\pkg{book}\&\pkg{beamer}) 以及更低层次的 \cmd{paragraph} (4) 与 \cmd{subparagraph} (5)。 }
\end{frame}

\begin{frame}[fragile]%
  \frametitle{组织文档}
  \begin{columns}
    \begin{column}{0.4\textwidth}
      \only<1>{
        \cmd{tableofcontents} 用来生成对于目次命令的目录。如果你想设定显示到哪个层级,在这个命令前使用 \cmd{setcounter\{tocdepth\}\{层次\}}
      }
      \only<2>{
        对于大型文档而言,使用多个文件管理源文件通常是更方便的。而 \cmd{include} 和 \cmd{input} 都以相对路径的方式插入其他 \TeX{} 文档。
        区别在于,\cmd{include} 命令会从新页开始并做一些内部调整,这基本上只对 \pkg{chapter} 这一级有用。而 \cmd{input} 会原样插入源代码。
      }
      \only<3>{
        但是 \cmd{include} 插入的文档可以使用 \cmd{includeonly} 管理当前要排印哪一部分的内容,利用所有章节辅助文件的同时,减少编译时间并专注于该部分的内容。
      }
    \end{column}
    \begin{column}{0.6\textwidth}
      \begin{codeblock}[]{主文档}
\documentclass{ctexrep}
|\only<3>{\highlightline}|\includeonly{learnlatex,sjtuthesis}
\begin{document}
|\only<1>{\highlightline}|  \tableofcontents
|\only<2-3>{\highlightline}|  \include{learnlatex}
|\only<3>{\highlightline}|  \include{sjtuthesis}
\end{document}
      \end{codeblock}
    \end{column}
  \end{columns}
\end{frame}

\begin{frame}[fragile]
  \frametitle{组织文档}
  \begin{columns}
    \begin{column}{0.4\textwidth}
      \begin{codeblock}[]{learnlatex.tex}
|\highlightline|\chapter{||学习 \LaTeX{}}
\section{||概念}
\subsection{\LaTeX{}}
\LaTeX{} 是一个用以排版高质量作品的文档准备系统。
      \end{codeblock}
      子文件中就不需要添加 \env{document} 环境了\footnotemark。
    \end{column}
    \begin{column}{0.6\textwidth}
      \begin{codeblock}[]{主文档}
|\highlightline|\documentclass{ctexrep}
\includeonly{learnlatex,sjtuthesis}
\begin{document}
  \tableofcontents
  \include{learnlatex}
  \include{sjtuthesis}
\end{document}
      \end{codeblock}
    \end{column}
  \end{columns}
  \footnotetext{如果想强制指定子文档的主文档,可以在文件第一行输入魔术命令:\texttt{\% !TeX root = main.tex}}
\end{frame}

\section{图}
\begin{frame}[fragile]%
  \frametitle{\temporal<5>{插图}{浮动体}{插图}}
  \begin{columns}
    \begin{column}{0.6\textwidth}
      \begin{codeblock}[]{插入单图\only<4->{最佳实践}}
\documentclass{ctexart}
|\only<2>{\highlightline}|\usepackage{graphicx}
|\only<2>{\highlightline}|\graphicspath{{figs/}{pics/}}
\begin{document}
|\only<5>{\highlightline}|\begin{figure}[ht]
|\only<6>{\highlightline}|  \centering
|\only<3>{\highlightline}|  \includegraphics[width=|\only<1-3>{4cm}\only<4->{0.4\textbackslash{}textwidth}|]{sjtug}
|\only<7>{\highlightline}|  \caption{SJTUG 徽标}\label{fig:sjtug}
|\only<5>{\highlightline}|\end{figure}
\end{document}
      \end{codeblock}
    \end{column}
    \begin{column}{0.4\textwidth}
      \only<1>{
        \includepdflarge{insertimage}
      }
      \only<2>{
        为了插入外部图片,需要使用 \pkg{graphicx} 宏包。之后在文档主体便可以使用 \cmd{includegraphics} 插入图片。导言区也可以加入 \cmd{graphicspath} 指定图片文件夹\footnotemark。
      }
      \only<3>{
        \cmd{includegraphics} 命令便以相对路径的方式插入图片,如果无同名图片,那么后缀名可以省略。可以使用可选参数指定插入的图片尺寸,最佳实践是使用 \cmd{textwidth} 或 \cmd{linewidth} 的相对值指定尺寸大小,以在未来可能的布局更改中保留一定的灵活性。
      }
      \only<4>{
        也可以通过键值对的方法设置图片的其他属性。
        \begin{center}
          \footnotesize
          \begin{stampbox}
            \begin{tabular}{rl}
              \pkg{width} & 宽度 \\
              \pkg{height} & 高度 \\
              \pkg{scale} & 缩放 \\
              \pkg{angle} & 角度 \\
            \end{tabular}
          \end{stampbox}
        \end{center}
      }
      \only<5>{
        \env{figure} 为一个浮动体环境(\env{table} 也是),可以将其移动到其他位置上以减少行文中的空白。可以添加可选参数以指定如何放置浮动体,最多可以使用四种位置描述符:
        \begin{center}
          \footnotesize
          \begin{stampbox}
            \begin{tabular}{cll}
              \pkg{h} & Here & 尽可能在这里 \\
              \pkg{t} & Top & 页面顶部 \\
              \pkg{b} & Bottom & 页面底部 \\
              \pkg{p} & Page & 浮动体专页 \\
            \end{tabular}
          \end{stampbox}
        \end{center}
        还可以只使用 \pkg{float} 宏包提供的 \pkg{H} 描述符以强制置于此处。
      }
      \only<6>{
        采用 \cmd{centering} 命令而不是 \env{center} 环境来水平居中图片。这将避免多余的纵向间距。
      }
      \only<7>{
        使用 \cmd{caption} 命令输入题注,如果这个命令写在插入图片前面,题注将会在上方(中文中一般对 \env{table} 环境这么做)。后面将会看到如何对留有标记(\cmd{label})的图片进行引用。
      }
    \end{column}
  \end{columns}
  \only<2>{\footnotetext{其命令参数每个为一个以 \texttt{/} 结尾的文件夹,每个文件夹需要使用大括号包裹起来。}}
\end{frame}

\begin{frame}[fragile]
  \begin{columns}
    \begin{column}{0.6\textwidth}
      \begin{codeblock}[]{插入双图}
\documentclass{ctexart}
\usepackage{graphicx}
\graphicspath{{figs/}{pics/}}
\begin{document}
  \begin{figure}[ht]
|\only<1>{\highlightline}|    \begin{minipage}{0.48\textwidth}
      \centering
      \includegraphics[height=2cm]{sjtug}
|\only<2>{\highlightline}|      \caption{SJTUG 徽标}\label{fig:sjtug}
|\only<1>{\highlightline}|    \end{minipage}\hfill
|\only<1>{\highlightline}|    \begin{minipage}{0.48\textwidth}
      \centering
      \includegraphics[height=2cm]{sjtugt}
|\only<2>{\highlightline}|      \caption{SJTUG||文字}\label{fig:sjtugt}
|\only<1>{\highlightline}|    \end{minipage}
  \end{figure}
\end{document}
      \end{codeblock}
    \end{column}
    \begin{column}{0.4\textwidth}
      \only<1>{
        在 \env{figure} 环境里使用 \env{minipage} 小页制作列盒子,以并排两图并分别编号,需要设定强制参数以指定列宽。两个小页之间添加 \cmd{hfill} 使两个小页两端对齐。
      }
      \only<2>{
        在每个小页内部分别使用 \cmd{caption} 添加标签。
      }
      \only<3>{
        \includepdflarge{doubleimages}
      }
    \end{column}
  \end{columns}
\end{frame}

\begin{frame}[fragile]%
  \begin{columns}
    \begin{column}{0.6\textwidth}
      \begin{codeblock}[]{}
\documentclass{ctexart}
\usepackage{graphicx}
|\highlightline|\usepackage{subcaption}
\graphicspath{{figs/}{pics/}}
\begin{document}
  \begin{figure}[ht]
|\highlightline|    \begin{subfigure}{0.48\textwidth}
      \centering
      \includegraphics[height=2cm]{sjtug}
      \caption{||徽标}
|\highlightline|    \end{subfigure}\hfill
|\highlightline|    \begin{subfigure}{0.48\textwidth}
      \centering
      \includegraphics[height=2cm]{sjtugt}
      \caption{||文字}
|\highlightline|    \end{subfigure}
    \caption{SJTUG}\label{fig:sjtug}
  \end{figure}
\end{document}
      \end{codeblock}
    \end{column}
    \begin{column}{0.4\textwidth}
      \includepdflarge{subfigures}\vspace{15pt}
      \pkg{subcaption} 宏包提供了 \env{subfigure} 环境(以及 \env{subtable}),可以用于以子图的形式插入,编号会增加一级。也可以为子图添加子集引用编号。
    \end{column}
  \end{columns}
\end{frame}

\section{表}
\begin{frame}[fragile]
  \frametitle{简单表格}
  \begin{columns}
    \begin{column}{0.6\textwidth}
      \begin{codeblock}[]{}
\documentclass{ctexart}
|\only<1-2>{\highlightline}|\usepackage{|\temporal<1>{array}{\highlight{array}}{array},\temporal<2>{booktabs}{\highlight{booktabs}}{booktabs}|}
\begin{document}
\begin{table}[ht]
  \centering
  \caption{||北京冬奥会吉祥物}
|\only<1>{\highlightline}|  \begin{tabular}{lp{3cm}}
|\only<2>{\highlightline}|    \toprule
|\only<3>{\highlightline}|吉祥物 & 描述                          \\
|\only<2>{\highlightline}|    \midrule
|\only<3>{\highlightline}|冰墩墩 & 2022 年北京冬季奥运会吉祥物  \\
|\only<3>{\highlightline}|雪容融 & 2022 年北京冬季残奥会吉祥物  \\
|\only<2>{\highlightline}|    \bottomrule
|\only<1>{\highlightline}|  \end{tabular}
\end{table}
\end{document}
      \end{codeblock}
    \end{column}
    \begin{column}{0.4\textwidth}
      \only<1>{
        使用 \env{tabular} 环境绘制表格。需要添加参数(称为\textbf{表格导言})以确定每一列的对齐方式。引入 \pkg{array} 宏包来使用更多的\textbf{引导符}。
        \begin{center}
          \footnotesize
          \begin{stampbox}
            \begin{tabular}{>{\ttfamily}ll}
              \alert{l} & 向左对齐 \\
              \alert{c} & 居中对齐 \\
              \alert{r} & 向右对齐 \\
              \alert{p\{3cm\}} & 固定列宽,两端对齐 \\
              \alert{m\{3cm\}} & \texttt{p} + 垂直居中对齐 \\
              \alert{>\{\textbackslash{}bfseries\}} & 后一列单元格前加命令 \\
              \alert{*\{3\}\{l\}} & 三个左对齐列 \\
            \end{tabular}
          \end{stampbox}
        \end{center}
      }
      \only<2>{
        \pkg{booktabs} 宏包提供了标准三线表格所需要的行分割线:\cmd{toprule},\cmd{midrule},\cmd{bottomrule}。也可以使用 \cmd{cmidrule\{1-2\}} 来部分地绘制行分割线。一般不推荐在专业表格中使用纵向分割线。
      }
      \only<3>{
        每行内容使用 \textbackslash\textbackslash{} 分隔开,每行中的单元格使用 \& 分隔开。
      }
      \only<4>{
        \includepdflarge{table}
      }
    \end{column}
  \end{columns}
\end{frame}

\begin{frame}[fragile]%
  \begin{columns}
    \begin{column}{0.6\textwidth}
      \begin{codeblock}[]{表头居中}
\documentclass{ctexart}
\usepackage{array,booktabs}
\begin{document}
\begin{table}[ht]
  \centering
  \caption{||北京冬奥会吉祥物}
  \begin{tabular}{lp{3cm}}
    \toprule
|\highlightline|\multicolumn{1}{c}{||吉祥物} &
|\highlightline|\multicolumn{1}{c}{||描述} \\
    \midrule
||冰墩墩 & 2022 年北京冬季奥运会吉祥物  \\
||雪容融 & 2022 年北京冬季残奥会吉祥物  \\
    \bottomrule
  \end{tabular}
\end{table}
\end{document}
      \end{codeblock}
    \end{column}
    \begin{column}{0.4\textwidth}
      \cmd{multicolumn} 命令不仅可以用于合并同行的单元格,还可以用于临时地屏蔽表格导言对该列的对齐方式定义。这里用于居中表头。
      \begin{center}
        \begin{stampbox}
          \parbox{0.85\linewidth}{
            \ttfamily\color{blue}\textbackslash{}multicolumn\{格数\}\{对齐方式\}\{内容\}
          }
        \end{stampbox}
      \end{center}
      跨页表格考虑使用 \pkg{longtable} 宏包。带标注的表格可以考虑使用 \pkg{threeparttable} 宏包。考虑使用在线工具生成表格代码 \link{https://www.tablesgenerator.com/latex_tables}。
    \end{column}
  \end{columns}
\end{frame}

\section{数学公式}
\begin{frame}
  \frametitle{数学模式}
  \begin{alertblock}{}
  输入数学公式是 \LaTeX{} 的绝对强项,很多常见的公式服务依赖于一些轻量级渲染引擎比如 MathJax, K\kern-.3ex\raise.4ex\hbox{\footnotesize A}\kern-.3ex\TeX{}。但是它们实际上使用的是 \LaTeX{} 语法变种,也就是说并没有使用 \LaTeX{} 后端。所以不要期望有完全一致的输出。
  \end{alertblock}
  
  为了更好的获得数学公式输入支持,请使用 \hologo{AmS}math 宏包。数学模式分为两种:
  \begin{description}
    \item[行内模式] 一般通过一对美元符号(\$$\cdots$\$)标记,可以使用编辑器内建的符号表输入数学符号,也可以使用在线工具手写识别 \link{https://detexify.kirelabs.org/classify.html}。
    \item[行间模式] 一般通过 \env{equation} 环境\footnote{这是有编号环境,其加星号的变种 \env{equation*} 用于生成无编号环境。}输入。如果需要使用多行公式,请使用 \env{align} 环境,并按照类似表格输入的方式,使用 \& 对齐符号,\textbackslash\textbackslash{} 换行。如果不想手动居中,可以考虑多行自动居中的 \env{gather} 和单个大型公式首尾两端对齐 \env{multline}。
  \end{description}
\end{frame}

\begin{frame}
  \frametitle{数学命令展示}
  \begin{columns}
    \begin{column}{0.33\textwidth}
      \begin{exampleblock}{}
        $PV=nRT$
      \end{exampleblock}
      \begin{exampleblock}{}
        $\sum_{i=1}^ki^2=\frac{n(n+1)(2n+1)}{6}$
      \end{exampleblock}
      \begin{exampleblock}{}
        $T(n) = aT\left(\left\lceil\frac{n}{b}\right\rceil\right) + \mathcal{O}(n^d)$
      \end{exampleblock}
      \begin{exampleblock}{}
        $\frac{x_{1}+x_{2}+x_{3}}{3}\geq \sqrt[3]{x_{1}x_{2}x_{3}}$
      \end{exampleblock}
      \begin{exampleblock}{}
        $n=(\underbrace{1\cdots 1}_{k\text{ of 1's}})_2=2^{k+1}-1$
      \end{exampleblock}
      \begin{exampleblock}{}
        $\nabla f (P)= \left.\left(\frac{\partial f}{\partial x},\frac{\partial f}{\partial y},\frac{\partial f}{\partial z}\right)\right|_{P}$
      \end{exampleblock}
    \end{column}
    \begin{column}{0.67\textwidth}
      \begin{exampleblock}{}
        \begin{equation*}
          \int_{a}^b f(x)\,\mathrm{d}x=\lim_{|P|\rightarrow 0}\sum_{i=1}^n f(\xi_i)\Delta x_i
        \end{equation*}
      \end{exampleblock}
      \begin{exampleblock}{}
        \begin{equation}
          T(n) = \begin{cases}
            \mathcal{O}(n^d),&\textrm{if } d>\log_b a, \\
            \mathcal{O}(n^d\log n), &\textrm{if } d=\log_b a,\\
            \mathcal{O}(n^{\log_b a}), &\textrm{if } d<\log_b a.
          \end{cases}
        \end{equation}
      \end{exampleblock}
      \begin{exampleblock}{}
        \begin{align}
          Q^{T}A&=R \\
          \begin{pmatrix}
            q_1^T \\ q_2^T \\ q_3^T
          \end{pmatrix}
          \begin{pmatrix}
            a_1 & a_2 & a_3
          \end{pmatrix}
          &=R
        \end{align}
      \end{exampleblock}
    \end{column}
  \end{columns}
\end{frame}

%更深入地讲解 mathtools, unicode-math, siunix

\section{引用}
\begin{frame}[fragile]
  \frametitle{交叉引用}
  \only<1>{
    正如之前所提到的,\LaTeX{} 中使用 \cmd{label} 标记,然后可以使用 \cmd{ref} 来引用这个标记。 \cmd{label} 可以放在使用计数器的对象之后。
  }
  \only<2>{
    为了使得对公式编号的引用带有括号,推荐使用 \hologo{AmS}math 宏包中的 \cmd{eqref} 命令。对于多行公式环境,每一个换行符前都可以添加一个 \cmd{label} 用于引用该行公式。
  }
  \begin{columns}
    \begin{column}{0.5\textwidth}
      \begin{codeblock}[]{图}
\begin{figure}
|\only<1>{\highlightline}|  \caption{||示例}\label{fig:example}
\end{figure}
      \end{codeblock}
      \begin{codeblock}[]{表}
\begin{table}
|\only<1>{\highlightline}|  \caption{||示例}\label{tab:example}
\end{table}
      \end{codeblock}
    \end{column}
    \begin{column}{0.5\textwidth}
\begin{codeblock}[]{目次}
|\only<1>{\highlightline}|\section{||示例}\label{sec:example}
\end{codeblock}

\begin{codeblock}[]{公式}
\begin{equation}
  a = b + c
|\only<1>{\highlightline}|\label{eq:example}
\end{equation}
|\only<2>{\highlightline}|如公式 \eqref{eq:example} 所示,
\end{codeblock}
    \end{column}
  \end{columns}
\end{frame}

\begin{frame}[fragile]
  \frametitle{文献引用}
  \LaTeX{} 管理参考文献可以采用专用数据库文件 \texttt{.bib},很多的文献管理文件比如 EndNote \link{https://lic.sjtu.edu.cn/Default/softshow/tag/MDAwMDAwMDAwMLGImKE}, Zotero \link{https://www.zotero.org/}, JabRef \link{https://www.jabref.org/} 都可以直接导出这种格式的文件用于 \LaTeX{} 论文中的引用。一般不需要手写数据库文件,你只需要注意每一个文献会在数据库中有一个主键,这个类似于上文的 \cmd{label} 标记,只是要引用数据库中的文献需要使用 \cmd{cite} 命令。
  
  \begin{codeblock}[]{ref.bib}
|\highlightline|@phdthesis{devoftech,|\hfill\alert{\% 类型为博士论文,主键为\texttt{devoftech}}|
  title={||新时期我国信息技术产业的发展},
  author={||江泽民},
  year={2008}
}
  \end{codeblock}
\end{frame}

\begin{frame}
  \frametitle{文献引用}
  而让 \LaTeX{} 处理 \texttt{.bib} 数据库文件与引用有两种工作流。你可能需要查清楚如何在编辑器中设置对应的工作流,或者采用后面所提到的高级编译方式 \texttt{latexmk}。
  \begin{columns}
    \begin{column}{0.5\textwidth}
      \begin{block}{\hologo{BibTeX} + \pkg{gbt7714}}
        一般期刊提交使用这种方法,\pkg{natbib} 宏包将提供命令 \cmd{citet}(文本引用) 和 \cmd{citep}(括号引用)。中文引用可以直接使用 \pkg{gbt7714} 宏包,但是角模式和正文模式不能混用。
      \end{block}
    \end{column}
    \begin{column}{0.5\textwidth}
      \begin{block}{\hologo{biber} + \pkg{biblatex}}
        这是更容易自定义的方法,与 \hologo{BibTeX} 的运作方式稍有不同。\pkg{biblatex} 提供了更加智能的引用命令。而中文引用可以使用 \pkg{biblatex} 宏包的样式 \pkg{gb7714-2015},使用该样式需要使用 \hologo{XeLaTeX} 编译。
      \end{block}
    \end{column}
  \end{columns}
\end{frame}

\begin{frame}[fragile]
  \frametitle{文献引用}
  \begin{columns}
    \begin{column}{0.5\textwidth}
      \begin{codeblock}[]{\hologo{BibTeX} + \pkg{gbt7714}}
\documentclass{ctexart}
\usepackage{gbt7714}
\bibliographystyle{gbt7714-numerial}
% \citestyle{numbers}  % 正文模式
\begin{document}
  ||他指出了...\cite{devoftech}
  \bibliography{ref}
\end{document}
      \end{codeblock}
    \end{column}
    \begin{column}{0.5\textwidth}
      \begin{codeblock}[]{\hologo{biber} + \pkg{biblatex}}
\documentclass{ctexart}
\usepackage[backend=biber,style=gb7714-2015]{biblatex}
\addbibresource{ref.bib}
\begin{document}
  ||他在文献 \parencite{devoftech}
  ||指出了...\cite{devoftech}
  \printbibliography
\end{document}
      \end{codeblock}
    \end{column}
  \end{columns}
\end{frame}

\begin{frame}
  \frametitle{文献引用}
  \begin{columns}
    \begin{column}{0.5\textwidth}
      \includepdflarge{bibtex}
    \end{column}
    \begin{column}{0.5\textwidth}
      \includepdflarge{biblatex}
    \end{column}
  \end{columns}
\end{frame}

} % End of customized shaded number logo

  % !TeX root = ..\..\latex-talk.tex

\part{SJTUThesis}

\begin{frame}
  \frametitle{简介}
  \begin{columns}
    \begin{column}{0.6\textwidth}
      \begin{itemize}
        \item 最早由韦建文于 2009 年 11 月发布 0.1a 版,2018 年起由 SJTUG 接手维护
        \item 最新版:\SJTUThesisVersion{} (\SJTUThesisDate)
        \item 支持本科、硕士、博士学位论文以及课程论文的排版
      \end{itemize}
    \end{column}
    \begin{column}{0.4\textwidth}
      \begin{exampleblock}{}
        \begin{minipage}[c]{1cm}
          \includegraphics[width=0.8cm]{\getcontribpath{sjtug}{vi/sjtug}}
        \end{minipage}
        \begin{minipage}[c]{2cm}
          \href{https://github.com/sjtug}{sjtug}/\href{https://github.com/sjtug/SJTUThesis}{SJTUThesis}
        \end{minipage}
      \end{exampleblock}
      \vspace{-8pt}
      \begin{block}{}
        \scriptsize
        上海交通大学 \hologo{XeLaTeX} 学位论文及课程论文模板 | Shanghai Jiao Tong University \hologo{XeLaTeX} Thesis Template
      \end{block}
      \vspace{-8pt}
      \begin{alertblock}{}
        \scriptsize
        \begin{tabular}{cl}
          \faStar & 2.4k \\
          \faEye & 55 \\
          \faCodeBranch & 701 \\
        \end{tabular}
      \end{alertblock}
    \end{column}
  \end{columns}
\end{frame}

\begin{frame}
  \frametitle{下载与编译}
  \alert{下载} 推荐安装 Git \link{https://git-scm.com/} 后,克隆 SJTUG 镜像仓库
  \begin{exampleblock}{\faGit*}
    \ttfamily\small
    git clone https://mirror.sjtu.edu.cn/git/SJTUThesis.git/
  \end{exampleblock}

  \alert{编译} 推荐使用 \pkg{latexmk} 编译\footnote{\hologo{MiKTeX} 用户需要手动安装 Perl 解释器 \link{https://www.perl.org/get.html} 才能使用 \pkg{latexmk}。},在不能够利用自带的 \texttt{.latexmkrc} 配置文件的情况下,需要查清楚在对应的编辑器中如何使用 \hologo{XeLaTeX} + \hologo{biber} 编译 \link{https://github.com/sjtug/SJTUThesis/blob/master/README.md}。
  \begin{exampleblock}{\faTerminal}
    \ttfamily\small
    latexmk -xelatex main
  \end{exampleblock}

  Overleaf 用户可以下载压缩包后,上传并采用 \hologo{XeLaTeX} 编译方式。
\end{frame}

\begin{frame}
  \frametitle{手动编译}
  \alert{第一次编译失败} 如果没有办法通过通常方式编译成功,请尝试使用文件夹内附带 \faLinux{}\,\faApple{} \texttt{Makefile} 和 \faWindows{} \texttt{Compile.bat} 进行编译。

  \alert{统计字数} 编写过程中也可以使用对应的命令调用 \TeX{}count 来统计正文字数。
  \begin{columns}
    \begin{column}{0.38\textwidth}
      \begin{exampleblock}{\faLinux{}\,\faApple}
        \ttfamily
        make all\\
        make clean\\
        make cleanall\\
        make wordcount
      \end{exampleblock}
    \end{column}
    \begin{column}{0.38\textwidth}
      \begin{exampleblock}{\faWindows}
        \ttfamily
        ./Compile.bat thesis\\
        ./Compile.bat clean\\
        ./Compile.bat cleanall\\
        ./Compile.bat wordcount
      \end{exampleblock}
    \end{column}
    \begin{column}{0.24\textwidth}
      \begin{block}{\faInfo}
        \ttfamily
        编译论文\\
        清理中间文件\\
        $\hookrightarrow +$删除论文\\
        统计字数
      \end{block}
    \end{column}
  \end{columns}
\end{frame}

\begin{frame}[label=compile]
  \frametitle{编译问题排查}
  \begin{columns}
    \begin{column}{0.33\textwidth}
      \begin{alertblock}{无法使用 \texttt{latexmk}\thesisissue{578}}
        \hologo{MiKTeX} 需要安装 Perl 解释器。
      \end{alertblock}  
      \begin{alertblock}{C\TeX{} 套装无法编译\thesisissue{446}}
        使用最新 \TeX{} 发行版。
      \end{alertblock}
      \begin{alertblock}{\hologo{pdfLaTeX} 无法编译\thesisissue{444}}
        请使用 \texttt{latexmk},或更改编辑器设置以 \hologo{XeLaTeX} 编译。
      \end{alertblock}
    \end{column}
    \begin{column}{0.33\textwidth}
      \begin{alertblock}{缺少字体\thesisissue{564} \thesisdiscuss{598}}
        更换字体集,或者安装对应字体。
      \end{alertblock}
      \begin{alertblock}{缺少汉字\thesisissue{533} \thesisdiscuss{617}}
        去除使用 fandol 字体集的设定。或者是安装字体后,改用 \texttt{fontset=adobe} 或 \texttt{fontset=founder}。
      \end{alertblock}
    \end{column}
    \begin{column}{0.33\textwidth}
      \begin{block}{\faInfoCircle{} README}
        不同编辑器的设置请首先参阅 README \link{https://github.com/sjtug/SJTUThesis/blob/master/README.md} 文档。
      \end{block}
      \begin{block}{\faBookOpen{} Wiki}
        其他编译问题推荐查阅 Wiki \link{https://github.com/sjtug/SJTUThesis/wiki} 的使用说明部分。
      \end{block}
    \end{column}
  \end{columns}
\end{frame}

\begin{frame}[fragile, label=covers]
  \begin{codeblock}[firstnumber=3]{main.tex}
|\alert{\% 载入 SJTUThesis 模版}|
\documentclass[|\only<1>{\highlight{type}}\only<2>{type}|=|\only<1>{bachelor}\only<2>{\highlight{bachelor}}|]{sjtuthesis}
  \end{codeblock}
  \begin{figure}
    \parbox{0.9\textwidth}{
      \begin{subfigure}{0.20\textwidth}
        \framebox{\includegraphics[width=\linewidth]{support/thesis/bachelor}}
        \caption{\only<1>{学士}\only<2>{\texttt{bachelor}}}
      \end{subfigure}\hfill
      \begin{subfigure}{0.20\textwidth}
        \framebox{\includegraphics[width=\linewidth]{support/thesis/master}}
        \caption{\only<1>{硕士}\only<2>{\texttt{master}}}
      \end{subfigure}\hfill
      \begin{subfigure}{0.20\textwidth}
        \framebox{\includegraphics[width=\linewidth]{support/thesis/doctor}}
        \caption{\only<1>{博士}\only<2>{\texttt{doctor}}}
      \end{subfigure}\hfill
      \begin{subfigure}{0.20\textwidth}
        \framebox{\includegraphics[width=\linewidth]{support/thesis/course}}
        \caption{\only<1>{课程}\only<2>{\texttt{course}}}
      \end{subfigure}
      \caption{论文类型示例\only<2>{ \texttt{type}}}
    }
  \end{figure}
\end{frame}

\begin{frame}[fragile]
  \frametitle{文档类选项}
  % \framesubtitle{\textbackslash{}documentclass\{sjtuthesis\}}
  \begin{columns}
    \begin{column}{0.45\textwidth}
      \includegraphics[page=10]{thesisdir}
    \end{column}
    \begin{column}{0.55\textwidth}
      \begin{table}[H]
        \caption{文档类选项}
        \footnotesize
        \begin{tabular}{>{\ttfamily}rll}
          \toprule
          选项 & 含义 & 相关 \\
          \midrule
          type= & 指定论文类型 & 第 \ref{covers} 页\\
          fontset= & 指定字体 & 第 \ref{compile} 页\\
          \midrule
          review & 开启盲审模式 & \thesisissue{195} \thesisissue{686} \\
          twoside & 双页模式 & \thesisissue{554} \\
          oneside & 单页模式 & \thesisissue{694} \\
          openright & 章从奇数页开始 & \thesisdiscuss{724} \\
          openany & 章从任意页开始 & \thesisissue{446} \\
          \bottomrule
        \end{tabular}
      \end{table}
    \end{column}
  \end{columns}
\end{frame}

\begin{frame}[fragile]
  \frametitle{基本配置}
  \framesubtitle{\textbackslash{}input\{setup\}}
  \begin{columns}
    \begin{column}{0.45\textwidth}
      \includegraphics[page=9]{thesisdir}
    \end{column}
    \begin{column}{0.55\textwidth}
      \begin{codeblock}[firstnumber=12]{main.tex}
|\highlightline<1>|% 论文基本配置,加载宏包等全局配置
|\highlightline<1>|\input{setup}

\begin{document}

%TC:ignore

|\highlightline<2>|% 标题页
|\highlightline<2>|\maketitle
      \end{codeblock}
      \visible<2>{
        \cmd{sjtusetup} 中的 \pkg{info} 将会修改封面的信息设置(见第 \ref{covers} 页)。
      }
    \end{column}
  \end{columns}
\end{frame}

\begin{frame}[fragile]
  \frametitle{基本配置}
  \framesubtitle{\textbackslash{}sjtusetup}
  \begin{columns}
    \begin{column}{0.45\textwidth}
      \includegraphics[page=1]{thesisdir}
    \end{column}
    \begin{column}{0.55\textwidth}
      \begin{codeblock}[firstnumber=3]{setup.tex}
\sjtusetup{
  info = {
    title    = {||上海交通大学学位论文 \LaTeX{} 模板示例文档},
    title*   = {A Sample for \LaTeX-based SJTU Thesis Template},
    author   = {||某\quad{}某},
    author* = {Mo Mo},
  },
  style = { header-logo-color = red, 
  },
  name = {
    publications = {||攻读学位期间完成的论文},
  },
}
      \end{codeblock}
    \end{column}
  \end{columns}
\end{frame}

\begin{frame}
  \frametitle{基本配置}
  \framesubtitle{\textbackslash{}sjtusetup}
  \begin{columns}
    \begin{column}{0.45\textwidth}
      \includegraphics[page=1]{thesisdir}
    \end{column}
    \begin{column}{0.55\textwidth}
      \begin{table}[H]
        \centering
        \caption{info 域}
        \footnotesize
        \begin{tabular}{lll} \toprule
          命令作用 & 中文对应选项 & 英文对应选项 \\ \midrule
          论文标题 & \texttt{title} & \texttt{title*} \\
          关键字列表 & \texttt{keywords} & \texttt{keywords*} \\
          作者姓名&  \texttt{author} &\texttt{author*}\\
          申请学位名称 & \texttt{degree}&\texttt{degree*}\\
          院系名称 & \texttt{department} & \texttt{department*}\\
          专业名称 & \texttt{major} & \texttt{major*}\\
          导师 & \texttt{supervisor} & \texttt{supervisor*}\\
          副导师 & \texttt{assisupervisor} & \texttt{assisupervisor*}\\
          日期 & \multicolumn{2}{c}{\texttt{date}}\\
          学号 & \multicolumn{2}{c}{\texttt{id}}\\ \bottomrule
          \end{tabular}
      \end{table}
    \end{column}
  \end{columns}
\end{frame}

\begin{frame}[fragile]
  \frametitle{版权页}
  \framesubtitle{\textbackslash{}copyrightpage}
  \begin{columns}
    \begin{column}{0.45\textwidth}
      \only<1>{
        \includegraphics[page=9]{thesisdir}
      }
      \only<2>{
        \includegraphics[page=2]{thesisdir}
      }
      \only<3>{
        \begin{figure}[H]
          \framebox{\includegraphics[page=2,width=0.4\linewidth]{bachelor}}
          \caption{版权页}
        \end{figure}
      }
    \end{column}
    \begin{column}{0.55\textwidth}
      \begin{codeblock}[firstnumber=22]{main.tex}
|\highlightline<1>|% 原创性声明及使用授权书
|\highlightline<1>|\copyrightpage
|\highlightline<2>|% 插入外置原创性声明及使用授权书
|\highlightline<2>|% \copyrightpage[scans/sample-copyright-old.pdf]
      \end{codeblock}
      \only<1>{
        \cmd{copyrightpages} 可以用于插入版权页。
      }
      \only<2>{
        \cmd{copyrightpages} 也接受一个可选参数,用于直接使用扫描件。
      }
    \end{column}
  \end{columns}
\end{frame}

\begin{frame}[fragile]
  \frametitle{前置部分}
  \framesubtitle{\textbackslash{}frontmatter}
  \begin{columns}
    \begin{column}{0.45\textwidth}
      \only<1>{
        \includegraphics[page=9]{thesisdir}
      }
      \only<2>{
        \includegraphics[page=3]{thesisdir}
      }
      \only<3>{
        \begin{figure}[H]
          \begin{subfigure}{0.45\textwidth}
            \framebox{\includegraphics[page=3,width=\linewidth]{bachelor}}
            \caption{中文}
          \end{subfigure}\hfill
          \begin{subfigure}{0.45\textwidth}
            \framebox{\includegraphics[page=4,width=\linewidth]{bachelor}}
            \caption{英文}
          \end{subfigure}
          \caption{摘要}
        \end{figure}
      }
      \only<4>{
        \begin{figure}[H]
          \begin{subfigure}{0.30\linewidth}
            \centering
            \framebox{\includegraphics[page=5,width=0.6\linewidth]{bachelor}}
            \caption{目录}
          \end{subfigure}
          \begin{subfigure}{0.30\linewidth}
            \centering
            \framebox{\includegraphics[page=6,width=0.6\linewidth]{bachelor}}
            \caption{插图}
          \end{subfigure}

          \begin{subfigure}{0.30\linewidth}
            \centering
            \framebox{\includegraphics[page=7,width=0.6\linewidth]{bachelor}}
            \caption{表格}
          \end{subfigure}
          \begin{subfigure}{0.30\linewidth}
            \centering
            \framebox{\includegraphics[page=8,width=0.6\linewidth]{bachelor}}
            \caption{算法}
          \end{subfigure}
          \caption{索引}
        \end{figure}
      }
      \only<5>{
        \includegraphics[page=4]{thesisdir}
      }
      \only<6>{
        \begin{figure}[H]
          \framebox{\includegraphics[page=9,width=0.5\linewidth]{bachelor}}
          \caption{符号对照表}
        \end{figure}
      }
    \end{column}
    \begin{column}{0.55\textwidth}
      \begin{codeblock}[firstnumber=30]{main.tex}
|\highlightline<2-3>|% 摘要
|\highlightline<2-3>|\input{contents/abstract}

|\highlightline<4>|% 目录
|\highlightline<4>|\tableofcontents
|\highlightline<4>|% 插图索引
|\highlightline<4>|\listoffigures*
|\highlightline<4>|% 表格索引
|\highlightline<4>|\listoftables*
|\highlightline<4>|% 算法索引
|\highlightline<4>|\listofalgorithms*

|\highlightline<5-6>|% 符号对照表
|\highlightline<5-6>|\input{contents/nomenclature}
      \end{codeblock}
    \end{column}
  \end{columns}
\end{frame}

\begin{frame}[fragile]
  \frametitle{主体部分}
  \framesubtitle{\textbackslash{}mainmatter}
  \begin{columns}
    \begin{column}{0.45\textwidth}
      \only<1>{
        \includegraphics[page=5]{thesisdir}
      }
      \only<2>{
        \begin{figure}[H]
          \begin{subfigure}{0.30\linewidth}
            \centering
            \framebox{\includegraphics[page=11,width=0.6\linewidth]{bachelor}}
            \caption{简介}
          \end{subfigure}
          \begin{subfigure}{0.30\linewidth}
            \centering
            \framebox{\includegraphics[page=13,width=0.6\linewidth]{bachelor}}
            \caption{数学}
          \end{subfigure}

          \begin{subfigure}{0.30\linewidth}
            \centering
            \framebox{\includegraphics[page=16,width=0.6\linewidth]{bachelor}}
            \caption{浮动体}
          \end{subfigure}
          \begin{subfigure}{0.30\linewidth}
            \centering
            \framebox{\includegraphics[page=22,width=0.6\linewidth]{bachelor}}
            \caption{总结}
          \end{subfigure}
          \caption{主体部分}
        \end{figure}
      }
    \end{column}
    \begin{column}{0.55\textwidth}
      \begin{codeblock}[firstnumber=47]{main.tex}
|\highlightline|% 正文内容
|\highlightline|\input{contents/intro}
|\highlightline|\input{contents/math_and_citations}
|\highlightline|\input{contents/floats}
|\highlightline|\input{contents/summary}

%TC:ignore

% 参考文献
\printbibliography[heading=bibintoc]
      \end{codeblock}
    \end{column}
  \end{columns}
\end{frame}

\begin{frame}
  \frametitle{数学}
  \begin{itemize}
    \item 公式示例:\nolinkurl{contents/math_and_citations.tex}
    \item \SJTUThesis{} 定义了常用的数学环境(需要手工引入 \texttt{ntheorem} 宏包):
      \begin{table}[h]
        \centering
        \footnotesize
        \begin{tabular}{*{7}{l}}\toprule
          assumption  & axiom   & conjecture & corollary    & definition  & example   & exercise  \\
          假设        & 公理    & 猜想       & 推论         & 定义        & 例        & 练习      \\\midrule
          lemma       & problem & proof      & proposition  & remark      & solution  & theorem   \\
          引理        & 问题    & 证明       & 命题         & 注          & 解        & 定理      \\\bottomrule
        \end{tabular}
      \end{table}
      \item \SJTUThesis{} 可以通过 \texttt{unimath} 选项使用 \pkg{unicode-math} 进行数学输入,注意与传统方式的区别。\thesisissue{555}
  \end{itemize}
\end{frame}

\begin{frame}[fragile]
  \frametitle{参考文献}
  \begin{columns}
    \begin{column}{0.45\textwidth}
      \includegraphics[page=6]{thesisdir}
    \end{column}
    \begin{column}{0.55\textwidth}
      \begin{codeblock}[firstnumber=111,numbersep=2pt]{setup.tex}
% 使用 BibLaTeX 处理参考文献
%   biblatex-gb7714-2015 常用选项
%     gbnamefmt=lowercase     姓名大小写由输入信息确定
%     gbpub=false             禁用出版信息缺失处理
\usepackage[backend=biber,style=gb7714-2015]{biblatex}
% 文献表字体
% \renewcommand{\bibfont}{\zihao{-5}}
% 文献表条目间的间距
\setlength{\bibitemsep}{0pt}
|\highlightline|% 导入参考文献数据库
|\highlightline|\addbibresource{bibdata/thesis.bib}
      \end{codeblock}
    \end{column}
  \end{columns}
\end{frame}

\begin{frame}[fragile]
  \frametitle{附录}
  \framesubtitle{\textbackslash{}appendix}
  \begin{columns}
    \begin{column}{0.45\textwidth}
      \only<1>{
        \includegraphics[page=7]{thesisdir}
      }
      \only<2>{
        \begin{figure}[H]
          \begin{subfigure}{0.45\linewidth}
            \framebox{\includegraphics[width=\linewidth,page=24]{bachelor}}
            \caption{}
          \end{subfigure}\hfill
          \begin{subfigure}{0.45\textwidth}
            \framebox{\includegraphics[width=\linewidth,page=25]{bachelor}}
            \caption{}
          \end{subfigure}
          \caption{附录}
        \end{figure}
      }
    \end{column}
    \begin{column}{0.55\textwidth}
      \begin{codeblock}[firstnumber=61]{main.tex}
% 附录中图表不加入索引
\captionsetup{list=no}

% 附录内容
|\highlightline|\input{contents/app_maxwell_equations}
|\highlightline|\input{contents/app_flow_chart}
      \end{codeblock}
    \end{column}
  \end{columns}
\end{frame}

\begin{frame}[fragile]
  \frametitle{结尾部分}
  \framesubtitle{\textbackslash{}backmatter}
  \begin{columns}
    \begin{column}{0.45\textwidth}
      \only<1>{
        \includegraphics[page=8]{thesisdir}
      }
      \only<2>{
        \begin{figure}[H]
          \begin{subfigure}{0.30\linewidth}
            \centering
            \framebox{\includegraphics[page=26,width=0.6\linewidth]{bachelor}}
            \caption{致谢}
          \end{subfigure}
          \begin{subfigure}{0.30\linewidth}
            \centering
            \framebox{\includegraphics[page=27,width=0.6\linewidth]{bachelor}}
            \caption{成就}
          \end{subfigure}

          \begin{subfigure}{0.30\linewidth}
            \centering
            \framebox{\includegraphics[page=28,width=0.6\linewidth]{bachelor}}
            \caption{简历}
          \end{subfigure}
          \begin{subfigure}{0.30\linewidth}
            \centering
            \framebox{\includegraphics[page=29,width=0.6\linewidth]{bachelor}}
            \caption{大摘要*}
          \end{subfigure}
          \caption{结尾部分}
        \end{figure}
      }
    \end{column}
    \begin{column}{0.55\textwidth}
      \begin{codeblock}[firstnumber=76]{main.tex}
% 致谢
\input{contents/acknowledgements}

% 发表论文及科研成果
% 盲审论文中,发表论文及科研成果等仅以第几作者注明即可,不要出现作者或他人姓名
\input{contents/achievements}

% 简历
\input{contents/resume}

% 学士学位论文要求在最后有一个大摘要,单独编页码
\input{contents/digest}
      \end{codeblock}
    \end{column}
  \end{columns}
\end{frame}

\begin{frame}
  \frametitle{还有其他问题?}
  \begin{columns}
    \begin{column}{0.75\textwidth}
    \begin{itemize}
      \item[{\faComment*[regular]}] 日常模板或 \LaTeX{} 使用问题可以前往 Discussions \link{https://github.com/sjtug/SJTUThesis/discussions} 提问
      
      (解决后别忘了 \textcolor{green}{\faCheckCircle{} Mark as answer}
      \item[{\faDotCircle[regular]}] 如果是 \textsc{SJTUThesis} 项目本身的 bug 和 feature request
      
      可以通过 Issues \link{https://github.com/sjtug/SJTUThesis/issues} 反馈。
      \item[{\faCodeBranch}] 如果你有好点子,可以贡献代码
     
      向 \textsc{SJTU\TeX{}}(v1) \link{https://github.com/sjtug/SJTUTeX/tree/v1} 存储库发 PR,\par
      而后把解包结果同步到 \textsc{SJTUThesis}。
  
      \item[{\faTag}] 如果你对正在基于 \LaTeX3 开发的新版感兴趣,\par
      也欢迎向 \textsc{SJTU\TeX{}}(v2) \link{https://github.com/sjtug/SJTUTeX/tree/v2} 发 PR。
  
      \item[{\faQq}] 也欢迎在 QQ 群即时讨论。
    \end{itemize}
    \end{column}
    \begin{column}{0.25\textwidth}
      \includegraphics[height=0.7\textheight]{qq.jpg}
    \end{column}
  \end{columns}
\end{frame}
\end{document}
      \end{codeblock}
    \end{column}
  \end{columns}
  \footnotetext{如果想强制指定子文档的主文档,可以在文件第一行输入魔术命令:\texttt{\% !TeX root = main.tex}}
\end{frame}

\section{图}
\begin{frame}[fragile]%
  \frametitle{\temporal<5>{插图}{浮动体}{插图}}
  \begin{columns}
    \begin{column}{0.6\textwidth}
      \begin{codeblock}[]{插入单图\only<4->{最佳实践}}
\documentclass{ctexart}
|\only<2>{\highlightline}|\usepackage{graphicx}
|\only<2>{\highlightline}|\graphicspath{{figs/}{pics/}}
\begin{document}
|\only<5>{\highlightline}|\begin{figure}[ht]
|\only<6>{\highlightline}|  \centering
|\only<3>{\highlightline}|  \includegraphics[width=|\only<1-3>{4cm}\only<4->{0.4\textbackslash{}textwidth}|]{sjtug}
|\only<7>{\highlightline}|  \caption{SJTUG 徽标}\label{fig:sjtug}
|\only<5>{\highlightline}|\end{figure}
\end{document}
      \end{codeblock}
    \end{column}
    \begin{column}{0.4\textwidth}
      \only<1>{
        \includepdflarge{insertimage}
      }
      \only<2>{
        为了插入外部图片,需要使用 \pkg{graphicx} 宏包。之后在文档主体便可以使用 \cmd{includegraphics} 插入图片。导言区也可以加入 \cmd{graphicspath} 指定图片文件夹\footnotemark。
      }
      \only<3>{
        \cmd{includegraphics} 命令便以相对路径的方式插入图片,如果无同名图片,那么后缀名可以省略。可以使用可选参数指定插入的图片尺寸,最佳实践是使用 \cmd{textwidth} 或 \cmd{linewidth} 的相对值指定尺寸大小,以在未来可能的布局更改中保留一定的灵活性。
      }
      \only<4>{
        也可以通过键值对的方法设置图片的其他属性。
        \begin{center}
          \footnotesize
          \begin{stampbox}
            \begin{tabular}{rl}
              \pkg{width} & 宽度 \\
              \pkg{height} & 高度 \\
              \pkg{scale} & 缩放 \\
              \pkg{angle} & 角度 \\
            \end{tabular}
          \end{stampbox}
        \end{center}
      }
      \only<5>{
        \env{figure} 为一个浮动体环境(\env{table} 也是),可以将其移动到其他位置上以减少行文中的空白。可以添加可选参数以指定如何放置浮动体,最多可以使用四种位置描述符:
        \begin{center}
          \footnotesize
          \begin{stampbox}
            \begin{tabular}{cll}
              \pkg{h} & Here & 尽可能在这里 \\
              \pkg{t} & Top & 页面顶部 \\
              \pkg{b} & Bottom & 页面底部 \\
              \pkg{p} & Page & 浮动体专页 \\
            \end{tabular}
          \end{stampbox}
        \end{center}
        还可以只使用 \pkg{float} 宏包提供的 \pkg{H} 描述符以强制置于此处。
      }
      \only<6>{
        采用 \cmd{centering} 命令而不是 \env{center} 环境来水平居中图片。这将避免多余的纵向间距。
      }
      \only<7>{
        使用 \cmd{caption} 命令输入题注,如果这个命令写在插入图片前面,题注将会在上方(中文中一般对 \env{table} 环境这么做)。后面将会看到如何对留有标记(\cmd{label})的图片进行引用。
      }
    \end{column}
  \end{columns}
  \only<2>{\footnotetext{其命令参数每个为一个以 \texttt{/} 结尾的文件夹,每个文件夹需要使用大括号包裹起来。}}
\end{frame}

\begin{frame}[fragile]
  \begin{columns}
    \begin{column}{0.6\textwidth}
      \begin{codeblock}[]{插入双图}
\documentclass{ctexart}
\usepackage{graphicx}
\graphicspath{{figs/}{pics/}}
\begin{document}
  \begin{figure}[ht]
|\only<1>{\highlightline}|    \begin{minipage}{0.48\textwidth}
      \centering
      \includegraphics[height=2cm]{sjtug}
|\only<2>{\highlightline}|      \caption{SJTUG 徽标}\label{fig:sjtug}
|\only<1>{\highlightline}|    \end{minipage}\hfill
|\only<1>{\highlightline}|    \begin{minipage}{0.48\textwidth}
      \centering
      \includegraphics[height=2cm]{sjtugt}
|\only<2>{\highlightline}|      \caption{SJTUG||文字}\label{fig:sjtugt}
|\only<1>{\highlightline}|    \end{minipage}
  \end{figure}
\end{document}
      \end{codeblock}
    \end{column}
    \begin{column}{0.4\textwidth}
      \only<1>{
        在 \env{figure} 环境里使用 \env{minipage} 小页制作列盒子,以并排两图并分别编号,需要设定强制参数以指定列宽。两个小页之间添加 \cmd{hfill} 使两个小页两端对齐。
      }
      \only<2>{
        在每个小页内部分别使用 \cmd{caption} 添加标签。
      }
      \only<3>{
        \includepdflarge{doubleimages}
      }
    \end{column}
  \end{columns}
\end{frame}

\begin{frame}[fragile]%
  \begin{columns}
    \begin{column}{0.6\textwidth}
      \begin{codeblock}[]{}
\documentclass{ctexart}
\usepackage{graphicx}
|\highlightline|\usepackage{subcaption}
\graphicspath{{figs/}{pics/}}
\begin{document}
  \begin{figure}[ht]
|\highlightline|    \begin{subfigure}{0.48\textwidth}
      \centering
      \includegraphics[height=2cm]{sjtug}
      \caption{||徽标}
|\highlightline|    \end{subfigure}\hfill
|\highlightline|    \begin{subfigure}{0.48\textwidth}
      \centering
      \includegraphics[height=2cm]{sjtugt}
      \caption{||文字}
|\highlightline|    \end{subfigure}
    \caption{SJTUG}\label{fig:sjtug}
  \end{figure}
\end{document}
      \end{codeblock}
    \end{column}
    \begin{column}{0.4\textwidth}
      \includepdflarge{subfigures}\vspace{15pt}
      \pkg{subcaption} 宏包提供了 \env{subfigure} 环境(以及 \env{subtable}),可以用于以子图的形式插入,编号会增加一级。也可以为子图添加子集引用编号。
    \end{column}
  \end{columns}
\end{frame}

\section{表}
\begin{frame}[fragile]
  \frametitle{简单表格}
  \begin{columns}
    \begin{column}{0.6\textwidth}
      \begin{codeblock}[]{}
\documentclass{ctexart}
|\only<1-2>{\highlightline}|\usepackage{|\temporal<1>{array}{\highlight{array}}{array},\temporal<2>{booktabs}{\highlight{booktabs}}{booktabs}|}
\begin{document}
\begin{table}[ht]
  \centering
  \caption{||北京冬奥会吉祥物}
|\only<1>{\highlightline}|  \begin{tabular}{lp{3cm}}
|\only<2>{\highlightline}|    \toprule
|\only<3>{\highlightline}|吉祥物 & 描述                          \\
|\only<2>{\highlightline}|    \midrule
|\only<3>{\highlightline}|冰墩墩 & 2022 年北京冬季奥运会吉祥物  \\
|\only<3>{\highlightline}|雪容融 & 2022 年北京冬季残奥会吉祥物  \\
|\only<2>{\highlightline}|    \bottomrule
|\only<1>{\highlightline}|  \end{tabular}
\end{table}
\end{document}
      \end{codeblock}
    \end{column}
    \begin{column}{0.4\textwidth}
      \only<1>{
        使用 \env{tabular} 环境绘制表格。需要添加参数(称为\textbf{表格导言})以确定每一列的对齐方式。引入 \pkg{array} 宏包来使用更多的\textbf{引导符}。
        \begin{center}
          \footnotesize
          \begin{stampbox}
            \begin{tabular}{>{\ttfamily}ll}
              \alert{l} & 向左对齐 \\
              \alert{c} & 居中对齐 \\
              \alert{r} & 向右对齐 \\
              \alert{p\{3cm\}} & 固定列宽,两端对齐 \\
              \alert{m\{3cm\}} & \texttt{p} + 垂直居中对齐 \\
              \alert{>\{\textbackslash{}bfseries\}} & 后一列单元格前加命令 \\
              \alert{*\{3\}\{l\}} & 三个左对齐列 \\
            \end{tabular}
          \end{stampbox}
        \end{center}
      }
      \only<2>{
        \pkg{booktabs} 宏包提供了标准三线表格所需要的行分割线:\cmd{toprule},\cmd{midrule},\cmd{bottomrule}。也可以使用 \cmd{cmidrule\{1-2\}} 来部分地绘制行分割线。一般不推荐在专业表格中使用纵向分割线。
      }
      \only<3>{
        每行内容使用 \textbackslash\textbackslash{} 分隔开,每行中的单元格使用 \& 分隔开。
      }
      \only<4>{
        \includepdflarge{table}
      }
    \end{column}
  \end{columns}
\end{frame}

\begin{frame}[fragile]%
  \begin{columns}
    \begin{column}{0.6\textwidth}
      \begin{codeblock}[]{表头居中}
\documentclass{ctexart}
\usepackage{array,booktabs}
\begin{document}
\begin{table}[ht]
  \centering
  \caption{||北京冬奥会吉祥物}
  \begin{tabular}{lp{3cm}}
    \toprule
|\highlightline|\multicolumn{1}{c}{||吉祥物} &
|\highlightline|\multicolumn{1}{c}{||描述} \\
    \midrule
||冰墩墩 & 2022 年北京冬季奥运会吉祥物  \\
||雪容融 & 2022 年北京冬季残奥会吉祥物  \\
    \bottomrule
  \end{tabular}
\end{table}
\end{document}
      \end{codeblock}
    \end{column}
    \begin{column}{0.4\textwidth}
      \cmd{multicolumn} 命令不仅可以用于合并同行的单元格,还可以用于临时地屏蔽表格导言对该列的对齐方式定义。这里用于居中表头。
      \begin{center}
        \begin{stampbox}
          \parbox{0.85\linewidth}{
            \ttfamily\color{blue}\textbackslash{}multicolumn\{格数\}\{对齐方式\}\{内容\}
          }
        \end{stampbox}
      \end{center}
      跨页表格考虑使用 \pkg{longtable} 宏包。带标注的表格可以考虑使用 \pkg{threeparttable} 宏包。考虑使用在线工具生成表格代码 \link{https://www.tablesgenerator.com/latex_tables}。
    \end{column}
  \end{columns}
\end{frame}

\section{数学公式}
\begin{frame}
  \frametitle{数学模式}
  \begin{alertblock}{}
  输入数学公式是 \LaTeX{} 的绝对强项,很多常见的公式服务依赖于一些轻量级渲染引擎比如 MathJax, K\kern-.3ex\raise.4ex\hbox{\footnotesize A}\kern-.3ex\TeX{}。但是它们实际上使用的是 \LaTeX{} 语法变种,也就是说并没有使用 \LaTeX{} 后端。所以不要期望有完全一致的输出。
  \end{alertblock}
  
  为了更好的获得数学公式输入支持,请使用 \hologo{AmS}math 宏包。数学模式分为两种:
  \begin{description}
    \item[行内模式] 一般通过一对美元符号(\$$\cdots$\$)标记,可以使用编辑器内建的符号表输入数学符号,也可以使用在线工具手写识别 \link{https://detexify.kirelabs.org/classify.html}。
    \item[行间模式] 一般通过 \env{equation} 环境\footnote{这是有编号环境,其加星号的变种 \env{equation*} 用于生成无编号环境。}输入。如果需要使用多行公式,请使用 \env{align} 环境,并按照类似表格输入的方式,使用 \& 对齐符号,\textbackslash\textbackslash{} 换行。如果不想手动居中,可以考虑多行自动居中的 \env{gather} 和单个大型公式首尾两端对齐 \env{multline}。
  \end{description}
\end{frame}

\begin{frame}
  \frametitle{数学命令展示}
  \begin{columns}
    \begin{column}{0.33\textwidth}
      \begin{exampleblock}{}
        $PV=nRT$
      \end{exampleblock}
      \begin{exampleblock}{}
        $\sum_{i=1}^ki^2=\frac{n(n+1)(2n+1)}{6}$
      \end{exampleblock}
      \begin{exampleblock}{}
        $T(n) = aT\left(\left\lceil\frac{n}{b}\right\rceil\right) + \mathcal{O}(n^d)$
      \end{exampleblock}
      \begin{exampleblock}{}
        $\frac{x_{1}+x_{2}+x_{3}}{3}\geq \sqrt[3]{x_{1}x_{2}x_{3}}$
      \end{exampleblock}
      \begin{exampleblock}{}
        $n=(\underbrace{1\cdots 1}_{k\text{ of 1's}})_2=2^{k+1}-1$
      \end{exampleblock}
      \begin{exampleblock}{}
        $\nabla f (P)= \left.\left(\frac{\partial f}{\partial x},\frac{\partial f}{\partial y},\frac{\partial f}{\partial z}\right)\right|_{P}$
      \end{exampleblock}
    \end{column}
    \begin{column}{0.67\textwidth}
      \begin{exampleblock}{}
        \begin{equation*}
          \int_{a}^b f(x)\,\mathrm{d}x=\lim_{|P|\rightarrow 0}\sum_{i=1}^n f(\xi_i)\Delta x_i
        \end{equation*}
      \end{exampleblock}
      \begin{exampleblock}{}
        \begin{equation}
          T(n) = \begin{cases}
            \mathcal{O}(n^d),&\textrm{if } d>\log_b a, \\
            \mathcal{O}(n^d\log n), &\textrm{if } d=\log_b a,\\
            \mathcal{O}(n^{\log_b a}), &\textrm{if } d<\log_b a.
          \end{cases}
        \end{equation}
      \end{exampleblock}
      \begin{exampleblock}{}
        \begin{align}
          Q^{T}A&=R \\
          \begin{pmatrix}
            q_1^T \\ q_2^T \\ q_3^T
          \end{pmatrix}
          \begin{pmatrix}
            a_1 & a_2 & a_3
          \end{pmatrix}
          &=R
        \end{align}
      \end{exampleblock}
    \end{column}
  \end{columns}
\end{frame}

%更深入地讲解 mathtools, unicode-math, siunix

\section{引用}
\begin{frame}[fragile]
  \frametitle{交叉引用}
  \only<1>{
    正如之前所提到的,\LaTeX{} 中使用 \cmd{label} 标记,然后可以使用 \cmd{ref} 来引用这个标记。 \cmd{label} 可以放在使用计数器的对象之后。
  }
  \only<2>{
    为了使得对公式编号的引用带有括号,推荐使用 \hologo{AmS}math 宏包中的 \cmd{eqref} 命令。对于多行公式环境,每一个换行符前都可以添加一个 \cmd{label} 用于引用该行公式。
  }
  \begin{columns}
    \begin{column}{0.5\textwidth}
      \begin{codeblock}[]{图}
\begin{figure}
|\only<1>{\highlightline}|  \caption{||示例}\label{fig:example}
\end{figure}
      \end{codeblock}
      \begin{codeblock}[]{表}
\begin{table}
|\only<1>{\highlightline}|  \caption{||示例}\label{tab:example}
\end{table}
      \end{codeblock}
    \end{column}
    \begin{column}{0.5\textwidth}
\begin{codeblock}[]{目次}
|\only<1>{\highlightline}|\section{||示例}\label{sec:example}
\end{codeblock}

\begin{codeblock}[]{公式}
\begin{equation}
  a = b + c
|\only<1>{\highlightline}|\label{eq:example}
\end{equation}
|\only<2>{\highlightline}|如公式 \eqref{eq:example} 所示,
\end{codeblock}
    \end{column}
  \end{columns}
\end{frame}

\begin{frame}[fragile]
  \frametitle{文献引用}
  \LaTeX{} 管理参考文献可以采用专用数据库文件 \texttt{.bib},很多的文献管理文件比如 EndNote \link{https://lic.sjtu.edu.cn/Default/softshow/tag/MDAwMDAwMDAwMLGImKE}, Zotero \link{https://www.zotero.org/}, JabRef \link{https://www.jabref.org/} 都可以直接导出这种格式的文件用于 \LaTeX{} 论文中的引用。一般不需要手写数据库文件,你只需要注意每一个文献会在数据库中有一个主键,这个类似于上文的 \cmd{label} 标记,只是要引用数据库中的文献需要使用 \cmd{cite} 命令。
  
  \begin{codeblock}[]{ref.bib}
|\highlightline|@phdthesis{devoftech,|\hfill\alert{\% 类型为博士论文,主键为\texttt{devoftech}}|
  title={||新时期我国信息技术产业的发展},
  author={||江泽民},
  year={2008}
}
  \end{codeblock}
\end{frame}

\begin{frame}
  \frametitle{文献引用}
  而让 \LaTeX{} 处理 \texttt{.bib} 数据库文件与引用有两种工作流。你可能需要查清楚如何在编辑器中设置对应的工作流,或者采用后面所提到的高级编译方式 \texttt{latexmk}。
  \begin{columns}
    \begin{column}{0.5\textwidth}
      \begin{block}{\hologo{BibTeX} + \pkg{gbt7714}}
        一般期刊提交使用这种方法,\pkg{natbib} 宏包将提供命令 \cmd{citet}(文本引用) 和 \cmd{citep}(括号引用)。中文引用可以直接使用 \pkg{gbt7714} 宏包,但是角模式和正文模式不能混用。
      \end{block}
    \end{column}
    \begin{column}{0.5\textwidth}
      \begin{block}{\hologo{biber} + \pkg{biblatex}}
        这是更容易自定义的方法,与 \hologo{BibTeX} 的运作方式稍有不同。\pkg{biblatex} 提供了更加智能的引用命令。而中文引用可以使用 \pkg{biblatex} 宏包的样式 \pkg{gb7714-2015},使用该样式需要使用 \hologo{XeLaTeX} 编译。
      \end{block}
    \end{column}
  \end{columns}
\end{frame}

\begin{frame}[fragile]
  \frametitle{文献引用}
  \begin{columns}
    \begin{column}{0.5\textwidth}
      \begin{codeblock}[]{\hologo{BibTeX} + \pkg{gbt7714}}
\documentclass{ctexart}
\usepackage{gbt7714}
\bibliographystyle{gbt7714-numerial}
% \citestyle{numbers}  % 正文模式
\begin{document}
  ||他指出了...\cite{devoftech}
  \bibliography{ref}
\end{document}
      \end{codeblock}
    \end{column}
    \begin{column}{0.5\textwidth}
      \begin{codeblock}[]{\hologo{biber} + \pkg{biblatex}}
\documentclass{ctexart}
\usepackage[backend=biber,style=gb7714-2015]{biblatex}
\addbibresource{ref.bib}
\begin{document}
  ||他在文献 \parencite{devoftech}
  ||指出了...\cite{devoftech}
  \printbibliography
\end{document}
      \end{codeblock}
    \end{column}
  \end{columns}
\end{frame}

\begin{frame}
  \frametitle{文献引用}
  \begin{columns}
    \begin{column}{0.5\textwidth}
      \includepdflarge{bibtex}
    \end{column}
    \begin{column}{0.5\textwidth}
      \includepdflarge{biblatex}
    \end{column}
  \end{columns}
\end{frame}

} % End of customized shaded number logo

|\only<3>{\highlightline}|  % !TeX root = ..\..\latex-talk.tex

\part{SJTUThesis}

\begin{frame}
  \frametitle{简介}
  \begin{columns}
    \begin{column}{0.6\textwidth}
      \begin{itemize}
        \item 最早由韦建文于 2009 年 11 月发布 0.1a 版,2018 年起由 SJTUG 接手维护
        \item 最新版:\SJTUThesisVersion{} (\SJTUThesisDate)
        \item 支持本科、硕士、博士学位论文以及课程论文的排版
      \end{itemize}
    \end{column}
    \begin{column}{0.4\textwidth}
      \begin{exampleblock}{}
        \begin{minipage}[c]{1cm}
          \includegraphics[width=0.8cm]{\getcontribpath{sjtug}{vi/sjtug}}
        \end{minipage}
        \begin{minipage}[c]{2cm}
          \href{https://github.com/sjtug}{sjtug}/\href{https://github.com/sjtug/SJTUThesis}{SJTUThesis}
        \end{minipage}
      \end{exampleblock}
      \vspace{-8pt}
      \begin{block}{}
        \scriptsize
        上海交通大学 \hologo{XeLaTeX} 学位论文及课程论文模板 | Shanghai Jiao Tong University \hologo{XeLaTeX} Thesis Template
      \end{block}
      \vspace{-8pt}
      \begin{alertblock}{}
        \scriptsize
        \begin{tabular}{cl}
          \faStar & 2.4k \\
          \faEye & 55 \\
          \faCodeBranch & 701 \\
        \end{tabular}
      \end{alertblock}
    \end{column}
  \end{columns}
\end{frame}

\begin{frame}
  \frametitle{下载与编译}
  \alert{下载} 推荐安装 Git \link{https://git-scm.com/} 后,克隆 SJTUG 镜像仓库
  \begin{exampleblock}{\faGit*}
    \ttfamily\small
    git clone https://mirror.sjtu.edu.cn/git/SJTUThesis.git/
  \end{exampleblock}

  \alert{编译} 推荐使用 \pkg{latexmk} 编译\footnote{\hologo{MiKTeX} 用户需要手动安装 Perl 解释器 \link{https://www.perl.org/get.html} 才能使用 \pkg{latexmk}。},在不能够利用自带的 \texttt{.latexmkrc} 配置文件的情况下,需要查清楚在对应的编辑器中如何使用 \hologo{XeLaTeX} + \hologo{biber} 编译 \link{https://github.com/sjtug/SJTUThesis/blob/master/README.md}。
  \begin{exampleblock}{\faTerminal}
    \ttfamily\small
    latexmk -xelatex main
  \end{exampleblock}

  Overleaf 用户可以下载压缩包后,上传并采用 \hologo{XeLaTeX} 编译方式。
\end{frame}

\begin{frame}
  \frametitle{手动编译}
  \alert{第一次编译失败} 如果没有办法通过通常方式编译成功,请尝试使用文件夹内附带 \faLinux{}\,\faApple{} \texttt{Makefile} 和 \faWindows{} \texttt{Compile.bat} 进行编译。

  \alert{统计字数} 编写过程中也可以使用对应的命令调用 \TeX{}count 来统计正文字数。
  \begin{columns}
    \begin{column}{0.38\textwidth}
      \begin{exampleblock}{\faLinux{}\,\faApple}
        \ttfamily
        make all\\
        make clean\\
        make cleanall\\
        make wordcount
      \end{exampleblock}
    \end{column}
    \begin{column}{0.38\textwidth}
      \begin{exampleblock}{\faWindows}
        \ttfamily
        ./Compile.bat thesis\\
        ./Compile.bat clean\\
        ./Compile.bat cleanall\\
        ./Compile.bat wordcount
      \end{exampleblock}
    \end{column}
    \begin{column}{0.24\textwidth}
      \begin{block}{\faInfo}
        \ttfamily
        编译论文\\
        清理中间文件\\
        $\hookrightarrow +$删除论文\\
        统计字数
      \end{block}
    \end{column}
  \end{columns}
\end{frame}

\begin{frame}[label=compile]
  \frametitle{编译问题排查}
  \begin{columns}
    \begin{column}{0.33\textwidth}
      \begin{alertblock}{无法使用 \texttt{latexmk}\thesisissue{578}}
        \hologo{MiKTeX} 需要安装 Perl 解释器。
      \end{alertblock}  
      \begin{alertblock}{C\TeX{} 套装无法编译\thesisissue{446}}
        使用最新 \TeX{} 发行版。
      \end{alertblock}
      \begin{alertblock}{\hologo{pdfLaTeX} 无法编译\thesisissue{444}}
        请使用 \texttt{latexmk},或更改编辑器设置以 \hologo{XeLaTeX} 编译。
      \end{alertblock}
    \end{column}
    \begin{column}{0.33\textwidth}
      \begin{alertblock}{缺少字体\thesisissue{564} \thesisdiscuss{598}}
        更换字体集,或者安装对应字体。
      \end{alertblock}
      \begin{alertblock}{缺少汉字\thesisissue{533} \thesisdiscuss{617}}
        去除使用 fandol 字体集的设定。或者是安装字体后,改用 \texttt{fontset=adobe} 或 \texttt{fontset=founder}。
      \end{alertblock}
    \end{column}
    \begin{column}{0.33\textwidth}
      \begin{block}{\faInfoCircle{} README}
        不同编辑器的设置请首先参阅 README \link{https://github.com/sjtug/SJTUThesis/blob/master/README.md} 文档。
      \end{block}
      \begin{block}{\faBookOpen{} Wiki}
        其他编译问题推荐查阅 Wiki \link{https://github.com/sjtug/SJTUThesis/wiki} 的使用说明部分。
      \end{block}
    \end{column}
  \end{columns}
\end{frame}

\begin{frame}[fragile, label=covers]
  \begin{codeblock}[firstnumber=3]{main.tex}
|\alert{\% 载入 SJTUThesis 模版}|
\documentclass[|\only<1>{\highlight{type}}\only<2>{type}|=|\only<1>{bachelor}\only<2>{\highlight{bachelor}}|]{sjtuthesis}
  \end{codeblock}
  \begin{figure}
    \parbox{0.9\textwidth}{
      \begin{subfigure}{0.20\textwidth}
        \framebox{\includegraphics[width=\linewidth]{support/thesis/bachelor}}
        \caption{\only<1>{学士}\only<2>{\texttt{bachelor}}}
      \end{subfigure}\hfill
      \begin{subfigure}{0.20\textwidth}
        \framebox{\includegraphics[width=\linewidth]{support/thesis/master}}
        \caption{\only<1>{硕士}\only<2>{\texttt{master}}}
      \end{subfigure}\hfill
      \begin{subfigure}{0.20\textwidth}
        \framebox{\includegraphics[width=\linewidth]{support/thesis/doctor}}
        \caption{\only<1>{博士}\only<2>{\texttt{doctor}}}
      \end{subfigure}\hfill
      \begin{subfigure}{0.20\textwidth}
        \framebox{\includegraphics[width=\linewidth]{support/thesis/course}}
        \caption{\only<1>{课程}\only<2>{\texttt{course}}}
      \end{subfigure}
      \caption{论文类型示例\only<2>{ \texttt{type}}}
    }
  \end{figure}
\end{frame}

\begin{frame}[fragile]
  \frametitle{文档类选项}
  % \framesubtitle{\textbackslash{}documentclass\{sjtuthesis\}}
  \begin{columns}
    \begin{column}{0.45\textwidth}
      \includegraphics[page=10]{thesisdir}
    \end{column}
    \begin{column}{0.55\textwidth}
      \begin{table}[H]
        \caption{文档类选项}
        \footnotesize
        \begin{tabular}{>{\ttfamily}rll}
          \toprule
          选项 & 含义 & 相关 \\
          \midrule
          type= & 指定论文类型 & 第 \ref{covers} 页\\
          fontset= & 指定字体 & 第 \ref{compile} 页\\
          \midrule
          review & 开启盲审模式 & \thesisissue{195} \thesisissue{686} \\
          twoside & 双页模式 & \thesisissue{554} \\
          oneside & 单页模式 & \thesisissue{694} \\
          openright & 章从奇数页开始 & \thesisdiscuss{724} \\
          openany & 章从任意页开始 & \thesisissue{446} \\
          \bottomrule
        \end{tabular}
      \end{table}
    \end{column}
  \end{columns}
\end{frame}

\begin{frame}[fragile]
  \frametitle{基本配置}
  \framesubtitle{\textbackslash{}input\{setup\}}
  \begin{columns}
    \begin{column}{0.45\textwidth}
      \includegraphics[page=9]{thesisdir}
    \end{column}
    \begin{column}{0.55\textwidth}
      \begin{codeblock}[firstnumber=12]{main.tex}
|\highlightline<1>|% 论文基本配置,加载宏包等全局配置
|\highlightline<1>|\input{setup}

\begin{document}

%TC:ignore

|\highlightline<2>|% 标题页
|\highlightline<2>|\maketitle
      \end{codeblock}
      \visible<2>{
        \cmd{sjtusetup} 中的 \pkg{info} 将会修改封面的信息设置(见第 \ref{covers} 页)。
      }
    \end{column}
  \end{columns}
\end{frame}

\begin{frame}[fragile]
  \frametitle{基本配置}
  \framesubtitle{\textbackslash{}sjtusetup}
  \begin{columns}
    \begin{column}{0.45\textwidth}
      \includegraphics[page=1]{thesisdir}
    \end{column}
    \begin{column}{0.55\textwidth}
      \begin{codeblock}[firstnumber=3]{setup.tex}
\sjtusetup{
  info = {
    title    = {||上海交通大学学位论文 \LaTeX{} 模板示例文档},
    title*   = {A Sample for \LaTeX-based SJTU Thesis Template},
    author   = {||某\quad{}某},
    author* = {Mo Mo},
  },
  style = { header-logo-color = red, 
  },
  name = {
    publications = {||攻读学位期间完成的论文},
  },
}
      \end{codeblock}
    \end{column}
  \end{columns}
\end{frame}

\begin{frame}
  \frametitle{基本配置}
  \framesubtitle{\textbackslash{}sjtusetup}
  \begin{columns}
    \begin{column}{0.45\textwidth}
      \includegraphics[page=1]{thesisdir}
    \end{column}
    \begin{column}{0.55\textwidth}
      \begin{table}[H]
        \centering
        \caption{info 域}
        \footnotesize
        \begin{tabular}{lll} \toprule
          命令作用 & 中文对应选项 & 英文对应选项 \\ \midrule
          论文标题 & \texttt{title} & \texttt{title*} \\
          关键字列表 & \texttt{keywords} & \texttt{keywords*} \\
          作者姓名&  \texttt{author} &\texttt{author*}\\
          申请学位名称 & \texttt{degree}&\texttt{degree*}\\
          院系名称 & \texttt{department} & \texttt{department*}\\
          专业名称 & \texttt{major} & \texttt{major*}\\
          导师 & \texttt{supervisor} & \texttt{supervisor*}\\
          副导师 & \texttt{assisupervisor} & \texttt{assisupervisor*}\\
          日期 & \multicolumn{2}{c}{\texttt{date}}\\
          学号 & \multicolumn{2}{c}{\texttt{id}}\\ \bottomrule
          \end{tabular}
      \end{table}
    \end{column}
  \end{columns}
\end{frame}

\begin{frame}[fragile]
  \frametitle{版权页}
  \framesubtitle{\textbackslash{}copyrightpage}
  \begin{columns}
    \begin{column}{0.45\textwidth}
      \only<1>{
        \includegraphics[page=9]{thesisdir}
      }
      \only<2>{
        \includegraphics[page=2]{thesisdir}
      }
      \only<3>{
        \begin{figure}[H]
          \framebox{\includegraphics[page=2,width=0.4\linewidth]{bachelor}}
          \caption{版权页}
        \end{figure}
      }
    \end{column}
    \begin{column}{0.55\textwidth}
      \begin{codeblock}[firstnumber=22]{main.tex}
|\highlightline<1>|% 原创性声明及使用授权书
|\highlightline<1>|\copyrightpage
|\highlightline<2>|% 插入外置原创性声明及使用授权书
|\highlightline<2>|% \copyrightpage[scans/sample-copyright-old.pdf]
      \end{codeblock}
      \only<1>{
        \cmd{copyrightpages} 可以用于插入版权页。
      }
      \only<2>{
        \cmd{copyrightpages} 也接受一个可选参数,用于直接使用扫描件。
      }
    \end{column}
  \end{columns}
\end{frame}

\begin{frame}[fragile]
  \frametitle{前置部分}
  \framesubtitle{\textbackslash{}frontmatter}
  \begin{columns}
    \begin{column}{0.45\textwidth}
      \only<1>{
        \includegraphics[page=9]{thesisdir}
      }
      \only<2>{
        \includegraphics[page=3]{thesisdir}
      }
      \only<3>{
        \begin{figure}[H]
          \begin{subfigure}{0.45\textwidth}
            \framebox{\includegraphics[page=3,width=\linewidth]{bachelor}}
            \caption{中文}
          \end{subfigure}\hfill
          \begin{subfigure}{0.45\textwidth}
            \framebox{\includegraphics[page=4,width=\linewidth]{bachelor}}
            \caption{英文}
          \end{subfigure}
          \caption{摘要}
        \end{figure}
      }
      \only<4>{
        \begin{figure}[H]
          \begin{subfigure}{0.30\linewidth}
            \centering
            \framebox{\includegraphics[page=5,width=0.6\linewidth]{bachelor}}
            \caption{目录}
          \end{subfigure}
          \begin{subfigure}{0.30\linewidth}
            \centering
            \framebox{\includegraphics[page=6,width=0.6\linewidth]{bachelor}}
            \caption{插图}
          \end{subfigure}

          \begin{subfigure}{0.30\linewidth}
            \centering
            \framebox{\includegraphics[page=7,width=0.6\linewidth]{bachelor}}
            \caption{表格}
          \end{subfigure}
          \begin{subfigure}{0.30\linewidth}
            \centering
            \framebox{\includegraphics[page=8,width=0.6\linewidth]{bachelor}}
            \caption{算法}
          \end{subfigure}
          \caption{索引}
        \end{figure}
      }
      \only<5>{
        \includegraphics[page=4]{thesisdir}
      }
      \only<6>{
        \begin{figure}[H]
          \framebox{\includegraphics[page=9,width=0.5\linewidth]{bachelor}}
          \caption{符号对照表}
        \end{figure}
      }
    \end{column}
    \begin{column}{0.55\textwidth}
      \begin{codeblock}[firstnumber=30]{main.tex}
|\highlightline<2-3>|% 摘要
|\highlightline<2-3>|\input{contents/abstract}

|\highlightline<4>|% 目录
|\highlightline<4>|\tableofcontents
|\highlightline<4>|% 插图索引
|\highlightline<4>|\listoffigures*
|\highlightline<4>|% 表格索引
|\highlightline<4>|\listoftables*
|\highlightline<4>|% 算法索引
|\highlightline<4>|\listofalgorithms*

|\highlightline<5-6>|% 符号对照表
|\highlightline<5-6>|\input{contents/nomenclature}
      \end{codeblock}
    \end{column}
  \end{columns}
\end{frame}

\begin{frame}[fragile]
  \frametitle{主体部分}
  \framesubtitle{\textbackslash{}mainmatter}
  \begin{columns}
    \begin{column}{0.45\textwidth}
      \only<1>{
        \includegraphics[page=5]{thesisdir}
      }
      \only<2>{
        \begin{figure}[H]
          \begin{subfigure}{0.30\linewidth}
            \centering
            \framebox{\includegraphics[page=11,width=0.6\linewidth]{bachelor}}
            \caption{简介}
          \end{subfigure}
          \begin{subfigure}{0.30\linewidth}
            \centering
            \framebox{\includegraphics[page=13,width=0.6\linewidth]{bachelor}}
            \caption{数学}
          \end{subfigure}

          \begin{subfigure}{0.30\linewidth}
            \centering
            \framebox{\includegraphics[page=16,width=0.6\linewidth]{bachelor}}
            \caption{浮动体}
          \end{subfigure}
          \begin{subfigure}{0.30\linewidth}
            \centering
            \framebox{\includegraphics[page=22,width=0.6\linewidth]{bachelor}}
            \caption{总结}
          \end{subfigure}
          \caption{主体部分}
        \end{figure}
      }
    \end{column}
    \begin{column}{0.55\textwidth}
      \begin{codeblock}[firstnumber=47]{main.tex}
|\highlightline|% 正文内容
|\highlightline|% !TeX root = ../../../latex-talk.tex

\section{是什么}

\begin{frame}
  \frametitle{\TeX{}}
  \begin{columns}[c]
    \begin{column}{0.7\textwidth}
      \begin{center}
        \rmfamily\Huge
        \highlight[structure]{\TeX{}}
      \end{center}
      \begin{center}
        \parbox{0.75\textwidth}{
          \TeX{} 是由斯坦福大学教授高德纳
          (Donald E.~Knuth)于 1977 年开始开发的排版引擎。目前仍在更新,最新版本号为 3.141592653 \link{https://tug.org/TUGboat/tb42-1/tb130knuth-tuneup21.pdf}。
        }
      \end{center}
    \end{column}
    \begin{column}{0.3\textwidth}
      \includegraphics[width=.8\columnwidth]{support/images/Knuth.jpg}
    \end{column}
  \end{columns}
  \note{\emph{这一部分背景介绍大家可以了解一下,暂时跳过。}
  \LaTeX{} 这个词由两个部分组成,\hologo{La} 和 \TeX{}。那我们首先了解一下 \TeX{} 是什么。
  \TeX{} 是由斯坦福大学的教授高德纳于 1977 年开始开发的排版引擎,它已经有三十多年的历史了,
  目前仍在更新,版本号(3.141592653)将会趋近于 $\pi$ 的取值,高德纳最近还在给 \textsl{TUGBoat} 写稿子
  \link{https://tug.org/TUGboat/tb42-1/tb130knuth-tuneup21.pdf},
  关于 \TeX{} 今年又做了哪些改进。}
\end{frame}

\begin{frame}
  \frametitle{\LaTeX{}}
  \begin{columns}[c]
    \begin{column}{0.7\textwidth}
      \begin{center}
        \rmfamily\Huge
        \highlight[structure]{\LaTeX{}}
      \end{center}
      \begin{center}
        \parbox{0.75\textwidth}{
          \LaTeX{} 是最早在 1985 年由现就职于微软的 Leslie Lamport 开发的一种 \TeX{} \textbf{格式}\footnotemark,使用一些列宏和扩展宏包来简化 \TeX{} 的使用。现在由 \LaTeX{} Project 的成员维护。现在广泛使用的版本是 \LaTeXe{},最新的版本为 \LaTeX3(2020 年 10 月后默认内置)。
        }
      \end{center}
    \end{column}
    \begin{column}{0.3\textwidth}
      \includegraphics[width=.8\columnwidth]{support/images/Lamport.jpg}
    \end{column}
  \end{columns}
  \footnotetext{\hologo{ConTeXt} 也是一种 \TeX{} 格式 \link{https://www.contextgarden.net/}。}
  \note{\emph{这一部分的背景介绍大家可以了解一下,暂时跳过。}
  \LaTeX{} 是最早由现就职于微软的 Leslie Lamport 开发的一种 \TeX{} 格式(与其对标的是
  \hologo{ConTeXt}\link{https://www.contextgarden.net/}),主要也是为了简化 \TeX{} 的使用。
  现在主要由 \LaTeX{} 开发组维护,现在广泛使用的版本是 \LaTeXe{},最新的版本为 \LaTeX3,
  在 2020 年 10 月后默认内置,所以要尽可能使用较新的发行版,以充分发挥其功能。}
\end{frame}

\begin{frame}
  \frametitle{程序}
  \begin{columns}[c]
    \begin{column}{0.7\textwidth}
      \begin{center}
        \rmfamily\Huge
        \highlight[structure]{\hologo{pdfLaTeX}}
      \end{center}
      \begin{center}
        \parbox{0.7\textwidth}{
          \hologo{pdfLaTeX} 是为了编译一个 \LaTeX{} 文档而运行的程序。实际上底层在运行一个叫 \hologo{pdfTeX} 的引擎,并预装了对应的 \LaTeX{} \textbf{格式}。为了利用临时文件,可能就需要多次运行程序。
        }
      \end{center}
    \end{column}
    \begin{column}{0.3\textwidth}
      \begin{block}{}
        \ttfamily\small
        > \highlight{pdflatex} main.tex\\
        This is pdfTeX, Version 3.141592653-
        2.6-1.40.23 (MiKTeX 21.10)\\
        entering extended mode\\
        \highlight{LaTeX2e} <2021-11-15>\\
        \highlight{L3} programming layer <2021-11-22>
      \end{block}
    \end{column}
  \end{columns}
  \note{\hologo{pdfLaTeX} 是为了编译一个 \LaTeX{} 文档而运行的程序。}
\end{frame}

% \begin{frame}
%   \frametitle{引擎}
%   \begin{columns}[c]
%     \begin{column}{0.7\textwidth}
%       \begin{center}
%         \rmfamily\Huge
%         \highlight[structure!70]{pdf}\hologo{La}\highlight[structure!70]{\TeX{}}
%       \end{center}
%       \begin{center}
%         \parbox{0.7\textwidth}{
%           pdf\TeX{} 是编译 \TeX{} 文档(以 \texttt{.tex} 结尾)的\textbf{引擎}---可以理解 \TeX{} 指令的\textbf{程序}。
%         }
%       \end{center}
%     \end{column}
%     \begin{column}{0.3\textwidth}
%       \begin{block}{}
%         \ttfamily\small
%         > pdflatex main.tex\\
%         This is \highlight[structure!70]{pdfTeX}, Version 3.141592653-
%         2.6-1.40.23 (MiKTeX 21.10)
%         entering extended mode\\
%         LaTeX2e <2021-11-15>\\
%         L3 programming layer <2021-11-22>
%       \end{block}
%     \end{column}
%   \end{columns}
%   \note{实际上底层在运行一个叫 \hologo{pdfTeX} 的引擎,并预装了对应的 \LaTeX{} 格式。}
% \end{frame}

\begin{frame}[label={frame:engine}]
  \frametitle{程序}
  \begin{table}
    \caption{主流 \hologo{(La)TeX} 程序
    \footnote{(u)p\TeX{} 是日语最常用的引擎,生成 \texttt{.dvi},支持 Unicode。}\footnote{Ap\TeX{} \link{https://github.com/clerkma/ptex-ng} 具有底层 CJK 支持,内联 Ruby,Color Emoji。}}
    \footnotesize
    \begin{stampbox}
      \begin{tabular}{c>{\raggedright}*{3}{p{3.5cm}}}
        \alert{引擎}     & \hologo{pdfTeX}   & \hologo{XeTeX}   & \hologo{LuaTeX}   \\
        \alert{程序}     & \hologo{pdfLaTeX} & \hologo{XeLaTeX} & \hologo{LuaLaTeX} \\
        \alert{特点}     & 直接生成 PDF,支持 micro-typography  & 支持 Unicode、OpenType 与复杂文字编排 (CTL) & 支持 Unicode,内联 Lua,支持 OpenType \\
      \end{tabular}
    \end{stampbox}
  \end{table}

  \begin{center}
    \parbox{.9\textwidth}{
      \hologo{pdfLaTeX} 不支持 Unicode。为了排版中文,大部分情况下应当使用 \hologo{XeLaTeX},而 \hologo{LuaLaTeX} 速度相对较慢。\faWindows{} 可以在一些情况下使用 \hologo{pdfLaTeX}。
    }
  \end{center}
  \note{当然为了排版中文,已经不再推荐使用 \hologo{pdfLaTeX} 了,应该使用
  \hologo{XeLaTeX} 或者 \hologo{LuaLaTeX},当然后者的速度还是相对较慢,
  它们支持 Unicode 编码,并可以使用 OpenType 字体的全部功能。
  当然 \faWindows{} 平台下在某些追求速度的情况下,
  还是可以试着使用 \hologo{pdfLaTeX} 的。

  \hologo{LuaLaTeX} 理想情况下不慢,但是使用一些宏包后会破坏理想状态,
  也会因配置产生不同的结果,不同的操作系统在 I/O 速度上的不同也会导致不同的时间。

  \hologo{pdfLaTeX} 也支持,只不过需要先生成 tfm \TeX{} 字体度量文件,后续使用 \TeX{}
  自身的配置方法,只能使用 7 比特或 8 比特字体。}
\end{frame}

% \begin{frame}
%   \paragraph{\hologo{pdfLaTeX}} \TeX{} 和 \LaTeX{} 被广泛使用之前,它们只需内置支持欧洲语言即可。在 Unicode 出现之前,\LaTeX{} 提供了许多种\textbf{文件编码}来允许很多语言的文字以原生的方式输入,\hologo{pdfLaTeX} 也只需要使用 8 位文件编码和 8 位字体。
% \end{frame}


|\highlightline|\input{contents/math_and_citations}
|\highlightline|\input{contents/floats}
|\highlightline|\input{contents/summary}

%TC:ignore

% 参考文献
\printbibliography[heading=bibintoc]
      \end{codeblock}
    \end{column}
  \end{columns}
\end{frame}

\begin{frame}
  \frametitle{数学}
  \begin{itemize}
    \item 公式示例:\nolinkurl{contents/math_and_citations.tex}
    \item \SJTUThesis{} 定义了常用的数学环境(需要手工引入 \texttt{ntheorem} 宏包):
      \begin{table}[h]
        \centering
        \footnotesize
        \begin{tabular}{*{7}{l}}\toprule
          assumption  & axiom   & conjecture & corollary    & definition  & example   & exercise  \\
          假设        & 公理    & 猜想       & 推论         & 定义        & 例        & 练习      \\\midrule
          lemma       & problem & proof      & proposition  & remark      & solution  & theorem   \\
          引理        & 问题    & 证明       & 命题         & 注          & 解        & 定理      \\\bottomrule
        \end{tabular}
      \end{table}
      \item \SJTUThesis{} 可以通过 \texttt{unimath} 选项使用 \pkg{unicode-math} 进行数学输入,注意与传统方式的区别。\thesisissue{555}
  \end{itemize}
\end{frame}

\begin{frame}[fragile]
  \frametitle{参考文献}
  \begin{columns}
    \begin{column}{0.45\textwidth}
      \includegraphics[page=6]{thesisdir}
    \end{column}
    \begin{column}{0.55\textwidth}
      \begin{codeblock}[firstnumber=111,numbersep=2pt]{setup.tex}
% 使用 BibLaTeX 处理参考文献
%   biblatex-gb7714-2015 常用选项
%     gbnamefmt=lowercase     姓名大小写由输入信息确定
%     gbpub=false             禁用出版信息缺失处理
\usepackage[backend=biber,style=gb7714-2015]{biblatex}
% 文献表字体
% \renewcommand{\bibfont}{\zihao{-5}}
% 文献表条目间的间距
\setlength{\bibitemsep}{0pt}
|\highlightline|% 导入参考文献数据库
|\highlightline|\addbibresource{bibdata/thesis.bib}
      \end{codeblock}
    \end{column}
  \end{columns}
\end{frame}

\begin{frame}[fragile]
  \frametitle{附录}
  \framesubtitle{\textbackslash{}appendix}
  \begin{columns}
    \begin{column}{0.45\textwidth}
      \only<1>{
        \includegraphics[page=7]{thesisdir}
      }
      \only<2>{
        \begin{figure}[H]
          \begin{subfigure}{0.45\linewidth}
            \framebox{\includegraphics[width=\linewidth,page=24]{bachelor}}
            \caption{}
          \end{subfigure}\hfill
          \begin{subfigure}{0.45\textwidth}
            \framebox{\includegraphics[width=\linewidth,page=25]{bachelor}}
            \caption{}
          \end{subfigure}
          \caption{附录}
        \end{figure}
      }
    \end{column}
    \begin{column}{0.55\textwidth}
      \begin{codeblock}[firstnumber=61]{main.tex}
% 附录中图表不加入索引
\captionsetup{list=no}

% 附录内容
|\highlightline|\input{contents/app_maxwell_equations}
|\highlightline|\input{contents/app_flow_chart}
      \end{codeblock}
    \end{column}
  \end{columns}
\end{frame}

\begin{frame}[fragile]
  \frametitle{结尾部分}
  \framesubtitle{\textbackslash{}backmatter}
  \begin{columns}
    \begin{column}{0.45\textwidth}
      \only<1>{
        \includegraphics[page=8]{thesisdir}
      }
      \only<2>{
        \begin{figure}[H]
          \begin{subfigure}{0.30\linewidth}
            \centering
            \framebox{\includegraphics[page=26,width=0.6\linewidth]{bachelor}}
            \caption{致谢}
          \end{subfigure}
          \begin{subfigure}{0.30\linewidth}
            \centering
            \framebox{\includegraphics[page=27,width=0.6\linewidth]{bachelor}}
            \caption{成就}
          \end{subfigure}

          \begin{subfigure}{0.30\linewidth}
            \centering
            \framebox{\includegraphics[page=28,width=0.6\linewidth]{bachelor}}
            \caption{简历}
          \end{subfigure}
          \begin{subfigure}{0.30\linewidth}
            \centering
            \framebox{\includegraphics[page=29,width=0.6\linewidth]{bachelor}}
            \caption{大摘要*}
          \end{subfigure}
          \caption{结尾部分}
        \end{figure}
      }
    \end{column}
    \begin{column}{0.55\textwidth}
      \begin{codeblock}[firstnumber=76]{main.tex}
% 致谢
\input{contents/acknowledgements}

% 发表论文及科研成果
% 盲审论文中,发表论文及科研成果等仅以第几作者注明即可,不要出现作者或他人姓名
\input{contents/achievements}

% 简历
\input{contents/resume}

% 学士学位论文要求在最后有一个大摘要,单独编页码
\input{contents/digest}
      \end{codeblock}
    \end{column}
  \end{columns}
\end{frame}

\begin{frame}
  \frametitle{还有其他问题?}
  \begin{columns}
    \begin{column}{0.75\textwidth}
    \begin{itemize}
      \item[{\faComment*[regular]}] 日常模板或 \LaTeX{} 使用问题可以前往 Discussions \link{https://github.com/sjtug/SJTUThesis/discussions} 提问
      
      (解决后别忘了 \textcolor{green}{\faCheckCircle{} Mark as answer}
      \item[{\faDotCircle[regular]}] 如果是 \textsc{SJTUThesis} 项目本身的 bug 和 feature request
      
      可以通过 Issues \link{https://github.com/sjtug/SJTUThesis/issues} 反馈。
      \item[{\faCodeBranch}] 如果你有好点子,可以贡献代码
     
      向 \textsc{SJTU\TeX{}}(v1) \link{https://github.com/sjtug/SJTUTeX/tree/v1} 存储库发 PR,\par
      而后把解包结果同步到 \textsc{SJTUThesis}。
  
      \item[{\faTag}] 如果你对正在基于 \LaTeX3 开发的新版感兴趣,\par
      也欢迎向 \textsc{SJTU\TeX{}}(v2) \link{https://github.com/sjtug/SJTUTeX/tree/v2} 发 PR。
  
      \item[{\faQq}] 也欢迎在 QQ 群即时讨论。
    \end{itemize}
    \end{column}
    \begin{column}{0.25\textwidth}
      \includegraphics[height=0.7\textheight]{qq.jpg}
    \end{column}
  \end{columns}
\end{frame}
\end{document}
      \end{codeblock}
    \end{column}
  \end{columns}
\end{frame}

\begin{frame}[fragile]
  \frametitle{组织文档}
  \begin{columns}
    \begin{column}{0.4\textwidth}
      \begin{codeblock}[]{learnlatex.tex}
|\highlightline|\chapter{||学习 \LaTeX{}}
\section{||概念}
\subsection{\LaTeX{}}
\LaTeX{} 是一个用以排版高质量作品的文档准备系统。
      \end{codeblock}
      子文件中就不需要添加 \env{document} 环境了\footnotemark。
    \end{column}
    \begin{column}{0.6\textwidth}
      \begin{codeblock}[]{主文档}
|\highlightline|\documentclass{ctexrep}
\includeonly{learnlatex,sjtuthesis}
\begin{document}
  \tableofcontents
  % !TeX root = ..\..\latex-talk.tex

\part{学习 \LaTeX{}}
% FIXME: Part Page miniframe overflow
% FIXME: footnote fault numbering

\begin{frame}[plain]
  \vfil
  \begin{center}
    \href{https://learnlatex.org}{
      \rmfamily
      Learn\,\lower1ex\hbox{\Huge\LaTeX{}}.org
    }
  \end{center}
  \vfil
  \begin{center}
    \parbox{0.75\linewidth}{
      Learn\LaTeX{}.org\cite{learnlatex} 提供了解 \LaTeX{} 的 16 篇简短的教程,并包含了一些可以在线运行的示例,可以通过亲自动手查看实验效果。本部分主要参考由 C\TeX{}-org 提供的中文翻译版本 \link{https://github.com/CTeX-org/learnlatex.github.io/tree/zh-Hans/zh-Hans/}。
    }
  \end{center}
  \vfil
\end{frame}

{ % Start of shaded number logo

\newcommand{\shadedfont}[2][1pt]{
  % #1 (optional): shadow distance
  % #2: the text needed to be shaded
  \hbox{\rlap{\color{gray}\hskip#1#2}#2}
}
\newcounter{learnsec}
\setcounter{learnsec}{-1}
\newcommand{\updatelogo}{
  % update the logo corresponding to current counter.
  \stepcounter{learnsec}
  \logo{
    \raise.3ex\hbox{\tiny\insertsection}\shadedfont{\arabic{learnsec}}
  }
}
\let\oldsection=\section
\renewcommand{\section}[1]{\oldsection{#1}\updatelogo}

\section{是什么}
\begin{frame}
  \frametitle{\TeX{}}
  \begin{columns}[c]
    \begin{column}{0.7\textwidth}
      \begin{center}
        \rmfamily\Huge
        \hologo{La}\highlight[structure!70]{\TeX{}}
      \end{center}
      \begin{center}
        \parbox{0.75\textwidth}{
          \TeX{} 是由斯坦福大学教授高德纳
          (Donald E.~Knuth)于 1977 年开始开发的排版引擎。目前仍在更新,最新版本号为 3.141592653 \link{https://tug.org/TUGboat/tb42-1/tb130knuth-tuneup21.pdf}。
        }
      \end{center}
    \end{column}
    \begin{column}{0.3\textwidth}
      \includegraphics[width=.8\columnwidth]{Knuth.jpg}
    \end{column}
  \end{columns}
\end{frame}

\begin{frame}
  \frametitle{\LaTeX{}}
  \begin{columns}[c]
    \begin{column}{0.7\textwidth}
      \begin{center}
        \rmfamily\Huge
        \highlight[structure]{\LaTeX{}}
      \end{center}
      \begin{center}
        \parbox{0.75\textwidth}{
          \LaTeX{} 是最早在 1985 年由现就职于微软的 Leslie Lamport 开发的一种 \TeX{} \textbf{格式}\footnotemark,使用一些列宏和扩展宏包来简化 \TeX{} 的使用。现在由 \LaTeX{} Project 的成员维护。现在广泛使用的版本是 \LaTeXe{},最新的版本为 \LaTeX3(2020 年 10 月后默认内置)。
        }
      \end{center}
    \end{column}
    \begin{column}{0.3\textwidth}
      \includegraphics[width=.8\columnwidth]{Lamport.jpg}
    \end{column}
  \end{columns}
  \footnotetext{\hologo{ConTeXt} 也是一种 \TeX{} 格式 \link{https://www.contextgarden.net/}。}
\end{frame}

\begin{frame}
  \frametitle{程序}
  \begin{columns}[c]
    \begin{column}{0.7\textwidth}
      \begin{center}
        \rmfamily\Huge
        \highlight[structure]{\hologo{pdfLaTeX}}
      \end{center}
      \begin{center}
        \parbox{0.7\textwidth}{
          \hologo{pdfLaTeX} 是为了编译一个 \LaTeX{} 文档而运行的程序。实际上底层在运行一个叫 \hologo{pdfTeX} 的引擎,并预装了对应的 \LaTeX{} \textbf{格式}。为了利用临时文件,可能就需要多次运行程序。
        }
      \end{center}
    \end{column}
    \begin{column}{0.3\textwidth}
      \begin{block}{}
        \ttfamily\small
        > \highlight{pdflatex} main.tex\\
        This is pdfTeX, Version 3.141592653-
        2.6-1.40.23 (MiKTeX 21.10)\\
        entering extended mode\\
        \highlight{LaTeX2e} <2021-11-15>\\
        \highlight{L3} programming layer <2021-11-22>
      \end{block}
    \end{column}
  \end{columns}
\end{frame}

\begin{frame}
  \frametitle{引擎}
  \begin{columns}[c]
    \begin{column}{0.7\textwidth}
      \begin{center}
        \rmfamily\Huge
        \highlight[structure!70]{pdf}\hologo{La}\highlight[structure!70]{\TeX{}}
      \end{center}
      \begin{center}
        \parbox{0.7\textwidth}{
          pdf\TeX{} 是编译 \TeX{} 文档(以 \texttt{.tex} 结尾)的\textbf{引擎}---可以理解 \TeX{} 指令的\textbf{程序}。
        }
      \end{center}
    \end{column}
    \begin{column}{0.3\textwidth}
      \begin{block}{}
        \ttfamily\small
        > pdflatex main.tex\\
        This is \highlight[structure!70]{pdfTeX}, Version 3.141592653-
        2.6-1.40.23 (MiKTeX 21.10)
        entering extended mode\\
        LaTeX2e <2021-11-15>\\
        L3 programming layer <2021-11-22>
      \end{block}
    \end{column}
  \end{columns}
\end{frame}

\begin{frame}
  \frametitle{Unicode 引擎}
  \begin{table}
    \caption{主流 \hologo{(La)TeX} 程序
    \footnote{(u)p\TeX{} 是日语最常用的引擎,生成 \texttt{.dvi},支持 Unicode。}\footnote{Ap\TeX{} 具有底层 CJK 支持,内联 Ruby,Color Emoji。}}
    \footnotesize
    \begin{stampbox}
      \begin{tabular}{c>{\raggedright}*{3}{p{3.5cm}}}
        \alert{引擎}     & \hologo{pdfTeX}   & \hologo{XeTeX}   & \hologo{LuaTeX}   \\
        \alert{程序}     & \hologo{pdfLaTeX} & \hologo{XeLaTeX} & \hologo{LuaLaTeX} \\
        \alert{特点}     & 直接生成 PDF,支持 micro-typography  & 支持 Unicode、OpenType 与复杂文字编排 (CTL) & 支持 Unicode,内联 Lua,支持 OpenType \\
      \end{tabular}
    \end{stampbox}
  \end{table}

  \begin{center}
    \parbox{.9\textwidth}{
      \hologo{pdfLaTeX} 不支持 Unicode。为了排版中文,大部分情况下 \faApple{}\,\faLinux{} 应当使用 \hologo{XeLaTeX},而 \hologo{LuaLaTeX} 速度相对较慢。\faWindows{} 可以在一些情况下使用 \hologo{pdfLaTeX}。
    }
  \end{center}
\end{frame}

% \begin{frame}
%   \paragraph{\hologo{pdfLaTeX}} \TeX{} 和 \LaTeX{} 被广泛使用之前,它们只需内置支持欧洲语言即可。在 Unicode 出现之前,\LaTeX{} 提供了许多种\textbf{文件编码}来允许很多语言的文字以原生的方式输入,\hologo{pdfLaTeX} 也只需要使用 8 位文件编码和 8 位字体。
% \end{frame}

\section{跑起来}
\begin{frame}
  \frametitle{发行版}
  \begin{table}
    \caption{\hologo{TeX} 发行版}
    \footnotesize
    \begin{stampbox}
      \begin{tabular}{c>{\raggedright}*{3}{p{3.2cm}}}
        \alert{发行版}     & \hologo{MiKTeX} \link{https://mirrors.sjtug.sjtu.edu.cn/ctan/systems/win32/miktex/setup/windows-x64/basic-miktex-21.12-x64.exe}   & \TeX{} Live \link{https://mirrors.sjtug.sjtu.edu.cn/ctan/systems/texlive/tlnet/install-tl.zip}   & Mac\TeX{} \link{https://mirrors.sjtug.sjtu.edu.cn/ctan/systems/mac/mactex/mactex-20210328.pkg}  \\[2pt]
        \alert{特点}      &  只安装必要文件,依赖用时更新  &  所有平台均可使用,每年发布一次 & Mac 系统专用,对 \TeX{} Live 的进一步打包 \\
        \alert{推荐平台}  & \faWindows  & \faLinux &  \faApple \\
      \end{tabular}
    \end{stampbox}
  \end{table}
  \begin{center}
    \parbox{.9\textwidth}{
      要让 \LaTeX{} 跑起来,核心就是要有一套 \TeX{} 发行版,来获取让 \LaTeX{} 工作所需的一组程序和文件。参考《一份简短的关于 \LaTeX{} 安装的介绍》\link{https://mirrors.sjtug.sjtu.edu.cn/ctan/info/install-latex-guide-zh-cn/install-latex-guide-zh-cn.pdf} 安装想使用的发行版。推荐使用发行版的最新版本\footnote{老版本 Linux 系统的包管理器自带 \TeX{} Live 发行版可能不是最新的 \link{https://repology.org/project/texlive/versions},尽量使用镜像安装,并手动将相关环境变量添加到路径 \link{https://www.tug.org/texlive/doc/texlive-zh-cn/texlive-zh-cn.pdf}。},并使用国内镜像。
    }
  \end{center}
\end{frame}

\begin{frame}[plain]
  \hbox to \textwidth{
    \hfil
    \vbox to 3cm{
      \hbox{
        \resizebox{3cm}{!}{\includegraphics{\getcontribpath{sjtug}{vi/sjtug.pdf}}}
      }
    }
    \hfil
    \vbox to 3cm{
      \vfill
      \hbox{\Large\bfseries\color{cprimary} 稳定、快速、现代的镜像服务。}
      \vskip2pt
      \hbox{托管于华东教育网骨干节点上海交通大学。}
      \vfill
    }
    \hskip20pt
    \hfil
  }

  \begin{center}
    \parbox{0.8\textwidth}{
      推荐使用 SJTUG 软件镜像服务,离得近,下得快。
      
      \begin{description}
        \footnotesize
        \item[\TeX{} Live]  {\ttfamily tlmgr option repository https://mirrors.sjtug.sjtu.edu.cn/CTAN/systems/texlive/tlnet}
        \item[\hologo{MiKTeX}] 在 \hologo{MiKTeX} Console 中设置镜像源为 \url{https://mirrors.sjtug.sjtu.edu.cn}
      \end{description}
    }
  \end{center}
\end{frame}

\begin{frame}
  \frametitle{编辑器}
  \begin{table}
    \caption{开源编辑器推荐}
    \footnotesize
    \begin{stampbox}
      \begin{tabular}{c>{\raggedright}*{3}{p{3.5cm}}}
        \alert{编辑器}     & \begin{tabular}{c}Visual Studio Code\\ \LaTeX{} Workshop\end{tabular}  & \TeX{}studio & \TeX{}works \\[5pt]
        \alert{特点}      &  搭配 VS Code 使用非常方便,易扩展  & 可以使用大量的菜单选项输入代码块,用户友好 & 只提供基础的高亮与运行方法,发行版自带\footnote{Mac\TeX{} 打包的是 \TeX{}Shop 编辑器。} \\
      \end{tabular}
    \end{stampbox}
  \end{table}
  \begin{center}
    \parbox{.9\textwidth}{
      使用专为 \LaTeX{} 设计的编辑器将带来更多便利,因为它们往往会提供一键编译、内置 PDF 阅读器以及语法高亮等功能。几乎所有现代的 \LaTeX{} 编辑器都提供 Sync\TeX{} 这一强大的功能,以在 PDF 和 代码间对应跳转。
    }
  \end{center}
\end{frame}

\begin{frame}
  \frametitle{在线平台}
  \begin{table}
    \caption{在线协作平台推荐}
    \footnotesize
    \begin{stampbox}
      \begin{tabular}{c>{\raggedright}*{2}{p{4cm}}}
        \alert{在线平台}     & Overleaf \link{https://www.overleaf.com/}  & 交大 \LaTeX{} 助手 \link{https://latex.sjtu.edu.cn/} \\[2pt]
        \alert{特点}      & 最流行的在线平台,提供大量的模板,但国内访问慢 & 校内平台,隐私保护有保障,共享项目限制少 \\
      \end{tabular}
    \end{stampbox}
  \end{table}
  \begin{center}
    \parbox{.9\textwidth}{
      在线平台允许你直接在网页中编辑文档,无需本地安装即可在后台运行 \LaTeX{},并显示生成的 PDF。可以参照 Overleaf 官方文档学习如何使用在线平台 \link{https://www.overleaf.com/learn}。
    }
  \end{center}
\end{frame}

\section{基本结构}
\begin{frame}[fragile]%
  \frametitle{文档部件}
  \begin{columns}[c]
    \begin{column}{0.4\textwidth}
      \only<1>{
        \cmd{documentclass} 命令加载了\textbf{文档类}。\pkg{article} 是由 \LaTeX{}提供的用于排版短文档的基本文档类。
        \begin{description}
          \footnotesize
          \item[\pkg{article}] 不包含章的短文档
          \item[\pkg{report}] 含有章的单面印刷文档
          \item[\pkg{book}] 含有章的双面印刷文档
          \item[\pkg{beamer}] 制作幻灯片
        \end{description}
      }
      \only<2>{
        \env{document} 环境用于指示文档主体的范围。\LaTeX{} 还有其他的使用 \cmd{begin} 和 \cmd{end} 的搭配,我们称这些为\textbf{环境}。它们将用来设定局部格式命令\footnotemark。
      }
      \only<3>{
        \includepdflarge{enminimal}
      }
    \end{column}
    \begin{column}{0.6\textwidth}
      \begin{codeblock}[]{排版英文最简示例}
|\only<1>{\highlightline}|\documentclass{article}
|\only<2>{\highlightline}|\begin{document}
|\only<3>{\highlightline}|  Together for a Shared Future
|\only<2>{\highlightline}|\end{document}
      \end{codeblock}
    \end{column}
  \end{columns}
  \only<2>{\footnotetext{环境实际上是一个组,只不过通过语义化的形式预装了对应的格式命令。普通的组可以直接使用一对大括号之间的内容 \{$\cdots$\} 表示。}}
\end{frame}

\section{扩展}
\begin{frame}[fragile]%
  \frametitle{中文排版}
  \begin{columns}[c]
    \begin{column}{0.4\textwidth}
      \only<1>{
        \cmd{usepackage} 用于使用宏包以向 \LaTeX{} 添加或修改功能,需要在\textbf{导言区}调用。
        这里使用 \pkg{ctex} 宏集以获得中文支持。其调用底层因随不同的引擎而不同。
        {
          \footnotesize
          \begin{stampbox}
            \begin{tabular}{c*{3}{c}}
              \alert{引擎}     & \hologo{pdfTeX}   & \hologo{XeTeX}   & \hologo{LuaTeX}   \\
              \alert{程序}     & \hologo{pdfLaTeX} & \hologo{XeLaTeX} & \hologo{LuaLaTeX} \\
              \alert{宏包}     & CJK\footnotemark & xeCJK & luatexja \\
              \alert{封装}     & \multicolumn{3}{c}{ctex} \\
            \end{tabular}
          \end{stampbox}
        }
        \vspace{-1cm}
      }
      \only<2>{
        C\TeX{} 建议对于之前提到的常规文档类,最佳实践是使用该宏集提供的四种中文文档类,以对特定类型提供额外的中文排版适配。
        \begin{center}
          \begin{stampbox}
            \footnotesize
            \begin{tabular}{cc}
              \pkg{ctexart} & \pkg{ctexrep} \\
              \pkg{ctexbook} & \pkg{ctexbeamer} \\
            \end{tabular}
          \end{stampbox}
        \end{center}
      }
      \only<3>{
        \includepdflarge{cnminimal}
      }
      \only<4>{
        大部分情况下,你都不应当在 \LaTeX{} 中强制断行:你几乎只是想另起一段,或者是想在段落之间添加空行(使用 \pkg{parskip} 宏包就可实现)。
        只有\alert{很少的}情况下你需要使用 \textbackslash{}\textbackslash{} 来另起一行而不另起一段。
      }
    \end{column}
    \begin{column}{0.6\textwidth}
      \begin{codeblock}[]{排版中文\only<2->{最佳实践}}
|\only<2>{\highlightline}|\documentclass{|\only<1>{article}\only<2->{ctexart}|}
|\only<1>{\highlightline\textbackslash{}usepackage\{ctex\}\hfill\color{cprimary}\% 导言区}|
\begin{document}
|\only<3>{\highlightline}|    一起向未来
|\only<4>{\highlightline}|
  Together for a Shared Future
\end{document}
      \end{codeblock}
    \end{column}
  \end{columns}
  \only<1>{\footnotetext{ctex 在 \faApple\,\faLinux{} 上已经不可以使用 \hologo{pdfLaTeX} 编译,以及在 \faWindows{} 上使用该引擎也会变更自动间距调整等功能的默认行为。}}
\end{frame}

\section{设定格式}
\begin{frame}[fragile]%
  \frametitle{字体样式}
  \begin{columns}
    \begin{column}{0.4\textwidth}
      \only<1>{
        \includepdflarge{fontstyle}
      }
      \only<2>{
        可以使用显示样式设定命令对小段做加粗、斜体、等宽等等的处理。
        \begin{center}
          \footnotesize
          \begin{stampbox}
            \begin{tabular}{rl}
              \cmd{textrm} & \textrm{衬线} \\
              \cmd{textbf} & \textbf{加粗} \\
              \cmd{textit} & \kaishu 斜体 \\
              \cmd{texttt} & \texttt{等宽} \\
              \cmd{textsf} & \textsf{无衬线} \\
              \cmd{textsc} & \textsc{Small Caps} \\
              \cmd{textsl} & \textsl{Slanted} \\
            \end{tabular}
          \end{stampbox}
        \end{center}
      }
      \only<3>{
        与 Word 不同的是,\LaTeX{} 一般情况下并不需要使用上面的显式命令,而是采用逻辑标记的方法,
        比如 \cmd{emph} 可以强调文字,以及下面将要提到的目次命令(第 \ref{sectioning} 页)。
        这样可以统一管理格式。
      }
    \end{column}
    \begin{column}{0.6\textwidth}
      \begin{codeblock}[]{样式}
\documentclass{ctexart}
\begin{document}
|\only<2>{\highlightline}|  \textbf{||一起向未来}

|\only<3>{\highlightline}|  \emph{Together for a Shared Future}
\end{document}
      \end{codeblock}
    \end{column}
  \end{columns}
\end{frame}

\begin{frame}[fragile]%
  \frametitle{\only<1-2>{字体大小}\only<3>{字体样式}}
  \begin{columns}
    \begin{column}{0.4\textwidth}
      \only<1>{
        \includepdflarge{fontsize}
      }
      \only<2>{
        同样地,你也可以显式地设定字体大小,但是这种命令会更改行文设置,所以需要使用一个组来限定作用范围\footnotemark。
        \begin{center}
          \footnotesize
          \begin{stampbox}
            \begin{tabular}{rl}
              \cmd{tiny} & \tiny 极小 \\
              \cmd{scriptsize} & \scriptsize 抄本大小  \\
              \cmd{footnotesize} & \footnotesize 脚注大小 \\
              \cmd{small} & \small 小 \\
              \cmd{normalsize} & \normalsize 正常大小 \\
              \cmd{large} & \large 大 \\
              \cmd{huge} & \Huge 巨大 \\
            \end{tabular}
          \end{stampbox}
        \end{center}
      }
      \only<3>{
        也可以使用字体样式对应的更改字体设置的命令,这对于大段文段的设置而言也是很方便的。
        \begin{center}
          \footnotesize
          \begin{stampbox}
            \begin{tabular}{ll}
              \cmd{textrm} & \cmd{rmfamily}\\
              \cmd{texttt} & \cmd{ttfamily}\\
              \cmd{textsf} & \cmd{sffamily}\\
              \cmd{textbf} & \cmd{bfseries}\\
              \cmd{textit} & \cmd{itshape}\\
              \cmd{textsc} & \cmd{scshape}\\
              \cmd{textsl} & \cmd{slshape}\\
            \end{tabular}
          \end{stampbox}
        \end{center}
      }
    \end{column}
    \begin{column}{0.6\textwidth}
      \begin{codeblock}[]{大小}
\documentclass{ctexart}
\begin{document}
|\only<2>{\highlightline}|  {\bfseries\Large 一起向未来\par}
|\only<3>{\highlightline}|  {\itshape Together for a Shared Future}
\end{document}
      \end{codeblock}
    \end{column}
  \end{columns}
  \only<2>{\footnotetext{注意最后显式地使用 \cmd{par} 在改回大小前结束该段,否则会导致下一行的行间距异常!}}
\end{frame}

\section{逻辑结构}
\begin{frame}[fragile]
  \frametitle{列表}
  \begin{columns}
    \begin{column}{0.35\textwidth}
      \begin{codeblock}[]{无序列表}
\documentclass{ctexart}
\begin{document}
|\highlightline|  \begin{itemize}
    \item 第一项
    \item 第二项
    \item 第三项
|\highlightline|  \end{itemize}
\end{document}
      \end{codeblock}
    \end{column}
    \begin{column}{0.35\textwidth}
      \begin{codeblock}[]{有序列表}
\documentclass{ctexart}
\begin{document}
|\highlightline|  \begin{enumerate}
    \item 第一项
    \item 第二项
    \item 第三项
|\highlightline|  \end{enumerate}
\end{document}
      \end{codeblock}
    \end{column}
    \begin{column}{0.35\textwidth}
      \begin{codeblock}[]{描述列表}
\documentclass{ctexart}
\begin{document}
|\highlightline|  \begin{description}
    \item[||第一] 文本
    \item[||第二] 文本
    \item[||第三] 文本  
|\highlightline|  \end{description}
\end{document}
      \end{codeblock}
    \end{column}
  \end{columns}
\end{frame}

%更深的列表技巧,定理环境等

\begin{frame}
  \frametitle{列表}
  \begin{columns}
    \begin{column}{0.35\textwidth}
      \includepdflarge{unordered}
    \end{column}
    \begin{column}{0.35\textwidth}
      \includepdflarge{ordered}
    \end{column}
    \begin{column}{0.35\textwidth}
      \includepdflarge{description}
    \end{column}
  \end{columns}
\end{frame}

\begin{frame}[fragile,label=sectioning]%
  \frametitle{目次结构}
  \begin{columns}
    \begin{column}{0.4\textwidth}
      \LaTeX{} 可以使用目次命令将文档划分层级\footnotemark,并自动设定对应字体样式和大小。
      \begin{center}
        \begin{stampbox}
          \footnotesize
          \begin{tabular}{rll}
           命令 & 中文 & 层次 \\
           \cmd{chapter} & 章\footnotemark & \sout{0} \\
           \cmd{section} & 节 & 1 \\
           \cmd{subsection} & 小节 & 2 \\
           \cmd{subsubsection} & 小小节 & 3 \\
          \end{tabular}
        \end{stampbox}
      \end{center}
    \end{column}
    \begin{column}{0.6\textwidth}
      \begin{codeblock}[]{目次}
\documentclass{ctexart}
\begin{document}
|\highlightline|  \section{||概念}
|\highlightline|  \subsection{\LaTeX{}}
  \LaTeX{} 是一个用以排版高质量作品的文档准备系统。
\end{document}
      \end{codeblock}
    \end{column}
  \end{columns}
  \footnotetext{章这一级只在 \pkg{report} 和 \pkg{book} 文档类(包括对应的中文文档类)有定义。还有不常用的 \cmd{part} (0@\pkg{article}/-1@\pkg{report}\&\pkg{book}\&\pkg{beamer}) 以及更低层次的 \cmd{paragraph} (4) 与 \cmd{subparagraph} (5)。 }
\end{frame}

\begin{frame}[fragile]%
  \frametitle{组织文档}
  \begin{columns}
    \begin{column}{0.4\textwidth}
      \only<1>{
        \cmd{tableofcontents} 用来生成对于目次命令的目录。如果你想设定显示到哪个层级,在这个命令前使用 \cmd{setcounter\{tocdepth\}\{层次\}}
      }
      \only<2>{
        对于大型文档而言,使用多个文件管理源文件通常是更方便的。而 \cmd{include} 和 \cmd{input} 都以相对路径的方式插入其他 \TeX{} 文档。
        区别在于,\cmd{include} 命令会从新页开始并做一些内部调整,这基本上只对 \pkg{chapter} 这一级有用。而 \cmd{input} 会原样插入源代码。
      }
      \only<3>{
        但是 \cmd{include} 插入的文档可以使用 \cmd{includeonly} 管理当前要排印哪一部分的内容,利用所有章节辅助文件的同时,减少编译时间并专注于该部分的内容。
      }
    \end{column}
    \begin{column}{0.6\textwidth}
      \begin{codeblock}[]{主文档}
\documentclass{ctexrep}
|\only<3>{\highlightline}|\includeonly{learnlatex,sjtuthesis}
\begin{document}
|\only<1>{\highlightline}|  \tableofcontents
|\only<2-3>{\highlightline}|  % !TeX root = ..\..\latex-talk.tex

\part{学习 \LaTeX{}}
% FIXME: Part Page miniframe overflow
% FIXME: footnote fault numbering

\begin{frame}[plain]
  \vfil
  \begin{center}
    \href{https://learnlatex.org}{
      \rmfamily
      Learn\,\lower1ex\hbox{\Huge\LaTeX{}}.org
    }
  \end{center}
  \vfil
  \begin{center}
    \parbox{0.75\linewidth}{
      Learn\LaTeX{}.org\cite{learnlatex} 提供了解 \LaTeX{} 的 16 篇简短的教程,并包含了一些可以在线运行的示例,可以通过亲自动手查看实验效果。本部分主要参考由 C\TeX{}-org 提供的中文翻译版本 \link{https://github.com/CTeX-org/learnlatex.github.io/tree/zh-Hans/zh-Hans/}。
    }
  \end{center}
  \vfil
\end{frame}

{ % Start of shaded number logo

\newcommand{\shadedfont}[2][1pt]{
  % #1 (optional): shadow distance
  % #2: the text needed to be shaded
  \hbox{\rlap{\color{gray}\hskip#1#2}#2}
}
\newcounter{learnsec}
\setcounter{learnsec}{-1}
\newcommand{\updatelogo}{
  % update the logo corresponding to current counter.
  \stepcounter{learnsec}
  \logo{
    \raise.3ex\hbox{\tiny\insertsection}\shadedfont{\arabic{learnsec}}
  }
}
\let\oldsection=\section
\renewcommand{\section}[1]{\oldsection{#1}\updatelogo}

\section{是什么}
\begin{frame}
  \frametitle{\TeX{}}
  \begin{columns}[c]
    \begin{column}{0.7\textwidth}
      \begin{center}
        \rmfamily\Huge
        \hologo{La}\highlight[structure!70]{\TeX{}}
      \end{center}
      \begin{center}
        \parbox{0.75\textwidth}{
          \TeX{} 是由斯坦福大学教授高德纳
          (Donald E.~Knuth)于 1977 年开始开发的排版引擎。目前仍在更新,最新版本号为 3.141592653 \link{https://tug.org/TUGboat/tb42-1/tb130knuth-tuneup21.pdf}。
        }
      \end{center}
    \end{column}
    \begin{column}{0.3\textwidth}
      \includegraphics[width=.8\columnwidth]{Knuth.jpg}
    \end{column}
  \end{columns}
\end{frame}

\begin{frame}
  \frametitle{\LaTeX{}}
  \begin{columns}[c]
    \begin{column}{0.7\textwidth}
      \begin{center}
        \rmfamily\Huge
        \highlight[structure]{\LaTeX{}}
      \end{center}
      \begin{center}
        \parbox{0.75\textwidth}{
          \LaTeX{} 是最早在 1985 年由现就职于微软的 Leslie Lamport 开发的一种 \TeX{} \textbf{格式}\footnotemark,使用一些列宏和扩展宏包来简化 \TeX{} 的使用。现在由 \LaTeX{} Project 的成员维护。现在广泛使用的版本是 \LaTeXe{},最新的版本为 \LaTeX3(2020 年 10 月后默认内置)。
        }
      \end{center}
    \end{column}
    \begin{column}{0.3\textwidth}
      \includegraphics[width=.8\columnwidth]{Lamport.jpg}
    \end{column}
  \end{columns}
  \footnotetext{\hologo{ConTeXt} 也是一种 \TeX{} 格式 \link{https://www.contextgarden.net/}。}
\end{frame}

\begin{frame}
  \frametitle{程序}
  \begin{columns}[c]
    \begin{column}{0.7\textwidth}
      \begin{center}
        \rmfamily\Huge
        \highlight[structure]{\hologo{pdfLaTeX}}
      \end{center}
      \begin{center}
        \parbox{0.7\textwidth}{
          \hologo{pdfLaTeX} 是为了编译一个 \LaTeX{} 文档而运行的程序。实际上底层在运行一个叫 \hologo{pdfTeX} 的引擎,并预装了对应的 \LaTeX{} \textbf{格式}。为了利用临时文件,可能就需要多次运行程序。
        }
      \end{center}
    \end{column}
    \begin{column}{0.3\textwidth}
      \begin{block}{}
        \ttfamily\small
        > \highlight{pdflatex} main.tex\\
        This is pdfTeX, Version 3.141592653-
        2.6-1.40.23 (MiKTeX 21.10)\\
        entering extended mode\\
        \highlight{LaTeX2e} <2021-11-15>\\
        \highlight{L3} programming layer <2021-11-22>
      \end{block}
    \end{column}
  \end{columns}
\end{frame}

\begin{frame}
  \frametitle{引擎}
  \begin{columns}[c]
    \begin{column}{0.7\textwidth}
      \begin{center}
        \rmfamily\Huge
        \highlight[structure!70]{pdf}\hologo{La}\highlight[structure!70]{\TeX{}}
      \end{center}
      \begin{center}
        \parbox{0.7\textwidth}{
          pdf\TeX{} 是编译 \TeX{} 文档(以 \texttt{.tex} 结尾)的\textbf{引擎}---可以理解 \TeX{} 指令的\textbf{程序}。
        }
      \end{center}
    \end{column}
    \begin{column}{0.3\textwidth}
      \begin{block}{}
        \ttfamily\small
        > pdflatex main.tex\\
        This is \highlight[structure!70]{pdfTeX}, Version 3.141592653-
        2.6-1.40.23 (MiKTeX 21.10)
        entering extended mode\\
        LaTeX2e <2021-11-15>\\
        L3 programming layer <2021-11-22>
      \end{block}
    \end{column}
  \end{columns}
\end{frame}

\begin{frame}
  \frametitle{Unicode 引擎}
  \begin{table}
    \caption{主流 \hologo{(La)TeX} 程序
    \footnote{(u)p\TeX{} 是日语最常用的引擎,生成 \texttt{.dvi},支持 Unicode。}\footnote{Ap\TeX{} 具有底层 CJK 支持,内联 Ruby,Color Emoji。}}
    \footnotesize
    \begin{stampbox}
      \begin{tabular}{c>{\raggedright}*{3}{p{3.5cm}}}
        \alert{引擎}     & \hologo{pdfTeX}   & \hologo{XeTeX}   & \hologo{LuaTeX}   \\
        \alert{程序}     & \hologo{pdfLaTeX} & \hologo{XeLaTeX} & \hologo{LuaLaTeX} \\
        \alert{特点}     & 直接生成 PDF,支持 micro-typography  & 支持 Unicode、OpenType 与复杂文字编排 (CTL) & 支持 Unicode,内联 Lua,支持 OpenType \\
      \end{tabular}
    \end{stampbox}
  \end{table}

  \begin{center}
    \parbox{.9\textwidth}{
      \hologo{pdfLaTeX} 不支持 Unicode。为了排版中文,大部分情况下 \faApple{}\,\faLinux{} 应当使用 \hologo{XeLaTeX},而 \hologo{LuaLaTeX} 速度相对较慢。\faWindows{} 可以在一些情况下使用 \hologo{pdfLaTeX}。
    }
  \end{center}
\end{frame}

% \begin{frame}
%   \paragraph{\hologo{pdfLaTeX}} \TeX{} 和 \LaTeX{} 被广泛使用之前,它们只需内置支持欧洲语言即可。在 Unicode 出现之前,\LaTeX{} 提供了许多种\textbf{文件编码}来允许很多语言的文字以原生的方式输入,\hologo{pdfLaTeX} 也只需要使用 8 位文件编码和 8 位字体。
% \end{frame}

\section{跑起来}
\begin{frame}
  \frametitle{发行版}
  \begin{table}
    \caption{\hologo{TeX} 发行版}
    \footnotesize
    \begin{stampbox}
      \begin{tabular}{c>{\raggedright}*{3}{p{3.2cm}}}
        \alert{发行版}     & \hologo{MiKTeX} \link{https://mirrors.sjtug.sjtu.edu.cn/ctan/systems/win32/miktex/setup/windows-x64/basic-miktex-21.12-x64.exe}   & \TeX{} Live \link{https://mirrors.sjtug.sjtu.edu.cn/ctan/systems/texlive/tlnet/install-tl.zip}   & Mac\TeX{} \link{https://mirrors.sjtug.sjtu.edu.cn/ctan/systems/mac/mactex/mactex-20210328.pkg}  \\[2pt]
        \alert{特点}      &  只安装必要文件,依赖用时更新  &  所有平台均可使用,每年发布一次 & Mac 系统专用,对 \TeX{} Live 的进一步打包 \\
        \alert{推荐平台}  & \faWindows  & \faLinux &  \faApple \\
      \end{tabular}
    \end{stampbox}
  \end{table}
  \begin{center}
    \parbox{.9\textwidth}{
      要让 \LaTeX{} 跑起来,核心就是要有一套 \TeX{} 发行版,来获取让 \LaTeX{} 工作所需的一组程序和文件。参考《一份简短的关于 \LaTeX{} 安装的介绍》\link{https://mirrors.sjtug.sjtu.edu.cn/ctan/info/install-latex-guide-zh-cn/install-latex-guide-zh-cn.pdf} 安装想使用的发行版。推荐使用发行版的最新版本\footnote{老版本 Linux 系统的包管理器自带 \TeX{} Live 发行版可能不是最新的 \link{https://repology.org/project/texlive/versions},尽量使用镜像安装,并手动将相关环境变量添加到路径 \link{https://www.tug.org/texlive/doc/texlive-zh-cn/texlive-zh-cn.pdf}。},并使用国内镜像。
    }
  \end{center}
\end{frame}

\begin{frame}[plain]
  \hbox to \textwidth{
    \hfil
    \vbox to 3cm{
      \hbox{
        \resizebox{3cm}{!}{\includegraphics{\getcontribpath{sjtug}{vi/sjtug.pdf}}}
      }
    }
    \hfil
    \vbox to 3cm{
      \vfill
      \hbox{\Large\bfseries\color{cprimary} 稳定、快速、现代的镜像服务。}
      \vskip2pt
      \hbox{托管于华东教育网骨干节点上海交通大学。}
      \vfill
    }
    \hskip20pt
    \hfil
  }

  \begin{center}
    \parbox{0.8\textwidth}{
      推荐使用 SJTUG 软件镜像服务,离得近,下得快。
      
      \begin{description}
        \footnotesize
        \item[\TeX{} Live]  {\ttfamily tlmgr option repository https://mirrors.sjtug.sjtu.edu.cn/CTAN/systems/texlive/tlnet}
        \item[\hologo{MiKTeX}] 在 \hologo{MiKTeX} Console 中设置镜像源为 \url{https://mirrors.sjtug.sjtu.edu.cn}
      \end{description}
    }
  \end{center}
\end{frame}

\begin{frame}
  \frametitle{编辑器}
  \begin{table}
    \caption{开源编辑器推荐}
    \footnotesize
    \begin{stampbox}
      \begin{tabular}{c>{\raggedright}*{3}{p{3.5cm}}}
        \alert{编辑器}     & \begin{tabular}{c}Visual Studio Code\\ \LaTeX{} Workshop\end{tabular}  & \TeX{}studio & \TeX{}works \\[5pt]
        \alert{特点}      &  搭配 VS Code 使用非常方便,易扩展  & 可以使用大量的菜单选项输入代码块,用户友好 & 只提供基础的高亮与运行方法,发行版自带\footnote{Mac\TeX{} 打包的是 \TeX{}Shop 编辑器。} \\
      \end{tabular}
    \end{stampbox}
  \end{table}
  \begin{center}
    \parbox{.9\textwidth}{
      使用专为 \LaTeX{} 设计的编辑器将带来更多便利,因为它们往往会提供一键编译、内置 PDF 阅读器以及语法高亮等功能。几乎所有现代的 \LaTeX{} 编辑器都提供 Sync\TeX{} 这一强大的功能,以在 PDF 和 代码间对应跳转。
    }
  \end{center}
\end{frame}

\begin{frame}
  \frametitle{在线平台}
  \begin{table}
    \caption{在线协作平台推荐}
    \footnotesize
    \begin{stampbox}
      \begin{tabular}{c>{\raggedright}*{2}{p{4cm}}}
        \alert{在线平台}     & Overleaf \link{https://www.overleaf.com/}  & 交大 \LaTeX{} 助手 \link{https://latex.sjtu.edu.cn/} \\[2pt]
        \alert{特点}      & 最流行的在线平台,提供大量的模板,但国内访问慢 & 校内平台,隐私保护有保障,共享项目限制少 \\
      \end{tabular}
    \end{stampbox}
  \end{table}
  \begin{center}
    \parbox{.9\textwidth}{
      在线平台允许你直接在网页中编辑文档,无需本地安装即可在后台运行 \LaTeX{},并显示生成的 PDF。可以参照 Overleaf 官方文档学习如何使用在线平台 \link{https://www.overleaf.com/learn}。
    }
  \end{center}
\end{frame}

\section{基本结构}
\begin{frame}[fragile]%
  \frametitle{文档部件}
  \begin{columns}[c]
    \begin{column}{0.4\textwidth}
      \only<1>{
        \cmd{documentclass} 命令加载了\textbf{文档类}。\pkg{article} 是由 \LaTeX{}提供的用于排版短文档的基本文档类。
        \begin{description}
          \footnotesize
          \item[\pkg{article}] 不包含章的短文档
          \item[\pkg{report}] 含有章的单面印刷文档
          \item[\pkg{book}] 含有章的双面印刷文档
          \item[\pkg{beamer}] 制作幻灯片
        \end{description}
      }
      \only<2>{
        \env{document} 环境用于指示文档主体的范围。\LaTeX{} 还有其他的使用 \cmd{begin} 和 \cmd{end} 的搭配,我们称这些为\textbf{环境}。它们将用来设定局部格式命令\footnotemark。
      }
      \only<3>{
        \includepdflarge{enminimal}
      }
    \end{column}
    \begin{column}{0.6\textwidth}
      \begin{codeblock}[]{排版英文最简示例}
|\only<1>{\highlightline}|\documentclass{article}
|\only<2>{\highlightline}|\begin{document}
|\only<3>{\highlightline}|  Together for a Shared Future
|\only<2>{\highlightline}|\end{document}
      \end{codeblock}
    \end{column}
  \end{columns}
  \only<2>{\footnotetext{环境实际上是一个组,只不过通过语义化的形式预装了对应的格式命令。普通的组可以直接使用一对大括号之间的内容 \{$\cdots$\} 表示。}}
\end{frame}

\section{扩展}
\begin{frame}[fragile]%
  \frametitle{中文排版}
  \begin{columns}[c]
    \begin{column}{0.4\textwidth}
      \only<1>{
        \cmd{usepackage} 用于使用宏包以向 \LaTeX{} 添加或修改功能,需要在\textbf{导言区}调用。
        这里使用 \pkg{ctex} 宏集以获得中文支持。其调用底层因随不同的引擎而不同。
        {
          \footnotesize
          \begin{stampbox}
            \begin{tabular}{c*{3}{c}}
              \alert{引擎}     & \hologo{pdfTeX}   & \hologo{XeTeX}   & \hologo{LuaTeX}   \\
              \alert{程序}     & \hologo{pdfLaTeX} & \hologo{XeLaTeX} & \hologo{LuaLaTeX} \\
              \alert{宏包}     & CJK\footnotemark & xeCJK & luatexja \\
              \alert{封装}     & \multicolumn{3}{c}{ctex} \\
            \end{tabular}
          \end{stampbox}
        }
        \vspace{-1cm}
      }
      \only<2>{
        C\TeX{} 建议对于之前提到的常规文档类,最佳实践是使用该宏集提供的四种中文文档类,以对特定类型提供额外的中文排版适配。
        \begin{center}
          \begin{stampbox}
            \footnotesize
            \begin{tabular}{cc}
              \pkg{ctexart} & \pkg{ctexrep} \\
              \pkg{ctexbook} & \pkg{ctexbeamer} \\
            \end{tabular}
          \end{stampbox}
        \end{center}
      }
      \only<3>{
        \includepdflarge{cnminimal}
      }
      \only<4>{
        大部分情况下,你都不应当在 \LaTeX{} 中强制断行:你几乎只是想另起一段,或者是想在段落之间添加空行(使用 \pkg{parskip} 宏包就可实现)。
        只有\alert{很少的}情况下你需要使用 \textbackslash{}\textbackslash{} 来另起一行而不另起一段。
      }
    \end{column}
    \begin{column}{0.6\textwidth}
      \begin{codeblock}[]{排版中文\only<2->{最佳实践}}
|\only<2>{\highlightline}|\documentclass{|\only<1>{article}\only<2->{ctexart}|}
|\only<1>{\highlightline\textbackslash{}usepackage\{ctex\}\hfill\color{cprimary}\% 导言区}|
\begin{document}
|\only<3>{\highlightline}|    一起向未来
|\only<4>{\highlightline}|
  Together for a Shared Future
\end{document}
      \end{codeblock}
    \end{column}
  \end{columns}
  \only<1>{\footnotetext{ctex 在 \faApple\,\faLinux{} 上已经不可以使用 \hologo{pdfLaTeX} 编译,以及在 \faWindows{} 上使用该引擎也会变更自动间距调整等功能的默认行为。}}
\end{frame}

\section{设定格式}
\begin{frame}[fragile]%
  \frametitle{字体样式}
  \begin{columns}
    \begin{column}{0.4\textwidth}
      \only<1>{
        \includepdflarge{fontstyle}
      }
      \only<2>{
        可以使用显示样式设定命令对小段做加粗、斜体、等宽等等的处理。
        \begin{center}
          \footnotesize
          \begin{stampbox}
            \begin{tabular}{rl}
              \cmd{textrm} & \textrm{衬线} \\
              \cmd{textbf} & \textbf{加粗} \\
              \cmd{textit} & \kaishu 斜体 \\
              \cmd{texttt} & \texttt{等宽} \\
              \cmd{textsf} & \textsf{无衬线} \\
              \cmd{textsc} & \textsc{Small Caps} \\
              \cmd{textsl} & \textsl{Slanted} \\
            \end{tabular}
          \end{stampbox}
        \end{center}
      }
      \only<3>{
        与 Word 不同的是,\LaTeX{} 一般情况下并不需要使用上面的显式命令,而是采用逻辑标记的方法,
        比如 \cmd{emph} 可以强调文字,以及下面将要提到的目次命令(第 \ref{sectioning} 页)。
        这样可以统一管理格式。
      }
    \end{column}
    \begin{column}{0.6\textwidth}
      \begin{codeblock}[]{样式}
\documentclass{ctexart}
\begin{document}
|\only<2>{\highlightline}|  \textbf{||一起向未来}

|\only<3>{\highlightline}|  \emph{Together for a Shared Future}
\end{document}
      \end{codeblock}
    \end{column}
  \end{columns}
\end{frame}

\begin{frame}[fragile]%
  \frametitle{\only<1-2>{字体大小}\only<3>{字体样式}}
  \begin{columns}
    \begin{column}{0.4\textwidth}
      \only<1>{
        \includepdflarge{fontsize}
      }
      \only<2>{
        同样地,你也可以显式地设定字体大小,但是这种命令会更改行文设置,所以需要使用一个组来限定作用范围\footnotemark。
        \begin{center}
          \footnotesize
          \begin{stampbox}
            \begin{tabular}{rl}
              \cmd{tiny} & \tiny 极小 \\
              \cmd{scriptsize} & \scriptsize 抄本大小  \\
              \cmd{footnotesize} & \footnotesize 脚注大小 \\
              \cmd{small} & \small 小 \\
              \cmd{normalsize} & \normalsize 正常大小 \\
              \cmd{large} & \large 大 \\
              \cmd{huge} & \Huge 巨大 \\
            \end{tabular}
          \end{stampbox}
        \end{center}
      }
      \only<3>{
        也可以使用字体样式对应的更改字体设置的命令,这对于大段文段的设置而言也是很方便的。
        \begin{center}
          \footnotesize
          \begin{stampbox}
            \begin{tabular}{ll}
              \cmd{textrm} & \cmd{rmfamily}\\
              \cmd{texttt} & \cmd{ttfamily}\\
              \cmd{textsf} & \cmd{sffamily}\\
              \cmd{textbf} & \cmd{bfseries}\\
              \cmd{textit} & \cmd{itshape}\\
              \cmd{textsc} & \cmd{scshape}\\
              \cmd{textsl} & \cmd{slshape}\\
            \end{tabular}
          \end{stampbox}
        \end{center}
      }
    \end{column}
    \begin{column}{0.6\textwidth}
      \begin{codeblock}[]{大小}
\documentclass{ctexart}
\begin{document}
|\only<2>{\highlightline}|  {\bfseries\Large 一起向未来\par}
|\only<3>{\highlightline}|  {\itshape Together for a Shared Future}
\end{document}
      \end{codeblock}
    \end{column}
  \end{columns}
  \only<2>{\footnotetext{注意最后显式地使用 \cmd{par} 在改回大小前结束该段,否则会导致下一行的行间距异常!}}
\end{frame}

\section{逻辑结构}
\begin{frame}[fragile]
  \frametitle{列表}
  \begin{columns}
    \begin{column}{0.35\textwidth}
      \begin{codeblock}[]{无序列表}
\documentclass{ctexart}
\begin{document}
|\highlightline|  \begin{itemize}
    \item 第一项
    \item 第二项
    \item 第三项
|\highlightline|  \end{itemize}
\end{document}
      \end{codeblock}
    \end{column}
    \begin{column}{0.35\textwidth}
      \begin{codeblock}[]{有序列表}
\documentclass{ctexart}
\begin{document}
|\highlightline|  \begin{enumerate}
    \item 第一项
    \item 第二项
    \item 第三项
|\highlightline|  \end{enumerate}
\end{document}
      \end{codeblock}
    \end{column}
    \begin{column}{0.35\textwidth}
      \begin{codeblock}[]{描述列表}
\documentclass{ctexart}
\begin{document}
|\highlightline|  \begin{description}
    \item[||第一] 文本
    \item[||第二] 文本
    \item[||第三] 文本  
|\highlightline|  \end{description}
\end{document}
      \end{codeblock}
    \end{column}
  \end{columns}
\end{frame}

%更深的列表技巧,定理环境等

\begin{frame}
  \frametitle{列表}
  \begin{columns}
    \begin{column}{0.35\textwidth}
      \includepdflarge{unordered}
    \end{column}
    \begin{column}{0.35\textwidth}
      \includepdflarge{ordered}
    \end{column}
    \begin{column}{0.35\textwidth}
      \includepdflarge{description}
    \end{column}
  \end{columns}
\end{frame}

\begin{frame}[fragile,label=sectioning]%
  \frametitle{目次结构}
  \begin{columns}
    \begin{column}{0.4\textwidth}
      \LaTeX{} 可以使用目次命令将文档划分层级\footnotemark,并自动设定对应字体样式和大小。
      \begin{center}
        \begin{stampbox}
          \footnotesize
          \begin{tabular}{rll}
           命令 & 中文 & 层次 \\
           \cmd{chapter} & 章\footnotemark & \sout{0} \\
           \cmd{section} & 节 & 1 \\
           \cmd{subsection} & 小节 & 2 \\
           \cmd{subsubsection} & 小小节 & 3 \\
          \end{tabular}
        \end{stampbox}
      \end{center}
    \end{column}
    \begin{column}{0.6\textwidth}
      \begin{codeblock}[]{目次}
\documentclass{ctexart}
\begin{document}
|\highlightline|  \section{||概念}
|\highlightline|  \subsection{\LaTeX{}}
  \LaTeX{} 是一个用以排版高质量作品的文档准备系统。
\end{document}
      \end{codeblock}
    \end{column}
  \end{columns}
  \footnotetext{章这一级只在 \pkg{report} 和 \pkg{book} 文档类(包括对应的中文文档类)有定义。还有不常用的 \cmd{part} (0@\pkg{article}/-1@\pkg{report}\&\pkg{book}\&\pkg{beamer}) 以及更低层次的 \cmd{paragraph} (4) 与 \cmd{subparagraph} (5)。 }
\end{frame}

\begin{frame}[fragile]%
  \frametitle{组织文档}
  \begin{columns}
    \begin{column}{0.4\textwidth}
      \only<1>{
        \cmd{tableofcontents} 用来生成对于目次命令的目录。如果你想设定显示到哪个层级,在这个命令前使用 \cmd{setcounter\{tocdepth\}\{层次\}}
      }
      \only<2>{
        对于大型文档而言,使用多个文件管理源文件通常是更方便的。而 \cmd{include} 和 \cmd{input} 都以相对路径的方式插入其他 \TeX{} 文档。
        区别在于,\cmd{include} 命令会从新页开始并做一些内部调整,这基本上只对 \pkg{chapter} 这一级有用。而 \cmd{input} 会原样插入源代码。
      }
      \only<3>{
        但是 \cmd{include} 插入的文档可以使用 \cmd{includeonly} 管理当前要排印哪一部分的内容,利用所有章节辅助文件的同时,减少编译时间并专注于该部分的内容。
      }
    \end{column}
    \begin{column}{0.6\textwidth}
      \begin{codeblock}[]{主文档}
\documentclass{ctexrep}
|\only<3>{\highlightline}|\includeonly{learnlatex,sjtuthesis}
\begin{document}
|\only<1>{\highlightline}|  \tableofcontents
|\only<2-3>{\highlightline}|  \include{learnlatex}
|\only<3>{\highlightline}|  \include{sjtuthesis}
\end{document}
      \end{codeblock}
    \end{column}
  \end{columns}
\end{frame}

\begin{frame}[fragile]
  \frametitle{组织文档}
  \begin{columns}
    \begin{column}{0.4\textwidth}
      \begin{codeblock}[]{learnlatex.tex}
|\highlightline|\chapter{||学习 \LaTeX{}}
\section{||概念}
\subsection{\LaTeX{}}
\LaTeX{} 是一个用以排版高质量作品的文档准备系统。
      \end{codeblock}
      子文件中就不需要添加 \env{document} 环境了\footnotemark。
    \end{column}
    \begin{column}{0.6\textwidth}
      \begin{codeblock}[]{主文档}
|\highlightline|\documentclass{ctexrep}
\includeonly{learnlatex,sjtuthesis}
\begin{document}
  \tableofcontents
  \include{learnlatex}
  \include{sjtuthesis}
\end{document}
      \end{codeblock}
    \end{column}
  \end{columns}
  \footnotetext{如果想强制指定子文档的主文档,可以在文件第一行输入魔术命令:\texttt{\% !TeX root = main.tex}}
\end{frame}

\section{图}
\begin{frame}[fragile]%
  \frametitle{\temporal<5>{插图}{浮动体}{插图}}
  \begin{columns}
    \begin{column}{0.6\textwidth}
      \begin{codeblock}[]{插入单图\only<4->{最佳实践}}
\documentclass{ctexart}
|\only<2>{\highlightline}|\usepackage{graphicx}
|\only<2>{\highlightline}|\graphicspath{{figs/}{pics/}}
\begin{document}
|\only<5>{\highlightline}|\begin{figure}[ht]
|\only<6>{\highlightline}|  \centering
|\only<3>{\highlightline}|  \includegraphics[width=|\only<1-3>{4cm}\only<4->{0.4\textbackslash{}textwidth}|]{sjtug}
|\only<7>{\highlightline}|  \caption{SJTUG 徽标}\label{fig:sjtug}
|\only<5>{\highlightline}|\end{figure}
\end{document}
      \end{codeblock}
    \end{column}
    \begin{column}{0.4\textwidth}
      \only<1>{
        \includepdflarge{insertimage}
      }
      \only<2>{
        为了插入外部图片,需要使用 \pkg{graphicx} 宏包。之后在文档主体便可以使用 \cmd{includegraphics} 插入图片。导言区也可以加入 \cmd{graphicspath} 指定图片文件夹\footnotemark。
      }
      \only<3>{
        \cmd{includegraphics} 命令便以相对路径的方式插入图片,如果无同名图片,那么后缀名可以省略。可以使用可选参数指定插入的图片尺寸,最佳实践是使用 \cmd{textwidth} 或 \cmd{linewidth} 的相对值指定尺寸大小,以在未来可能的布局更改中保留一定的灵活性。
      }
      \only<4>{
        也可以通过键值对的方法设置图片的其他属性。
        \begin{center}
          \footnotesize
          \begin{stampbox}
            \begin{tabular}{rl}
              \pkg{width} & 宽度 \\
              \pkg{height} & 高度 \\
              \pkg{scale} & 缩放 \\
              \pkg{angle} & 角度 \\
            \end{tabular}
          \end{stampbox}
        \end{center}
      }
      \only<5>{
        \env{figure} 为一个浮动体环境(\env{table} 也是),可以将其移动到其他位置上以减少行文中的空白。可以添加可选参数以指定如何放置浮动体,最多可以使用四种位置描述符:
        \begin{center}
          \footnotesize
          \begin{stampbox}
            \begin{tabular}{cll}
              \pkg{h} & Here & 尽可能在这里 \\
              \pkg{t} & Top & 页面顶部 \\
              \pkg{b} & Bottom & 页面底部 \\
              \pkg{p} & Page & 浮动体专页 \\
            \end{tabular}
          \end{stampbox}
        \end{center}
        还可以只使用 \pkg{float} 宏包提供的 \pkg{H} 描述符以强制置于此处。
      }
      \only<6>{
        采用 \cmd{centering} 命令而不是 \env{center} 环境来水平居中图片。这将避免多余的纵向间距。
      }
      \only<7>{
        使用 \cmd{caption} 命令输入题注,如果这个命令写在插入图片前面,题注将会在上方(中文中一般对 \env{table} 环境这么做)。后面将会看到如何对留有标记(\cmd{label})的图片进行引用。
      }
    \end{column}
  \end{columns}
  \only<2>{\footnotetext{其命令参数每个为一个以 \texttt{/} 结尾的文件夹,每个文件夹需要使用大括号包裹起来。}}
\end{frame}

\begin{frame}[fragile]
  \begin{columns}
    \begin{column}{0.6\textwidth}
      \begin{codeblock}[]{插入双图}
\documentclass{ctexart}
\usepackage{graphicx}
\graphicspath{{figs/}{pics/}}
\begin{document}
  \begin{figure}[ht]
|\only<1>{\highlightline}|    \begin{minipage}{0.48\textwidth}
      \centering
      \includegraphics[height=2cm]{sjtug}
|\only<2>{\highlightline}|      \caption{SJTUG 徽标}\label{fig:sjtug}
|\only<1>{\highlightline}|    \end{minipage}\hfill
|\only<1>{\highlightline}|    \begin{minipage}{0.48\textwidth}
      \centering
      \includegraphics[height=2cm]{sjtugt}
|\only<2>{\highlightline}|      \caption{SJTUG||文字}\label{fig:sjtugt}
|\only<1>{\highlightline}|    \end{minipage}
  \end{figure}
\end{document}
      \end{codeblock}
    \end{column}
    \begin{column}{0.4\textwidth}
      \only<1>{
        在 \env{figure} 环境里使用 \env{minipage} 小页制作列盒子,以并排两图并分别编号,需要设定强制参数以指定列宽。两个小页之间添加 \cmd{hfill} 使两个小页两端对齐。
      }
      \only<2>{
        在每个小页内部分别使用 \cmd{caption} 添加标签。
      }
      \only<3>{
        \includepdflarge{doubleimages}
      }
    \end{column}
  \end{columns}
\end{frame}

\begin{frame}[fragile]%
  \begin{columns}
    \begin{column}{0.6\textwidth}
      \begin{codeblock}[]{}
\documentclass{ctexart}
\usepackage{graphicx}
|\highlightline|\usepackage{subcaption}
\graphicspath{{figs/}{pics/}}
\begin{document}
  \begin{figure}[ht]
|\highlightline|    \begin{subfigure}{0.48\textwidth}
      \centering
      \includegraphics[height=2cm]{sjtug}
      \caption{||徽标}
|\highlightline|    \end{subfigure}\hfill
|\highlightline|    \begin{subfigure}{0.48\textwidth}
      \centering
      \includegraphics[height=2cm]{sjtugt}
      \caption{||文字}
|\highlightline|    \end{subfigure}
    \caption{SJTUG}\label{fig:sjtug}
  \end{figure}
\end{document}
      \end{codeblock}
    \end{column}
    \begin{column}{0.4\textwidth}
      \includepdflarge{subfigures}\vspace{15pt}
      \pkg{subcaption} 宏包提供了 \env{subfigure} 环境(以及 \env{subtable}),可以用于以子图的形式插入,编号会增加一级。也可以为子图添加子集引用编号。
    \end{column}
  \end{columns}
\end{frame}

\section{表}
\begin{frame}[fragile]
  \frametitle{简单表格}
  \begin{columns}
    \begin{column}{0.6\textwidth}
      \begin{codeblock}[]{}
\documentclass{ctexart}
|\only<1-2>{\highlightline}|\usepackage{|\temporal<1>{array}{\highlight{array}}{array},\temporal<2>{booktabs}{\highlight{booktabs}}{booktabs}|}
\begin{document}
\begin{table}[ht]
  \centering
  \caption{||北京冬奥会吉祥物}
|\only<1>{\highlightline}|  \begin{tabular}{lp{3cm}}
|\only<2>{\highlightline}|    \toprule
|\only<3>{\highlightline}|吉祥物 & 描述                          \\
|\only<2>{\highlightline}|    \midrule
|\only<3>{\highlightline}|冰墩墩 & 2022 年北京冬季奥运会吉祥物  \\
|\only<3>{\highlightline}|雪容融 & 2022 年北京冬季残奥会吉祥物  \\
|\only<2>{\highlightline}|    \bottomrule
|\only<1>{\highlightline}|  \end{tabular}
\end{table}
\end{document}
      \end{codeblock}
    \end{column}
    \begin{column}{0.4\textwidth}
      \only<1>{
        使用 \env{tabular} 环境绘制表格。需要添加参数(称为\textbf{表格导言})以确定每一列的对齐方式。引入 \pkg{array} 宏包来使用更多的\textbf{引导符}。
        \begin{center}
          \footnotesize
          \begin{stampbox}
            \begin{tabular}{>{\ttfamily}ll}
              \alert{l} & 向左对齐 \\
              \alert{c} & 居中对齐 \\
              \alert{r} & 向右对齐 \\
              \alert{p\{3cm\}} & 固定列宽,两端对齐 \\
              \alert{m\{3cm\}} & \texttt{p} + 垂直居中对齐 \\
              \alert{>\{\textbackslash{}bfseries\}} & 后一列单元格前加命令 \\
              \alert{*\{3\}\{l\}} & 三个左对齐列 \\
            \end{tabular}
          \end{stampbox}
        \end{center}
      }
      \only<2>{
        \pkg{booktabs} 宏包提供了标准三线表格所需要的行分割线:\cmd{toprule},\cmd{midrule},\cmd{bottomrule}。也可以使用 \cmd{cmidrule\{1-2\}} 来部分地绘制行分割线。一般不推荐在专业表格中使用纵向分割线。
      }
      \only<3>{
        每行内容使用 \textbackslash\textbackslash{} 分隔开,每行中的单元格使用 \& 分隔开。
      }
      \only<4>{
        \includepdflarge{table}
      }
    \end{column}
  \end{columns}
\end{frame}

\begin{frame}[fragile]%
  \begin{columns}
    \begin{column}{0.6\textwidth}
      \begin{codeblock}[]{表头居中}
\documentclass{ctexart}
\usepackage{array,booktabs}
\begin{document}
\begin{table}[ht]
  \centering
  \caption{||北京冬奥会吉祥物}
  \begin{tabular}{lp{3cm}}
    \toprule
|\highlightline|\multicolumn{1}{c}{||吉祥物} &
|\highlightline|\multicolumn{1}{c}{||描述} \\
    \midrule
||冰墩墩 & 2022 年北京冬季奥运会吉祥物  \\
||雪容融 & 2022 年北京冬季残奥会吉祥物  \\
    \bottomrule
  \end{tabular}
\end{table}
\end{document}
      \end{codeblock}
    \end{column}
    \begin{column}{0.4\textwidth}
      \cmd{multicolumn} 命令不仅可以用于合并同行的单元格,还可以用于临时地屏蔽表格导言对该列的对齐方式定义。这里用于居中表头。
      \begin{center}
        \begin{stampbox}
          \parbox{0.85\linewidth}{
            \ttfamily\color{blue}\textbackslash{}multicolumn\{格数\}\{对齐方式\}\{内容\}
          }
        \end{stampbox}
      \end{center}
      跨页表格考虑使用 \pkg{longtable} 宏包。带标注的表格可以考虑使用 \pkg{threeparttable} 宏包。考虑使用在线工具生成表格代码 \link{https://www.tablesgenerator.com/latex_tables}。
    \end{column}
  \end{columns}
\end{frame}

\section{数学公式}
\begin{frame}
  \frametitle{数学模式}
  \begin{alertblock}{}
  输入数学公式是 \LaTeX{} 的绝对强项,很多常见的公式服务依赖于一些轻量级渲染引擎比如 MathJax, K\kern-.3ex\raise.4ex\hbox{\footnotesize A}\kern-.3ex\TeX{}。但是它们实际上使用的是 \LaTeX{} 语法变种,也就是说并没有使用 \LaTeX{} 后端。所以不要期望有完全一致的输出。
  \end{alertblock}
  
  为了更好的获得数学公式输入支持,请使用 \hologo{AmS}math 宏包。数学模式分为两种:
  \begin{description}
    \item[行内模式] 一般通过一对美元符号(\$$\cdots$\$)标记,可以使用编辑器内建的符号表输入数学符号,也可以使用在线工具手写识别 \link{https://detexify.kirelabs.org/classify.html}。
    \item[行间模式] 一般通过 \env{equation} 环境\footnote{这是有编号环境,其加星号的变种 \env{equation*} 用于生成无编号环境。}输入。如果需要使用多行公式,请使用 \env{align} 环境,并按照类似表格输入的方式,使用 \& 对齐符号,\textbackslash\textbackslash{} 换行。如果不想手动居中,可以考虑多行自动居中的 \env{gather} 和单个大型公式首尾两端对齐 \env{multline}。
  \end{description}
\end{frame}

\begin{frame}
  \frametitle{数学命令展示}
  \begin{columns}
    \begin{column}{0.33\textwidth}
      \begin{exampleblock}{}
        $PV=nRT$
      \end{exampleblock}
      \begin{exampleblock}{}
        $\sum_{i=1}^ki^2=\frac{n(n+1)(2n+1)}{6}$
      \end{exampleblock}
      \begin{exampleblock}{}
        $T(n) = aT\left(\left\lceil\frac{n}{b}\right\rceil\right) + \mathcal{O}(n^d)$
      \end{exampleblock}
      \begin{exampleblock}{}
        $\frac{x_{1}+x_{2}+x_{3}}{3}\geq \sqrt[3]{x_{1}x_{2}x_{3}}$
      \end{exampleblock}
      \begin{exampleblock}{}
        $n=(\underbrace{1\cdots 1}_{k\text{ of 1's}})_2=2^{k+1}-1$
      \end{exampleblock}
      \begin{exampleblock}{}
        $\nabla f (P)= \left.\left(\frac{\partial f}{\partial x},\frac{\partial f}{\partial y},\frac{\partial f}{\partial z}\right)\right|_{P}$
      \end{exampleblock}
    \end{column}
    \begin{column}{0.67\textwidth}
      \begin{exampleblock}{}
        \begin{equation*}
          \int_{a}^b f(x)\,\mathrm{d}x=\lim_{|P|\rightarrow 0}\sum_{i=1}^n f(\xi_i)\Delta x_i
        \end{equation*}
      \end{exampleblock}
      \begin{exampleblock}{}
        \begin{equation}
          T(n) = \begin{cases}
            \mathcal{O}(n^d),&\textrm{if } d>\log_b a, \\
            \mathcal{O}(n^d\log n), &\textrm{if } d=\log_b a,\\
            \mathcal{O}(n^{\log_b a}), &\textrm{if } d<\log_b a.
          \end{cases}
        \end{equation}
      \end{exampleblock}
      \begin{exampleblock}{}
        \begin{align}
          Q^{T}A&=R \\
          \begin{pmatrix}
            q_1^T \\ q_2^T \\ q_3^T
          \end{pmatrix}
          \begin{pmatrix}
            a_1 & a_2 & a_3
          \end{pmatrix}
          &=R
        \end{align}
      \end{exampleblock}
    \end{column}
  \end{columns}
\end{frame}

%更深入地讲解 mathtools, unicode-math, siunix

\section{引用}
\begin{frame}[fragile]
  \frametitle{交叉引用}
  \only<1>{
    正如之前所提到的,\LaTeX{} 中使用 \cmd{label} 标记,然后可以使用 \cmd{ref} 来引用这个标记。 \cmd{label} 可以放在使用计数器的对象之后。
  }
  \only<2>{
    为了使得对公式编号的引用带有括号,推荐使用 \hologo{AmS}math 宏包中的 \cmd{eqref} 命令。对于多行公式环境,每一个换行符前都可以添加一个 \cmd{label} 用于引用该行公式。
  }
  \begin{columns}
    \begin{column}{0.5\textwidth}
      \begin{codeblock}[]{图}
\begin{figure}
|\only<1>{\highlightline}|  \caption{||示例}\label{fig:example}
\end{figure}
      \end{codeblock}
      \begin{codeblock}[]{表}
\begin{table}
|\only<1>{\highlightline}|  \caption{||示例}\label{tab:example}
\end{table}
      \end{codeblock}
    \end{column}
    \begin{column}{0.5\textwidth}
\begin{codeblock}[]{目次}
|\only<1>{\highlightline}|\section{||示例}\label{sec:example}
\end{codeblock}

\begin{codeblock}[]{公式}
\begin{equation}
  a = b + c
|\only<1>{\highlightline}|\label{eq:example}
\end{equation}
|\only<2>{\highlightline}|如公式 \eqref{eq:example} 所示,
\end{codeblock}
    \end{column}
  \end{columns}
\end{frame}

\begin{frame}[fragile]
  \frametitle{文献引用}
  \LaTeX{} 管理参考文献可以采用专用数据库文件 \texttt{.bib},很多的文献管理文件比如 EndNote \link{https://lic.sjtu.edu.cn/Default/softshow/tag/MDAwMDAwMDAwMLGImKE}, Zotero \link{https://www.zotero.org/}, JabRef \link{https://www.jabref.org/} 都可以直接导出这种格式的文件用于 \LaTeX{} 论文中的引用。一般不需要手写数据库文件,你只需要注意每一个文献会在数据库中有一个主键,这个类似于上文的 \cmd{label} 标记,只是要引用数据库中的文献需要使用 \cmd{cite} 命令。
  
  \begin{codeblock}[]{ref.bib}
|\highlightline|@phdthesis{devoftech,|\hfill\alert{\% 类型为博士论文,主键为\texttt{devoftech}}|
  title={||新时期我国信息技术产业的发展},
  author={||江泽民},
  year={2008}
}
  \end{codeblock}
\end{frame}

\begin{frame}
  \frametitle{文献引用}
  而让 \LaTeX{} 处理 \texttt{.bib} 数据库文件与引用有两种工作流。你可能需要查清楚如何在编辑器中设置对应的工作流,或者采用后面所提到的高级编译方式 \texttt{latexmk}。
  \begin{columns}
    \begin{column}{0.5\textwidth}
      \begin{block}{\hologo{BibTeX} + \pkg{gbt7714}}
        一般期刊提交使用这种方法,\pkg{natbib} 宏包将提供命令 \cmd{citet}(文本引用) 和 \cmd{citep}(括号引用)。中文引用可以直接使用 \pkg{gbt7714} 宏包,但是角模式和正文模式不能混用。
      \end{block}
    \end{column}
    \begin{column}{0.5\textwidth}
      \begin{block}{\hologo{biber} + \pkg{biblatex}}
        这是更容易自定义的方法,与 \hologo{BibTeX} 的运作方式稍有不同。\pkg{biblatex} 提供了更加智能的引用命令。而中文引用可以使用 \pkg{biblatex} 宏包的样式 \pkg{gb7714-2015},使用该样式需要使用 \hologo{XeLaTeX} 编译。
      \end{block}
    \end{column}
  \end{columns}
\end{frame}

\begin{frame}[fragile]
  \frametitle{文献引用}
  \begin{columns}
    \begin{column}{0.5\textwidth}
      \begin{codeblock}[]{\hologo{BibTeX} + \pkg{gbt7714}}
\documentclass{ctexart}
\usepackage{gbt7714}
\bibliographystyle{gbt7714-numerial}
% \citestyle{numbers}  % 正文模式
\begin{document}
  ||他指出了...\cite{devoftech}
  \bibliography{ref}
\end{document}
      \end{codeblock}
    \end{column}
    \begin{column}{0.5\textwidth}
      \begin{codeblock}[]{\hologo{biber} + \pkg{biblatex}}
\documentclass{ctexart}
\usepackage[backend=biber,style=gb7714-2015]{biblatex}
\addbibresource{ref.bib}
\begin{document}
  ||他在文献 \parencite{devoftech}
  ||指出了...\cite{devoftech}
  \printbibliography
\end{document}
      \end{codeblock}
    \end{column}
  \end{columns}
\end{frame}

\begin{frame}
  \frametitle{文献引用}
  \begin{columns}
    \begin{column}{0.5\textwidth}
      \includepdflarge{bibtex}
    \end{column}
    \begin{column}{0.5\textwidth}
      \includepdflarge{biblatex}
    \end{column}
  \end{columns}
\end{frame}

} % End of customized shaded number logo

|\only<3>{\highlightline}|  % !TeX root = ..\..\latex-talk.tex

\part{SJTUThesis}

\begin{frame}
  \frametitle{简介}
  \begin{columns}
    \begin{column}{0.6\textwidth}
      \begin{itemize}
        \item 最早由韦建文于 2009 年 11 月发布 0.1a 版,2018 年起由 SJTUG 接手维护
        \item 最新版:\SJTUThesisVersion{} (\SJTUThesisDate)
        \item 支持本科、硕士、博士学位论文以及课程论文的排版
      \end{itemize}
    \end{column}
    \begin{column}{0.4\textwidth}
      \begin{exampleblock}{}
        \begin{minipage}[c]{1cm}
          \includegraphics[width=0.8cm]{\getcontribpath{sjtug}{vi/sjtug}}
        \end{minipage}
        \begin{minipage}[c]{2cm}
          \href{https://github.com/sjtug}{sjtug}/\href{https://github.com/sjtug/SJTUThesis}{SJTUThesis}
        \end{minipage}
      \end{exampleblock}
      \vspace{-8pt}
      \begin{block}{}
        \scriptsize
        上海交通大学 \hologo{XeLaTeX} 学位论文及课程论文模板 | Shanghai Jiao Tong University \hologo{XeLaTeX} Thesis Template
      \end{block}
      \vspace{-8pt}
      \begin{alertblock}{}
        \scriptsize
        \begin{tabular}{cl}
          \faStar & 2.4k \\
          \faEye & 55 \\
          \faCodeBranch & 701 \\
        \end{tabular}
      \end{alertblock}
    \end{column}
  \end{columns}
\end{frame}

\begin{frame}
  \frametitle{下载与编译}
  \alert{下载} 推荐安装 Git \link{https://git-scm.com/} 后,克隆 SJTUG 镜像仓库
  \begin{exampleblock}{\faGit*}
    \ttfamily\small
    git clone https://mirror.sjtu.edu.cn/git/SJTUThesis.git/
  \end{exampleblock}

  \alert{编译} 推荐使用 \pkg{latexmk} 编译\footnote{\hologo{MiKTeX} 用户需要手动安装 Perl 解释器 \link{https://www.perl.org/get.html} 才能使用 \pkg{latexmk}。},在不能够利用自带的 \texttt{.latexmkrc} 配置文件的情况下,需要查清楚在对应的编辑器中如何使用 \hologo{XeLaTeX} + \hologo{biber} 编译 \link{https://github.com/sjtug/SJTUThesis/blob/master/README.md}。
  \begin{exampleblock}{\faTerminal}
    \ttfamily\small
    latexmk -xelatex main
  \end{exampleblock}

  Overleaf 用户可以下载压缩包后,上传并采用 \hologo{XeLaTeX} 编译方式。
\end{frame}

\begin{frame}
  \frametitle{手动编译}
  \alert{第一次编译失败} 如果没有办法通过通常方式编译成功,请尝试使用文件夹内附带 \faLinux{}\,\faApple{} \texttt{Makefile} 和 \faWindows{} \texttt{Compile.bat} 进行编译。

  \alert{统计字数} 编写过程中也可以使用对应的命令调用 \TeX{}count 来统计正文字数。
  \begin{columns}
    \begin{column}{0.38\textwidth}
      \begin{exampleblock}{\faLinux{}\,\faApple}
        \ttfamily
        make all\\
        make clean\\
        make cleanall\\
        make wordcount
      \end{exampleblock}
    \end{column}
    \begin{column}{0.38\textwidth}
      \begin{exampleblock}{\faWindows}
        \ttfamily
        ./Compile.bat thesis\\
        ./Compile.bat clean\\
        ./Compile.bat cleanall\\
        ./Compile.bat wordcount
      \end{exampleblock}
    \end{column}
    \begin{column}{0.24\textwidth}
      \begin{block}{\faInfo}
        \ttfamily
        编译论文\\
        清理中间文件\\
        $\hookrightarrow +$删除论文\\
        统计字数
      \end{block}
    \end{column}
  \end{columns}
\end{frame}

\begin{frame}[label=compile]
  \frametitle{编译问题排查}
  \begin{columns}
    \begin{column}{0.33\textwidth}
      \begin{alertblock}{无法使用 \texttt{latexmk}\thesisissue{578}}
        \hologo{MiKTeX} 需要安装 Perl 解释器。
      \end{alertblock}  
      \begin{alertblock}{C\TeX{} 套装无法编译\thesisissue{446}}
        使用最新 \TeX{} 发行版。
      \end{alertblock}
      \begin{alertblock}{\hologo{pdfLaTeX} 无法编译\thesisissue{444}}
        请使用 \texttt{latexmk},或更改编辑器设置以 \hologo{XeLaTeX} 编译。
      \end{alertblock}
    \end{column}
    \begin{column}{0.33\textwidth}
      \begin{alertblock}{缺少字体\thesisissue{564} \thesisdiscuss{598}}
        更换字体集,或者安装对应字体。
      \end{alertblock}
      \begin{alertblock}{缺少汉字\thesisissue{533} \thesisdiscuss{617}}
        去除使用 fandol 字体集的设定。或者是安装字体后,改用 \texttt{fontset=adobe} 或 \texttt{fontset=founder}。
      \end{alertblock}
    \end{column}
    \begin{column}{0.33\textwidth}
      \begin{block}{\faInfoCircle{} README}
        不同编辑器的设置请首先参阅 README \link{https://github.com/sjtug/SJTUThesis/blob/master/README.md} 文档。
      \end{block}
      \begin{block}{\faBookOpen{} Wiki}
        其他编译问题推荐查阅 Wiki \link{https://github.com/sjtug/SJTUThesis/wiki} 的使用说明部分。
      \end{block}
    \end{column}
  \end{columns}
\end{frame}

\begin{frame}[fragile, label=covers]
  \begin{codeblock}[firstnumber=3]{main.tex}
|\alert{\% 载入 SJTUThesis 模版}|
\documentclass[|\only<1>{\highlight{type}}\only<2>{type}|=|\only<1>{bachelor}\only<2>{\highlight{bachelor}}|]{sjtuthesis}
  \end{codeblock}
  \begin{figure}
    \parbox{0.9\textwidth}{
      \begin{subfigure}{0.20\textwidth}
        \framebox{\includegraphics[width=\linewidth]{support/thesis/bachelor}}
        \caption{\only<1>{学士}\only<2>{\texttt{bachelor}}}
      \end{subfigure}\hfill
      \begin{subfigure}{0.20\textwidth}
        \framebox{\includegraphics[width=\linewidth]{support/thesis/master}}
        \caption{\only<1>{硕士}\only<2>{\texttt{master}}}
      \end{subfigure}\hfill
      \begin{subfigure}{0.20\textwidth}
        \framebox{\includegraphics[width=\linewidth]{support/thesis/doctor}}
        \caption{\only<1>{博士}\only<2>{\texttt{doctor}}}
      \end{subfigure}\hfill
      \begin{subfigure}{0.20\textwidth}
        \framebox{\includegraphics[width=\linewidth]{support/thesis/course}}
        \caption{\only<1>{课程}\only<2>{\texttt{course}}}
      \end{subfigure}
      \caption{论文类型示例\only<2>{ \texttt{type}}}
    }
  \end{figure}
\end{frame}

\begin{frame}[fragile]
  \frametitle{文档类选项}
  % \framesubtitle{\textbackslash{}documentclass\{sjtuthesis\}}
  \begin{columns}
    \begin{column}{0.45\textwidth}
      \includegraphics[page=10]{thesisdir}
    \end{column}
    \begin{column}{0.55\textwidth}
      \begin{table}[H]
        \caption{文档类选项}
        \footnotesize
        \begin{tabular}{>{\ttfamily}rll}
          \toprule
          选项 & 含义 & 相关 \\
          \midrule
          type= & 指定论文类型 & 第 \ref{covers} 页\\
          fontset= & 指定字体 & 第 \ref{compile} 页\\
          \midrule
          review & 开启盲审模式 & \thesisissue{195} \thesisissue{686} \\
          twoside & 双页模式 & \thesisissue{554} \\
          oneside & 单页模式 & \thesisissue{694} \\
          openright & 章从奇数页开始 & \thesisdiscuss{724} \\
          openany & 章从任意页开始 & \thesisissue{446} \\
          \bottomrule
        \end{tabular}
      \end{table}
    \end{column}
  \end{columns}
\end{frame}

\begin{frame}[fragile]
  \frametitle{基本配置}
  \framesubtitle{\textbackslash{}input\{setup\}}
  \begin{columns}
    \begin{column}{0.45\textwidth}
      \includegraphics[page=9]{thesisdir}
    \end{column}
    \begin{column}{0.55\textwidth}
      \begin{codeblock}[firstnumber=12]{main.tex}
|\highlightline<1>|% 论文基本配置,加载宏包等全局配置
|\highlightline<1>|\input{setup}

\begin{document}

%TC:ignore

|\highlightline<2>|% 标题页
|\highlightline<2>|\maketitle
      \end{codeblock}
      \visible<2>{
        \cmd{sjtusetup} 中的 \pkg{info} 将会修改封面的信息设置(见第 \ref{covers} 页)。
      }
    \end{column}
  \end{columns}
\end{frame}

\begin{frame}[fragile]
  \frametitle{基本配置}
  \framesubtitle{\textbackslash{}sjtusetup}
  \begin{columns}
    \begin{column}{0.45\textwidth}
      \includegraphics[page=1]{thesisdir}
    \end{column}
    \begin{column}{0.55\textwidth}
      \begin{codeblock}[firstnumber=3]{setup.tex}
\sjtusetup{
  info = {
    title    = {||上海交通大学学位论文 \LaTeX{} 模板示例文档},
    title*   = {A Sample for \LaTeX-based SJTU Thesis Template},
    author   = {||某\quad{}某},
    author* = {Mo Mo},
  },
  style = { header-logo-color = red, 
  },
  name = {
    publications = {||攻读学位期间完成的论文},
  },
}
      \end{codeblock}
    \end{column}
  \end{columns}
\end{frame}

\begin{frame}
  \frametitle{基本配置}
  \framesubtitle{\textbackslash{}sjtusetup}
  \begin{columns}
    \begin{column}{0.45\textwidth}
      \includegraphics[page=1]{thesisdir}
    \end{column}
    \begin{column}{0.55\textwidth}
      \begin{table}[H]
        \centering
        \caption{info 域}
        \footnotesize
        \begin{tabular}{lll} \toprule
          命令作用 & 中文对应选项 & 英文对应选项 \\ \midrule
          论文标题 & \texttt{title} & \texttt{title*} \\
          关键字列表 & \texttt{keywords} & \texttt{keywords*} \\
          作者姓名&  \texttt{author} &\texttt{author*}\\
          申请学位名称 & \texttt{degree}&\texttt{degree*}\\
          院系名称 & \texttt{department} & \texttt{department*}\\
          专业名称 & \texttt{major} & \texttt{major*}\\
          导师 & \texttt{supervisor} & \texttt{supervisor*}\\
          副导师 & \texttt{assisupervisor} & \texttt{assisupervisor*}\\
          日期 & \multicolumn{2}{c}{\texttt{date}}\\
          学号 & \multicolumn{2}{c}{\texttt{id}}\\ \bottomrule
          \end{tabular}
      \end{table}
    \end{column}
  \end{columns}
\end{frame}

\begin{frame}[fragile]
  \frametitle{版权页}
  \framesubtitle{\textbackslash{}copyrightpage}
  \begin{columns}
    \begin{column}{0.45\textwidth}
      \only<1>{
        \includegraphics[page=9]{thesisdir}
      }
      \only<2>{
        \includegraphics[page=2]{thesisdir}
      }
      \only<3>{
        \begin{figure}[H]
          \framebox{\includegraphics[page=2,width=0.4\linewidth]{bachelor}}
          \caption{版权页}
        \end{figure}
      }
    \end{column}
    \begin{column}{0.55\textwidth}
      \begin{codeblock}[firstnumber=22]{main.tex}
|\highlightline<1>|% 原创性声明及使用授权书
|\highlightline<1>|\copyrightpage
|\highlightline<2>|% 插入外置原创性声明及使用授权书
|\highlightline<2>|% \copyrightpage[scans/sample-copyright-old.pdf]
      \end{codeblock}
      \only<1>{
        \cmd{copyrightpages} 可以用于插入版权页。
      }
      \only<2>{
        \cmd{copyrightpages} 也接受一个可选参数,用于直接使用扫描件。
      }
    \end{column}
  \end{columns}
\end{frame}

\begin{frame}[fragile]
  \frametitle{前置部分}
  \framesubtitle{\textbackslash{}frontmatter}
  \begin{columns}
    \begin{column}{0.45\textwidth}
      \only<1>{
        \includegraphics[page=9]{thesisdir}
      }
      \only<2>{
        \includegraphics[page=3]{thesisdir}
      }
      \only<3>{
        \begin{figure}[H]
          \begin{subfigure}{0.45\textwidth}
            \framebox{\includegraphics[page=3,width=\linewidth]{bachelor}}
            \caption{中文}
          \end{subfigure}\hfill
          \begin{subfigure}{0.45\textwidth}
            \framebox{\includegraphics[page=4,width=\linewidth]{bachelor}}
            \caption{英文}
          \end{subfigure}
          \caption{摘要}
        \end{figure}
      }
      \only<4>{
        \begin{figure}[H]
          \begin{subfigure}{0.30\linewidth}
            \centering
            \framebox{\includegraphics[page=5,width=0.6\linewidth]{bachelor}}
            \caption{目录}
          \end{subfigure}
          \begin{subfigure}{0.30\linewidth}
            \centering
            \framebox{\includegraphics[page=6,width=0.6\linewidth]{bachelor}}
            \caption{插图}
          \end{subfigure}

          \begin{subfigure}{0.30\linewidth}
            \centering
            \framebox{\includegraphics[page=7,width=0.6\linewidth]{bachelor}}
            \caption{表格}
          \end{subfigure}
          \begin{subfigure}{0.30\linewidth}
            \centering
            \framebox{\includegraphics[page=8,width=0.6\linewidth]{bachelor}}
            \caption{算法}
          \end{subfigure}
          \caption{索引}
        \end{figure}
      }
      \only<5>{
        \includegraphics[page=4]{thesisdir}
      }
      \only<6>{
        \begin{figure}[H]
          \framebox{\includegraphics[page=9,width=0.5\linewidth]{bachelor}}
          \caption{符号对照表}
        \end{figure}
      }
    \end{column}
    \begin{column}{0.55\textwidth}
      \begin{codeblock}[firstnumber=30]{main.tex}
|\highlightline<2-3>|% 摘要
|\highlightline<2-3>|\input{contents/abstract}

|\highlightline<4>|% 目录
|\highlightline<4>|\tableofcontents
|\highlightline<4>|% 插图索引
|\highlightline<4>|\listoffigures*
|\highlightline<4>|% 表格索引
|\highlightline<4>|\listoftables*
|\highlightline<4>|% 算法索引
|\highlightline<4>|\listofalgorithms*

|\highlightline<5-6>|% 符号对照表
|\highlightline<5-6>|\input{contents/nomenclature}
      \end{codeblock}
    \end{column}
  \end{columns}
\end{frame}

\begin{frame}[fragile]
  \frametitle{主体部分}
  \framesubtitle{\textbackslash{}mainmatter}
  \begin{columns}
    \begin{column}{0.45\textwidth}
      \only<1>{
        \includegraphics[page=5]{thesisdir}
      }
      \only<2>{
        \begin{figure}[H]
          \begin{subfigure}{0.30\linewidth}
            \centering
            \framebox{\includegraphics[page=11,width=0.6\linewidth]{bachelor}}
            \caption{简介}
          \end{subfigure}
          \begin{subfigure}{0.30\linewidth}
            \centering
            \framebox{\includegraphics[page=13,width=0.6\linewidth]{bachelor}}
            \caption{数学}
          \end{subfigure}

          \begin{subfigure}{0.30\linewidth}
            \centering
            \framebox{\includegraphics[page=16,width=0.6\linewidth]{bachelor}}
            \caption{浮动体}
          \end{subfigure}
          \begin{subfigure}{0.30\linewidth}
            \centering
            \framebox{\includegraphics[page=22,width=0.6\linewidth]{bachelor}}
            \caption{总结}
          \end{subfigure}
          \caption{主体部分}
        \end{figure}
      }
    \end{column}
    \begin{column}{0.55\textwidth}
      \begin{codeblock}[firstnumber=47]{main.tex}
|\highlightline|% 正文内容
|\highlightline|\input{contents/intro}
|\highlightline|\input{contents/math_and_citations}
|\highlightline|\input{contents/floats}
|\highlightline|\input{contents/summary}

%TC:ignore

% 参考文献
\printbibliography[heading=bibintoc]
      \end{codeblock}
    \end{column}
  \end{columns}
\end{frame}

\begin{frame}
  \frametitle{数学}
  \begin{itemize}
    \item 公式示例:\nolinkurl{contents/math_and_citations.tex}
    \item \SJTUThesis{} 定义了常用的数学环境(需要手工引入 \texttt{ntheorem} 宏包):
      \begin{table}[h]
        \centering
        \footnotesize
        \begin{tabular}{*{7}{l}}\toprule
          assumption  & axiom   & conjecture & corollary    & definition  & example   & exercise  \\
          假设        & 公理    & 猜想       & 推论         & 定义        & 例        & 练习      \\\midrule
          lemma       & problem & proof      & proposition  & remark      & solution  & theorem   \\
          引理        & 问题    & 证明       & 命题         & 注          & 解        & 定理      \\\bottomrule
        \end{tabular}
      \end{table}
      \item \SJTUThesis{} 可以通过 \texttt{unimath} 选项使用 \pkg{unicode-math} 进行数学输入,注意与传统方式的区别。\thesisissue{555}
  \end{itemize}
\end{frame}

\begin{frame}[fragile]
  \frametitle{参考文献}
  \begin{columns}
    \begin{column}{0.45\textwidth}
      \includegraphics[page=6]{thesisdir}
    \end{column}
    \begin{column}{0.55\textwidth}
      \begin{codeblock}[firstnumber=111,numbersep=2pt]{setup.tex}
% 使用 BibLaTeX 处理参考文献
%   biblatex-gb7714-2015 常用选项
%     gbnamefmt=lowercase     姓名大小写由输入信息确定
%     gbpub=false             禁用出版信息缺失处理
\usepackage[backend=biber,style=gb7714-2015]{biblatex}
% 文献表字体
% \renewcommand{\bibfont}{\zihao{-5}}
% 文献表条目间的间距
\setlength{\bibitemsep}{0pt}
|\highlightline|% 导入参考文献数据库
|\highlightline|\addbibresource{bibdata/thesis.bib}
      \end{codeblock}
    \end{column}
  \end{columns}
\end{frame}

\begin{frame}[fragile]
  \frametitle{附录}
  \framesubtitle{\textbackslash{}appendix}
  \begin{columns}
    \begin{column}{0.45\textwidth}
      \only<1>{
        \includegraphics[page=7]{thesisdir}
      }
      \only<2>{
        \begin{figure}[H]
          \begin{subfigure}{0.45\linewidth}
            \framebox{\includegraphics[width=\linewidth,page=24]{bachelor}}
            \caption{}
          \end{subfigure}\hfill
          \begin{subfigure}{0.45\textwidth}
            \framebox{\includegraphics[width=\linewidth,page=25]{bachelor}}
            \caption{}
          \end{subfigure}
          \caption{附录}
        \end{figure}
      }
    \end{column}
    \begin{column}{0.55\textwidth}
      \begin{codeblock}[firstnumber=61]{main.tex}
% 附录中图表不加入索引
\captionsetup{list=no}

% 附录内容
|\highlightline|\input{contents/app_maxwell_equations}
|\highlightline|\input{contents/app_flow_chart}
      \end{codeblock}
    \end{column}
  \end{columns}
\end{frame}

\begin{frame}[fragile]
  \frametitle{结尾部分}
  \framesubtitle{\textbackslash{}backmatter}
  \begin{columns}
    \begin{column}{0.45\textwidth}
      \only<1>{
        \includegraphics[page=8]{thesisdir}
      }
      \only<2>{
        \begin{figure}[H]
          \begin{subfigure}{0.30\linewidth}
            \centering
            \framebox{\includegraphics[page=26,width=0.6\linewidth]{bachelor}}
            \caption{致谢}
          \end{subfigure}
          \begin{subfigure}{0.30\linewidth}
            \centering
            \framebox{\includegraphics[page=27,width=0.6\linewidth]{bachelor}}
            \caption{成就}
          \end{subfigure}

          \begin{subfigure}{0.30\linewidth}
            \centering
            \framebox{\includegraphics[page=28,width=0.6\linewidth]{bachelor}}
            \caption{简历}
          \end{subfigure}
          \begin{subfigure}{0.30\linewidth}
            \centering
            \framebox{\includegraphics[page=29,width=0.6\linewidth]{bachelor}}
            \caption{大摘要*}
          \end{subfigure}
          \caption{结尾部分}
        \end{figure}
      }
    \end{column}
    \begin{column}{0.55\textwidth}
      \begin{codeblock}[firstnumber=76]{main.tex}
% 致谢
\input{contents/acknowledgements}

% 发表论文及科研成果
% 盲审论文中,发表论文及科研成果等仅以第几作者注明即可,不要出现作者或他人姓名
\input{contents/achievements}

% 简历
\input{contents/resume}

% 学士学位论文要求在最后有一个大摘要,单独编页码
\input{contents/digest}
      \end{codeblock}
    \end{column}
  \end{columns}
\end{frame}

\begin{frame}
  \frametitle{还有其他问题?}
  \begin{columns}
    \begin{column}{0.75\textwidth}
    \begin{itemize}
      \item[{\faComment*[regular]}] 日常模板或 \LaTeX{} 使用问题可以前往 Discussions \link{https://github.com/sjtug/SJTUThesis/discussions} 提问
      
      (解决后别忘了 \textcolor{green}{\faCheckCircle{} Mark as answer}
      \item[{\faDotCircle[regular]}] 如果是 \textsc{SJTUThesis} 项目本身的 bug 和 feature request
      
      可以通过 Issues \link{https://github.com/sjtug/SJTUThesis/issues} 反馈。
      \item[{\faCodeBranch}] 如果你有好点子,可以贡献代码
     
      向 \textsc{SJTU\TeX{}}(v1) \link{https://github.com/sjtug/SJTUTeX/tree/v1} 存储库发 PR,\par
      而后把解包结果同步到 \textsc{SJTUThesis}。
  
      \item[{\faTag}] 如果你对正在基于 \LaTeX3 开发的新版感兴趣,\par
      也欢迎向 \textsc{SJTU\TeX{}}(v2) \link{https://github.com/sjtug/SJTUTeX/tree/v2} 发 PR。
  
      \item[{\faQq}] 也欢迎在 QQ 群即时讨论。
    \end{itemize}
    \end{column}
    \begin{column}{0.25\textwidth}
      \includegraphics[height=0.7\textheight]{qq.jpg}
    \end{column}
  \end{columns}
\end{frame}
\end{document}
      \end{codeblock}
    \end{column}
  \end{columns}
\end{frame}

\begin{frame}[fragile]
  \frametitle{组织文档}
  \begin{columns}
    \begin{column}{0.4\textwidth}
      \begin{codeblock}[]{learnlatex.tex}
|\highlightline|\chapter{||学习 \LaTeX{}}
\section{||概念}
\subsection{\LaTeX{}}
\LaTeX{} 是一个用以排版高质量作品的文档准备系统。
      \end{codeblock}
      子文件中就不需要添加 \env{document} 环境了\footnotemark。
    \end{column}
    \begin{column}{0.6\textwidth}
      \begin{codeblock}[]{主文档}
|\highlightline|\documentclass{ctexrep}
\includeonly{learnlatex,sjtuthesis}
\begin{document}
  \tableofcontents
  % !TeX root = ..\..\latex-talk.tex

\part{学习 \LaTeX{}}
% FIXME: Part Page miniframe overflow
% FIXME: footnote fault numbering

\begin{frame}[plain]
  \vfil
  \begin{center}
    \href{https://learnlatex.org}{
      \rmfamily
      Learn\,\lower1ex\hbox{\Huge\LaTeX{}}.org
    }
  \end{center}
  \vfil
  \begin{center}
    \parbox{0.75\linewidth}{
      Learn\LaTeX{}.org\cite{learnlatex} 提供了解 \LaTeX{} 的 16 篇简短的教程,并包含了一些可以在线运行的示例,可以通过亲自动手查看实验效果。本部分主要参考由 C\TeX{}-org 提供的中文翻译版本 \link{https://github.com/CTeX-org/learnlatex.github.io/tree/zh-Hans/zh-Hans/}。
    }
  \end{center}
  \vfil
\end{frame}

{ % Start of shaded number logo

\newcommand{\shadedfont}[2][1pt]{
  % #1 (optional): shadow distance
  % #2: the text needed to be shaded
  \hbox{\rlap{\color{gray}\hskip#1#2}#2}
}
\newcounter{learnsec}
\setcounter{learnsec}{-1}
\newcommand{\updatelogo}{
  % update the logo corresponding to current counter.
  \stepcounter{learnsec}
  \logo{
    \raise.3ex\hbox{\tiny\insertsection}\shadedfont{\arabic{learnsec}}
  }
}
\let\oldsection=\section
\renewcommand{\section}[1]{\oldsection{#1}\updatelogo}

\section{是什么}
\begin{frame}
  \frametitle{\TeX{}}
  \begin{columns}[c]
    \begin{column}{0.7\textwidth}
      \begin{center}
        \rmfamily\Huge
        \hologo{La}\highlight[structure!70]{\TeX{}}
      \end{center}
      \begin{center}
        \parbox{0.75\textwidth}{
          \TeX{} 是由斯坦福大学教授高德纳
          (Donald E.~Knuth)于 1977 年开始开发的排版引擎。目前仍在更新,最新版本号为 3.141592653 \link{https://tug.org/TUGboat/tb42-1/tb130knuth-tuneup21.pdf}。
        }
      \end{center}
    \end{column}
    \begin{column}{0.3\textwidth}
      \includegraphics[width=.8\columnwidth]{Knuth.jpg}
    \end{column}
  \end{columns}
\end{frame}

\begin{frame}
  \frametitle{\LaTeX{}}
  \begin{columns}[c]
    \begin{column}{0.7\textwidth}
      \begin{center}
        \rmfamily\Huge
        \highlight[structure]{\LaTeX{}}
      \end{center}
      \begin{center}
        \parbox{0.75\textwidth}{
          \LaTeX{} 是最早在 1985 年由现就职于微软的 Leslie Lamport 开发的一种 \TeX{} \textbf{格式}\footnotemark,使用一些列宏和扩展宏包来简化 \TeX{} 的使用。现在由 \LaTeX{} Project 的成员维护。现在广泛使用的版本是 \LaTeXe{},最新的版本为 \LaTeX3(2020 年 10 月后默认内置)。
        }
      \end{center}
    \end{column}
    \begin{column}{0.3\textwidth}
      \includegraphics[width=.8\columnwidth]{Lamport.jpg}
    \end{column}
  \end{columns}
  \footnotetext{\hologo{ConTeXt} 也是一种 \TeX{} 格式 \link{https://www.contextgarden.net/}。}
\end{frame}

\begin{frame}
  \frametitle{程序}
  \begin{columns}[c]
    \begin{column}{0.7\textwidth}
      \begin{center}
        \rmfamily\Huge
        \highlight[structure]{\hologo{pdfLaTeX}}
      \end{center}
      \begin{center}
        \parbox{0.7\textwidth}{
          \hologo{pdfLaTeX} 是为了编译一个 \LaTeX{} 文档而运行的程序。实际上底层在运行一个叫 \hologo{pdfTeX} 的引擎,并预装了对应的 \LaTeX{} \textbf{格式}。为了利用临时文件,可能就需要多次运行程序。
        }
      \end{center}
    \end{column}
    \begin{column}{0.3\textwidth}
      \begin{block}{}
        \ttfamily\small
        > \highlight{pdflatex} main.tex\\
        This is pdfTeX, Version 3.141592653-
        2.6-1.40.23 (MiKTeX 21.10)\\
        entering extended mode\\
        \highlight{LaTeX2e} <2021-11-15>\\
        \highlight{L3} programming layer <2021-11-22>
      \end{block}
    \end{column}
  \end{columns}
\end{frame}

\begin{frame}
  \frametitle{引擎}
  \begin{columns}[c]
    \begin{column}{0.7\textwidth}
      \begin{center}
        \rmfamily\Huge
        \highlight[structure!70]{pdf}\hologo{La}\highlight[structure!70]{\TeX{}}
      \end{center}
      \begin{center}
        \parbox{0.7\textwidth}{
          pdf\TeX{} 是编译 \TeX{} 文档(以 \texttt{.tex} 结尾)的\textbf{引擎}---可以理解 \TeX{} 指令的\textbf{程序}。
        }
      \end{center}
    \end{column}
    \begin{column}{0.3\textwidth}
      \begin{block}{}
        \ttfamily\small
        > pdflatex main.tex\\
        This is \highlight[structure!70]{pdfTeX}, Version 3.141592653-
        2.6-1.40.23 (MiKTeX 21.10)
        entering extended mode\\
        LaTeX2e <2021-11-15>\\
        L3 programming layer <2021-11-22>
      \end{block}
    \end{column}
  \end{columns}
\end{frame}

\begin{frame}
  \frametitle{Unicode 引擎}
  \begin{table}
    \caption{主流 \hologo{(La)TeX} 程序
    \footnote{(u)p\TeX{} 是日语最常用的引擎,生成 \texttt{.dvi},支持 Unicode。}\footnote{Ap\TeX{} 具有底层 CJK 支持,内联 Ruby,Color Emoji。}}
    \footnotesize
    \begin{stampbox}
      \begin{tabular}{c>{\raggedright}*{3}{p{3.5cm}}}
        \alert{引擎}     & \hologo{pdfTeX}   & \hologo{XeTeX}   & \hologo{LuaTeX}   \\
        \alert{程序}     & \hologo{pdfLaTeX} & \hologo{XeLaTeX} & \hologo{LuaLaTeX} \\
        \alert{特点}     & 直接生成 PDF,支持 micro-typography  & 支持 Unicode、OpenType 与复杂文字编排 (CTL) & 支持 Unicode,内联 Lua,支持 OpenType \\
      \end{tabular}
    \end{stampbox}
  \end{table}

  \begin{center}
    \parbox{.9\textwidth}{
      \hologo{pdfLaTeX} 不支持 Unicode。为了排版中文,大部分情况下 \faApple{}\,\faLinux{} 应当使用 \hologo{XeLaTeX},而 \hologo{LuaLaTeX} 速度相对较慢。\faWindows{} 可以在一些情况下使用 \hologo{pdfLaTeX}。
    }
  \end{center}
\end{frame}

% \begin{frame}
%   \paragraph{\hologo{pdfLaTeX}} \TeX{} 和 \LaTeX{} 被广泛使用之前,它们只需内置支持欧洲语言即可。在 Unicode 出现之前,\LaTeX{} 提供了许多种\textbf{文件编码}来允许很多语言的文字以原生的方式输入,\hologo{pdfLaTeX} 也只需要使用 8 位文件编码和 8 位字体。
% \end{frame}

\section{跑起来}
\begin{frame}
  \frametitle{发行版}
  \begin{table}
    \caption{\hologo{TeX} 发行版}
    \footnotesize
    \begin{stampbox}
      \begin{tabular}{c>{\raggedright}*{3}{p{3.2cm}}}
        \alert{发行版}     & \hologo{MiKTeX} \link{https://mirrors.sjtug.sjtu.edu.cn/ctan/systems/win32/miktex/setup/windows-x64/basic-miktex-21.12-x64.exe}   & \TeX{} Live \link{https://mirrors.sjtug.sjtu.edu.cn/ctan/systems/texlive/tlnet/install-tl.zip}   & Mac\TeX{} \link{https://mirrors.sjtug.sjtu.edu.cn/ctan/systems/mac/mactex/mactex-20210328.pkg}  \\[2pt]
        \alert{特点}      &  只安装必要文件,依赖用时更新  &  所有平台均可使用,每年发布一次 & Mac 系统专用,对 \TeX{} Live 的进一步打包 \\
        \alert{推荐平台}  & \faWindows  & \faLinux &  \faApple \\
      \end{tabular}
    \end{stampbox}
  \end{table}
  \begin{center}
    \parbox{.9\textwidth}{
      要让 \LaTeX{} 跑起来,核心就是要有一套 \TeX{} 发行版,来获取让 \LaTeX{} 工作所需的一组程序和文件。参考《一份简短的关于 \LaTeX{} 安装的介绍》\link{https://mirrors.sjtug.sjtu.edu.cn/ctan/info/install-latex-guide-zh-cn/install-latex-guide-zh-cn.pdf} 安装想使用的发行版。推荐使用发行版的最新版本\footnote{老版本 Linux 系统的包管理器自带 \TeX{} Live 发行版可能不是最新的 \link{https://repology.org/project/texlive/versions},尽量使用镜像安装,并手动将相关环境变量添加到路径 \link{https://www.tug.org/texlive/doc/texlive-zh-cn/texlive-zh-cn.pdf}。},并使用国内镜像。
    }
  \end{center}
\end{frame}

\begin{frame}[plain]
  \hbox to \textwidth{
    \hfil
    \vbox to 3cm{
      \hbox{
        \resizebox{3cm}{!}{\includegraphics{\getcontribpath{sjtug}{vi/sjtug.pdf}}}
      }
    }
    \hfil
    \vbox to 3cm{
      \vfill
      \hbox{\Large\bfseries\color{cprimary} 稳定、快速、现代的镜像服务。}
      \vskip2pt
      \hbox{托管于华东教育网骨干节点上海交通大学。}
      \vfill
    }
    \hskip20pt
    \hfil
  }

  \begin{center}
    \parbox{0.8\textwidth}{
      推荐使用 SJTUG 软件镜像服务,离得近,下得快。
      
      \begin{description}
        \footnotesize
        \item[\TeX{} Live]  {\ttfamily tlmgr option repository https://mirrors.sjtug.sjtu.edu.cn/CTAN/systems/texlive/tlnet}
        \item[\hologo{MiKTeX}] 在 \hologo{MiKTeX} Console 中设置镜像源为 \url{https://mirrors.sjtug.sjtu.edu.cn}
      \end{description}
    }
  \end{center}
\end{frame}

\begin{frame}
  \frametitle{编辑器}
  \begin{table}
    \caption{开源编辑器推荐}
    \footnotesize
    \begin{stampbox}
      \begin{tabular}{c>{\raggedright}*{3}{p{3.5cm}}}
        \alert{编辑器}     & \begin{tabular}{c}Visual Studio Code\\ \LaTeX{} Workshop\end{tabular}  & \TeX{}studio & \TeX{}works \\[5pt]
        \alert{特点}      &  搭配 VS Code 使用非常方便,易扩展  & 可以使用大量的菜单选项输入代码块,用户友好 & 只提供基础的高亮与运行方法,发行版自带\footnote{Mac\TeX{} 打包的是 \TeX{}Shop 编辑器。} \\
      \end{tabular}
    \end{stampbox}
  \end{table}
  \begin{center}
    \parbox{.9\textwidth}{
      使用专为 \LaTeX{} 设计的编辑器将带来更多便利,因为它们往往会提供一键编译、内置 PDF 阅读器以及语法高亮等功能。几乎所有现代的 \LaTeX{} 编辑器都提供 Sync\TeX{} 这一强大的功能,以在 PDF 和 代码间对应跳转。
    }
  \end{center}
\end{frame}

\begin{frame}
  \frametitle{在线平台}
  \begin{table}
    \caption{在线协作平台推荐}
    \footnotesize
    \begin{stampbox}
      \begin{tabular}{c>{\raggedright}*{2}{p{4cm}}}
        \alert{在线平台}     & Overleaf \link{https://www.overleaf.com/}  & 交大 \LaTeX{} 助手 \link{https://latex.sjtu.edu.cn/} \\[2pt]
        \alert{特点}      & 最流行的在线平台,提供大量的模板,但国内访问慢 & 校内平台,隐私保护有保障,共享项目限制少 \\
      \end{tabular}
    \end{stampbox}
  \end{table}
  \begin{center}
    \parbox{.9\textwidth}{
      在线平台允许你直接在网页中编辑文档,无需本地安装即可在后台运行 \LaTeX{},并显示生成的 PDF。可以参照 Overleaf 官方文档学习如何使用在线平台 \link{https://www.overleaf.com/learn}。
    }
  \end{center}
\end{frame}

\section{基本结构}
\begin{frame}[fragile]%
  \frametitle{文档部件}
  \begin{columns}[c]
    \begin{column}{0.4\textwidth}
      \only<1>{
        \cmd{documentclass} 命令加载了\textbf{文档类}。\pkg{article} 是由 \LaTeX{}提供的用于排版短文档的基本文档类。
        \begin{description}
          \footnotesize
          \item[\pkg{article}] 不包含章的短文档
          \item[\pkg{report}] 含有章的单面印刷文档
          \item[\pkg{book}] 含有章的双面印刷文档
          \item[\pkg{beamer}] 制作幻灯片
        \end{description}
      }
      \only<2>{
        \env{document} 环境用于指示文档主体的范围。\LaTeX{} 还有其他的使用 \cmd{begin} 和 \cmd{end} 的搭配,我们称这些为\textbf{环境}。它们将用来设定局部格式命令\footnotemark。
      }
      \only<3>{
        \includepdflarge{enminimal}
      }
    \end{column}
    \begin{column}{0.6\textwidth}
      \begin{codeblock}[]{排版英文最简示例}
|\only<1>{\highlightline}|\documentclass{article}
|\only<2>{\highlightline}|\begin{document}
|\only<3>{\highlightline}|  Together for a Shared Future
|\only<2>{\highlightline}|\end{document}
      \end{codeblock}
    \end{column}
  \end{columns}
  \only<2>{\footnotetext{环境实际上是一个组,只不过通过语义化的形式预装了对应的格式命令。普通的组可以直接使用一对大括号之间的内容 \{$\cdots$\} 表示。}}
\end{frame}

\section{扩展}
\begin{frame}[fragile]%
  \frametitle{中文排版}
  \begin{columns}[c]
    \begin{column}{0.4\textwidth}
      \only<1>{
        \cmd{usepackage} 用于使用宏包以向 \LaTeX{} 添加或修改功能,需要在\textbf{导言区}调用。
        这里使用 \pkg{ctex} 宏集以获得中文支持。其调用底层因随不同的引擎而不同。
        {
          \footnotesize
          \begin{stampbox}
            \begin{tabular}{c*{3}{c}}
              \alert{引擎}     & \hologo{pdfTeX}   & \hologo{XeTeX}   & \hologo{LuaTeX}   \\
              \alert{程序}     & \hologo{pdfLaTeX} & \hologo{XeLaTeX} & \hologo{LuaLaTeX} \\
              \alert{宏包}     & CJK\footnotemark & xeCJK & luatexja \\
              \alert{封装}     & \multicolumn{3}{c}{ctex} \\
            \end{tabular}
          \end{stampbox}
        }
        \vspace{-1cm}
      }
      \only<2>{
        C\TeX{} 建议对于之前提到的常规文档类,最佳实践是使用该宏集提供的四种中文文档类,以对特定类型提供额外的中文排版适配。
        \begin{center}
          \begin{stampbox}
            \footnotesize
            \begin{tabular}{cc}
              \pkg{ctexart} & \pkg{ctexrep} \\
              \pkg{ctexbook} & \pkg{ctexbeamer} \\
            \end{tabular}
          \end{stampbox}
        \end{center}
      }
      \only<3>{
        \includepdflarge{cnminimal}
      }
      \only<4>{
        大部分情况下,你都不应当在 \LaTeX{} 中强制断行:你几乎只是想另起一段,或者是想在段落之间添加空行(使用 \pkg{parskip} 宏包就可实现)。
        只有\alert{很少的}情况下你需要使用 \textbackslash{}\textbackslash{} 来另起一行而不另起一段。
      }
    \end{column}
    \begin{column}{0.6\textwidth}
      \begin{codeblock}[]{排版中文\only<2->{最佳实践}}
|\only<2>{\highlightline}|\documentclass{|\only<1>{article}\only<2->{ctexart}|}
|\only<1>{\highlightline\textbackslash{}usepackage\{ctex\}\hfill\color{cprimary}\% 导言区}|
\begin{document}
|\only<3>{\highlightline}|    一起向未来
|\only<4>{\highlightline}|
  Together for a Shared Future
\end{document}
      \end{codeblock}
    \end{column}
  \end{columns}
  \only<1>{\footnotetext{ctex 在 \faApple\,\faLinux{} 上已经不可以使用 \hologo{pdfLaTeX} 编译,以及在 \faWindows{} 上使用该引擎也会变更自动间距调整等功能的默认行为。}}
\end{frame}

\section{设定格式}
\begin{frame}[fragile]%
  \frametitle{字体样式}
  \begin{columns}
    \begin{column}{0.4\textwidth}
      \only<1>{
        \includepdflarge{fontstyle}
      }
      \only<2>{
        可以使用显示样式设定命令对小段做加粗、斜体、等宽等等的处理。
        \begin{center}
          \footnotesize
          \begin{stampbox}
            \begin{tabular}{rl}
              \cmd{textrm} & \textrm{衬线} \\
              \cmd{textbf} & \textbf{加粗} \\
              \cmd{textit} & \kaishu 斜体 \\
              \cmd{texttt} & \texttt{等宽} \\
              \cmd{textsf} & \textsf{无衬线} \\
              \cmd{textsc} & \textsc{Small Caps} \\
              \cmd{textsl} & \textsl{Slanted} \\
            \end{tabular}
          \end{stampbox}
        \end{center}
      }
      \only<3>{
        与 Word 不同的是,\LaTeX{} 一般情况下并不需要使用上面的显式命令,而是采用逻辑标记的方法,
        比如 \cmd{emph} 可以强调文字,以及下面将要提到的目次命令(第 \ref{sectioning} 页)。
        这样可以统一管理格式。
      }
    \end{column}
    \begin{column}{0.6\textwidth}
      \begin{codeblock}[]{样式}
\documentclass{ctexart}
\begin{document}
|\only<2>{\highlightline}|  \textbf{||一起向未来}

|\only<3>{\highlightline}|  \emph{Together for a Shared Future}
\end{document}
      \end{codeblock}
    \end{column}
  \end{columns}
\end{frame}

\begin{frame}[fragile]%
  \frametitle{\only<1-2>{字体大小}\only<3>{字体样式}}
  \begin{columns}
    \begin{column}{0.4\textwidth}
      \only<1>{
        \includepdflarge{fontsize}
      }
      \only<2>{
        同样地,你也可以显式地设定字体大小,但是这种命令会更改行文设置,所以需要使用一个组来限定作用范围\footnotemark。
        \begin{center}
          \footnotesize
          \begin{stampbox}
            \begin{tabular}{rl}
              \cmd{tiny} & \tiny 极小 \\
              \cmd{scriptsize} & \scriptsize 抄本大小  \\
              \cmd{footnotesize} & \footnotesize 脚注大小 \\
              \cmd{small} & \small 小 \\
              \cmd{normalsize} & \normalsize 正常大小 \\
              \cmd{large} & \large 大 \\
              \cmd{huge} & \Huge 巨大 \\
            \end{tabular}
          \end{stampbox}
        \end{center}
      }
      \only<3>{
        也可以使用字体样式对应的更改字体设置的命令,这对于大段文段的设置而言也是很方便的。
        \begin{center}
          \footnotesize
          \begin{stampbox}
            \begin{tabular}{ll}
              \cmd{textrm} & \cmd{rmfamily}\\
              \cmd{texttt} & \cmd{ttfamily}\\
              \cmd{textsf} & \cmd{sffamily}\\
              \cmd{textbf} & \cmd{bfseries}\\
              \cmd{textit} & \cmd{itshape}\\
              \cmd{textsc} & \cmd{scshape}\\
              \cmd{textsl} & \cmd{slshape}\\
            \end{tabular}
          \end{stampbox}
        \end{center}
      }
    \end{column}
    \begin{column}{0.6\textwidth}
      \begin{codeblock}[]{大小}
\documentclass{ctexart}
\begin{document}
|\only<2>{\highlightline}|  {\bfseries\Large 一起向未来\par}
|\only<3>{\highlightline}|  {\itshape Together for a Shared Future}
\end{document}
      \end{codeblock}
    \end{column}
  \end{columns}
  \only<2>{\footnotetext{注意最后显式地使用 \cmd{par} 在改回大小前结束该段,否则会导致下一行的行间距异常!}}
\end{frame}

\section{逻辑结构}
\begin{frame}[fragile]
  \frametitle{列表}
  \begin{columns}
    \begin{column}{0.35\textwidth}
      \begin{codeblock}[]{无序列表}
\documentclass{ctexart}
\begin{document}
|\highlightline|  \begin{itemize}
    \item 第一项
    \item 第二项
    \item 第三项
|\highlightline|  \end{itemize}
\end{document}
      \end{codeblock}
    \end{column}
    \begin{column}{0.35\textwidth}
      \begin{codeblock}[]{有序列表}
\documentclass{ctexart}
\begin{document}
|\highlightline|  \begin{enumerate}
    \item 第一项
    \item 第二项
    \item 第三项
|\highlightline|  \end{enumerate}
\end{document}
      \end{codeblock}
    \end{column}
    \begin{column}{0.35\textwidth}
      \begin{codeblock}[]{描述列表}
\documentclass{ctexart}
\begin{document}
|\highlightline|  \begin{description}
    \item[||第一] 文本
    \item[||第二] 文本
    \item[||第三] 文本  
|\highlightline|  \end{description}
\end{document}
      \end{codeblock}
    \end{column}
  \end{columns}
\end{frame}

%更深的列表技巧,定理环境等

\begin{frame}
  \frametitle{列表}
  \begin{columns}
    \begin{column}{0.35\textwidth}
      \includepdflarge{unordered}
    \end{column}
    \begin{column}{0.35\textwidth}
      \includepdflarge{ordered}
    \end{column}
    \begin{column}{0.35\textwidth}
      \includepdflarge{description}
    \end{column}
  \end{columns}
\end{frame}

\begin{frame}[fragile,label=sectioning]%
  \frametitle{目次结构}
  \begin{columns}
    \begin{column}{0.4\textwidth}
      \LaTeX{} 可以使用目次命令将文档划分层级\footnotemark,并自动设定对应字体样式和大小。
      \begin{center}
        \begin{stampbox}
          \footnotesize
          \begin{tabular}{rll}
           命令 & 中文 & 层次 \\
           \cmd{chapter} & 章\footnotemark & \sout{0} \\
           \cmd{section} & 节 & 1 \\
           \cmd{subsection} & 小节 & 2 \\
           \cmd{subsubsection} & 小小节 & 3 \\
          \end{tabular}
        \end{stampbox}
      \end{center}
    \end{column}
    \begin{column}{0.6\textwidth}
      \begin{codeblock}[]{目次}
\documentclass{ctexart}
\begin{document}
|\highlightline|  \section{||概念}
|\highlightline|  \subsection{\LaTeX{}}
  \LaTeX{} 是一个用以排版高质量作品的文档准备系统。
\end{document}
      \end{codeblock}
    \end{column}
  \end{columns}
  \footnotetext{章这一级只在 \pkg{report} 和 \pkg{book} 文档类(包括对应的中文文档类)有定义。还有不常用的 \cmd{part} (0@\pkg{article}/-1@\pkg{report}\&\pkg{book}\&\pkg{beamer}) 以及更低层次的 \cmd{paragraph} (4) 与 \cmd{subparagraph} (5)。 }
\end{frame}

\begin{frame}[fragile]%
  \frametitle{组织文档}
  \begin{columns}
    \begin{column}{0.4\textwidth}
      \only<1>{
        \cmd{tableofcontents} 用来生成对于目次命令的目录。如果你想设定显示到哪个层级,在这个命令前使用 \cmd{setcounter\{tocdepth\}\{层次\}}
      }
      \only<2>{
        对于大型文档而言,使用多个文件管理源文件通常是更方便的。而 \cmd{include} 和 \cmd{input} 都以相对路径的方式插入其他 \TeX{} 文档。
        区别在于,\cmd{include} 命令会从新页开始并做一些内部调整,这基本上只对 \pkg{chapter} 这一级有用。而 \cmd{input} 会原样插入源代码。
      }
      \only<3>{
        但是 \cmd{include} 插入的文档可以使用 \cmd{includeonly} 管理当前要排印哪一部分的内容,利用所有章节辅助文件的同时,减少编译时间并专注于该部分的内容。
      }
    \end{column}
    \begin{column}{0.6\textwidth}
      \begin{codeblock}[]{主文档}
\documentclass{ctexrep}
|\only<3>{\highlightline}|\includeonly{learnlatex,sjtuthesis}
\begin{document}
|\only<1>{\highlightline}|  \tableofcontents
|\only<2-3>{\highlightline}|  \include{learnlatex}
|\only<3>{\highlightline}|  \include{sjtuthesis}
\end{document}
      \end{codeblock}
    \end{column}
  \end{columns}
\end{frame}

\begin{frame}[fragile]
  \frametitle{组织文档}
  \begin{columns}
    \begin{column}{0.4\textwidth}
      \begin{codeblock}[]{learnlatex.tex}
|\highlightline|\chapter{||学习 \LaTeX{}}
\section{||概念}
\subsection{\LaTeX{}}
\LaTeX{} 是一个用以排版高质量作品的文档准备系统。
      \end{codeblock}
      子文件中就不需要添加 \env{document} 环境了\footnotemark。
    \end{column}
    \begin{column}{0.6\textwidth}
      \begin{codeblock}[]{主文档}
|\highlightline|\documentclass{ctexrep}
\includeonly{learnlatex,sjtuthesis}
\begin{document}
  \tableofcontents
  \include{learnlatex}
  \include{sjtuthesis}
\end{document}
      \end{codeblock}
    \end{column}
  \end{columns}
  \footnotetext{如果想强制指定子文档的主文档,可以在文件第一行输入魔术命令:\texttt{\% !TeX root = main.tex}}
\end{frame}

\section{图}
\begin{frame}[fragile]%
  \frametitle{\temporal<5>{插图}{浮动体}{插图}}
  \begin{columns}
    \begin{column}{0.6\textwidth}
      \begin{codeblock}[]{插入单图\only<4->{最佳实践}}
\documentclass{ctexart}
|\only<2>{\highlightline}|\usepackage{graphicx}
|\only<2>{\highlightline}|\graphicspath{{figs/}{pics/}}
\begin{document}
|\only<5>{\highlightline}|\begin{figure}[ht]
|\only<6>{\highlightline}|  \centering
|\only<3>{\highlightline}|  \includegraphics[width=|\only<1-3>{4cm}\only<4->{0.4\textbackslash{}textwidth}|]{sjtug}
|\only<7>{\highlightline}|  \caption{SJTUG 徽标}\label{fig:sjtug}
|\only<5>{\highlightline}|\end{figure}
\end{document}
      \end{codeblock}
    \end{column}
    \begin{column}{0.4\textwidth}
      \only<1>{
        \includepdflarge{insertimage}
      }
      \only<2>{
        为了插入外部图片,需要使用 \pkg{graphicx} 宏包。之后在文档主体便可以使用 \cmd{includegraphics} 插入图片。导言区也可以加入 \cmd{graphicspath} 指定图片文件夹\footnotemark。
      }
      \only<3>{
        \cmd{includegraphics} 命令便以相对路径的方式插入图片,如果无同名图片,那么后缀名可以省略。可以使用可选参数指定插入的图片尺寸,最佳实践是使用 \cmd{textwidth} 或 \cmd{linewidth} 的相对值指定尺寸大小,以在未来可能的布局更改中保留一定的灵活性。
      }
      \only<4>{
        也可以通过键值对的方法设置图片的其他属性。
        \begin{center}
          \footnotesize
          \begin{stampbox}
            \begin{tabular}{rl}
              \pkg{width} & 宽度 \\
              \pkg{height} & 高度 \\
              \pkg{scale} & 缩放 \\
              \pkg{angle} & 角度 \\
            \end{tabular}
          \end{stampbox}
        \end{center}
      }
      \only<5>{
        \env{figure} 为一个浮动体环境(\env{table} 也是),可以将其移动到其他位置上以减少行文中的空白。可以添加可选参数以指定如何放置浮动体,最多可以使用四种位置描述符:
        \begin{center}
          \footnotesize
          \begin{stampbox}
            \begin{tabular}{cll}
              \pkg{h} & Here & 尽可能在这里 \\
              \pkg{t} & Top & 页面顶部 \\
              \pkg{b} & Bottom & 页面底部 \\
              \pkg{p} & Page & 浮动体专页 \\
            \end{tabular}
          \end{stampbox}
        \end{center}
        还可以只使用 \pkg{float} 宏包提供的 \pkg{H} 描述符以强制置于此处。
      }
      \only<6>{
        采用 \cmd{centering} 命令而不是 \env{center} 环境来水平居中图片。这将避免多余的纵向间距。
      }
      \only<7>{
        使用 \cmd{caption} 命令输入题注,如果这个命令写在插入图片前面,题注将会在上方(中文中一般对 \env{table} 环境这么做)。后面将会看到如何对留有标记(\cmd{label})的图片进行引用。
      }
    \end{column}
  \end{columns}
  \only<2>{\footnotetext{其命令参数每个为一个以 \texttt{/} 结尾的文件夹,每个文件夹需要使用大括号包裹起来。}}
\end{frame}

\begin{frame}[fragile]
  \begin{columns}
    \begin{column}{0.6\textwidth}
      \begin{codeblock}[]{插入双图}
\documentclass{ctexart}
\usepackage{graphicx}
\graphicspath{{figs/}{pics/}}
\begin{document}
  \begin{figure}[ht]
|\only<1>{\highlightline}|    \begin{minipage}{0.48\textwidth}
      \centering
      \includegraphics[height=2cm]{sjtug}
|\only<2>{\highlightline}|      \caption{SJTUG 徽标}\label{fig:sjtug}
|\only<1>{\highlightline}|    \end{minipage}\hfill
|\only<1>{\highlightline}|    \begin{minipage}{0.48\textwidth}
      \centering
      \includegraphics[height=2cm]{sjtugt}
|\only<2>{\highlightline}|      \caption{SJTUG||文字}\label{fig:sjtugt}
|\only<1>{\highlightline}|    \end{minipage}
  \end{figure}
\end{document}
      \end{codeblock}
    \end{column}
    \begin{column}{0.4\textwidth}
      \only<1>{
        在 \env{figure} 环境里使用 \env{minipage} 小页制作列盒子,以并排两图并分别编号,需要设定强制参数以指定列宽。两个小页之间添加 \cmd{hfill} 使两个小页两端对齐。
      }
      \only<2>{
        在每个小页内部分别使用 \cmd{caption} 添加标签。
      }
      \only<3>{
        \includepdflarge{doubleimages}
      }
    \end{column}
  \end{columns}
\end{frame}

\begin{frame}[fragile]%
  \begin{columns}
    \begin{column}{0.6\textwidth}
      \begin{codeblock}[]{}
\documentclass{ctexart}
\usepackage{graphicx}
|\highlightline|\usepackage{subcaption}
\graphicspath{{figs/}{pics/}}
\begin{document}
  \begin{figure}[ht]
|\highlightline|    \begin{subfigure}{0.48\textwidth}
      \centering
      \includegraphics[height=2cm]{sjtug}
      \caption{||徽标}
|\highlightline|    \end{subfigure}\hfill
|\highlightline|    \begin{subfigure}{0.48\textwidth}
      \centering
      \includegraphics[height=2cm]{sjtugt}
      \caption{||文字}
|\highlightline|    \end{subfigure}
    \caption{SJTUG}\label{fig:sjtug}
  \end{figure}
\end{document}
      \end{codeblock}
    \end{column}
    \begin{column}{0.4\textwidth}
      \includepdflarge{subfigures}\vspace{15pt}
      \pkg{subcaption} 宏包提供了 \env{subfigure} 环境(以及 \env{subtable}),可以用于以子图的形式插入,编号会增加一级。也可以为子图添加子集引用编号。
    \end{column}
  \end{columns}
\end{frame}

\section{表}
\begin{frame}[fragile]
  \frametitle{简单表格}
  \begin{columns}
    \begin{column}{0.6\textwidth}
      \begin{codeblock}[]{}
\documentclass{ctexart}
|\only<1-2>{\highlightline}|\usepackage{|\temporal<1>{array}{\highlight{array}}{array},\temporal<2>{booktabs}{\highlight{booktabs}}{booktabs}|}
\begin{document}
\begin{table}[ht]
  \centering
  \caption{||北京冬奥会吉祥物}
|\only<1>{\highlightline}|  \begin{tabular}{lp{3cm}}
|\only<2>{\highlightline}|    \toprule
|\only<3>{\highlightline}|吉祥物 & 描述                          \\
|\only<2>{\highlightline}|    \midrule
|\only<3>{\highlightline}|冰墩墩 & 2022 年北京冬季奥运会吉祥物  \\
|\only<3>{\highlightline}|雪容融 & 2022 年北京冬季残奥会吉祥物  \\
|\only<2>{\highlightline}|    \bottomrule
|\only<1>{\highlightline}|  \end{tabular}
\end{table}
\end{document}
      \end{codeblock}
    \end{column}
    \begin{column}{0.4\textwidth}
      \only<1>{
        使用 \env{tabular} 环境绘制表格。需要添加参数(称为\textbf{表格导言})以确定每一列的对齐方式。引入 \pkg{array} 宏包来使用更多的\textbf{引导符}。
        \begin{center}
          \footnotesize
          \begin{stampbox}
            \begin{tabular}{>{\ttfamily}ll}
              \alert{l} & 向左对齐 \\
              \alert{c} & 居中对齐 \\
              \alert{r} & 向右对齐 \\
              \alert{p\{3cm\}} & 固定列宽,两端对齐 \\
              \alert{m\{3cm\}} & \texttt{p} + 垂直居中对齐 \\
              \alert{>\{\textbackslash{}bfseries\}} & 后一列单元格前加命令 \\
              \alert{*\{3\}\{l\}} & 三个左对齐列 \\
            \end{tabular}
          \end{stampbox}
        \end{center}
      }
      \only<2>{
        \pkg{booktabs} 宏包提供了标准三线表格所需要的行分割线:\cmd{toprule},\cmd{midrule},\cmd{bottomrule}。也可以使用 \cmd{cmidrule\{1-2\}} 来部分地绘制行分割线。一般不推荐在专业表格中使用纵向分割线。
      }
      \only<3>{
        每行内容使用 \textbackslash\textbackslash{} 分隔开,每行中的单元格使用 \& 分隔开。
      }
      \only<4>{
        \includepdflarge{table}
      }
    \end{column}
  \end{columns}
\end{frame}

\begin{frame}[fragile]%
  \begin{columns}
    \begin{column}{0.6\textwidth}
      \begin{codeblock}[]{表头居中}
\documentclass{ctexart}
\usepackage{array,booktabs}
\begin{document}
\begin{table}[ht]
  \centering
  \caption{||北京冬奥会吉祥物}
  \begin{tabular}{lp{3cm}}
    \toprule
|\highlightline|\multicolumn{1}{c}{||吉祥物} &
|\highlightline|\multicolumn{1}{c}{||描述} \\
    \midrule
||冰墩墩 & 2022 年北京冬季奥运会吉祥物  \\
||雪容融 & 2022 年北京冬季残奥会吉祥物  \\
    \bottomrule
  \end{tabular}
\end{table}
\end{document}
      \end{codeblock}
    \end{column}
    \begin{column}{0.4\textwidth}
      \cmd{multicolumn} 命令不仅可以用于合并同行的单元格,还可以用于临时地屏蔽表格导言对该列的对齐方式定义。这里用于居中表头。
      \begin{center}
        \begin{stampbox}
          \parbox{0.85\linewidth}{
            \ttfamily\color{blue}\textbackslash{}multicolumn\{格数\}\{对齐方式\}\{内容\}
          }
        \end{stampbox}
      \end{center}
      跨页表格考虑使用 \pkg{longtable} 宏包。带标注的表格可以考虑使用 \pkg{threeparttable} 宏包。考虑使用在线工具生成表格代码 \link{https://www.tablesgenerator.com/latex_tables}。
    \end{column}
  \end{columns}
\end{frame}

\section{数学公式}
\begin{frame}
  \frametitle{数学模式}
  \begin{alertblock}{}
  输入数学公式是 \LaTeX{} 的绝对强项,很多常见的公式服务依赖于一些轻量级渲染引擎比如 MathJax, K\kern-.3ex\raise.4ex\hbox{\footnotesize A}\kern-.3ex\TeX{}。但是它们实际上使用的是 \LaTeX{} 语法变种,也就是说并没有使用 \LaTeX{} 后端。所以不要期望有完全一致的输出。
  \end{alertblock}
  
  为了更好的获得数学公式输入支持,请使用 \hologo{AmS}math 宏包。数学模式分为两种:
  \begin{description}
    \item[行内模式] 一般通过一对美元符号(\$$\cdots$\$)标记,可以使用编辑器内建的符号表输入数学符号,也可以使用在线工具手写识别 \link{https://detexify.kirelabs.org/classify.html}。
    \item[行间模式] 一般通过 \env{equation} 环境\footnote{这是有编号环境,其加星号的变种 \env{equation*} 用于生成无编号环境。}输入。如果需要使用多行公式,请使用 \env{align} 环境,并按照类似表格输入的方式,使用 \& 对齐符号,\textbackslash\textbackslash{} 换行。如果不想手动居中,可以考虑多行自动居中的 \env{gather} 和单个大型公式首尾两端对齐 \env{multline}。
  \end{description}
\end{frame}

\begin{frame}
  \frametitle{数学命令展示}
  \begin{columns}
    \begin{column}{0.33\textwidth}
      \begin{exampleblock}{}
        $PV=nRT$
      \end{exampleblock}
      \begin{exampleblock}{}
        $\sum_{i=1}^ki^2=\frac{n(n+1)(2n+1)}{6}$
      \end{exampleblock}
      \begin{exampleblock}{}
        $T(n) = aT\left(\left\lceil\frac{n}{b}\right\rceil\right) + \mathcal{O}(n^d)$
      \end{exampleblock}
      \begin{exampleblock}{}
        $\frac{x_{1}+x_{2}+x_{3}}{3}\geq \sqrt[3]{x_{1}x_{2}x_{3}}$
      \end{exampleblock}
      \begin{exampleblock}{}
        $n=(\underbrace{1\cdots 1}_{k\text{ of 1's}})_2=2^{k+1}-1$
      \end{exampleblock}
      \begin{exampleblock}{}
        $\nabla f (P)= \left.\left(\frac{\partial f}{\partial x},\frac{\partial f}{\partial y},\frac{\partial f}{\partial z}\right)\right|_{P}$
      \end{exampleblock}
    \end{column}
    \begin{column}{0.67\textwidth}
      \begin{exampleblock}{}
        \begin{equation*}
          \int_{a}^b f(x)\,\mathrm{d}x=\lim_{|P|\rightarrow 0}\sum_{i=1}^n f(\xi_i)\Delta x_i
        \end{equation*}
      \end{exampleblock}
      \begin{exampleblock}{}
        \begin{equation}
          T(n) = \begin{cases}
            \mathcal{O}(n^d),&\textrm{if } d>\log_b a, \\
            \mathcal{O}(n^d\log n), &\textrm{if } d=\log_b a,\\
            \mathcal{O}(n^{\log_b a}), &\textrm{if } d<\log_b a.
          \end{cases}
        \end{equation}
      \end{exampleblock}
      \begin{exampleblock}{}
        \begin{align}
          Q^{T}A&=R \\
          \begin{pmatrix}
            q_1^T \\ q_2^T \\ q_3^T
          \end{pmatrix}
          \begin{pmatrix}
            a_1 & a_2 & a_3
          \end{pmatrix}
          &=R
        \end{align}
      \end{exampleblock}
    \end{column}
  \end{columns}
\end{frame}

%更深入地讲解 mathtools, unicode-math, siunix

\section{引用}
\begin{frame}[fragile]
  \frametitle{交叉引用}
  \only<1>{
    正如之前所提到的,\LaTeX{} 中使用 \cmd{label} 标记,然后可以使用 \cmd{ref} 来引用这个标记。 \cmd{label} 可以放在使用计数器的对象之后。
  }
  \only<2>{
    为了使得对公式编号的引用带有括号,推荐使用 \hologo{AmS}math 宏包中的 \cmd{eqref} 命令。对于多行公式环境,每一个换行符前都可以添加一个 \cmd{label} 用于引用该行公式。
  }
  \begin{columns}
    \begin{column}{0.5\textwidth}
      \begin{codeblock}[]{图}
\begin{figure}
|\only<1>{\highlightline}|  \caption{||示例}\label{fig:example}
\end{figure}
      \end{codeblock}
      \begin{codeblock}[]{表}
\begin{table}
|\only<1>{\highlightline}|  \caption{||示例}\label{tab:example}
\end{table}
      \end{codeblock}
    \end{column}
    \begin{column}{0.5\textwidth}
\begin{codeblock}[]{目次}
|\only<1>{\highlightline}|\section{||示例}\label{sec:example}
\end{codeblock}

\begin{codeblock}[]{公式}
\begin{equation}
  a = b + c
|\only<1>{\highlightline}|\label{eq:example}
\end{equation}
|\only<2>{\highlightline}|如公式 \eqref{eq:example} 所示,
\end{codeblock}
    \end{column}
  \end{columns}
\end{frame}

\begin{frame}[fragile]
  \frametitle{文献引用}
  \LaTeX{} 管理参考文献可以采用专用数据库文件 \texttt{.bib},很多的文献管理文件比如 EndNote \link{https://lic.sjtu.edu.cn/Default/softshow/tag/MDAwMDAwMDAwMLGImKE}, Zotero \link{https://www.zotero.org/}, JabRef \link{https://www.jabref.org/} 都可以直接导出这种格式的文件用于 \LaTeX{} 论文中的引用。一般不需要手写数据库文件,你只需要注意每一个文献会在数据库中有一个主键,这个类似于上文的 \cmd{label} 标记,只是要引用数据库中的文献需要使用 \cmd{cite} 命令。
  
  \begin{codeblock}[]{ref.bib}
|\highlightline|@phdthesis{devoftech,|\hfill\alert{\% 类型为博士论文,主键为\texttt{devoftech}}|
  title={||新时期我国信息技术产业的发展},
  author={||江泽民},
  year={2008}
}
  \end{codeblock}
\end{frame}

\begin{frame}
  \frametitle{文献引用}
  而让 \LaTeX{} 处理 \texttt{.bib} 数据库文件与引用有两种工作流。你可能需要查清楚如何在编辑器中设置对应的工作流,或者采用后面所提到的高级编译方式 \texttt{latexmk}。
  \begin{columns}
    \begin{column}{0.5\textwidth}
      \begin{block}{\hologo{BibTeX} + \pkg{gbt7714}}
        一般期刊提交使用这种方法,\pkg{natbib} 宏包将提供命令 \cmd{citet}(文本引用) 和 \cmd{citep}(括号引用)。中文引用可以直接使用 \pkg{gbt7714} 宏包,但是角模式和正文模式不能混用。
      \end{block}
    \end{column}
    \begin{column}{0.5\textwidth}
      \begin{block}{\hologo{biber} + \pkg{biblatex}}
        这是更容易自定义的方法,与 \hologo{BibTeX} 的运作方式稍有不同。\pkg{biblatex} 提供了更加智能的引用命令。而中文引用可以使用 \pkg{biblatex} 宏包的样式 \pkg{gb7714-2015},使用该样式需要使用 \hologo{XeLaTeX} 编译。
      \end{block}
    \end{column}
  \end{columns}
\end{frame}

\begin{frame}[fragile]
  \frametitle{文献引用}
  \begin{columns}
    \begin{column}{0.5\textwidth}
      \begin{codeblock}[]{\hologo{BibTeX} + \pkg{gbt7714}}
\documentclass{ctexart}
\usepackage{gbt7714}
\bibliographystyle{gbt7714-numerial}
% \citestyle{numbers}  % 正文模式
\begin{document}
  ||他指出了...\cite{devoftech}
  \bibliography{ref}
\end{document}
      \end{codeblock}
    \end{column}
    \begin{column}{0.5\textwidth}
      \begin{codeblock}[]{\hologo{biber} + \pkg{biblatex}}
\documentclass{ctexart}
\usepackage[backend=biber,style=gb7714-2015]{biblatex}
\addbibresource{ref.bib}
\begin{document}
  ||他在文献 \parencite{devoftech}
  ||指出了...\cite{devoftech}
  \printbibliography
\end{document}
      \end{codeblock}
    \end{column}
  \end{columns}
\end{frame}

\begin{frame}
  \frametitle{文献引用}
  \begin{columns}
    \begin{column}{0.5\textwidth}
      \includepdflarge{bibtex}
    \end{column}
    \begin{column}{0.5\textwidth}
      \includepdflarge{biblatex}
    \end{column}
  \end{columns}
\end{frame}

} % End of customized shaded number logo

  % !TeX root = ..\..\latex-talk.tex

\part{SJTUThesis}

\begin{frame}
  \frametitle{简介}
  \begin{columns}
    \begin{column}{0.6\textwidth}
      \begin{itemize}
        \item 最早由韦建文于 2009 年 11 月发布 0.1a 版,2018 年起由 SJTUG 接手维护
        \item 最新版:\SJTUThesisVersion{} (\SJTUThesisDate)
        \item 支持本科、硕士、博士学位论文以及课程论文的排版
      \end{itemize}
    \end{column}
    \begin{column}{0.4\textwidth}
      \begin{exampleblock}{}
        \begin{minipage}[c]{1cm}
          \includegraphics[width=0.8cm]{\getcontribpath{sjtug}{vi/sjtug}}
        \end{minipage}
        \begin{minipage}[c]{2cm}
          \href{https://github.com/sjtug}{sjtug}/\href{https://github.com/sjtug/SJTUThesis}{SJTUThesis}
        \end{minipage}
      \end{exampleblock}
      \vspace{-8pt}
      \begin{block}{}
        \scriptsize
        上海交通大学 \hologo{XeLaTeX} 学位论文及课程论文模板 | Shanghai Jiao Tong University \hologo{XeLaTeX} Thesis Template
      \end{block}
      \vspace{-8pt}
      \begin{alertblock}{}
        \scriptsize
        \begin{tabular}{cl}
          \faStar & 2.4k \\
          \faEye & 55 \\
          \faCodeBranch & 701 \\
        \end{tabular}
      \end{alertblock}
    \end{column}
  \end{columns}
\end{frame}

\begin{frame}
  \frametitle{下载与编译}
  \alert{下载} 推荐安装 Git \link{https://git-scm.com/} 后,克隆 SJTUG 镜像仓库
  \begin{exampleblock}{\faGit*}
    \ttfamily\small
    git clone https://mirror.sjtu.edu.cn/git/SJTUThesis.git/
  \end{exampleblock}

  \alert{编译} 推荐使用 \pkg{latexmk} 编译\footnote{\hologo{MiKTeX} 用户需要手动安装 Perl 解释器 \link{https://www.perl.org/get.html} 才能使用 \pkg{latexmk}。},在不能够利用自带的 \texttt{.latexmkrc} 配置文件的情况下,需要查清楚在对应的编辑器中如何使用 \hologo{XeLaTeX} + \hologo{biber} 编译 \link{https://github.com/sjtug/SJTUThesis/blob/master/README.md}。
  \begin{exampleblock}{\faTerminal}
    \ttfamily\small
    latexmk -xelatex main
  \end{exampleblock}

  Overleaf 用户可以下载压缩包后,上传并采用 \hologo{XeLaTeX} 编译方式。
\end{frame}

\begin{frame}
  \frametitle{手动编译}
  \alert{第一次编译失败} 如果没有办法通过通常方式编译成功,请尝试使用文件夹内附带 \faLinux{}\,\faApple{} \texttt{Makefile} 和 \faWindows{} \texttt{Compile.bat} 进行编译。

  \alert{统计字数} 编写过程中也可以使用对应的命令调用 \TeX{}count 来统计正文字数。
  \begin{columns}
    \begin{column}{0.38\textwidth}
      \begin{exampleblock}{\faLinux{}\,\faApple}
        \ttfamily
        make all\\
        make clean\\
        make cleanall\\
        make wordcount
      \end{exampleblock}
    \end{column}
    \begin{column}{0.38\textwidth}
      \begin{exampleblock}{\faWindows}
        \ttfamily
        ./Compile.bat thesis\\
        ./Compile.bat clean\\
        ./Compile.bat cleanall\\
        ./Compile.bat wordcount
      \end{exampleblock}
    \end{column}
    \begin{column}{0.24\textwidth}
      \begin{block}{\faInfo}
        \ttfamily
        编译论文\\
        清理中间文件\\
        $\hookrightarrow +$删除论文\\
        统计字数
      \end{block}
    \end{column}
  \end{columns}
\end{frame}

\begin{frame}[label=compile]
  \frametitle{编译问题排查}
  \begin{columns}
    \begin{column}{0.33\textwidth}
      \begin{alertblock}{无法使用 \texttt{latexmk}\thesisissue{578}}
        \hologo{MiKTeX} 需要安装 Perl 解释器。
      \end{alertblock}  
      \begin{alertblock}{C\TeX{} 套装无法编译\thesisissue{446}}
        使用最新 \TeX{} 发行版。
      \end{alertblock}
      \begin{alertblock}{\hologo{pdfLaTeX} 无法编译\thesisissue{444}}
        请使用 \texttt{latexmk},或更改编辑器设置以 \hologo{XeLaTeX} 编译。
      \end{alertblock}
    \end{column}
    \begin{column}{0.33\textwidth}
      \begin{alertblock}{缺少字体\thesisissue{564} \thesisdiscuss{598}}
        更换字体集,或者安装对应字体。
      \end{alertblock}
      \begin{alertblock}{缺少汉字\thesisissue{533} \thesisdiscuss{617}}
        去除使用 fandol 字体集的设定。或者是安装字体后,改用 \texttt{fontset=adobe} 或 \texttt{fontset=founder}。
      \end{alertblock}
    \end{column}
    \begin{column}{0.33\textwidth}
      \begin{block}{\faInfoCircle{} README}
        不同编辑器的设置请首先参阅 README \link{https://github.com/sjtug/SJTUThesis/blob/master/README.md} 文档。
      \end{block}
      \begin{block}{\faBookOpen{} Wiki}
        其他编译问题推荐查阅 Wiki \link{https://github.com/sjtug/SJTUThesis/wiki} 的使用说明部分。
      \end{block}
    \end{column}
  \end{columns}
\end{frame}

\begin{frame}[fragile, label=covers]
  \begin{codeblock}[firstnumber=3]{main.tex}
|\alert{\% 载入 SJTUThesis 模版}|
\documentclass[|\only<1>{\highlight{type}}\only<2>{type}|=|\only<1>{bachelor}\only<2>{\highlight{bachelor}}|]{sjtuthesis}
  \end{codeblock}
  \begin{figure}
    \parbox{0.9\textwidth}{
      \begin{subfigure}{0.20\textwidth}
        \framebox{\includegraphics[width=\linewidth]{support/thesis/bachelor}}
        \caption{\only<1>{学士}\only<2>{\texttt{bachelor}}}
      \end{subfigure}\hfill
      \begin{subfigure}{0.20\textwidth}
        \framebox{\includegraphics[width=\linewidth]{support/thesis/master}}
        \caption{\only<1>{硕士}\only<2>{\texttt{master}}}
      \end{subfigure}\hfill
      \begin{subfigure}{0.20\textwidth}
        \framebox{\includegraphics[width=\linewidth]{support/thesis/doctor}}
        \caption{\only<1>{博士}\only<2>{\texttt{doctor}}}
      \end{subfigure}\hfill
      \begin{subfigure}{0.20\textwidth}
        \framebox{\includegraphics[width=\linewidth]{support/thesis/course}}
        \caption{\only<1>{课程}\only<2>{\texttt{course}}}
      \end{subfigure}
      \caption{论文类型示例\only<2>{ \texttt{type}}}
    }
  \end{figure}
\end{frame}

\begin{frame}[fragile]
  \frametitle{文档类选项}
  % \framesubtitle{\textbackslash{}documentclass\{sjtuthesis\}}
  \begin{columns}
    \begin{column}{0.45\textwidth}
      \includegraphics[page=10]{thesisdir}
    \end{column}
    \begin{column}{0.55\textwidth}
      \begin{table}[H]
        \caption{文档类选项}
        \footnotesize
        \begin{tabular}{>{\ttfamily}rll}
          \toprule
          选项 & 含义 & 相关 \\
          \midrule
          type= & 指定论文类型 & 第 \ref{covers} 页\\
          fontset= & 指定字体 & 第 \ref{compile} 页\\
          \midrule
          review & 开启盲审模式 & \thesisissue{195} \thesisissue{686} \\
          twoside & 双页模式 & \thesisissue{554} \\
          oneside & 单页模式 & \thesisissue{694} \\
          openright & 章从奇数页开始 & \thesisdiscuss{724} \\
          openany & 章从任意页开始 & \thesisissue{446} \\
          \bottomrule
        \end{tabular}
      \end{table}
    \end{column}
  \end{columns}
\end{frame}

\begin{frame}[fragile]
  \frametitle{基本配置}
  \framesubtitle{\textbackslash{}input\{setup\}}
  \begin{columns}
    \begin{column}{0.45\textwidth}
      \includegraphics[page=9]{thesisdir}
    \end{column}
    \begin{column}{0.55\textwidth}
      \begin{codeblock}[firstnumber=12]{main.tex}
|\highlightline<1>|% 论文基本配置,加载宏包等全局配置
|\highlightline<1>|\input{setup}

\begin{document}

%TC:ignore

|\highlightline<2>|% 标题页
|\highlightline<2>|\maketitle
      \end{codeblock}
      \visible<2>{
        \cmd{sjtusetup} 中的 \pkg{info} 将会修改封面的信息设置(见第 \ref{covers} 页)。
      }
    \end{column}
  \end{columns}
\end{frame}

\begin{frame}[fragile]
  \frametitle{基本配置}
  \framesubtitle{\textbackslash{}sjtusetup}
  \begin{columns}
    \begin{column}{0.45\textwidth}
      \includegraphics[page=1]{thesisdir}
    \end{column}
    \begin{column}{0.55\textwidth}
      \begin{codeblock}[firstnumber=3]{setup.tex}
\sjtusetup{
  info = {
    title    = {||上海交通大学学位论文 \LaTeX{} 模板示例文档},
    title*   = {A Sample for \LaTeX-based SJTU Thesis Template},
    author   = {||某\quad{}某},
    author* = {Mo Mo},
  },
  style = { header-logo-color = red, 
  },
  name = {
    publications = {||攻读学位期间完成的论文},
  },
}
      \end{codeblock}
    \end{column}
  \end{columns}
\end{frame}

\begin{frame}
  \frametitle{基本配置}
  \framesubtitle{\textbackslash{}sjtusetup}
  \begin{columns}
    \begin{column}{0.45\textwidth}
      \includegraphics[page=1]{thesisdir}
    \end{column}
    \begin{column}{0.55\textwidth}
      \begin{table}[H]
        \centering
        \caption{info 域}
        \footnotesize
        \begin{tabular}{lll} \toprule
          命令作用 & 中文对应选项 & 英文对应选项 \\ \midrule
          论文标题 & \texttt{title} & \texttt{title*} \\
          关键字列表 & \texttt{keywords} & \texttt{keywords*} \\
          作者姓名&  \texttt{author} &\texttt{author*}\\
          申请学位名称 & \texttt{degree}&\texttt{degree*}\\
          院系名称 & \texttt{department} & \texttt{department*}\\
          专业名称 & \texttt{major} & \texttt{major*}\\
          导师 & \texttt{supervisor} & \texttt{supervisor*}\\
          副导师 & \texttt{assisupervisor} & \texttt{assisupervisor*}\\
          日期 & \multicolumn{2}{c}{\texttt{date}}\\
          学号 & \multicolumn{2}{c}{\texttt{id}}\\ \bottomrule
          \end{tabular}
      \end{table}
    \end{column}
  \end{columns}
\end{frame}

\begin{frame}[fragile]
  \frametitle{版权页}
  \framesubtitle{\textbackslash{}copyrightpage}
  \begin{columns}
    \begin{column}{0.45\textwidth}
      \only<1>{
        \includegraphics[page=9]{thesisdir}
      }
      \only<2>{
        \includegraphics[page=2]{thesisdir}
      }
      \only<3>{
        \begin{figure}[H]
          \framebox{\includegraphics[page=2,width=0.4\linewidth]{bachelor}}
          \caption{版权页}
        \end{figure}
      }
    \end{column}
    \begin{column}{0.55\textwidth}
      \begin{codeblock}[firstnumber=22]{main.tex}
|\highlightline<1>|% 原创性声明及使用授权书
|\highlightline<1>|\copyrightpage
|\highlightline<2>|% 插入外置原创性声明及使用授权书
|\highlightline<2>|% \copyrightpage[scans/sample-copyright-old.pdf]
      \end{codeblock}
      \only<1>{
        \cmd{copyrightpages} 可以用于插入版权页。
      }
      \only<2>{
        \cmd{copyrightpages} 也接受一个可选参数,用于直接使用扫描件。
      }
    \end{column}
  \end{columns}
\end{frame}

\begin{frame}[fragile]
  \frametitle{前置部分}
  \framesubtitle{\textbackslash{}frontmatter}
  \begin{columns}
    \begin{column}{0.45\textwidth}
      \only<1>{
        \includegraphics[page=9]{thesisdir}
      }
      \only<2>{
        \includegraphics[page=3]{thesisdir}
      }
      \only<3>{
        \begin{figure}[H]
          \begin{subfigure}{0.45\textwidth}
            \framebox{\includegraphics[page=3,width=\linewidth]{bachelor}}
            \caption{中文}
          \end{subfigure}\hfill
          \begin{subfigure}{0.45\textwidth}
            \framebox{\includegraphics[page=4,width=\linewidth]{bachelor}}
            \caption{英文}
          \end{subfigure}
          \caption{摘要}
        \end{figure}
      }
      \only<4>{
        \begin{figure}[H]
          \begin{subfigure}{0.30\linewidth}
            \centering
            \framebox{\includegraphics[page=5,width=0.6\linewidth]{bachelor}}
            \caption{目录}
          \end{subfigure}
          \begin{subfigure}{0.30\linewidth}
            \centering
            \framebox{\includegraphics[page=6,width=0.6\linewidth]{bachelor}}
            \caption{插图}
          \end{subfigure}

          \begin{subfigure}{0.30\linewidth}
            \centering
            \framebox{\includegraphics[page=7,width=0.6\linewidth]{bachelor}}
            \caption{表格}
          \end{subfigure}
          \begin{subfigure}{0.30\linewidth}
            \centering
            \framebox{\includegraphics[page=8,width=0.6\linewidth]{bachelor}}
            \caption{算法}
          \end{subfigure}
          \caption{索引}
        \end{figure}
      }
      \only<5>{
        \includegraphics[page=4]{thesisdir}
      }
      \only<6>{
        \begin{figure}[H]
          \framebox{\includegraphics[page=9,width=0.5\linewidth]{bachelor}}
          \caption{符号对照表}
        \end{figure}
      }
    \end{column}
    \begin{column}{0.55\textwidth}
      \begin{codeblock}[firstnumber=30]{main.tex}
|\highlightline<2-3>|% 摘要
|\highlightline<2-3>|\input{contents/abstract}

|\highlightline<4>|% 目录
|\highlightline<4>|\tableofcontents
|\highlightline<4>|% 插图索引
|\highlightline<4>|\listoffigures*
|\highlightline<4>|% 表格索引
|\highlightline<4>|\listoftables*
|\highlightline<4>|% 算法索引
|\highlightline<4>|\listofalgorithms*

|\highlightline<5-6>|% 符号对照表
|\highlightline<5-6>|\input{contents/nomenclature}
      \end{codeblock}
    \end{column}
  \end{columns}
\end{frame}

\begin{frame}[fragile]
  \frametitle{主体部分}
  \framesubtitle{\textbackslash{}mainmatter}
  \begin{columns}
    \begin{column}{0.45\textwidth}
      \only<1>{
        \includegraphics[page=5]{thesisdir}
      }
      \only<2>{
        \begin{figure}[H]
          \begin{subfigure}{0.30\linewidth}
            \centering
            \framebox{\includegraphics[page=11,width=0.6\linewidth]{bachelor}}
            \caption{简介}
          \end{subfigure}
          \begin{subfigure}{0.30\linewidth}
            \centering
            \framebox{\includegraphics[page=13,width=0.6\linewidth]{bachelor}}
            \caption{数学}
          \end{subfigure}

          \begin{subfigure}{0.30\linewidth}
            \centering
            \framebox{\includegraphics[page=16,width=0.6\linewidth]{bachelor}}
            \caption{浮动体}
          \end{subfigure}
          \begin{subfigure}{0.30\linewidth}
            \centering
            \framebox{\includegraphics[page=22,width=0.6\linewidth]{bachelor}}
            \caption{总结}
          \end{subfigure}
          \caption{主体部分}
        \end{figure}
      }
    \end{column}
    \begin{column}{0.55\textwidth}
      \begin{codeblock}[firstnumber=47]{main.tex}
|\highlightline|% 正文内容
|\highlightline|\input{contents/intro}
|\highlightline|\input{contents/math_and_citations}
|\highlightline|\input{contents/floats}
|\highlightline|\input{contents/summary}

%TC:ignore

% 参考文献
\printbibliography[heading=bibintoc]
      \end{codeblock}
    \end{column}
  \end{columns}
\end{frame}

\begin{frame}
  \frametitle{数学}
  \begin{itemize}
    \item 公式示例:\nolinkurl{contents/math_and_citations.tex}
    \item \SJTUThesis{} 定义了常用的数学环境(需要手工引入 \texttt{ntheorem} 宏包):
      \begin{table}[h]
        \centering
        \footnotesize
        \begin{tabular}{*{7}{l}}\toprule
          assumption  & axiom   & conjecture & corollary    & definition  & example   & exercise  \\
          假设        & 公理    & 猜想       & 推论         & 定义        & 例        & 练习      \\\midrule
          lemma       & problem & proof      & proposition  & remark      & solution  & theorem   \\
          引理        & 问题    & 证明       & 命题         & 注          & 解        & 定理      \\\bottomrule
        \end{tabular}
      \end{table}
      \item \SJTUThesis{} 可以通过 \texttt{unimath} 选项使用 \pkg{unicode-math} 进行数学输入,注意与传统方式的区别。\thesisissue{555}
  \end{itemize}
\end{frame}

\begin{frame}[fragile]
  \frametitle{参考文献}
  \begin{columns}
    \begin{column}{0.45\textwidth}
      \includegraphics[page=6]{thesisdir}
    \end{column}
    \begin{column}{0.55\textwidth}
      \begin{codeblock}[firstnumber=111,numbersep=2pt]{setup.tex}
% 使用 BibLaTeX 处理参考文献
%   biblatex-gb7714-2015 常用选项
%     gbnamefmt=lowercase     姓名大小写由输入信息确定
%     gbpub=false             禁用出版信息缺失处理
\usepackage[backend=biber,style=gb7714-2015]{biblatex}
% 文献表字体
% \renewcommand{\bibfont}{\zihao{-5}}
% 文献表条目间的间距
\setlength{\bibitemsep}{0pt}
|\highlightline|% 导入参考文献数据库
|\highlightline|\addbibresource{bibdata/thesis.bib}
      \end{codeblock}
    \end{column}
  \end{columns}
\end{frame}

\begin{frame}[fragile]
  \frametitle{附录}
  \framesubtitle{\textbackslash{}appendix}
  \begin{columns}
    \begin{column}{0.45\textwidth}
      \only<1>{
        \includegraphics[page=7]{thesisdir}
      }
      \only<2>{
        \begin{figure}[H]
          \begin{subfigure}{0.45\linewidth}
            \framebox{\includegraphics[width=\linewidth,page=24]{bachelor}}
            \caption{}
          \end{subfigure}\hfill
          \begin{subfigure}{0.45\textwidth}
            \framebox{\includegraphics[width=\linewidth,page=25]{bachelor}}
            \caption{}
          \end{subfigure}
          \caption{附录}
        \end{figure}
      }
    \end{column}
    \begin{column}{0.55\textwidth}
      \begin{codeblock}[firstnumber=61]{main.tex}
% 附录中图表不加入索引
\captionsetup{list=no}

% 附录内容
|\highlightline|\input{contents/app_maxwell_equations}
|\highlightline|\input{contents/app_flow_chart}
      \end{codeblock}
    \end{column}
  \end{columns}
\end{frame}

\begin{frame}[fragile]
  \frametitle{结尾部分}
  \framesubtitle{\textbackslash{}backmatter}
  \begin{columns}
    \begin{column}{0.45\textwidth}
      \only<1>{
        \includegraphics[page=8]{thesisdir}
      }
      \only<2>{
        \begin{figure}[H]
          \begin{subfigure}{0.30\linewidth}
            \centering
            \framebox{\includegraphics[page=26,width=0.6\linewidth]{bachelor}}
            \caption{致谢}
          \end{subfigure}
          \begin{subfigure}{0.30\linewidth}
            \centering
            \framebox{\includegraphics[page=27,width=0.6\linewidth]{bachelor}}
            \caption{成就}
          \end{subfigure}

          \begin{subfigure}{0.30\linewidth}
            \centering
            \framebox{\includegraphics[page=28,width=0.6\linewidth]{bachelor}}
            \caption{简历}
          \end{subfigure}
          \begin{subfigure}{0.30\linewidth}
            \centering
            \framebox{\includegraphics[page=29,width=0.6\linewidth]{bachelor}}
            \caption{大摘要*}
          \end{subfigure}
          \caption{结尾部分}
        \end{figure}
      }
    \end{column}
    \begin{column}{0.55\textwidth}
      \begin{codeblock}[firstnumber=76]{main.tex}
% 致谢
\input{contents/acknowledgements}

% 发表论文及科研成果
% 盲审论文中,发表论文及科研成果等仅以第几作者注明即可,不要出现作者或他人姓名
\input{contents/achievements}

% 简历
\input{contents/resume}

% 学士学位论文要求在最后有一个大摘要,单独编页码
\input{contents/digest}
      \end{codeblock}
    \end{column}
  \end{columns}
\end{frame}

\begin{frame}
  \frametitle{还有其他问题?}
  \begin{columns}
    \begin{column}{0.75\textwidth}
    \begin{itemize}
      \item[{\faComment*[regular]}] 日常模板或 \LaTeX{} 使用问题可以前往 Discussions \link{https://github.com/sjtug/SJTUThesis/discussions} 提问
      
      (解决后别忘了 \textcolor{green}{\faCheckCircle{} Mark as answer}
      \item[{\faDotCircle[regular]}] 如果是 \textsc{SJTUThesis} 项目本身的 bug 和 feature request
      
      可以通过 Issues \link{https://github.com/sjtug/SJTUThesis/issues} 反馈。
      \item[{\faCodeBranch}] 如果你有好点子,可以贡献代码
     
      向 \textsc{SJTU\TeX{}}(v1) \link{https://github.com/sjtug/SJTUTeX/tree/v1} 存储库发 PR,\par
      而后把解包结果同步到 \textsc{SJTUThesis}。
  
      \item[{\faTag}] 如果你对正在基于 \LaTeX3 开发的新版感兴趣,\par
      也欢迎向 \textsc{SJTU\TeX{}}(v2) \link{https://github.com/sjtug/SJTUTeX/tree/v2} 发 PR。
  
      \item[{\faQq}] 也欢迎在 QQ 群即时讨论。
    \end{itemize}
    \end{column}
    \begin{column}{0.25\textwidth}
      \includegraphics[height=0.7\textheight]{qq.jpg}
    \end{column}
  \end{columns}
\end{frame}
\end{document}
      \end{codeblock}
    \end{column}
  \end{columns}
  \footnotetext{如果想强制指定子文档的主文档,可以在文件第一行输入魔术命令:\texttt{\% !TeX root = main.tex}}
\end{frame}

\section{图}
\begin{frame}[fragile]%
  \frametitle{\temporal<5>{插图}{浮动体}{插图}}
  \begin{columns}
    \begin{column}{0.6\textwidth}
      \begin{codeblock}[]{插入单图\only<4->{最佳实践}}
\documentclass{ctexart}
|\only<2>{\highlightline}|\usepackage{graphicx}
|\only<2>{\highlightline}|\graphicspath{{figs/}{pics/}}
\begin{document}
|\only<5>{\highlightline}|\begin{figure}[ht]
|\only<6>{\highlightline}|  \centering
|\only<3>{\highlightline}|  \includegraphics[width=|\only<1-3>{4cm}\only<4->{0.4\textbackslash{}textwidth}|]{sjtug}
|\only<7>{\highlightline}|  \caption{SJTUG 徽标}\label{fig:sjtug}
|\only<5>{\highlightline}|\end{figure}
\end{document}
      \end{codeblock}
    \end{column}
    \begin{column}{0.4\textwidth}
      \only<1>{
        \includepdflarge{insertimage}
      }
      \only<2>{
        为了插入外部图片,需要使用 \pkg{graphicx} 宏包。之后在文档主体便可以使用 \cmd{includegraphics} 插入图片。导言区也可以加入 \cmd{graphicspath} 指定图片文件夹\footnotemark。
      }
      \only<3>{
        \cmd{includegraphics} 命令便以相对路径的方式插入图片,如果无同名图片,那么后缀名可以省略。可以使用可选参数指定插入的图片尺寸,最佳实践是使用 \cmd{textwidth} 或 \cmd{linewidth} 的相对值指定尺寸大小,以在未来可能的布局更改中保留一定的灵活性。
      }
      \only<4>{
        也可以通过键值对的方法设置图片的其他属性。
        \begin{center}
          \footnotesize
          \begin{stampbox}
            \begin{tabular}{rl}
              \pkg{width} & 宽度 \\
              \pkg{height} & 高度 \\
              \pkg{scale} & 缩放 \\
              \pkg{angle} & 角度 \\
            \end{tabular}
          \end{stampbox}
        \end{center}
      }
      \only<5>{
        \env{figure} 为一个浮动体环境(\env{table} 也是),可以将其移动到其他位置上以减少行文中的空白。可以添加可选参数以指定如何放置浮动体,最多可以使用四种位置描述符:
        \begin{center}
          \footnotesize
          \begin{stampbox}
            \begin{tabular}{cll}
              \pkg{h} & Here & 尽可能在这里 \\
              \pkg{t} & Top & 页面顶部 \\
              \pkg{b} & Bottom & 页面底部 \\
              \pkg{p} & Page & 浮动体专页 \\
            \end{tabular}
          \end{stampbox}
        \end{center}
        还可以只使用 \pkg{float} 宏包提供的 \pkg{H} 描述符以强制置于此处。
      }
      \only<6>{
        采用 \cmd{centering} 命令而不是 \env{center} 环境来水平居中图片。这将避免多余的纵向间距。
      }
      \only<7>{
        使用 \cmd{caption} 命令输入题注,如果这个命令写在插入图片前面,题注将会在上方(中文中一般对 \env{table} 环境这么做)。后面将会看到如何对留有标记(\cmd{label})的图片进行引用。
      }
    \end{column}
  \end{columns}
  \only<2>{\footnotetext{其命令参数每个为一个以 \texttt{/} 结尾的文件夹,每个文件夹需要使用大括号包裹起来。}}
\end{frame}

\begin{frame}[fragile]
  \begin{columns}
    \begin{column}{0.6\textwidth}
      \begin{codeblock}[]{插入双图}
\documentclass{ctexart}
\usepackage{graphicx}
\graphicspath{{figs/}{pics/}}
\begin{document}
  \begin{figure}[ht]
|\only<1>{\highlightline}|    \begin{minipage}{0.48\textwidth}
      \centering
      \includegraphics[height=2cm]{sjtug}
|\only<2>{\highlightline}|      \caption{SJTUG 徽标}\label{fig:sjtug}
|\only<1>{\highlightline}|    \end{minipage}\hfill
|\only<1>{\highlightline}|    \begin{minipage}{0.48\textwidth}
      \centering
      \includegraphics[height=2cm]{sjtugt}
|\only<2>{\highlightline}|      \caption{SJTUG||文字}\label{fig:sjtugt}
|\only<1>{\highlightline}|    \end{minipage}
  \end{figure}
\end{document}
      \end{codeblock}
    \end{column}
    \begin{column}{0.4\textwidth}
      \only<1>{
        在 \env{figure} 环境里使用 \env{minipage} 小页制作列盒子,以并排两图并分别编号,需要设定强制参数以指定列宽。两个小页之间添加 \cmd{hfill} 使两个小页两端对齐。
      }
      \only<2>{
        在每个小页内部分别使用 \cmd{caption} 添加标签。
      }
      \only<3>{
        \includepdflarge{doubleimages}
      }
    \end{column}
  \end{columns}
\end{frame}

\begin{frame}[fragile]%
  \begin{columns}
    \begin{column}{0.6\textwidth}
      \begin{codeblock}[]{}
\documentclass{ctexart}
\usepackage{graphicx}
|\highlightline|\usepackage{subcaption}
\graphicspath{{figs/}{pics/}}
\begin{document}
  \begin{figure}[ht]
|\highlightline|    \begin{subfigure}{0.48\textwidth}
      \centering
      \includegraphics[height=2cm]{sjtug}
      \caption{||徽标}
|\highlightline|    \end{subfigure}\hfill
|\highlightline|    \begin{subfigure}{0.48\textwidth}
      \centering
      \includegraphics[height=2cm]{sjtugt}
      \caption{||文字}
|\highlightline|    \end{subfigure}
    \caption{SJTUG}\label{fig:sjtug}
  \end{figure}
\end{document}
      \end{codeblock}
    \end{column}
    \begin{column}{0.4\textwidth}
      \includepdflarge{subfigures}\vspace{15pt}
      \pkg{subcaption} 宏包提供了 \env{subfigure} 环境(以及 \env{subtable}),可以用于以子图的形式插入,编号会增加一级。也可以为子图添加子集引用编号。
    \end{column}
  \end{columns}
\end{frame}

\section{表}
\begin{frame}[fragile]
  \frametitle{简单表格}
  \begin{columns}
    \begin{column}{0.6\textwidth}
      \begin{codeblock}[]{}
\documentclass{ctexart}
|\only<1-2>{\highlightline}|\usepackage{|\temporal<1>{array}{\highlight{array}}{array},\temporal<2>{booktabs}{\highlight{booktabs}}{booktabs}|}
\begin{document}
\begin{table}[ht]
  \centering
  \caption{||北京冬奥会吉祥物}
|\only<1>{\highlightline}|  \begin{tabular}{lp{3cm}}
|\only<2>{\highlightline}|    \toprule
|\only<3>{\highlightline}|吉祥物 & 描述                          \\
|\only<2>{\highlightline}|    \midrule
|\only<3>{\highlightline}|冰墩墩 & 2022 年北京冬季奥运会吉祥物  \\
|\only<3>{\highlightline}|雪容融 & 2022 年北京冬季残奥会吉祥物  \\
|\only<2>{\highlightline}|    \bottomrule
|\only<1>{\highlightline}|  \end{tabular}
\end{table}
\end{document}
      \end{codeblock}
    \end{column}
    \begin{column}{0.4\textwidth}
      \only<1>{
        使用 \env{tabular} 环境绘制表格。需要添加参数(称为\textbf{表格导言})以确定每一列的对齐方式。引入 \pkg{array} 宏包来使用更多的\textbf{引导符}。
        \begin{center}
          \footnotesize
          \begin{stampbox}
            \begin{tabular}{>{\ttfamily}ll}
              \alert{l} & 向左对齐 \\
              \alert{c} & 居中对齐 \\
              \alert{r} & 向右对齐 \\
              \alert{p\{3cm\}} & 固定列宽,两端对齐 \\
              \alert{m\{3cm\}} & \texttt{p} + 垂直居中对齐 \\
              \alert{>\{\textbackslash{}bfseries\}} & 后一列单元格前加命令 \\
              \alert{*\{3\}\{l\}} & 三个左对齐列 \\
            \end{tabular}
          \end{stampbox}
        \end{center}
      }
      \only<2>{
        \pkg{booktabs} 宏包提供了标准三线表格所需要的行分割线:\cmd{toprule},\cmd{midrule},\cmd{bottomrule}。也可以使用 \cmd{cmidrule\{1-2\}} 来部分地绘制行分割线。一般不推荐在专业表格中使用纵向分割线。
      }
      \only<3>{
        每行内容使用 \textbackslash\textbackslash{} 分隔开,每行中的单元格使用 \& 分隔开。
      }
      \only<4>{
        \includepdflarge{table}
      }
    \end{column}
  \end{columns}
\end{frame}

\begin{frame}[fragile]%
  \begin{columns}
    \begin{column}{0.6\textwidth}
      \begin{codeblock}[]{表头居中}
\documentclass{ctexart}
\usepackage{array,booktabs}
\begin{document}
\begin{table}[ht]
  \centering
  \caption{||北京冬奥会吉祥物}
  \begin{tabular}{lp{3cm}}
    \toprule
|\highlightline|\multicolumn{1}{c}{||吉祥物} &
|\highlightline|\multicolumn{1}{c}{||描述} \\
    \midrule
||冰墩墩 & 2022 年北京冬季奥运会吉祥物  \\
||雪容融 & 2022 年北京冬季残奥会吉祥物  \\
    \bottomrule
  \end{tabular}
\end{table}
\end{document}
      \end{codeblock}
    \end{column}
    \begin{column}{0.4\textwidth}
      \cmd{multicolumn} 命令不仅可以用于合并同行的单元格,还可以用于临时地屏蔽表格导言对该列的对齐方式定义。这里用于居中表头。
      \begin{center}
        \begin{stampbox}
          \parbox{0.85\linewidth}{
            \ttfamily\color{blue}\textbackslash{}multicolumn\{格数\}\{对齐方式\}\{内容\}
          }
        \end{stampbox}
      \end{center}
      跨页表格考虑使用 \pkg{longtable} 宏包。带标注的表格可以考虑使用 \pkg{threeparttable} 宏包。考虑使用在线工具生成表格代码 \link{https://www.tablesgenerator.com/latex_tables}。
    \end{column}
  \end{columns}
\end{frame}

\section{数学公式}
\begin{frame}
  \frametitle{数学模式}
  \begin{alertblock}{}
  输入数学公式是 \LaTeX{} 的绝对强项,很多常见的公式服务依赖于一些轻量级渲染引擎比如 MathJax, K\kern-.3ex\raise.4ex\hbox{\footnotesize A}\kern-.3ex\TeX{}。但是它们实际上使用的是 \LaTeX{} 语法变种,也就是说并没有使用 \LaTeX{} 后端。所以不要期望有完全一致的输出。
  \end{alertblock}
  
  为了更好的获得数学公式输入支持,请使用 \hologo{AmS}math 宏包。数学模式分为两种:
  \begin{description}
    \item[行内模式] 一般通过一对美元符号(\$$\cdots$\$)标记,可以使用编辑器内建的符号表输入数学符号,也可以使用在线工具手写识别 \link{https://detexify.kirelabs.org/classify.html}。
    \item[行间模式] 一般通过 \env{equation} 环境\footnote{这是有编号环境,其加星号的变种 \env{equation*} 用于生成无编号环境。}输入。如果需要使用多行公式,请使用 \env{align} 环境,并按照类似表格输入的方式,使用 \& 对齐符号,\textbackslash\textbackslash{} 换行。如果不想手动居中,可以考虑多行自动居中的 \env{gather} 和单个大型公式首尾两端对齐 \env{multline}。
  \end{description}
\end{frame}

\begin{frame}
  \frametitle{数学命令展示}
  \begin{columns}
    \begin{column}{0.33\textwidth}
      \begin{exampleblock}{}
        $PV=nRT$
      \end{exampleblock}
      \begin{exampleblock}{}
        $\sum_{i=1}^ki^2=\frac{n(n+1)(2n+1)}{6}$
      \end{exampleblock}
      \begin{exampleblock}{}
        $T(n) = aT\left(\left\lceil\frac{n}{b}\right\rceil\right) + \mathcal{O}(n^d)$
      \end{exampleblock}
      \begin{exampleblock}{}
        $\frac{x_{1}+x_{2}+x_{3}}{3}\geq \sqrt[3]{x_{1}x_{2}x_{3}}$
      \end{exampleblock}
      \begin{exampleblock}{}
        $n=(\underbrace{1\cdots 1}_{k\text{ of 1's}})_2=2^{k+1}-1$
      \end{exampleblock}
      \begin{exampleblock}{}
        $\nabla f (P)= \left.\left(\frac{\partial f}{\partial x},\frac{\partial f}{\partial y},\frac{\partial f}{\partial z}\right)\right|_{P}$
      \end{exampleblock}
    \end{column}
    \begin{column}{0.67\textwidth}
      \begin{exampleblock}{}
        \begin{equation*}
          \int_{a}^b f(x)\,\mathrm{d}x=\lim_{|P|\rightarrow 0}\sum_{i=1}^n f(\xi_i)\Delta x_i
        \end{equation*}
      \end{exampleblock}
      \begin{exampleblock}{}
        \begin{equation}
          T(n) = \begin{cases}
            \mathcal{O}(n^d),&\textrm{if } d>\log_b a, \\
            \mathcal{O}(n^d\log n), &\textrm{if } d=\log_b a,\\
            \mathcal{O}(n^{\log_b a}), &\textrm{if } d<\log_b a.
          \end{cases}
        \end{equation}
      \end{exampleblock}
      \begin{exampleblock}{}
        \begin{align}
          Q^{T}A&=R \\
          \begin{pmatrix}
            q_1^T \\ q_2^T \\ q_3^T
          \end{pmatrix}
          \begin{pmatrix}
            a_1 & a_2 & a_3
          \end{pmatrix}
          &=R
        \end{align}
      \end{exampleblock}
    \end{column}
  \end{columns}
\end{frame}

%更深入地讲解 mathtools, unicode-math, siunix

\section{引用}
\begin{frame}[fragile]
  \frametitle{交叉引用}
  \only<1>{
    正如之前所提到的,\LaTeX{} 中使用 \cmd{label} 标记,然后可以使用 \cmd{ref} 来引用这个标记。 \cmd{label} 可以放在使用计数器的对象之后。
  }
  \only<2>{
    为了使得对公式编号的引用带有括号,推荐使用 \hologo{AmS}math 宏包中的 \cmd{eqref} 命令。对于多行公式环境,每一个换行符前都可以添加一个 \cmd{label} 用于引用该行公式。
  }
  \begin{columns}
    \begin{column}{0.5\textwidth}
      \begin{codeblock}[]{图}
\begin{figure}
|\only<1>{\highlightline}|  \caption{||示例}\label{fig:example}
\end{figure}
      \end{codeblock}
      \begin{codeblock}[]{表}
\begin{table}
|\only<1>{\highlightline}|  \caption{||示例}\label{tab:example}
\end{table}
      \end{codeblock}
    \end{column}
    \begin{column}{0.5\textwidth}
\begin{codeblock}[]{目次}
|\only<1>{\highlightline}|\section{||示例}\label{sec:example}
\end{codeblock}

\begin{codeblock}[]{公式}
\begin{equation}
  a = b + c
|\only<1>{\highlightline}|\label{eq:example}
\end{equation}
|\only<2>{\highlightline}|如公式 \eqref{eq:example} 所示,
\end{codeblock}
    \end{column}
  \end{columns}
\end{frame}

\begin{frame}[fragile]
  \frametitle{文献引用}
  \LaTeX{} 管理参考文献可以采用专用数据库文件 \texttt{.bib},很多的文献管理文件比如 EndNote \link{https://lic.sjtu.edu.cn/Default/softshow/tag/MDAwMDAwMDAwMLGImKE}, Zotero \link{https://www.zotero.org/}, JabRef \link{https://www.jabref.org/} 都可以直接导出这种格式的文件用于 \LaTeX{} 论文中的引用。一般不需要手写数据库文件,你只需要注意每一个文献会在数据库中有一个主键,这个类似于上文的 \cmd{label} 标记,只是要引用数据库中的文献需要使用 \cmd{cite} 命令。
  
  \begin{codeblock}[]{ref.bib}
|\highlightline|@phdthesis{devoftech,|\hfill\alert{\% 类型为博士论文,主键为\texttt{devoftech}}|
  title={||新时期我国信息技术产业的发展},
  author={||江泽民},
  year={2008}
}
  \end{codeblock}
\end{frame}

\begin{frame}
  \frametitle{文献引用}
  而让 \LaTeX{} 处理 \texttt{.bib} 数据库文件与引用有两种工作流。你可能需要查清楚如何在编辑器中设置对应的工作流,或者采用后面所提到的高级编译方式 \texttt{latexmk}。
  \begin{columns}
    \begin{column}{0.5\textwidth}
      \begin{block}{\hologo{BibTeX} + \pkg{gbt7714}}
        一般期刊提交使用这种方法,\pkg{natbib} 宏包将提供命令 \cmd{citet}(文本引用) 和 \cmd{citep}(括号引用)。中文引用可以直接使用 \pkg{gbt7714} 宏包,但是角模式和正文模式不能混用。
      \end{block}
    \end{column}
    \begin{column}{0.5\textwidth}
      \begin{block}{\hologo{biber} + \pkg{biblatex}}
        这是更容易自定义的方法,与 \hologo{BibTeX} 的运作方式稍有不同。\pkg{biblatex} 提供了更加智能的引用命令。而中文引用可以使用 \pkg{biblatex} 宏包的样式 \pkg{gb7714-2015},使用该样式需要使用 \hologo{XeLaTeX} 编译。
      \end{block}
    \end{column}
  \end{columns}
\end{frame}

\begin{frame}[fragile]
  \frametitle{文献引用}
  \begin{columns}
    \begin{column}{0.5\textwidth}
      \begin{codeblock}[]{\hologo{BibTeX} + \pkg{gbt7714}}
\documentclass{ctexart}
\usepackage{gbt7714}
\bibliographystyle{gbt7714-numerial}
% \citestyle{numbers}  % 正文模式
\begin{document}
  ||他指出了...\cite{devoftech}
  \bibliography{ref}
\end{document}
      \end{codeblock}
    \end{column}
    \begin{column}{0.5\textwidth}
      \begin{codeblock}[]{\hologo{biber} + \pkg{biblatex}}
\documentclass{ctexart}
\usepackage[backend=biber,style=gb7714-2015]{biblatex}
\addbibresource{ref.bib}
\begin{document}
  ||他在文献 \parencite{devoftech}
  ||指出了...\cite{devoftech}
  \printbibliography
\end{document}
      \end{codeblock}
    \end{column}
  \end{columns}
\end{frame}

\begin{frame}
  \frametitle{文献引用}
  \begin{columns}
    \begin{column}{0.5\textwidth}
      \includepdflarge{bibtex}
    \end{column}
    \begin{column}{0.5\textwidth}
      \includepdflarge{biblatex}
    \end{column}
  \end{columns}
\end{frame}

} % End of customized shaded number logo

  % !TeX root = ..\..\latex-talk.tex

\part{SJTUThesis}

\begin{frame}
  \frametitle{简介}
  \begin{columns}
    \begin{column}{0.6\textwidth}
      \begin{itemize}
        \item 最早由韦建文于 2009 年 11 月发布 0.1a 版,2018 年起由 SJTUG 接手维护
        \item 最新版:\SJTUThesisVersion{} (\SJTUThesisDate)
        \item 支持本科、硕士、博士学位论文以及课程论文的排版
      \end{itemize}
    \end{column}
    \begin{column}{0.4\textwidth}
      \begin{exampleblock}{}
        \begin{minipage}[c]{1cm}
          \includegraphics[width=0.8cm]{\getcontribpath{sjtug}{vi/sjtug}}
        \end{minipage}
        \begin{minipage}[c]{2cm}
          \href{https://github.com/sjtug}{sjtug}/\href{https://github.com/sjtug/SJTUThesis}{SJTUThesis}
        \end{minipage}
      \end{exampleblock}
      \vspace{-8pt}
      \begin{block}{}
        \scriptsize
        上海交通大学 \hologo{XeLaTeX} 学位论文及课程论文模板 | Shanghai Jiao Tong University \hologo{XeLaTeX} Thesis Template
      \end{block}
      \vspace{-8pt}
      \begin{alertblock}{}
        \scriptsize
        \begin{tabular}{cl}
          \faStar & 2.4k \\
          \faEye & 55 \\
          \faCodeBranch & 701 \\
        \end{tabular}
      \end{alertblock}
    \end{column}
  \end{columns}
\end{frame}

\begin{frame}
  \frametitle{下载与编译}
  \alert{下载} 推荐安装 Git \link{https://git-scm.com/} 后,克隆 SJTUG 镜像仓库
  \begin{exampleblock}{\faGit*}
    \ttfamily\small
    git clone https://mirror.sjtu.edu.cn/git/SJTUThesis.git/
  \end{exampleblock}

  \alert{编译} 推荐使用 \pkg{latexmk} 编译\footnote{\hologo{MiKTeX} 用户需要手动安装 Perl 解释器 \link{https://www.perl.org/get.html} 才能使用 \pkg{latexmk}。},在不能够利用自带的 \texttt{.latexmkrc} 配置文件的情况下,需要查清楚在对应的编辑器中如何使用 \hologo{XeLaTeX} + \hologo{biber} 编译 \link{https://github.com/sjtug/SJTUThesis/blob/master/README.md}。
  \begin{exampleblock}{\faTerminal}
    \ttfamily\small
    latexmk -xelatex main
  \end{exampleblock}

  Overleaf 用户可以下载压缩包后,上传并采用 \hologo{XeLaTeX} 编译方式。
\end{frame}

\begin{frame}
  \frametitle{手动编译}
  \alert{第一次编译失败} 如果没有办法通过通常方式编译成功,请尝试使用文件夹内附带 \faLinux{}\,\faApple{} \texttt{Makefile} 和 \faWindows{} \texttt{Compile.bat} 进行编译。

  \alert{统计字数} 编写过程中也可以使用对应的命令调用 \TeX{}count 来统计正文字数。
  \begin{columns}
    \begin{column}{0.38\textwidth}
      \begin{exampleblock}{\faLinux{}\,\faApple}
        \ttfamily
        make all\\
        make clean\\
        make cleanall\\
        make wordcount
      \end{exampleblock}
    \end{column}
    \begin{column}{0.38\textwidth}
      \begin{exampleblock}{\faWindows}
        \ttfamily
        ./Compile.bat thesis\\
        ./Compile.bat clean\\
        ./Compile.bat cleanall\\
        ./Compile.bat wordcount
      \end{exampleblock}
    \end{column}
    \begin{column}{0.24\textwidth}
      \begin{block}{\faInfo}
        \ttfamily
        编译论文\\
        清理中间文件\\
        $\hookrightarrow +$删除论文\\
        统计字数
      \end{block}
    \end{column}
  \end{columns}
\end{frame}

\begin{frame}[label=compile]
  \frametitle{编译问题排查}
  \begin{columns}
    \begin{column}{0.33\textwidth}
      \begin{alertblock}{无法使用 \texttt{latexmk}\thesisissue{578}}
        \hologo{MiKTeX} 需要安装 Perl 解释器。
      \end{alertblock}  
      \begin{alertblock}{C\TeX{} 套装无法编译\thesisissue{446}}
        使用最新 \TeX{} 发行版。
      \end{alertblock}
      \begin{alertblock}{\hologo{pdfLaTeX} 无法编译\thesisissue{444}}
        请使用 \texttt{latexmk},或更改编辑器设置以 \hologo{XeLaTeX} 编译。
      \end{alertblock}
    \end{column}
    \begin{column}{0.33\textwidth}
      \begin{alertblock}{缺少字体\thesisissue{564} \thesisdiscuss{598}}
        更换字体集,或者安装对应字体。
      \end{alertblock}
      \begin{alertblock}{缺少汉字\thesisissue{533} \thesisdiscuss{617}}
        去除使用 fandol 字体集的设定。或者是安装字体后,改用 \texttt{fontset=adobe} 或 \texttt{fontset=founder}。
      \end{alertblock}
    \end{column}
    \begin{column}{0.33\textwidth}
      \begin{block}{\faInfoCircle{} README}
        不同编辑器的设置请首先参阅 README \link{https://github.com/sjtug/SJTUThesis/blob/master/README.md} 文档。
      \end{block}
      \begin{block}{\faBookOpen{} Wiki}
        其他编译问题推荐查阅 Wiki \link{https://github.com/sjtug/SJTUThesis/wiki} 的使用说明部分。
      \end{block}
    \end{column}
  \end{columns}
\end{frame}

\begin{frame}[fragile, label=covers]
  \begin{codeblock}[firstnumber=3]{main.tex}
|\alert{\% 载入 SJTUThesis 模版}|
\documentclass[|\only<1>{\highlight{type}}\only<2>{type}|=|\only<1>{bachelor}\only<2>{\highlight{bachelor}}|]{sjtuthesis}
  \end{codeblock}
  \begin{figure}
    \parbox{0.9\textwidth}{
      \begin{subfigure}{0.20\textwidth}
        \framebox{\includegraphics[width=\linewidth]{support/thesis/bachelor}}
        \caption{\only<1>{学士}\only<2>{\texttt{bachelor}}}
      \end{subfigure}\hfill
      \begin{subfigure}{0.20\textwidth}
        \framebox{\includegraphics[width=\linewidth]{support/thesis/master}}
        \caption{\only<1>{硕士}\only<2>{\texttt{master}}}
      \end{subfigure}\hfill
      \begin{subfigure}{0.20\textwidth}
        \framebox{\includegraphics[width=\linewidth]{support/thesis/doctor}}
        \caption{\only<1>{博士}\only<2>{\texttt{doctor}}}
      \end{subfigure}\hfill
      \begin{subfigure}{0.20\textwidth}
        \framebox{\includegraphics[width=\linewidth]{support/thesis/course}}
        \caption{\only<1>{课程}\only<2>{\texttt{course}}}
      \end{subfigure}
      \caption{论文类型示例\only<2>{ \texttt{type}}}
    }
  \end{figure}
\end{frame}

\begin{frame}[fragile]
  \frametitle{文档类选项}
  % \framesubtitle{\textbackslash{}documentclass\{sjtuthesis\}}
  \begin{columns}
    \begin{column}{0.45\textwidth}
      \includegraphics[page=10]{thesisdir}
    \end{column}
    \begin{column}{0.55\textwidth}
      \begin{table}[H]
        \caption{文档类选项}
        \footnotesize
        \begin{tabular}{>{\ttfamily}rll}
          \toprule
          选项 & 含义 & 相关 \\
          \midrule
          type= & 指定论文类型 & 第 \ref{covers} 页\\
          fontset= & 指定字体 & 第 \ref{compile} 页\\
          \midrule
          review & 开启盲审模式 & \thesisissue{195} \thesisissue{686} \\
          twoside & 双页模式 & \thesisissue{554} \\
          oneside & 单页模式 & \thesisissue{694} \\
          openright & 章从奇数页开始 & \thesisdiscuss{724} \\
          openany & 章从任意页开始 & \thesisissue{446} \\
          \bottomrule
        \end{tabular}
      \end{table}
    \end{column}
  \end{columns}
\end{frame}

\begin{frame}[fragile]
  \frametitle{基本配置}
  \framesubtitle{\textbackslash{}input\{setup\}}
  \begin{columns}
    \begin{column}{0.45\textwidth}
      \includegraphics[page=9]{thesisdir}
    \end{column}
    \begin{column}{0.55\textwidth}
      \begin{codeblock}[firstnumber=12]{main.tex}
|\highlightline<1>|% 论文基本配置,加载宏包等全局配置
|\highlightline<1>|\input{setup}

\begin{document}

%TC:ignore

|\highlightline<2>|% 标题页
|\highlightline<2>|\maketitle
      \end{codeblock}
      \visible<2>{
        \cmd{sjtusetup} 中的 \pkg{info} 将会修改封面的信息设置(见第 \ref{covers} 页)。
      }
    \end{column}
  \end{columns}
\end{frame}

\begin{frame}[fragile]
  \frametitle{基本配置}
  \framesubtitle{\textbackslash{}sjtusetup}
  \begin{columns}
    \begin{column}{0.45\textwidth}
      \includegraphics[page=1]{thesisdir}
    \end{column}
    \begin{column}{0.55\textwidth}
      \begin{codeblock}[firstnumber=3]{setup.tex}
\sjtusetup{
  info = {
    title    = {||上海交通大学学位论文 \LaTeX{} 模板示例文档},
    title*   = {A Sample for \LaTeX-based SJTU Thesis Template},
    author   = {||某\quad{}某},
    author* = {Mo Mo},
  },
  style = { header-logo-color = red, 
  },
  name = {
    publications = {||攻读学位期间完成的论文},
  },
}
      \end{codeblock}
    \end{column}
  \end{columns}
\end{frame}

\begin{frame}
  \frametitle{基本配置}
  \framesubtitle{\textbackslash{}sjtusetup}
  \begin{columns}
    \begin{column}{0.45\textwidth}
      \includegraphics[page=1]{thesisdir}
    \end{column}
    \begin{column}{0.55\textwidth}
      \begin{table}[H]
        \centering
        \caption{info 域}
        \footnotesize
        \begin{tabular}{lll} \toprule
          命令作用 & 中文对应选项 & 英文对应选项 \\ \midrule
          论文标题 & \texttt{title} & \texttt{title*} \\
          关键字列表 & \texttt{keywords} & \texttt{keywords*} \\
          作者姓名&  \texttt{author} &\texttt{author*}\\
          申请学位名称 & \texttt{degree}&\texttt{degree*}\\
          院系名称 & \texttt{department} & \texttt{department*}\\
          专业名称 & \texttt{major} & \texttt{major*}\\
          导师 & \texttt{supervisor} & \texttt{supervisor*}\\
          副导师 & \texttt{assisupervisor} & \texttt{assisupervisor*}\\
          日期 & \multicolumn{2}{c}{\texttt{date}}\\
          学号 & \multicolumn{2}{c}{\texttt{id}}\\ \bottomrule
          \end{tabular}
      \end{table}
    \end{column}
  \end{columns}
\end{frame}

\begin{frame}[fragile]
  \frametitle{版权页}
  \framesubtitle{\textbackslash{}copyrightpage}
  \begin{columns}
    \begin{column}{0.45\textwidth}
      \only<1>{
        \includegraphics[page=9]{thesisdir}
      }
      \only<2>{
        \includegraphics[page=2]{thesisdir}
      }
      \only<3>{
        \begin{figure}[H]
          \framebox{\includegraphics[page=2,width=0.4\linewidth]{bachelor}}
          \caption{版权页}
        \end{figure}
      }
    \end{column}
    \begin{column}{0.55\textwidth}
      \begin{codeblock}[firstnumber=22]{main.tex}
|\highlightline<1>|% 原创性声明及使用授权书
|\highlightline<1>|\copyrightpage
|\highlightline<2>|% 插入外置原创性声明及使用授权书
|\highlightline<2>|% \copyrightpage[scans/sample-copyright-old.pdf]
      \end{codeblock}
      \only<1>{
        \cmd{copyrightpages} 可以用于插入版权页。
      }
      \only<2>{
        \cmd{copyrightpages} 也接受一个可选参数,用于直接使用扫描件。
      }
    \end{column}
  \end{columns}
\end{frame}

\begin{frame}[fragile]
  \frametitle{前置部分}
  \framesubtitle{\textbackslash{}frontmatter}
  \begin{columns}
    \begin{column}{0.45\textwidth}
      \only<1>{
        \includegraphics[page=9]{thesisdir}
      }
      \only<2>{
        \includegraphics[page=3]{thesisdir}
      }
      \only<3>{
        \begin{figure}[H]
          \begin{subfigure}{0.45\textwidth}
            \framebox{\includegraphics[page=3,width=\linewidth]{bachelor}}
            \caption{中文}
          \end{subfigure}\hfill
          \begin{subfigure}{0.45\textwidth}
            \framebox{\includegraphics[page=4,width=\linewidth]{bachelor}}
            \caption{英文}
          \end{subfigure}
          \caption{摘要}
        \end{figure}
      }
      \only<4>{
        \begin{figure}[H]
          \begin{subfigure}{0.30\linewidth}
            \centering
            \framebox{\includegraphics[page=5,width=0.6\linewidth]{bachelor}}
            \caption{目录}
          \end{subfigure}
          \begin{subfigure}{0.30\linewidth}
            \centering
            \framebox{\includegraphics[page=6,width=0.6\linewidth]{bachelor}}
            \caption{插图}
          \end{subfigure}

          \begin{subfigure}{0.30\linewidth}
            \centering
            \framebox{\includegraphics[page=7,width=0.6\linewidth]{bachelor}}
            \caption{表格}
          \end{subfigure}
          \begin{subfigure}{0.30\linewidth}
            \centering
            \framebox{\includegraphics[page=8,width=0.6\linewidth]{bachelor}}
            \caption{算法}
          \end{subfigure}
          \caption{索引}
        \end{figure}
      }
      \only<5>{
        \includegraphics[page=4]{thesisdir}
      }
      \only<6>{
        \begin{figure}[H]
          \framebox{\includegraphics[page=9,width=0.5\linewidth]{bachelor}}
          \caption{符号对照表}
        \end{figure}
      }
    \end{column}
    \begin{column}{0.55\textwidth}
      \begin{codeblock}[firstnumber=30]{main.tex}
|\highlightline<2-3>|% 摘要
|\highlightline<2-3>|\input{contents/abstract}

|\highlightline<4>|% 目录
|\highlightline<4>|\tableofcontents
|\highlightline<4>|% 插图索引
|\highlightline<4>|\listoffigures*
|\highlightline<4>|% 表格索引
|\highlightline<4>|\listoftables*
|\highlightline<4>|% 算法索引
|\highlightline<4>|\listofalgorithms*

|\highlightline<5-6>|% 符号对照表
|\highlightline<5-6>|\input{contents/nomenclature}
      \end{codeblock}
    \end{column}
  \end{columns}
\end{frame}

\begin{frame}[fragile]
  \frametitle{主体部分}
  \framesubtitle{\textbackslash{}mainmatter}
  \begin{columns}
    \begin{column}{0.45\textwidth}
      \only<1>{
        \includegraphics[page=5]{thesisdir}
      }
      \only<2>{
        \begin{figure}[H]
          \begin{subfigure}{0.30\linewidth}
            \centering
            \framebox{\includegraphics[page=11,width=0.6\linewidth]{bachelor}}
            \caption{简介}
          \end{subfigure}
          \begin{subfigure}{0.30\linewidth}
            \centering
            \framebox{\includegraphics[page=13,width=0.6\linewidth]{bachelor}}
            \caption{数学}
          \end{subfigure}

          \begin{subfigure}{0.30\linewidth}
            \centering
            \framebox{\includegraphics[page=16,width=0.6\linewidth]{bachelor}}
            \caption{浮动体}
          \end{subfigure}
          \begin{subfigure}{0.30\linewidth}
            \centering
            \framebox{\includegraphics[page=22,width=0.6\linewidth]{bachelor}}
            \caption{总结}
          \end{subfigure}
          \caption{主体部分}
        \end{figure}
      }
    \end{column}
    \begin{column}{0.55\textwidth}
      \begin{codeblock}[firstnumber=47]{main.tex}
|\highlightline|% 正文内容
|\highlightline|% !TeX root = ../../../latex-talk.tex

\section{是什么}

\begin{frame}
  \frametitle{\TeX{}}
  \begin{columns}[c]
    \begin{column}{0.7\textwidth}
      \begin{center}
        \rmfamily\Huge
        \highlight[structure]{\TeX{}}
      \end{center}
      \begin{center}
        \parbox{0.75\textwidth}{
          \TeX{} 是由斯坦福大学教授高德纳
          (Donald E.~Knuth)于 1977 年开始开发的排版引擎。目前仍在更新,最新版本号为 3.141592653 \link{https://tug.org/TUGboat/tb42-1/tb130knuth-tuneup21.pdf}。
        }
      \end{center}
    \end{column}
    \begin{column}{0.3\textwidth}
      \includegraphics[width=.8\columnwidth]{support/images/Knuth.jpg}
    \end{column}
  \end{columns}
  \note{\emph{这一部分背景介绍大家可以了解一下,暂时跳过。}
  \LaTeX{} 这个词由两个部分组成,\hologo{La} 和 \TeX{}。那我们首先了解一下 \TeX{} 是什么。
  \TeX{} 是由斯坦福大学的教授高德纳于 1977 年开始开发的排版引擎,它已经有三十多年的历史了,
  目前仍在更新,版本号(3.141592653)将会趋近于 $\pi$ 的取值,高德纳最近还在给 \textsl{TUGBoat} 写稿子
  \link{https://tug.org/TUGboat/tb42-1/tb130knuth-tuneup21.pdf},
  关于 \TeX{} 今年又做了哪些改进。}
\end{frame}

\begin{frame}
  \frametitle{\LaTeX{}}
  \begin{columns}[c]
    \begin{column}{0.7\textwidth}
      \begin{center}
        \rmfamily\Huge
        \highlight[structure]{\LaTeX{}}
      \end{center}
      \begin{center}
        \parbox{0.75\textwidth}{
          \LaTeX{} 是最早在 1985 年由现就职于微软的 Leslie Lamport 开发的一种 \TeX{} \textbf{格式}\footnotemark,使用一些列宏和扩展宏包来简化 \TeX{} 的使用。现在由 \LaTeX{} Project 的成员维护。现在广泛使用的版本是 \LaTeXe{},最新的版本为 \LaTeX3(2020 年 10 月后默认内置)。
        }
      \end{center}
    \end{column}
    \begin{column}{0.3\textwidth}
      \includegraphics[width=.8\columnwidth]{support/images/Lamport.jpg}
    \end{column}
  \end{columns}
  \footnotetext{\hologo{ConTeXt} 也是一种 \TeX{} 格式 \link{https://www.contextgarden.net/}。}
  \note{\emph{这一部分的背景介绍大家可以了解一下,暂时跳过。}
  \LaTeX{} 是最早由现就职于微软的 Leslie Lamport 开发的一种 \TeX{} 格式(与其对标的是
  \hologo{ConTeXt}\link{https://www.contextgarden.net/}),主要也是为了简化 \TeX{} 的使用。
  现在主要由 \LaTeX{} 开发组维护,现在广泛使用的版本是 \LaTeXe{},最新的版本为 \LaTeX3,
  在 2020 年 10 月后默认内置,所以要尽可能使用较新的发行版,以充分发挥其功能。}
\end{frame}

\begin{frame}
  \frametitle{程序}
  \begin{columns}[c]
    \begin{column}{0.7\textwidth}
      \begin{center}
        \rmfamily\Huge
        \highlight[structure]{\hologo{pdfLaTeX}}
      \end{center}
      \begin{center}
        \parbox{0.7\textwidth}{
          \hologo{pdfLaTeX} 是为了编译一个 \LaTeX{} 文档而运行的程序。实际上底层在运行一个叫 \hologo{pdfTeX} 的引擎,并预装了对应的 \LaTeX{} \textbf{格式}。为了利用临时文件,可能就需要多次运行程序。
        }
      \end{center}
    \end{column}
    \begin{column}{0.3\textwidth}
      \begin{block}{}
        \ttfamily\small
        > \highlight{pdflatex} main.tex\\
        This is pdfTeX, Version 3.141592653-
        2.6-1.40.23 (MiKTeX 21.10)\\
        entering extended mode\\
        \highlight{LaTeX2e} <2021-11-15>\\
        \highlight{L3} programming layer <2021-11-22>
      \end{block}
    \end{column}
  \end{columns}
  \note{\hologo{pdfLaTeX} 是为了编译一个 \LaTeX{} 文档而运行的程序。}
\end{frame}

% \begin{frame}
%   \frametitle{引擎}
%   \begin{columns}[c]
%     \begin{column}{0.7\textwidth}
%       \begin{center}
%         \rmfamily\Huge
%         \highlight[structure!70]{pdf}\hologo{La}\highlight[structure!70]{\TeX{}}
%       \end{center}
%       \begin{center}
%         \parbox{0.7\textwidth}{
%           pdf\TeX{} 是编译 \TeX{} 文档(以 \texttt{.tex} 结尾)的\textbf{引擎}---可以理解 \TeX{} 指令的\textbf{程序}。
%         }
%       \end{center}
%     \end{column}
%     \begin{column}{0.3\textwidth}
%       \begin{block}{}
%         \ttfamily\small
%         > pdflatex main.tex\\
%         This is \highlight[structure!70]{pdfTeX}, Version 3.141592653-
%         2.6-1.40.23 (MiKTeX 21.10)
%         entering extended mode\\
%         LaTeX2e <2021-11-15>\\
%         L3 programming layer <2021-11-22>
%       \end{block}
%     \end{column}
%   \end{columns}
%   \note{实际上底层在运行一个叫 \hologo{pdfTeX} 的引擎,并预装了对应的 \LaTeX{} 格式。}
% \end{frame}

\begin{frame}[label={frame:engine}]
  \frametitle{程序}
  \begin{table}
    \caption{主流 \hologo{(La)TeX} 程序
    \footnote{(u)p\TeX{} 是日语最常用的引擎,生成 \texttt{.dvi},支持 Unicode。}\footnote{Ap\TeX{} \link{https://github.com/clerkma/ptex-ng} 具有底层 CJK 支持,内联 Ruby,Color Emoji。}}
    \footnotesize
    \begin{stampbox}
      \begin{tabular}{c>{\raggedright}*{3}{p{3.5cm}}}
        \alert{引擎}     & \hologo{pdfTeX}   & \hologo{XeTeX}   & \hologo{LuaTeX}   \\
        \alert{程序}     & \hologo{pdfLaTeX} & \hologo{XeLaTeX} & \hologo{LuaLaTeX} \\
        \alert{特点}     & 直接生成 PDF,支持 micro-typography  & 支持 Unicode、OpenType 与复杂文字编排 (CTL) & 支持 Unicode,内联 Lua,支持 OpenType \\
      \end{tabular}
    \end{stampbox}
  \end{table}

  \begin{center}
    \parbox{.9\textwidth}{
      \hologo{pdfLaTeX} 不支持 Unicode。为了排版中文,大部分情况下应当使用 \hologo{XeLaTeX},而 \hologo{LuaLaTeX} 速度相对较慢。\faWindows{} 可以在一些情况下使用 \hologo{pdfLaTeX}。
    }
  \end{center}
  \note{当然为了排版中文,已经不再推荐使用 \hologo{pdfLaTeX} 了,应该使用
  \hologo{XeLaTeX} 或者 \hologo{LuaLaTeX},当然后者的速度还是相对较慢,
  它们支持 Unicode 编码,并可以使用 OpenType 字体的全部功能。
  当然 \faWindows{} 平台下在某些追求速度的情况下,
  还是可以试着使用 \hologo{pdfLaTeX} 的。

  \hologo{LuaLaTeX} 理想情况下不慢,但是使用一些宏包后会破坏理想状态,
  也会因配置产生不同的结果,不同的操作系统在 I/O 速度上的不同也会导致不同的时间。

  \hologo{pdfLaTeX} 也支持,只不过需要先生成 tfm \TeX{} 字体度量文件,后续使用 \TeX{}
  自身的配置方法,只能使用 7 比特或 8 比特字体。}
\end{frame}

% \begin{frame}
%   \paragraph{\hologo{pdfLaTeX}} \TeX{} 和 \LaTeX{} 被广泛使用之前,它们只需内置支持欧洲语言即可。在 Unicode 出现之前,\LaTeX{} 提供了许多种\textbf{文件编码}来允许很多语言的文字以原生的方式输入,\hologo{pdfLaTeX} 也只需要使用 8 位文件编码和 8 位字体。
% \end{frame}


|\highlightline|\input{contents/math_and_citations}
|\highlightline|\input{contents/floats}
|\highlightline|\input{contents/summary}

%TC:ignore

% 参考文献
\printbibliography[heading=bibintoc]
      \end{codeblock}
    \end{column}
  \end{columns}
\end{frame}

\begin{frame}
  \frametitle{数学}
  \begin{itemize}
    \item 公式示例:\nolinkurl{contents/math_and_citations.tex}
    \item \SJTUThesis{} 定义了常用的数学环境(需要手工引入 \texttt{ntheorem} 宏包):
      \begin{table}[h]
        \centering
        \footnotesize
        \begin{tabular}{*{7}{l}}\toprule
          assumption  & axiom   & conjecture & corollary    & definition  & example   & exercise  \\
          假设        & 公理    & 猜想       & 推论         & 定义        & 例        & 练习      \\\midrule
          lemma       & problem & proof      & proposition  & remark      & solution  & theorem   \\
          引理        & 问题    & 证明       & 命题         & 注          & 解        & 定理      \\\bottomrule
        \end{tabular}
      \end{table}
      \item \SJTUThesis{} 可以通过 \texttt{unimath} 选项使用 \pkg{unicode-math} 进行数学输入,注意与传统方式的区别。\thesisissue{555}
  \end{itemize}
\end{frame}

\begin{frame}[fragile]
  \frametitle{参考文献}
  \begin{columns}
    \begin{column}{0.45\textwidth}
      \includegraphics[page=6]{thesisdir}
    \end{column}
    \begin{column}{0.55\textwidth}
      \begin{codeblock}[firstnumber=111,numbersep=2pt]{setup.tex}
% 使用 BibLaTeX 处理参考文献
%   biblatex-gb7714-2015 常用选项
%     gbnamefmt=lowercase     姓名大小写由输入信息确定
%     gbpub=false             禁用出版信息缺失处理
\usepackage[backend=biber,style=gb7714-2015]{biblatex}
% 文献表字体
% \renewcommand{\bibfont}{\zihao{-5}}
% 文献表条目间的间距
\setlength{\bibitemsep}{0pt}
|\highlightline|% 导入参考文献数据库
|\highlightline|\addbibresource{bibdata/thesis.bib}
      \end{codeblock}
    \end{column}
  \end{columns}
\end{frame}

\begin{frame}[fragile]
  \frametitle{附录}
  \framesubtitle{\textbackslash{}appendix}
  \begin{columns}
    \begin{column}{0.45\textwidth}
      \only<1>{
        \includegraphics[page=7]{thesisdir}
      }
      \only<2>{
        \begin{figure}[H]
          \begin{subfigure}{0.45\linewidth}
            \framebox{\includegraphics[width=\linewidth,page=24]{bachelor}}
            \caption{}
          \end{subfigure}\hfill
          \begin{subfigure}{0.45\textwidth}
            \framebox{\includegraphics[width=\linewidth,page=25]{bachelor}}
            \caption{}
          \end{subfigure}
          \caption{附录}
        \end{figure}
      }
    \end{column}
    \begin{column}{0.55\textwidth}
      \begin{codeblock}[firstnumber=61]{main.tex}
% 附录中图表不加入索引
\captionsetup{list=no}

% 附录内容
|\highlightline|\input{contents/app_maxwell_equations}
|\highlightline|\input{contents/app_flow_chart}
      \end{codeblock}
    \end{column}
  \end{columns}
\end{frame}

\begin{frame}[fragile]
  \frametitle{结尾部分}
  \framesubtitle{\textbackslash{}backmatter}
  \begin{columns}
    \begin{column}{0.45\textwidth}
      \only<1>{
        \includegraphics[page=8]{thesisdir}
      }
      \only<2>{
        \begin{figure}[H]
          \begin{subfigure}{0.30\linewidth}
            \centering
            \framebox{\includegraphics[page=26,width=0.6\linewidth]{bachelor}}
            \caption{致谢}
          \end{subfigure}
          \begin{subfigure}{0.30\linewidth}
            \centering
            \framebox{\includegraphics[page=27,width=0.6\linewidth]{bachelor}}
            \caption{成就}
          \end{subfigure}

          \begin{subfigure}{0.30\linewidth}
            \centering
            \framebox{\includegraphics[page=28,width=0.6\linewidth]{bachelor}}
            \caption{简历}
          \end{subfigure}
          \begin{subfigure}{0.30\linewidth}
            \centering
            \framebox{\includegraphics[page=29,width=0.6\linewidth]{bachelor}}
            \caption{大摘要*}
          \end{subfigure}
          \caption{结尾部分}
        \end{figure}
      }
    \end{column}
    \begin{column}{0.55\textwidth}
      \begin{codeblock}[firstnumber=76]{main.tex}
% 致谢
\input{contents/acknowledgements}

% 发表论文及科研成果
% 盲审论文中,发表论文及科研成果等仅以第几作者注明即可,不要出现作者或他人姓名
\input{contents/achievements}

% 简历
\input{contents/resume}

% 学士学位论文要求在最后有一个大摘要,单独编页码
\input{contents/digest}
      \end{codeblock}
    \end{column}
  \end{columns}
\end{frame}

\begin{frame}
  \frametitle{还有其他问题?}
  \begin{columns}
    \begin{column}{0.75\textwidth}
    \begin{itemize}
      \item[{\faComment*[regular]}] 日常模板或 \LaTeX{} 使用问题可以前往 Discussions \link{https://github.com/sjtug/SJTUThesis/discussions} 提问
      
      (解决后别忘了 \textcolor{green}{\faCheckCircle{} Mark as answer}
      \item[{\faDotCircle[regular]}] 如果是 \textsc{SJTUThesis} 项目本身的 bug 和 feature request
      
      可以通过 Issues \link{https://github.com/sjtug/SJTUThesis/issues} 反馈。
      \item[{\faCodeBranch}] 如果你有好点子,可以贡献代码
     
      向 \textsc{SJTU\TeX{}}(v1) \link{https://github.com/sjtug/SJTUTeX/tree/v1} 存储库发 PR,\par
      而后把解包结果同步到 \textsc{SJTUThesis}。
  
      \item[{\faTag}] 如果你对正在基于 \LaTeX3 开发的新版感兴趣,\par
      也欢迎向 \textsc{SJTU\TeX{}}(v2) \link{https://github.com/sjtug/SJTUTeX/tree/v2} 发 PR。
  
      \item[{\faQq}] 也欢迎在 QQ 群即时讨论。
    \end{itemize}
    \end{column}
    \begin{column}{0.25\textwidth}
      \includegraphics[height=0.7\textheight]{qq.jpg}
    \end{column}
  \end{columns}
\end{frame}
\end{document}
      \end{codeblock}
    \end{column}
  \end{columns}
  \footnotetext{如果想强制指定子文档的主文档,可以在文件第一行输入魔术命令:\texttt{\% !TeX root = main.tex}}
\end{frame}

\section{图}
\begin{frame}[fragile]%
  \frametitle{\temporal<5>{插图}{浮动体}{插图}}
  \begin{columns}
    \begin{column}{0.6\textwidth}
      \begin{codeblock}[]{插入单图\only<4->{最佳实践}}
\documentclass{ctexart}
|\only<2>{\highlightline}|\usepackage{graphicx}
|\only<2>{\highlightline}|\graphicspath{{figs/}{pics/}}
\begin{document}
|\only<5>{\highlightline}|\begin{figure}[ht]
|\only<6>{\highlightline}|  \centering
|\only<3>{\highlightline}|  \includegraphics[width=|\only<1-3>{4cm}\only<4->{0.4\textbackslash{}textwidth}|]{sjtug}
|\only<7>{\highlightline}|  \caption{SJTUG 徽标}\label{fig:sjtug}
|\only<5>{\highlightline}|\end{figure}
\end{document}
      \end{codeblock}
    \end{column}
    \begin{column}{0.4\textwidth}
      \only<1>{
        \includepdflarge{insertimage}
      }
      \only<2>{
        为了插入外部图片,需要使用 \pkg{graphicx} 宏包。之后在文档主体便可以使用 \cmd{includegraphics} 插入图片。导言区也可以加入 \cmd{graphicspath} 指定图片文件夹\footnotemark。
      }
      \only<3>{
        \cmd{includegraphics} 命令便以相对路径的方式插入图片,如果无同名图片,那么后缀名可以省略。可以使用可选参数指定插入的图片尺寸,最佳实践是使用 \cmd{textwidth} 或 \cmd{linewidth} 的相对值指定尺寸大小,以在未来可能的布局更改中保留一定的灵活性。
      }
      \only<4>{
        也可以通过键值对的方法设置图片的其他属性。
        \begin{center}
          \footnotesize
          \begin{stampbox}
            \begin{tabular}{rl}
              \pkg{width} & 宽度 \\
              \pkg{height} & 高度 \\
              \pkg{scale} & 缩放 \\
              \pkg{angle} & 角度 \\
            \end{tabular}
          \end{stampbox}
        \end{center}
      }
      \only<5>{
        \env{figure} 为一个浮动体环境(\env{table} 也是),可以将其移动到其他位置上以减少行文中的空白。可以添加可选参数以指定如何放置浮动体,最多可以使用四种位置描述符:
        \begin{center}
          \footnotesize
          \begin{stampbox}
            \begin{tabular}{cll}
              \pkg{h} & Here & 尽可能在这里 \\
              \pkg{t} & Top & 页面顶部 \\
              \pkg{b} & Bottom & 页面底部 \\
              \pkg{p} & Page & 浮动体专页 \\
            \end{tabular}
          \end{stampbox}
        \end{center}
        还可以只使用 \pkg{float} 宏包提供的 \pkg{H} 描述符以强制置于此处。
      }
      \only<6>{
        采用 \cmd{centering} 命令而不是 \env{center} 环境来水平居中图片。这将避免多余的纵向间距。
      }
      \only<7>{
        使用 \cmd{caption} 命令输入题注,如果这个命令写在插入图片前面,题注将会在上方(中文中一般对 \env{table} 环境这么做)。后面将会看到如何对留有标记(\cmd{label})的图片进行引用。
      }
    \end{column}
  \end{columns}
  \only<2>{\footnotetext{其命令参数每个为一个以 \texttt{/} 结尾的文件夹,每个文件夹需要使用大括号包裹起来。}}
\end{frame}

\begin{frame}[fragile]
  \begin{columns}
    \begin{column}{0.6\textwidth}
      \begin{codeblock}[]{插入双图}
\documentclass{ctexart}
\usepackage{graphicx}
\graphicspath{{figs/}{pics/}}
\begin{document}
  \begin{figure}[ht]
|\only<1>{\highlightline}|    \begin{minipage}{0.48\textwidth}
      \centering
      \includegraphics[height=2cm]{sjtug}
|\only<2>{\highlightline}|      \caption{SJTUG 徽标}\label{fig:sjtug}
|\only<1>{\highlightline}|    \end{minipage}\hfill
|\only<1>{\highlightline}|    \begin{minipage}{0.48\textwidth}
      \centering
      \includegraphics[height=2cm]{sjtugt}
|\only<2>{\highlightline}|      \caption{SJTUG||文字}\label{fig:sjtugt}
|\only<1>{\highlightline}|    \end{minipage}
  \end{figure}
\end{document}
      \end{codeblock}
    \end{column}
    \begin{column}{0.4\textwidth}
      \only<1>{
        在 \env{figure} 环境里使用 \env{minipage} 小页制作列盒子,以并排两图并分别编号,需要设定强制参数以指定列宽。两个小页之间添加 \cmd{hfill} 使两个小页两端对齐。
      }
      \only<2>{
        在每个小页内部分别使用 \cmd{caption} 添加标签。
      }
      \only<3>{
        \includepdflarge{doubleimages}
      }
    \end{column}
  \end{columns}
\end{frame}

\begin{frame}[fragile]%
  \begin{columns}
    \begin{column}{0.6\textwidth}
      \begin{codeblock}[]{}
\documentclass{ctexart}
\usepackage{graphicx}
|\highlightline|\usepackage{subcaption}
\graphicspath{{figs/}{pics/}}
\begin{document}
  \begin{figure}[ht]
|\highlightline|    \begin{subfigure}{0.48\textwidth}
      \centering
      \includegraphics[height=2cm]{sjtug}
      \caption{||徽标}
|\highlightline|    \end{subfigure}\hfill
|\highlightline|    \begin{subfigure}{0.48\textwidth}
      \centering
      \includegraphics[height=2cm]{sjtugt}
      \caption{||文字}
|\highlightline|    \end{subfigure}
    \caption{SJTUG}\label{fig:sjtug}
  \end{figure}
\end{document}
      \end{codeblock}
    \end{column}
    \begin{column}{0.4\textwidth}
      \includepdflarge{subfigures}\vspace{15pt}
      \pkg{subcaption} 宏包提供了 \env{subfigure} 环境(以及 \env{subtable}),可以用于以子图的形式插入,编号会增加一级。也可以为子图添加子集引用编号。
    \end{column}
  \end{columns}
\end{frame}

\section{表}
\begin{frame}[fragile]
  \frametitle{简单表格}
  \begin{columns}
    \begin{column}{0.6\textwidth}
      \begin{codeblock}[]{}
\documentclass{ctexart}
|\only<1-2>{\highlightline}|\usepackage{|\temporal<1>{array}{\highlight{array}}{array},\temporal<2>{booktabs}{\highlight{booktabs}}{booktabs}|}
\begin{document}
\begin{table}[ht]
  \centering
  \caption{||北京冬奥会吉祥物}
|\only<1>{\highlightline}|  \begin{tabular}{lp{3cm}}
|\only<2>{\highlightline}|    \toprule
|\only<3>{\highlightline}|吉祥物 & 描述                          \\
|\only<2>{\highlightline}|    \midrule
|\only<3>{\highlightline}|冰墩墩 & 2022 年北京冬季奥运会吉祥物  \\
|\only<3>{\highlightline}|雪容融 & 2022 年北京冬季残奥会吉祥物  \\
|\only<2>{\highlightline}|    \bottomrule
|\only<1>{\highlightline}|  \end{tabular}
\end{table}
\end{document}
      \end{codeblock}
    \end{column}
    \begin{column}{0.4\textwidth}
      \only<1>{
        使用 \env{tabular} 环境绘制表格。需要添加参数(称为\textbf{表格导言})以确定每一列的对齐方式。引入 \pkg{array} 宏包来使用更多的\textbf{引导符}。
        \begin{center}
          \footnotesize
          \begin{stampbox}
            \begin{tabular}{>{\ttfamily}ll}
              \alert{l} & 向左对齐 \\
              \alert{c} & 居中对齐 \\
              \alert{r} & 向右对齐 \\
              \alert{p\{3cm\}} & 固定列宽,两端对齐 \\
              \alert{m\{3cm\}} & \texttt{p} + 垂直居中对齐 \\
              \alert{>\{\textbackslash{}bfseries\}} & 后一列单元格前加命令 \\
              \alert{*\{3\}\{l\}} & 三个左对齐列 \\
            \end{tabular}
          \end{stampbox}
        \end{center}
      }
      \only<2>{
        \pkg{booktabs} 宏包提供了标准三线表格所需要的行分割线:\cmd{toprule},\cmd{midrule},\cmd{bottomrule}。也可以使用 \cmd{cmidrule\{1-2\}} 来部分地绘制行分割线。一般不推荐在专业表格中使用纵向分割线。
      }
      \only<3>{
        每行内容使用 \textbackslash\textbackslash{} 分隔开,每行中的单元格使用 \& 分隔开。
      }
      \only<4>{
        \includepdflarge{table}
      }
    \end{column}
  \end{columns}
\end{frame}

\begin{frame}[fragile]%
  \begin{columns}
    \begin{column}{0.6\textwidth}
      \begin{codeblock}[]{表头居中}
\documentclass{ctexart}
\usepackage{array,booktabs}
\begin{document}
\begin{table}[ht]
  \centering
  \caption{||北京冬奥会吉祥物}
  \begin{tabular}{lp{3cm}}
    \toprule
|\highlightline|\multicolumn{1}{c}{||吉祥物} &
|\highlightline|\multicolumn{1}{c}{||描述} \\
    \midrule
||冰墩墩 & 2022 年北京冬季奥运会吉祥物  \\
||雪容融 & 2022 年北京冬季残奥会吉祥物  \\
    \bottomrule
  \end{tabular}
\end{table}
\end{document}
      \end{codeblock}
    \end{column}
    \begin{column}{0.4\textwidth}
      \cmd{multicolumn} 命令不仅可以用于合并同行的单元格,还可以用于临时地屏蔽表格导言对该列的对齐方式定义。这里用于居中表头。
      \begin{center}
        \begin{stampbox}
          \parbox{0.85\linewidth}{
            \ttfamily\color{blue}\textbackslash{}multicolumn\{格数\}\{对齐方式\}\{内容\}
          }
        \end{stampbox}
      \end{center}
      跨页表格考虑使用 \pkg{longtable} 宏包。带标注的表格可以考虑使用 \pkg{threeparttable} 宏包。考虑使用在线工具生成表格代码 \link{https://www.tablesgenerator.com/latex_tables}。
    \end{column}
  \end{columns}
\end{frame}

\section{数学公式}
\begin{frame}
  \frametitle{数学模式}
  \begin{alertblock}{}
  输入数学公式是 \LaTeX{} 的绝对强项,很多常见的公式服务依赖于一些轻量级渲染引擎比如 MathJax, K\kern-.3ex\raise.4ex\hbox{\footnotesize A}\kern-.3ex\TeX{}。但是它们实际上使用的是 \LaTeX{} 语法变种,也就是说并没有使用 \LaTeX{} 后端。所以不要期望有完全一致的输出。
  \end{alertblock}
  
  为了更好的获得数学公式输入支持,请使用 \hologo{AmS}math 宏包。数学模式分为两种:
  \begin{description}
    \item[行内模式] 一般通过一对美元符号(\$$\cdots$\$)标记,可以使用编辑器内建的符号表输入数学符号,也可以使用在线工具手写识别 \link{https://detexify.kirelabs.org/classify.html}。
    \item[行间模式] 一般通过 \env{equation} 环境\footnote{这是有编号环境,其加星号的变种 \env{equation*} 用于生成无编号环境。}输入。如果需要使用多行公式,请使用 \env{align} 环境,并按照类似表格输入的方式,使用 \& 对齐符号,\textbackslash\textbackslash{} 换行。如果不想手动居中,可以考虑多行自动居中的 \env{gather} 和单个大型公式首尾两端对齐 \env{multline}。
  \end{description}
\end{frame}

\begin{frame}
  \frametitle{数学命令展示}
  \begin{columns}
    \begin{column}{0.33\textwidth}
      \begin{exampleblock}{}
        $PV=nRT$
      \end{exampleblock}
      \begin{exampleblock}{}
        $\sum_{i=1}^ki^2=\frac{n(n+1)(2n+1)}{6}$
      \end{exampleblock}
      \begin{exampleblock}{}
        $T(n) = aT\left(\left\lceil\frac{n}{b}\right\rceil\right) + \mathcal{O}(n^d)$
      \end{exampleblock}
      \begin{exampleblock}{}
        $\frac{x_{1}+x_{2}+x_{3}}{3}\geq \sqrt[3]{x_{1}x_{2}x_{3}}$
      \end{exampleblock}
      \begin{exampleblock}{}
        $n=(\underbrace{1\cdots 1}_{k\text{ of 1's}})_2=2^{k+1}-1$
      \end{exampleblock}
      \begin{exampleblock}{}
        $\nabla f (P)= \left.\left(\frac{\partial f}{\partial x},\frac{\partial f}{\partial y},\frac{\partial f}{\partial z}\right)\right|_{P}$
      \end{exampleblock}
    \end{column}
    \begin{column}{0.67\textwidth}
      \begin{exampleblock}{}
        \begin{equation*}
          \int_{a}^b f(x)\,\mathrm{d}x=\lim_{|P|\rightarrow 0}\sum_{i=1}^n f(\xi_i)\Delta x_i
        \end{equation*}
      \end{exampleblock}
      \begin{exampleblock}{}
        \begin{equation}
          T(n) = \begin{cases}
            \mathcal{O}(n^d),&\textrm{if } d>\log_b a, \\
            \mathcal{O}(n^d\log n), &\textrm{if } d=\log_b a,\\
            \mathcal{O}(n^{\log_b a}), &\textrm{if } d<\log_b a.
          \end{cases}
        \end{equation}
      \end{exampleblock}
      \begin{exampleblock}{}
        \begin{align}
          Q^{T}A&=R \\
          \begin{pmatrix}
            q_1^T \\ q_2^T \\ q_3^T
          \end{pmatrix}
          \begin{pmatrix}
            a_1 & a_2 & a_3
          \end{pmatrix}
          &=R
        \end{align}
      \end{exampleblock}
    \end{column}
  \end{columns}
\end{frame}

%更深入地讲解 mathtools, unicode-math, siunix

\section{引用}
\begin{frame}[fragile]
  \frametitle{交叉引用}
  \only<1>{
    正如之前所提到的,\LaTeX{} 中使用 \cmd{label} 标记,然后可以使用 \cmd{ref} 来引用这个标记。 \cmd{label} 可以放在使用计数器的对象之后。
  }
  \only<2>{
    为了使得对公式编号的引用带有括号,推荐使用 \hologo{AmS}math 宏包中的 \cmd{eqref} 命令。对于多行公式环境,每一个换行符前都可以添加一个 \cmd{label} 用于引用该行公式。
  }
  \begin{columns}
    \begin{column}{0.5\textwidth}
      \begin{codeblock}[]{图}
\begin{figure}
|\only<1>{\highlightline}|  \caption{||示例}\label{fig:example}
\end{figure}
      \end{codeblock}
      \begin{codeblock}[]{表}
\begin{table}
|\only<1>{\highlightline}|  \caption{||示例}\label{tab:example}
\end{table}
      \end{codeblock}
    \end{column}
    \begin{column}{0.5\textwidth}
\begin{codeblock}[]{目次}
|\only<1>{\highlightline}|\section{||示例}\label{sec:example}
\end{codeblock}

\begin{codeblock}[]{公式}
\begin{equation}
  a = b + c
|\only<1>{\highlightline}|\label{eq:example}
\end{equation}
|\only<2>{\highlightline}|如公式 \eqref{eq:example} 所示,
\end{codeblock}
    \end{column}
  \end{columns}
\end{frame}

\begin{frame}[fragile]
  \frametitle{文献引用}
  \LaTeX{} 管理参考文献可以采用专用数据库文件 \texttt{.bib},很多的文献管理文件比如 EndNote \link{https://lic.sjtu.edu.cn/Default/softshow/tag/MDAwMDAwMDAwMLGImKE}, Zotero \link{https://www.zotero.org/}, JabRef \link{https://www.jabref.org/} 都可以直接导出这种格式的文件用于 \LaTeX{} 论文中的引用。一般不需要手写数据库文件,你只需要注意每一个文献会在数据库中有一个主键,这个类似于上文的 \cmd{label} 标记,只是要引用数据库中的文献需要使用 \cmd{cite} 命令。
  
  \begin{codeblock}[]{ref.bib}
|\highlightline|@phdthesis{devoftech,|\hfill\alert{\% 类型为博士论文,主键为\texttt{devoftech}}|
  title={||新时期我国信息技术产业的发展},
  author={||江泽民},
  year={2008}
}
  \end{codeblock}
\end{frame}

\begin{frame}
  \frametitle{文献引用}
  而让 \LaTeX{} 处理 \texttt{.bib} 数据库文件与引用有两种工作流。你可能需要查清楚如何在编辑器中设置对应的工作流,或者采用后面所提到的高级编译方式 \texttt{latexmk}。
  \begin{columns}
    \begin{column}{0.5\textwidth}
      \begin{block}{\hologo{BibTeX} + \pkg{gbt7714}}
        一般期刊提交使用这种方法,\pkg{natbib} 宏包将提供命令 \cmd{citet}(文本引用) 和 \cmd{citep}(括号引用)。中文引用可以直接使用 \pkg{gbt7714} 宏包,但是角模式和正文模式不能混用。
      \end{block}
    \end{column}
    \begin{column}{0.5\textwidth}
      \begin{block}{\hologo{biber} + \pkg{biblatex}}
        这是更容易自定义的方法,与 \hologo{BibTeX} 的运作方式稍有不同。\pkg{biblatex} 提供了更加智能的引用命令。而中文引用可以使用 \pkg{biblatex} 宏包的样式 \pkg{gb7714-2015},使用该样式需要使用 \hologo{XeLaTeX} 编译。
      \end{block}
    \end{column}
  \end{columns}
\end{frame}

\begin{frame}[fragile]
  \frametitle{文献引用}
  \begin{columns}
    \begin{column}{0.5\textwidth}
      \begin{codeblock}[]{\hologo{BibTeX} + \pkg{gbt7714}}
\documentclass{ctexart}
\usepackage{gbt7714}
\bibliographystyle{gbt7714-numerial}
% \citestyle{numbers}  % 正文模式
\begin{document}
  ||他指出了...\cite{devoftech}
  \bibliography{ref}
\end{document}
      \end{codeblock}
    \end{column}
    \begin{column}{0.5\textwidth}
      \begin{codeblock}[]{\hologo{biber} + \pkg{biblatex}}
\documentclass{ctexart}
\usepackage[backend=biber,style=gb7714-2015]{biblatex}
\addbibresource{ref.bib}
\begin{document}
  ||他在文献 \parencite{devoftech}
  ||指出了...\cite{devoftech}
  \printbibliography
\end{document}
      \end{codeblock}
    \end{column}
  \end{columns}
\end{frame}

\begin{frame}
  \frametitle{文献引用}
  \begin{columns}
    \begin{column}{0.5\textwidth}
      \includepdflarge{bibtex}
    \end{column}
    \begin{column}{0.5\textwidth}
      \includepdflarge{biblatex}
    \end{column}
  \end{columns}
\end{frame}

} % End of customized shaded number logo

  % !TeX root = ..\..\latex-talk.tex

\part{SJTUThesis}

\begin{frame}
  \frametitle{简介}
  \begin{columns}
    \begin{column}{0.6\textwidth}
      \begin{itemize}
        \item 最早由韦建文于 2009 年 11 月发布 0.1a 版,2018 年起由 SJTUG 接手维护
        \item 最新版:\SJTUThesisVersion{} (\SJTUThesisDate)
        \item 支持本科、硕士、博士学位论文以及课程论文的排版
      \end{itemize}
    \end{column}
    \begin{column}{0.4\textwidth}
      \begin{exampleblock}{}
        \begin{minipage}[c]{1cm}
          \includegraphics[width=0.8cm]{\getcontribpath{sjtug}{vi/sjtug}}
        \end{minipage}
        \begin{minipage}[c]{2cm}
          \href{https://github.com/sjtug}{sjtug}/\href{https://github.com/sjtug/SJTUThesis}{SJTUThesis}
        \end{minipage}
      \end{exampleblock}
      \vspace{-8pt}
      \begin{block}{}
        \scriptsize
        上海交通大学 \hologo{XeLaTeX} 学位论文及课程论文模板 | Shanghai Jiao Tong University \hologo{XeLaTeX} Thesis Template
      \end{block}
      \vspace{-8pt}
      \begin{alertblock}{}
        \scriptsize
        \begin{tabular}{cl}
          \faStar & 2.4k \\
          \faEye & 55 \\
          \faCodeBranch & 701 \\
        \end{tabular}
      \end{alertblock}
    \end{column}
  \end{columns}
\end{frame}

\begin{frame}
  \frametitle{下载与编译}
  \alert{下载} 推荐安装 Git \link{https://git-scm.com/} 后,克隆 SJTUG 镜像仓库
  \begin{exampleblock}{\faGit*}
    \ttfamily\small
    git clone https://mirror.sjtu.edu.cn/git/SJTUThesis.git/
  \end{exampleblock}

  \alert{编译} 推荐使用 \pkg{latexmk} 编译\footnote{\hologo{MiKTeX} 用户需要手动安装 Perl 解释器 \link{https://www.perl.org/get.html} 才能使用 \pkg{latexmk}。},在不能够利用自带的 \texttt{.latexmkrc} 配置文件的情况下,需要查清楚在对应的编辑器中如何使用 \hologo{XeLaTeX} + \hologo{biber} 编译 \link{https://github.com/sjtug/SJTUThesis/blob/master/README.md}。
  \begin{exampleblock}{\faTerminal}
    \ttfamily\small
    latexmk -xelatex main
  \end{exampleblock}

  Overleaf 用户可以下载压缩包后,上传并采用 \hologo{XeLaTeX} 编译方式。
\end{frame}

\begin{frame}
  \frametitle{手动编译}
  \alert{第一次编译失败} 如果没有办法通过通常方式编译成功,请尝试使用文件夹内附带 \faLinux{}\,\faApple{} \texttt{Makefile} 和 \faWindows{} \texttt{Compile.bat} 进行编译。

  \alert{统计字数} 编写过程中也可以使用对应的命令调用 \TeX{}count 来统计正文字数。
  \begin{columns}
    \begin{column}{0.38\textwidth}
      \begin{exampleblock}{\faLinux{}\,\faApple}
        \ttfamily
        make all\\
        make clean\\
        make cleanall\\
        make wordcount
      \end{exampleblock}
    \end{column}
    \begin{column}{0.38\textwidth}
      \begin{exampleblock}{\faWindows}
        \ttfamily
        ./Compile.bat thesis\\
        ./Compile.bat clean\\
        ./Compile.bat cleanall\\
        ./Compile.bat wordcount
      \end{exampleblock}
    \end{column}
    \begin{column}{0.24\textwidth}
      \begin{block}{\faInfo}
        \ttfamily
        编译论文\\
        清理中间文件\\
        $\hookrightarrow +$删除论文\\
        统计字数
      \end{block}
    \end{column}
  \end{columns}
\end{frame}

\begin{frame}[label=compile]
  \frametitle{编译问题排查}
  \begin{columns}
    \begin{column}{0.33\textwidth}
      \begin{alertblock}{无法使用 \texttt{latexmk}\thesisissue{578}}
        \hologo{MiKTeX} 需要安装 Perl 解释器。
      \end{alertblock}  
      \begin{alertblock}{C\TeX{} 套装无法编译\thesisissue{446}}
        使用最新 \TeX{} 发行版。
      \end{alertblock}
      \begin{alertblock}{\hologo{pdfLaTeX} 无法编译\thesisissue{444}}
        请使用 \texttt{latexmk},或更改编辑器设置以 \hologo{XeLaTeX} 编译。
      \end{alertblock}
    \end{column}
    \begin{column}{0.33\textwidth}
      \begin{alertblock}{缺少字体\thesisissue{564} \thesisdiscuss{598}}
        更换字体集,或者安装对应字体。
      \end{alertblock}
      \begin{alertblock}{缺少汉字\thesisissue{533} \thesisdiscuss{617}}
        去除使用 fandol 字体集的设定。或者是安装字体后,改用 \texttt{fontset=adobe} 或 \texttt{fontset=founder}。
      \end{alertblock}
    \end{column}
    \begin{column}{0.33\textwidth}
      \begin{block}{\faInfoCircle{} README}
        不同编辑器的设置请首先参阅 README \link{https://github.com/sjtug/SJTUThesis/blob/master/README.md} 文档。
      \end{block}
      \begin{block}{\faBookOpen{} Wiki}
        其他编译问题推荐查阅 Wiki \link{https://github.com/sjtug/SJTUThesis/wiki} 的使用说明部分。
      \end{block}
    \end{column}
  \end{columns}
\end{frame}

\begin{frame}[fragile, label=covers]
  \begin{codeblock}[firstnumber=3]{main.tex}
|\alert{\% 载入 SJTUThesis 模版}|
\documentclass[|\only<1>{\highlight{type}}\only<2>{type}|=|\only<1>{bachelor}\only<2>{\highlight{bachelor}}|]{sjtuthesis}
  \end{codeblock}
  \begin{figure}
    \parbox{0.9\textwidth}{
      \begin{subfigure}{0.20\textwidth}
        \framebox{\includegraphics[width=\linewidth]{support/thesis/bachelor}}
        \caption{\only<1>{学士}\only<2>{\texttt{bachelor}}}
      \end{subfigure}\hfill
      \begin{subfigure}{0.20\textwidth}
        \framebox{\includegraphics[width=\linewidth]{support/thesis/master}}
        \caption{\only<1>{硕士}\only<2>{\texttt{master}}}
      \end{subfigure}\hfill
      \begin{subfigure}{0.20\textwidth}
        \framebox{\includegraphics[width=\linewidth]{support/thesis/doctor}}
        \caption{\only<1>{博士}\only<2>{\texttt{doctor}}}
      \end{subfigure}\hfill
      \begin{subfigure}{0.20\textwidth}
        \framebox{\includegraphics[width=\linewidth]{support/thesis/course}}
        \caption{\only<1>{课程}\only<2>{\texttt{course}}}
      \end{subfigure}
      \caption{论文类型示例\only<2>{ \texttt{type}}}
    }
  \end{figure}
\end{frame}

\begin{frame}[fragile]
  \frametitle{文档类选项}
  % \framesubtitle{\textbackslash{}documentclass\{sjtuthesis\}}
  \begin{columns}
    \begin{column}{0.45\textwidth}
      \includegraphics[page=10]{thesisdir}
    \end{column}
    \begin{column}{0.55\textwidth}
      \begin{table}[H]
        \caption{文档类选项}
        \footnotesize
        \begin{tabular}{>{\ttfamily}rll}
          \toprule
          选项 & 含义 & 相关 \\
          \midrule
          type= & 指定论文类型 & 第 \ref{covers} 页\\
          fontset= & 指定字体 & 第 \ref{compile} 页\\
          \midrule
          review & 开启盲审模式 & \thesisissue{195} \thesisissue{686} \\
          twoside & 双页模式 & \thesisissue{554} \\
          oneside & 单页模式 & \thesisissue{694} \\
          openright & 章从奇数页开始 & \thesisdiscuss{724} \\
          openany & 章从任意页开始 & \thesisissue{446} \\
          \bottomrule
        \end{tabular}
      \end{table}
    \end{column}
  \end{columns}
\end{frame}

\begin{frame}[fragile]
  \frametitle{基本配置}
  \framesubtitle{\textbackslash{}input\{setup\}}
  \begin{columns}
    \begin{column}{0.45\textwidth}
      \includegraphics[page=9]{thesisdir}
    \end{column}
    \begin{column}{0.55\textwidth}
      \begin{codeblock}[firstnumber=12]{main.tex}
|\highlightline<1>|% 论文基本配置,加载宏包等全局配置
|\highlightline<1>|\input{setup}

\begin{document}

%TC:ignore

|\highlightline<2>|% 标题页
|\highlightline<2>|\maketitle
      \end{codeblock}
      \visible<2>{
        \cmd{sjtusetup} 中的 \pkg{info} 将会修改封面的信息设置(见第 \ref{covers} 页)。
      }
    \end{column}
  \end{columns}
\end{frame}

\begin{frame}[fragile]
  \frametitle{基本配置}
  \framesubtitle{\textbackslash{}sjtusetup}
  \begin{columns}
    \begin{column}{0.45\textwidth}
      \includegraphics[page=1]{thesisdir}
    \end{column}
    \begin{column}{0.55\textwidth}
      \begin{codeblock}[firstnumber=3]{setup.tex}
\sjtusetup{
  info = {
    title    = {||上海交通大学学位论文 \LaTeX{} 模板示例文档},
    title*   = {A Sample for \LaTeX-based SJTU Thesis Template},
    author   = {||某\quad{}某},
    author* = {Mo Mo},
  },
  style = { header-logo-color = red, 
  },
  name = {
    publications = {||攻读学位期间完成的论文},
  },
}
      \end{codeblock}
    \end{column}
  \end{columns}
\end{frame}

\begin{frame}
  \frametitle{基本配置}
  \framesubtitle{\textbackslash{}sjtusetup}
  \begin{columns}
    \begin{column}{0.45\textwidth}
      \includegraphics[page=1]{thesisdir}
    \end{column}
    \begin{column}{0.55\textwidth}
      \begin{table}[H]
        \centering
        \caption{info 域}
        \footnotesize
        \begin{tabular}{lll} \toprule
          命令作用 & 中文对应选项 & 英文对应选项 \\ \midrule
          论文标题 & \texttt{title} & \texttt{title*} \\
          关键字列表 & \texttt{keywords} & \texttt{keywords*} \\
          作者姓名&  \texttt{author} &\texttt{author*}\\
          申请学位名称 & \texttt{degree}&\texttt{degree*}\\
          院系名称 & \texttt{department} & \texttt{department*}\\
          专业名称 & \texttt{major} & \texttt{major*}\\
          导师 & \texttt{supervisor} & \texttt{supervisor*}\\
          副导师 & \texttt{assisupervisor} & \texttt{assisupervisor*}\\
          日期 & \multicolumn{2}{c}{\texttt{date}}\\
          学号 & \multicolumn{2}{c}{\texttt{id}}\\ \bottomrule
          \end{tabular}
      \end{table}
    \end{column}
  \end{columns}
\end{frame}

\begin{frame}[fragile]
  \frametitle{版权页}
  \framesubtitle{\textbackslash{}copyrightpage}
  \begin{columns}
    \begin{column}{0.45\textwidth}
      \only<1>{
        \includegraphics[page=9]{thesisdir}
      }
      \only<2>{
        \includegraphics[page=2]{thesisdir}
      }
      \only<3>{
        \begin{figure}[H]
          \framebox{\includegraphics[page=2,width=0.4\linewidth]{bachelor}}
          \caption{版权页}
        \end{figure}
      }
    \end{column}
    \begin{column}{0.55\textwidth}
      \begin{codeblock}[firstnumber=22]{main.tex}
|\highlightline<1>|% 原创性声明及使用授权书
|\highlightline<1>|\copyrightpage
|\highlightline<2>|% 插入外置原创性声明及使用授权书
|\highlightline<2>|% \copyrightpage[scans/sample-copyright-old.pdf]
      \end{codeblock}
      \only<1>{
        \cmd{copyrightpages} 可以用于插入版权页。
      }
      \only<2>{
        \cmd{copyrightpages} 也接受一个可选参数,用于直接使用扫描件。
      }
    \end{column}
  \end{columns}
\end{frame}

\begin{frame}[fragile]
  \frametitle{前置部分}
  \framesubtitle{\textbackslash{}frontmatter}
  \begin{columns}
    \begin{column}{0.45\textwidth}
      \only<1>{
        \includegraphics[page=9]{thesisdir}
      }
      \only<2>{
        \includegraphics[page=3]{thesisdir}
      }
      \only<3>{
        \begin{figure}[H]
          \begin{subfigure}{0.45\textwidth}
            \framebox{\includegraphics[page=3,width=\linewidth]{bachelor}}
            \caption{中文}
          \end{subfigure}\hfill
          \begin{subfigure}{0.45\textwidth}
            \framebox{\includegraphics[page=4,width=\linewidth]{bachelor}}
            \caption{英文}
          \end{subfigure}
          \caption{摘要}
        \end{figure}
      }
      \only<4>{
        \begin{figure}[H]
          \begin{subfigure}{0.30\linewidth}
            \centering
            \framebox{\includegraphics[page=5,width=0.6\linewidth]{bachelor}}
            \caption{目录}
          \end{subfigure}
          \begin{subfigure}{0.30\linewidth}
            \centering
            \framebox{\includegraphics[page=6,width=0.6\linewidth]{bachelor}}
            \caption{插图}
          \end{subfigure}

          \begin{subfigure}{0.30\linewidth}
            \centering
            \framebox{\includegraphics[page=7,width=0.6\linewidth]{bachelor}}
            \caption{表格}
          \end{subfigure}
          \begin{subfigure}{0.30\linewidth}
            \centering
            \framebox{\includegraphics[page=8,width=0.6\linewidth]{bachelor}}
            \caption{算法}
          \end{subfigure}
          \caption{索引}
        \end{figure}
      }
      \only<5>{
        \includegraphics[page=4]{thesisdir}
      }
      \only<6>{
        \begin{figure}[H]
          \framebox{\includegraphics[page=9,width=0.5\linewidth]{bachelor}}
          \caption{符号对照表}
        \end{figure}
      }
    \end{column}
    \begin{column}{0.55\textwidth}
      \begin{codeblock}[firstnumber=30]{main.tex}
|\highlightline<2-3>|% 摘要
|\highlightline<2-3>|\input{contents/abstract}

|\highlightline<4>|% 目录
|\highlightline<4>|\tableofcontents
|\highlightline<4>|% 插图索引
|\highlightline<4>|\listoffigures*
|\highlightline<4>|% 表格索引
|\highlightline<4>|\listoftables*
|\highlightline<4>|% 算法索引
|\highlightline<4>|\listofalgorithms*

|\highlightline<5-6>|% 符号对照表
|\highlightline<5-6>|\input{contents/nomenclature}
      \end{codeblock}
    \end{column}
  \end{columns}
\end{frame}

\begin{frame}[fragile]
  \frametitle{主体部分}
  \framesubtitle{\textbackslash{}mainmatter}
  \begin{columns}
    \begin{column}{0.45\textwidth}
      \only<1>{
        \includegraphics[page=5]{thesisdir}
      }
      \only<2>{
        \begin{figure}[H]
          \begin{subfigure}{0.30\linewidth}
            \centering
            \framebox{\includegraphics[page=11,width=0.6\linewidth]{bachelor}}
            \caption{简介}
          \end{subfigure}
          \begin{subfigure}{0.30\linewidth}
            \centering
            \framebox{\includegraphics[page=13,width=0.6\linewidth]{bachelor}}
            \caption{数学}
          \end{subfigure}

          \begin{subfigure}{0.30\linewidth}
            \centering
            \framebox{\includegraphics[page=16,width=0.6\linewidth]{bachelor}}
            \caption{浮动体}
          \end{subfigure}
          \begin{subfigure}{0.30\linewidth}
            \centering
            \framebox{\includegraphics[page=22,width=0.6\linewidth]{bachelor}}
            \caption{总结}
          \end{subfigure}
          \caption{主体部分}
        \end{figure}
      }
    \end{column}
    \begin{column}{0.55\textwidth}
      \begin{codeblock}[firstnumber=47]{main.tex}
|\highlightline|% 正文内容
|\highlightline|% !TeX root = ../../../latex-talk.tex

\section{是什么}

\begin{frame}
  \frametitle{\TeX{}}
  \begin{columns}[c]
    \begin{column}{0.7\textwidth}
      \begin{center}
        \rmfamily\Huge
        \highlight[structure]{\TeX{}}
      \end{center}
      \begin{center}
        \parbox{0.75\textwidth}{
          \TeX{} 是由斯坦福大学教授高德纳
          (Donald E.~Knuth)于 1977 年开始开发的排版引擎。目前仍在更新,最新版本号为 3.141592653 \link{https://tug.org/TUGboat/tb42-1/tb130knuth-tuneup21.pdf}。
        }
      \end{center}
    \end{column}
    \begin{column}{0.3\textwidth}
      \includegraphics[width=.8\columnwidth]{support/images/Knuth.jpg}
    \end{column}
  \end{columns}
  \note{\emph{这一部分背景介绍大家可以了解一下,暂时跳过。}
  \LaTeX{} 这个词由两个部分组成,\hologo{La} 和 \TeX{}。那我们首先了解一下 \TeX{} 是什么。
  \TeX{} 是由斯坦福大学的教授高德纳于 1977 年开始开发的排版引擎,它已经有三十多年的历史了,
  目前仍在更新,版本号(3.141592653)将会趋近于 $\pi$ 的取值,高德纳最近还在给 \textsl{TUGBoat} 写稿子
  \link{https://tug.org/TUGboat/tb42-1/tb130knuth-tuneup21.pdf},
  关于 \TeX{} 今年又做了哪些改进。}
\end{frame}

\begin{frame}
  \frametitle{\LaTeX{}}
  \begin{columns}[c]
    \begin{column}{0.7\textwidth}
      \begin{center}
        \rmfamily\Huge
        \highlight[structure]{\LaTeX{}}
      \end{center}
      \begin{center}
        \parbox{0.75\textwidth}{
          \LaTeX{} 是最早在 1985 年由现就职于微软的 Leslie Lamport 开发的一种 \TeX{} \textbf{格式}\footnotemark,使用一些列宏和扩展宏包来简化 \TeX{} 的使用。现在由 \LaTeX{} Project 的成员维护。现在广泛使用的版本是 \LaTeXe{},最新的版本为 \LaTeX3(2020 年 10 月后默认内置)。
        }
      \end{center}
    \end{column}
    \begin{column}{0.3\textwidth}
      \includegraphics[width=.8\columnwidth]{support/images/Lamport.jpg}
    \end{column}
  \end{columns}
  \footnotetext{\hologo{ConTeXt} 也是一种 \TeX{} 格式 \link{https://www.contextgarden.net/}。}
  \note{\emph{这一部分的背景介绍大家可以了解一下,暂时跳过。}
  \LaTeX{} 是最早由现就职于微软的 Leslie Lamport 开发的一种 \TeX{} 格式(与其对标的是
  \hologo{ConTeXt}\link{https://www.contextgarden.net/}),主要也是为了简化 \TeX{} 的使用。
  现在主要由 \LaTeX{} 开发组维护,现在广泛使用的版本是 \LaTeXe{},最新的版本为 \LaTeX3,
  在 2020 年 10 月后默认内置,所以要尽可能使用较新的发行版,以充分发挥其功能。}
\end{frame}

\begin{frame}
  \frametitle{程序}
  \begin{columns}[c]
    \begin{column}{0.7\textwidth}
      \begin{center}
        \rmfamily\Huge
        \highlight[structure]{\hologo{pdfLaTeX}}
      \end{center}
      \begin{center}
        \parbox{0.7\textwidth}{
          \hologo{pdfLaTeX} 是为了编译一个 \LaTeX{} 文档而运行的程序。实际上底层在运行一个叫 \hologo{pdfTeX} 的引擎,并预装了对应的 \LaTeX{} \textbf{格式}。为了利用临时文件,可能就需要多次运行程序。
        }
      \end{center}
    \end{column}
    \begin{column}{0.3\textwidth}
      \begin{block}{}
        \ttfamily\small
        > \highlight{pdflatex} main.tex\\
        This is pdfTeX, Version 3.141592653-
        2.6-1.40.23 (MiKTeX 21.10)\\
        entering extended mode\\
        \highlight{LaTeX2e} <2021-11-15>\\
        \highlight{L3} programming layer <2021-11-22>
      \end{block}
    \end{column}
  \end{columns}
  \note{\hologo{pdfLaTeX} 是为了编译一个 \LaTeX{} 文档而运行的程序。}
\end{frame}

% \begin{frame}
%   \frametitle{引擎}
%   \begin{columns}[c]
%     \begin{column}{0.7\textwidth}
%       \begin{center}
%         \rmfamily\Huge
%         \highlight[structure!70]{pdf}\hologo{La}\highlight[structure!70]{\TeX{}}
%       \end{center}
%       \begin{center}
%         \parbox{0.7\textwidth}{
%           pdf\TeX{} 是编译 \TeX{} 文档(以 \texttt{.tex} 结尾)的\textbf{引擎}---可以理解 \TeX{} 指令的\textbf{程序}。
%         }
%       \end{center}
%     \end{column}
%     \begin{column}{0.3\textwidth}
%       \begin{block}{}
%         \ttfamily\small
%         > pdflatex main.tex\\
%         This is \highlight[structure!70]{pdfTeX}, Version 3.141592653-
%         2.6-1.40.23 (MiKTeX 21.10)
%         entering extended mode\\
%         LaTeX2e <2021-11-15>\\
%         L3 programming layer <2021-11-22>
%       \end{block}
%     \end{column}
%   \end{columns}
%   \note{实际上底层在运行一个叫 \hologo{pdfTeX} 的引擎,并预装了对应的 \LaTeX{} 格式。}
% \end{frame}

\begin{frame}[label={frame:engine}]
  \frametitle{程序}
  \begin{table}
    \caption{主流 \hologo{(La)TeX} 程序
    \footnote{(u)p\TeX{} 是日语最常用的引擎,生成 \texttt{.dvi},支持 Unicode。}\footnote{Ap\TeX{} \link{https://github.com/clerkma/ptex-ng} 具有底层 CJK 支持,内联 Ruby,Color Emoji。}}
    \footnotesize
    \begin{stampbox}
      \begin{tabular}{c>{\raggedright}*{3}{p{3.5cm}}}
        \alert{引擎}     & \hologo{pdfTeX}   & \hologo{XeTeX}   & \hologo{LuaTeX}   \\
        \alert{程序}     & \hologo{pdfLaTeX} & \hologo{XeLaTeX} & \hologo{LuaLaTeX} \\
        \alert{特点}     & 直接生成 PDF,支持 micro-typography  & 支持 Unicode、OpenType 与复杂文字编排 (CTL) & 支持 Unicode,内联 Lua,支持 OpenType \\
      \end{tabular}
    \end{stampbox}
  \end{table}

  \begin{center}
    \parbox{.9\textwidth}{
      \hologo{pdfLaTeX} 不支持 Unicode。为了排版中文,大部分情况下应当使用 \hologo{XeLaTeX},而 \hologo{LuaLaTeX} 速度相对较慢。\faWindows{} 可以在一些情况下使用 \hologo{pdfLaTeX}。
    }
  \end{center}
  \note{当然为了排版中文,已经不再推荐使用 \hologo{pdfLaTeX} 了,应该使用
  \hologo{XeLaTeX} 或者 \hologo{LuaLaTeX},当然后者的速度还是相对较慢,
  它们支持 Unicode 编码,并可以使用 OpenType 字体的全部功能。
  当然 \faWindows{} 平台下在某些追求速度的情况下,
  还是可以试着使用 \hologo{pdfLaTeX} 的。

  \hologo{LuaLaTeX} 理想情况下不慢,但是使用一些宏包后会破坏理想状态,
  也会因配置产生不同的结果,不同的操作系统在 I/O 速度上的不同也会导致不同的时间。

  \hologo{pdfLaTeX} 也支持,只不过需要先生成 tfm \TeX{} 字体度量文件,后续使用 \TeX{}
  自身的配置方法,只能使用 7 比特或 8 比特字体。}
\end{frame}

% \begin{frame}
%   \paragraph{\hologo{pdfLaTeX}} \TeX{} 和 \LaTeX{} 被广泛使用之前,它们只需内置支持欧洲语言即可。在 Unicode 出现之前,\LaTeX{} 提供了许多种\textbf{文件编码}来允许很多语言的文字以原生的方式输入,\hologo{pdfLaTeX} 也只需要使用 8 位文件编码和 8 位字体。
% \end{frame}


|\highlightline|\input{contents/math_and_citations}
|\highlightline|\input{contents/floats}
|\highlightline|\input{contents/summary}

%TC:ignore

% 参考文献
\printbibliography[heading=bibintoc]
      \end{codeblock}
    \end{column}
  \end{columns}
\end{frame}

\begin{frame}
  \frametitle{数学}
  \begin{itemize}
    \item 公式示例:\nolinkurl{contents/math_and_citations.tex}
    \item \SJTUThesis{} 定义了常用的数学环境(需要手工引入 \texttt{ntheorem} 宏包):
      \begin{table}[h]
        \centering
        \footnotesize
        \begin{tabular}{*{7}{l}}\toprule
          assumption  & axiom   & conjecture & corollary    & definition  & example   & exercise  \\
          假设        & 公理    & 猜想       & 推论         & 定义        & 例        & 练习      \\\midrule
          lemma       & problem & proof      & proposition  & remark      & solution  & theorem   \\
          引理        & 问题    & 证明       & 命题         & 注          & 解        & 定理      \\\bottomrule
        \end{tabular}
      \end{table}
      \item \SJTUThesis{} 可以通过 \texttt{unimath} 选项使用 \pkg{unicode-math} 进行数学输入,注意与传统方式的区别。\thesisissue{555}
  \end{itemize}
\end{frame}

\begin{frame}[fragile]
  \frametitle{参考文献}
  \begin{columns}
    \begin{column}{0.45\textwidth}
      \includegraphics[page=6]{thesisdir}
    \end{column}
    \begin{column}{0.55\textwidth}
      \begin{codeblock}[firstnumber=111,numbersep=2pt]{setup.tex}
% 使用 BibLaTeX 处理参考文献
%   biblatex-gb7714-2015 常用选项
%     gbnamefmt=lowercase     姓名大小写由输入信息确定
%     gbpub=false             禁用出版信息缺失处理
\usepackage[backend=biber,style=gb7714-2015]{biblatex}
% 文献表字体
% \renewcommand{\bibfont}{\zihao{-5}}
% 文献表条目间的间距
\setlength{\bibitemsep}{0pt}
|\highlightline|% 导入参考文献数据库
|\highlightline|\addbibresource{bibdata/thesis.bib}
      \end{codeblock}
    \end{column}
  \end{columns}
\end{frame}

\begin{frame}[fragile]
  \frametitle{附录}
  \framesubtitle{\textbackslash{}appendix}
  \begin{columns}
    \begin{column}{0.45\textwidth}
      \only<1>{
        \includegraphics[page=7]{thesisdir}
      }
      \only<2>{
        \begin{figure}[H]
          \begin{subfigure}{0.45\linewidth}
            \framebox{\includegraphics[width=\linewidth,page=24]{bachelor}}
            \caption{}
          \end{subfigure}\hfill
          \begin{subfigure}{0.45\textwidth}
            \framebox{\includegraphics[width=\linewidth,page=25]{bachelor}}
            \caption{}
          \end{subfigure}
          \caption{附录}
        \end{figure}
      }
    \end{column}
    \begin{column}{0.55\textwidth}
      \begin{codeblock}[firstnumber=61]{main.tex}
% 附录中图表不加入索引
\captionsetup{list=no}

% 附录内容
|\highlightline|\input{contents/app_maxwell_equations}
|\highlightline|\input{contents/app_flow_chart}
      \end{codeblock}
    \end{column}
  \end{columns}
\end{frame}

\begin{frame}[fragile]
  \frametitle{结尾部分}
  \framesubtitle{\textbackslash{}backmatter}
  \begin{columns}
    \begin{column}{0.45\textwidth}
      \only<1>{
        \includegraphics[page=8]{thesisdir}
      }
      \only<2>{
        \begin{figure}[H]
          \begin{subfigure}{0.30\linewidth}
            \centering
            \framebox{\includegraphics[page=26,width=0.6\linewidth]{bachelor}}
            \caption{致谢}
          \end{subfigure}
          \begin{subfigure}{0.30\linewidth}
            \centering
            \framebox{\includegraphics[page=27,width=0.6\linewidth]{bachelor}}
            \caption{成就}
          \end{subfigure}

          \begin{subfigure}{0.30\linewidth}
            \centering
            \framebox{\includegraphics[page=28,width=0.6\linewidth]{bachelor}}
            \caption{简历}
          \end{subfigure}
          \begin{subfigure}{0.30\linewidth}
            \centering
            \framebox{\includegraphics[page=29,width=0.6\linewidth]{bachelor}}
            \caption{大摘要*}
          \end{subfigure}
          \caption{结尾部分}
        \end{figure}
      }
    \end{column}
    \begin{column}{0.55\textwidth}
      \begin{codeblock}[firstnumber=76]{main.tex}
% 致谢
\input{contents/acknowledgements}

% 发表论文及科研成果
% 盲审论文中,发表论文及科研成果等仅以第几作者注明即可,不要出现作者或他人姓名
\input{contents/achievements}

% 简历
\input{contents/resume}

% 学士学位论文要求在最后有一个大摘要,单独编页码
\input{contents/digest}
      \end{codeblock}
    \end{column}
  \end{columns}
\end{frame}

\begin{frame}
  \frametitle{还有其他问题?}
  \begin{columns}
    \begin{column}{0.75\textwidth}
    \begin{itemize}
      \item[{\faComment*[regular]}] 日常模板或 \LaTeX{} 使用问题可以前往 Discussions \link{https://github.com/sjtug/SJTUThesis/discussions} 提问
      
      (解决后别忘了 \textcolor{green}{\faCheckCircle{} Mark as answer}
      \item[{\faDotCircle[regular]}] 如果是 \textsc{SJTUThesis} 项目本身的 bug 和 feature request
      
      可以通过 Issues \link{https://github.com/sjtug/SJTUThesis/issues} 反馈。
      \item[{\faCodeBranch}] 如果你有好点子,可以贡献代码
     
      向 \textsc{SJTU\TeX{}}(v1) \link{https://github.com/sjtug/SJTUTeX/tree/v1} 存储库发 PR,\par
      而后把解包结果同步到 \textsc{SJTUThesis}。
  
      \item[{\faTag}] 如果你对正在基于 \LaTeX3 开发的新版感兴趣,\par
      也欢迎向 \textsc{SJTU\TeX{}}(v2) \link{https://github.com/sjtug/SJTUTeX/tree/v2} 发 PR。
  
      \item[{\faQq}] 也欢迎在 QQ 群即时讨论。
    \end{itemize}
    \end{column}
    \begin{column}{0.25\textwidth}
      \includegraphics[height=0.7\textheight]{qq.jpg}
    \end{column}
  \end{columns}
\end{frame}
\end{document}
      \end{codeblock}
    \end{column}
  \end{columns}
  \footnotetext{如果想强制指定子文档的主文档,可以在文件第一行输入魔术命令:\texttt{\% !TeX root = main.tex}}
\end{frame}

\section{图}
\begin{frame}[fragile]%
  \frametitle{\temporal<5>{插图}{浮动体}{插图}}
  \begin{columns}
    \begin{column}{0.6\textwidth}
      \begin{codeblock}[]{插入单图\only<4->{最佳实践}}
\documentclass{ctexart}
|\highlightline<2>|\usepackage{graphicx}
|\highlightline<2>|\graphicspath{{figs/}{pics/}}
\begin{document}
|\highlightline<5>|\begin{figure}[ht]
|\highlightline<6>|  \centering
|\highlightline<3>|  \includegraphics[width=|\only<1-3>{4cm}\only<4->{0.4\textbackslash{}textwidth}|]{sjtug}
|\highlightline<7>|  \caption{SJTUG 徽标}\label{fig:sjtug}
|\highlightline<5>|\end{figure}
\end{document}
      \end{codeblock}
    \end{column}
    \begin{column}{0.4\textwidth}
      \only<1>{
        \includepdflarge{support/examples/insertimage.pdf}
      }

      \only<2>{
        为了插入外部图片,需要使用 \pkg{graphicx} 宏包。之后在文档主体便可以使用 \cmd{includegraphics} 插入图片。导言区也可以加入 \cmd{graphicspath} 指定图片文件夹\footnotemark。
      }

      \only<3>{
        \cmd{includegraphics} 命令便以相对路径的方式插入图片,如果无同名图片,那么后缀名可以省略。可以使用可选参数指定插入的图片尺寸,最佳实践是使用 \cmd{textwidth} 或 \cmd{linewidth} 的相对值指定尺寸大小,以在未来可能的布局更改中保留一定的灵活性。
        \note{比如我未来想变更为幻灯片的时候。}
      }

      \only<4>{
        也可以通过键值对的方法设置图片的其他属性。
        \note{事实上,\LaTeX{} 很多命令都是使用方括号添加可选参数的。}
        \begin{center}
          \footnotesize
          \begin{tabular}{rl}
            \opt{width} & 宽度 \\
            \opt{height} & 高度 \\
            \opt{scale} & 缩放 \\
            \opt{angle} & 角度 \\
          \end{tabular}
        \end{center}
      }

      \only<5>{
        \env{figure} 为一个浮动体环境(\env{table} 也是),可以将其移动到其他位置上以减少行文中的空白。可以添加可选参数以指定如何放置浮动体,最多可以使用四种位置描述符:
        \begin{center}
          \footnotesize
          \begin{tabular}{cll}
            \opt{h} & Here & 尽可能在这里 \\
            \opt{t} & Top & 页面顶部 \\
            \opt{b} & Bottom & 页面底部 \\
            \opt{p} & Page & 浮动体专页 \\
          \end{tabular}
        \end{center}
        还可以只使用 \pkg{float} 宏包提供的 \opt{H} 描述符以强制置于此处。
      }

      \only<6>{
        采用 \cmd{centering} 命令而不是 \env{center} 环境来水平居中图片。这将避免多余的纵向间距。
      }

      \only<7>{
        使用 \cmd{caption} 命令输入题注,如果这个命令写在插入图片前面,题注将会在上方(中文中一般对 \env{table} 环境这么做)。后面将会看到如何对留有标记(\cmd{label})的图片进行引用。
      }
    \end{column}
  \end{columns}
  \only<2>{\footnotetext{其命令参数每个为一个以 \texttt{/} 结尾的文件夹,每个文件夹需要使用大括号包裹起来。}}
\end{frame}

\begin{frame}[fragile]
  \begin{columns}
    \begin{column}{0.6\textwidth}
      \begin{codeblock}[]{插入双图}
\documentclass{ctexart}
\usepackage{graphicx}
\graphicspath{{figs/}{pics/}}
\begin{document}
  \begin{figure}[ht]
|\highlightline<1>|    \begin{minipage}{0.48\textwidth}
      \centering
      \includegraphics[height=2cm]{sjtug}
|\highlightline<2>|      \caption{SJTUG 徽标}\label{fig:sjtug}
|\highlightline<1>|    \end{minipage}\hfill
|\highlightline<1>|    \begin{minipage}{0.48\textwidth}
      \centering
      \includegraphics[height=2cm]{sjtugt}
|\highlightline<2>|      \caption{SJTUG|\phantom{}|文字}\label{fig:sjtugt}
|\highlightline<1>|    \end{minipage}
  \end{figure}
\end{document}
      \end{codeblock}
    \end{column}
    \begin{column}{0.4\textwidth}

      \only<1>{
        在 \env{figure} 环境里使用 \env{minipage} 小页制作列盒子,以并排两图并分别编号,需要设定强制参数以指定列宽。两个小页之间添加 \cmd{hfill} 使两个小页两端对齐。
      }

      \only<2>{
        在每个小页内部分别使用 \cmd{caption} 添加标签。
      }

      \only<3>{
        \includepdflarge{support/examples/doubleimages.pdf}
      }
    \end{column}
  \end{columns}
\end{frame}

\begin{frame}[fragile]%
  \begin{columns}
    \begin{column}{0.6\textwidth}
      \begin{codeblock}[]{}
\documentclass{ctexart}
\usepackage{graphicx}
|\highlightline|\usepackage{subcaption}
\graphicspath{{figs/}{pics/}}
\begin{document}
  \begin{figure}[ht]
|\highlightline|    \begin{subfigure}{0.48\textwidth}
      \centering
      \includegraphics[height=2cm]{sjtug}
      \caption{|\phantom{}|徽标}
|\highlightline|    \end{subfigure}\hfill
|\highlightline|    \begin{subfigure}{0.48\textwidth}
      \centering
      \includegraphics[height=2cm]{sjtugt}
      \caption{|\phantom{}|文字}
|\highlightline|    \end{subfigure}
    \caption{SJTUG}\label{fig:sjtug}
  \end{figure}
\end{document}
      \end{codeblock}
    \end{column}
    \begin{column}{0.4\textwidth}
      \includepdflarge{support/examples/subfigures.pdf}\vspace{15pt}
      \pkg{subcaption} 宏包提供了 \env{subfigure} 环境(以及 \env{subtable}),可以用于以子图的形式插入,编号会增加一级。也可以为子图添加子级引用编号。
    \end{column}
  \end{columns}
\end{frame}

\section{表}
\begin{frame}[fragile]
  \frametitle{简单表格}
  \begin{columns}
    \begin{column}{0.6\textwidth}
      \begin{codeblock}[]{}
\documentclass{ctexart}
|\only<1-2>{\highlightline}|\usepackage{|\temporal<1>{array}{\highlight{array}}{array},\temporal<2>{booktabs}{\highlight{booktabs}}{booktabs}|}
\begin{document}
\begin{table}[ht]
  \centering
  \caption{|\phantom{}|北京冬奥会吉祥物}
|\highlightline<1>|  \begin{tabular}{lp{3cm}}
|\highlightline<2>|    \toprule
|\highlightline<3>|吉祥物 & 描述                          \\
|\highlightline<2>|    \midrule
|\highlightline<3>|冰墩墩 & 2022 年北京冬季奥运会吉祥物  \\
|\highlightline<3>|雪容融 & 2022 年北京冬季残奥会吉祥物  \\
|\highlightline<2>|    \bottomrule
|\highlightline<1>|  \end{tabular}
\end{table}
\end{document}
      \end{codeblock}
    \end{column}
    \begin{column}{0.4\textwidth}

      \only<1>{
        使用 \env{tabular} 环境绘制表格。需要添加参数(称为\textbf{表格导言})以确定每一列的对齐方式。引入 \pkg{array} 宏包来使用更多的\textbf{引导符}。
        \begin{center}
          \footnotesize
          \begin{tabular}{>{\ttfamily}ll}
            \alert{l} & 向左对齐 \\
            \alert{c} & 居中对齐 \\
            \alert{r} & 向右对齐 \\
            \alert{p\{3cm\}} & 固定列宽,两端对齐 \\
            \alert{m\{3cm\}} & \texttt{p} + 垂直居中对齐 \\
            \alert{>\{\textbackslash{}bfseries\}} & 后一列单元格前加命令 \\
            \alert{*\{3\}\{l\}} & 三个左对齐列 \\
          \end{tabular}
        \end{center}
      }

      \only<2>{
        \pkg{booktabs} 宏包提供了标准三线表格所需要的行分割线:\cmd{toprule},\cmd{midrule},\cmd{bottomrule}。也可以使用 \cmd{cmidrule\{1-2\}} 来部分地绘制行分割线。一般不推荐在专业表格中使用纵向分割线。
      }

      \only<3>{
        每行内容使用 \textbackslash\textbackslash{} 分隔开,每行中的单元格使用 \& 分隔开。
      }

      \only<4>{
        \includepdflarge{support/examples/table.pdf}
      }
    \end{column}
  \end{columns}
\end{frame}

\begin{frame}[fragile]%
  \begin{columns}
    \begin{column}{0.6\textwidth}
      \begin{codeblock}[]{表头居中}
\documentclass{ctexart}
\usepackage{array,booktabs}
\begin{document}
\begin{table}[ht]
  \centering
  \caption{|\phantom{}|北京冬奥会吉祥物}
  \begin{tabular}{lp{3cm}}
    \toprule
|\highlightline|\multicolumn{1}{c}{|\phantom{}|吉祥物} &
|\highlightline|\multicolumn{1}{c}{|\phantom{}|描述} \\
    \midrule
|\phantom{}|冰墩墩 & 2022 年北京冬季奥运会吉祥物  \\
|\phantom{}|雪容融 & 2022 年北京冬季残奥会吉祥物  \\
    \bottomrule
  \end{tabular}
\end{table}
\end{document}
      \end{codeblock}
    \end{column}
    \begin{column}{0.4\textwidth}
      \cmd{multicolumn} 命令不仅可以用于合并同行的单元格,还可以用于临时地屏蔽表格导言对该列的对齐方式定义。这里用于居中表头。
      \begin{center}
        \parbox{0.85\linewidth}{
          \cmd{multicolumn\{格数\}\{对齐方式\}\{内容\}}
        }
      \end{center}
      跨页表格考虑使用 \pkg{longtable} 宏包。带标注的表格可以考虑使用 \pkg{threeparttable} 宏包。考虑使用在线工具生成表格代码 \link{https://www.tablesgenerator.com/latex_tables}。
      \note{复杂的使用方法在 \SJTUThesis{} 示例文档中都有提及。}
    \end{column}
  \end{columns}
\end{frame}

\section{数学公式}
\begin{frame}
  \frametitle{数学模式}
  \begin{alertblock}{}
  输入数学公式是 \LaTeX{} 的绝对强项,很多常见的公式服务依赖于一些轻量级渲染引擎比如 MathJax, K\kern-.3ex\raise.4ex\hbox{\footnotesize A}\kern-.3ex\TeX{}。但是它们实际上使用的是 \LaTeX{} 语法变种,也就是说并没有使用 \LaTeX{} 后端。所以不要期望有完全一致的输出。
  \end{alertblock}

  为了更好的获得数学公式输入支持,请使用 \hologo{AmS}math 宏包。数学模式分为两种:
  \begin{description}
    \item[行内模式] 一般通过一对美元符号(\$$\cdots$\$)标记,可以使用编辑器内建的符号表输入数学符号,也可以使用在线工具手写识别 \link{https://detexify.kirelabs.org/classify.html}。
    \item[行间模式] 一般通过 \env{equation} 环境\footnote{这是有编号环境,其加星号的变种 \env{equation*} 用于生成无编号环境。}输入。如果需要使用多行公式,请使用 \env{align} 环境,并按照类似表格输入的方式,使用 \& 对齐符号,\textbackslash\textbackslash{} 换行。如果不想手动居中,可以考虑多行自动居中的 \env{gather} 和单个大型公式首尾两端对齐 \env{multline}。
  \end{description}
  \note{关于表格和数学公式,如果不太熟悉如何输入,或者符号表记不住,推荐从比较容易上手的编辑器起步,比如 \TeX{}studio 提供了用户友好的界面(\faWindows{} 上的向导 $\rightarrow$ 数学助手)。我相信输入半年后,就可以对这些符号的输入很熟练了。

  你会发现我的这套教程没有讲很多的数学公式输入技巧,因为这些东西只有你自己熟练了才能体会。而且 \LaTeX{} 本来就不是完全关于数学公式的。}
\end{frame}

\begin{frame}
  \frametitle{数学命令展示}
  \begin{columns}
    \begin{column}{0.33\textwidth}
      \begin{exampleblock}{}
        $PV=nRT$
      \end{exampleblock}
      \begin{exampleblock}{}
        $\sum_{i=1}^ki^2=\frac{n(n+1)(2n+1)}{6}$
      \end{exampleblock}
      \begin{exampleblock}{}
        $T(n) = aT\left(\left\lceil\frac{n}{b}\right\rceil\right) + \mathcal{O}(n^d)$
      \end{exampleblock}
      \begin{exampleblock}{}
        $\frac{x_{1}+x_{2}+x_{3}}{3}\geq \sqrt[3]{x_{1}x_{2}x_{3}}$
      \end{exampleblock}
      \begin{exampleblock}{}
        $n=(\underbrace{1\cdots 1}_{k\text{ of 1's}})_2=2^{k+1}-1$
      \end{exampleblock}
      \begin{exampleblock}{}
        $\nabla f (P)= \left.\left(\frac{\partial f}{\partial x},\frac{\partial f}{\partial y},\frac{\partial f}{\partial z}\right)\right|_{P}$
      \end{exampleblock}
    \end{column}
    \begin{column}{0.67\textwidth}
      \begin{exampleblock}{}
        \begin{equation*}
          \int_{a}^b f(x)\,\mathrm{d}x=\lim_{|P|\rightarrow 0}\sum_{i=1}^n f(\xi_i)\Delta x_i
        \end{equation*}
        \note{在 SJTUThesis 中,你或许更希望使用 \cmd{increment} 而不是 \cmd{Delta} 来表示增量。}
      \end{exampleblock}
      \begin{exampleblock}{}
        \begin{equation}
          T(n) = \begin{cases}
            \mathcal{O}(n^d),&\textrm{if } d>\log_b a, \\
            \mathcal{O}(n^d\log n), &\textrm{if } d=\log_b a,\\
            \mathcal{O}(n^{\log_b a}), &\textrm{if } d<\log_b a.
          \end{cases}
        \end{equation}
      \end{exampleblock}
      \begin{exampleblock}{}
        \begin{align}
          Q^{T}A&=R \\
          \begin{pmatrix}
            q_1^T \\ q_2^T \\ q_3^T
          \end{pmatrix}
          \begin{pmatrix}
            a_1 & a_2 & a_3
          \end{pmatrix}
          &=R
        \end{align}
      \end{exampleblock}
    \end{column}
  \end{columns}
  \note{关于如果 \LaTeX{} 报出了错误,比如说数学模式下不能有空行,想要学习如何修复这些错误,
  可以详见 Learn\LaTeX{}.org 的相关章节
  \link{https://github.com/CTeX-org/learnlatex.github.io/blob/zh-Hans/zh-Hans/lesson-15.md}。}
\end{frame}

%更深入地讲解 mathtools, unicode-math, siunix

\section{引用}
\begin{frame}[fragile]
  \frametitle{交叉引用}
  \only<1>{
    正如之前所提到的,\LaTeX{} 中使用 \cmd{label} 标记,然后可以使用 \cmd{ref} 来引用这个标记。 \cmd{label} 可以放在使用计数器的对象之后。
  }

  \only<2>{
    为了使得对公式编号的引用带有括号,推荐使用 \hologo{AmS}math 宏包中的 \cmd{eqref} 命令。对于多行公式环境,每一个换行符前都可以添加一个 \cmd{label} 用于引用该行公式。
  }

  \begin{columns}
    \begin{column}{0.5\textwidth}
      \begin{codeblock}[]{图}
\begin{figure}
|\highlightline<1>|  \caption{|\phantom{}|示例}\label{fig:example}
\end{figure}
      \end{codeblock}
      \begin{codeblock}[]{表}
\begin{table}
|\highlightline<1>|  \caption{|\phantom{}|示例}\label{tab:example}
\end{table}
      \end{codeblock}
    \end{column}
    \begin{column}{0.5\textwidth}
\begin{codeblock}[]{目次}
|\highlightline<1>|\section{|\phantom{}|示例}\label{sec:example}
\end{codeblock}

\begin{codeblock}[]{公式}
\begin{equation}
  a = b + c
|\highlightline<1>|\label{eq:example}
\end{equation}
|\highlightline<2>|如公式 \eqref{eq:example} 所示,
\end{codeblock}
    \end{column}
  \end{columns}
\end{frame}

\begin{frame}[fragile]
  \frametitle{文献引用}
  \LaTeX{} 可以通过专用数据库文件 \texttt{.bib} 自动生成参考文献,很多的文献管理文件比如 EndNote \link{https://lic.sjtu.edu.cn/Default/softshow/tag/MDAwMDAwMDAwMLGImKE}, Zotero \link{https://www.zotero.org/}, JabRef \link{https://www.jabref.org/} 都可以直接导出这种格式的文件用于 \LaTeX{} 论文中的引用。一般不需要手写数据库文件,你只需要注意每一个文献会在数据库中有一个主键,这个类似于上文的 \cmd{label} 标记,只是要引用数据库中的文献需要使用 \cmd{cite} 命令。

  \begin{codeblock}[]{ref.bib}
|\highlightline|@article{devoftech,|\hfill\alert{\% 类型为期刊论文,主键为\texttt{devoftech}}|
  title={|\phantom{}|新时期我国信息技术产业的发展},
  author={|\phantom{}|江泽民},
  year={2008}
}
  \end{codeblock}
\end{frame}

\begin{frame}
  \frametitle{文献引用}
  而让 \LaTeX{} 处理 \texttt{.bib} 数据库文件与引用有两种工作流。你可能需要查清楚如何在编辑器中设置对应的工作流,或者采用后面所提到的高级编译方式 \texttt{latexmk}。
  \begin{columns}
    \begin{column}{0.5\textwidth}
      \begin{block}{\hologo{BibTeX} + \pkg{natbib}}
        一般期刊提交使用这种方法,\pkg{natbib} 宏包将提供命令 \cmd{citet}(文本引用) 和 \cmd{citep}(括号引用)。
      \end{block}
      \begin{alertblock}{\hologo{BibTeX} + \pkg{gbt7714}}
        中文引用可以直接使用 \pkg{gbt7714} 宏包,但是角模式和正文模式不能混用。
      \end{alertblock}
    \end{column}
    \begin{column}{0.5\textwidth}
      \begin{block}{\hologo{biber} + \pkg{biblatex}}
        这是更容易自定义的方法,与 \hologo{BibTeX} 的运作方式稍有不同。\pkg{biblatex} 提供了更加智能的引用命令。
      \end{block}
      \begin{alertblock}{\hologo{biber} + \pkg{biblatex-gb7714-2015}}
        而中文引用可以使用 \pkg{biblatex} 宏包的样式 \pkg{gb7714-2015}。
      \end{alertblock}
    \end{column}
  \end{columns}
\end{frame}

\begin{frame}[fragile]
  \frametitle{文献引用}
  \begin{columns}
    \begin{column}{0.5\textwidth}
      \begin{codeblock}[]{\hologo{BibTeX} + \pkg{gbt7714}}
\documentclass{ctexart}
\usepackage{gbt7714}
\bibliographystyle{gbt7714-numerial}
% \citestyle{numbers}  % 正文模式
\begin{document}
  |\phantom{}|他指出了...\cite{devoftech}
  \bibliography{ref}
\end{document}
      \end{codeblock}
    \end{column}
    \begin{column}{0.5\textwidth}
      \begin{codeblock}[]{\hologo{biber} + \pkg{biblatex-gb7714-2015}}
\documentclass{ctexart}
\usepackage[backend=biber,style=gb7714-2015]{biblatex}
\addbibresource{ref.bib}
\begin{document}
  |\phantom{}|他在文献 \parencite{devoftech}
  |\phantom{}|指出了...\cite{devoftech}
  \printbibliography
\end{document}
      \end{codeblock}
    \end{column}
  \end{columns}
\end{frame}

\begin{frame}
  \frametitle{文献引用}
  \begin{columns}
    \begin{column}{0.5\textwidth}
      \includepdflarge{support/examples/bibtex.pdf}
    \end{column}
    \begin{column}{0.5\textwidth}
      \includepdflarge{support/examples/biblatex.pdf}
    \end{column}
  \end{columns}
  \note{这页有一篇上过《新闻联播》的论文。}
\end{frame}

\end{shadedsection}
