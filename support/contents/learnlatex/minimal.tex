% !TeX root = ../../../latex-talk.tex

\section{基本要素}

\begin{frame}[fragile]%
  \frametitle{最小示例}
  \begin{columns}[c]
    \begin{column}{0.4\textwidth}
      \only<1>{
        \includepdflarge{support/examples/enminimal.pdf}
      }

      \only<2>{
        \begin{center}\highlight[structure]{文档类}\end{center}
        \cmd{documentclass} 命令加载了\textbf{文档类}。\cls{article} 是由 \LaTeX{} 提供的用于排版短文档的基本文档类。
        \begin{description}
          \footnotesize
          \item[\cls{article}] 不包含章的短文档
          \item[\cls{report}] 含有章的单面印刷文档
          \item[\cls{book}] 含有章的双面印刷文档
          \item[\cls{beamer}] 幻灯片
        \end{description}
      }

      \only<3>{
        \begin{center}\highlight[structure]{\texttt{document} 环境}\end{center}
        \cmd{begin} 和 \cmd{end} 用于创建\textbf{环境},可以多次、嵌套使用。环境用来指定一组文档元素的局部格式\footnotemark。\env{document} 环境是文档中必须有的环境,用于指示文档主体的范围。
      }
    \end{column}
    \begin{column}{0.6\textwidth}
      \begin{codeblock}[]{排版英文最小示例}
|\highlightline<2>|\documentclass{article}
|\highlightline<3>|\begin{document}
  Together for a Shared Future
|\highlightline<3>|\end{document}
      \end{codeblock}
    \end{column}
  \end{columns}

  \only<3>{\footnotetext{环境实际上是一个组,只不过通过语义化的形式预装了对应的格式命令。普通的组可以直接使用一对大括号之间的内容 \{$\cdots$\} 表示。}}
\end{frame}

\begin{frame}[fragile]%
  \frametitle{中文最小示例}
  \begin{columns}[c]
    \begin{column}{0.4\textwidth}

      \only<1>{
        \begin{center}\highlight[structure]{导言区}\end{center}
        \cmd{usepackage} 用于引入宏包,从而使用扩展功能,需要在\textbf{导言区}调用。这里使用 \pkg{ctex} 宏集以获得中文支持。
      }

    \end{column}
    \begin{column}{0.6\textwidth}
      \begin{codeblock}[]{排版中文最小示例}
\documentclass{article}
|\highlightline<1>\textbackslash{}usepackage\{ctex\}\hfill\color{structure}\% 导言区|
\begin{document}
  一起向未来

  Together for a Shared Future
\end{document}
      \end{codeblock}
    \end{column}
  \end{columns}
\end{frame}

\begin{frame}[fragile]%
  \frametitle{中文最小示例(更好版本)}
  \begin{columns}[c]
    \begin{column}{0.4\textwidth}

      \only<1>{
        \CTeX{} 建议对于之前提到的常规文档类,使用该宏集提供的四种中文文档类,以对特定类型提供额外的中文排版适配。
        \begin{center}
          \footnotesize
          \begin{tabular}{cc}
            \cls{ctexart} & \cls{ctexrep} \\
            \cls{ctexbook} & \cls{ctexbeamer} \\
          \end{tabular}
        \end{center}
      }

      \only<2>{
        \includepdflarge{support/examples/cnminimal.pdf}
      }

      \only<3>{
        \LaTeX{} 中通过空行来开启新的段落。一般情况下,\alert{不建议}在一段中强制断行(使用 \textbackslash{}\textbackslash{})。
      }

    \end{column}
    \begin{column}{0.6\textwidth}
      \begin{codeblock}[]{排版中文最小示例(更好版本)}
|\highlightline<1>|\documentclass{ctexart}
\begin{document}
|\highlightline<3>|  一起向未来
|\highlightline<3>|
|\highlightline<3>|  Together for a Shared Future
\end{document}
      \end{codeblock}
    \end{column}
  \end{columns}
\end{frame}
