% !TeX root = ../../../latex-talk.tex

\section{图}

\begin{frame}[fragile]%
  \frametitle{插图}
  \begin{columns}
    \begin{column}{0.6\textwidth}
      \begin{codeblock}[]{插入单图\only<4->{最佳实践}}
\documentclass{ctexart}
|\highlightline<1,2>|\usepackage{graphicx}
|\highlightline<1,2>|\graphicspath{{figs/}{pics/}}
\begin{document}
|\highlightline<1,5>|\begin{figure}[ht]
|\highlightline<1,6>|  \centering
|\highlightline<1,3,4>|  \includegraphics[width=|\only<1-3>{4cm}\only<4->{0.4\textbackslash{}textwidth}|]{sjtug}
|\highlightline<1,7>|  \caption{SJTUG 徽标}\label{fig:sjtug}
|\highlightline<1,5>|\end{figure}
\end{document}
      \end{codeblock}
    \end{column}
    \begin{column}{0.4\textwidth}

      \only<1>{
        \includepdflarge{support/examples/insertimage.pdf}
      }

      \only<2>{
        要插入图片,需使用 \pkg{graphicx} 宏包。之后在文档主体便可以使用 \cmd{includegraphics} 插入图片。在导言区可通过 \cmd{graphicspath} 指定图片文件夹\footnotemark。
      }

      \only<3>{
        \cmd{includegraphics} 命令以相对路径插入图片,后缀名可以省略。可以使用可选参数指定插入的图片尺寸。
        \note{比如我未来想变更为幻灯片的时候。}
      }

      \only<4>{
        最佳实践是使用 \cmd{textwidth} 或 \cmd{linewidth} 的相对值指定尺寸大小,以在未来可能的布局更改中保留一定的灵活性。除了 \opt{width} 还有其他属性:
        \note{事实上,\LaTeX{} 很多命令都是使用方括号添加可选参数的。}
        \begin{center}
          \footnotesize
          \begin{tabular}{rl}
            \opt{width} & 宽度 \\
            \opt{height} & 高度 \\
            \opt{scale} & 缩放 \\
            \opt{angle} & 角度 \\
          \end{tabular}
        \end{center}
      }

      \only<5>{
        \env{figure} 为一个\textbf{浮动体}环境(\env{table} 也是),可以将其移动到其他位置上以使排版更紧凑。可以添加可选参数以指定如何放置浮动体,最多可以使用四种位置描述符:
        \begin{center}
          \footnotesize
          \begin{tabular}{cll}
            \opt{h} & Here & 尽可能在这里 \\
            \opt{t} & Top & 页面顶部 \\
            \opt{b} & Bottom & 页面底部 \\
            \opt{p} & Page & 浮动体专页 \\
          \end{tabular}
        \end{center}
        还可以只使用 \pkg{float} 宏包提供的 \opt{H} 描述符以强制置于此处。
      }

      \only<6>{
        使用 \cmd{centering} 命令可将图片水平居中。
      }

      \only<7>{
        使用 \cmd{caption} 命令输入题注,如果这个命令写在插入图片前面,题注将会在上方(中文中一般对 \env{table} 环境这么做)。\cmd{label} 为图片添加标记名称,可在随后进行交叉引用。
      }

    \end{column}
  \end{columns}
  \only<2>{\footnotetext{其命令参数每个为一个以 \texttt{/} 结尾的文件夹,每个文件夹需要使用大括号包裹起来。}}
\end{frame}

\begin{frame}[fragile]
  \begin{columns}
    \begin{column}{0.6\textwidth}
      \begin{codeblock}[]{插入双图}
\documentclass{ctexart}
\usepackage{graphicx}
\graphicspath{{figs/}{pics/}}
\begin{document}
|\highlightline<1>|\begin{figure}[ht]
|\highlightline<1,2>|  \begin{minipage}{0.48\textwidth}
|\highlightline<1>|    \centering
|\highlightline<1>|    \includegraphics[height=2cm]{sjtug}
|\highlightline<1,3>|    \caption{SJTUG 徽标}\label{fig:sjtug}
|\highlightline<1,2>|  \end{minipage}\hfill
|\highlightline<1,2>|  \begin{minipage}{0.48\textwidth}
|\highlightline<1>|    \centering
|\highlightline<1>|    \includegraphics[height=2cm]{sjtugt}
|\highlightline<1,3>|    \caption{SJTUG 文字}\label{fig:sjtugt}
|\highlightline<1,2>|  \end{minipage}
|\highlightline<1>|\end{figure}
\end{document}
      \end{codeblock}
    \end{column}
    \begin{column}{0.4\textwidth}

      \only<1>{
        \includepdflarge{support/examples/doubleimages.pdf}
      }

      \only<2>{
        在 \env{figure} 环境里使用 \env{minipage} 小页制作列盒子,以并排两图并分别编号,需要设定强制参数以指定列宽。两个小页之间添加 \cmd{hfill} 使两个小页两端对齐。
      }

      \only<3>{
        在每个小页内部分别使用 \cmd{caption} 添加标签。
      }
    \end{column}
  \end{columns}
\end{frame}

\begin{frame}[fragile]%
  \begin{columns}
    \begin{column}{0.6\textwidth}
      \begin{codeblock}[]{}
\documentclass{ctexart}
\usepackage{graphicx}
|\highlightline|\usepackage{subcaption}
\graphicspath{{figs/}{pics/}}
\begin{document}
\begin{figure}[ht]
|\highlightline|  \begin{subfigure}{0.48\textwidth}
    \centering
    \includegraphics[height=2cm]{sjtug}
    \caption{徽标}
|\highlightline|  \end{subfigure}\hfill
|\highlightline|  \begin{subfigure}{0.48\textwidth}
    \centering
    \includegraphics[height=2cm]{sjtugt}
    \caption{文字}
|\highlightline|  \end{subfigure}
  \caption{SJTUG}\label{fig:sjtug}
\end{figure}
\end{document}
      \end{codeblock}
    \end{column}
    \begin{column}{0.4\textwidth}
      \includepdflarge{support/examples/subfigures.pdf}\vspace{15pt}
      \pkg{subcaption} 宏包提供了 \env{subfigure} 环境(以及 \env{subtable}),可以用于以子图的形式插入,编号会增加一级。也可以为子图添加子级引用编号。
    \end{column}
  \end{columns}
\end{frame}
