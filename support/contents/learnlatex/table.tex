% !TeX root = ../../../latex-talk.tex

\section{表}

\begin{frame}[fragile]
  \frametitle{三线表}
  \begin{columns}
    \begin{column}{0.6\textwidth}
      \begin{codeblock}[]{}
\documentclass{ctexart}
|\only<1-3>{\highlightline}|\usepackage{|\temporal<2>{array}{\highlight{array}}{array},\temporal<3>{booktabs}{\highlight{booktabs}}{booktabs}|}
\begin{document}
|\highlightline<1>|\begin{table}[ht]
  \centering
  \caption{|\phantom{}|北京冬奥会吉祥物}
|\highlightline<1,2>|  \begin{tabular}{lp{3cm}}
|\highlightline<1,3>|    \toprule
|\highlightline<1,4>|吉祥物 & 描述                          \\
|\highlightline<1,3>|    \midrule
|\highlightline<1,4>|冰墩墩 & 2022 年北京冬季奥运会吉祥物  \\
|\highlightline<1,4>|雪容融 & 2022 年北京冬季残奥会吉祥物  \\
|\highlightline<1,3>|    \bottomrule
|\highlightline<1,2>|  \end{tabular}
|\highlightline<1>|\end{table}
\end{document}
      \end{codeblock}
    \end{column}
    \begin{column}{0.4\textwidth}

      \only<1>{
        \includepdflarge{support/examples/table.pdf}
      }

      \only<2>{
        使用 \env{tabular} 环境绘制表格。需要添加参数(称为\textbf{表格导言})以确定每一列的对齐方式。引入 \pkg{array} 宏包来使用更多的\textbf{引导符}。
        \begin{center}
          \footnotesize
          \begin{tabular}{>{\ttfamily}ll}
            \alert{l} & 向左对齐 \\
            \alert{c} & 居中对齐 \\
            \alert{r} & 向右对齐 \\
            \alert{p\{3cm\}} & 固定列宽,两端对齐 \\
            \alert{m\{3cm\}} & \texttt{p} + 垂直居中对齐 \\
            \alert{>\{\textbackslash{}bfseries\}} & 后一列单元格前加命令 \\
            \alert{*\{3\}\{l\}} & 三个左对齐列 \\
          \end{tabular}
        \end{center}
      }

      \only<3>{
        \pkg{booktabs} 宏包提供了标准三线表格所需要的行分割线:\cmd{toprule},\cmd{midrule},\cmd{bottomrule}。也可以使用 \cmd{cmidrule\{1-2\}} 来部分地绘制行分割线。一般不推荐在专业表格中使用纵向分割线。
      }

      \only<4>{
        每行内容使用 \textbackslash\textbackslash{} 分隔开,每行中的单元格使用 \& 分隔开。
      }

    \end{column}
  \end{columns}
\end{frame}

\begin{frame}[fragile]%
  \begin{columns}
    \begin{column}{0.6\textwidth}
      \begin{codeblock}[]{表头居中}
\documentclass{ctexart}
\usepackage{array,booktabs}
\begin{document}
\begin{table}[ht]
  \centering
  \caption{|\phantom{}|北京冬奥会吉祥物}
  \begin{tabular}{lp{3cm}}
    \toprule
|\highlightline|\multicolumn{1}{c}{|\phantom{}|吉祥物} &
|\highlightline|\multicolumn{1}{c}{|\phantom{}|描述} \\
    \midrule
|\phantom{}|冰墩墩 & 2022 年北京冬季奥运会吉祥物  \\
|\phantom{}|雪容融 & 2022 年北京冬季残奥会吉祥物  \\
    \bottomrule
  \end{tabular}
\end{table}
\end{document}
      \end{codeblock}
    \end{column}
    \begin{column}{0.4\textwidth}
      \cmd{multicolumn} 命令不仅可以用于合并同行的单元格,还可以用于临时地屏蔽表格导言对该列的对齐方式定义。这里用于居中表头。
      \begin{center}
        \parbox{0.85\linewidth}{
          \cmd{multicolumn\{格数\}\{对齐方式\}\{内容\}}
        }
      \end{center}
      跨页表格考虑使用 \pkg{longtable} 宏包。带标注的表格可以考虑使用 \pkg{threeparttable} 宏包。考虑使用在线工具生成表格代码 \link{https://www.tablesgenerator.com/latex_tables}。
      \note{复杂的使用方法在 \SJTUThesis{} 示例文档中都有提及。}
    \end{column}
  \end{columns}
\end{frame}
