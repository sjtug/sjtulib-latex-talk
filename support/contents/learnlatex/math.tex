% !TeX root = ../../../latex-talk.tex

\section{数学公式}

\begin{frame}
  \frametitle{数学模式}
  \begin{alertblock}{}
  输入数学公式是 \LaTeX{} 的绝对强项,很多常见的公式服务依赖于一些轻量级渲染引擎比如 MathJax、K\kern-.3ex\raise.4ex\hbox{\footnotesize A}\kern-.3ex\TeX{}。但是它们实际上使用的是 \LaTeX{} 语法变种,也就是说并没有使用 \LaTeX{} 后端。所以不要期望有完全一致的输出。
  \end{alertblock}

  为了更好的获得数学公式输入支持,请使用 \hologo{AmS}math 宏包。数学模式分为两种:
  \begin{description}
    \item[行内模式] 一般通过一对美元符号(\$$\cdots$\$)标记,可以使用编辑器内建的符号表输入数学符号,也可以使用在线工具手写识别 \link{https://detexify.kirelabs.org/classify.html} \link{https://mathpix.com/equation-to-latex}。
    \item[行间模式] 一般通过 \env{equation} 环境\footnote{这是有编号环境,其加星号的变种 \env{equation*} 用于生成无编号环境。}输入。如果需要使用多行公式,请使用 \env{align} 环境,并按照类似表格输入的方式,使用 \& 对齐符号,\textbackslash\textbackslash{} 换行。如果不想手动居中,可以考虑多行自动居中的 \env{gather} 和单个大型公式首尾两端对齐 \env{multline}。
  \end{description}
  \note{关于表格和数学公式,如果不太熟悉如何输入,或者符号表记不住,推荐从比较容易上手的编辑器起步,比如 \TeX{}studio 提供了用户友好的界面(\faWindows{} 上的向导 $\rightarrow$ 数学助手)。我相信输入半年后,就可以对这些符号的输入很熟练了。

  你会发现我的这套教程没有讲很多的数学公式输入技巧,因为这些东西只有你自己熟练了才能体会。而且 \LaTeX{} 本来就不是完全关于数学公式的。}
\end{frame}

\begin{frame}
  \frametitle{一些例子}
  \begin{columns}
    \begin{column}{0.33\textwidth}
      \begin{exampleblock}{}
        $PV=nRT$
      \end{exampleblock}
      \begin{exampleblock}{}
        $\sum_{i=1}^ki^2=\frac{n(n+1)(2n+1)}{6}$
      \end{exampleblock}
      \begin{exampleblock}{}
        $T(n) = aT\left(\left\lceil\frac{n}{b}\right\rceil\right) + \mathcal{O}(n^d)$
      \end{exampleblock}
      \begin{exampleblock}{}
        $\frac{x_{1}+x_{2}+x_{3}}{3}\geq \sqrt[3]{x_{1}x_{2}x_{3}}$
      \end{exampleblock}
      \begin{exampleblock}{}
        $n=(\underbrace{1\cdots 1}_{k\text{ of 1's}})_2=2^{k+1}-1$
      \end{exampleblock}
      \begin{exampleblock}{}
        $\nabla f (P)= \left.\left(\frac{\partial f}{\partial x},\frac{\partial f}{\partial y},\frac{\partial f}{\partial z}\right)\right|_{P}$
      \end{exampleblock}
    \end{column}
    \begin{column}{0.67\textwidth}
      \begin{exampleblock}{}
        \begin{equation*}
          \int_{a}^b f(x)\,\mathrm{d}x=\lim_{|P|\rightarrow 0}\sum_{i=1}^n f(\xi_i)\Delta x_i
        \end{equation*}
        \note{在 SJTUThesis 中,你或许更希望使用 \cmd{increment} 而不是 \cmd{Delta} 来表示增量。}
      \end{exampleblock}
      \begin{exampleblock}{}
        \begin{equation}
          T(n) = \begin{cases}
            \mathcal{O}(n^d),&\textrm{if } d>\log_b a, \\
            \mathcal{O}(n^d\log n), &\textrm{if } d=\log_b a,\\
            \mathcal{O}(n^{\log_b a}), &\textrm{if } d<\log_b a.
          \end{cases}
        \end{equation}
      \end{exampleblock}
      \begin{exampleblock}{}
        \begin{align}
          Q^{T}A&=R \\
          \begin{pmatrix}
            q_1^T \\ q_2^T \\ q_3^T
          \end{pmatrix}
          \begin{pmatrix}
            a_1 & a_2 & a_3
          \end{pmatrix}
          &=R
        \end{align}
      \end{exampleblock}
    \end{column}
  \end{columns}
  \note{关于如果 \LaTeX{} 报出了错误,比如说数学模式下不能有空行,想要学习如何修复这些错误,
  可以详见 Learn\LaTeX{}.org 的相关章节
  \link{https://github.com/CTeX-org/learnlatex.github.io/blob/zh-Hans/zh-Hans/lesson-15.md}。}
\end{frame}

%更深入地讲解 mathtools, unicode-math, siunix
