% !TeX root = ../../../latex-talk.tex

\section{安装和编辑}

\begin{frame}
  \frametitle{发行版}
  \begin{table}
    \caption{\hologo{TeX} 发行版}
    \footnotesize
    \begin{stampbox}
      \begin{tabular}{c>{\raggedright}*{3}{p{3.2cm}}}
        \alert{发行版}     & \hologo{MiKTeX} \link{https://miktex.org/}   & \TeX{} Live \link{https://www.tug.org/texlive/}   & Mac\TeX{} \link{https://www.tug.org/mactex/}  \\[2pt]
        \alert{特点}      &  只安装必要文件,依赖用时更新  &  所有平台均可使用,每年发布一次 & Mac 系统专用,对 \TeX{} Live 的进一步打包 \\
        \alert{推荐平台}  & \faWindows  & \faWindows\,\faLinux &  \faApple \\
      \end{tabular}
    \end{stampbox}
  \end{table}
  \begin{center}
    \parbox{.9\textwidth}{
      要让 \LaTeX{} 跑起来,核心就是要有一套 \TeX{} 发行版,来获取让 \LaTeX{} 工作所需的一组程序和文件。参考《一份简短的关于 \LaTeX{} 安装的介绍》\link{https://mirrors.sjtug.sjtu.edu.cn/ctan/info/install-latex-guide-zh-cn/install-latex-guide-zh-cn.pdf} 安装想使用的发行版。推荐使用发行版的最新版本\footnote{老版本 Linux 系统的包管理器自带 \TeX{} Live 发行版可能不是最新的 \link{https://repology.org/project/texlive/versions},尽量使用镜像安装,并手动将相关环境变量添加到路径 \link{https://www.tug.org/texlive/doc/texlive-zh-cn/texlive-zh-cn.pdf}。},并使用国内镜像。
    }
  \end{center}
  \note{要让 \LaTeX{} 跑起来,核心就是要有一套 \TeX{} 发行版,来获取让
  \LaTeX{} 工作所需的一组程序和文件。参考《一份简短的关于 \LaTeX{} 安装的介绍》
  安装想使用的发行版,里面介绍了 \faWindows{}, \faApple{}, \faLinux{}, WSL 等系统上
  \TeX{} Live 的安装,非常全面,一步一步做就可以成功安装。目前最新的 \TeX{} Live 版本为
  2022,\SJTUThesis{} 用户不应当安装 \TeX{} Live 2020 以下的版本(后面会讲)。

  事实上,我认为这几个发行版各有操作系统偏好,虽然前两者是跨平台的。

  \hologo{MiKTeX} 对 Windows 用户较为友好,安装简单,占用空间不大,安装时间短,
  而且有完整的安装与卸载程序。可以给大家看一下 \hologo{MiKTeX} Console 的情况。

  \TeX{} Live 更符合 Linux 的更新传统。老版本 Linux 系统的包管理器自带 \TeX{} Live
  发行版可能不是最新的(到时间也会锁定依赖库的版本),尽量使用镜像安装(当然也推荐使用最新的
  Linux 发行版,这样它的版本也就一直是最新的),并手动将相关环境变量添加到路径。

  Mac\TeX{} 发行版有 pkg 安装包封装,并且附带了 \TeX{}Shop 基本编辑软件,更加适合 Mac OS。
  我用下来的话,感觉除了 \TeX{} Live Utility 外都有点过时了。}
\end{frame}

\begin{frame}[plain]
  \hbox to \textwidth{
    \hfil
    \vbox to 3cm{
      \hbox{
        \resizebox{3cm}{!}{\includegraphics{support/examples/pics/sjtug}}
      }
    }
    \hfil
    \vbox to 3cm{
      \vfill
      \hbox{\Large\bfseries\color{structure} 稳定、快速、现代的镜像服务。}
      \vskip2pt
      \hbox{托管于华东教育网骨干节点上海交通大学。}
      \vfill
    }
    \hskip20pt
    \hfil
  }

  \begin{center}
    \parbox{0.8\textwidth}{
      推荐使用 SJTUG 软件镜像服务 \link{https://mirror.sjtu.edu.cn/},离得近,下得快。

      \begin{description}
        \footnotesize
        \item[\hologo{MiKTeX}] \url{https://mirrors.sjtug.sjtu.edu.cn/CTAN/systems/win32/miktex/setup/windows-x64/} \\ 并在 \hologo{MiKTeX} Console 中设置镜像源为 \url{https://mirrors.sjtug.sjtu.edu.cn}
        \item[\TeX{} Live] \url{https://mirrors.sjtug.sjtu.edu.cn/CTAN/systems/texlive/tlnet}
        \item[Mac\TeX{}] \url{https://mirrors.sjtug.sjtu.edu.cn/CTAN/systems/mac/mactex/}
        \item[\faTelegram] 可以在 SJTUG 镜像站通知频道 \link{https://t.me/sjtug_mirrors_news} 获得更多信息,加入关联群组参与讨论。
      \end{description}
    }
  \end{center}
  \note{说到镜像,像后两者的安装包都很大(4GB 左右),由于一些原因,不采用镜像的话不知道要下到什么时候,对下载速度的要求高;
  而 \hologo{MiKTeX} 需要随时更新,宏包大小颗粒度大,对延迟的要求高。
  那么采用 SJTUG 镜像源将同时解决这两个问题,位于图信大楼的机房,凭借校内的高速网络,稳定快速下载,
  现在由 LightQuantum 维护的镜像站欢迎大家的使用,主页上还有更多的其他镜像可供使用,加入 Telegram 群组参与讨论。}
\end{frame}

\begin{frame}
  \frametitle{编辑器}
  \begin{table}
    \caption{开源编辑器推荐}
    \footnotesize
    \begin{stampbox}
      \begin{tabular}{c>{\raggedright}*{3}{p{3.5cm}}}
        \alert{编辑器}     & \begin{tabular}{c}Visual Studio Code \link{https://code.visualstudio.com}\\ +\,\LaTeX{} Workshop \link{https://marketplace.visualstudio.com/items?itemName=James-Yu.latex-workshop}\end{tabular}  & \TeX{}studio \link{https://texstudio.org} & \TeX{}works \\[5pt]
        \alert{特点}      &  搭配 VS Code 使用非常方便,易扩展  & 可以使用大量的菜单选项输入代码块,用户友好 & 只提供基础的高亮与运行方法,发行版自带\footnote{Mac\TeX{} 打包的是 \TeX{}Shop 编辑器。} \\
      \end{tabular}
    \end{stampbox}
  \end{table}
  \begin{center}
    \parbox{.9\textwidth}{
      使用专为 \LaTeX{} 设计的编辑器将带来更多便利,因为它们往往会提供一键编译、内置 PDF 阅读器以及语法高亮等功能。几乎所有现代的 \LaTeX{} 编辑器都提供 Sync\TeX{} 这一强大的功能,以在 PDF 和代码间对应跳转。
    }
  \end{center}
  \note{编辑器的种类很多,我无法一一列举,但是对编写 \TeX{} 常用的开源编辑器我推荐这三个。
  其中 \TeX{}studio 的安装包可能下得有点慢。这里我对这些编辑器都演示一下,初学者我更推荐使用
  \TeX{} studio 编辑器,如果平时就码很多代码的话,我更推荐使用 VS Code 加插件这种方式。

  使用专为 \LaTeX{} 设计的编辑器将带来更多便利,因为它们往往会提供一键编译、内置 PDF 阅读器
  以及语法高亮等功能。几乎所有现代的 \LaTeX{} 编辑器都提供 Sync\TeX{} 这一强大的功能
  (VS Code 的方法是 Ctrl + 某处,Overleaf 的方法是直接双击),以在 PDF 和代码间对应跳转。
  当然如果你不喜欢使用这种 GUI 编辑器,\TeX{} 文档本身就是纯文本,对 Vim, Emacs 等终端用户
  也很友好。}
\end{frame}

\begin{frame}
  \frametitle{在线平台}
  \begin{table}
    \caption{在线协作平台推荐}
    \footnotesize
    \begin{stampbox}
      \begin{tabular}{c>{\raggedright}*{2}{p{4cm}}}
        \alert{在线平台}     & Overleaf \link{https://www.overleaf.com/}  & 交大 \LaTeX{} 助手 \link{https://latex.sjtu.edu.cn/} \\[2pt]
        \alert{特点}      & 最流行的在线平台,提供大量的模板,但国内访问慢 & 校内平台,隐私保护有保障,共享项目限制少 \\
      \end{tabular}
    \end{stampbox}
  \end{table}
  \begin{center}
    \parbox{.9\textwidth}{
      在线平台允许你直接在网页中编辑文档,无需本地安装即可在后台运行 \LaTeX{},并显示生成的 PDF。可以参照 Overleaf 官方文档学习如何使用在线平台 \link{https://www.overleaf.com/learn}。
    }
  \end{center}
  \note{当然使用在线平台省去了安装发行版的麻烦,这里列出两种在线写作平台。

  如果有数据合规需求的话,可以考虑使用由网络信息中心维护的交大 \LaTeX{} 助手,最近更新了 \TeX{} Live 2022,还是很不错的。

  当然国内还有 \TeX{} Page \link{https://www.texpage.com/},Slagger \link{https://www.slager.cn/} 等。
  一般来讲,这种平台使用的都是 Linux 操作系统,所以在排版中文的时候考虑将编译引擎更改为 \hologo{XeLaTeX},
  学习如何使用在线平台参见 Overleaf 的官方文档 \link{https://www.overleaf.com/learn}。}
\end{frame}
