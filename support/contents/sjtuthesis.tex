% !TeX root = ../../latex-talk.tex

\part{SJTUThesis}

\begin{frame}
  \frametitle{本部分主要参考}
  \begin{bibliolist}{00}
    \onlineitem \textsc{SJTUG}.
    \newblock \textsc{SJTUThesis} 示例文档[EB/OL].
    \newblock 2022. \url{https://github.com/sjtug/SJTUThesis}.

    \onlineitem \textsc{SJTUG}.
    \newblock \textsc{SJTUThesis} 用户文档[EB/OL].
    \newblock 2022. \url{https://github.com/sjtug/SJTUTeX}.
  \end{bibliolist}
\end{frame}

\begin{frame}
  \frametitle{简介}
  \begin{columns}
    \begin{column}{0.6\textwidth}
      \begin{itemize}
        \item 最早由韦建文于 2009 年 11 月发布 0.1a 版
        \item 2018 年起由 SJTUG 接手维护
        \item 2019 年 6 月吴伟健重构了整个宏包的代码,升级版本号为 1.0
        \item 2022 年 11 月模板改版后,吴伟健、张驰等人使用 \LaTeX3 重构 2.0 版本
        \item 最新版:\SJTUThesisVersion{} (\SJTUThesisDate)
        \item 支持本科、硕士、博士学位论文的排版
        \item 推荐使用最新版本的 \TeX{} 发行版编译
      \end{itemize}
    \end{column}
    \begin{column}{0.4\textwidth}
      \begin{exampleblock}{}
        \begin{minipage}[c]{1cm}
          \includegraphics[width=0.8cm]{\getcontribpath{sjtug}{vi/sjtug}}
        \end{minipage}
        \begin{minipage}[c]{3cm}
          \href{https://github.com/sjtug}{sjtug}/\href{https://github.com/sjtug/SJTUThesis}{SJTUThesis}
        \end{minipage}
      \end{exampleblock}
      \vspace{-8pt}
      \begin{block}{}
        \scriptsize
        上海交通大学 \hologo{XeLaTeX} 学位论文及课程论文模板 | Shanghai Jiao Tong University \hologo{XeLaTeX} Thesis Template
      \end{block}
      \vspace{-8pt}
      \begin{alertblock}{}
        \scriptsize
        \begin{tabular}{cl}
          \faStar & 2.6k \\
          \faEye & 52 \\
          \faCodeBranch & 726 \\
        \end{tabular}
      \end{alertblock}
    \end{column}
  \end{columns}
\end{frame}

\begin{frame}
  \frametitle{\only<1>{为什么使用 \LaTeX{} 排版论文?}\only<2>{当然它们也互相学习}}
  \begin{columns}[t]
    \begin{column}{0.25\textwidth}
      \begin{exampleblock}{\faMarkdown{} Markdown}
        \begin{itemize}
          \item[\faPlus] 技术文档流行
          \item[\faPlus] 语法简单 
          \item[\faMinus] 不内置格式控制
        \end{itemize}
      \end{exampleblock}
      \only<2>{
        \begin{block}{}
          \begin{itemize}
            \item[\faBolt] R Markdown (Bookdown) 模板 \link{https://github.com/bubifengyun/SJTUThesis-Rmd} \link{https://github.com/bubifengyun/SJTUThesis-Rmd}
            \item[\faAsterisk] 配套 MathJax 渲染公式  
          \end{itemize}
        \end{block}
      }
    \end{column}
    \begin{column}{0.25\textwidth}
      \begin{exampleblock}{\faFileWord{} Word}
        \begin{itemize}
          \item[\faPlus] 通用论文格式
          \item[\faPlus] 所见即所得
          \item[\faMinus] 进阶排版仍困难 
        \end{itemize}
      \end{exampleblock}
      \only<2>{
        \begin{block}{}
          \begin{itemize}
            \item[\faBolt] 数学公式可以直接通过 \LaTeX{} 格式转换
            \item[\faAsterisk] 也就是 Unicode Math 输入方式 
          \end{itemize}
        \end{block}
      }
    \end{column}
    \begin{column}{0.25\textwidth}
      \begin{block}{\LaTeX{} SJTUThesis}
        \begin{itemize}
          \item[\faPlus] 学术论文格式
          \item[\faPlus] 内容样式分离
          \item[\faMinus] 上手有门槛 
        \end{itemize}
      \end{block}
      \only<2>{
        \begin{exampleblock}{}
          \begin{itemize}
            \item[\faBolt] \TeX{} 的可视前端 \hologo{LyX} \link{https://www.lyx.org/Download} Overleaf Rich Text 模式
            \item[\faAsterisk] \TeX{} 的可视改良 \TeX{}\raise-0.25em\hbox{\footnotesize MACS} \link{http://texmacs.org/tmweb/home/welcome.en.html} \link{https://mogan.app}
          \end{itemize}
        \end{exampleblock}
      }
    \end{column}
    \begin{column}{0.25\textwidth}
      \begin{exampleblock}{\faAdobe{} InDesign}
        \begin{itemize}
          \item[\faPlus] 专业杂志排版
          \item[\faPlus] 精细调整
          \item[\faMinus] 过于繁琐专业  
        \end{itemize}
      \end{exampleblock}
      \only<2>{
        \begin{block}{}
          \begin{itemize}
            \item[\faBolt] 传说用了 \TeX{} 的一些算法 \link{https://mp.weixin.qq.com/s/GASGHK-GsIg2Fwb2jWwpvw}
          \end{itemize}
        \end{block}
      }
    \end{column}
  \end{columns}
  \note{\emph{这页仅作简要介绍。}

  让我们来讨论 the elephant in the room:为什么用 \LaTeX{} 排版论文?
  }
  \note<1>[item]{Markdown 很好啊,方便的语法,一般技术文档也常用。但就是因为它太简单了,没有内置样式控制,一般需要借助 HTML,CSS 那一套东西。}
  \note<2>[item]{也会有一些人尝试通过 R Markdown(Bookdown)改进,当然你也可以试着使用 \LaTeX{} 里的 \pkg{markdown} 宏包
  (这有点像各种前端博客框架渲染 Markdown 为 HTML,只不过这里渲染 \LaTeX{} 生成 PDF,代码抄录方面已经有 Sphinx 这个工具 \link{https://github.com/sphinx-doc/sphinx}),
  以及可以通过 MathJax 渲染 \TeX{} 公式。
  }
  \note<1>[item]{Word 很好啊,官方钦定的论文写作方法,可见即可得,但是进阶排版仍然可以困难。就拿排版公式来说,
  大家以前初高中学的、或者是计算机二级考的 Word 2003 要排版公式,一种是装 MathType 插件,版本间不兼容、一种是搞个域代码+替换字体。}
  \note<2>[item]{Word 2007 之后添加了插入正经的 Unicode 公式功能,但默认的 Calibri Math 字体以及符号布局仍然赶不上 \TeX{} 的美感。
  }
  \note<1>[item]{根据之前学到的一些技巧,看起来不难对吧(虽然有我诱导的成分),然后它内容与样式分离的设计理念已经渗入了很多领域。以及很多学术论文都需要 \LaTeX{} 的提交。}
  \note<2>[item]{当然现在也推出了可见即可得的编辑器 \hologo{LyX},Overelaf 的可视模式(虽然这两个并不是一个东西);以及迟先生很喜欢的 TeXmacs,一种不使用 \TeX{} 底层的、但是效果相像的排版程序。
  }
  \note<1>[item]{设计相关专业的同学可能更喜欢 Adobe Indesign,对图文混排更为擅长。但是它也足够复杂,虽然提供了几乎所有的排版用具,但是我还没见过用它排版几十页充满公式的论文的(或许有人会开先河?)。}
  \note<2>[item]{以及传说它用了 \TeX{} 的一些算法,所以 \TeX{} 还是老大哥,兼具美感和批量化处理的折中方法。}
\end{frame}

\begin{frame}
  \frametitle{开始使用}
  \alert{下载} 推荐安装 Git \link{https://git-scm.com/} 后,克隆 SJTUG 镜像仓库
  \begin{exampleblock}{\faGit*}
    \ttfamily\small
    git clone https://mirror.sjtu.edu.cn/git/SJTUThesis.git/
  \end{exampleblock}

  \alert{编译} 推荐使用 \pkg{latexmk} 编译\footnote{\hologo{MiKTeX} 用户需要手动安装 Perl 解释器 \link{https://www.perl.org/get.html} 才能使用 \pkg{latexmk}。},在不能够利用自带的 \texttt{.latexmkrc} 配置文件的情况下,需要查清楚在对应的编辑器中如何使用 \hologo{XeLaTeX} + \hologo{biber} 编译\footnote{这种情况下,你可能需要查清楚如何全局安装该文档类,并刷新文件名数据库。} \link{https://github.com/sjtug/SJTUThesis/blob/master/README.md}。
  \begin{exampleblock}{\faTerminal}
    \ttfamily\small
    latexmk -xelatex main
  \end{exampleblock}

  \alert{在线} 直接使用 Overleaf 链接 \link{https://www.overleaf.com/latex/templates/sjtuthesis-latex-thesis-template-for-shanghai-jiao-tong-university/mkdwbyjbtfgg?r=sdkbtJ4qGS8kDZQQ&rm=d&rs=b}。
  其他在线平台用户可以下载压缩包,上传至对应平台并采用 \hologo{XeLaTeX} 编译,请注意使用最新版本的 \TeX{} Live。
  \note{不会 Git 的同学可以直接 Download ZIP。}
\end{frame}

\begin{frame}
  \frametitle{手动编译}
  \alert{第一次编译失败} 如果没有办法通过通常方式编译成功,请尝试使用文件夹内附带 \faLinux{}\,\faApple{} \texttt{Makefile} 和 \faWindows{} \texttt{Compile.bat} 进行编译。

  \alert{统计字数} 编写过程中也可以使用对应的命令调用 \TeX{}count 来统计正文字数。
  \begin{columns}
    \begin{column}{0.38\textwidth}
      \begin{exampleblock}{\faLinux{}\,\faApple}
        \ttfamily
        make all\\
        make clean\\
        make cleanall\\
        make wordcount
      \end{exampleblock}
    \end{column}
    \begin{column}{0.38\textwidth}
      \begin{exampleblock}{\faWindows}
        \ttfamily
        ./Compile.bat thesis\\
        ./Compile.bat clean\\
        ./Compile.bat cleanall\\
        ./Compile.bat wordcount
      \end{exampleblock}
    \end{column}
    \begin{column}{0.24\textwidth}
      \begin{block}{\faInfo}
        \ttfamily
        编译论文\\
        清理中间文件\\
        $\hookrightarrow +$删除论文\\
        统计字数
      \end{block}
    \end{column}
  \end{columns}
\end{frame}

\begin{frame}[label=compile]
  \frametitle{编译问题排查}
  \begin{columns}
    \begin{column}{0.33\textwidth}
      \begin{alertblock}{无法使用 \texttt{latexmk}\thesisissue{578}}
        \hologo{MiKTeX} 需要安装 Perl 解释器。
      \end{alertblock}  
      \begin{alertblock}{\CTeX{} 套装无法编译\thesisissue{446}}
        使用最新 \TeX{} 发行版。\link{https://github.com/Aloft-Lab/CTeX-Installer}
      \end{alertblock}
      \begin{alertblock}{\hologo{pdfLaTeX} 无法编译\thesisissue{444}}
        请使用 \texttt{latexmk},或更改编辑器设置以 \hologo{XeLaTeX} 编译。
      \end{alertblock}
    \end{column}
    \begin{column}{0.33\textwidth}
      \begin{alertblock}{缺少字体\thesisissue{564} \thesisdiscuss{598}}
        更换字体集,或者安装对应字体。
      \end{alertblock}
      \begin{alertblock}{缺少汉字\thesisissue{533} \thesisdiscuss{617}}
        去除使用 fandol 字体集的设定。或者是安装字体后,改用 \texttt{cjk-font=adobe} 或 \texttt{cjk-font=founder}。
      \end{alertblock}
    \end{column}
    \begin{column}{0.33\textwidth}
      \begin{block}{\faInfoCircle{} README}
        不同编辑器的设置请首先参阅 README \link{https://github.com/sjtug/SJTUThesis/blob/master/README.md} 文档。
      \end{block}
      \begin{block}{\faBookOpen{} Wiki}
        其他编译问题推荐查阅 Wiki \link{https://github.com/sjtug/SJTUThesis/wiki} 的使用说明部分。
      \end{block}
    \end{column}
  \end{columns}
  \note{两个进阶问题:
  
  如果之前出现了编译错误,在重新编译前,最好清理一下临时文件(\texttt{make clean})。

  如果是 biber 出现了问题,还可以尝试 \texttt{rm -rfv \$(biber --cache)}。\thesisdiscuss{774}
  }
\end{frame}

\begin{frame}[label=covers]
  \frametitle{论文组成}
  \begin{figure}[h]
    \centering
    \foreach \thesispage/\thesisnote in {
      1/{中文封面},3/{英文封面},5/{版权页},7/{中文摘要},9/{英文摘要},11/{目录},13/{插图目录},15/{表格目录},
      17/{正文},27/{参考文献},29/{附录},31/{成果},33/{致谢},35/{大摘要}} {%
      \begin{subfigure}{.13\textwidth}
        \centering
        \fzerobox{\includegraphics[width=\textwidth,page=\thesispage]{support/thesis/sample-thesis-zh.pdf}}
        \caption{\thesisnote}
      \end{subfigure}
    }
  \end{figure}
  \note{虽然说符号表在新版中是要放在附录中,但是学位论文规范中却说可以放在正文前,所以你可以自行选择。}
  \note{文档中的彩色框是标识超链接,印刷时不会输出,如果希望关闭可以向 \pkg{hyperref} 宏包添加可选参数 \opt{hidelinks}。}
\end{frame}

\begin{frame}[fragile]
  \frametitle{文档类选项}
  \begin{columns}
    \begin{column}{0.35\textwidth}
      \includegraphics[page=1]{support/figures/thesisdir.pdf}
    \end{column}
    \begin{column}{0.65\textwidth}
      文档类选项是指在载入文档类时的可选选项,多个选项使用逗号隔开,文档类选项会对所有宏包可见。
      \begin{codeblock}[escapechar="]{main.tex}
% !TeX encoding = UTF-8

% 载入 SJTUThesis 模版
"\highlightline"\documentclass[type=master]{sjtuthesis}
% 选项
%   type=[doctor|master|bachelor],
%   zihao=[-4|5],
%   lang=[zh|en],
%   review,
%   [twoside|oneside],
%   math-style=[ISO|TeX],
      \end{codeblock}
    \end{column}
  \end{columns}
\end{frame}

\begin{frame}[fragile]
  \frametitle{文档类选项}
  \begin{columns}
    \begin{column}{0.3\textwidth}
      \includegraphics[page=1]{support/figures/thesisdir.pdf}
    \end{column}
    \begin{column}{0.7\textwidth}
      我是学士,写英文论文
      \begin{codeblock}[]{}
|\phantom{}|\documentclass[type=bachelor,lang=en]{sjtuthesis}
      \end{codeblock}
      我是硕士,盲审
      \begin{codeblock}[]{}
|\phantom{}|\documentclass[type=master,review]{sjtuthesis}
      \end{codeblock}
      我是博士,先写着电子版不空页
      \begin{codeblock}[]{}
|\phantom{}|\documentclass[type=doctor,oneside]{sjtuthesis}
      \end{codeblock}
    \end{column}
  \end{columns}
  \note{有 ChatGPT 那味了。}
\end{frame}

\begin{frame}
  \frametitle{文档类选项}
  \begin{columns}
    \begin{column}{0.35\textwidth}
      \includegraphics[page=2,scale=0.9]{support/figures/thesisdir.pdf}
    \end{column}
    \begin{column}{0.65\textwidth}
      \begin{table}
        \caption{文档类选项}
        \footnotesize
        \begin{tabular}{>{\ttfamily}rll}
          \toprule
          选项 & 含义 & 相关 \\
          \midrule
          type= & 指定论文类型 & 第 \ref{covers} 页\\
          \midrule
          cjk-font= & 指定中文字体 & \\
          text-font= & 指定西文字体 & 第 \ref{frame:fonts} 页\\
          math-font= & 指定数学字体 & \\
          math-style= & 指定数学符号样式 & 第 \ref{frame:math-style} 页\\
          \midrule
          review & 开启盲审模式 & \thesisissue{195} \thesisissue{686} \\
          twoside & 双页模式 & \thesisissue{554} \\
          oneside & 单页模式 & \thesisissue{694} \\
          openright & 章从奇数页开始 & \thesisdiscuss{724} \\
          openany & 章从任意页开始 & \thesisissue{446} \\
          \bottomrule
        \end{tabular}
      \end{table}

      更多文档类选项查阅 \textsc{SJTU\TeX{}} 的开发文档 \link{https://github.com/sjtug/SJTUTeX/releases/download/v1.1.0/sjtuthesis.pdf}。
    \end{column}
  \end{columns}
\end{frame}

\begin{frame}[label={frame:fonts}]
  \frametitle{字体配置}
  \newcommand{\noticemark}{\alert{\normalshape \textopenbullet}}
  \begin{columns}
    \begin{column}{0.35\textwidth}
      \includegraphics[page=3,scale=0.9]{support/figures/thesisdir.pdf}
    \end{column}
    \begin{column}{.65\textwidth}
      相较于 \CTeX{} 使用 \texttt{fontset} 设定中文字体集,
      \SJTUThesis{} 还提供了西文、数学字体集的设定\footnotemark。

      {
        \medskip
        \ttfamily\scriptsize
        \alert{cjk-font=...\hfill 中文字体}
        
        \stamphrule\medskip

        \foreach \sjtufontname/\sjtufontdesc in {adobe/{adobe \faAdobe\noticemark},{fandol}/{fandol \faLinux{} \noticemark},{founder},mac/{mac \faApple{} \noticemark},windows/{windows \faWindows},ubuntu/{ubuntu \noticemark}}{
          \begin{minipage}{2.2cm}
            \centering
            \includefontpreview{support/thesis/cjkfont-\sjtufontname.pdf}\\
            \raisebox{0.8ex}{\sjtufontdesc}
          \end{minipage}
        }

        \bigskip

        \alert{text-font=..., math-font=...\hfill 西文与数学字体}

        \stamphrule\medskip

        \foreach \sjtufontname/\sjtufontdesc in {{cambria}/{cambria \noticemark},{lm},{newcm}/{newcm \noticemark},{newpx},{newtx},{stixtwo}/{stixtwo \noticemark},{times},{xits}/{xits \noticemark}}{
          \begin{minipage}{2cm}
            \centering
            \includefontpreview{support/thesis/latinfont-\sjtufontname.pdf}\\
            \raisebox{0.8ex}{\sjtufontdesc}
          \end{minipage}
        }
      }
    \end{column}
  \end{columns}
  \footnotetext{\noticemark 表示无法使用 pdf\LaTeX{} 编译。}
\end{frame}

\begin{frame}[label={frame:math-style}]
  \frametitle{数学符号样式}
  \begin{columns}
    \begin{column}{0.35\textwidth}
      \includegraphics[page=2,scale=0.9]{support/figures/thesisdir.pdf}
    \end{column}
    \begin{column}{0.65\textwidth}
      新增数学符号样式 \texttt{math-style} 文档类选项,现在默认为 \texttt{ISO},
      如果更喜欢原始的 \TeX{} 数学样式,可以切换为 \texttt{TeX}。

      \begin{minipage}[c]{10em}
        \texttt{math-style=ISO}
      \end{minipage}
      \begin{minipage}[c]{5cm}
        \includemathstylepreview{support/thesis/mathstyle-ISO.pdf}
      \end{minipage}
       
      \begin{minipage}[c]{10em}
        \texttt{math-style=TeX}
      \end{minipage}
      \begin{minipage}[c]{5cm}
        \includemathstylepreview{support/thesis/mathstyle-TeX.pdf}
      \end{minipage}

      \begin{block}{}
        请注意在默认情况下(\texttt{math-style=ISO})应当使用 \cmd{increment} 而不是 \cmd{Delta} 表示有限增量。
      \end{block}
    \end{column}
  \end{columns}
\end{frame}

\begin{frame}[fragile]
  \frametitle{基本配置}
  \begin{columns}
    \begin{column}{0.35\textwidth}
      \includegraphics[page=1]{support/figures/thesisdir.pdf}
    \end{column}
    \begin{column}{0.65\textwidth}

      \only<1>{
        在 \texttt{main.tex} 中引入 \texttt{setup.tex} 来导入主要的信息录入与宏包加载配置。
      }

      \only<2>{
        \alert{\textbf{(a,b)}} 其中 \cmd{sjtusetup}(第 \ref{sjtusetup} 页)中的 \opt{info} 将会修改封面的信息设置。
      }

      \begin{codeblock}[firstnumber=12]{main.tex}
|\highlightline<1>|% 论文基本配置,加载宏包等全局配置
|\highlightline<1>|\input{setup}

\begin{document}

%TC:ignore

|\highlightline<2>|% 标题页
|\highlightline<2>|\maketitle
      \end{codeblock}
    \end{column}
  \end{columns}
\end{frame}

\begin{frame}[fragile, label=sjtusetup]
  \frametitle{基本配置}
  \begin{columns}
    \begin{column}{0.35\textwidth}
      \includegraphics[page=4]{support/figures/thesisdir.pdf}
    \end{column}
    \begin{column}{0.65\textwidth}
      \vspace*{-0.2cm}
      \begin{codeblock}[firstnumber=3]{setup.tex}
\sjtusetup{
  info = {
    zh/title  = {|\phantom{}|上海交通大学学位论文 \LaTeX{} 模板示例文档},
    en/title  = {A Sample for \LaTeX-based SJTU Thesis Template},
    zh/author = {|\phantom{}|某\quad{}某},
    en/author = {Mo Mo},
  },
  style = { float-seperator = {--}, },
  name = {
    achv = {|\phantom{}|攻读学位期间完成的论文},
  },
}
      \end{codeblock}
    \end{column}
  \end{columns}
\end{frame}

\begin{frame}[label=setup]
  \frametitle{基本配置}
  \begin{columns}
    \begin{column}{0.35\textwidth}
      \includegraphics[page=4]{support/figures/thesisdir.pdf}
    \end{column}
    \begin{column}{0.65\textwidth}
      \begin{table}
        \centering
        \caption{info 域}
        \footnotesize
        \begin{tabular}{lll} \toprule
          命令作用     & 中文对应选项                      & 英文对应选项                 \\ \midrule
          论文标题     & \texttt{zh/title}                 & \texttt{en/title}            \\
          关键字列表   & \texttt{zh/keywords}              & \texttt{en/keywords}        \\
          作者姓名     & \texttt{zh/author}                & \texttt{en/author}           \\
          申请学位名称 & \texttt{zh/degree}                & \texttt{en/degree}           \\
          院系名称     & \texttt{zh/department}            & \texttt{en/department}       \\
          专业名称     & \texttt{zh/major}                 & \texttt{en/major}            \\
          导师         & \texttt{zh/supervisor}            & \texttt{en/supervisor}       \\
          副导师       & \texttt{zh/assoc-supervisor}      & \texttt{en/assoc-supervisor} \\
          联培导师     & \texttt{zh/co-supervisor}         & \texttt{en/co-supervisor}    \\
          日期         & \multicolumn{2}{c}{\texttt{date}}                                \\
          学号         & \multicolumn{2}{c}{\texttt{id}}                                  \\ \bottomrule
        \end{tabular}
      \end{table}
    \end{column}
  \end{columns}
  \note{有些选项是 v2 更名或新增的。}
  \note{注意现在使用语言前缀作为键名。}
\end{frame}

\begin{frame}[fragile]
  \frametitle{版权页}
  \begin{columns}
    \begin{column}{0.4\textwidth}
      \includegraphics[page=5]{support/figures/thesisdir.pdf}
    \end{column}
    \begin{column}{0.6\textwidth}
      \alert{\textbf{(c)}} \cmd{copyrightpage} 可以用于插入版权页。
      也可接受一个可选参数,用于直接使用扫描件,此时需要载入 \pkg{pdfpages} 包。\thesisissue{473}

      \begin{codeblock}[firstnumber=22]{main.tex}
% 原创性声明及使用授权书
\copyrightpage
% 插入外置原创性声明及使用授权书
% 导言区添加 \usepackage{pdfpages}
% \copyrightpage[scans/sample-copyright-old.pdf]
      \end{codeblock}
    \end{column}
  \end{columns}
\end{frame}

\begin{frame}[fragile]
  \frametitle{三个部分}
  \framesubtitle{前置部分}
  \begin{columns}
    \begin{column}{0.4\textwidth}
      \includegraphics[page=8]{support/figures/thesisdir.pdf}
    \end{column}
    \begin{column}{0.6\textwidth}
      \alert{\textbf{(d,e,f,g,h)}} 前言从 \cmd{frontmatter} 处开始,页码设置为大写罗马数字,主要包含摘要和目录内容。
      \begin{codeblock}[firstnumber=27]{main.tex}
|\highlightline|% 前置部分
|\highlightline|\frontmatter

% 摘要
\input{contents/abstract}

% 目录
\tableofcontents
% ...
      \end{codeblock}
    \end{column}
  \end{columns}
\end{frame}

\begin{frame}[fragile]
  \frametitle{三个部分}
  \framesubtitle{正文部分}
  \begin{columns}
    \begin{column}{0.4\textwidth}
      \includegraphics[page=9]{support/figures/thesisdir.pdf}
    \end{column}
    \begin{column}{0.6\textwidth}
      \alert{\textbf{(i,j,k)}} 正文从 \cmd{mainmatter} 处开始,页码设置为正常数字,包含正文、参考文献、附录内容。
      \begin{codeblock}[firstnumber=47]{main.tex}
|\highlightline|% 主体部分
|\highlightline|\mainmatter

% 正文内容
% !TeX root = ../../../latex-talk.tex

\section{是什么}

\begin{frame}
  \frametitle{\TeX{}}
  \begin{columns}[c]
    \begin{column}{0.7\textwidth}
      \begin{center}
        \rmfamily\Huge
        \highlight[structure]{\TeX{}}
      \end{center}
      \begin{center}
        \parbox{0.75\textwidth}{
          \TeX{} 是由斯坦福大学教授高德纳
          (Donald E.~Knuth)于 1977 年开始开发的排版引擎。目前仍在更新,最新版本号为 3.141592653 \link{https://tug.org/TUGboat/tb42-1/tb130knuth-tuneup21.pdf}。
        }
      \end{center}
    \end{column}
    \begin{column}{0.3\textwidth}
      \includegraphics[width=.8\columnwidth]{support/images/Knuth.jpg}
    \end{column}
  \end{columns}
  \note{\emph{这一部分背景介绍大家可以了解一下,暂时跳过。}
  \LaTeX{} 这个词由两个部分组成,\hologo{La} 和 \TeX{}。那我们首先了解一下 \TeX{} 是什么。
  \TeX{} 是由斯坦福大学的教授高德纳于 1977 年开始开发的排版引擎,它已经有三十多年的历史了,
  目前仍在更新,版本号(3.141592653)将会趋近于 $\pi$ 的取值,高德纳最近还在给 \textsl{TUGBoat} 写稿子
  \link{https://tug.org/TUGboat/tb42-1/tb130knuth-tuneup21.pdf},
  关于 \TeX{} 今年又做了哪些改进。}
\end{frame}

\begin{frame}
  \frametitle{\LaTeX{}}
  \begin{columns}[c]
    \begin{column}{0.7\textwidth}
      \begin{center}
        \rmfamily\Huge
        \highlight[structure]{\LaTeX{}}
      \end{center}
      \begin{center}
        \parbox{0.75\textwidth}{
          \LaTeX{} 是最早在 1985 年由现就职于微软的 Leslie Lamport 开发的一种 \TeX{} \textbf{格式}\footnotemark,使用一些列宏和扩展宏包来简化 \TeX{} 的使用。现在由 \LaTeX{} Project 的成员维护。现在广泛使用的版本是 \LaTeXe{},最新的版本为 \LaTeX3(2020 年 10 月后默认内置)。
        }
      \end{center}
    \end{column}
    \begin{column}{0.3\textwidth}
      \includegraphics[width=.8\columnwidth]{support/images/Lamport.jpg}
    \end{column}
  \end{columns}
  \footnotetext{\hologo{ConTeXt} 也是一种 \TeX{} 格式 \link{https://www.contextgarden.net/}。}
  \note{\emph{这一部分的背景介绍大家可以了解一下,暂时跳过。}
  \LaTeX{} 是最早由现就职于微软的 Leslie Lamport 开发的一种 \TeX{} 格式(与其对标的是
  \hologo{ConTeXt}\link{https://www.contextgarden.net/}),主要也是为了简化 \TeX{} 的使用。
  现在主要由 \LaTeX{} 开发组维护,现在广泛使用的版本是 \LaTeXe{},最新的版本为 \LaTeX3,
  在 2020 年 10 月后默认内置,所以要尽可能使用较新的发行版,以充分发挥其功能。}
\end{frame}

\begin{frame}
  \frametitle{程序}
  \begin{columns}[c]
    \begin{column}{0.7\textwidth}
      \begin{center}
        \rmfamily\Huge
        \highlight[structure]{\hologo{pdfLaTeX}}
      \end{center}
      \begin{center}
        \parbox{0.7\textwidth}{
          \hologo{pdfLaTeX} 是为了编译一个 \LaTeX{} 文档而运行的程序。实际上底层在运行一个叫 \hologo{pdfTeX} 的引擎,并预装了对应的 \LaTeX{} \textbf{格式}。为了利用临时文件,可能就需要多次运行程序。
        }
      \end{center}
    \end{column}
    \begin{column}{0.3\textwidth}
      \begin{block}{}
        \ttfamily\small
        > \highlight{pdflatex} main.tex\\
        This is pdfTeX, Version 3.141592653-
        2.6-1.40.23 (MiKTeX 21.10)\\
        entering extended mode\\
        \highlight{LaTeX2e} <2021-11-15>\\
        \highlight{L3} programming layer <2021-11-22>
      \end{block}
    \end{column}
  \end{columns}
  \note{\hologo{pdfLaTeX} 是为了编译一个 \LaTeX{} 文档而运行的程序。}
\end{frame}

% \begin{frame}
%   \frametitle{引擎}
%   \begin{columns}[c]
%     \begin{column}{0.7\textwidth}
%       \begin{center}
%         \rmfamily\Huge
%         \highlight[structure!70]{pdf}\hologo{La}\highlight[structure!70]{\TeX{}}
%       \end{center}
%       \begin{center}
%         \parbox{0.7\textwidth}{
%           pdf\TeX{} 是编译 \TeX{} 文档(以 \texttt{.tex} 结尾)的\textbf{引擎}---可以理解 \TeX{} 指令的\textbf{程序}。
%         }
%       \end{center}
%     \end{column}
%     \begin{column}{0.3\textwidth}
%       \begin{block}{}
%         \ttfamily\small
%         > pdflatex main.tex\\
%         This is \highlight[structure!70]{pdfTeX}, Version 3.141592653-
%         2.6-1.40.23 (MiKTeX 21.10)
%         entering extended mode\\
%         LaTeX2e <2021-11-15>\\
%         L3 programming layer <2021-11-22>
%       \end{block}
%     \end{column}
%   \end{columns}
%   \note{实际上底层在运行一个叫 \hologo{pdfTeX} 的引擎,并预装了对应的 \LaTeX{} 格式。}
% \end{frame}

\begin{frame}[label={frame:engine}]
  \frametitle{程序}
  \begin{table}
    \caption{主流 \hologo{(La)TeX} 程序
    \footnote{(u)p\TeX{} 是日语最常用的引擎,生成 \texttt{.dvi},支持 Unicode。}\footnote{Ap\TeX{} \link{https://github.com/clerkma/ptex-ng} 具有底层 CJK 支持,内联 Ruby,Color Emoji。}}
    \footnotesize
    \begin{stampbox}
      \begin{tabular}{c>{\raggedright}*{3}{p{3.5cm}}}
        \alert{引擎}     & \hologo{pdfTeX}   & \hologo{XeTeX}   & \hologo{LuaTeX}   \\
        \alert{程序}     & \hologo{pdfLaTeX} & \hologo{XeLaTeX} & \hologo{LuaLaTeX} \\
        \alert{特点}     & 直接生成 PDF,支持 micro-typography  & 支持 Unicode、OpenType 与复杂文字编排 (CTL) & 支持 Unicode,内联 Lua,支持 OpenType \\
      \end{tabular}
    \end{stampbox}
  \end{table}

  \begin{center}
    \parbox{.9\textwidth}{
      \hologo{pdfLaTeX} 不支持 Unicode。为了排版中文,大部分情况下应当使用 \hologo{XeLaTeX},而 \hologo{LuaLaTeX} 速度相对较慢。\faWindows{} 可以在一些情况下使用 \hologo{pdfLaTeX}。
    }
  \end{center}
  \note{当然为了排版中文,已经不再推荐使用 \hologo{pdfLaTeX} 了,应该使用
  \hologo{XeLaTeX} 或者 \hologo{LuaLaTeX},当然后者的速度还是相对较慢,
  它们支持 Unicode 编码,并可以使用 OpenType 字体的全部功能。
  当然 \faWindows{} 平台下在某些追求速度的情况下,
  还是可以试着使用 \hologo{pdfLaTeX} 的。

  \hologo{LuaLaTeX} 理想情况下不慢,但是使用一些宏包后会破坏理想状态,
  也会因配置产生不同的结果,不同的操作系统在 I/O 速度上的不同也会导致不同的时间。

  \hologo{pdfLaTeX} 也支持,只不过需要先生成 tfm \TeX{} 字体度量文件,后续使用 \TeX{}
  自身的配置方法,只能使用 7 比特或 8 比特字体。}
\end{frame}

% \begin{frame}
%   \paragraph{\hologo{pdfLaTeX}} \TeX{} 和 \LaTeX{} 被广泛使用之前,它们只需内置支持欧洲语言即可。在 Unicode 出现之前,\LaTeX{} 提供了许多种\textbf{文件编码}来允许很多语言的文字以原生的方式输入,\hologo{pdfLaTeX} 也只需要使用 8 位文件编码和 8 位字体。
% \end{frame}


\input{contents/math_and_citations}
\input{contents/floats}
\input{contents/summary}

%TC:ignore

% 参考文献
\printbibliography[heading=bibintoc]

% 附录
\appendix
      \end{codeblock}
    \end{column}
  \end{columns}
\end{frame}

\begin{frame}[fragile]
  \frametitle{三个部分}
  \framesubtitle{结尾部分}
  \begin{columns}
    \begin{column}{0.4\textwidth}
      \includegraphics[page=10]{support/figures/thesisdir.pdf}
    \end{column}
    \begin{column}{0.6\textwidth}
      \alert{\textbf{(k,l,m,n)}} 结尾从 \cmd{backmatter} 处开始,页码设置为正常数字,包含致谢等相关情况。
      \begin{codeblock}[firstnumber=71]{main.tex}
|\highlightline|% 结尾部分
|\highlightline|\backmatter

% 用于盲审的论文需隐去致谢、发表论文、科研成果、简历

% 致谢
\input{contents/acknowledgements}

% 发表论文及科研成果
% 盲审论文中,发表论文及科研成果等仅以第几作者注明即可,不要出现作者或他人姓名
\input{contents/achievements}

%...
      \end{codeblock}
    \end{column}
  \end{columns}
\end{frame}

\begin{frame}
  \frametitle{数学定理环境}
  \begin{columns}
    \begin{column}{0.4\textwidth}
      \includegraphics[page=6]{support/figures/thesisdir.pdf}
    \end{column}
    \begin{column}{0.6\textwidth}
      \SJTUThesis{} 定义了常用的数学环境(需要引入 \pkg{ntheorem} 或者 \pkg{amsthm} 宏包)。

      \begin{table}
        \centering
        \caption{\textsc{SJTUThesis} 定义的数学环境}
        \footnotesize
        \begin{tabular}{>{\ttfamily}rl|>{\ttfamily}rl}
          \toprule
          assumption  & 假设  & lemma       & 引理 \\
          axiom       & 公理  & problem     & 问题 \\
          conjecture  & 猜想  & proof       & 证明 \\
          corollary   & 推论  & proposition & 命题 \\
          definition  & 定义  & remark      & 注   \\
          example     & 例    & solution    & 解   \\
          exercise    & 练习  & theorem     & 定理 \\
          \bottomrule
        \end{tabular}
      \end{table}
    \end{column}
  \end{columns}
\end{frame}

\begin{frame}[fragile]
  \frametitle{参考文献}
  \begin{columns}
    \begin{column}{0.4\textwidth}
      \includegraphics[page=7]{support/figures/thesisdir.pdf}
    \end{column}
    \begin{column}{0.6\textwidth}
      \begin{codeblock}[firstnumber=111,numbersep=2pt]{setup.tex}
% 使用 BibLaTeX 处理参考文献
%   biblatex-gb7714-2015 常用选项
%     gbnamefmt=lowercase     姓名大小写由输入信息确定
%     gbpub=false             禁用出版信息缺失处理
\usepackage[backend=biber,style=gb7714-2015]{biblatex}
% 文献表字体
% \renewcommand{\bibfont}{\zihao{-5}}
% 文献表条目间的间距
\setlength{\bibitemsep}{0pt}
|\highlightline|% 导入参考文献数据库
|\highlightline|\addbibresource{bibdata/thesis.bib}
      \end{codeblock}
    \end{column}
  \end{columns}
\end{frame}

\begin{frame}
  \frametitle{Word}
  \begin{itemize}
    \item[{\faQuestionCircle[regular]}] 跟 Word 的参考实现略有不同 
    \item[{\faCheckCircle[regular]}] 毕设论文的格式只要不违背《上海交通大学关于本科生毕业设计(论文)工作的指导意见》\link{https://github.com/sjtug/SJTUThesis/files/6505296/default.pdf} \thesisissue{621}、《上海交通大学博士、硕士学位论文撰写指南》\link{https://www.gs.sjtu.edu.cn/info/1143/5801.htm} \thesisissue{652} 即可,其他细节上的修改可以先搜索解决方案,再反馈给我们。
    \item[{\faQuestionCircle[regular]}] 我需要转为 Word 文档
    \item[{\faCheckCircle[regular]}] PDF 转为 Word 文档属于逆向工程,暂时不存在完全正确的转换方法 \link{https://www.bilibili.com/video/BV1Vi4y1C71M},从 \LaTeX{} 源代码出发的转换可以使用其他工具实现 \thesisissue{480} \thesisissue{500}。
  \end{itemize}
\end{frame}

\begin{frame}
  \frametitle{还有其他问题?}
  % \begin{columns}
    % \begin{column}{0.73\textwidth}
      \begin{itemize}
        \item[{\faComment*[regular]}] 日常模板或 \LaTeX{} 使用问题可以前往 Discussions \link{https://github.com/sjtug/SJTUThesis/discussions} 提问

        (解决后别忘了 \textcolor{green}{\faCheckCircle{} Mark as answer}
        \item[{\faDotCircle[regular]}] 如果是 \textsc{SJTUThesis} 项目本身的 bug 和 feature request

        可以通过 Issues \link{https://github.com/sjtug/SJTUThesis/issues} 反馈。
        \item[{\faCodeBranch}] 如果你有好点子,可以贡献代码

          向 \textsc{SJTU\TeX{}} \link{https://github.com/sjtug/SJTUTeX} 存储库发 PR,\par
          而后把解包结果同步到 \textsc{SJTUThesis}。
        
        \item[{\faQq}] 也欢迎在 QQ 群(715273806)即时讨论。
        \note{群之前满了,社长给腾讯充了钱,让它可以接着塞人。}
      \end{itemize}
    % \end{column}
    % \begin{column}{0.27\textwidth}
    %   \includegraphics[height=0.7\textheight]{support/images/qq.jpg}
    % \end{column}
  % \end{columns}
\end{frame}
