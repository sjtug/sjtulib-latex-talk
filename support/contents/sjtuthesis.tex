% !TeX root = ..\..\latex-talk.tex

\part{SJTUThesis}

\begin{frame}
  \frametitle{简介}
  \begin{columns}
    \begin{column}{0.6\textwidth}
      \begin{itemize}
        \item 最早由韦建文于 2009 年 11 月发布 0.1a 版,2018 年起由 SJTUG 接手维护
        \item 最新版:\SJTUThesisVersion{} (\SJTUThesisDate)
        \item 支持本科、硕士、博士学位论文以及课程论文的排版
      \end{itemize}
    \end{column}
    \begin{column}{0.4\textwidth}
      \begin{exampleblock}{}
        \begin{minipage}[c]{1cm}
          \includegraphics[width=0.8cm]{\getcontribpath{sjtug}{vi/sjtug}}
        \end{minipage}
        \begin{minipage}[c]{2cm}
          \href{https://github.com/sjtug}{sjtug}/\href{https://github.com/sjtug/SJTUThesis}{SJTUThesis}
        \end{minipage}
      \end{exampleblock}
      \vspace{-8pt}
      \begin{block}{}
        \scriptsize
        上海交通大学 \hologo{XeLaTeX} 学位论文及课程论文模板 | Shanghai Jiao Tong University \hologo{XeLaTeX} Thesis Template
      \end{block}
      \vspace{-8pt}
      \begin{alertblock}{}
        \scriptsize
        \begin{tabular}{cl}
          \faStar & 2.4k \\
          \faEye & 55 \\
          \faCodeBranch & 701 \\
        \end{tabular}
      \end{alertblock}
    \end{column}
  \end{columns}
\end{frame}

\begin{frame}
  \frametitle{下载与编译}
  \alert{下载} 推荐安装 Git \link{https://git-scm.com/} 后,克隆 SJTUG 镜像仓库
  \begin{exampleblock}{\faGit*}
    \ttfamily\small
    git clone https://mirror.sjtu.edu.cn/git/SJTUThesis.git/
  \end{exampleblock}

  \alert{编译} 推荐使用 \pkg{latexmk} 编译\footnote{\hologo{MiKTeX} 用户需要手动安装 Perl 解释器 \link{https://www.perl.org/get.html} 才能使用 \pkg{latexmk}。},在不能够利用自带的 \texttt{.latexmkrc} 配置文件的情况下,需要查清楚在对应的编辑器中如何使用 \hologo{XeLaTeX} + \hologo{biber} 编译 \link{https://github.com/sjtug/SJTUThesis/blob/master/README.md}。
  \begin{exampleblock}{\faTerminal}
    \ttfamily\small
    latexmk -xelatex main
  \end{exampleblock}

  Overleaf 用户可以下载压缩包后,上传并采用 \hologo{XeLaTeX} 编译方式。
\end{frame}

\begin{frame}
  \frametitle{手动编译}
  \alert{第一次编译失败} 如果没有办法通过通常方式编译成功,请尝试使用文件夹内附带 \faLinux{}\,\faApple{} \texttt{Makefile} 和 \faWindows{} \texttt{Compile.bat} 进行编译。

  \alert{统计字数} 编写过程中也可以使用对应的命令调用 \TeX{}count 来统计正文字数。
  \begin{columns}
    \begin{column}{0.38\textwidth}
      \begin{exampleblock}{\faLinux{}\,\faApple}
        \ttfamily
        make all\\
        make clean\\
        make cleanall\\
        make wordcount
      \end{exampleblock}
    \end{column}
    \begin{column}{0.38\textwidth}
      \begin{exampleblock}{\faWindows}
        \ttfamily
        ./Compile.bat thesis\\
        ./Compile.bat clean\\
        ./Compile.bat cleanall\\
        ./Compile.bat wordcount
      \end{exampleblock}
    \end{column}
    \begin{column}{0.24\textwidth}
      \begin{block}{\faInfo}
        \ttfamily
        编译论文\\
        清理中间文件\\
        $\hookrightarrow +$删除论文\\
        统计字数
      \end{block}
    \end{column}
  \end{columns}
\end{frame}

\begin{frame}[label=compile]
  \frametitle{编译问题排查}
  \begin{columns}
    \begin{column}{0.33\textwidth}
      \begin{alertblock}{无法使用 \texttt{latexmk}\thesisissue{578}}
        \hologo{MiKTeX} 需要安装 Perl 解释器。
      \end{alertblock}  
      \begin{alertblock}{C\TeX{} 套装无法编译\thesisissue{446}}
        使用最新 \TeX{} 发行版。
      \end{alertblock}
      \begin{alertblock}{\hologo{pdfLaTeX} 无法编译\thesisissue{444}}
        请使用 \texttt{latexmk},或更改编辑器设置以 \hologo{XeLaTeX} 编译。
      \end{alertblock}
    \end{column}
    \begin{column}{0.33\textwidth}
      \begin{alertblock}{缺少字体\thesisissue{564} \thesisdiscuss{598}}
        更换字体集,或者安装对应字体。
      \end{alertblock}
      \begin{alertblock}{缺少汉字\thesisissue{533} \thesisdiscuss{617}}
        去除使用 fandol 字体集的设定。或者是安装字体后,改用 \texttt{fontset=adobe} 或 \texttt{fontset=founder}。
      \end{alertblock}
    \end{column}
    \begin{column}{0.33\textwidth}
      \begin{block}{\faInfoCircle{} README}
        不同编辑器的设置请首先参阅 README \link{https://github.com/sjtug/SJTUThesis/blob/master/README.md} 文档。
      \end{block}
      \begin{block}{\faBookOpen{} Wiki}
        其他编译问题推荐查阅 Wiki \link{https://github.com/sjtug/SJTUThesis/wiki} 的使用说明部分。
      \end{block}
    \end{column}
  \end{columns}
\end{frame}

\begin{frame}[fragile, label=covers]
  \begin{codeblock}[firstnumber=3]{main.tex}
|\alert{\% 载入 SJTUThesis 模版}|
\documentclass[|\only<1>{\highlight{type}}\only<2>{type}|=|\only<1>{bachelor}\only<2>{\highlight{bachelor}}|]{sjtuthesis}
  \end{codeblock}
  \begin{figure}
    \parbox{0.9\textwidth}{
      \begin{subfigure}{0.20\textwidth}
        \framebox{\includegraphics[width=\linewidth]{support/thesis/bachelor}}
        \caption{\only<1>{学士}\only<2>{\texttt{bachelor}}}
      \end{subfigure}\hfill
      \begin{subfigure}{0.20\textwidth}
        \framebox{\includegraphics[width=\linewidth]{support/thesis/master}}
        \caption{\only<1>{硕士}\only<2>{\texttt{master}}}
      \end{subfigure}\hfill
      \begin{subfigure}{0.20\textwidth}
        \framebox{\includegraphics[width=\linewidth]{support/thesis/doctor}}
        \caption{\only<1>{博士}\only<2>{\texttt{doctor}}}
      \end{subfigure}\hfill
      \begin{subfigure}{0.20\textwidth}
        \framebox{\includegraphics[width=\linewidth]{support/thesis/course}}
        \caption{\only<1>{课程}\only<2>{\texttt{course}}}
      \end{subfigure}
      \caption{论文类型示例\only<2>{ \texttt{type}}}
    }
  \end{figure}
\end{frame}

\begin{frame}[fragile]
  \frametitle{文档类选项}
  % \framesubtitle{\textbackslash{}documentclass\{sjtuthesis\}}
  \begin{columns}
    \begin{column}{0.45\textwidth}
      \includegraphics[page=10]{thesisdir}
    \end{column}
    \begin{column}{0.55\textwidth}
      \begin{table}[H]
        \caption{文档类选项}
        \footnotesize
        \begin{tabular}{>{\ttfamily}rll}
          \toprule
          选项 & 含义 & 相关 \\
          \midrule
          type= & 指定论文类型 & 第 \ref{covers} 页\\
          fontset= & 指定字体 & 第 \ref{compile} 页\\
          \midrule
          review & 开启盲审模式 & \thesisissue{195} \thesisissue{686} \\
          twoside & 双页模式 & \thesisissue{554} \\
          oneside & 单页模式 & \thesisissue{694} \\
          openright & 章从奇数页开始 & \thesisdiscuss{724} \\
          openany & 章从任意页开始 & \thesisissue{446} \\
          \bottomrule
        \end{tabular}
      \end{table}
    \end{column}
  \end{columns}
\end{frame}

\begin{frame}[fragile]
  \frametitle{基本配置}
  \framesubtitle{\textbackslash{}input\{setup\}}
  \begin{columns}
    \begin{column}{0.45\textwidth}
      \includegraphics[page=9]{thesisdir}
    \end{column}
    \begin{column}{0.55\textwidth}
      \begin{codeblock}[firstnumber=12]{main.tex}
|\highlightline<1>|% 论文基本配置,加载宏包等全局配置
|\highlightline<1>|\input{setup}

\begin{document}

%TC:ignore

|\highlightline<2>|% 标题页
|\highlightline<2>|\maketitle
      \end{codeblock}
      \visible<2>{
        \cmd{sjtusetup} 中的 \pkg{info} 将会修改封面的信息设置(见第 \ref{covers} 页)。
      }
    \end{column}
  \end{columns}
\end{frame}

\begin{frame}[fragile]
  \frametitle{基本配置}
  \framesubtitle{\textbackslash{}sjtusetup}
  \begin{columns}
    \begin{column}{0.45\textwidth}
      \includegraphics[page=1]{thesisdir}
    \end{column}
    \begin{column}{0.55\textwidth}
      \begin{codeblock}[firstnumber=3]{setup.tex}
\sjtusetup{
  info = {
    title    = {||上海交通大学学位论文 \LaTeX{} 模板示例文档},
    title*   = {A Sample for \LaTeX-based SJTU Thesis Template},
    author   = {||某\quad{}某},
    author* = {Mo Mo},
  },
  style = { header-logo-color = red, 
  },
  name = {
    publications = {||攻读学位期间完成的论文},
  },
}
      \end{codeblock}
    \end{column}
  \end{columns}
\end{frame}

\begin{frame}
  \frametitle{基本配置}
  \framesubtitle{\textbackslash{}sjtusetup}
  \begin{columns}
    \begin{column}{0.45\textwidth}
      \includegraphics[page=1]{thesisdir}
    \end{column}
    \begin{column}{0.55\textwidth}
      \begin{table}[H]
        \centering
        \caption{info 域}
        \footnotesize
        \begin{tabular}{lll} \toprule
          命令作用 & 中文对应选项 & 英文对应选项 \\ \midrule
          论文标题 & \texttt{title} & \texttt{title*} \\
          关键字列表 & \texttt{keywords} & \texttt{keywords*} \\
          作者姓名&  \texttt{author} &\texttt{author*}\\
          申请学位名称 & \texttt{degree}&\texttt{degree*}\\
          院系名称 & \texttt{department} & \texttt{department*}\\
          专业名称 & \texttt{major} & \texttt{major*}\\
          导师 & \texttt{supervisor} & \texttt{supervisor*}\\
          副导师 & \texttt{assisupervisor} & \texttt{assisupervisor*}\\
          日期 & \multicolumn{2}{c}{\texttt{date}}\\
          学号 & \multicolumn{2}{c}{\texttt{id}}\\ \bottomrule
          \end{tabular}
      \end{table}
    \end{column}
  \end{columns}
\end{frame}

\begin{frame}[fragile]
  \frametitle{版权页}
  \framesubtitle{\textbackslash{}copyrightpage}
  \begin{columns}
    \begin{column}{0.45\textwidth}
      \only<1>{
        \includegraphics[page=9]{thesisdir}
      }
      \only<2>{
        \includegraphics[page=2]{thesisdir}
      }
      \only<3>{
        \begin{figure}[H]
          \framebox{\includegraphics[page=2,width=0.4\linewidth]{bachelor}}
          \caption{版权页}
        \end{figure}
      }
    \end{column}
    \begin{column}{0.55\textwidth}
      \begin{codeblock}[firstnumber=22]{main.tex}
|\highlightline<1>|% 原创性声明及使用授权书
|\highlightline<1>|\copyrightpage
|\highlightline<2>|% 插入外置原创性声明及使用授权书
|\highlightline<2>|% \copyrightpage[scans/sample-copyright-old.pdf]
      \end{codeblock}
      \only<1>{
        \cmd{copyrightpages} 可以用于插入版权页。
      }
      \only<2>{
        \cmd{copyrightpages} 也接受一个可选参数,用于直接使用扫描件。
      }
    \end{column}
  \end{columns}
\end{frame}

\begin{frame}[fragile]
  \frametitle{前置部分}
  \framesubtitle{\textbackslash{}frontmatter}
  \begin{columns}
    \begin{column}{0.45\textwidth}
      \only<1>{
        \includegraphics[page=9]{thesisdir}
      }
      \only<2>{
        \includegraphics[page=3]{thesisdir}
      }
      \only<3>{
        \begin{figure}[H]
          \begin{subfigure}{0.45\textwidth}
            \framebox{\includegraphics[page=3,width=\linewidth]{bachelor}}
            \caption{中文}
          \end{subfigure}\hfill
          \begin{subfigure}{0.45\textwidth}
            \framebox{\includegraphics[page=4,width=\linewidth]{bachelor}}
            \caption{英文}
          \end{subfigure}
          \caption{摘要}
        \end{figure}
      }
      \only<4>{
        \begin{figure}[H]
          \begin{subfigure}{0.30\linewidth}
            \centering
            \framebox{\includegraphics[page=5,width=0.6\linewidth]{bachelor}}
            \caption{目录}
          \end{subfigure}
          \begin{subfigure}{0.30\linewidth}
            \centering
            \framebox{\includegraphics[page=6,width=0.6\linewidth]{bachelor}}
            \caption{插图}
          \end{subfigure}

          \begin{subfigure}{0.30\linewidth}
            \centering
            \framebox{\includegraphics[page=7,width=0.6\linewidth]{bachelor}}
            \caption{表格}
          \end{subfigure}
          \begin{subfigure}{0.30\linewidth}
            \centering
            \framebox{\includegraphics[page=8,width=0.6\linewidth]{bachelor}}
            \caption{算法}
          \end{subfigure}
          \caption{索引}
        \end{figure}
      }
      \only<5>{
        \includegraphics[page=4]{thesisdir}
      }
      \only<6>{
        \begin{figure}[H]
          \framebox{\includegraphics[page=9,width=0.5\linewidth]{bachelor}}
          \caption{符号对照表}
        \end{figure}
      }
    \end{column}
    \begin{column}{0.55\textwidth}
      \begin{codeblock}[firstnumber=30]{main.tex}
|\highlightline<2-3>|% 摘要
|\highlightline<2-3>|\input{contents/abstract}

|\highlightline<4>|% 目录
|\highlightline<4>|\tableofcontents
|\highlightline<4>|% 插图索引
|\highlightline<4>|\listoffigures*
|\highlightline<4>|% 表格索引
|\highlightline<4>|\listoftables*
|\highlightline<4>|% 算法索引
|\highlightline<4>|\listofalgorithms*

|\highlightline<5-6>|% 符号对照表
|\highlightline<5-6>|\input{contents/nomenclature}
      \end{codeblock}
    \end{column}
  \end{columns}
\end{frame}

\begin{frame}[fragile]
  \frametitle{主体部分}
  \framesubtitle{\textbackslash{}mainmatter}
  \begin{columns}
    \begin{column}{0.45\textwidth}
      \only<1>{
        \includegraphics[page=5]{thesisdir}
      }
      \only<2>{
        \begin{figure}[H]
          \begin{subfigure}{0.30\linewidth}
            \centering
            \framebox{\includegraphics[page=11,width=0.6\linewidth]{bachelor}}
            \caption{简介}
          \end{subfigure}
          \begin{subfigure}{0.30\linewidth}
            \centering
            \framebox{\includegraphics[page=13,width=0.6\linewidth]{bachelor}}
            \caption{数学}
          \end{subfigure}

          \begin{subfigure}{0.30\linewidth}
            \centering
            \framebox{\includegraphics[page=16,width=0.6\linewidth]{bachelor}}
            \caption{浮动体}
          \end{subfigure}
          \begin{subfigure}{0.30\linewidth}
            \centering
            \framebox{\includegraphics[page=22,width=0.6\linewidth]{bachelor}}
            \caption{总结}
          \end{subfigure}
          \caption{主体部分}
        \end{figure}
      }
    \end{column}
    \begin{column}{0.55\textwidth}
      \begin{codeblock}[firstnumber=47]{main.tex}
|\highlightline|% 正文内容
|\highlightline|% !TeX root = ../../../latex-talk.tex

\section{是什么}

\begin{frame}
  \frametitle{\TeX{}}
  \begin{columns}[c]
    \begin{column}{0.7\textwidth}
      \begin{center}
        \rmfamily\Huge
        \highlight[structure]{\TeX{}}
      \end{center}
      \begin{center}
        \parbox{0.75\textwidth}{
          \TeX{} 是由斯坦福大学教授高德纳
          (Donald E.~Knuth)于 1977 年开始开发的排版引擎。目前仍在更新,最新版本号为 3.141592653 \link{https://tug.org/TUGboat/tb42-1/tb130knuth-tuneup21.pdf}。
        }
      \end{center}
    \end{column}
    \begin{column}{0.3\textwidth}
      \includegraphics[width=.8\columnwidth]{support/images/Knuth.jpg}
    \end{column}
  \end{columns}
  \note{\emph{这一部分背景介绍大家可以了解一下,暂时跳过。}
  \LaTeX{} 这个词由两个部分组成,\hologo{La} 和 \TeX{}。那我们首先了解一下 \TeX{} 是什么。
  \TeX{} 是由斯坦福大学的教授高德纳于 1977 年开始开发的排版引擎,它已经有三十多年的历史了,
  目前仍在更新,版本号(3.141592653)将会趋近于 $\pi$ 的取值,高德纳最近还在给 \textsl{TUGBoat} 写稿子
  \link{https://tug.org/TUGboat/tb42-1/tb130knuth-tuneup21.pdf},
  关于 \TeX{} 今年又做了哪些改进。}
\end{frame}

\begin{frame}
  \frametitle{\LaTeX{}}
  \begin{columns}[c]
    \begin{column}{0.7\textwidth}
      \begin{center}
        \rmfamily\Huge
        \highlight[structure]{\LaTeX{}}
      \end{center}
      \begin{center}
        \parbox{0.75\textwidth}{
          \LaTeX{} 是最早在 1985 年由现就职于微软的 Leslie Lamport 开发的一种 \TeX{} \textbf{格式}\footnotemark,使用一些列宏和扩展宏包来简化 \TeX{} 的使用。现在由 \LaTeX{} Project 的成员维护。现在广泛使用的版本是 \LaTeXe{},最新的版本为 \LaTeX3(2020 年 10 月后默认内置)。
        }
      \end{center}
    \end{column}
    \begin{column}{0.3\textwidth}
      \includegraphics[width=.8\columnwidth]{support/images/Lamport.jpg}
    \end{column}
  \end{columns}
  \footnotetext{\hologo{ConTeXt} 也是一种 \TeX{} 格式 \link{https://www.contextgarden.net/}。}
  \note{\emph{这一部分的背景介绍大家可以了解一下,暂时跳过。}
  \LaTeX{} 是最早由现就职于微软的 Leslie Lamport 开发的一种 \TeX{} 格式(与其对标的是
  \hologo{ConTeXt}\link{https://www.contextgarden.net/}),主要也是为了简化 \TeX{} 的使用。
  现在主要由 \LaTeX{} 开发组维护,现在广泛使用的版本是 \LaTeXe{},最新的版本为 \LaTeX3,
  在 2020 年 10 月后默认内置,所以要尽可能使用较新的发行版,以充分发挥其功能。}
\end{frame}

\begin{frame}
  \frametitle{程序}
  \begin{columns}[c]
    \begin{column}{0.7\textwidth}
      \begin{center}
        \rmfamily\Huge
        \highlight[structure]{\hologo{pdfLaTeX}}
      \end{center}
      \begin{center}
        \parbox{0.7\textwidth}{
          \hologo{pdfLaTeX} 是为了编译一个 \LaTeX{} 文档而运行的程序。实际上底层在运行一个叫 \hologo{pdfTeX} 的引擎,并预装了对应的 \LaTeX{} \textbf{格式}。为了利用临时文件,可能就需要多次运行程序。
        }
      \end{center}
    \end{column}
    \begin{column}{0.3\textwidth}
      \begin{block}{}
        \ttfamily\small
        > \highlight{pdflatex} main.tex\\
        This is pdfTeX, Version 3.141592653-
        2.6-1.40.23 (MiKTeX 21.10)\\
        entering extended mode\\
        \highlight{LaTeX2e} <2021-11-15>\\
        \highlight{L3} programming layer <2021-11-22>
      \end{block}
    \end{column}
  \end{columns}
  \note{\hologo{pdfLaTeX} 是为了编译一个 \LaTeX{} 文档而运行的程序。}
\end{frame}

% \begin{frame}
%   \frametitle{引擎}
%   \begin{columns}[c]
%     \begin{column}{0.7\textwidth}
%       \begin{center}
%         \rmfamily\Huge
%         \highlight[structure!70]{pdf}\hologo{La}\highlight[structure!70]{\TeX{}}
%       \end{center}
%       \begin{center}
%         \parbox{0.7\textwidth}{
%           pdf\TeX{} 是编译 \TeX{} 文档(以 \texttt{.tex} 结尾)的\textbf{引擎}---可以理解 \TeX{} 指令的\textbf{程序}。
%         }
%       \end{center}
%     \end{column}
%     \begin{column}{0.3\textwidth}
%       \begin{block}{}
%         \ttfamily\small
%         > pdflatex main.tex\\
%         This is \highlight[structure!70]{pdfTeX}, Version 3.141592653-
%         2.6-1.40.23 (MiKTeX 21.10)
%         entering extended mode\\
%         LaTeX2e <2021-11-15>\\
%         L3 programming layer <2021-11-22>
%       \end{block}
%     \end{column}
%   \end{columns}
%   \note{实际上底层在运行一个叫 \hologo{pdfTeX} 的引擎,并预装了对应的 \LaTeX{} 格式。}
% \end{frame}

\begin{frame}[label={frame:engine}]
  \frametitle{程序}
  \begin{table}
    \caption{主流 \hologo{(La)TeX} 程序
    \footnote{(u)p\TeX{} 是日语最常用的引擎,生成 \texttt{.dvi},支持 Unicode。}\footnote{Ap\TeX{} \link{https://github.com/clerkma/ptex-ng} 具有底层 CJK 支持,内联 Ruby,Color Emoji。}}
    \footnotesize
    \begin{stampbox}
      \begin{tabular}{c>{\raggedright}*{3}{p{3.5cm}}}
        \alert{引擎}     & \hologo{pdfTeX}   & \hologo{XeTeX}   & \hologo{LuaTeX}   \\
        \alert{程序}     & \hologo{pdfLaTeX} & \hologo{XeLaTeX} & \hologo{LuaLaTeX} \\
        \alert{特点}     & 直接生成 PDF,支持 micro-typography  & 支持 Unicode、OpenType 与复杂文字编排 (CTL) & 支持 Unicode,内联 Lua,支持 OpenType \\
      \end{tabular}
    \end{stampbox}
  \end{table}

  \begin{center}
    \parbox{.9\textwidth}{
      \hologo{pdfLaTeX} 不支持 Unicode。为了排版中文,大部分情况下应当使用 \hologo{XeLaTeX},而 \hologo{LuaLaTeX} 速度相对较慢。\faWindows{} 可以在一些情况下使用 \hologo{pdfLaTeX}。
    }
  \end{center}
  \note{当然为了排版中文,已经不再推荐使用 \hologo{pdfLaTeX} 了,应该使用
  \hologo{XeLaTeX} 或者 \hologo{LuaLaTeX},当然后者的速度还是相对较慢,
  它们支持 Unicode 编码,并可以使用 OpenType 字体的全部功能。
  当然 \faWindows{} 平台下在某些追求速度的情况下,
  还是可以试着使用 \hologo{pdfLaTeX} 的。

  \hologo{LuaLaTeX} 理想情况下不慢,但是使用一些宏包后会破坏理想状态,
  也会因配置产生不同的结果,不同的操作系统在 I/O 速度上的不同也会导致不同的时间。

  \hologo{pdfLaTeX} 也支持,只不过需要先生成 tfm \TeX{} 字体度量文件,后续使用 \TeX{}
  自身的配置方法,只能使用 7 比特或 8 比特字体。}
\end{frame}

% \begin{frame}
%   \paragraph{\hologo{pdfLaTeX}} \TeX{} 和 \LaTeX{} 被广泛使用之前,它们只需内置支持欧洲语言即可。在 Unicode 出现之前,\LaTeX{} 提供了许多种\textbf{文件编码}来允许很多语言的文字以原生的方式输入,\hologo{pdfLaTeX} 也只需要使用 8 位文件编码和 8 位字体。
% \end{frame}


|\highlightline|\input{contents/math_and_citations}
|\highlightline|\input{contents/floats}
|\highlightline|\input{contents/summary}

%TC:ignore

% 参考文献
\printbibliography[heading=bibintoc]
      \end{codeblock}
    \end{column}
  \end{columns}
\end{frame}

\begin{frame}
  \frametitle{数学}
  \begin{itemize}
    \item 公式示例:\nolinkurl{contents/math_and_citations.tex}
    \item \SJTUThesis{} 定义了常用的数学环境(需要手工引入 \texttt{ntheorem} 宏包):
      \begin{table}[h]
        \centering
        \footnotesize
        \begin{tabular}{*{7}{l}}\toprule
          assumption  & axiom   & conjecture & corollary    & definition  & example   & exercise  \\
          假设        & 公理    & 猜想       & 推论         & 定义        & 例        & 练习      \\\midrule
          lemma       & problem & proof      & proposition  & remark      & solution  & theorem   \\
          引理        & 问题    & 证明       & 命题         & 注          & 解        & 定理      \\\bottomrule
        \end{tabular}
      \end{table}
      \item \SJTUThesis{} 可以通过 \texttt{unimath} 选项使用 \pkg{unicode-math} 进行数学输入,注意与传统方式的区别。\thesisissue{555}
  \end{itemize}
\end{frame}

\begin{frame}[fragile]
  \frametitle{参考文献}
  \begin{columns}
    \begin{column}{0.45\textwidth}
      \includegraphics[page=6]{thesisdir}
    \end{column}
    \begin{column}{0.55\textwidth}
      \begin{codeblock}[firstnumber=111,numbersep=2pt]{setup.tex}
% 使用 BibLaTeX 处理参考文献
%   biblatex-gb7714-2015 常用选项
%     gbnamefmt=lowercase     姓名大小写由输入信息确定
%     gbpub=false             禁用出版信息缺失处理
\usepackage[backend=biber,style=gb7714-2015]{biblatex}
% 文献表字体
% \renewcommand{\bibfont}{\zihao{-5}}
% 文献表条目间的间距
\setlength{\bibitemsep}{0pt}
|\highlightline|% 导入参考文献数据库
|\highlightline|\addbibresource{bibdata/thesis.bib}
      \end{codeblock}
    \end{column}
  \end{columns}
\end{frame}

\begin{frame}[fragile]
  \frametitle{附录}
  \framesubtitle{\textbackslash{}appendix}
  \begin{columns}
    \begin{column}{0.45\textwidth}
      \only<1>{
        \includegraphics[page=7]{thesisdir}
      }
      \only<2>{
        \begin{figure}[H]
          \begin{subfigure}{0.45\linewidth}
            \framebox{\includegraphics[width=\linewidth,page=24]{bachelor}}
            \caption{}
          \end{subfigure}\hfill
          \begin{subfigure}{0.45\textwidth}
            \framebox{\includegraphics[width=\linewidth,page=25]{bachelor}}
            \caption{}
          \end{subfigure}
          \caption{附录}
        \end{figure}
      }
    \end{column}
    \begin{column}{0.55\textwidth}
      \begin{codeblock}[firstnumber=61]{main.tex}
% 附录中图表不加入索引
\captionsetup{list=no}

% 附录内容
|\highlightline|\input{contents/app_maxwell_equations}
|\highlightline|\input{contents/app_flow_chart}
      \end{codeblock}
    \end{column}
  \end{columns}
\end{frame}

\begin{frame}[fragile]
  \frametitle{结尾部分}
  \framesubtitle{\textbackslash{}backmatter}
  \begin{columns}
    \begin{column}{0.45\textwidth}
      \only<1>{
        \includegraphics[page=8]{thesisdir}
      }
      \only<2>{
        \begin{figure}[H]
          \begin{subfigure}{0.30\linewidth}
            \centering
            \framebox{\includegraphics[page=26,width=0.6\linewidth]{bachelor}}
            \caption{致谢}
          \end{subfigure}
          \begin{subfigure}{0.30\linewidth}
            \centering
            \framebox{\includegraphics[page=27,width=0.6\linewidth]{bachelor}}
            \caption{成就}
          \end{subfigure}

          \begin{subfigure}{0.30\linewidth}
            \centering
            \framebox{\includegraphics[page=28,width=0.6\linewidth]{bachelor}}
            \caption{简历}
          \end{subfigure}
          \begin{subfigure}{0.30\linewidth}
            \centering
            \framebox{\includegraphics[page=29,width=0.6\linewidth]{bachelor}}
            \caption{大摘要*}
          \end{subfigure}
          \caption{结尾部分}
        \end{figure}
      }
    \end{column}
    \begin{column}{0.55\textwidth}
      \begin{codeblock}[firstnumber=76]{main.tex}
% 致谢
\input{contents/acknowledgements}

% 发表论文及科研成果
% 盲审论文中,发表论文及科研成果等仅以第几作者注明即可,不要出现作者或他人姓名
\input{contents/achievements}

% 简历
\input{contents/resume}

% 学士学位论文要求在最后有一个大摘要,单独编页码
\input{contents/digest}
      \end{codeblock}
    \end{column}
  \end{columns}
\end{frame}

\begin{frame}
  \frametitle{还有其他问题?}
  \begin{columns}
    \begin{column}{0.75\textwidth}
    \begin{itemize}
      \item[{\faComment*[regular]}] 日常模板或 \LaTeX{} 使用问题可以前往 Discussions \link{https://github.com/sjtug/SJTUThesis/discussions} 提问
      
      (解决后别忘了 \textcolor{green}{\faCheckCircle{} Mark as answer}
      \item[{\faDotCircle[regular]}] 如果是 \textsc{SJTUThesis} 项目本身的 bug 和 feature request
      
      可以通过 Issues \link{https://github.com/sjtug/SJTUThesis/issues} 反馈。
      \item[{\faCodeBranch}] 如果你有好点子,可以贡献代码
     
      向 \textsc{SJTU\TeX{}}(v1) \link{https://github.com/sjtug/SJTUTeX/tree/v1} 存储库发 PR,\par
      而后把解包结果同步到 \textsc{SJTUThesis}。
  
      \item[{\faTag}] 如果你对正在基于 \LaTeX3 开发的新版感兴趣,\par
      也欢迎向 \textsc{SJTU\TeX{}}(v2) \link{https://github.com/sjtug/SJTUTeX/tree/v2} 发 PR。
  
      \item[{\faQq}] 也欢迎在 QQ 群即时讨论。
    \end{itemize}
    \end{column}
    \begin{column}{0.25\textwidth}
      \includegraphics[height=0.7\textheight]{qq.jpg}
    \end{column}
  \end{columns}
\end{frame}